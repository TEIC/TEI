
\section[{Certainty, Precision, and Responsibility}]{Certainty, Precision, and Responsibility}\label{CE}\par
Encoders of text often find it useful to indicate that some aspects of the encoded text are problematic or uncertain, and to indicate who is responsible for various aspects of the markup of the electronic text. These Guidelines provide several methods of recording uncertainty about the text or its markup: \begin{itemize}
\item the \hyperref[TEI.note]{<note>} element defined in section \textit{\hyperref[CONO]{3.9.\ Notes, Annotation, and Indexing}} may be used with a value of certainty for its {\itshape type} attribute.
\item the \hyperref[TEI.certainty]{<certainty>} element defined in this chapter may be used to record the nature and degree of the uncertainty in a more structured way.
\item the \hyperref[TEI.precision]{<precision>} element defined in this chapter may be used to record the accuracy with which some numerical value (such as a date or quantity) is provided by some other element or attribute.
\item the \hyperref[TEI.alt]{<alt>} element defined in the module for linking and segmentation may be used to provide alternative encodings for parts of a text, as described in section \textit{\hyperref[SAAT]{16.8.\ Alternation}}.
\end{itemize}  There are three methods of indicating responsibility for different aspects of the electronic text: \begin{itemize}
\item the TEI header records who is responsible for an electronic text by means of the \hyperref[TEI.respStmt]{<respStmt>} element and other more specific elements (\hyperref[TEI.author]{<author>}, \hyperref[TEI.sponsor]{<sponsor>}, \hyperref[TEI.funder]{<funder>}, \hyperref[TEI.principal]{<principal>}, etc.) used within the \hyperref[TEI.titleStmt]{<titleStmt>}, \hyperref[TEI.editionStmt]{<editionStmt>}, and \hyperref[TEI.revisionDesc]{<revisionDesc>} elements.
\item the \hyperref[TEI.note]{<note>} element may be used with a value of resp or responsibility in its {\itshape type} attribute.
\item the \hyperref[TEI.respons]{<respons>} element defined in this chapter may be used to record fine-grained structured information about responsibility for individual tags in the text.
\end{itemize}  No special steps are needed to use the \hyperref[TEI.note]{<note>} and \hyperref[TEI.respStmt]{<respStmt>} elements, since they are defined in the core module and header respectively. The \hyperref[TEI.alt]{<alt>} element is only available when the module for linking has been selected, as described in chapter \textit{\hyperref[SA]{16.\ Linking, Segmentation, and Alignment}}. To use the \hyperref[TEI.certainty]{<certainty>}, \hyperref[TEI.precision]{<precision>} or \hyperref[TEI.respons]{<respons>} elements, the module for certainty and responsibility should be selected. \par
These three elements are all members of an attribute class called \textsf{att.scoping} from which they inherit the following attributes: 
\begin{sansreflist}
  
\item [\textbf{att.scoping}] provides attributes for selecting particular elements within a document.\hfil\\[-10pt]\begin{sansreflist}
    \item[@{\itshape target}]
  points at one or more sets of zero or more elements each.
    \item[@{\itshape match}]
  supplies an XPath selection pattern using the syntax defined in \cite{XSLT3} which identifies a set of nodes, selected within the context identified by the {\itshape target} attribute if this is supplied, or within the context of the parent element if it is not.
\end{sansreflist}  
\end{sansreflist}
\par
These attributes enable statements about certainty, precision, or responsibility to be made with respect to the whole of a document, or any part or parts of it which can be identified using standard XML location methods. Several examples are given in the discussion of the \hyperref[TEI.certainty]{<certainty>} element below; the same mechanisms are available for all three elements discussed in this chapter.
\subsection[{Levels of Certainty}]{Levels of Certainty}\label{CECERT}\par
Many types of uncertainty may be distinguished. The \hyperref[TEI.certainty]{<certainty>} element is designed to encode the following sorts: \begin{itemize}
\item a given tag may or may not correctly apply (e.g. a given word may be a personal name, or perhaps not)
\item the precise point at which an element begins or ends is uncertain
\item the value given for an attribute is uncertain
\item the content given for an element is unreliable for any reason.
\end{itemize} \par
The following types of uncertainty are \textit{not} indicated with the \hyperref[TEI.certainty]{<certainty>} element: \begin{itemize}
\item the numerical precision associated with a number or date (for this use the \hyperref[TEI.precision]{<precision>} element discussed in \textit{\hyperref[CEPREC]{21.2.\ Indications of Precision}})
\item the content of the document being transcribed is identifiable, but may be read or understood in different ways (for this use the transcriptional elements such as \hyperref[TEI.unclear]{<unclear>}, discussed in chapter \textit{\hyperref[PH]{11.\ Representation of Primary Sources}})
\item a transcriber, editor, or author wishes to indicate a level of confidence in a factual assertion made in the text (for this use the interpretative mechanisms discussed in \textit{\hyperref[AI]{17.\ Simple Analytic Mechanisms}} and \textit{\hyperref[FS]{18.\ Feature Structures}})
\end{itemize} 
\subsubsection[{Using Notes to Record Uncertainty}]{Using Notes to Record Uncertainty}\label{CECENO}\par
The simplest way of recording uncertainty about markup is to attach a note to the element or location about which one is unsure. In the following (invented) paragraph, for example, an encoder might be uncertain whether to mark ‘Essex’ as a place name or a personal name, since both might be plausible in the given context: 
\begin{quote}Elizabeth went to Essex. She had always liked Essex.\end{quote}
 Using \hyperref[TEI.note]{<note>}, the uncertainty here may be recorded quite simply: \par\bgroup\index{persName=<persName>|exampleindex}\index{placeName=<placeName>|exampleindex}\index{placeName=<placeName>|exampleindex}\index{note=<note>|exampleindex}\index{type=@type!<note>|exampleindex}\index{resp=@resp!<note>|exampleindex}\index{mentioned=<mentioned>|exampleindex}\exampleFont \begin{shaded}\noindent\mbox{}{<\textbf{persName}>}Elizabeth{</\textbf{persName}>} went to {<\textbf{placeName}>}Essex{</\textbf{placeName}>}. She had always liked {<\textbf{placeName}>}Essex{</\textbf{placeName}>}.\mbox{}\newline 
{<\textbf{note}\hspace*{1em}{type}="{certainty}"\hspace*{1em}{resp}="{\#MSM}">}It is not\mbox{}\newline 
 clear here whether {<\textbf{mentioned}>}Essex{</\textbf{mentioned}>}\mbox{}\newline 
 refers to the place or to the nobleman. -MSM{</\textbf{note}>}\end{shaded}\egroup\par \par
Using the normal mechanisms, the note may be associated unambiguously with specific elements of the text, thus: \par\bgroup\index{persName=<persName>|exampleindex}\index{placeName=<placeName>|exampleindex}\index{placeName=<placeName>|exampleindex}\index{note=<note>|exampleindex}\index{type=@type!<note>|exampleindex}\index{resp=@resp!<note>|exampleindex}\index{target=@target!<note>|exampleindex}\index{mentioned=<mentioned>|exampleindex}\index{name=<name>|exampleindex}\exampleFont \begin{shaded}\noindent\mbox{}{<\textbf{persName}>}Elizabeth{</\textbf{persName}>} went to {<\textbf{placeName}\hspace*{1em}{xml:id}="{CE-p1a}">}Essex{</\textbf{placeName}>}.\mbox{}\newline 
 She had always liked {<\textbf{placeName}\hspace*{1em}{xml:id}="{CE-p1b}">}Essex{</\textbf{placeName}>}.\mbox{}\newline 
{<\textbf{note}\hspace*{1em}{type}="{certainty}"\hspace*{1em}{resp}="{\#MSM}"\mbox{}\newline 
\hspace*{1em}{target}="{\#CE-p1a \#CE-p1b}">}It\mbox{}\newline 
 is not clear here whether {<\textbf{mentioned}>}Essex{</\textbf{mentioned}>}\mbox{}\newline 
 refers to the place or to the nobleman. If the latter,\mbox{}\newline 
 it should be tagged as a personal name. -{<\textbf{name}\hspace*{1em}{xml:id}="{MSM}">}Michael{</\textbf{name}>}\mbox{}\newline 
{</\textbf{note}>}\end{shaded}\egroup\par \par
The advantage of this technique is its relative simplicity. Its disadvantage is that the nature and degree of uncertainty are not conveyed in any systematic way and thus are not susceptible to any sort of automatic processing.
\subsubsection[{Structured Indications of Uncertainty}]{Structured Indications of Uncertainty}\label{CECECE}\par
To record uncertainty in a more structured way, susceptible of at least simple automatic processing, the \hyperref[TEI.certainty]{<certainty>} element may be used: 
\begin{sansreflist}
  
\item [\textbf{<certainty>}] indicates the degree of certainty associated with some aspect of the text markup.\hfil\\[-10pt]\begin{sansreflist}
    \item[@{\itshape locus}]
  indicates more exactly the aspect concerning which certainty is being expressed: specifically, whether the markup is correctly located, whether the correct element or attribute name has been used, or whether the content of the element or attribute is correct, etc.
    \item[@{\itshape degree}]
  indicates the degree of confidence assigned to the aspect of the markup named by the {\itshape locus} attribute.
\end{sansreflist}  
\end{sansreflist}
\par
Returning to the example, the \hyperref[TEI.certainty]{<certainty>} element may be used to record doubts about the proper encoding of ‘Essex’ in several ways of varying precision. To record merely that we are not certain that ‘Essex’ is in fact a place name, as it is tagged, we use the {\itshape target} attribute to identify the element in question, and the {\itshape locus} attribute to indicate which aspect of the markup we are uncertain about (in this case, whether we have used the correct ‘name’ for the element used to mark it): \par\bgroup\index{placeName=<placeName>|exampleindex}\index{certainty=<certainty>|exampleindex}\index{target=@target!<certainty>|exampleindex}\index{locus=@locus!<certainty>|exampleindex}\index{desc=<desc>|exampleindex}\exampleFont \begin{shaded}\noindent\mbox{}Elizabeth went to\mbox{}\newline 
{<\textbf{placeName}\hspace*{1em}{xml:id}="{CE-pl1}">}Essex{</\textbf{placeName}>}.\mbox{}\newline 
\mbox{}\newline 
\textit{<!-- ... elsewhere in the document ... -->}\mbox{}\newline 
{<\textbf{certainty}\hspace*{1em}{target}="{\#CE-pl1}"\hspace*{1em}{locus}="{name}">}\mbox{}\newline 
\hspace*{1em}{<\textbf{desc}>}possibly not a placename{</\textbf{desc}>}\mbox{}\newline 
{</\textbf{certainty}>}\end{shaded}\egroup\par \noindent  There are no particular constraints as to where the \hyperref[TEI.certainty]{<certainty>} element is placed in a document; it may be placed adjacent to the target element, or elsewhere in the same or another document. Its position is however significant when the {\itshape target} attribute is not specified as further discussed below.\par
We may wish to record the probability, assessed in some subjective way, that ‘Essex’ really is a place name here. The {\itshape degree} attribute is used to indicate the degree of confidence associated with the \hyperref[TEI.certainty]{<certainty>} element, expressed as a number between 0 and 1: \par\bgroup\index{certainty=<certainty>|exampleindex}\index{target=@target!<certainty>|exampleindex}\index{locus=@locus!<certainty>|exampleindex}\index{degree=@degree!<certainty>|exampleindex}\exampleFont \begin{shaded}\noindent\mbox{}\mbox{}\newline 
\textit{<!-- ... -->}{<\textbf{certainty}\hspace*{1em}{target}="{\#CE-pl1}"\hspace*{1em}{locus}="{name}"\mbox{}\newline 
\hspace*{1em}{degree}="{0.6}"/>}\end{shaded}\egroup\par \noindent  This expresses the point of view that there is a 60 percent chance of ‘Essex’ being a place name here, and hence a 40 percent chance of its being a personal name. We can use two \hyperref[TEI.certainty]{<certainty>} elements to indicate the two probabilities independently. Both elements indicate the same location in the text, but the second provides an alternative choice of name identifier (in this case \hyperref[TEI.persName]{<persName>}), which is given as the value of the {\itshape assertedValue} attribute: \par\bgroup\index{certainty=<certainty>|exampleindex}\index{target=@target!<certainty>|exampleindex}\index{locus=@locus!<certainty>|exampleindex}\index{degree=@degree!<certainty>|exampleindex}\index{desc=<desc>|exampleindex}\index{certainty=<certainty>|exampleindex}\index{target=@target!<certainty>|exampleindex}\index{locus=@locus!<certainty>|exampleindex}\index{degree=@degree!<certainty>|exampleindex}\index{assertedValue=@assertedValue!<certainty>|exampleindex}\index{desc=<desc>|exampleindex}\exampleFont \begin{shaded}\noindent\mbox{}\mbox{}\newline 
\textit{<!-- ... -->}{<\textbf{certainty}\hspace*{1em}{target}="{\#CE-pl1}"\hspace*{1em}{locus}="{name}"\mbox{}\newline 
\hspace*{1em}{degree}="{0.6}">}\mbox{}\newline 
\hspace*{1em}{<\textbf{desc}>}probably a placename, but possibly not{</\textbf{desc}>}\mbox{}\newline 
{</\textbf{certainty}>}\mbox{}\newline 
{<\textbf{certainty}\hspace*{1em}{target}="{\#CE-pl1}"\hspace*{1em}{locus}="{name}"\mbox{}\newline 
\hspace*{1em}{degree}="{0.4}"\hspace*{1em}{assertedValue}="{persName}">}\mbox{}\newline 
\hspace*{1em}{<\textbf{desc}>}may refer to the Earl of Essex{</\textbf{desc}>}\mbox{}\newline 
{</\textbf{certainty}>}\end{shaded}\egroup\par \par
In the simplest case, it is also possible to place the \hyperref[TEI.certainty]{<certainty>} element within the element concerned: \par\bgroup\index{placeName=<placeName>|exampleindex}\index{certainty=<certainty>|exampleindex}\index{locus=@locus!<certainty>|exampleindex}\index{degree=@degree!<certainty>|exampleindex}\exampleFont \begin{shaded}\noindent\mbox{}Elizabeth went to\mbox{}\newline 
{<\textbf{placeName}>}Essex\mbox{}\newline 
{<\textbf{certainty}\hspace*{1em}{locus}="{name}"\hspace*{1em}{degree}="{0.6}"/>}\mbox{}\newline 
{</\textbf{placeName}>}.\end{shaded}\egroup\par \noindent  When no {\itshape target} is specified, by default the proposed certainty applies to its parent element, in this case the \hyperref[TEI.placeName]{<placeName>} element. The {\itshape match} attribute discussed below may be used to further vary this behaviour.
\paragraph[{Contingent Conditions}]{Contingent Conditions}\label{CEconcon}\par
Finally, we may wish to make our probability estimates contingent on some condition. In the passage ‘Elizabeth went to Essex; she had always liked Essex,’ for example, we may feel there is a 60 percent chance that the county is meant, and a 40 percent chance that the earl is meant. But the two occurrences of the word are not independent: there is (we may feel) no chance at all that the first occurrence refers to the county and the second to the earl. We can express this by using the {\itshape given} attribute to list the identifiers of \hyperref[TEI.certainty]{<certainty>} elements. \par\bgroup\index{placeName=<placeName>|exampleindex}\index{placeName=<placeName>|exampleindex}\index{certainty=<certainty>|exampleindex}\index{target=@target!<certainty>|exampleindex}\index{locus=@locus!<certainty>|exampleindex}\index{degree=@degree!<certainty>|exampleindex}\index{desc=<desc>|exampleindex}\index{certainty=<certainty>|exampleindex}\index{target=@target!<certainty>|exampleindex}\index{locus=@locus!<certainty>|exampleindex}\index{assertedValue=@assertedValue!<certainty>|exampleindex}\index{degree=@degree!<certainty>|exampleindex}\index{desc=<desc>|exampleindex}\index{certainty=<certainty>|exampleindex}\index{target=@target!<certainty>|exampleindex}\index{locus=@locus!<certainty>|exampleindex}\index{given=@given!<certainty>|exampleindex}\index{degree=@degree!<certainty>|exampleindex}\index{desc=<desc>|exampleindex}\index{certainty=<certainty>|exampleindex}\index{target=@target!<certainty>|exampleindex}\index{locus=@locus!<certainty>|exampleindex}\index{assertedValue=@assertedValue!<certainty>|exampleindex}\index{degree=@degree!<certainty>|exampleindex}\index{given=@given!<certainty>|exampleindex}\index{desc=<desc>|exampleindex}\exampleFont \begin{shaded}\noindent\mbox{}Elizabeth went to {<\textbf{placeName}\hspace*{1em}{xml:id}="{CE-PL1}">}Essex{</\textbf{placeName}>}.\mbox{}\newline 
 She had always liked {<\textbf{placeName}\hspace*{1em}{xml:id}="{CE-PL2}">}Essex{</\textbf{placeName}>}.\mbox{}\newline 
\mbox{}\newline 
\textit{<!-- ... -->}\mbox{}\newline 
\textit{<!-- 60\% chance that P1 is a placename,\newline
     40\% chance a personal name. -->}\mbox{}\newline 
{<\textbf{certainty}\hspace*{1em}{xml:id}="{cert-1}"\hspace*{1em}{target}="{\#CE-PL1}"\mbox{}\newline 
\hspace*{1em}{locus}="{name}"\hspace*{1em}{degree}="{0.6}">}\mbox{}\newline 
\hspace*{1em}{<\textbf{desc}>}probably a placename, but possibly not"{</\textbf{desc}>}\mbox{}\newline 
{</\textbf{certainty}>}\mbox{}\newline 
{<\textbf{certainty}\hspace*{1em}{xml:id}="{cert-2}"\hspace*{1em}{target}="{\#CE-PL1}"\mbox{}\newline 
\hspace*{1em}{locus}="{name}"\hspace*{1em}{assertedValue}="{persName}"\hspace*{1em}{degree}="{0.4}">}\mbox{}\newline 
\hspace*{1em}{<\textbf{desc}>}may refer to the Earl of Essex"{</\textbf{desc}>}\mbox{}\newline 
{</\textbf{certainty}>}\mbox{}\newline 
\textit{<!-- 60\% chance that P2 is a placename,\newline
     40\% chance a personal name.\newline
    100\% chance that it agrees with P1. -->}\mbox{}\newline 
{<\textbf{certainty}\hspace*{1em}{target}="{\#CE-PL2}"\hspace*{1em}{locus}="{name}"\mbox{}\newline 
\hspace*{1em}{given}="{\#cert-1}"\hspace*{1em}{degree}="{1.0}">}\mbox{}\newline 
\hspace*{1em}{<\textbf{desc}>}if CE-PL1 is a placename, CE-PL2 certainly is"{</\textbf{desc}>}\mbox{}\newline 
{</\textbf{certainty}>}\mbox{}\newline 
{<\textbf{certainty}\hspace*{1em}{target}="{\#CE-PL2}"\hspace*{1em}{locus}="{name}"\mbox{}\newline 
\hspace*{1em}{assertedValue}="{persName}"\hspace*{1em}{degree}="{1.0}"\hspace*{1em}{given}="{\#cert-2}">}\mbox{}\newline 
\hspace*{1em}{<\textbf{desc}>}if CE-PL1 is a personal name, then so is CE-PL2{</\textbf{desc}>}\mbox{}\newline 
{</\textbf{certainty}>}\end{shaded}\egroup\par \noindent  When {\itshape given} conditions are listed, the \hyperref[TEI.certainty]{<certainty>} element is interpreted as claiming a given degree of confidence in a particular markup given the assertional content of the \hyperref[TEI.certainty]{<certainty>} elements indicated. That is, a conjectural assertion is being made solely on the assumption that the interpretation indicated by the element named by the {\itshape given} attribute is actually correct.\par
Conditional confidence may be less than 100 percent: given the sentence ‘Ernest went to old Saybrook’, we may interpret ‘Saybrook’ as a personal name or a place name, assigning a 60 percent probability to the former. If it is a place name, there may be a 50 percent chance that the place name actually in question is ‘Old Saybrook’ rather than ‘Saybrook’, while if it is correctly tagged as a personal name, it is much more likely (say, 90 percent certain) that the name is ‘Saybrook’. Hence there is uncertainty about the correct location for the markup as well as about which markup to use. This state of affairs can be expressed using the \hyperref[TEI.certainty]{<certainty>} element thus: \par\bgroup\index{anchor=<anchor>|exampleindex}\index{persName=<persName>|exampleindex}\index{certainty=<certainty>|exampleindex}\index{target=@target!<certainty>|exampleindex}\index{locus=@locus!<certainty>|exampleindex}\index{degree=@degree!<certainty>|exampleindex}\index{certainty=<certainty>|exampleindex}\index{target=@target!<certainty>|exampleindex}\index{locus=@locus!<certainty>|exampleindex}\index{given=@given!<certainty>|exampleindex}\index{degree=@degree!<certainty>|exampleindex}\index{certainty=<certainty>|exampleindex}\index{target=@target!<certainty>|exampleindex}\index{locus=@locus!<certainty>|exampleindex}\index{assertedValue=@assertedValue!<certainty>|exampleindex}\index{degree=@degree!<certainty>|exampleindex}\index{certainty=<certainty>|exampleindex}\index{target=@target!<certainty>|exampleindex}\index{locus=@locus!<certainty>|exampleindex}\index{given=@given!<certainty>|exampleindex}\index{degree=@degree!<certainty>|exampleindex}\index{certainty=<certainty>|exampleindex}\index{target=@target!<certainty>|exampleindex}\index{locus=@locus!<certainty>|exampleindex}\index{assertedValue=@assertedValue!<certainty>|exampleindex}\index{given=@given!<certainty>|exampleindex}\index{degree=@degree!<certainty>|exampleindex}\index{certainty=<certainty>|exampleindex}\index{target=@target!<certainty>|exampleindex}\index{locus=@locus!<certainty>|exampleindex}\index{assertedValue=@assertedValue!<certainty>|exampleindex}\index{given=@given!<certainty>|exampleindex}\index{degree=@degree!<certainty>|exampleindex}\exampleFont \begin{shaded}\noindent\mbox{}Ernest went to {<\textbf{anchor}\hspace*{1em}{xml:id}="{CE-a1}"/>} old {<\textbf{persName}\hspace*{1em}{xml:id}="{CE-p2}">}Saybrook{</\textbf{persName}>}.\mbox{}\newline 
\mbox{}\newline 
{<\textbf{certainty}\hspace*{1em}{xml:id}="{cert1}"\hspace*{1em}{target}="{\#CE-p2}"\mbox{}\newline 
\hspace*{1em}{locus}="{name}"\hspace*{1em}{degree}="{0.6}"/>}\mbox{}\newline 
{<\textbf{certainty}\hspace*{1em}{target}="{\#CE-p2}"\hspace*{1em}{locus}="{start}"\mbox{}\newline 
\hspace*{1em}{given}="{\#cert1}"\hspace*{1em}{degree}="{0.9}"/>}\mbox{}\newline 
{<\textbf{certainty}\hspace*{1em}{xml:id}="{cert2}"\hspace*{1em}{target}="{\#CE-p2}"\mbox{}\newline 
\hspace*{1em}{locus}="{name}"\hspace*{1em}{assertedValue}="{placeName}"\hspace*{1em}{degree}="{0.4}"/>}\mbox{}\newline 
{<\textbf{certainty}\hspace*{1em}{target}="{\#CE-p2}"\hspace*{1em}{locus}="{start}"\mbox{}\newline 
\hspace*{1em}{given}="{\#cert2}"\hspace*{1em}{degree}="{0.5}"/>}\mbox{}\newline 
{<\textbf{certainty}\hspace*{1em}{xml:id}="{cert3}"\hspace*{1em}{target}="{\#CE-p2}"\mbox{}\newline 
\hspace*{1em}{locus}="{start}"\hspace*{1em}{assertedValue}="{\#CE-a1}"\hspace*{1em}{given}="{\#cert1}"\mbox{}\newline 
\hspace*{1em}{degree}="{0.1}"/>}\mbox{}\newline 
{<\textbf{certainty}\hspace*{1em}{xml:id}="{cert4}"\hspace*{1em}{target}="{\#CE-p2}"\mbox{}\newline 
\hspace*{1em}{locus}="{start}"\hspace*{1em}{assertedValue}="{\#CE-a1}"\hspace*{1em}{given}="{\#cert2}"\mbox{}\newline 
\hspace*{1em}{degree}="{0.5}"/>}\end{shaded}\egroup\par \noindent  Note the use of the {\itshape assertedValue} on \hyperref[TEI.certainty]{<certainty>} elements cert3 and cert4 to reference the \hyperref[TEI.anchor]{<anchor>} element placed at the alternative starting point for the element.\par
Multiplying the numeric values out, this markup may be interpreted as assigning specific probabilities to three different ways of marking up the sentence: \par\bgroup\index{persName=<persName>|exampleindex}\index{placeName=<placeName>|exampleindex}\index{placeName=<placeName>|exampleindex}\exampleFont \begin{shaded}\noindent\mbox{}Ernest went to old {<\textbf{persName}>}Saybrook{</\textbf{persName}>}. (0.6 * 0.9, or 0.54)\mbox{}\newline 
 Ernest went to old {<\textbf{placeName}>}Saybrook{</\textbf{placeName}>}. (0.4 * 0.5, or 0.20)\mbox{}\newline 
 Ernest went to {<\textbf{placeName}>}old Saybrook{</\textbf{placeName}>}. (0.4 * 0.5, or 0.20)\end{shaded}\egroup\par \noindent  The probabilities do not add up to 1.00 because the markup indicates that if ‘Saybrook’ is (part of) a personal name, there is a 10 percent likelihood that the element should start somewhere other than the place indicated, without however giving an alternative location; there is thus a 6 percent chance (0.1 × 0.6) that none of the alternatives given is correct.
\paragraph[{Pervasive Conditions}]{Pervasive Conditions}\par
We may also wish to indicate confidence in some aspect of the tagging throughout a document, rather than (as discussed so far) in one particular instance. The {\itshape match} attribute may be used to supply a pattern identifying the portion of a document concerning which certainty is being expressed. The value of the {\itshape match} attribute is an XSLT selection pattern using the syntax defined in \cite{XSLT3}. In the following example, we wish to indicate a low degree of confidence that the \hyperref[TEI.persName]{<persName>} elements used throughout the whole document have been correctly applied: \par\bgroup\index{certainty=<certainty>|exampleindex}\index{locus=@locus!<certainty>|exampleindex}\index{degree=@degree!<certainty>|exampleindex}\index{match=@match!<certainty>|exampleindex}\exampleFont \begin{shaded}\noindent\mbox{}{<\textbf{certainty}\hspace*{1em}{locus}="{name}"\hspace*{1em}{degree}="{0.3}"\mbox{}\newline 
\hspace*{1em}{match}="{//persName}"/>}\end{shaded}\egroup\par \noindent  No {\itshape target} has been supplied here, and so by default the \hyperref[TEI.certainty]{<certainty>} expressed would therefore apply to the parent element. However, in this case the XPath supplied as the value for {\itshape match} returns a set of all the \hyperref[TEI.persName]{<persName>} elements in the document, independent of the current context. By contrast, in the following example \par\bgroup\index{div=<div>|exampleindex}\index{p=<p>|exampleindex}\index{div=<div>|exampleindex}\index{certainty=<certainty>|exampleindex}\index{locus=@locus!<certainty>|exampleindex}\index{degree=@degree!<certainty>|exampleindex}\index{match=@match!<certainty>|exampleindex}\exampleFont \begin{shaded}\noindent\mbox{}{<\textbf{div}>}\mbox{}\newline 
\hspace*{1em}{<\textbf{p}>}[...]{</\textbf{p}>}\mbox{}\newline 
{</\textbf{div}>}\mbox{}\newline 
{<\textbf{div}>}\mbox{}\newline 
\hspace*{1em}{<\textbf{certainty}\hspace*{1em}{locus}="{name}"\hspace*{1em}{degree}="{0.3}"\mbox{}\newline 
\hspace*{1em}\hspace*{1em}{match}="{.//persName}"/>}\mbox{}\newline 
{</\textbf{div}>}\end{shaded}\egroup\par \noindent  only the \hyperref[TEI.persName]{<persName>} elements within the second \hyperref[TEI.div]{<div>} element are in question. Similarly, we may indicate that we have more confidence in the \hyperref[TEI.persName]{<persName>} tagging within those \hyperref[TEI.div]{<div>} elements which have a {\itshape type} value of \texttt{checked}: \par\bgroup\index{certainty=<certainty>|exampleindex}\index{locus=@locus!<certainty>|exampleindex}\index{degree=@degree!<certainty>|exampleindex}\index{match=@match!<certainty>|exampleindex}\exampleFont \begin{shaded}\noindent\mbox{}{<\textbf{certainty}\hspace*{1em}{locus}="{name}"\hspace*{1em}{degree}="{0.7}"\mbox{}\newline 
\hspace*{1em}{match}="{//div[@type='checked']//persName}"/>}\end{shaded}\egroup\par \noindent  If an element in a document is matched by more than one match expression, then the most specific pattern applies. \footnote{Specificity of pattern matching is defined further in the XSLT3 reference cited above (see \url{https://www.w3.org/TR/xslt-30/\#default-priority})} As a simple case, if both the preceding \hyperref[TEI.certainty]{<certainty>} elements were present in the same document, a \hyperref[TEI.persName]{<persName>} occurring within a <div type="checked"> element would potentially match both pattern expressions. However because the second pattern is more specific than the former, in fact this is the only one that would apply. If multiple patterns match and have the same priority, then the first one (in document order) is applied. Only those statements of certainty which have matched in this sense are available for conditional application using the {\itshape given} attribute mentioned above.\par
When the {\itshape match} attribute is processed, the namespace bindings in force are those in effect at that point in the document. For example, \par\bgroup\index{div=<div>|exampleindex}\index{certainty=<certainty>|exampleindex}\index{match=@match!<certainty>|exampleindex}\index{locus=@locus!<certainty>|exampleindex}\index{degree=@degree!<certainty>|exampleindex}\exampleFont \begin{shaded}\noindent\mbox{}{<\textbf{div}>}\mbox{}\newline 
\textit{<!-- ... -->}\mbox{}\newline 
\hspace*{1em}{<\textbf{certainty}\hspace*{1em}{match}="{.//my:*}"\hspace*{1em}{locus}="{value}"\mbox{}\newline 
\hspace*{1em}\hspace*{1em}{degree}="{0.9}"/>}\mbox{}\newline 
{</\textbf{div}>}\end{shaded}\egroup\par \noindent  might be used to indicate a high degree of certainty about the content of any elements taken the namespace associated with the prefix \texttt{my}. This namespace prefix must be associated with an appropriate namespace definition, either on the \hyperref[TEI.certainty]{<certainty>} element itself, or on one of its ancestor elements.
\paragraph[{Content Uncertainty}]{Content Uncertainty}\par
Doubts about whether the content of an element is correct may also be expressed by assigning to {\itshape locus} the value \textit{value}. For example, if the source is hard to read and so the transcription is uncertain: \par\bgroup\index{emph=<emph>|exampleindex}\index{certainty=<certainty>|exampleindex}\index{target=@target!<certainty>|exampleindex}\index{locus=@locus!<certainty>|exampleindex}\index{degree=@degree!<certainty>|exampleindex}\exampleFont \begin{shaded}\noindent\mbox{}I have a {<\textbf{emph}\hspace*{1em}{xml:id}="{CE-p3}">}bun{</\textbf{emph}>}.\mbox{}\newline 
\mbox{}\newline 
{<\textbf{certainty}\hspace*{1em}{target}="{\#CE-p3}"\hspace*{1em}{locus}="{value}"\mbox{}\newline 
\hspace*{1em}{degree}="{0.5}"/>}\end{shaded}\egroup\par \par
Degrees of confidence in the proper expansion of abbreviations may also be expressed, as in the following example:\par\bgroup\index{choice=<choice>|exampleindex}\index{expan=<expan>|exampleindex}\index{expan=<expan>|exampleindex}\index{abbr=<abbr>|exampleindex}\index{certainty=<certainty>|exampleindex}\index{target=@target!<certainty>|exampleindex}\index{locus=@locus!<certainty>|exampleindex}\index{degree=@degree!<certainty>|exampleindex}\index{certainty=<certainty>|exampleindex}\index{target=@target!<certainty>|exampleindex}\index{locus=@locus!<certainty>|exampleindex}\index{degree=@degree!<certainty>|exampleindex}\exampleFont \begin{shaded}\noindent\mbox{}You will want to use\mbox{}\newline 
{<\textbf{choice}>}\mbox{}\newline 
\hspace*{1em}{<\textbf{expan}\hspace*{1em}{xml:id}="{CE-e1}">}Standard\mbox{}\newline 
\hspace*{1em}\hspace*{1em} Generalized Markup Language{</\textbf{expan}>}\mbox{}\newline 
\hspace*{1em}{<\textbf{expan}\hspace*{1em}{xml:id}="{CE-e40}">}Some Grandiose Methodology for Losers{</\textbf{expan}>}\mbox{}\newline 
\hspace*{1em}{<\textbf{abbr}>}SGML{</\textbf{abbr}>}\mbox{}\newline 
{</\textbf{choice}>} ...\mbox{}\newline 
\mbox{}\newline 
\textit{<!-- ... -->}\mbox{}\newline 
{<\textbf{certainty}\hspace*{1em}{target}="{\#CE-e1}"\hspace*{1em}{locus}="{value}"\mbox{}\newline 
\hspace*{1em}{degree}="{0.9}"/>}\mbox{}\newline 
{<\textbf{certainty}\hspace*{1em}{target}="{\#CE-e40}"\hspace*{1em}{locus}="{value}"\mbox{}\newline 
\hspace*{1em}{degree}="{0.5}"/>}\end{shaded}\egroup\par \par
The {\itshape assertedValue} attribute should be used to provide an alternative value for whatever aspect of the markup is in doubt: an alternative name, or the identifier of an alternative starting or ending point, as already shown, an alternative attribute value, or alternative element content, as in this example: \par\bgroup\index{emph=<emph>|exampleindex}\index{certainty=<certainty>|exampleindex}\index{target=@target!<certainty>|exampleindex}\index{locus=@locus!<certainty>|exampleindex}\index{assertedValue=@assertedValue!<certainty>|exampleindex}\index{degree=@degree!<certainty>|exampleindex}\index{desc=<desc>|exampleindex}\exampleFont \begin{shaded}\noindent\mbox{}I have a {<\textbf{emph}\hspace*{1em}{xml:id}="{CE-P3}">}bun{</\textbf{emph}>}.\mbox{}\newline 
\mbox{}\newline 
{<\textbf{certainty}\hspace*{1em}{target}="{\#CE-P3}"\hspace*{1em}{locus}="{value}"\mbox{}\newline 
\hspace*{1em}{assertedValue}="{gun}"\hspace*{1em}{degree}="{0.8}">}\mbox{}\newline 
\hspace*{1em}{<\textbf{desc}>}a gun makes more sense in a holdup{</\textbf{desc}>}\mbox{}\newline 
{</\textbf{certainty}>}\end{shaded}\egroup\par \noindent  Since attribute values have no internal substructure, the {\itshape assertedValue} attribute is not generally useful for specifying alternative transcriptions; it cannot for example be used if the alternative reading contains markup of any kind. More robust methods of handling uncertainties of transcription are the \hyperref[TEI.unclear]{<unclear>} element and the \hyperref[TEI.app]{<app>} and \hyperref[TEI.rdg]{<rdg>} elements described in chapter \textit{\hyperref[TC]{12.\ Critical Apparatus}}. The \hyperref[TEI.certainty]{<certainty>} element allows for indications of uncertainty to be structured with at least as much detail and clarity as appears to be currently required in most ongoing text projects.
\paragraph[{Target or Match?}]{Target or Match?}\par
As noted in \textit{\hyperref[SA]{16.\ Linking, Segmentation, and Alignment}}, the {\itshape target} attribute may take any general teidata.pointer as values and may thus also contain an XPath expression of arbitrary complexity. Because full support for XPath is not provided by current processors, it is not generally recommended TEI practice. There are however some simple cases in which XPath syntax is to be preferred, notably those in which the {\itshape xml:id} attribute is used to identify a single element occurrence. The usage \#A (to indicate the element whose {\itshape xml:id} attribute has the value A) is syntactically much simpler than the equivalent xpath2 expression //*[@xml:id='A'] and is hence preferred throughout these guidelines.\par
For similar reasons, the \hyperref[TEI.certainty]{<certainty>} element may specify both a {\itshape target} value (expressed as an URI) and a {\itshape match} value (expressed as an XPath). The former defines the context within which the latter is to be evaluated. As previously noted, if no value is supplied for {\itshape target}, the context within which the value of {\itshape match} should be evaluated is the parent element of the \hyperref[TEI.certainty]{<certainty>} element itself.\par
A typical case where it may be convenient to specify both {\itshape target} and {\itshape match} is that where we wish to indicate that the value of an attribute on some specific element is uncertain. In this case, the {\itshape locus} attribute takes the value value. For example, supposing there is only a 50 percent chance that the question was spoken by participant A: \par\bgroup\index{u=<u>|exampleindex}\index{who=@who!<u>|exampleindex}\index{certainty=<certainty>|exampleindex}\index{target=@target!<certainty>|exampleindex}\index{match=@match!<certainty>|exampleindex}\index{locus=@locus!<certainty>|exampleindex}\index{degree=@degree!<certainty>|exampleindex}\exampleFont \begin{shaded}\noindent\mbox{}{<\textbf{u}\hspace*{1em}{xml:id}="{CE-u1}"\hspace*{1em}{who}="{\#A}">}Have you heard the election results?{</\textbf{u}>}\mbox{}\newline 
{<\textbf{certainty}\hspace*{1em}{target}="{\#CE-u1}"\hspace*{1em}{match}="{@who}"\mbox{}\newline 
\hspace*{1em}{locus}="{value}"\hspace*{1em}{degree}="{0.5}"/>}\end{shaded}\egroup\par \noindent  or, equivalently and without the need to define a target, \par\bgroup\index{u=<u>|exampleindex}\index{who=@who!<u>|exampleindex}\index{certainty=<certainty>|exampleindex}\index{match=@match!<certainty>|exampleindex}\index{locus=@locus!<certainty>|exampleindex}\index{degree=@degree!<certainty>|exampleindex}\exampleFont \begin{shaded}\noindent\mbox{}{<\textbf{u}\hspace*{1em}{who}="{\#A}">}Have you heard the election results?{<\textbf{certainty}\hspace*{1em}{match}="{@who}"\hspace*{1em}{locus}="{value}"\mbox{}\newline 
\hspace*{1em}\hspace*{1em}{degree}="{0.5}"/>}\mbox{}\newline 
{</\textbf{u}>}\end{shaded}\egroup\par \par
The {\itshape match} and {\itshape target} attributes together provide a powerful mechanism which can be used to indicate precision for a large number of assertions throughout an encoded document in an economical way. Some further examples follow: \par\bgroup\index{certainty=<certainty>|exampleindex}\index{match=@match!<certainty>|exampleindex}\index{locus=@locus!<certainty>|exampleindex}\index{degree=@degree!<certainty>|exampleindex}\exampleFont \begin{shaded}\noindent\mbox{}{<\textbf{certainty}\hspace*{1em}{match}="{//p}"\hspace*{1em}{locus}="{location}"\mbox{}\newline 
\hspace*{1em}{degree}="{0.2}"/>}\end{shaded}\egroup\par \noindent  This encoding indicates that there is only a 0.2 certainty that the boundaries of all \hyperref[TEI.p]{<p>} elements in the document have been correctly identified.\par
\par\bgroup\index{certainty=<certainty>|exampleindex}\index{target=@target!<certainty>|exampleindex}\index{match=@match!<certainty>|exampleindex}\index{locus=@locus!<certainty>|exampleindex}\index{degree=@degree!<certainty>|exampleindex}\exampleFont \begin{shaded}\noindent\mbox{}{<\textbf{certainty}\hspace*{1em}{target}="{\#a101}"\hspace*{1em}{match}="{p}"\mbox{}\newline 
\hspace*{1em}{locus}="{location}"\hspace*{1em}{degree}="{0.2}"/>}\end{shaded}\egroup\par \noindent  This encoding indicates that there is only a 0.2 certainty that the boundaries of the \hyperref[TEI.p]{<p>} elements contained by the element with {\itshape xml:id} value a101 have been correctly identified.\par
\par\bgroup\index{persName=<persName>|exampleindex}\index{resp=@resp!<persName>|exampleindex}\index{certainty=<certainty>|exampleindex}\index{match=@match!<certainty>|exampleindex}\index{locus=@locus!<certainty>|exampleindex}\index{degree=@degree!<certainty>|exampleindex}\exampleFont \begin{shaded}\noindent\mbox{}{<\textbf{persName}\hspace*{1em}{resp}="{\#LB}">}Essex\mbox{}\newline 
{<\textbf{certainty}\hspace*{1em}{match}="{@resp}"\hspace*{1em}{locus}="{value}"\mbox{}\newline 
\hspace*{1em}\hspace*{1em}{degree}="{0.2}"/>}\mbox{}\newline 
{</\textbf{persName}>}\end{shaded}\egroup\par \noindent  This encoding indicates that there is only a 0.2 certainty that the value for the {\itshape resp} attribute on the given \hyperref[TEI.persName]{<persName>} element is correct.\par
\par\bgroup\index{certainty=<certainty>|exampleindex}\index{match=@match!<certainty>|exampleindex}\index{locus=@locus!<certainty>|exampleindex}\index{degree=@degree!<certainty>|exampleindex}\exampleFont \begin{shaded}\noindent\mbox{}{<\textbf{certainty}\hspace*{1em}{match}="{//*/@resp}"\hspace*{1em}{locus}="{value}"\mbox{}\newline 
\hspace*{1em}{degree}="{0.2}"/>}\end{shaded}\egroup\par \noindent  This encoding indicates that there is only a 0.2 certainty that any value for the {\itshape resp} attribute is correct, wherever it appears in the document.\par
\par\bgroup\index{certainty=<certainty>|exampleindex}\index{target=@target!<certainty>|exampleindex}\index{match=@match!<certainty>|exampleindex}\index{locus=@locus!<certainty>|exampleindex}\index{degree=@degree!<certainty>|exampleindex}\exampleFont \begin{shaded}\noindent\mbox{}{<\textbf{certainty}\hspace*{1em}{target}="{\#dd001}"\hspace*{1em}{match}="{@resp}"\mbox{}\newline 
\hspace*{1em}{locus}="{value}"\hspace*{1em}{degree}="{0.2}"/>}\end{shaded}\egroup\par \noindent  This encoding indicates that there is only a 0.2 certainty that the value for the {\itshape resp} attribute of the element indicated by the pointer \#dd001 is correct\par
\par\bgroup\index{certainty=<certainty>|exampleindex}\index{match=@match!<certainty>|exampleindex}\index{locus=@locus!<certainty>|exampleindex}\index{degree=@degree!<certainty>|exampleindex}\exampleFont \begin{shaded}\noindent\mbox{}{<\textbf{certainty}\hspace*{1em}{match}="{//*[@resp='\#LB']}"\mbox{}\newline 
\hspace*{1em}{locus}="{value}"\hspace*{1em}{degree}="{0.2}"/>}\end{shaded}\egroup\par \noindent  This encoding indicates that there is only a 0.2 certainty that the content of any element the {\itshape resp} attribute of which has the value \#LB is correct, wherever it appears in the document.\par
The \hyperref[TEI.certainty]{<certainty>} element and the other TEI mechanisms for indicating uncertainty provide a range of methods of graduated complexity. Simple expressions of uncertainty may be made by using the \hyperref[TEI.note]{<note>} element. This is simple and convenient, and can accommodate either a discursive and unstructured indication of uncertainty, or a complex and structured but probably project-specific expression of uncertainty. In general, however, unless special steps are taken, the \hyperref[TEI.note]{<note>} element does not provide as much expressive power as the \hyperref[TEI.certainty]{<certainty>} element, and in cases where highly structured certainty information are needed, it is recommended that the \hyperref[TEI.certainty]{<certainty>} element be preferred.
\subsection[{Indications of Precision}]{Indications of Precision}\label{CEPREC}\par
As noted above, certainty about the accuracy of an encoding or its content is not the same thing as the \textit{precision} with which a value is specified. In the case of a date or a quantity, for example, we might be certain that the value given is imprecise, or uncertain about whether or not the value given is correct. The latter possibility would be represented by the \hyperref[TEI.certainty]{<certainty>} element discussed in the previous section; the former by the \hyperref[TEI.precision]{<precision>} element discussed in this section.\par
The elements concerning which statements of precision are to be made are identified using the same {\itshape target} and {\itshape match} attributes inherited from the \textsf{att.scoping} class discussed in the previous section and in the same way. Other aspects are provided by other attributes as further discussed below. 
\begin{sansreflist}
  
\item [\textbf{<precision>}] indicates the numerical accuracy or precision associated with some aspect of the text markup.\hfil\\[-10pt]\begin{sansreflist}
    \item[@{\itshape precision}]
  characterizes the precision of the element or attribute pointed to by the \hyperref[TEI.precision]{<precision>} element.
    \item[@{\itshape stdDeviation}]
  supplies a standard deviation associated with the value in question
\end{sansreflist}  
\end{sansreflist}
\par
In \textit{\hyperref[CONANU]{3.6.3.\ Numbers and Measures}} several ways of indicating ranges of values were introduced. For example, if we know that a date falls between 1930 and 1935, without being certain exactly where, this fact may be encoded using attributes {\itshape notBefore} and {\itshape notAfter}, as in the following example: \par\bgroup\index{date=<date>|exampleindex}\index{notBefore=@notBefore!<date>|exampleindex}\index{notAfter=@notAfter!<date>|exampleindex}\exampleFont \begin{shaded}\noindent\mbox{}{<\textbf{date}\hspace*{1em}{notBefore}="{1930}"\hspace*{1em}{notAfter}="{1935}">}Early in the 1930s{</\textbf{date}>}...\end{shaded}\egroup\par \noindent  Equally, if we know that every page of a manuscript has a width of at least 10 cm but no more than 30, we can use the attributes {\itshape atLeast} and {\itshape atMost}, as in the following examples: \par\bgroup\index{width=<width>|exampleindex}\index{atLeast=@atLeast!<width>|exampleindex}\index{atMost=@atMost!<width>|exampleindex}\index{unit=@unit!<width>|exampleindex}\index{scope=@scope!<width>|exampleindex}\exampleFont \begin{shaded}\noindent\mbox{}{<\textbf{width}\hspace*{1em}{atLeast}="{10}"\hspace*{1em}{atMost}="{30}"\hspace*{1em}{unit}="{cm}"\mbox{}\newline 
\hspace*{1em}{scope}="{all}"/>}\end{shaded}\egroup\par \par
Suppose however that the precision with which the value of such an attribute can be specified is variable. For example, suppose an event is dated ‘about fifty years after the death of Augustus’. In this case, the precision of one end of the range (the death of Augustus) is higher than the other, assuming we know when Augustus died. We can say that the latest possible date is probably 50 years after that, but with less confidence than we can attach to the earliest possible date.\par
The \hyperref[TEI.precision]{<precision>} element allows us to indicate the two attributes concerned and attach different levels of precision to them, using a similar mechanism as that provided for the \hyperref[TEI.certainty]{<certainty>} element: \par\bgroup\index{date=<date>|exampleindex}\index{notBefore=@notBefore!<date>|exampleindex}\index{notAfter=@notAfter!<date>|exampleindex}\index{precision=<precision>|exampleindex}\index{target=@target!<precision>|exampleindex}\index{match=@match!<precision>|exampleindex}\index{precision=@precision!<precision>|exampleindex}\index{precision=<precision>|exampleindex}\index{target=@target!<precision>|exampleindex}\index{match=@match!<precision>|exampleindex}\index{precision=@precision!<precision>|exampleindex}\exampleFont \begin{shaded}\noindent\mbox{}{<\textbf{date}\hspace*{1em}{xml:id}="{d001}"\hspace*{1em}{notBefore}="{0014}"\mbox{}\newline 
\hspace*{1em}{notAfter}="{0064}">}About 50\mbox{}\newline 
 years after the death of Augustus{</\textbf{date}>}\mbox{}\newline 
{<\textbf{precision}\hspace*{1em}{target}="{\#d001}"\hspace*{1em}{match}="{@notAfter}"\mbox{}\newline 
\hspace*{1em}{precision}="{low}"/>}\mbox{}\newline 
{<\textbf{precision}\hspace*{1em}{target}="{\#d001}"\mbox{}\newline 
\hspace*{1em}{match}="{@notBefore}"\hspace*{1em}{precision}="{high}"/>}\end{shaded}\egroup\par \par
In much the same way, we may wish to indicate different levels of precision about the dating of either end of a historical period. For example, the elements defined for encoding personal data all bear a similar set of attributes to indicate normalized values for earliest or latest dates, etc. (see section \textit{\hyperref[NDATTSda]{13.1.2.\ Dating Attributes}}); the precision of these attribute values may be indicated in exactly the same way. For example, \par\bgroup\index{residence=<residence>|exampleindex}\index{from=@from!<residence>|exampleindex}\index{notAfter=@notAfter!<residence>|exampleindex}\index{precision=<precision>|exampleindex}\index{match=@match!<precision>|exampleindex}\index{precision=@precision!<precision>|exampleindex}\exampleFont \begin{shaded}\noindent\mbox{}{<\textbf{residence}\hspace*{1em}{from}="{1857-03-01}"\mbox{}\newline 
\hspace*{1em}{notAfter}="{1857-04-30}">}From the 1st of March to\mbox{}\newline 
 some time in April of 1857.\mbox{}\newline 
{<\textbf{precision}\hspace*{1em}{match}="{@notAfter}"\mbox{}\newline 
\hspace*{1em}\hspace*{1em}{precision}="{medium}"/>}\mbox{}\newline 
{</\textbf{residence}>}\end{shaded}\egroup\par \par
It may also be useful to indicate that the precisions given for minimum and maximum quanta differ. For example, to indicate that all pages measure at least 10 cm wide, and at most \textit{about} 30: \par\bgroup\index{width=<width>|exampleindex}\index{atLeast=@atLeast!<width>|exampleindex}\index{atMost=@atMost!<width>|exampleindex}\index{unit=@unit!<width>|exampleindex}\index{scope=@scope!<width>|exampleindex}\index{precision=<precision>|exampleindex}\index{target=@target!<precision>|exampleindex}\index{match=@match!<precision>|exampleindex}\index{precision=@precision!<precision>|exampleindex}\exampleFont \begin{shaded}\noindent\mbox{}{<\textbf{width}\hspace*{1em}{xml:id}="{w00t}"\hspace*{1em}{atLeast}="{10}"\mbox{}\newline 
\hspace*{1em}{atMost}="{30}"\hspace*{1em}{unit}="{cm}"\hspace*{1em}{scope}="{all}"/>}\mbox{}\newline 
{<\textbf{precision}\hspace*{1em}{target}="{\#w00t}"\hspace*{1em}{match}="{@atMost}"\mbox{}\newline 
\hspace*{1em}{precision}="{low}"/>}\end{shaded}\egroup\par \par
The {\itshape stdDeviation} attribute may be used to indicate the standard deviation for a range of values. The generic \hyperref[TEI.dim]{<dim>} element introduced in \textit{\hyperref[msdim]{10.3.4.\ Dimensions}} might be used to record the average number of characters per line in a typescript. If in addition we wish to record the standard deviation for the values summarized by that average, this would require an additional \hyperref[TEI.precision]{<precision>} element, as in the following example: \par\bgroup\index{dim=<dim>|exampleindex}\index{type=@type!<dim>|exampleindex}\index{unit=@unit!<dim>|exampleindex}\index{quantity=@quantity!<dim>|exampleindex}\index{precision=<precision>|exampleindex}\index{target=@target!<precision>|exampleindex}\index{stdDeviation=@stdDeviation!<precision>|exampleindex}\exampleFont \begin{shaded}\noindent\mbox{}{<\textbf{dim}\hspace*{1em}{xml:id}="{dim1}"\hspace*{1em}{type}="{avgLineLength}"\mbox{}\newline 
\hspace*{1em}{unit}="{chars}"\hspace*{1em}{quantity}="{62.4}"/>}\mbox{}\newline 
{<\textbf{precision}\hspace*{1em}{target}="{\#dim1}"\hspace*{1em}{stdDeviation}="{4}"/>}\end{shaded}\egroup\par 
\subsection[{Attribution of Responsibility}]{Attribution of Responsibility}\label{CERESP}\par
In general, attribution of responsibility for the transcription and markup of an electronic text is made by \hyperref[TEI.respStmt]{<respStmt>} elements within the header: specifically, within the title statement, the edition statement(s), and the revision history.\par
In some cases, however, more detailed element-by-element information may be desired. For example, an encoder may wish to distinguish between the individuals responsible for transcribing the content and those responsible for determining that a given word or phrase constitutes a proper noun. Where such fine-grained attribution of responsibility is required, the \hyperref[TEI.respons]{<respons>} element can be used. 
\begin{sansreflist}
  
\item [\textbf{<respons>}] (responsibility) identifies the individual(s) responsible for some aspect of the content or markup of particular element(s).\hfil\\[-10pt]\begin{sansreflist}
    \item[@{\itshape locus}]
  indicates the specific aspect of the encoding (markup or content) for which responsibility is being assigned.
\end{sansreflist}  
\end{sansreflist}
\par
This element allows one or more aspects of the markup to be attributed to a given individual. This element inherits the {\itshape target} and {\itshape match} attributes from the \textsf{att.scoping} class, in the same way as the \hyperref[TEI.certainty]{<certainty>} and \hyperref[TEI.precision]{<precision>} elements. Its {\itshape locus} attribute functions in the same way as that on the \hyperref[TEI.certainty]{<certainty>} element (see \textit{\hyperref[CECERT]{21.1.\ Levels of Certainty}}). It inherits the {\itshape resp} and {\itshape cert} attributes from the \textsf{att.global.responsibility} class.\par
For example, the following encoding indicates that RC is responsible for transcribing an illegible word, and that PMWR is responsible for identifying that word as a proper noun, i.e. deciding to mark it with the \hyperref[TEI.persName]{<persName>} element at the location indicated: \par\bgroup\index{persName=<persName>|exampleindex}\index{rend=@rend!<persName>|exampleindex}\index{respons=<respons>|exampleindex}\index{target=@target!<respons>|exampleindex}\index{locus=@locus!<respons>|exampleindex}\index{resp=@resp!<respons>|exampleindex}\index{respons=<respons>|exampleindex}\index{target=@target!<respons>|exampleindex}\index{locus=@locus!<respons>|exampleindex}\index{resp=@resp!<respons>|exampleindex}\index{list=<list>|exampleindex}\index{type=@type!<list>|exampleindex}\index{item=<item>|exampleindex}\index{item=<item>|exampleindex}\exampleFont \begin{shaded}\noindent\mbox{}Ernest went to old\mbox{}\newline 
{<\textbf{persName}\hspace*{1em}{xml:id}="{CE-p5}"\hspace*{1em}{rend}="{it}">}Saybrook{</\textbf{persName}>}.\mbox{}\newline 
\mbox{}\newline 
\textit{<!-- ... -->}\mbox{}\newline 
{<\textbf{respons}\hspace*{1em}{target}="{\#CE-p5}"\hspace*{1em}{locus}="{value}"\mbox{}\newline 
\hspace*{1em}{resp}="{\#RC}"/>}\mbox{}\newline 
{<\textbf{respons}\hspace*{1em}{target}="{\#CE-p5}"\mbox{}\newline 
\hspace*{1em}{locus}="{name location}"\hspace*{1em}{resp}="{\#PMWR}"/>}\mbox{}\newline 
{<\textbf{list}\hspace*{1em}{type}="{encoders}">}\mbox{}\newline 
\hspace*{1em}{<\textbf{item}\hspace*{1em}{xml:id}="{PMWR}"/>}\mbox{}\newline 
\hspace*{1em}{<\textbf{item}\hspace*{1em}{xml:id}="{RC}"/>}\mbox{}\newline 
{</\textbf{list}>}\end{shaded}\egroup\par \par
Similarly, in the following example, we indicate that RC is responsible for proposing the value of the {\itshape rend} attribute: \par\bgroup\index{respons=<respons>|exampleindex}\index{target=@target!<respons>|exampleindex}\index{match=@match!<respons>|exampleindex}\index{locus=@locus!<respons>|exampleindex}\index{resp=@resp!<respons>|exampleindex}\exampleFont \begin{shaded}\noindent\mbox{}{<\textbf{respons}\hspace*{1em}{target}="{\#CE-p5}"\hspace*{1em}{match}="{@rend}"\mbox{}\newline 
\hspace*{1em}{locus}="{value}"\hspace*{1em}{resp}="{\#RC}"/>}\end{shaded}\egroup\par 
\subsection[{The Certainty Module}]{The Certainty Module}\par
The module described in this chapter makes available the following additional elements: \begin{description}

\item[{Module certainty: Certainty, Precision, and Responsibility}]\hspace{1em}\hfill\linebreak
\mbox{}\\[-10pt] \begin{itemize}
\item {\itshape Elements defined}: \hyperref[TEI.certainty]{certainty} \hyperref[TEI.precision]{precision} \hyperref[TEI.respons]{respons}
\end{itemize} 
\end{description}   The selection and combination of modules to form a TEI schema is described in \textit{\hyperref[STIN]{1.2.\ Defining a TEI Schema}}.