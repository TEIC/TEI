
\section[{Manuscript Description}]{Manuscript Description}\label{MS}
\subsection[{Overview}]{Overview}\label{msov}\par
The \textsf{msdescription} module\footnote{This chapter is based on the work of the European MASTER (Manuscript Access through Standards for Electronic Records) project, funded by the European Union from January 1999 to June 2001, and led by Peter Robinson, then at the Centre for Technology and the Arts at De Montfort University, Leicester (UK). Significant input also came from a TEI Workgroup headed by Consuelo W. Dutschke of the Rare Book and Manuscript Library, Columbia University (USA) and Ambrogio Piazzoni of the Biblioteca Apostolica Vaticana (IT) during 1998-2000.} defines a special purpose element which can be used to provide detailed descriptive information about handwritten primary sources and other text-bearing objects. Although originally developed to meet the needs of cataloguers and scholars working with medieval manuscripts in the European tradition, the scheme presented here is general enough that it can also be extended to other traditions and materials, and is potentially useful for any kind of text-bearing artefact. Where the textuality of an object is not the primary concern, encoders may wish to use the \hyperref[TEI.object]{<object>} element which provides a very similar system of description (see \textit{\hyperref[NDOBJ]{13.3.5.\ Objects}}.\par
The scheme described here is also intended to accommodate the needs of many different classes of encoders. On the one hand, encoders may be engaged in \textit{retrospective conversion} of existing detailed descriptions and catalogues into machine tractable form; on the other, they may be engaged in cataloguing \textit{ex nihilo}, that is, creating new detailed descriptions for materials never before catalogued. Some may be primarily concerned to represent accurately the description itself, as opposed to the ideas and interpretations the description represents; others may have entirely opposite priorities. At one extreme, a project may simply wish to capture an existing catalogue in a form that can be displayed on the Web, and which can be searched for literal strings, or for such features such as titles, authors and dates; at the other, a project may wish to create, in highly structured and encoded form, a detailed database of information about the physical characteristics, history, interpretation, etc. of the material, able to support practitioners of \textit{quantitative codicology} as well as librarians.\par
To cater for this diversity, here as elsewhere, these Guidelines propose a flexible strategy, in which encoders must choose for themselves the approach appropriate to their needs, and are provided with a choice of encoding mechanisms to support those differing degrees.
\subsection[{The Manuscript Description Element}]{The Manuscript Description Element}\label{msdesc}\par
The \hyperref[TEI.msDesc]{<msDesc>} element will normally appear within the \hyperref[TEI.sourceDesc]{<sourceDesc>} element of the header of a TEI-conformant document, where the document being encoded is a digital representation of some manuscript original, whether as an encoded transcription, as a collection of digital images (as described in \textit{\hyperref[PHFAX]{11.1.\ Digital Facsimiles}}), or as some combination of the two. However, in cases where the document being encoded is essentially a collection of manuscript descriptions, the \hyperref[TEI.msDesc]{<msDesc>} element may be used in the same way as the bibliographic elements (\hyperref[TEI.bibl]{<bibl>}, \hyperref[TEI.biblFull]{<biblFull>}, and \hyperref[TEI.biblStruct]{<biblStruct>}) making up the TEI element class \textsf{model.biblLike}. These typically appear within the \hyperref[TEI.listBibl]{<listBibl>} element. 
\begin{sansreflist}
  
\item [\textbf{<msDesc>}] (manuscript description) contains a description of a single identifiable manuscript or other text-bearing object such as early printed books.
\end{sansreflist}
\par
The \hyperref[TEI.msDesc]{<msDesc>} element has the following components, which provide more detailed information under a number of headings. Each of these component elements is further described in the remainder of this chapter. 
\begin{sansreflist}
  
\item [\textbf{<msIdentifier>}] (manuscript identifier) contains the information required to identify the manuscript or similar object being described.
\item [\textbf{<head>}] (heading) contains any type of heading, for example the title of a section, or the heading of a list, glossary, manuscript description, etc.
\item [\textbf{<msContents>}] (manuscript contents) describes the intellectual content of a manuscript, manuscript part, or other object either as a series of paragraphs or as a series of structured manuscript items.
\item [\textbf{<physDesc>}] (physical description) contains a full physical description of a manuscript, manuscript part, or other object optionally subdivided using more specialized elements from the \textsf{model.physDescPart} class.
\item [\textbf{<history>}] (history) groups elements describing the full history of a manuscript, manuscript part, or other object.
\item [\textbf{<additional>}] (additional) groups additional information, combining bibliographic information about a manuscript or other object, or surrogate copies of it, with curatorial or administrative information.
\item [\textbf{<msPart>}] (manuscript part) contains information about an originally distinct manuscript or part of a manuscript, which is now part of a composite manuscript.
\item [\textbf{<msFrag>}] (manuscript fragment) contains information about a fragment described in relation to a prior context, typically as a description of a virtual reconstruction of a manuscript or other object whose fragments were catalogued separately
\end{sansreflist}
\par
The first of these components, \hyperref[TEI.msIdentifier]{<msIdentifier>}, is the only one which is mandatory; it is described in more detail in \textit{\hyperref[msid]{10.4.\ The Manuscript Identifier}} below. It is followed optionally by one or more \hyperref[TEI.head]{<head>} elements, each holding a brief heading (see \textit{\hyperref[msdo]{10.5.\ The Manuscript Heading}}), and then either one or more paragraphs, marked up as a series of \hyperref[TEI.p]{<p>} elements, or one or more of the specialized elements \hyperref[TEI.msContents]{<msContents>} (\textit{\hyperref[msco]{10.6.\ Intellectual Content}}), \hyperref[TEI.physDesc]{<physDesc>} (\textit{\hyperref[msph]{10.7.\ Physical Description}}), \hyperref[TEI.history]{<history>} (\textit{\hyperref[mshy]{10.8.\ History}}), and \hyperref[TEI.additional]{<additional>} (\textit{\hyperref[msad]{10.9.\ Additional Information}}). These elements are all optional, but if used they must appear in the order given here. Finally, in the case of a composite manuscript (a manuscript composed of several codicological units) or a fragmented manuscript (a manuscript whose parts are now dispersed and kept at different places), a full description may also contain one or more \hyperref[TEI.msPart]{<msPart>} (\textit{\hyperref[mspt]{10.10.\ Manuscript Parts}}) elements and \hyperref[TEI.msFrag]{<msFrag>} (\textit{\hyperref[msfg]{10.11.\ Manuscript Fragments}}) elements, respectively.\par
To demonstrate the use of this module, consider the following sample manuscript description, chosen more or less at random from the Bodleian Library's \textit{Summary catalogue} (\hyperref[MS-eg-001]{[184]}) \begin{figure}[htbp]
\noindent\noindent\includegraphics[width=450pt,]{Images/MSadda61.png}
\caption{Entry for Bodleian MS. Add. A. 61 in Madan et al. 1895-1953}\end{figure}
\par
The simplest way of digitizing this catalogue entry would simply be to key in the text, tagging the relevant parts of it which make up the mandatory \hyperref[TEI.msIdentifier]{<msIdentifier>} element, as follows: \par\bgroup\index{msDesc=<msDesc>|exampleindex}\index{msIdentifier=<msIdentifier>|exampleindex}\index{settlement=<settlement>|exampleindex}\index{repository=<repository>|exampleindex}\index{idno=<idno>|exampleindex}\index{altIdentifier=<altIdentifier>|exampleindex}\index{type=@type!<altIdentifier>|exampleindex}\index{idno=<idno>|exampleindex}\index{p=<p>|exampleindex}\index{p=<p>|exampleindex}\index{p=<p>|exampleindex}\exampleFont \begin{shaded}\noindent\mbox{}{<\textbf{msDesc}>}\mbox{}\newline 
\hspace*{1em}{<\textbf{msIdentifier}>}\mbox{}\newline 
\hspace*{1em}\hspace*{1em}{<\textbf{settlement}>}Oxford{</\textbf{settlement}>}\mbox{}\newline 
\hspace*{1em}\hspace*{1em}{<\textbf{repository}>}Bodleian Library{</\textbf{repository}>}\mbox{}\newline 
\hspace*{1em}\hspace*{1em}{<\textbf{idno}>}MS. Add. A. 61{</\textbf{idno}>}\mbox{}\newline 
\hspace*{1em}\hspace*{1em}{<\textbf{altIdentifier}\hspace*{1em}{type}="{SC}">}\mbox{}\newline 
\hspace*{1em}\hspace*{1em}\hspace*{1em}{<\textbf{idno}>}28843{</\textbf{idno}>}\mbox{}\newline 
\hspace*{1em}\hspace*{1em}{</\textbf{altIdentifier}>}\mbox{}\newline 
\hspace*{1em}{</\textbf{msIdentifier}>}\mbox{}\newline 
\hspace*{1em}{<\textbf{p}>}In Latin, on parchment: written in more than one hand of the 13th cent. in\mbox{}\newline 
\hspace*{1em}\hspace*{1em} England: 7¼ x 5⅜ in., i + 55 leaves, in double columns: with a few coloured\mbox{}\newline 
\hspace*{1em}\hspace*{1em} capitals.{</\textbf{p}>}\mbox{}\newline 
\hspace*{1em}{<\textbf{p}>}'Hic incipit Bruitus Anglie,' the De origine et gestis Regum Angliae of\mbox{}\newline 
\hspace*{1em}\hspace*{1em} Geoffrey of Monmouth (Galfridus Monumetensis: beg. 'Cum mecum multa \& de\mbox{}\newline 
\hspace*{1em}\hspace*{1em} multis.'{</\textbf{p}>}\mbox{}\newline 
\hspace*{1em}{<\textbf{p}>}On fol. 54v very faint is 'Iste liber est fratris guillelmi de buria de ...\mbox{}\newline 
\hspace*{1em}\hspace*{1em} Roberti ordinis fratrum Pred[icatorum],' 14th cent. (?): 'hanauilla' is written\mbox{}\newline 
\hspace*{1em}\hspace*{1em} at the foot of the page (15th cent.). Bought from the rev. W. D. Macray on March\mbox{}\newline 
\hspace*{1em}\hspace*{1em} 17, 1863, for £1 10s.{</\textbf{p}>}\mbox{}\newline 
{</\textbf{msDesc}>}\end{shaded}\egroup\par \noindent  With a suitable stylesheet, this encoding would be as readable as the original; it would not, however, be very useful for search purposes since only shelfmarks and other identifiers are distinguished. To improve on this, one might wrap the paragraphs in the appropriate special-purpose first-child-level elements of \hyperref[TEI.msDesc]{<msDesc>} and add some of the additional phrase-level elements available when this module is in use: \par\bgroup\index{msDesc=<msDesc>|exampleindex}\index{msIdentifier=<msIdentifier>|exampleindex}\index{settlement=<settlement>|exampleindex}\index{repository=<repository>|exampleindex}\index{idno=<idno>|exampleindex}\index{altIdentifier=<altIdentifier>|exampleindex}\index{type=@type!<altIdentifier>|exampleindex}\index{idno=<idno>|exampleindex}\index{msContents=<msContents>|exampleindex}\index{p=<p>|exampleindex}\index{quote=<quote>|exampleindex}\index{title=<title>|exampleindex}\index{quote=<quote>|exampleindex}\index{physDesc=<physDesc>|exampleindex}\index{p=<p>|exampleindex}\index{material=<material>|exampleindex}\index{function=@function!<material>|exampleindex}\index{history=<history>|exampleindex}\index{p=<p>|exampleindex}\index{origPlace=<origPlace>|exampleindex}\index{origDate=<origDate>|exampleindex}\index{quote=<quote>|exampleindex}\index{quote=<quote>|exampleindex}\exampleFont \begin{shaded}\noindent\mbox{}{<\textbf{msDesc}>}\mbox{}\newline 
\hspace*{1em}{<\textbf{msIdentifier}>}\mbox{}\newline 
\hspace*{1em}\hspace*{1em}{<\textbf{settlement}>}Oxford{</\textbf{settlement}>}\mbox{}\newline 
\hspace*{1em}\hspace*{1em}{<\textbf{repository}>}Bodleian Library{</\textbf{repository}>}\mbox{}\newline 
\hspace*{1em}\hspace*{1em}{<\textbf{idno}>}MS. Add. A. 61{</\textbf{idno}>}\mbox{}\newline 
\hspace*{1em}\hspace*{1em}{<\textbf{altIdentifier}\hspace*{1em}{type}="{SC}">}\mbox{}\newline 
\hspace*{1em}\hspace*{1em}\hspace*{1em}{<\textbf{idno}>}28843{</\textbf{idno}>}\mbox{}\newline 
\hspace*{1em}\hspace*{1em}{</\textbf{altIdentifier}>}\mbox{}\newline 
\hspace*{1em}{</\textbf{msIdentifier}>}\mbox{}\newline 
\hspace*{1em}{<\textbf{msContents}>}\mbox{}\newline 
\hspace*{1em}\hspace*{1em}{<\textbf{p}>}\mbox{}\newline 
\hspace*{1em}\hspace*{1em}\hspace*{1em}{<\textbf{quote}>}Hic incipit Bruitus Anglie,{</\textbf{quote}>} the {<\textbf{title}>}De origine et gestis\mbox{}\newline 
\hspace*{1em}\hspace*{1em}\hspace*{1em}\hspace*{1em}\hspace*{1em}\hspace*{1em} Regum Angliae{</\textbf{title}>} of Geoffrey of Monmouth (Galfridus Monumetensis): beg.\mbox{}\newline 
\hspace*{1em}\hspace*{1em}{<\textbf{quote}>}Cum mecum multa \& de multis.{</\textbf{quote}>} In Latin.{</\textbf{p}>}\mbox{}\newline 
\hspace*{1em}{</\textbf{msContents}>}\mbox{}\newline 
\hspace*{1em}{<\textbf{physDesc}>}\mbox{}\newline 
\hspace*{1em}\hspace*{1em}{<\textbf{p}>}\mbox{}\newline 
\hspace*{1em}\hspace*{1em}\hspace*{1em}{<\textbf{material}\hspace*{1em}{function}="{support}">}Parchment{</\textbf{material}>}: written in\mbox{}\newline 
\hspace*{1em}\hspace*{1em}\hspace*{1em}\hspace*{1em} more than one hand: 7¼ x 5⅜ in., i + 55 leaves, in double columns:\mbox{}\newline 
\hspace*{1em}\hspace*{1em}\hspace*{1em}\hspace*{1em} with a few coloured capitals.{</\textbf{p}>}\mbox{}\newline 
\hspace*{1em}{</\textbf{physDesc}>}\mbox{}\newline 
\hspace*{1em}{<\textbf{history}>}\mbox{}\newline 
\hspace*{1em}\hspace*{1em}{<\textbf{p}>}Written in {<\textbf{origPlace}>}England{</\textbf{origPlace}>} in the {<\textbf{origDate}>}13th\mbox{}\newline 
\hspace*{1em}\hspace*{1em}\hspace*{1em}\hspace*{1em}\hspace*{1em}\hspace*{1em} cent.{</\textbf{origDate}>} On fol. 54v very faint is {<\textbf{quote}>}Iste liber est fratris\mbox{}\newline 
\hspace*{1em}\hspace*{1em}\hspace*{1em}\hspace*{1em}\hspace*{1em}\hspace*{1em} guillelmi de buria de ... Roberti ordinis fratrum Pred[icatorum],{</\textbf{quote}>} 14th\mbox{}\newline 
\hspace*{1em}\hspace*{1em}\hspace*{1em}\hspace*{1em} cent. (?): {<\textbf{quote}>}hanauilla{</\textbf{quote}>} is written at the foot of the page (15th\mbox{}\newline 
\hspace*{1em}\hspace*{1em}\hspace*{1em}\hspace*{1em} cent.). Bought from the rev. W. D. Macray on March 17, 1863, for £1 10s.{</\textbf{p}>}\mbox{}\newline 
\hspace*{1em}{</\textbf{history}>}\mbox{}\newline 
{</\textbf{msDesc}>}\end{shaded}\egroup\par \noindent  Note that in this version the text has been slightly reorganized, but no actual rewriting has been necessary. The encoding now allows the user to search for such features as title, material, and date and place of origin; it is also possible to distinguish quoted material from descriptive passages and to search within descriptions relating to a particular topic (for example, history as distinct from material).\par
This process could be continued further, restructuring the whole entry so as to take full advantage of many more of the encoding possibilities provided by the module described in this chapter: \par\bgroup\index{msDesc=<msDesc>|exampleindex}\index{msIdentifier=<msIdentifier>|exampleindex}\index{settlement=<settlement>|exampleindex}\index{repository=<repository>|exampleindex}\index{idno=<idno>|exampleindex}\index{altIdentifier=<altIdentifier>|exampleindex}\index{type=@type!<altIdentifier>|exampleindex}\index{idno=<idno>|exampleindex}\index{msContents=<msContents>|exampleindex}\index{msItem=<msItem>|exampleindex}\index{author=<author>|exampleindex}\index{author=<author>|exampleindex}\index{title=<title>|exampleindex}\index{type=@type!<title>|exampleindex}\index{rubric=<rubric>|exampleindex}\index{incipit=<incipit>|exampleindex}\index{textLang=<textLang>|exampleindex}\index{mainLang=@mainLang!<textLang>|exampleindex}\index{physDesc=<physDesc>|exampleindex}\index{objectDesc=<objectDesc>|exampleindex}\index{form=@form!<objectDesc>|exampleindex}\index{supportDesc=<supportDesc>|exampleindex}\index{material=@material!<supportDesc>|exampleindex}\index{support=<support>|exampleindex}\index{p=<p>|exampleindex}\index{extent=<extent>|exampleindex}\index{dimensions=<dimensions>|exampleindex}\index{scope=@scope!<dimensions>|exampleindex}\index{type=@type!<dimensions>|exampleindex}\index{unit=@unit!<dimensions>|exampleindex}\index{height=<height>|exampleindex}\index{width=<width>|exampleindex}\index{layoutDesc=<layoutDesc>|exampleindex}\index{layout=<layout>|exampleindex}\index{columns=@columns!<layout>|exampleindex}\index{p=<p>|exampleindex}\index{handDesc=<handDesc>|exampleindex}\index{p=<p>|exampleindex}\index{decoDesc=<decoDesc>|exampleindex}\index{p=<p>|exampleindex}\index{history=<history>|exampleindex}\index{origin=<origin>|exampleindex}\index{p=<p>|exampleindex}\index{origPlace=<origPlace>|exampleindex}\index{origDate=<origDate>|exampleindex}\index{notAfter=@notAfter!<origDate>|exampleindex}\index{notBefore=@notBefore!<origDate>|exampleindex}\index{provenance=<provenance>|exampleindex}\index{p=<p>|exampleindex}\index{quote=<quote>|exampleindex}\index{gap=<gap>|exampleindex}\index{ex=<ex>|exampleindex}\index{quote=<quote>|exampleindex}\index{acquisition=<acquisition>|exampleindex}\index{p=<p>|exampleindex}\index{name=<name>|exampleindex}\index{key=@key!<name>|exampleindex}\index{date=<date>|exampleindex}\index{when=@when!<date>|exampleindex}\exampleFont \begin{shaded}\noindent\mbox{}{<\textbf{msDesc}>}\mbox{}\newline 
\hspace*{1em}{<\textbf{msIdentifier}>}\mbox{}\newline 
\hspace*{1em}\hspace*{1em}{<\textbf{settlement}>}Oxford{</\textbf{settlement}>}\mbox{}\newline 
\hspace*{1em}\hspace*{1em}{<\textbf{repository}>}Bodleian Library{</\textbf{repository}>}\mbox{}\newline 
\hspace*{1em}\hspace*{1em}{<\textbf{idno}>}MS. Add. A. 61{</\textbf{idno}>}\mbox{}\newline 
\hspace*{1em}\hspace*{1em}{<\textbf{altIdentifier}\hspace*{1em}{type}="{SC}">}\mbox{}\newline 
\hspace*{1em}\hspace*{1em}\hspace*{1em}{<\textbf{idno}>}28843{</\textbf{idno}>}\mbox{}\newline 
\hspace*{1em}\hspace*{1em}{</\textbf{altIdentifier}>}\mbox{}\newline 
\hspace*{1em}{</\textbf{msIdentifier}>}\mbox{}\newline 
\hspace*{1em}{<\textbf{msContents}>}\mbox{}\newline 
\hspace*{1em}\hspace*{1em}{<\textbf{msItem}>}\mbox{}\newline 
\hspace*{1em}\hspace*{1em}\hspace*{1em}{<\textbf{author}\hspace*{1em}{xml:lang}="{en}">}Geoffrey of Monmouth{</\textbf{author}>}\mbox{}\newline 
\hspace*{1em}\hspace*{1em}\hspace*{1em}{<\textbf{author}\hspace*{1em}{xml:lang}="{la}">}Galfridus Monumetensis{</\textbf{author}>}\mbox{}\newline 
\hspace*{1em}\hspace*{1em}\hspace*{1em}{<\textbf{title}\hspace*{1em}{type}="{uniform}"\hspace*{1em}{xml:lang}="{la}">}De origine et gestis Regum Angliae{</\textbf{title}>}\mbox{}\newline 
\hspace*{1em}\hspace*{1em}\hspace*{1em}{<\textbf{rubric}\hspace*{1em}{xml:lang}="{la}">}Hic incipit Bruitus Anglie{</\textbf{rubric}>}\mbox{}\newline 
\hspace*{1em}\hspace*{1em}\hspace*{1em}{<\textbf{incipit}\hspace*{1em}{xml:lang}="{la}">}Cum mecum multa \& de multis{</\textbf{incipit}>}\mbox{}\newline 
\hspace*{1em}\hspace*{1em}\hspace*{1em}{<\textbf{textLang}\hspace*{1em}{mainLang}="{la}">}Latin{</\textbf{textLang}>}\mbox{}\newline 
\hspace*{1em}\hspace*{1em}{</\textbf{msItem}>}\mbox{}\newline 
\hspace*{1em}{</\textbf{msContents}>}\mbox{}\newline 
\hspace*{1em}{<\textbf{physDesc}>}\mbox{}\newline 
\hspace*{1em}\hspace*{1em}{<\textbf{objectDesc}\hspace*{1em}{form}="{codex}">}\mbox{}\newline 
\hspace*{1em}\hspace*{1em}\hspace*{1em}{<\textbf{supportDesc}\hspace*{1em}{material}="{perg}">}\mbox{}\newline 
\hspace*{1em}\hspace*{1em}\hspace*{1em}\hspace*{1em}{<\textbf{support}>}\mbox{}\newline 
\hspace*{1em}\hspace*{1em}\hspace*{1em}\hspace*{1em}\hspace*{1em}{<\textbf{p}>}Parchment.{</\textbf{p}>}\mbox{}\newline 
\hspace*{1em}\hspace*{1em}\hspace*{1em}\hspace*{1em}{</\textbf{support}>}\mbox{}\newline 
\hspace*{1em}\hspace*{1em}\hspace*{1em}\hspace*{1em}{<\textbf{extent}>}i + 55 leaves {<\textbf{dimensions}\hspace*{1em}{scope}="{all}"\hspace*{1em}{type}="{leaf}"\mbox{}\newline 
\hspace*{1em}\hspace*{1em}\hspace*{1em}\hspace*{1em}\hspace*{1em}\hspace*{1em}{unit}="{inch}">}\mbox{}\newline 
\hspace*{1em}\hspace*{1em}\hspace*{1em}\hspace*{1em}\hspace*{1em}\hspace*{1em}{<\textbf{height}>}7¼{</\textbf{height}>}\mbox{}\newline 
\hspace*{1em}\hspace*{1em}\hspace*{1em}\hspace*{1em}\hspace*{1em}\hspace*{1em}{<\textbf{width}>}5⅜{</\textbf{width}>}\mbox{}\newline 
\hspace*{1em}\hspace*{1em}\hspace*{1em}\hspace*{1em}\hspace*{1em}{</\textbf{dimensions}>}\mbox{}\newline 
\hspace*{1em}\hspace*{1em}\hspace*{1em}\hspace*{1em}{</\textbf{extent}>}\mbox{}\newline 
\hspace*{1em}\hspace*{1em}\hspace*{1em}{</\textbf{supportDesc}>}\mbox{}\newline 
\hspace*{1em}\hspace*{1em}\hspace*{1em}{<\textbf{layoutDesc}>}\mbox{}\newline 
\hspace*{1em}\hspace*{1em}\hspace*{1em}\hspace*{1em}{<\textbf{layout}\hspace*{1em}{columns}="{2}">}\mbox{}\newline 
\hspace*{1em}\hspace*{1em}\hspace*{1em}\hspace*{1em}\hspace*{1em}{<\textbf{p}>}In double columns.{</\textbf{p}>}\mbox{}\newline 
\hspace*{1em}\hspace*{1em}\hspace*{1em}\hspace*{1em}{</\textbf{layout}>}\mbox{}\newline 
\hspace*{1em}\hspace*{1em}\hspace*{1em}{</\textbf{layoutDesc}>}\mbox{}\newline 
\hspace*{1em}\hspace*{1em}{</\textbf{objectDesc}>}\mbox{}\newline 
\hspace*{1em}\hspace*{1em}{<\textbf{handDesc}>}\mbox{}\newline 
\hspace*{1em}\hspace*{1em}\hspace*{1em}{<\textbf{p}>}Written in more than one hand.{</\textbf{p}>}\mbox{}\newline 
\hspace*{1em}\hspace*{1em}{</\textbf{handDesc}>}\mbox{}\newline 
\hspace*{1em}\hspace*{1em}{<\textbf{decoDesc}>}\mbox{}\newline 
\hspace*{1em}\hspace*{1em}\hspace*{1em}{<\textbf{p}>}With a few coloured capitals.{</\textbf{p}>}\mbox{}\newline 
\hspace*{1em}\hspace*{1em}{</\textbf{decoDesc}>}\mbox{}\newline 
\hspace*{1em}{</\textbf{physDesc}>}\mbox{}\newline 
\hspace*{1em}{<\textbf{history}>}\mbox{}\newline 
\hspace*{1em}\hspace*{1em}{<\textbf{origin}>}\mbox{}\newline 
\hspace*{1em}\hspace*{1em}\hspace*{1em}{<\textbf{p}>}Written in {<\textbf{origPlace}>}England{</\textbf{origPlace}>} in the {<\textbf{origDate}\hspace*{1em}{notAfter}="{1300}"\mbox{}\newline 
\hspace*{1em}\hspace*{1em}\hspace*{1em}\hspace*{1em}\hspace*{1em}{notBefore}="{1200}">}13th cent.{</\textbf{origDate}>}\mbox{}\newline 
\hspace*{1em}\hspace*{1em}\hspace*{1em}{</\textbf{p}>}\mbox{}\newline 
\hspace*{1em}\hspace*{1em}{</\textbf{origin}>}\mbox{}\newline 
\hspace*{1em}\hspace*{1em}{<\textbf{provenance}>}\mbox{}\newline 
\hspace*{1em}\hspace*{1em}\hspace*{1em}{<\textbf{p}>}On fol. 54v very faint is {<\textbf{quote}\hspace*{1em}{xml:lang}="{la}">}Iste liber est fratris\mbox{}\newline 
\hspace*{1em}\hspace*{1em}\hspace*{1em}\hspace*{1em}\hspace*{1em}\hspace*{1em}\hspace*{1em}\hspace*{1em} guillelmi de buria de {<\textbf{gap}/>} Roberti ordinis fratrum\mbox{}\newline 
\hspace*{1em}\hspace*{1em}\hspace*{1em}\hspace*{1em}\hspace*{1em}\hspace*{1em}\hspace*{1em}\hspace*{1em} Pred{<\textbf{ex}>}icatorum{</\textbf{ex}>}\mbox{}\newline 
\hspace*{1em}\hspace*{1em}\hspace*{1em}\hspace*{1em}{</\textbf{quote}>}, 14th cent. (?): {<\textbf{quote}>}hanauilla{</\textbf{quote}>} is\mbox{}\newline 
\hspace*{1em}\hspace*{1em}\hspace*{1em}\hspace*{1em}\hspace*{1em}\hspace*{1em} written at the foot of the page (15th cent.).{</\textbf{p}>}\mbox{}\newline 
\hspace*{1em}\hspace*{1em}{</\textbf{provenance}>}\mbox{}\newline 
\hspace*{1em}\hspace*{1em}{<\textbf{acquisition}>}\mbox{}\newline 
\hspace*{1em}\hspace*{1em}\hspace*{1em}{<\textbf{p}>}Bought from the rev. {<\textbf{name}\hspace*{1em}{key}="{MCRAYWD}">}W. D. Macray{</\textbf{name}>} on {<\textbf{date}\hspace*{1em}{when}="{1863-03-17}">}March 17, 1863{</\textbf{date}>}, for £1 10s.{</\textbf{p}>}\mbox{}\newline 
\hspace*{1em}\hspace*{1em}{</\textbf{acquisition}>}\mbox{}\newline 
\hspace*{1em}{</\textbf{history}>}\mbox{}\newline 
{</\textbf{msDesc}>}\end{shaded}\egroup\par \noindent  In the remainder of this chapter we discuss all of the encoding features demonstrated above, together with many other related matters.
\subsection[{Phrase-level Elements}]{Phrase-level Elements}\label{msphrase}\par
When the \textsf{msdescription} module is in use, several extra elements are added to the phrase level class, and thus become available within paragraphs and elsewhere in the document. These elements are listed below in alphabetical order: 
\begin{sansreflist}
  
\item [\textbf{<catchwords>}] (catchwords) describes the system used to ensure correct ordering of the quires or similar making up a codex, incunable, or other object typically by means of annotations at the foot of the page.
\item [\textbf{<dimensions>}] (dimensions) contains a dimensional specification.
\item [\textbf{<heraldry>}] (heraldry) contains a heraldic formula or phrase, typically found as part of a blazon, coat of arms, etc. 
\item [\textbf{<locus>}] (locus) defines a location within a manuscript, manuscript part, or other object typically as a (possibly discontinuous) sequence of folio references.
\item [\textbf{<locusGrp>}] (locus group) groups a number of locations which together form a distinct but discontinuous item within a manuscript, manuscript part, or other object.
\item [\textbf{<material>}] (material) contains a word or phrase describing the material of which the object being described is composed.
\item [\textbf{<watermark>}] (watermark) contains a word or phrase describing a watermark or similar device.
\item [\textbf{<objectType>}] (object type) contains a word or phrase describing the type of object being referred to.
\item [\textbf{<origDate>}] (origin date) contains any form of date, used to identify the date of origin for a manuscript, manuscript part, or other object.
\item [\textbf{<origPlace>}] (origin place) contains any form of place name, used to identify the place of origin for a manuscript, manuscript part, or other object.
\item [\textbf{<secFol>}] (second folio) marks the word or words taken from a fixed point in a codex (typically the beginning of the second leaf) in order to provide a unique identifier for it. 
\item [\textbf{<signatures>}] (signatures) contains discussion of the leaf or quire signatures found within a codex or similar object.
\end{sansreflist}
\par
Within a manuscript description, many other standard TEI phrase level elements are available, notably those described in the Core module (\textit{\hyperref[CO]{3.\ Elements Available in All TEI Documents}}). Additional elements of particular relevance to manuscript description, such as those for names and dates, may also be made available by including the relevant module in one's schema.
\subsubsection[{Origination}]{Origination}\label{msdates}\par
The following elements may be used to provide information about the origins of any aspect of a manuscript: 
\begin{sansreflist}
  
\item [\textbf{<origDate>}] (origin date) contains any form of date, used to identify the date of origin for a manuscript, manuscript part, or other object.
\item [\textbf{<origPlace>}] (origin place) contains any form of place name, used to identify the place of origin for a manuscript, manuscript part, or other object.
\end{sansreflist}
\par
The \hyperref[TEI.origDate]{<origDate>} and \hyperref[TEI.origPlace]{<origPlace>} elements are specialized forms of the existing \hyperref[TEI.date]{<date>} and \hyperref[TEI.name]{<name>} elements respectively, used to indicate specifically the date and place of origin of a manuscript or manuscript part. Such information would normally be encoded within the \hyperref[TEI.history]{<history>} element, discussed in section \textit{\hyperref[mshy]{10.8.\ History}}. \hyperref[TEI.origDate]{<origDate>} and \hyperref[TEI.origPlace]{<origPlace>} can also be used to identify the place or date of origin of any aspect of the manuscript, such as its decoration or binding, when these are not of the same date or from the same location as rest of the manuscript. Both these elements are members of the \textsf{att.editLike} class, from which they inherit many attributes. \par
The \hyperref[TEI.origDate]{<origDate>} element is a member of the \textsf{att.datable} class, and may thus also carry additional attributes giving normalized values for the associated dating. 
\subsubsection[{Material and Object Type}]{Material and Object Type}\label{msmat}\par
The \hyperref[TEI.material]{<material>} element can be used to tag any specific term used for the physical material of which a manuscript (or binding, seal, etc.) is composed. The \hyperref[TEI.objectType]{<objectType>} element may be used to tag any term specifying the type of object or manuscript upon with the text is written. 
\begin{sansreflist}
  
\item [\textbf{<material>}] (material) contains a word or phrase describing the material of which the object being described is composed.\hfil\\[-10pt]\begin{sansreflist}
    \item[@{\itshape function}]
  describes the function or use of the material in relation to the object as a whole.
\end{sansreflist}  
\item [\textbf{<objectType>}] (object type) contains a word or phrase describing the type of object being referred to.
\end{sansreflist}
\par
These elements may appear wherever a term regarded as significant by the encoder occurs, as in the following examples: \par\bgroup\index{binding=<binding>|exampleindex}\index{p=<p>|exampleindex}\index{material=<material>|exampleindex}\exampleFont \begin{shaded}\noindent\mbox{}{<\textbf{binding}>}\mbox{}\newline 
\hspace*{1em}{<\textbf{p}>}Brown {<\textbf{material}>}calfskin{</\textbf{material}>}, previously with two clasps.{</\textbf{p}>}\mbox{}\newline 
{</\textbf{binding}>}\end{shaded}\egroup\par \noindent  \par\bgroup\index{support=<support>|exampleindex}\index{p=<p>|exampleindex}\index{material=<material>|exampleindex}\index{function=@function!<material>|exampleindex}\index{objectType=<objectType>|exampleindex}\index{material=<material>|exampleindex}\index{function=@function!<material>|exampleindex}\exampleFont \begin{shaded}\noindent\mbox{}{<\textbf{support}>}\mbox{}\newline 
\hspace*{1em}{<\textbf{p}>}\mbox{}\newline 
\hspace*{1em}\hspace*{1em}{<\textbf{material}\hspace*{1em}{function}="{support}">}Parchment{</\textbf{material}>}\mbox{}\newline 
\hspace*{1em}\hspace*{1em}{<\textbf{objectType}>}codex{</\textbf{objectType}>} with half {<\textbf{material}\hspace*{1em}{function}="{binding}">}goat-leather{</\textbf{material}>}\mbox{}\newline 
\hspace*{1em}\hspace*{1em} binding.{</\textbf{p}>}\mbox{}\newline 
{</\textbf{support}>}\end{shaded}\egroup\par 
\subsubsection[{Watermarks and Stamps}]{Watermarks and Stamps}\label{mswat}\par
Two further elements are provided to mark up other decorative features characteristic of manuscript leaves and bindings: 
\begin{sansreflist}
  
\item [\textbf{<watermark>}] (watermark) contains a word or phrase describing a watermark or similar device.
\item [\textbf{<stamp>}] (stamp) contains a word or phrase describing a stamp or similar device.
\end{sansreflist}
\par
These element may appear wherever a term regarded as significant by the encoder occurs. The \hyperref[TEI.watermark]{<watermark>} element is most likely to be of use within the \hyperref[TEI.support]{<support>} element discussed in \textit{\hyperref[msph1sup]{10.7.1.1.\ Support}} below. We give a simple example here: \par\bgroup\index{support=<support>|exampleindex}\index{material=<material>|exampleindex}\index{watermark=<watermark>|exampleindex}\exampleFont \begin{shaded}\noindent\mbox{}{<\textbf{support}>}\mbox{}\newline 
\hspace*{1em}{<\textbf{material}>}Rag\mbox{}\newline 
\hspace*{1em}\hspace*{1em} paper{</\textbf{material}>} with {<\textbf{watermark}>}anchor{</\textbf{watermark}>} watermark\mbox{}\newline 
{</\textbf{support}>}\end{shaded}\egroup\par \par
The \hyperref[TEI.stamp]{<stamp>} element will typically appear when text from the source is being transcribed, for example within a rubric in the following case: \par\bgroup\index{rubric=<rubric>|exampleindex}\index{lb=<lb>|exampleindex}\index{lb=<lb>|exampleindex}\index{lb=<lb>|exampleindex}\index{lb=<lb>|exampleindex}\index{stamp=<stamp>|exampleindex}\index{lb=<lb>|exampleindex}\exampleFont \begin{shaded}\noindent\mbox{}{<\textbf{rubric}>}\mbox{}\newline 
\hspace*{1em}{<\textbf{lb}/>}Apologyticu TTVLLIANI AC IGNORATIA IN XPO IHV\mbox{}\newline 
{<\textbf{lb}/>}SI NON LICET\mbox{}\newline 
{<\textbf{lb}/>}NOBIS RO\mbox{}\newline 
{<\textbf{lb}/>}manii imperii {<\textbf{stamp}>}Bodleian stamp{</\textbf{stamp}>}\mbox{}\newline 
\hspace*{1em}{<\textbf{lb}/>}\mbox{}\newline 
{</\textbf{rubric}>}\end{shaded}\egroup\par \par
It may also appear as part of the detailed description of a binding: \par\bgroup\index{binding=<binding>|exampleindex}\index{p=<p>|exampleindex}\index{stamp=<stamp>|exampleindex}\index{mentioned=<mentioned>|exampleindex}\exampleFont \begin{shaded}\noindent\mbox{}{<\textbf{binding}>}\mbox{}\newline 
\hspace*{1em}{<\textbf{p}>}Modern calf recasing with original armorial stamp {<\textbf{stamp}>}with legend\mbox{}\newline 
\hspace*{1em}\hspace*{1em}{<\textbf{mentioned}\hspace*{1em}{xml:lang}="{la}">}Ex Bibliotheca J. Richard\mbox{}\newline 
\hspace*{1em}\hspace*{1em}\hspace*{1em}\hspace*{1em}\hspace*{1em}\hspace*{1em} D.M.{</\textbf{mentioned}>}\mbox{}\newline 
\hspace*{1em}\hspace*{1em}{</\textbf{stamp}>}\mbox{}\newline 
\hspace*{1em}{</\textbf{p}>}\mbox{}\newline 
{</\textbf{binding}>}\end{shaded}\egroup\par \par
If, as here, any text contained by a stamp is included in its description it should be clearly distinguished from that description. The element \hyperref[TEI.mentioned]{<mentioned>} may be used for this purpose, as shown above.
\subsubsection[{Dimensions}]{Dimensions}\label{msdim}\par
The \hyperref[TEI.dimensions]{<dimensions>} element can be used to specify the size of some aspect of the manuscript, and thus may be thought of as a specialized form of the existing TEI \hyperref[TEI.measure]{<measure>} element. 
\begin{sansreflist}
  
\item [\textbf{<dimensions>}] (dimensions) contains a dimensional specification.\hfil\\[-10pt]\begin{sansreflist}
    \item[@{\itshape type}]
  indicates which aspect of the object is being measured.
\end{sansreflist}  
\end{sansreflist}
\par
The \hyperref[TEI.dimensions]{<dimensions>} element will normally occur within the element describing the particular feature or aspect of a manuscript whose dimensions are being given; thus the size of the leaves would be specified within the \hyperref[TEI.support]{<support>} or \hyperref[TEI.extent]{<extent>} element (part of the \hyperref[TEI.physDesc]{<physDesc>} element discussed in \textit{\hyperref[msph1]{10.7.1.\ Object Description}}), while the dimensions of other specific parts of a manuscript, such as accompanying materials, binding, etc., would be given in other parts of the description, as appropriate.\par
The following elements are available within the \hyperref[TEI.dimensions]{<dimensions>} element: 
\begin{sansreflist}
  
\item [\textbf{<height>}] (height) contains a measurement measured along the axis at a right angle to the bottom of the object.
\item [\textbf{<width>}] (width) contains a measurement of an object along the axis parallel to its bottom, e.g. perpendicular to the spine of a book or codex.
\item [\textbf{<depth>}] (depth) contains a measurement from the front to the back of an object, perpendicular to the measurement given by the \hyperref[TEI.width]{<width>} element.
\item [\textbf{<dim>}] contains any single measurement forming part of a dimensional specification of some sort.
\end{sansreflist}
\par
These elements, as well as \hyperref[TEI.dimensions]{<dimensions>} itself, are all members of the \textsf{att.dimensions} class, which also inherits attributes from the \textsf{att.ranging} class. They all thus carry the following attributes: 
\begin{sansreflist}
  
\item [\textbf{att.dimensions}] provides attributes for describing the size of physical objects.\hfil\\[-10pt]\begin{sansreflist}
    \item[@{\itshape scope}]
  where the measurement summarizes more than one observation, specifies the applicability of this measurement.
    \item[@{\itshape extent}]
  indicates the size of the object concerned using a project-specific vocabulary combining quantity and units in a single string of words.
    \item[@{\itshape unit}]
  names the unit used for the measurement
    \item[@{\itshape quantity}]
  specifies the length in the units specified
\end{sansreflist}  
\item [\textbf{att.ranging}] provides attributes for describing numerical ranges.\hfil\\[-10pt]\begin{sansreflist}
    \item[@{\itshape atLeast}]
  gives a minimum estimated value for the approximate measurement.
    \item[@{\itshape atMost}]
  gives a maximum estimated value for the approximate measurement.
    \item[@{\itshape min}]
  where the measurement summarizes more than one observation or a range, supplies the minimum value observed.
    \item[@{\itshape max}]
  where the measurement summarizes more than one observation or a range, supplies the maximum value observed.
\end{sansreflist}  
\end{sansreflist}
\par
Attributes {\itshape scope}, {\itshape min}, and {\itshape max} are used only when the measurement applies to several items, for example the size of all leaves in a manuscript; attributes {\itshape atLeast} and {\itshape atMost} are used when the measurement applies to a single item, for example the size of a specific codex, but has had to be estimated. Attribute {\itshape quantity} is used when the measurement can be given exactly, and applies to a single item; this is the usual situation. In this case, the units in which dimensions are measured may be specified using the {\itshape unit} attribute, the value of which will normally be taken from a closed set of values appropriate to the project, using standard units of measurement wherever possible, such as cm, mm, in, line, char. If however the only data available for the measurement uses some other unit, or it is preferred to normalize it in some other way, then it may be supplied as a string value by means of the {\itshape extent} attribute.\par
In the simplest case, only the {\itshape extent} attribute may be supplied: \par\bgroup\index{width=<width>|exampleindex}\index{extent=@extent!<width>|exampleindex}\exampleFont \begin{shaded}\noindent\mbox{}{<\textbf{width}\hspace*{1em}{extent}="{6 cubit}">}six\mbox{}\newline 
 cubits{</\textbf{width}>}\end{shaded}\egroup\par \noindent  More usually, the measurement will be normalized into a value and an appropriate SI unit: \par\bgroup\index{width=<width>|exampleindex}\index{quantity=@quantity!<width>|exampleindex}\index{unit=@unit!<width>|exampleindex}\exampleFont \begin{shaded}\noindent\mbox{}{<\textbf{width}\hspace*{1em}{quantity}="{270}"\hspace*{1em}{unit}="{cm}">}six cubits{</\textbf{width}>}\end{shaded}\egroup\par \noindent  Where the exact value is uncertain, the attributes {\itshape atLeast} and {\itshape atMost} may be used to indicate the upper and lower bounds of an estimated value: \par\bgroup\index{width=<width>|exampleindex}\index{atLeast=@atLeast!<width>|exampleindex}\index{atMost=@atMost!<width>|exampleindex}\index{unit=@unit!<width>|exampleindex}\exampleFont \begin{shaded}\noindent\mbox{}{<\textbf{width}\hspace*{1em}{atLeast}="{250}"\hspace*{1em}{atMost}="{300}"\hspace*{1em}{unit}="{cm}">}six cubits{</\textbf{width}>}\end{shaded}\egroup\par \par
It is often convenient to supply a measurement which applies to a number of discrete observations: for example, the number of ruled lines on the pages of a manuscript (which may not all be the same), or the diameter of an object like a bell, which will differ depending where it is measured. In such cases, the {\itshape scope} attribute may be used to specify the observations for which this measurement is applicable: \par\bgroup\index{height=<height>|exampleindex}\index{unit=@unit!<height>|exampleindex}\index{scope=@scope!<height>|exampleindex}\index{atLeast=@atLeast!<height>|exampleindex}\exampleFont \begin{shaded}\noindent\mbox{}{<\textbf{height}\hspace*{1em}{unit}="{line}"\hspace*{1em}{scope}="{most}"\mbox{}\newline 
\hspace*{1em}{atLeast}="{20}"/>}\end{shaded}\egroup\par \noindent  This indicates that most pages have at least 20 lines. The attributes {\itshape min} and {\itshape max} can also be used to specify the possible range of values: for example, to show that all pages have between 12 and 30 lines: \par\bgroup\index{height=<height>|exampleindex}\index{unit=@unit!<height>|exampleindex}\index{scope=@scope!<height>|exampleindex}\index{min=@min!<height>|exampleindex}\index{max=@max!<height>|exampleindex}\exampleFont \begin{shaded}\noindent\mbox{}{<\textbf{height}\hspace*{1em}{unit}="{line}"\hspace*{1em}{scope}="{all}"\hspace*{1em}{min}="{12}"\mbox{}\newline 
\hspace*{1em}{max}="{30}"/>}\end{shaded}\egroup\par \par
The \hyperref[TEI.dimensions]{<dimensions>} element may be repeated as often as necessary, with appropriate attribute values to indicate the nature and scope of the measurement concerned. For example, in the following case the leaf size and ruled space of the leaves of the manuscript are specified: \par\bgroup\index{dimensions=<dimensions>|exampleindex}\index{type=@type!<dimensions>|exampleindex}\index{unit=@unit!<dimensions>|exampleindex}\index{height=<height>|exampleindex}\index{scope=@scope!<height>|exampleindex}\index{quantity=@quantity!<height>|exampleindex}\index{unit=@unit!<height>|exampleindex}\index{width=<width>|exampleindex}\index{scope=@scope!<width>|exampleindex}\index{quantity=@quantity!<width>|exampleindex}\index{unit=@unit!<width>|exampleindex}\index{dimensions=<dimensions>|exampleindex}\index{type=@type!<dimensions>|exampleindex}\index{height=<height>|exampleindex}\index{min=@min!<height>|exampleindex}\index{max=@max!<height>|exampleindex}\index{unit=@unit!<height>|exampleindex}\index{width=<width>|exampleindex}\index{quantity=@quantity!<width>|exampleindex}\exampleFont \begin{shaded}\noindent\mbox{}{<\textbf{dimensions}\hspace*{1em}{type}="{ruled}"\hspace*{1em}{unit}="{mm}">}\mbox{}\newline 
\hspace*{1em}{<\textbf{height}\hspace*{1em}{scope}="{most}"\hspace*{1em}{quantity}="{90}"\mbox{}\newline 
\hspace*{1em}\hspace*{1em}{unit}="{mm}"/>}\mbox{}\newline 
\hspace*{1em}{<\textbf{width}\hspace*{1em}{scope}="{most}"\hspace*{1em}{quantity}="{48}"\hspace*{1em}{unit}="{mm}"/>}\mbox{}\newline 
{</\textbf{dimensions}>}\mbox{}\newline 
{<\textbf{dimensions}\hspace*{1em}{type}="{leaves}">}\mbox{}\newline 
\hspace*{1em}{<\textbf{height}\hspace*{1em}{min}="{157}"\hspace*{1em}{max}="{160}"\hspace*{1em}{unit}="{mm}"/>}\mbox{}\newline 
\hspace*{1em}{<\textbf{width}\hspace*{1em}{quantity}="{105}"/>}\mbox{}\newline 
{</\textbf{dimensions}>}\end{shaded}\egroup\par \noindent  This indicates that for most leaves of the manuscript being described the ruled space is 90 mm high and 48 mm wide, while the leaves throughout are between 157 and 160 mm in height and 105 mm in width.\par
The \hyperref[TEI.dim]{<dim>} element is provided for cases where some measurement other than height, width, or depth is required. Its {\itshape type} attribute is used to indicate the type of measurement involved: \par\bgroup\index{dimensions=<dimensions>|exampleindex}\index{unit=@unit!<dimensions>|exampleindex}\index{dim=<dim>|exampleindex}\index{type=@type!<dim>|exampleindex}\index{quantity=@quantity!<dim>|exampleindex}\index{unit=@unit!<dim>|exampleindex}\index{height=<height>|exampleindex}\index{quantity=@quantity!<height>|exampleindex}\index{unit=@unit!<height>|exampleindex}\exampleFont \begin{shaded}\noindent\mbox{}{<\textbf{dimensions}\hspace*{1em}{unit}="{cm}">}\mbox{}\newline 
\hspace*{1em}{<\textbf{dim}\hspace*{1em}{type}="{circumference}"\hspace*{1em}{quantity}="{48}"\mbox{}\newline 
\hspace*{1em}\hspace*{1em}{unit}="{mm}"/>}\mbox{}\newline 
\hspace*{1em}{<\textbf{height}\hspace*{1em}{quantity}="{90}"\hspace*{1em}{unit}="{mm}"/>}\mbox{}\newline 
{</\textbf{dimensions}>}\end{shaded}\egroup\par \par
The order in which components of the \hyperref[TEI.dimensions]{<dimensions>} element may be supplied is not constrained. 
\subsubsection[{References to Locations within a Manuscript}]{References to Locations within a Manuscript}\label{msloc}\par
The \hyperref[TEI.locus]{<locus>} and its grouping element \hyperref[TEI.locusGrp]{<locusGrp>} element are specialized forms of the \hyperref[TEI.ref]{<ref>} element, used to indicate a location, or sequence of locations, within a manuscript. 
\begin{sansreflist}
  
\item [\textbf{<locus>}] (locus) defines a location within a manuscript, manuscript part, or other object typically as a (possibly discontinuous) sequence of folio references.\hfil\\[-10pt]\begin{sansreflist}
    \item[@{\itshape from}]
  (from) specifies the starting point of the location in a normalized form, typically a page number.
    \item[@{\itshape to}]
  (to) specifies the end-point of the location in a normalized form, typically as a page number.
    \item[@{\itshape scheme}]
  (scheme) identifies the foliation scheme in terms of which the location is being specified by pointing to some \hyperref[TEI.foliation]{<foliation>} element defining it, or to some other equivalent resource.
\end{sansreflist}  
\item [\textbf{<locusGrp>}] (locus group) groups a number of locations which together form a distinct but discontinuous item within a manuscript, manuscript part, or other object.\hfil\\[-10pt]\begin{sansreflist}
    \item[@{\itshape scheme}]
  (scheme) identifies the foliation scheme in terms of which all the locations contained by the group are specified by pointing to some \hyperref[TEI.foliation]{<foliation>} element defining it, or to some other equivalent resource.
\end{sansreflist}  
\end{sansreflist}
\par
The \hyperref[TEI.locus]{<locus>} element is used to reference a single location within a manuscript, typically to specify the location occupied by the element within which it appears. If, for example, it is used as the first component of an \hyperref[TEI.msItem]{<msItem>} or \hyperref[TEI.msItemStruct]{<msItemStruct>} element, or of any of the more specific elements appearing within one (see further section \textit{\hyperref[msco]{10.6.\ Intellectual Content}} below) then it is understood to specify the location (or locations) of that item within the manuscript being described.
\paragraph[{Identifying a Location}]{Identifying a Location}\par
A \hyperref[TEI.locus]{<locus>} element can be used to identify any reference to one or more folios within a manuscript, wherever such a reference is appropriate. Locations are conventionally specified as a sequence of folio or page numbers, but may also be a discontinuous list, or a combination of the two. This specification should be given as the content of the \hyperref[TEI.locus]{<locus>} element, using the conventions appropriate to the individual scholar or holding institution, as in the following example: \par\bgroup\index{msItem=<msItem>|exampleindex}\index{n=@n!<msItem>|exampleindex}\index{locus=<locus>|exampleindex}\index{title=<title>|exampleindex}\exampleFont \begin{shaded}\noindent\mbox{}{<\textbf{msItem}\hspace*{1em}{n}="{1}">}\mbox{}\newline 
\hspace*{1em}{<\textbf{locus}>}ff. 1-24r{</\textbf{locus}>}\mbox{}\newline 
\hspace*{1em}{<\textbf{title}>}Apocalypsis beati Ioannis Apostoli{</\textbf{title}>}\mbox{}\newline 
{</\textbf{msItem}>}\end{shaded}\egroup\par \par
A normalized form of the location can also be supplied, using special purpose attributes on the \hyperref[TEI.locus]{<locus>} element, as in the following revision of the above example: \par\bgroup\index{msItem=<msItem>|exampleindex}\index{n=@n!<msItem>|exampleindex}\index{locus=<locus>|exampleindex}\index{from=@from!<locus>|exampleindex}\index{to=@to!<locus>|exampleindex}\index{title=<title>|exampleindex}\exampleFont \begin{shaded}\noindent\mbox{}{<\textbf{msItem}\hspace*{1em}{n}="{1}">}\mbox{}\newline 
\hspace*{1em}{<\textbf{locus}\hspace*{1em}{from}="{1r}"\hspace*{1em}{to}="{24r}">}ff. 1-24r{</\textbf{locus}>}\mbox{}\newline 
\hspace*{1em}{<\textbf{title}>}Apocalypsis beati Ioannis Apostoli{</\textbf{title}>}\mbox{}\newline 
{</\textbf{msItem}>}\end{shaded}\egroup\par \par
When the item concerned occupies a discontinuous sequence of pages, this may simply be indicated in the body of the \hyperref[TEI.locus]{<locus>} element: \par\bgroup\index{msItem=<msItem>|exampleindex}\index{n=@n!<msItem>|exampleindex}\index{locus=<locus>|exampleindex}\index{title=<title>|exampleindex}\exampleFont \begin{shaded}\noindent\mbox{}{<\textbf{msItem}\hspace*{1em}{n}="{1}">}\mbox{}\newline 
\hspace*{1em}{<\textbf{locus}>}ff. 1-12v, 18-24r{</\textbf{locus}>}\mbox{}\newline 
\hspace*{1em}{<\textbf{title}>}Apocalypsis beati Ioannis Apostoli{</\textbf{title}>}\mbox{}\newline 
{</\textbf{msItem}>}\end{shaded}\egroup\par \noindent  Alternatively, if it is desired to indicate normalized values for each part of the sequence, a sequence of \hyperref[TEI.locus]{<locus>} elements can be supplied, grouped within the \hyperref[TEI.locusGrp]{<locusGrp>} element: \par\bgroup\index{msItem=<msItem>|exampleindex}\index{n=@n!<msItem>|exampleindex}\index{locusGrp=<locusGrp>|exampleindex}\index{locus=<locus>|exampleindex}\index{from=@from!<locus>|exampleindex}\index{to=@to!<locus>|exampleindex}\index{locus=<locus>|exampleindex}\index{from=@from!<locus>|exampleindex}\index{to=@to!<locus>|exampleindex}\index{title=<title>|exampleindex}\exampleFont \begin{shaded}\noindent\mbox{}{<\textbf{msItem}\hspace*{1em}{n}="{1}">}\mbox{}\newline 
\hspace*{1em}{<\textbf{locusGrp}>}\mbox{}\newline 
\hspace*{1em}\hspace*{1em}{<\textbf{locus}\hspace*{1em}{from}="{1r}"\hspace*{1em}{to}="{12v}">}ff. 1-12v{</\textbf{locus}>}\mbox{}\newline 
\hspace*{1em}\hspace*{1em}{<\textbf{locus}\hspace*{1em}{from}="{18}"\hspace*{1em}{to}="{24r}">}ff. 18-24r{</\textbf{locus}>}\mbox{}\newline 
\hspace*{1em}{</\textbf{locusGrp}>}\mbox{}\newline 
\hspace*{1em}{<\textbf{title}>}Apocalypsis beati Ioannis Apostoli{</\textbf{title}>}\mbox{}\newline 
{</\textbf{msItem}>}\end{shaded}\egroup\par \noindent  If an existing catalogue is being transcribed and it is desirable to retain formatting of the reference (e.g. superscript or italic text) then the \hyperref[TEI.hi]{<hi>} element may be used. If encoding multiple semantic divisions in a single location reference then a nested \hyperref[TEI.locus]{<locus>} may be used to separate or annotate these.\par
Finally, the content of the \hyperref[TEI.locus]{<locus>} element may be omitted if a formatting application can construct it automatically from the values of the {\itshape from} and {\itshape to} attributes: \par\bgroup\index{msItem=<msItem>|exampleindex}\index{n=@n!<msItem>|exampleindex}\index{locusGrp=<locusGrp>|exampleindex}\index{locus=<locus>|exampleindex}\index{from=@from!<locus>|exampleindex}\index{to=@to!<locus>|exampleindex}\index{locus=<locus>|exampleindex}\index{from=@from!<locus>|exampleindex}\index{to=@to!<locus>|exampleindex}\index{title=<title>|exampleindex}\exampleFont \begin{shaded}\noindent\mbox{}{<\textbf{msItem}\hspace*{1em}{n}="{1}">}\mbox{}\newline 
\hspace*{1em}{<\textbf{locusGrp}>}\mbox{}\newline 
\hspace*{1em}\hspace*{1em}{<\textbf{locus}\hspace*{1em}{from}="{1r}"\hspace*{1em}{to}="{12v}"/>}\mbox{}\newline 
\hspace*{1em}\hspace*{1em}{<\textbf{locus}\hspace*{1em}{from}="{18}"\hspace*{1em}{to}="{24r}"/>}\mbox{}\newline 
\hspace*{1em}{</\textbf{locusGrp}>}\mbox{}\newline 
\hspace*{1em}{<\textbf{title}>}Apocalypsis beati Ioannis Apostoli{</\textbf{title}>}\mbox{}\newline 
{</\textbf{msItem}>}\end{shaded}\egroup\par 
\paragraph[{Linking a Location to a Transcription or an Image}]{Linking a Location to a Transcription or an Image}\par
The \hyperref[TEI.locus]{<locus>} attribute can also be used to associate a location within a manuscript with facsimile images of that location, using the {\itshape facs} attribute, or with a transcription of the text occurring at that location. The former association is effected by means of the {\itshape facs} attribute; the latter by means of the {\itshape target} attribute.\par
The {\itshape facs} is available only when the \textsf{transcr} module described in chapter \textit{\hyperref[PH]{11.\ Representation of Primary Sources}} is included in a schema. It associates a \hyperref[TEI.locus]{<locus>} element with one or more digitized images, as in the following example: \par\bgroup\index{msItem=<msItem>|exampleindex}\index{locus=<locus>|exampleindex}\index{facs=@facs!<locus>|exampleindex}\index{title=<title>|exampleindex}\index{bibl=<bibl>|exampleindex}\index{title=<title>|exampleindex}\index{biblScope=<biblScope>|exampleindex}\exampleFont \begin{shaded}\noindent\mbox{}{<\textbf{msItem}>}\mbox{}\newline 
\hspace*{1em}{<\textbf{locus}\hspace*{1em}{facs}="{images/08v.jpg images/09r.jpg images/09v.jpg images/10r.jpg images/10v.jpg}">}fols. 8v-10v{</\textbf{locus}>}\mbox{}\newline 
\hspace*{1em}{<\textbf{title}>}Birds Praise of Love{</\textbf{title}>}\mbox{}\newline 
\hspace*{1em}{<\textbf{bibl}>}\mbox{}\newline 
\hspace*{1em}\hspace*{1em}{<\textbf{title}>}IMEV{</\textbf{title}>}\mbox{}\newline 
\hspace*{1em}\hspace*{1em}{<\textbf{biblScope}>}1506{</\textbf{biblScope}>}\mbox{}\newline 
\hspace*{1em}{</\textbf{bibl}>}\mbox{}\newline 
{</\textbf{msItem}>}\end{shaded}\egroup\par \noindent  Here, the {\itshape facs} attribute uses a URI reference to point directly to images of the relevant pages. This method may be found cumbersome when many images are to be associated with a single location. It is of most use when specific pages are referenced within a description, as in the following example: \par\bgroup\index{decoDesc=<decoDesc>|exampleindex}\index{p=<p>|exampleindex}\index{locus=<locus>|exampleindex}\index{facs=@facs!<locus>|exampleindex}\index{locus=<locus>|exampleindex}\index{facs=@facs!<locus>|exampleindex}\exampleFont \begin{shaded}\noindent\mbox{}{<\textbf{decoDesc}>}\mbox{}\newline 
\hspace*{1em}{<\textbf{p}>}Several of the miniatures in this section have been damaged and overpainted\mbox{}\newline 
\hspace*{1em}\hspace*{1em} at a later date (e.g. the figure of Christ on {<\textbf{locus}\hspace*{1em}{facs}="{http://www.example.com/images.fr\#F33R}">}fol. 33r{</\textbf{locus}>}; the face of the\mbox{}\newline 
\hspace*{1em}\hspace*{1em} Shepherdess on {<\textbf{locus}\hspace*{1em}{facs}="{http://www.example.com/images.fr\#F59V}">}fol.\mbox{}\newline 
\hspace*{1em}\hspace*{1em}\hspace*{1em}\hspace*{1em} 59v{</\textbf{locus}>}, etc.).{</\textbf{p}>}\mbox{}\newline 
{</\textbf{decoDesc}>}\end{shaded}\egroup\par \noindent  For further discussion of the {\itshape facs} attribute, see section \textit{\hyperref[PHFAX]{11.1.\ Digital Facsimiles}}.\par
Where a transcription of the relevant pages is available, this may be associated with the \hyperref[TEI.locus]{<locus>} element using its {\itshape target} attribute, as in the following example: \par\bgroup\index{msItem=<msItem>|exampleindex}\index{n=@n!<msItem>|exampleindex}\index{locus=<locus>|exampleindex}\index{target=@target!<locus>|exampleindex}\index{author=<author>|exampleindex}\index{title=<title>|exampleindex}\index{rubric=<rubric>|exampleindex}\index{rend=@rend!<rubric>|exampleindex}\index{lb=<lb>|exampleindex}\index{lb=<lb>|exampleindex}\index{incipit=<incipit>|exampleindex}\index{explicit=<explicit>|exampleindex}\index{bibl=<bibl>|exampleindex}\index{name=<name>|exampleindex}\index{title=<title>|exampleindex}\index{pb=<pb>|exampleindex}\index{pb=<pb>|exampleindex}\index{pb=<pb>|exampleindex}\exampleFont \begin{shaded}\noindent\mbox{}{<\textbf{msItem}\hspace*{1em}{n}="{1}">}\mbox{}\newline 
\hspace*{1em}{<\textbf{locus}\hspace*{1em}{target}="{\#f1r \#f1v \#f2r}">}ff. 1r-2r{</\textbf{locus}>}\mbox{}\newline 
\hspace*{1em}{<\textbf{author}>}Ben Jonson{</\textbf{author}>}\mbox{}\newline 
\hspace*{1em}{<\textbf{title}>}Ode to himself{</\textbf{title}>}\mbox{}\newline 
\hspace*{1em}{<\textbf{rubric}\hspace*{1em}{rend}="{italics}">}\mbox{}\newline 
\hspace*{1em}\hspace*{1em}{<\textbf{lb}/>} An Ode\mbox{}\newline 
\hspace*{1em}{<\textbf{lb}/>} to him selfe.{</\textbf{rubric}>}\mbox{}\newline 
\hspace*{1em}{<\textbf{incipit}>}Com leaue the loathed stage{</\textbf{incipit}>}\mbox{}\newline 
\hspace*{1em}{<\textbf{explicit}>}And see his chariot triumph ore his wayne.{</\textbf{explicit}>}\mbox{}\newline 
\hspace*{1em}{<\textbf{bibl}>}\mbox{}\newline 
\hspace*{1em}\hspace*{1em}{<\textbf{name}>}Beal{</\textbf{name}>}, {<\textbf{title}>}Index 1450-1625{</\textbf{title}>}, JnB 380{</\textbf{bibl}>}\mbox{}\newline 
{</\textbf{msItem}>}\mbox{}\newline 
\textit{<!-- within transcription ... -->}\mbox{}\newline 
{<\textbf{pb}\hspace*{1em}{xml:id}="{f1r}"/>}\mbox{}\newline 
\textit{<!-- ... -->}\mbox{}\newline 
{<\textbf{pb}\hspace*{1em}{xml:id}="{f1v}"/>}\mbox{}\newline 
\textit{<!-- ... -->}\mbox{}\newline 
{<\textbf{pb}\hspace*{1em}{xml:id}="{f2r}"/>}\mbox{}\newline 
\textit{<!-- ... -->}\end{shaded}\egroup\par \par
When (as in this example) a sequence of elements is to be supplied as target value, it may be given explicitly as above, or using the xPointer range() syntax defined at \textit{\hyperref[SATSRN]{16.2.4.6.\ range()}}. Note however that support for this pointer mechanism is not widespread in current XML processing systems.\par
The {\itshape target} attribute should only be used to point to elements that contain or indicate a transcription of the locus being described. To associate a \hyperref[TEI.locus]{<locus>} element with a page image or other comparable representation, the global {\itshape facs} attribute should be used instead.
\paragraph[{Using Multiple Location Schemes}]{Using Multiple Location Schemes}\par
Where a manuscript contains more than one foliation, the {\itshape scheme} attribute may be used to distinguish them. For example, MS 65 Corpus Christi College, Cambridge contains two fly leaves bearing music. These leaves have modern foliation 135 and 136 respectively, but are also marked with an older foliation. This may be preserved in an encoding such as the following: \par\bgroup\index{locus=<locus>|exampleindex}\index{scheme=@scheme!<locus>|exampleindex}\index{locus=<locus>|exampleindex}\index{scheme=@scheme!<locus>|exampleindex}\exampleFont \begin{shaded}\noindent\mbox{}{<\textbf{locus}\hspace*{1em}{scheme}="{\#original}">}XCIII{</\textbf{locus}>}\mbox{}\newline 
{<\textbf{locus}\hspace*{1em}{scheme}="{\#modern}">}135{</\textbf{locus}>}\end{shaded}\egroup\par \noindent  Here the {\itshape scheme} attribute points to a \hyperref[TEI.foliation]{<foliation>} element providing more details about the scheme used, as further discussed in \textit{\hyperref[msphfo]{10.7.1.4.\ Foliation}} below.\par
Where discontinuous sequences are identified within two different foliations, the {\itshape scheme} attribute should be supplied on the \hyperref[TEI.locusGrp]{<locusGrp>} element in preference, as in the following: \par\bgroup\index{locusGrp=<locusGrp>|exampleindex}\index{scheme=@scheme!<locusGrp>|exampleindex}\index{locus=<locus>|exampleindex}\index{locus=<locus>|exampleindex}\index{locusGrp=<locusGrp>|exampleindex}\index{scheme=@scheme!<locusGrp>|exampleindex}\index{locus=<locus>|exampleindex}\index{locus=<locus>|exampleindex}\exampleFont \begin{shaded}\noindent\mbox{}{<\textbf{locusGrp}\hspace*{1em}{scheme}="{\#original}">}\mbox{}\newline 
\hspace*{1em}{<\textbf{locus}>}XCIII{</\textbf{locus}>}\mbox{}\newline 
\hspace*{1em}{<\textbf{locus}>}CC-CCII{</\textbf{locus}>}\mbox{}\newline 
{</\textbf{locusGrp}>}\mbox{}\newline 
{<\textbf{locusGrp}\hspace*{1em}{scheme}="{\#modern}">}\mbox{}\newline 
\hspace*{1em}{<\textbf{locus}>}135{</\textbf{locus}>}\mbox{}\newline 
\hspace*{1em}{<\textbf{locus}>}197-204{</\textbf{locus}>}\mbox{}\newline 
{</\textbf{locusGrp}>}\end{shaded}\egroup\par 
\subsubsection[{Names of Persons, Places, and Organizations}]{Names of Persons, Places, and Organizations}\label{msnames}\par
The standard TEI element \hyperref[TEI.name]{<name>} may be used to identify names of any kind occurring within a description: 
\begin{sansreflist}
  
\item [\textbf{<name>}] (name, proper noun) contains a proper noun or noun phrase.
\end{sansreflist}
 As further discussed in \textit{\hyperref[CONARS]{3.6.1.\ Referring Strings}}, this element is a member of the class \textsf{att.canonical}, from which it inherits the following attributes: 
\begin{sansreflist}
  
\item [\textbf{att.canonical}] provides attributes which can be used to associate a representation such as a name or title with canonical information about the object being named or referenced.\hfil\\[-10pt]\begin{sansreflist}
    \item[@{\itshape key}]
  provides an externally-defined means of identifying the entity (or entities) being named, using a coded value of some kind.
    \item[@{\itshape ref}]
  (reference) provides an explicit means of locating a full definition or identity for the entity being named by means of one or more URIs.
\end{sansreflist}  
\end{sansreflist}
\par
Here are some examples of the use of the \hyperref[TEI.name]{<name>} element: \par\bgroup\index{name=<name>|exampleindex}\index{type=@type!<name>|exampleindex}\index{name=<name>|exampleindex}\index{type=@type!<name>|exampleindex}\index{name=<name>|exampleindex}\index{type=@type!<name>|exampleindex}\index{name=<name>|exampleindex}\index{type=@type!<name>|exampleindex}\index{ref=@ref!<name>|exampleindex}\exampleFont \begin{shaded}\noindent\mbox{}{<\textbf{name}\hspace*{1em}{type}="{person}">}Thomas Hoccleve{</\textbf{name}>}\mbox{}\newline 
{<\textbf{name}\hspace*{1em}{type}="{place}">}Villingaholt{</\textbf{name}>}\mbox{}\newline 
{<\textbf{name}\hspace*{1em}{type}="{org}">}Vetus Latina Institut{</\textbf{name}>}\mbox{}\newline 
{<\textbf{name}\hspace*{1em}{type}="{person}"\hspace*{1em}{ref}="{\#HOC001}">}Occleve{</\textbf{name}>}\end{shaded}\egroup\par \par
Note that the \hyperref[TEI.name]{<name>} element is defined as providing information about a \textit{name}, not the person, place, or organization to which that name refers. In the last example above, the {\itshape ref} attribute is used to associate the name with a more detailed description of the person named. This is provided by means of the \hyperref[TEI.person]{<person>} element, which becomes available when the \textsf{namesdates} module described in chapter \textit{\hyperref[ND]{13.\ Names, Dates, People, and Places}} is included in a schema. An element such as the following might then be used to provide detailed information about the person indicated by the name: \par\bgroup\index{person=<person>|exampleindex}\index{persName=<persName>|exampleindex}\index{surname=<surname>|exampleindex}\index{forename=<forename>|exampleindex}\index{birth=<birth>|exampleindex}\index{notBefore=@notBefore!<birth>|exampleindex}\index{occupation=<occupation>|exampleindex}\exampleFont \begin{shaded}\noindent\mbox{}{<\textbf{person}\hspace*{1em}{xml:id}="{HOC001}">}\mbox{}\newline 
\hspace*{1em}{<\textbf{persName}>}\mbox{}\newline 
\hspace*{1em}\hspace*{1em}{<\textbf{surname}>}Hoccleve{</\textbf{surname}>}\mbox{}\newline 
\hspace*{1em}\hspace*{1em}{<\textbf{forename}>}Thomas{</\textbf{forename}>}\mbox{}\newline 
\hspace*{1em}{</\textbf{persName}>}\mbox{}\newline 
\hspace*{1em}{<\textbf{birth}\hspace*{1em}{notBefore}="{1368}"/>}\mbox{}\newline 
\hspace*{1em}{<\textbf{occupation}>}poet{</\textbf{occupation}>}\mbox{}\newline 
\textit{<!-- other personal data -->}\mbox{}\newline 
{</\textbf{person}>}\end{shaded}\egroup\par \noindent   Note that an instance of the \hyperref[TEI.person]{<person>} element must be provided for each distinct {\itshape ref} value specified. For example, in the case above, the value HOC001 must be found as the {\itshape xml:id} attribute of some \hyperref[TEI.person]{<person>} element; the same value will be used as the {\itshape ref} attribute of every reference to Hoccleve in the document (however spelled), but there will only be one \hyperref[TEI.person]{<person>} element with this identifier.\par
Alternatively, the {\itshape key} attribute may be used to supply a unique identifying code for the person referenced by the name independently of both the existence of a \hyperref[TEI.person]{<person>} element and the use of the standard URI reference mechanism. If, for example, a project maintains as its authority file some non-digital resource, or uses a database which cannot readily be integrated with other digital resources for this purpose, the unique codes used by such ‘offline’ resources may be used as values for the {\itshape key} attribute. Although such practices clearly reduce the interchangeability of the resulting encoded texts, they may be judged more convenient or practical in certain situations. As explained in \textit{\hyperref[CONARS]{3.6.1.\ Referring Strings}}, interchange is improved by use of tag URIs in {\itshape ref} instead of {\itshape key}.\par
All the \hyperref[TEI.person]{<person>} elements referenced by a particular document set should be collected together within a \hyperref[TEI.listPerson]{<listPerson>}  element, located in a \hyperref[TEI.standOff]{<standOff>} element. This functions as a kind of prosopography for all the people referenced by the set of manuscripts being described, in much the same way as a \hyperref[TEI.listBibl]{<listBibl>} element may be used to hold bibliographic information for all the works referenced.\par
When the \textsf{namesdates} module described in chapter \textit{\hyperref[ND]{13.\ Names, Dates, People, and Places}} is included in a schema, similar mechanisms are used to maintain and reference canonical lists of places or organizations, as further discussed in sections \textit{\hyperref[NDPLAC]{13.2.3.\ Place Names}} and \textit{\hyperref[NDORG]{13.2.2.\ Organizational Names}} respectively.
\subsubsection[{Catchwords, Signatures, Secundo Folio}]{Catchwords, Signatures, Secundo Folio}\label{msmisc}\par
The \hyperref[TEI.catchwords]{<catchwords>} element is used to describe one method by which correct ordering of the quires of a codex is ensured. Typically, this takes the form of a word or phrase written in the lower margin of the last leaf verso of a gathering, which provides a preview of the first recto leaf of the successive gathering. This may be a simple phrase such as the following: \par\bgroup\index{catchwords=<catchwords>|exampleindex}\exampleFont \begin{shaded}\noindent\mbox{}{<\textbf{catchwords}>}Quires signed on the\mbox{}\newline 
 last leaf verso in roman numerals.{</\textbf{catchwords}>}\end{shaded}\egroup\par \noindent  Alternatively, it may contain more details: \par\bgroup\index{catchwords=<catchwords>|exampleindex}\exampleFont \begin{shaded}\noindent\mbox{}{<\textbf{catchwords}>}Vertical catchwords in the hand of the scribe placed along the inner\mbox{}\newline 
 bounding line, reading from top to bottom.{</\textbf{catchwords}>}\end{shaded}\egroup\par \par
The ‘Signatures’ element is used, in a similar way, to describe a similar system in which quires or leaves are marked progressively in order to facilitate arrangement during binding. For example: \par\bgroup\index{signatures=<signatures>|exampleindex}\exampleFont \begin{shaded}\noindent\mbox{}{<\textbf{signatures}>}At the bottom of the\mbox{}\newline 
 first four leaves of quires 1-14 are the remains of a series of quire signatures\mbox{}\newline 
 a-o plus roman figures in a cursive hand of the fourteenth century.{</\textbf{signatures}>}\end{shaded}\egroup\par \par
The \hyperref[TEI.signatures]{<signatures>} element can be used for either leaf signatures, or a combination of quire and leaf signatures, whether the marking is alphabetic, alphanumeric, or some ad hoc system, as in the following more complex example: \par\bgroup\index{signatures=<signatures>|exampleindex}\exampleFont \begin{shaded}\noindent\mbox{}{<\textbf{signatures}>}Quire and leaf\mbox{}\newline 
 signatures in letters, [b]-v, and roman numerals; those in quires 10 (1) and 17\mbox{}\newline 
 (s) in red ink and different from others; every third quire also signed with red\mbox{}\newline 
 crayon in arabic numerals in the centre lower margin of the first leaf recto:\mbox{}\newline 
 "2" for quire 4 (f. 19), "3" for quire 7 (f. 43); "4", barely visible, for quire\mbox{}\newline 
 10 (f. 65), "5", in a later hand, for quire 13 (f. 89), "6", in a later hand,\mbox{}\newline 
 for quire 16 (f. 113).{</\textbf{signatures}>}\end{shaded}\egroup\par \par
The \hyperref[TEI.secFol]{<secFol>} element (for ‘secundo folio’) is used to record an identifying phrase (also called \textit{dictio probatoria}) taken from a specific known point in a codex (for example the first few words on the second leaf). Since these words will differ from one copy of a text to another, the practice originated in the middle ages of using them when cataloguing a manuscript in order to distinguish individual copies of a work in a way which its opening words could not. \par\bgroup\index{secFol=<secFol>|exampleindex}\exampleFont \begin{shaded}\noindent\mbox{}{<\textbf{secFol}>}(ando-)ssene in una villa{</\textbf{secFol}>}\end{shaded}\egroup\par \noindent  
\subsubsection[{Heraldry}]{Heraldry}\label{mshera}\par
Descriptions of heraldic arms, supporters, devices, and mottos may appear at various points in the description of a manuscript, usually in the context of ownership information, binding descriptions, or detailed accounts of illustrations. A full description may also contain a detailed account of the heraldic components of a manuscript independently considered. Frequently, however, heraldic descriptions will be cited as short phrases within other parts of the record. The phrase level element \hyperref[TEI.heraldry]{<heraldry>} is provided to allow such phrases to be marked for further analysis, as in the following examples: \par\bgroup\index{p=<p>|exampleindex}\index{heraldry=<heraldry>|exampleindex}\index{quote=<quote>|exampleindex}\index{p=<p>|exampleindex}\index{heraldry=<heraldry>|exampleindex}\exampleFont \begin{shaded}\noindent\mbox{}{<\textbf{p}>}Ownership stamp (xvii cent.) on i recto with the arms {<\textbf{heraldry}>}A bull\mbox{}\newline 
\hspace*{1em}\hspace*{1em} passant within a bordure bezanty, in chief a crescent for difference{</\textbf{heraldry}>}\mbox{}\newline 
 [Cole], crest, and the legend {<\textbf{quote}>}Cole Deum{</\textbf{quote}>}.{</\textbf{p}>}\mbox{}\newline 
\textit{<!-- ... -->}\mbox{}\newline 
{<\textbf{p}>}A c. 8r fregio su due lati, {<\textbf{heraldry}>}stemma e imprese medicee{</\textbf{heraldry}>}\mbox{}\newline 
 racchiudono l'inizio dell'epistolario di Paolino.{</\textbf{p}>}\end{shaded}\egroup\par 
\subsection[{The Manuscript Identifier}]{The Manuscript Identifier}\label{msid}\par
The \hyperref[TEI.msIdentifier]{<msIdentifier>} element is intended to provide an unambiguous means of uniquely identifying a particular manuscript. This may be done in a structured way, by providing information about the holding institution and the call number, shelfmark, or other identifier used to indicate its location within that institution. Alternatively, or in addition, a manuscript may be identified simply by a commonly used name. 
\begin{sansreflist}
  
\item [\textbf{<msIdentifier>}] (manuscript identifier) contains the information required to identify the manuscript or similar object being described.
\end{sansreflist}
\par
A manuscript's actual physical location may occasionally be different from its place of ownership; at Cambridge University, for example, manuscripts owned by various colleges are kept in the central University Library. Normally, it is the ownership of the manuscript which should be specified in the manuscript identifier, while additional or more precise information on the physical location of the manuscript can be given within the \hyperref[TEI.adminInfo]{<adminInfo>} element, discussed in section \textit{\hyperref[msadad]{10.9.1.\ Administrative Information}} below.\par
The following elements are available within \hyperref[TEI.msIdentifier]{<msIdentifier>} to identify the holding institution: 
\begin{sansreflist}
  
\item [\textbf{<country>}] (country) contains the name of a geo-political unit, such as a nation, country, colony, or commonwealth, larger than or administratively superior to a region and smaller than a bloc.
\item [\textbf{<region>}] (region) contains the name of an administrative unit such as a state, province, or county, larger than a settlement, but smaller than a country.
\item [\textbf{<settlement>}] (settlement) contains the name of a settlement such as a city, town, or village identified as a single geo-political or administrative unit.
\item [\textbf{<institution>}] (institution) contains the name of an organization such as a university or library, with which a manuscript or other object is identified, generally its holding institution.
\item [\textbf{<repository>}] (repository) contains the name of a repository within which manuscripts or other objects are stored, possibly forming part of an institution.
\end{sansreflist}
\par
These elements are all structurally equivalent to the standard TEI \hyperref[TEI.name]{<name>} element with an appropriate value for its {\itshape type} attribute; however the use of this ‘syntactic sugar’ enables the model for \hyperref[TEI.msIdentifier]{<msIdentifier>} to be constrained rather more tightly than would otherwise be possible. Specifically, only one of each of the elements listed above may appear within the \hyperref[TEI.msIdentifier]{<msIdentifier>} and they must, if present, appear in the order given.\par
Like \hyperref[TEI.name]{<name>}, these elements are all also members of the attribute class \textsf{att.canonical}, and thus can use the attributes {\itshape key} or {\itshape ref} to reference a single standardized source of information about the entity named.\par
The following elements are used within \hyperref[TEI.msIdentifier]{<msIdentifier>} to provide different ways of identifying the manuscript within its holding institution: 
\begin{sansreflist}
  
\item [\textbf{<collection>}] (collection) contains the name of a collection of manuscripts or other objects, not necessarily located within a single repository.
\item [\textbf{<idno>}] (identifier) supplies any form of identifier used to identify some object, such as a bibliographic item, a person, a title, an organization, etc. in a standardized way.
\item [\textbf{<altIdentifier>}] (alternative identifier) contains an alternative or former structured identifier used for a manuscript or other object, such as a former catalogue number.
\item [\textbf{<msName>}] (alternative name) contains any form of unstructured alternative name used for a manuscript or other object, such as an ‘ocellus nominum’, or nickname.
\end{sansreflist}
\par
Major manuscript repositories will usually have a preferred form of citation for manuscript shelfmarks, including rules about punctuation, spacing, abbreviation, etc., which should be adhered to. Where such a format also contains information which might additionally be supplied as a distinct subcomponent of the \hyperref[TEI.msIdentifier]{<msIdentifier>}, for example a collection name, a decision must be taken as to whether to use the more specific element, or to include such information within the \hyperref[TEI.idno]{<idno>} element. For example, the manuscript formally identified as ‘El 26 C 9’ forms a part of the Ellesmere (‘El’) collection. Either of the following encodings is therefore feasible: \par\bgroup\index{msIdentifier=<msIdentifier>|exampleindex}\index{country=<country>|exampleindex}\index{region=<region>|exampleindex}\index{settlement=<settlement>|exampleindex}\index{repository=<repository>|exampleindex}\index{collection=<collection>|exampleindex}\index{idno=<idno>|exampleindex}\index{msName=<msName>|exampleindex}\exampleFont \begin{shaded}\noindent\mbox{}{<\textbf{msIdentifier}>}\mbox{}\newline 
\hspace*{1em}{<\textbf{country}>}USA{</\textbf{country}>}\mbox{}\newline 
\hspace*{1em}{<\textbf{region}>}California{</\textbf{region}>}\mbox{}\newline 
\hspace*{1em}{<\textbf{settlement}>}San Marino{</\textbf{settlement}>}\mbox{}\newline 
\hspace*{1em}{<\textbf{repository}>}Huntington Library{</\textbf{repository}>}\mbox{}\newline 
\hspace*{1em}{<\textbf{collection}>}El{</\textbf{collection}>}\mbox{}\newline 
\hspace*{1em}{<\textbf{idno}>}26 C 9{</\textbf{idno}>}\mbox{}\newline 
\hspace*{1em}{<\textbf{msName}>}The Ellesmere Chaucer{</\textbf{msName}>}\mbox{}\newline 
{</\textbf{msIdentifier}>}\end{shaded}\egroup\par \noindent  \par\bgroup\index{msIdentifier=<msIdentifier>|exampleindex}\index{country=<country>|exampleindex}\index{region=<region>|exampleindex}\index{settlement=<settlement>|exampleindex}\index{repository=<repository>|exampleindex}\index{idno=<idno>|exampleindex}\index{msName=<msName>|exampleindex}\exampleFont \begin{shaded}\noindent\mbox{}{<\textbf{msIdentifier}>}\mbox{}\newline 
\hspace*{1em}{<\textbf{country}>}USA{</\textbf{country}>}\mbox{}\newline 
\hspace*{1em}{<\textbf{region}>}California{</\textbf{region}>}\mbox{}\newline 
\hspace*{1em}{<\textbf{settlement}>}San Marino{</\textbf{settlement}>}\mbox{}\newline 
\hspace*{1em}{<\textbf{repository}>}Huntington Library{</\textbf{repository}>}\mbox{}\newline 
\hspace*{1em}{<\textbf{idno}>}El 26 C 9{</\textbf{idno}>}\mbox{}\newline 
\hspace*{1em}{<\textbf{msName}>}The Ellesmere Chaucer{</\textbf{msName}>}\mbox{}\newline 
{</\textbf{msIdentifier}>}\end{shaded}\egroup\par \par
In the former example, the preferred form of the identifier can be retrieved by prefixing the content of the \hyperref[TEI.idno]{<idno>} element with that of the \hyperref[TEI.collection]{<collection>} element, while in the latter it is given explicitly. The advantage of the former is that it simplifies accurate retrieval of all manuscripts from a given collection; the disadvantage is that encoded abbreviations of this kind may not be as immediately comprehensible. Care should be taken to avoid redundancy: for example \par\bgroup\index{collection=<collection>|exampleindex}\index{idno=<idno>|exampleindex}\exampleFont \begin{shaded}\noindent\mbox{}{<\textbf{collection}>}El{</\textbf{collection}>}\mbox{}\newline 
{<\textbf{idno}>}El 26 C 9{</\textbf{idno}>}\end{shaded}\egroup\par \noindent  would clearly be inappropriate. Equally clearly, \par\bgroup\index{collection=<collection>|exampleindex}\index{idno=<idno>|exampleindex}\exampleFont \begin{shaded}\noindent\mbox{}{<\textbf{collection}>}Ellesmere{</\textbf{collection}>}\mbox{}\newline 
{<\textbf{idno}>}El 26 C 9{</\textbf{idno}>}\end{shaded}\egroup\par \noindent  might be considered helpful in some circumstances (if, for example, some of the items in the Ellesmere collection had shelfmarks which did not begin ‘El’).\par
In some cases the shelfmark may contain no information about the collection; in other cases, the item may be regarded as belonging to more than one collection. The \hyperref[TEI.collection]{<collection>} element may be added, and repeated as often as necessary to cater for such situations: \par\bgroup\index{msIdentifier=<msIdentifier>|exampleindex}\index{country=<country>|exampleindex}\index{settlement=<settlement>|exampleindex}\index{repository=<repository>|exampleindex}\index{collection=<collection>|exampleindex}\index{collection=<collection>|exampleindex}\index{idno=<idno>|exampleindex}\exampleFont \begin{shaded}\noindent\mbox{}{<\textbf{msIdentifier}>}\mbox{}\newline 
\hspace*{1em}{<\textbf{country}>}Hungary{</\textbf{country}>}\mbox{}\newline 
\hspace*{1em}{<\textbf{settlement}>}Budapest{</\textbf{settlement}>}\mbox{}\newline 
\hspace*{1em}{<\textbf{repository}\hspace*{1em}{xml:lang}="{fr}">} Bibliothèque de l'Académie des Sciences de Hongrie {</\textbf{repository}>}\mbox{}\newline 
\hspace*{1em}{<\textbf{collection}>}Oriental Collection{</\textbf{collection}>}\mbox{}\newline 
\hspace*{1em}{<\textbf{collection}>}Sandor Kégl Bequest{</\textbf{collection}>}\mbox{}\newline 
\hspace*{1em}{<\textbf{idno}>}MS 1265{</\textbf{idno}>}\mbox{}\newline 
{</\textbf{msIdentifier}>}\end{shaded}\egroup\par \par
\par\bgroup\index{msIdentifier=<msIdentifier>|exampleindex}\index{country=<country>|exampleindex}\index{region=<region>|exampleindex}\index{settlement=<settlement>|exampleindex}\index{repository=<repository>|exampleindex}\index{collection=<collection>|exampleindex}\index{idno=<idno>|exampleindex}\index{msName=<msName>|exampleindex}\exampleFont \begin{shaded}\noindent\mbox{}{<\textbf{msIdentifier}>}\mbox{}\newline 
\hspace*{1em}{<\textbf{country}>}USA{</\textbf{country}>}\mbox{}\newline 
\hspace*{1em}{<\textbf{region}>}New Jersey{</\textbf{region}>}\mbox{}\newline 
\hspace*{1em}{<\textbf{settlement}>}Princeton{</\textbf{settlement}>}\mbox{}\newline 
\hspace*{1em}{<\textbf{repository}>}Princeton University Library{</\textbf{repository}>}\mbox{}\newline 
\hspace*{1em}{<\textbf{collection}>}Scheide Library{</\textbf{collection}>}\mbox{}\newline 
\hspace*{1em}{<\textbf{idno}>}MS 71{</\textbf{idno}>}\mbox{}\newline 
\hspace*{1em}{<\textbf{msName}>}Blickling Homiliary{</\textbf{msName}>}\mbox{}\newline 
{</\textbf{msIdentifier}>}\end{shaded}\egroup\par \par
Note in the latter case the use of the \hyperref[TEI.msName]{<msName>} element to provide a common name other than the shelfmark by which a manuscript is known. Where a manuscript has several such names, more than one of these elements may be used, as in the following example: \par\bgroup\index{msIdentifier=<msIdentifier>|exampleindex}\index{country=<country>|exampleindex}\index{settlement=<settlement>|exampleindex}\index{repository=<repository>|exampleindex}\index{idno=<idno>|exampleindex}\index{msName=<msName>|exampleindex}\index{msName=<msName>|exampleindex}\exampleFont \begin{shaded}\noindent\mbox{}{<\textbf{msIdentifier}>}\mbox{}\newline 
\hspace*{1em}{<\textbf{country}>}Danmark{</\textbf{country}>}\mbox{}\newline 
\hspace*{1em}{<\textbf{settlement}>}København{</\textbf{settlement}>}\mbox{}\newline 
\hspace*{1em}{<\textbf{repository}>}Det Arnamagnæanske Institut{</\textbf{repository}>}\mbox{}\newline 
\hspace*{1em}{<\textbf{idno}>}AM 45 fol.{</\textbf{idno}>}\mbox{}\newline 
\hspace*{1em}{<\textbf{msName}\hspace*{1em}{xml:lang}="{la}">}Codex Frisianus{</\textbf{msName}>}\mbox{}\newline 
\hspace*{1em}{<\textbf{msName}\hspace*{1em}{xml:lang}="{is}">}Fríssbók{</\textbf{msName}>}\mbox{}\newline 
{</\textbf{msIdentifier}>}\end{shaded}\egroup\par \noindent  Here the globally available {\itshape xml:lang} attribute has been used to specify the language of the alternative names. \par
In very rare cases a repository may have only one manuscript (or only one of any significance), which will have no shelfmark as such but will be known by a particular name or names. In such circumstances, the \hyperref[TEI.idno]{<idno>} element may be omitted, and the manuscript identified by the name or names used for it, using one or more \hyperref[TEI.msName]{<msName>} elements, as in the following example: \par\bgroup\index{msIdentifier=<msIdentifier>|exampleindex}\index{settlement=<settlement>|exampleindex}\index{repository=<repository>|exampleindex}\index{msName=<msName>|exampleindex}\index{msName=<msName>|exampleindex}\index{msName=<msName>|exampleindex}\exampleFont \begin{shaded}\noindent\mbox{}{<\textbf{msIdentifier}>}\mbox{}\newline 
\hspace*{1em}{<\textbf{settlement}>}Rossano{</\textbf{settlement}>}\mbox{}\newline 
\hspace*{1em}{<\textbf{repository}\hspace*{1em}{xml:lang}="{it}">}Biblioteca arcivescovile{</\textbf{repository}>}\mbox{}\newline 
\hspace*{1em}{<\textbf{msName}\hspace*{1em}{xml:lang}="{la}">}Codex Rossanensis{</\textbf{msName}>}\mbox{}\newline 
\hspace*{1em}{<\textbf{msName}\hspace*{1em}{xml:lang}="{la}">}Codex purpureus{</\textbf{msName}>}\mbox{}\newline 
\hspace*{1em}{<\textbf{msName}\hspace*{1em}{xml:lang}="{en}">}The Rossano Gospels{</\textbf{msName}>}\mbox{}\newline 
{</\textbf{msIdentifier}>}\end{shaded}\egroup\par \noindent  If a manuscript name contains a name or referencing string that it is useful to annotate (e.g. by referring to an authority list) then \hyperref[TEI.name]{<name>} or \hyperref[TEI.rs]{<rs>} may be used for this purpose.\par
Where manuscripts have moved from one institution to another, or even within the same institution, they may have identifiers additional to the ones currently used, such as former shelfmarks, which are sometimes retained even after they have been officially superseded. In such cases it may be useful to supply an alternative identifier, with a detailed structure similar to that of the \hyperref[TEI.msIdentifier]{<msIdentifier>} itself. The following example shows a manuscript which had shelfmark \texttt{II-M-5} in the collection of the Duque de Osuna, but which now has the shelfmark \texttt{MS 10237} in the National Library in Madrid: \par\bgroup\index{msIdentifier=<msIdentifier>|exampleindex}\index{settlement=<settlement>|exampleindex}\index{repository=<repository>|exampleindex}\index{idno=<idno>|exampleindex}\index{altIdentifier=<altIdentifier>|exampleindex}\index{region=<region>|exampleindex}\index{settlement=<settlement>|exampleindex}\index{repository=<repository>|exampleindex}\index{idno=<idno>|exampleindex}\exampleFont \begin{shaded}\noindent\mbox{}{<\textbf{msIdentifier}>}\mbox{}\newline 
\hspace*{1em}{<\textbf{settlement}>}Madrid{</\textbf{settlement}>}\mbox{}\newline 
\hspace*{1em}{<\textbf{repository}>}Biblioteca Nacional{</\textbf{repository}>}\mbox{}\newline 
\hspace*{1em}{<\textbf{idno}>}MS 10237{</\textbf{idno}>}\mbox{}\newline 
\hspace*{1em}{<\textbf{altIdentifier}>}\mbox{}\newline 
\hspace*{1em}\hspace*{1em}{<\textbf{region}>}Andalucia{</\textbf{region}>}\mbox{}\newline 
\hspace*{1em}\hspace*{1em}{<\textbf{settlement}>}Osuna{</\textbf{settlement}>}\mbox{}\newline 
\hspace*{1em}\hspace*{1em}{<\textbf{repository}>}Duque de Osuna{</\textbf{repository}>}\mbox{}\newline 
\hspace*{1em}\hspace*{1em}{<\textbf{idno}>}II-M-5{</\textbf{idno}>}\mbox{}\newline 
\hspace*{1em}{</\textbf{altIdentifier}>}\mbox{}\newline 
{</\textbf{msIdentifier}>}\end{shaded}\egroup\par \noindent  Normally, such information would be dealt with under \hyperref[TEI.history]{<history>}, except in cases where a manuscript is likely still to be referred to or known by its former identifier. For example, an institution may have changed its call number system but still wish to retain a record of the earlier number, perhaps because the manuscript concerned is frequently cited in print under its previous number: \par\bgroup\index{msIdentifier=<msIdentifier>|exampleindex}\index{settlement=<settlement>|exampleindex}\index{institution=<institution>|exampleindex}\index{repository=<repository>|exampleindex}\index{idno=<idno>|exampleindex}\index{altIdentifier=<altIdentifier>|exampleindex}\index{idno=<idno>|exampleindex}\exampleFont \begin{shaded}\noindent\mbox{}{<\textbf{msIdentifier}>}\mbox{}\newline 
\hspace*{1em}{<\textbf{settlement}>}Berkeley{</\textbf{settlement}>}\mbox{}\newline 
\hspace*{1em}{<\textbf{institution}>}University of California{</\textbf{institution}>}\mbox{}\newline 
\hspace*{1em}{<\textbf{repository}>}Bancroft Library{</\textbf{repository}>}\mbox{}\newline 
\hspace*{1em}{<\textbf{idno}>}UCB 16{</\textbf{idno}>}\mbox{}\newline 
\hspace*{1em}{<\textbf{altIdentifier}>}\mbox{}\newline 
\hspace*{1em}\hspace*{1em}{<\textbf{idno}>}2MS BS1145 I8{</\textbf{idno}>}\mbox{}\newline 
\hspace*{1em}{</\textbf{altIdentifier}>}\mbox{}\newline 
{</\textbf{msIdentifier}>}\end{shaded}\egroup\par \noindent  Where (as in this example) no repository is specified for the \hyperref[TEI.altIdentifier]{<altIdentifier>}, it is assumed to be the same as that of the parent \hyperref[TEI.msIdentifier]{<msIdentifier>}. Where the holding institution has only one preferred form of citation but wishes to retain the other for internal administrative purposes, the secondary could be given within \hyperref[TEI.altIdentifier]{<altIdentifier>} with an appropriate value on the {\itshape type} attribute: \par\bgroup\index{msIdentifier=<msIdentifier>|exampleindex}\index{settlement=<settlement>|exampleindex}\index{repository=<repository>|exampleindex}\index{idno=<idno>|exampleindex}\index{altIdentifier=<altIdentifier>|exampleindex}\index{type=@type!<altIdentifier>|exampleindex}\index{idno=<idno>|exampleindex}\exampleFont \begin{shaded}\noindent\mbox{}{<\textbf{msIdentifier}>}\mbox{}\newline 
\hspace*{1em}{<\textbf{settlement}>}Oxford{</\textbf{settlement}>}\mbox{}\newline 
\hspace*{1em}{<\textbf{repository}>}Bodleian Library{</\textbf{repository}>}\mbox{}\newline 
\hspace*{1em}{<\textbf{idno}>}MS. Bodley 406{</\textbf{idno}>}\mbox{}\newline 
\hspace*{1em}{<\textbf{altIdentifier}\hspace*{1em}{type}="{SC}">}\mbox{}\newline 
\hspace*{1em}\hspace*{1em}{<\textbf{idno}>}2297{</\textbf{idno}>}\mbox{}\newline 
\hspace*{1em}{</\textbf{altIdentifier}>}\mbox{}\newline 
{</\textbf{msIdentifier}>}\end{shaded}\egroup\par \noindent  It might, however, be preferable to include such information within the \hyperref[TEI.adminInfo]{<adminInfo>} element discussed in section \textit{\hyperref[msadad]{10.9.1.\ Administrative Information}} below.\par
Cases of such changed or alternative identifiers should be clearly distinguished from cases of ‘fragmented’ (\textit{\hyperref[msfg]{10.11.\ Manuscript Fragments}}) manuscripts, that is to say manuscripts which although physically disjoint are nevertheless generally treated as single units.\par
As mentioned above, the smallest possible description is one that contains only the element \hyperref[TEI.msIdentifier]{<msIdentifier>}; good practice in all but exceptional circumstances requires the presence within it of the three sub-elements \hyperref[TEI.settlement]{<settlement>}, \hyperref[TEI.repository]{<repository>}, and \hyperref[TEI.idno]{<idno>}, since they provide what is, by common consent, the minimum amount of information necessary to identify a manuscript.
\subsection[{The Manuscript Heading}]{The Manuscript Heading}\label{msdo}\par
Historically, the briefest possible meaningful description of a manuscript consists of no more than a title, e.g. \textit{Polychronicon}. This will often have been enough to identify a manuscript in a small collection because the identity of the author is implicit. Where a title does not imply the author, and is thus insufficient to identify the main text of a manuscript, the author should be stated explicitly (e.g. \textit{Augustinus, Sermones} or \textit{Cicero, Letters}). Many inventories of manuscripts consist of no more than an author and title, with some form of copy-specific identifier, such as a shelfmark or ‘secundo folio’ reference (e.g. \textit{Arch. B. 3. 2: Evangelium Matthei cum glossa}, \textit{126. Isidori Originum libri octo}, \textit{Biblia Hieronimi, 2o fo. opus est}); information on date and place of writing will sometimes also be included. The standard TEI element \hyperref[TEI.head]{<head>} element can be used to provide a brief description of this kind. 
\begin{sansreflist}
  
\item [\textbf{<head>}] (heading) contains any type of heading, for example the title of a section, or the heading of a list, glossary, manuscript description, etc.
\end{sansreflist}
 In this way the cataloguer or scholar can supply in one place a minimum of essential information, such as might be displayed or printed as the heading of a full description. For example: \par\bgroup\index{head=<head>|exampleindex}\exampleFont \begin{shaded}\noindent\mbox{}{<\textbf{head}>}Marsilius de Inghen, Abbreviata phisicorum Aristotelis; Italy,\mbox{}\newline 
 1463.{</\textbf{head}>}\end{shaded}\egroup\par \noindent  Any phrase-level elements, such as \hyperref[TEI.title]{<title>}, \hyperref[TEI.name]{<name>}, \hyperref[TEI.date]{<date>}, or the specialized elements \hyperref[TEI.origPlace]{<origPlace>} and \hyperref[TEI.origDate]{<origDate>}, can also be used within a \hyperref[TEI.head]{<head>} element, but it should be remembered that the \hyperref[TEI.head]{<head>} element is intended principally to contain a heading. More structured information concerning the contents, physical form, or history of the manuscript should be given within the specialized elements described below, \hyperref[TEI.msContents]{<msContents>}, \hyperref[TEI.physDesc]{<physDesc>}, \hyperref[TEI.history]{<history>}, etc. However, in simple cases, the \hyperref[TEI.p]{<p>} element may also be used to supply an unstructured collection of such information, as in the example given above (\textit{\hyperref[msdesc]{10.2.\ The Manuscript Description Element}}).
\subsection[{Intellectual Content}]{Intellectual Content}\label{msco}\par
The \hyperref[TEI.msContents]{<msContents>} element is used to describe the intellectual content of a manuscript or manuscript part. It comprises \textit{either} a series of informal prose paragraphs \textit{or} a series of \hyperref[TEI.msItem]{<msItem>} or \hyperref[TEI.msItemStruct]{<msItemStruct>} elements, each of which provides a more detailed description of a single item contained within the manuscript. These may be prefaced, if desired, by a \hyperref[TEI.summary]{<summary>} element, which is especially useful where one wishes to provide an overview of a manuscript's contents and describe only some of the items in detail. 
\begin{sansreflist}
  
\item [\textbf{<msContents>}] (manuscript contents) describes the intellectual content of a manuscript, manuscript part, or other object either as a series of paragraphs or as a series of structured manuscript items.
\item [\textbf{<msItem>}] (manuscript item) describes an individual work or item within the intellectual content of a manuscript, manuscript part, or other object.
\item [\textbf{<msItemStruct>}] (structured manuscript item) contains a structured description for an individual work or item within the intellectual content of a manuscript, manuscript part, or other object.
\item [\textbf{<summary>}] contains an overview of the available information concerning some aspect of an item or object (for example, its intellectual content, history, layout, typography etc.) as a complement or alternative to the more detailed information carried by more specific elements.
\end{sansreflist}
\par
In the simplest case, only a brief description may be provided, as in the following examples: \par\bgroup\index{msContents=<msContents>|exampleindex}\index{p=<p>|exampleindex}\index{msContents=<msContents>|exampleindex}\index{p=<p>|exampleindex}\index{msContents=<msContents>|exampleindex}\index{p=<p>|exampleindex}\exampleFont \begin{shaded}\noindent\mbox{}{<\textbf{msContents}>}\mbox{}\newline 
\hspace*{1em}{<\textbf{p}>}A collection of Lollard sermons{</\textbf{p}>}\mbox{}\newline 
{</\textbf{msContents}>}\mbox{}\newline 
{<\textbf{msContents}>}\mbox{}\newline 
\hspace*{1em}{<\textbf{p}>}Atlas of the world from Western Europe and Africa to Indochina, containing 27\mbox{}\newline 
\hspace*{1em}\hspace*{1em} maps and 26 tables{</\textbf{p}>}\mbox{}\newline 
{</\textbf{msContents}>}\mbox{}\newline 
{<\textbf{msContents}>}\mbox{}\newline 
\hspace*{1em}{<\textbf{p}>}Biblia sacra: Antiguo y Nuevo Testamento, con prefacios, prólogos y\mbox{}\newline 
\hspace*{1em}\hspace*{1em} argumentos de san Jerónimo y de otros. Interpretaciones de los nombres\mbox{}\newline 
\hspace*{1em}\hspace*{1em} hebreos.{</\textbf{p}>}\mbox{}\newline 
{</\textbf{msContents}>}\end{shaded}\egroup\par \par
This description may of course be expanded to include any of the TEI elements generally available within a \hyperref[TEI.p]{<p>} element, such as \hyperref[TEI.title]{<title>}, \hyperref[TEI.bibl]{<bibl>}, or \hyperref[TEI.list]{<list>}. More usually, however, each individual work within a manuscript will be given its own description, using the \hyperref[TEI.msItem]{<msItem>} or \hyperref[TEI.msItemStruct]{<msItemStruct>} element described in the next section, as in the following example: \par\bgroup\index{msContents=<msContents>|exampleindex}\index{msItem=<msItem>|exampleindex}\index{n=@n!<msItem>|exampleindex}\index{locus=<locus>|exampleindex}\index{title=<title>|exampleindex}\index{bibl=<bibl>|exampleindex}\index{title=<title>|exampleindex}\index{biblScope=<biblScope>|exampleindex}\index{msItem=<msItem>|exampleindex}\index{n=@n!<msItem>|exampleindex}\index{locus=<locus>|exampleindex}\index{title=<title>|exampleindex}\index{bibl=<bibl>|exampleindex}\index{title=<title>|exampleindex}\index{biblScope=<biblScope>|exampleindex}\index{msItem=<msItem>|exampleindex}\index{n=@n!<msItem>|exampleindex}\index{locus=<locus>|exampleindex}\index{title=<title>|exampleindex}\index{bibl=<bibl>|exampleindex}\index{title=<title>|exampleindex}\index{biblScope=<biblScope>|exampleindex}\index{msItem=<msItem>|exampleindex}\index{n=@n!<msItem>|exampleindex}\index{locus=<locus>|exampleindex}\index{title=<title>|exampleindex}\index{bibl=<bibl>|exampleindex}\index{title=<title>|exampleindex}\index{biblScope=<biblScope>|exampleindex}\index{msItem=<msItem>|exampleindex}\index{n=@n!<msItem>|exampleindex}\index{locus=<locus>|exampleindex}\index{title=<title>|exampleindex}\index{title=<title>|exampleindex}\index{bibl=<bibl>|exampleindex}\index{title=<title>|exampleindex}\index{biblScope=<biblScope>|exampleindex}\index{msItem=<msItem>|exampleindex}\index{n=@n!<msItem>|exampleindex}\index{locus=<locus>|exampleindex}\index{title=<title>|exampleindex}\index{note=<note>|exampleindex}\exampleFont \begin{shaded}\noindent\mbox{}{<\textbf{msContents}>}\mbox{}\newline 
\hspace*{1em}{<\textbf{msItem}\hspace*{1em}{n}="{1}">}\mbox{}\newline 
\hspace*{1em}\hspace*{1em}{<\textbf{locus}>}fols. 5r -7v{</\textbf{locus}>}\mbox{}\newline 
\hspace*{1em}\hspace*{1em}{<\textbf{title}>}An ABC{</\textbf{title}>}\mbox{}\newline 
\hspace*{1em}\hspace*{1em}{<\textbf{bibl}>}\mbox{}\newline 
\hspace*{1em}\hspace*{1em}\hspace*{1em}{<\textbf{title}>}IMEV{</\textbf{title}>}\mbox{}\newline 
\hspace*{1em}\hspace*{1em}\hspace*{1em}{<\textbf{biblScope}>}239{</\textbf{biblScope}>}\mbox{}\newline 
\hspace*{1em}\hspace*{1em}{</\textbf{bibl}>}\mbox{}\newline 
\hspace*{1em}{</\textbf{msItem}>}\mbox{}\newline 
\hspace*{1em}{<\textbf{msItem}\hspace*{1em}{n}="{2}">}\mbox{}\newline 
\hspace*{1em}\hspace*{1em}{<\textbf{locus}>}fols. 7v -8v{</\textbf{locus}>}\mbox{}\newline 
\hspace*{1em}\hspace*{1em}{<\textbf{title}\hspace*{1em}{xml:lang}="{fr}">}Lenvoy de Chaucer a Scogan{</\textbf{title}>}\mbox{}\newline 
\hspace*{1em}\hspace*{1em}{<\textbf{bibl}>}\mbox{}\newline 
\hspace*{1em}\hspace*{1em}\hspace*{1em}{<\textbf{title}>}IMEV{</\textbf{title}>}\mbox{}\newline 
\hspace*{1em}\hspace*{1em}\hspace*{1em}{<\textbf{biblScope}>}3747{</\textbf{biblScope}>}\mbox{}\newline 
\hspace*{1em}\hspace*{1em}{</\textbf{bibl}>}\mbox{}\newline 
\hspace*{1em}{</\textbf{msItem}>}\mbox{}\newline 
\hspace*{1em}{<\textbf{msItem}\hspace*{1em}{n}="{3}">}\mbox{}\newline 
\hspace*{1em}\hspace*{1em}{<\textbf{locus}>}fol. 8v{</\textbf{locus}>}\mbox{}\newline 
\hspace*{1em}\hspace*{1em}{<\textbf{title}>}Truth{</\textbf{title}>}\mbox{}\newline 
\hspace*{1em}\hspace*{1em}{<\textbf{bibl}>}\mbox{}\newline 
\hspace*{1em}\hspace*{1em}\hspace*{1em}{<\textbf{title}>}IMEV{</\textbf{title}>}\mbox{}\newline 
\hspace*{1em}\hspace*{1em}\hspace*{1em}{<\textbf{biblScope}>}809{</\textbf{biblScope}>}\mbox{}\newline 
\hspace*{1em}\hspace*{1em}{</\textbf{bibl}>}\mbox{}\newline 
\hspace*{1em}{</\textbf{msItem}>}\mbox{}\newline 
\hspace*{1em}{<\textbf{msItem}\hspace*{1em}{n}="{4}">}\mbox{}\newline 
\hspace*{1em}\hspace*{1em}{<\textbf{locus}>}fols. 8v-10v{</\textbf{locus}>}\mbox{}\newline 
\hspace*{1em}\hspace*{1em}{<\textbf{title}>}Birds Praise of Love{</\textbf{title}>}\mbox{}\newline 
\hspace*{1em}\hspace*{1em}{<\textbf{bibl}>}\mbox{}\newline 
\hspace*{1em}\hspace*{1em}\hspace*{1em}{<\textbf{title}>}IMEV{</\textbf{title}>}\mbox{}\newline 
\hspace*{1em}\hspace*{1em}\hspace*{1em}{<\textbf{biblScope}>}1506{</\textbf{biblScope}>}\mbox{}\newline 
\hspace*{1em}\hspace*{1em}{</\textbf{bibl}>}\mbox{}\newline 
\hspace*{1em}{</\textbf{msItem}>}\mbox{}\newline 
\hspace*{1em}{<\textbf{msItem}\hspace*{1em}{n}="{5}">}\mbox{}\newline 
\hspace*{1em}\hspace*{1em}{<\textbf{locus}>}fols. 10v -11v{</\textbf{locus}>}\mbox{}\newline 
\hspace*{1em}\hspace*{1em}{<\textbf{title}\hspace*{1em}{xml:lang}="{la}">}De amico ad amicam{</\textbf{title}>}\mbox{}\newline 
\hspace*{1em}\hspace*{1em}{<\textbf{title}\hspace*{1em}{xml:lang}="{la}">}Responcio{</\textbf{title}>}\mbox{}\newline 
\hspace*{1em}\hspace*{1em}{<\textbf{bibl}>}\mbox{}\newline 
\hspace*{1em}\hspace*{1em}\hspace*{1em}{<\textbf{title}>}IMEV{</\textbf{title}>}\mbox{}\newline 
\hspace*{1em}\hspace*{1em}\hspace*{1em}{<\textbf{biblScope}>}16 \& 19{</\textbf{biblScope}>}\mbox{}\newline 
\hspace*{1em}\hspace*{1em}{</\textbf{bibl}>}\mbox{}\newline 
\hspace*{1em}{</\textbf{msItem}>}\mbox{}\newline 
\hspace*{1em}{<\textbf{msItem}\hspace*{1em}{n}="{6}">}\mbox{}\newline 
\hspace*{1em}\hspace*{1em}{<\textbf{locus}>}fols. 14r-126v{</\textbf{locus}>}\mbox{}\newline 
\hspace*{1em}\hspace*{1em}{<\textbf{title}>}Troilus and Criseyde{</\textbf{title}>}\mbox{}\newline 
\hspace*{1em}\hspace*{1em}{<\textbf{note}>}Bk. 1:71-Bk. 5:1701, with additional losses due to mutilation\mbox{}\newline 
\hspace*{1em}\hspace*{1em}\hspace*{1em}\hspace*{1em} throughout{</\textbf{note}>}\mbox{}\newline 
\hspace*{1em}{</\textbf{msItem}>}\mbox{}\newline 
{</\textbf{msContents}>}\end{shaded}\egroup\par \par
The \hyperref[TEI.summary]{<summary>} element may be used in conjunction with one or more \hyperref[TEI.msItem]{<msItem>} elements if it is desired to provide both a general summary of the contents of a manuscript and more detail about some or all of the individual items within it. It may not however be used within an individual \hyperref[TEI.msItem]{<msItem>} element.\par\bgroup\index{msContents=<msContents>|exampleindex}\index{summary=<summary>|exampleindex}\index{msItem=<msItem>|exampleindex}\index{n=@n!<msItem>|exampleindex}\index{locus=<locus>|exampleindex}\index{title=<title>|exampleindex}\index{msItem=<msItem>|exampleindex}\index{n=@n!<msItem>|exampleindex}\index{locus=<locus>|exampleindex}\index{title=<title>|exampleindex}\exampleFont \begin{shaded}\noindent\mbox{}{<\textbf{msContents}>}\mbox{}\newline 
\hspace*{1em}{<\textbf{summary}>}A collection of Lollard sermons{</\textbf{summary}>}\mbox{}\newline 
\hspace*{1em}{<\textbf{msItem}\hspace*{1em}{n}="{1}">}\mbox{}\newline 
\hspace*{1em}\hspace*{1em}{<\textbf{locus}>}fol. 4r-8r{</\textbf{locus}>}\mbox{}\newline 
\hspace*{1em}\hspace*{1em}{<\textbf{title}>}3rd Sunday Before Lent{</\textbf{title}>}\mbox{}\newline 
\hspace*{1em}{</\textbf{msItem}>}\mbox{}\newline 
\hspace*{1em}{<\textbf{msItem}\hspace*{1em}{n}="{2}">}\mbox{}\newline 
\hspace*{1em}\hspace*{1em}{<\textbf{locus}>}fol. 9r-16v{</\textbf{locus}>}\mbox{}\newline 
\hspace*{1em}\hspace*{1em}{<\textbf{title}>}Sexagesima{</\textbf{title}>}\mbox{}\newline 
\hspace*{1em}{</\textbf{msItem}>}\mbox{}\newline 
{</\textbf{msContents}>}\end{shaded}\egroup\par 
\subsubsection[{The msItem and msItemStruct Elements}]{The \texttt{<msItem>} and \texttt{<msItemStruct>} Elements}\label{mscoit}\par
Each discrete item in a manuscript or manuscript part can be described within a distinct \hyperref[TEI.msItem]{<msItem>} or \hyperref[TEI.msItemStruct]{<msItemStruct>} element, and may be classified using the {\itshape class} attribute.\par
These are the possible component elements of \hyperref[TEI.msItem]{<msItem>} and \hyperref[TEI.msItemStruct]{<msItemStruct>}. 
\begin{sansreflist}
  
\item [\textbf{<author>}] (author) in a bibliographic reference, contains the name(s) of an author, personal or corporate, of a work; for example in the same form as that provided by a recognized bibliographic name authority.
\item [\textbf{<respStmt>}] (statement of responsibility) supplies a statement of responsibility for the intellectual content of a text, edition, recording, or series, where the specialized elements for authors, editors, etc. do not suffice or do not apply. May also be used to encode information about individuals or organizations which have played a role in the production or distribution of a bibliographic work.
\item [\textbf{<title>}] (title) contains a title for any kind of work.\hfil\\[-10pt]\begin{sansreflist}
    \item[@{\itshape type}]
  classifies the title according to some convenient typology.
\end{sansreflist}  
\item [\textbf{<rubric>}] (rubric) contains the text of any \textit{rubric} or heading attached to a particular manuscript item, that is, a string of words through which a manuscript or other object signals the beginning of a text division, often with an assertion as to its author and title, which is in some way set off from the text itself, typically in red ink, or by use of different size or type of script, or some other such visual device.
\item [\textbf{<incipit>}] contains the \textit{incipit} of a manuscript or similar object item, that is the opening words of the text proper, exclusive of any \textit{rubric} which might precede it, of sufficient length to identify the work uniquely; such incipits were, in former times, frequently used a means of reference to a work, in place of a title.
\item [\textbf{<quote>}] (quotation) contains a phrase or passage attributed by the narrator or author to some agency external to the text.
\item [\textbf{<explicit>}] (explicit) contains the \textit{explicit} of a item, that is, the closing words of the text proper, exclusive of any rubric or colophon which might follow it.
\item [\textbf{<finalRubric>}] (final rubric) contains the string of words that denotes the end of a text division, often with an assertion as to its author and title, usually set off from the text itself by red ink, by a different size or type of script, or by some other such visual device.
\item [\textbf{<colophon>}] (colophon) contains the \textit{colophon} of an item: that is, a statement providing information regarding the date, place, agency, or reason for production of the manuscript or other object.
\item [\textbf{<decoNote>}] (note on decoration) contains a note describing either a decorative component of a manuscript or other object, or a fairly homogenous class of such components.
\item [\textbf{<listBibl>}] (citation list) contains a list of bibliographic citations of any kind.
\item [\textbf{<bibl>}] (bibliographic citation) contains a loosely-structured bibliographic citation of which the sub-components may or may not be explicitly tagged.
\item [\textbf{<filiation>}] (filiation) contains information concerning the manuscript or other object's \textit{filiation}, i.e. its relationship to other surviving manuscripts or other objects of the same text or contents, its \textit{protographs}, \textit{antigraphs} and \textit{apographs}.
\item [\textbf{<note>}] (note) contains a note or annotation.
\item [\textbf{<textLang>}] (text language) describes the languages and writing systems identified within the bibliographic work being described, rather than its description.
\end{sansreflist}
\par
In addition, an \hyperref[TEI.msItemStruct]{<msItemStruct>} may contain nested \hyperref[TEI.msItemStruct]{<msItemStruct>} elements, just as an \hyperref[TEI.msItem]{<msItem>} may contain nested \hyperref[TEI.msItem]{<msItem>} elements.\par
The main difference between \hyperref[TEI.msItem]{<msItem>} and \hyperref[TEI.msItemStruct]{<msItemStruct>} is that in the former, the order and number of child elements is not constrained; any element, in other words, may be given in any order, and repeated as often as is judged necessary. In the latter, however, the sub-elements, if used, must be given in the order specified above and only some of them may be repeated; specifically, \hyperref[TEI.rubric]{<rubric>}, \hyperref[TEI.finalRubric]{<finalRubric>}. \hyperref[TEI.incipit]{<incipit>}, \hyperref[TEI.textLang]{<textLang>} and \hyperref[TEI.explicit]{<explicit>} can appear only once.\par
While neither \hyperref[TEI.msItem]{<msItem>} nor \hyperref[TEI.msItemStruct]{<msItemStruct>} may contain untagged running text, both permit an unstructured description to be provided in the form of one or more paragraphs of text. They differ in this respect also: if paragraphs are supplied as the content of an \hyperref[TEI.msItem]{<msItem>}, then none of the other component elements listed above is permitted; in the \hyperref[TEI.msItemStruct]{<msItemStruct>} case, however, paragraphs may appear anywhere as an alternative to any of the component elements listed above.\par
As noted above, both \hyperref[TEI.msItem]{<msItem>} and \hyperref[TEI.msItemStruct]{<msItemStruct>} elements may also nest, where a number of separate items in a manuscript are grouped under a single title or rubric, as is the case, for example, with a work like \textit{The Canterbury Tales}.\par
The elements \hyperref[TEI.msContents]{<msContents>}, \hyperref[TEI.msItem]{<msItem>}, \hyperref[TEI.msItemStruct]{<msItemStruct>}, \hyperref[TEI.incipit]{<incipit>}, and \hyperref[TEI.explicit]{<explicit>} are all members of the class \textsf{att.msExcerpt} from which they inherit the {\itshape defective} attribute. 
\begin{sansreflist}
  
\item [\textbf{att.msExcerpt}] (manuscript excerpt) provides attributes used to describe excerpts from a manuscript placed in a description thereof.\hfil\\[-10pt]\begin{sansreflist}
    \item[@{\itshape defective}]
  indicates whether the passage being quoted is defective, i.e. incomplete through loss or damage.
\end{sansreflist}  
\end{sansreflist}
 This attribute can be used for example with collections of fragments, where each fragment is given as a separate \hyperref[TEI.msItem]{<msItem>} and the first and last words of each fragment are transcribed as defective incipits and explicits, as in the following example, a manuscript containing four fragments of a single work: \par\bgroup\index{msContents=<msContents>|exampleindex}\index{msItem=<msItem>|exampleindex}\index{defective=@defective!<msItem>|exampleindex}\index{locus=<locus>|exampleindex}\index{from=@from!<locus>|exampleindex}\index{to=@to!<locus>|exampleindex}\index{title=<title>|exampleindex}\index{msItem=<msItem>|exampleindex}\index{n=@n!<msItem>|exampleindex}\index{locus=<locus>|exampleindex}\index{from=@from!<locus>|exampleindex}\index{to=@to!<locus>|exampleindex}\index{incipit=<incipit>|exampleindex}\index{defective=@defective!<incipit>|exampleindex}\index{ex=<ex>|exampleindex}\index{ex=<ex>|exampleindex}\index{explicit=<explicit>|exampleindex}\index{defective=@defective!<explicit>|exampleindex}\index{expan=<expan>|exampleindex}\index{ex=<ex>|exampleindex}\index{am=<am>|exampleindex}\index{g=<g>|exampleindex}\index{ref=@ref!<g>|exampleindex}\index{ex=<ex>|exampleindex}\exampleFont \begin{shaded}\noindent\mbox{}{<\textbf{msContents}>}\mbox{}\newline 
\hspace*{1em}{<\textbf{msItem}\hspace*{1em}{defective}="{true}">}\mbox{}\newline 
\hspace*{1em}\hspace*{1em}{<\textbf{locus}\hspace*{1em}{from}="{1r}"\hspace*{1em}{to}="{9v}">}1r-9v{</\textbf{locus}>}\mbox{}\newline 
\hspace*{1em}\hspace*{1em}{<\textbf{title}>}Knýtlinga saga{</\textbf{title}>}\mbox{}\newline 
\hspace*{1em}\hspace*{1em}{<\textbf{msItem}\hspace*{1em}{n}="{1.1}">}\mbox{}\newline 
\hspace*{1em}\hspace*{1em}\hspace*{1em}{<\textbf{locus}\hspace*{1em}{from}="{1r:1}"\hspace*{1em}{to}="{2v:30}">}1r:1-2v:30{</\textbf{locus}>}\mbox{}\newline 
\hspace*{1em}\hspace*{1em}\hspace*{1em}{<\textbf{incipit}\hspace*{1em}{defective}="{true}">}dan{<\textbf{ex}>}n{</\textbf{ex}>}a a engl{<\textbf{ex}>}an{</\textbf{ex}>}di{</\textbf{incipit}>}\mbox{}\newline 
\hspace*{1em}\hspace*{1em}\hspace*{1em}{<\textbf{explicit}\hspace*{1em}{defective}="{true}">}en meðan {<\textbf{expan}>}haraldr{</\textbf{expan}>} hein hafði\mbox{}\newline 
\hspace*{1em}\hspace*{1em}\hspace*{1em}\hspace*{1em}\hspace*{1em}\hspace*{1em} k{<\textbf{ex}>}onung{</\textbf{ex}>}r v{<\textbf{am}>}\mbox{}\newline 
\hspace*{1em}\hspace*{1em}\hspace*{1em}\hspace*{1em}\hspace*{1em}{<\textbf{g}\hspace*{1em}{ref}="{http://www.example.com/abbrevs.xml\#er}"/>}\mbox{}\newline 
\hspace*{1em}\hspace*{1em}\hspace*{1em}\hspace*{1em}{</\textbf{am}>}it\mbox{}\newline 
\hspace*{1em}\hspace*{1em}\hspace*{1em}\hspace*{1em}\hspace*{1em}\hspace*{1em} yf{<\textbf{ex}>}ir{</\textbf{ex}>} danmork{</\textbf{explicit}>}\mbox{}\newline 
\hspace*{1em}\hspace*{1em}{</\textbf{msItem}>}\mbox{}\newline 
\textit{<!-- msItems 1.2 to 1.4 -->}\mbox{}\newline 
\hspace*{1em}{</\textbf{msItem}>}\mbox{}\newline 
{</\textbf{msContents}>}\end{shaded}\egroup\par \par
The elements \hyperref[TEI.ex]{<ex>}, \hyperref[TEI.am]{<am>}, and \hyperref[TEI.expan]{<expan>} used in the above example are further discussed in section \textit{\hyperref[PHAB]{11.3.1.2.\ Abbreviation and Expansion}}; they are available only when the \textsf{transcr} module defined by that chapter is selected. Similarly, the \hyperref[TEI.g]{<g>} element used in this example to represent the abbreviation mark is defined by the \textsf{gaiji} module documented in chapter \textit{\hyperref[WD]{5.\ Characters, Glyphs, and Writing Modes}}.
\subsubsection[{Authors and Titles}]{Authors and Titles}\label{msat}\par
When used within a manuscript description, the \hyperref[TEI.title]{<title>} element should be used to supply a regularized form of the item's title, as distinct from any rubric quoted from the manuscript. If the item concerned has a standardized distinctive title, e.g. \textit{Roman de la Rose}, then this should be the form given as content of the \hyperref[TEI.title]{<title>} element, with the value of the {\itshape type} attribute given as \texttt{uniform}. If no uniform title exists for an item, or none has been yet identified, or if one wishes to provide a general designation of the contents, then a ‘supplied’ title can be given, e.g. \textit{missal}, in which case the {\itshape type} attribute on the \hyperref[TEI.title]{<title>} should be given the value \texttt{supplied}.\par
Similarly, if used within a manuscript description, the \hyperref[TEI.author]{<author>} element should always contain the normalized form of an author's name, irrespective of how (or whether) this form of the name is cited in the manuscript. If it is desired to retain the form of the author's name as given in the manuscript, this may be tagged as a distinct \hyperref[TEI.name]{<name>} element, within the text at the point where it occurs.\par
Note that the {\itshape key} attribute can also be used, as on names in general, to specify the identifier of a \hyperref[TEI.person]{<person>} element carrying full details of the person concerned (see further \textit{\hyperref[msnames]{10.3.6.\ Names of Persons, Places, and Organizations}}).\par
The \hyperref[TEI.respStmt]{<respStmt>} element can be used to supply the name and role of a person other than the author who is responsible for some aspect of the intellectual content of the manuscript: \par\bgroup\index{author=<author>|exampleindex}\index{respStmt=<respStmt>|exampleindex}\index{resp=<resp>|exampleindex}\index{name=<name>|exampleindex}\exampleFont \begin{shaded}\noindent\mbox{}{<\textbf{author}>}Diogenes Laertius{</\textbf{author}>}\mbox{}\newline 
{<\textbf{respStmt}>}\mbox{}\newline 
\hspace*{1em}{<\textbf{resp}>}in the translation of{</\textbf{resp}>}\mbox{}\newline 
\hspace*{1em}{<\textbf{name}>}Ambrogio Traversari{</\textbf{name}>}\mbox{}\newline 
{</\textbf{respStmt}>}\end{shaded}\egroup\par \par
The \hyperref[TEI.respStmt]{<respStmt>} element can also be used where there is a discrepancy between the author of an item as given in the manuscript and the accepted scholarly view, as in the following example: \par\bgroup\index{title=<title>|exampleindex}\index{type=@type!<title>|exampleindex}\index{respStmt=<respStmt>|exampleindex}\index{resp=<resp>|exampleindex}\index{name=<name>|exampleindex}\exampleFont \begin{shaded}\noindent\mbox{}{<\textbf{title}\hspace*{1em}{type}="{supplied}">}Sermons on the Epistles and the Gospels{</\textbf{title}>}\mbox{}\newline 
{<\textbf{respStmt}>}\mbox{}\newline 
\hspace*{1em}{<\textbf{resp}>}here erroneously attributed to{</\textbf{resp}>}\mbox{}\newline 
\hspace*{1em}{<\textbf{name}>}St. Bonaventura{</\textbf{name}>}\mbox{}\newline 
{</\textbf{respStmt}>}\end{shaded}\egroup\par \noindent  Note that such attributions of authorship, both correct and incorrect, are frequently found in the rubric or final rubric (and occasionally also elsewhere in the text), and can therefore be transcribed and included in the description, if desired, using the \hyperref[TEI.rubric]{<rubric>}, \hyperref[TEI.finalRubric]{<finalRubric>}, or \hyperref[TEI.quote]{<quote>} elements, as appropriate.
\subsubsection[{Rubrics, Incipits, Explicits, and Other Quotations from the Text}]{Rubrics, Incipits, Explicits, and Other Quotations from the Text}\label{mscorie}\par
It is customary in a manuscript description to record the opening and closing words of a text as well as any headings or colophons it might have, and the specialized elements \hyperref[TEI.rubric]{<rubric>}, \hyperref[TEI.incipit]{<incipit>}, \hyperref[TEI.explicit]{<explicit>}, \hyperref[TEI.finalRubric]{<finalRubric>}, and \hyperref[TEI.colophon]{<colophon>} are available within \hyperref[TEI.msItem]{<msItem>} for doing so, along with the more general \hyperref[TEI.quote]{<quote>}, for recording other bits of the text not covered by these elements. Each of these elements has the same substructure, containing a mixture of phrase-level elements and plain text. A \hyperref[TEI.locus]{<locus>} element can be included within each, in order to specify the location of the component, as in the following example: \par\bgroup\index{msContents=<msContents>|exampleindex}\index{msItem=<msItem>|exampleindex}\index{locus=<locus>|exampleindex}\index{author=<author>|exampleindex}\index{title=<title>|exampleindex}\index{incipit=<incipit>|exampleindex}\index{locus=<locus>|exampleindex}\index{explicit=<explicit>|exampleindex}\index{locus=<locus>|exampleindex}\exampleFont \begin{shaded}\noindent\mbox{}{<\textbf{msContents}>}\mbox{}\newline 
\hspace*{1em}{<\textbf{msItem}>}\mbox{}\newline 
\hspace*{1em}\hspace*{1em}{<\textbf{locus}>}f. 1-223{</\textbf{locus}>}\mbox{}\newline 
\hspace*{1em}\hspace*{1em}{<\textbf{author}>}Radulphus Flaviacensis{</\textbf{author}>}\mbox{}\newline 
\hspace*{1em}\hspace*{1em}{<\textbf{title}>}Expositio super Leviticum {</\textbf{title}>}\mbox{}\newline 
\hspace*{1em}\hspace*{1em}{<\textbf{incipit}>}\mbox{}\newline 
\hspace*{1em}\hspace*{1em}\hspace*{1em}{<\textbf{locus}>}f. 1r{</\textbf{locus}>} Forte Hervei monachi{</\textbf{incipit}>}\mbox{}\newline 
\hspace*{1em}\hspace*{1em}{<\textbf{explicit}>}\mbox{}\newline 
\hspace*{1em}\hspace*{1em}\hspace*{1em}{<\textbf{locus}>}f. 223v{</\textbf{locus}>} Benedictio salis et aquae{</\textbf{explicit}>}\mbox{}\newline 
\hspace*{1em}{</\textbf{msItem}>}\mbox{}\newline 
{</\textbf{msContents}>}\end{shaded}\egroup\par \par
In the following example, standard TEI elements for the transcription of primary sources have been used to mark the expansion of abbreviations and other features present in the original: \par\bgroup\index{msItem=<msItem>|exampleindex}\index{defective=@defective!<msItem>|exampleindex}\index{locus=<locus>|exampleindex}\index{title=<title>|exampleindex}\index{type=@type!<title>|exampleindex}\index{incipit=<incipit>|exampleindex}\index{defective=@defective!<incipit>|exampleindex}\index{lb=<lb>|exampleindex}\index{ex=<ex>|exampleindex}\index{gap=<gap>|exampleindex}\index{reason=@reason!<gap>|exampleindex}\index{quantity=@quantity!<gap>|exampleindex}\index{unit=@unit!<gap>|exampleindex}\index{lb=<lb>|exampleindex}\index{ex=<ex>|exampleindex}\index{explicit=<explicit>|exampleindex}\index{defective=@defective!<explicit>|exampleindex}\index{ex=<ex>|exampleindex}\index{ex=<ex>|exampleindex}\index{ex=<ex>|exampleindex}\index{lb=<lb>|exampleindex}\index{ex=<ex>|exampleindex}\index{ex=<ex>|exampleindex}\index{ex=<ex>|exampleindex}\exampleFont \begin{shaded}\noindent\mbox{}{<\textbf{msItem}\hspace*{1em}{defective}="{true}">}\mbox{}\newline 
\hspace*{1em}{<\textbf{locus}>}ff. 1r-24v{</\textbf{locus}>}\mbox{}\newline 
\hspace*{1em}{<\textbf{title}\hspace*{1em}{type}="{uniform}">}Ágrip af Noregs konunga sǫgum{</\textbf{title}>}\mbox{}\newline 
\hspace*{1em}{<\textbf{incipit}\hspace*{1em}{defective}="{true}">}\mbox{}\newline 
\hspace*{1em}\hspace*{1em}{<\textbf{lb}/>}regi oc h{<\textbf{ex}>}ann{</\textbf{ex}>} seti ho{<\textbf{gap}\hspace*{1em}{reason}="{illegible}"\hspace*{1em}{quantity}="{7}"\mbox{}\newline 
\hspace*{1em}\hspace*{1em}\hspace*{1em}{unit}="{mm}"/>}\mbox{}\newline 
\hspace*{1em}\hspace*{1em}{<\textbf{lb}/>}sc heim se{<\textbf{ex}>}m{</\textbf{ex}>} þio{</\textbf{incipit}>}\mbox{}\newline 
\hspace*{1em}{<\textbf{explicit}\hspace*{1em}{defective}="{true}">}h{<\textbf{ex}>}on{</\textbf{ex}>} hev{<\textbf{ex}>}er{</\textbf{ex}>}\mbox{}\newline 
\hspace*{1em}\hspace*{1em}{<\textbf{ex}>}oc{</\textbf{ex}>} þa buit hesta .ij.\mbox{}\newline 
\hspace*{1em}{<\textbf{lb}/>}annan viþ fé en h{<\textbf{ex}>}on{</\textbf{ex}>}o{<\textbf{ex}>}m{</\textbf{ex}>} annan til\mbox{}\newline 
\hspace*{1em}\hspace*{1em} reiþ{<\textbf{ex}>}ar{</\textbf{ex}>}\mbox{}\newline 
\hspace*{1em}{</\textbf{explicit}>}\mbox{}\newline 
{</\textbf{msItem}>}\end{shaded}\egroup\par \noindent  Note here also the use of the {\itshape defective} attribute on \hyperref[TEI.incipit]{<incipit>} and \hyperref[TEI.explicit]{<explicit>} to indicate that the text begins and ends defectively.\par
The {\itshape xml:lang} attribute for \hyperref[TEI.colophon]{<colophon>}, \hyperref[TEI.explicit]{<explicit>}, \hyperref[TEI.incipit]{<incipit>}, \hyperref[TEI.quote]{<quote>}, and \hyperref[TEI.rubric]{<rubric>} may always be used to identify the language of the text quoted, if this is different from the default language specified by the {\itshape mainLang} attribute on \hyperref[TEI.textLang]{<textLang>}.
\subsubsection[{Filiation}]{Filiation}\label{msfil}\par
The \hyperref[TEI.filiation]{<filiation>} element can be used to provide information on the relationship between the manuscript and other surviving manuscripts of the same text, either specifically or in a general way, as in the following example: \par\bgroup\index{msItem=<msItem>|exampleindex}\index{locus=<locus>|exampleindex}\index{incipit=<incipit>|exampleindex}\index{ref=<ref>|exampleindex}\index{explicit=<explicit>|exampleindex}\index{bibl=<bibl>|exampleindex}\index{filiation=<filiation>|exampleindex}\exampleFont \begin{shaded}\noindent\mbox{}{<\textbf{msItem}>}\mbox{}\newline 
\hspace*{1em}{<\textbf{locus}>}118rb{</\textbf{locus}>}\mbox{}\newline 
\hspace*{1em}{<\textbf{incipit}>}Ecce morior cum nichil horum ... {<\textbf{ref}>}[Dn 13, 43]{</\textbf{ref}>}. Verba ista\mbox{}\newline 
\hspace*{1em}\hspace*{1em} dixit Susanna de illis{</\textbf{incipit}>}\mbox{}\newline 
\hspace*{1em}{<\textbf{explicit}>}ut bonum comune conservatur.{</\textbf{explicit}>}\mbox{}\newline 
\hspace*{1em}{<\textbf{bibl}>}Schneyer 3, 436 (Johannes Contractus OFM){</\textbf{bibl}>}\mbox{}\newline 
\hspace*{1em}{<\textbf{filiation}>}weitere Überl. Uppsala C 181, 35r.{</\textbf{filiation}>}\mbox{}\newline 
{</\textbf{msItem}>}\end{shaded}\egroup\par 
\subsubsection[{Text Classification}]{Text Classification}\label{msclass}\par
One or more text classification or text-type codes may be specified, either for the whole of the \hyperref[TEI.msContents]{<msContents>} element, or for one or more of its constituent \hyperref[TEI.msItem]{<msItem>} elements, using the {\itshape class} attribute as specified above: \par\bgroup\index{msContents=<msContents>|exampleindex}\index{msItem=<msItem>|exampleindex}\index{n=@n!<msItem>|exampleindex}\index{defective=@defective!<msItem>|exampleindex}\index{class=@class!<msItem>|exampleindex}\index{locus=<locus>|exampleindex}\index{from=@from!<locus>|exampleindex}\index{to=@to!<locus>|exampleindex}\index{title=<title>|exampleindex}\index{type=@type!<title>|exampleindex}\index{incipit=<incipit>|exampleindex}\index{ex=<ex>|exampleindex}\index{ex=<ex>|exampleindex}\index{explicit=<explicit>|exampleindex}\index{ex=<ex>|exampleindex}\index{ex=<ex>|exampleindex}\index{ex=<ex>|exampleindex}\exampleFont \begin{shaded}\noindent\mbox{}{<\textbf{msContents}>}\mbox{}\newline 
\hspace*{1em}{<\textbf{msItem}\hspace*{1em}{n}="{1}"\hspace*{1em}{defective}="{false}"\mbox{}\newline 
\hspace*{1em}\hspace*{1em}{class}="{\#law}">}\mbox{}\newline 
\hspace*{1em}\hspace*{1em}{<\textbf{locus}\hspace*{1em}{from}="{1v}"\hspace*{1em}{to}="{71v}">}1v-71v{</\textbf{locus}>}\mbox{}\newline 
\hspace*{1em}\hspace*{1em}{<\textbf{title}\hspace*{1em}{type}="{uniform}">}Jónsbók{</\textbf{title}>}\mbox{}\newline 
\hspace*{1em}\hspace*{1em}{<\textbf{incipit}>}Magnus m{<\textbf{ex}>}ed{</\textbf{ex}>} guds miskun Noregs k{<\textbf{ex}>}onungu{</\textbf{ex}>}r{</\textbf{incipit}>}\mbox{}\newline 
\hspace*{1em}\hspace*{1em}{<\textbf{explicit}>}en{<\textbf{ex}>}n{</\textbf{ex}>} u{<\textbf{ex}>}ir{</\textbf{ex}>}da þo t{<\textbf{ex}>}il{</\textbf{ex}>} fullra aura{</\textbf{explicit}>}\mbox{}\newline 
\hspace*{1em}{</\textbf{msItem}>}\mbox{}\newline 
{</\textbf{msContents}>}\end{shaded}\egroup\par \noindent  The value used for the {\itshape class} attribute in this example points to a \hyperref[TEI.category]{<category>} element with the identifier \texttt{law}, which defines the classification concerned. Such \hyperref[TEI.category]{<category>} elements will typically appear within a \hyperref[TEI.taxonomy]{<taxonomy>} element, within the \hyperref[TEI.classDecl]{<classDecl>} element of the TEI header (\textit{\hyperref[HD55]{2.3.7.\ The Classification Declaration}}) as in the following example: \par\bgroup\index{classDecl=<classDecl>|exampleindex}\index{taxonomy=<taxonomy>|exampleindex}\index{category=<category>|exampleindex}\index{catDesc=<catDesc>|exampleindex}\index{category=<category>|exampleindex}\index{catDesc=<catDesc>|exampleindex}\exampleFont \begin{shaded}\noindent\mbox{}{<\textbf{classDecl}>}\mbox{}\newline 
\hspace*{1em}{<\textbf{taxonomy}>}\mbox{}\newline 
\textit{<!-- -->}\mbox{}\newline 
\hspace*{1em}\hspace*{1em}{<\textbf{category}\hspace*{1em}{xml:id}="{law}">}\mbox{}\newline 
\hspace*{1em}\hspace*{1em}\hspace*{1em}{<\textbf{catDesc}>}Legislation{</\textbf{catDesc}>}\mbox{}\newline 
\hspace*{1em}\hspace*{1em}{</\textbf{category}>}\mbox{}\newline 
\hspace*{1em}\hspace*{1em}{<\textbf{category}\hspace*{1em}{xml:id}="{war}">}\mbox{}\newline 
\hspace*{1em}\hspace*{1em}\hspace*{1em}{<\textbf{catDesc}>}Military topics{</\textbf{catDesc}>}\mbox{}\newline 
\hspace*{1em}\hspace*{1em}{</\textbf{category}>}\mbox{}\newline 
\textit{<!-- -->}\mbox{}\newline 
\hspace*{1em}{</\textbf{taxonomy}>}\mbox{}\newline 
{</\textbf{classDecl}>}\end{shaded}\egroup\par \noindent  More than one classification may apply to a single item. Another text, concerned with legislation about military topics might thus be specified as follows: \par\bgroup\index{msItem=<msItem>|exampleindex}\index{class=@class!<msItem>|exampleindex}\index{p=<p>|exampleindex}\exampleFont \begin{shaded}\noindent\mbox{}{<\textbf{msItem}\hspace*{1em}{class}="{\#law \#war}">}\mbox{}\newline 
\hspace*{1em}{<\textbf{p}>}A treatise on Clausewitz{</\textbf{p}>}\mbox{}\newline 
\textit{<!-- details of the item here -->}\mbox{}\newline 
{</\textbf{msItem}>}\end{shaded}\egroup\par 
\subsubsection[{Languages and Writing Systems}]{Languages and Writing Systems}\label{mslangs}\par
The \hyperref[TEI.textLang]{<textLang>} element should be used to provide information about the languages used within a manuscript item. It may take the form of a simple note, as in the following example: \par\bgroup\index{textLang=<textLang>|exampleindex}\exampleFont \begin{shaded}\noindent\mbox{}{<\textbf{textLang}>}Old Church Slavonic, written in Cyrillic script.{</\textbf{textLang}>}\end{shaded}\egroup\par \par
Where, for validation and indexing purposes, it is thought convenient to add keywords identifying the particular languages used, the {\itshape mainLang} attribute may be used. This attribute takes the same range of values as the global {\itshape xml:lang} attribute, on which see further \textit{\hyperref[CHSH]{vi.1\ Language Identification}}. In the following example a manuscript written chiefly in Old Church Slavonic is described: \par\bgroup\index{textLang=<textLang>|exampleindex}\index{mainLang=@mainLang!<textLang>|exampleindex}\exampleFont \begin{shaded}\noindent\mbox{}{<\textbf{textLang}\hspace*{1em}{mainLang}="{chu}">}Old Church Slavonic{</\textbf{textLang}>}\end{shaded}\egroup\par \par
A manuscript item will sometimes contain material in more than one language. The {\itshape mainLang} attribute should be used only for the chief language. Other languages used may be specified using the {\itshape otherLangs} attribute as in the following example: \par\bgroup\index{textLang=<textLang>|exampleindex}\index{mainLang=@mainLang!<textLang>|exampleindex}\index{otherLangs=@otherLangs!<textLang>|exampleindex}\exampleFont \begin{shaded}\noindent\mbox{}{<\textbf{textLang}\hspace*{1em}{mainLang}="{chu}"\mbox{}\newline 
\hspace*{1em}{otherLangs}="{RUS HEL}">}Mostly Old Church Slavonic, with\mbox{}\newline 
 some Russian and Greek material{</\textbf{textLang}>}\end{shaded}\egroup\par \par
Since Old Church Slavonic may be written in either Cyrillic or Glagolitic scripts, and even occasionally in both within the same manuscript, it might be preferable to use a more explicit identifier: \par\bgroup\index{textLang=<textLang>|exampleindex}\index{mainLang=@mainLang!<textLang>|exampleindex}\exampleFont \begin{shaded}\noindent\mbox{}{<\textbf{textLang}\hspace*{1em}{mainLang}="{chu-Cyrs}">}Old Church Slavonic in Cyrillic script{</\textbf{textLang}>}\end{shaded}\egroup\par \par
The form and scope of language identifiers recommended by these Guidelines is based on the IANA standard described at \textit{\hyperref[CHSH]{vi.1\ Language Identification}} and should be followed throughout. Where additional detail is needed correctly to describe a language, or to discuss its deployment in a given text, this should be done using the \hyperref[TEI.langUsage]{<langUsage>} element in the TEI header, within which individual \hyperref[TEI.language]{<language>} elements document the languages used: see \textit{\hyperref[HD41]{2.4.2.\ Language Usage}}.\par
Note that the \hyperref[TEI.language]{<language>} element defines a particular combination of human language and writing system. Only one \hyperref[TEI.language]{<language>} element may be supplied for each such combination. Standard TEI practice also allows this element to be referenced by any element using the global {\itshape xml:lang} attribute in order to specify the language applicable to the content of that element. For example, assuming that \hyperref[TEI.language]{<language>} elements have been defined with the identifiers \textsf{fr} (for French), \textsf{la} (for Latin), and \textsf{de} (for German), a manuscript description written in French which specifies that a particular manuscript contains predominantly German but also some Latin material, might have a \hyperref[TEI.textLang]{<textLang>} element like the following: \par\bgroup\index{textLang=<textLang>|exampleindex}\index{mainLang=@mainLang!<textLang>|exampleindex}\index{otherLangs=@otherLangs!<textLang>|exampleindex}\exampleFont \begin{shaded}\noindent\mbox{}{<\textbf{textLang}\hspace*{1em}{xml:lang}="{fr}"\hspace*{1em}{mainLang}="{de}"\mbox{}\newline 
\hspace*{1em}{otherLangs}="{la}">}allemand et\mbox{}\newline 
 latin{</\textbf{textLang}>}\end{shaded}\egroup\par 
\subsection[{Physical Description}]{Physical Description}\label{msph}\par
Under the general heading ‘physical description’ we subsume a large number of different aspects generally regarded as useful in the description of a given manuscript. These include: \begin{itemize}
\item aspects of the form, support, extent, and quire structure of the manuscript object and of the way in which the text is laid out on the page (\textit{\hyperref[msph1]{10.7.1.\ Object Description}});
\item the styles of writing, such as the way it is laid out on the page, the styles of writing, decorative features, any musical notation employed and any annotations or marginalia (\textit{\hyperref[msph2]{10.7.2.\ Writing, Decoration, and Other Notations}});
\item and discussion of its binding, seals, and any accompanying material (\textit{\hyperref[msph3]{10.7.3.\ Bindings, Seals, and Additional Material}}).
\end{itemize} \par
Most manuscript descriptions touch on several of these categories of information though few include them all, and not all distinguish them as clearly as we propose here. In particular, it is often the case that an existing description will include information for which we propose distinct elements within a single paragraph, or even sentence. The encoder must then decide whether to rewrite the description using the structure proposed here, or to retain the existing prose, marked up simply as a series of \hyperref[TEI.p]{<p>} elements, directly within the \hyperref[TEI.physDesc]{<physDesc>} element.\par
The \hyperref[TEI.physDesc]{<physDesc>} element may thus be used in either of two distinct ways. It may contain a series of paragraphs addressing topics listed above and similar ones. Alternatively, it may act as a container for any choice of the more specialized elements described in the remainder of this section, each of which itself contains a series of paragraphs, and may also have more specific attributes. \par
In general, it is not recommended to combine unstructured prose description with usage of the more specialized elements, as such an approach complicates processing, and may lead to inconsistency within a single manuscript description. A single \hyperref[TEI.physDesc]{<physDesc>} element will normally contain either a series of \textsf{model.pLike} elements, or a sequence of specialized elements from the \textsf{model.physDescPart} class. There are however circumstances in which this is not feasible, for example: \begin{itemize}
\item the description already exists in a prose form where some of the specialized topics are treated together in paragraphs of prose, but others are treated distinctly;
\item although all parts of the description are clearly distinguished, some of them cannot be mapped to a pre-existing specialized element.
\end{itemize} \par
In such situations, both specialized and generic (\textsf{model.pLike}) elements may be combined in a single \hyperref[TEI.physDesc]{<physDesc>}. Note however that all generic elements given must precede the first specialized element in the description. Thus the following is valid: \par\bgroup\index{physDesc=<physDesc>|exampleindex}\index{p=<p>|exampleindex}\index{objectDesc=<objectDesc>|exampleindex}\index{form=@form!<objectDesc>|exampleindex}\exampleFont \begin{shaded}\noindent\mbox{}{<\textbf{physDesc}>}\mbox{}\newline 
\hspace*{1em}{<\textbf{p}>}Generic descriptive prose...{</\textbf{p}>}\mbox{}\newline 
\textit{<!-- other generic elements here -->}\mbox{}\newline 
\hspace*{1em}{<\textbf{objectDesc}\hspace*{1em}{form}="{codex}">}\mbox{}\newline 
\textit{<!-- ... -->}\mbox{}\newline 
\hspace*{1em}{</\textbf{objectDesc}>}\mbox{}\newline 
\textit{<!-- other specific elements here -->}\mbox{}\newline 
{</\textbf{physDesc}>}\end{shaded}\egroup\par \noindent  but neither of the following is valid: \par\bgroup\exampleFont \begin{shaded}\noindent\mbox{}\newline
<physDesc>\newline
<objectDesc form="codex"> \newline
<!-- ... -->\newline
</objectDesc>\newline
<p>Generic descriptive prose...</p>\newline
</physDesc>\end{shaded}\egroup\par \noindent  \par\bgroup\exampleFont \begin{shaded}\noindent\mbox{}\newline
<physDesc>\newline
<objectDesc form="codex"> \newline
  <!-- ... -->\newline
</objectDesc>\newline
<p>Generic descriptive prose...</p>\newline
<!-- other specific elements here -->\newline
</physDesc>\end{shaded}\egroup\par \noindent  The order in which specific elements may appear is also constrained by the content model; again this is for simplicity of processing. They may of course be processed or displayed in any desired order, but for ease of validation, they must be given in the order specified below.
\subsubsection[{Object Description}]{Object Description}\label{msph1}\par
The \hyperref[TEI.objectDesc]{<objectDesc>} element is used to group together those parts of the physical description which relate specifically to the text-bearing object, its format, constitution, layout, etc. The \hyperref[TEI.objectDesc]{<objectDesc>} element is used for grouping elements relating to the physicality of a text-bearing object as part of a manuscript description. If a full description of an object (text-bearing or not) is desired, the more general \hyperref[TEI.object]{<object>} element may be preferred.\par
The {\itshape form} attribute is used to indicate the specific type of writing vehicle being described, for example, as a codex, roll, tablet, etc. If used it must appear first in the sequence of specialized elements. The \hyperref[TEI.objectDesc]{<objectDesc>} element has two parts: a description of the \textit{support}, i.e. the physical carrier on which the text is inscribed; and a description of the \textit{layout}, i.e. the way text is organized on the carrier.\par
Taking these in turn, the description of the support is tagged using the following elements, each of which is discussed in more detail below: 
\begin{sansreflist}
  
\item [\textbf{<supportDesc>}] (support description) groups elements describing the physical support for the written part of a manuscript or other object.
\item [\textbf{<support>}] (support) contains a description of the materials etc. which make up the physical support for the written part of a manuscript or other object.
\item [\textbf{<extent>}] (extent) describes the approximate size of a text stored on some carrier medium or of some other object, digital or non-digital, specified in any convenient units.
\item [\textbf{<collation>}] (collation) contains a description of how the leaves, bifolia, or similar objects are physically arranged.
\item [\textbf{<foliation>}] (foliation) describes the numbering system or systems used to count the leaves or pages in a codex or similar object.
\item [\textbf{<condition>}] (condition) contains a description of the physical condition of the manuscript or object.
\end{sansreflist}
\par
Each of these elements contains paragraphs relating to the topic concerned. Within these paragraphs, phrase-level elements (in particular those discussed above at \textit{\hyperref[msphrase]{10.3.\ Phrase-level Elements}}), may be used to tag specific terms of interest if so desired. \par\bgroup\index{objectDesc=<objectDesc>|exampleindex}\index{form=@form!<objectDesc>|exampleindex}\index{supportDesc=<supportDesc>|exampleindex}\index{p=<p>|exampleindex}\index{material=<material>|exampleindex}\index{watermark=<watermark>|exampleindex}\index{ref=<ref>|exampleindex}\index{watermark=<watermark>|exampleindex}\index{ref=<ref>|exampleindex}\exampleFont \begin{shaded}\noindent\mbox{}{<\textbf{objectDesc}\hspace*{1em}{form}="{codex}">}\mbox{}\newline 
\hspace*{1em}{<\textbf{supportDesc}>}\mbox{}\newline 
\hspace*{1em}\hspace*{1em}{<\textbf{p}>}Mostly {<\textbf{material}>}paper{</\textbf{material}>}, with watermarks\mbox{}\newline 
\hspace*{1em}\hspace*{1em}{<\textbf{watermark}>}unicorn{</\textbf{watermark}>} ({<\textbf{ref}>}Briquet 9993{</\textbf{ref}>}) and\mbox{}\newline 
\hspace*{1em}\hspace*{1em}{<\textbf{watermark}>}ox{</\textbf{watermark}>} (close to {<\textbf{ref}>}Briquet 2785{</\textbf{ref}>}). The first and last\mbox{}\newline 
\hspace*{1em}\hspace*{1em}\hspace*{1em}\hspace*{1em} leaf of each quire, with the exception of quires xvi and xviii, are constituted\mbox{}\newline 
\hspace*{1em}\hspace*{1em}\hspace*{1em}\hspace*{1em} by bifolia of parchment, and all seven miniatures have been painted on inserted\mbox{}\newline 
\hspace*{1em}\hspace*{1em}\hspace*{1em}\hspace*{1em} singletons of parchment.{</\textbf{p}>}\mbox{}\newline 
\hspace*{1em}{</\textbf{supportDesc}>}\mbox{}\newline 
{</\textbf{objectDesc}>}\end{shaded}\egroup\par \par
This example combines information which might alternatively be more precisely tagged using the more specific elements described in the following subsections.
\paragraph[{Support}]{Support}\label{msph1sup}\par
The \hyperref[TEI.support]{<support>} element groups together information about the physical carrier. Typically, for western manuscripts, this will entail discussion of the material (parchment, paper, or a combination of the two) written on. For paper, a discussion of any watermarks present may also be useful. If this discussion makes reference to standard catalogues of such items, these may be tagged using the standard \hyperref[TEI.ref]{<ref>} element as in the following example: \par\bgroup\index{support=<support>|exampleindex}\index{p=<p>|exampleindex}\index{material=<material>|exampleindex}\index{watermark=<watermark>|exampleindex}\index{watermark=<watermark>|exampleindex}\index{ref=<ref>|exampleindex}\index{date=<date>|exampleindex}\exampleFont \begin{shaded}\noindent\mbox{}{<\textbf{support}>}\mbox{}\newline 
\hspace*{1em}{<\textbf{p}>}\mbox{}\newline 
\hspace*{1em}\hspace*{1em}{<\textbf{material}>}Paper{</\textbf{material}>} with watermark: {<\textbf{watermark}>}anchor in a circle with\mbox{}\newline 
\hspace*{1em}\hspace*{1em}\hspace*{1em}\hspace*{1em} star on top{</\textbf{watermark}>}, {<\textbf{watermark}>}countermark B-B with trefoil{</\textbf{watermark}>}\mbox{}\newline 
\hspace*{1em}\hspace*{1em} similar to {<\textbf{ref}>}Moschin, Anchor N 1680{</\textbf{ref}>}\mbox{}\newline 
\hspace*{1em}\hspace*{1em}{<\textbf{date}>}1570-1585{</\textbf{date}>}.{</\textbf{p}>}\mbox{}\newline 
{</\textbf{support}>}\end{shaded}\egroup\par 
\paragraph[{Extent}]{Extent}\label{msph1ext}\par
The \hyperref[TEI.extent]{<extent>} element, defined in the TEI header, may also be used in a manuscript description to specify the number of leaves a manuscript contains, as in the following example: \par\bgroup\index{extent=<extent>|exampleindex}\exampleFont \begin{shaded}\noindent\mbox{}{<\textbf{extent}>}ii + 97 + ii{</\textbf{extent}>}\end{shaded}\egroup\par \noindent  Information regarding the size of the leaves may be specifically marked using the phrase level \hyperref[TEI.dimensions]{<dimensions>} element, as in the following example, or left as plain prose. \par\bgroup\index{extent=<extent>|exampleindex}\index{dimensions=<dimensions>|exampleindex}\index{unit=@unit!<dimensions>|exampleindex}\index{height=<height>|exampleindex}\index{width=<width>|exampleindex}\exampleFont \begin{shaded}\noindent\mbox{}{<\textbf{extent}>}ii + 321 leaves {<\textbf{dimensions}\hspace*{1em}{unit}="{cm}">}\mbox{}\newline 
\hspace*{1em}\hspace*{1em}{<\textbf{height}>}35{</\textbf{height}>}\mbox{}\newline 
\hspace*{1em}\hspace*{1em}{<\textbf{width}>}27{</\textbf{width}>}\mbox{}\newline 
\hspace*{1em}{</\textbf{dimensions}>}\mbox{}\newline 
{</\textbf{extent}>}\end{shaded}\egroup\par \par
Alternatively, the generic \hyperref[TEI.measure]{<measure>} element might be used within \hyperref[TEI.extent]{<extent>}, as in the following example: \par\bgroup\index{extent=<extent>|exampleindex}\index{measure=<measure>|exampleindex}\index{type=@type!<measure>|exampleindex}\index{unit=@unit!<measure>|exampleindex}\index{quantity=@quantity!<measure>|exampleindex}\index{measure=<measure>|exampleindex}\index{type=@type!<measure>|exampleindex}\index{quantity=@quantity!<measure>|exampleindex}\index{unit=@unit!<measure>|exampleindex}\index{measure=<measure>|exampleindex}\index{type=@type!<measure>|exampleindex}\index{quantity=@quantity!<measure>|exampleindex}\index{unit=@unit!<measure>|exampleindex}\exampleFont \begin{shaded}\noindent\mbox{}{<\textbf{extent}>}\mbox{}\newline 
\hspace*{1em}{<\textbf{measure}\hspace*{1em}{type}="{composition}"\hspace*{1em}{unit}="{leaf}"\mbox{}\newline 
\hspace*{1em}\hspace*{1em}{quantity}="{10}">}10 Bl.{</\textbf{measure}>}\mbox{}\newline 
\hspace*{1em}{<\textbf{measure}\hspace*{1em}{type}="{height}"\hspace*{1em}{quantity}="{37}"\mbox{}\newline 
\hspace*{1em}\hspace*{1em}{unit}="{cm}">}37{</\textbf{measure}>} x {<\textbf{measure}\hspace*{1em}{type}="{width}"\hspace*{1em}{quantity}="{29}"\mbox{}\newline 
\hspace*{1em}\hspace*{1em}{unit}="{cm}">}29{</\textbf{measure}>} cm \mbox{}\newline 
{</\textbf{extent}>}\end{shaded}\egroup\par 
\paragraph[{Collation}]{Collation}\label{msph1col}\par
The \hyperref[TEI.collation]{<collation>} element should be used to provide a description of a book's current and original structure, that is, the arrangement of its leaves and quires. This information may be conveyed using informal prose, or any appropriate notational convention. Although no specific notation is defined here, an appropriate element to enclose such an expression would be the \hyperref[TEI.formula]{<formula>} element, which is provided when the \textsf{figures} module is included in a schema. Here are some examples of different ways of treating collation: \par\bgroup\index{collation=<collation>|exampleindex}\index{p=<p>|exampleindex}\index{formula=<formula>|exampleindex}\index{collation=<collation>|exampleindex}\index{p=<p>|exampleindex}\index{list=<list>|exampleindex}\index{item=<item>|exampleindex}\index{locus=<locus>|exampleindex}\index{locus=<locus>|exampleindex}\index{locus=<locus>|exampleindex}\index{locus=<locus>|exampleindex}\index{locus=<locus>|exampleindex}\index{locus=<locus>|exampleindex}\index{locus=<locus>|exampleindex}\index{item=<item>|exampleindex}\index{item=<item>|exampleindex}\index{locus=<locus>|exampleindex}\index{locus=<locus>|exampleindex}\index{item=<item>|exampleindex}\index{collation=<collation>|exampleindex}\index{p=<p>|exampleindex}\index{collation=<collation>|exampleindex}\index{p=<p>|exampleindex}\index{formula=<formula>|exampleindex}\exampleFont \begin{shaded}\noindent\mbox{}{<\textbf{collation}>}\mbox{}\newline 
\hspace*{1em}{<\textbf{p}>}\mbox{}\newline 
\hspace*{1em}\hspace*{1em}{<\textbf{formula}>}1-3:8, 4:6, 5-13:8{</\textbf{formula}>}\mbox{}\newline 
\hspace*{1em}{</\textbf{p}>}\mbox{}\newline 
{</\textbf{collation}>}\mbox{}\newline 
{<\textbf{collation}>}\mbox{}\newline 
\hspace*{1em}{<\textbf{p}>}There are now four gatherings, the first, second and fourth originally\mbox{}\newline 
\hspace*{1em}\hspace*{1em} consisting of eight leaves, the third of seven. A fifth gathering thought to\mbox{}\newline 
\hspace*{1em}\hspace*{1em} have followed has left no trace. {<\textbf{list}>}\mbox{}\newline 
\hspace*{1em}\hspace*{1em}\hspace*{1em}{<\textbf{item}>}Gathering I consists of 7 leaves, a first leaf, originally conjoint with\mbox{}\newline 
\hspace*{1em}\hspace*{1em}\hspace*{1em}{<\textbf{locus}>}fol. 7{</\textbf{locus}>}, having been cut away leaving only a narrow strip along\mbox{}\newline 
\hspace*{1em}\hspace*{1em}\hspace*{1em}\hspace*{1em}\hspace*{1em}\hspace*{1em} the gutter; the others, {<\textbf{locus}>}fols 1{</\textbf{locus}>} and {<\textbf{locus}>}6{</\textbf{locus}>},\mbox{}\newline 
\hspace*{1em}\hspace*{1em}\hspace*{1em}{<\textbf{locus}>}2{</\textbf{locus}>} and {<\textbf{locus}>}5{</\textbf{locus}>}, and {<\textbf{locus}>}3{</\textbf{locus}>} and\mbox{}\newline 
\hspace*{1em}\hspace*{1em}\hspace*{1em}{<\textbf{locus}>}4{</\textbf{locus}>}, are bifolia.{</\textbf{item}>}\mbox{}\newline 
\hspace*{1em}\hspace*{1em}\hspace*{1em}{<\textbf{item}>}Gathering II consists of 8 leaves, 4 bifolia.{</\textbf{item}>}\mbox{}\newline 
\hspace*{1em}\hspace*{1em}\hspace*{1em}{<\textbf{item}>}Gathering III consists of 7 leaves; {<\textbf{locus}>}fols 16{</\textbf{locus}>} and\mbox{}\newline 
\hspace*{1em}\hspace*{1em}\hspace*{1em}{<\textbf{locus}>}22{</\textbf{locus}>} are conjoint, the others singletons.{</\textbf{item}>}\mbox{}\newline 
\hspace*{1em}\hspace*{1em}\hspace*{1em}{<\textbf{item}>}Gathering IV consists of 2 leaves, a bifolium.{</\textbf{item}>}\mbox{}\newline 
\hspace*{1em}\hspace*{1em}{</\textbf{list}>}\mbox{}\newline 
\hspace*{1em}{</\textbf{p}>}\mbox{}\newline 
{</\textbf{collation}>}\mbox{}\newline 
{<\textbf{collation}>}\mbox{}\newline 
\hspace*{1em}{<\textbf{p}>}I (1, 2+9, 3+8, 4+7, 5+6, 10); II (11, 12+17, 13, 14, 15, 16, 18,\mbox{}\newline 
\hspace*{1em}\hspace*{1em} 19).{</\textbf{p}>}\mbox{}\newline 
{</\textbf{collation}>}\mbox{}\newline 
{<\textbf{collation}>}\mbox{}\newline 
\hspace*{1em}{<\textbf{p}>}\mbox{}\newline 
\hspace*{1em}\hspace*{1em}{<\textbf{formula}>}1-5.8 6.6 (catchword, f. 46, does not match following text) 7-8.8\mbox{}\newline 
\hspace*{1em}\hspace*{1em}\hspace*{1em}\hspace*{1em} 9.10, 11.2 (through f. 82) 12-14.8 15.8(-7){</\textbf{formula}>}\mbox{}\newline 
\hspace*{1em}{</\textbf{p}>}\mbox{}\newline 
{</\textbf{collation}>}\end{shaded}\egroup\par 
\paragraph[{Foliation}]{Foliation}\label{msphfo}\par
The \hyperref[TEI.foliation]{<foliation>} element may be used to indicate the scheme, medium or location of folio, page, column, or line numbers written in the manuscript, frequently including a statement about when and, if known, by whom, the numbering was done. \par\bgroup\index{foliation=<foliation>|exampleindex}\index{p=<p>|exampleindex}\index{foliation=<foliation>|exampleindex}\index{p=<p>|exampleindex}\exampleFont \begin{shaded}\noindent\mbox{}{<\textbf{foliation}>}\mbox{}\newline 
\hspace*{1em}{<\textbf{p}>}Neuere Foliierung, die auch das Vorsatzblatt mitgezählt\mbox{}\newline 
\hspace*{1em}\hspace*{1em} hat.{</\textbf{p}>}\mbox{}\newline 
{</\textbf{foliation}>}\mbox{}\newline 
{<\textbf{foliation}>}\mbox{}\newline 
\hspace*{1em}{<\textbf{p}>}Folio numbers were added in brown ink by Árni Magnússon ca.\mbox{}\newline 
\hspace*{1em}\hspace*{1em} 1720-1730 in the upper right corner of all recto-pages.{</\textbf{p}>}\mbox{}\newline 
{</\textbf{foliation}>}\end{shaded}\egroup\par \par
Where a manuscript contains traces of more than one foliation, each should be recorded as a distinct \hyperref[TEI.foliation]{<foliation>} element and optionally given a distinct value for its {\itshape xml:id} attribute. The \hyperref[TEI.locus]{<locus>} element discussed in \textit{\hyperref[msloc]{10.3.5.\ References to Locations within a Manuscript}} can then indicate which foliation scheme is being cited by means of its {\itshape scheme} attribute, which points to this identifier: \par\bgroup\index{foliation=<foliation>|exampleindex}\index{p=<p>|exampleindex}\index{foliation=<foliation>|exampleindex}\index{p=<p>|exampleindex}\index{locus=<locus>|exampleindex}\index{scheme=@scheme!<locus>|exampleindex}\exampleFont \begin{shaded}\noindent\mbox{}{<\textbf{foliation}\hspace*{1em}{xml:id}="{original}">}\mbox{}\newline 
\hspace*{1em}{<\textbf{p}>}Original foliation in red roman numerals in the\mbox{}\newline 
\hspace*{1em}\hspace*{1em} middle of the outer margin of each recto{</\textbf{p}>}\mbox{}\newline 
{</\textbf{foliation}>}\mbox{}\newline 
{<\textbf{foliation}\hspace*{1em}{xml:id}="{modern}">}\mbox{}\newline 
\hspace*{1em}{<\textbf{p}>}Foliated in pencil in the top right corner of each\mbox{}\newline 
\hspace*{1em}\hspace*{1em} recto page.{</\textbf{p}>}\mbox{}\newline 
{</\textbf{foliation}>}\mbox{}\newline 
\textit{<!-- ... -->}\mbox{}\newline 
{<\textbf{locus}\hspace*{1em}{scheme}="{\#modern}">}ff 1-20{</\textbf{locus}>}\end{shaded}\egroup\par 
\paragraph[{Condition}]{Condition}\label{msphco}\par
The \hyperref[TEI.condition]{<condition>} element is used to summarize the overall physical state of a manuscript, in particular where such information is not recorded elsewhere in the description. It should not, however, be used to describe changes or repairs to a manuscript, as these are more appropriately described as a part of its custodial history (see \textit{\hyperref[msadch]{10.9.1.2.\ Availability and Custodial History}}). It should be supplied within the \hyperref[TEI.supportDesc]{<supportDesc>} element, if it discusses the condition of the physical support of the manuscript; within the \hyperref[TEI.bindingDesc]{<bindingDesc>} or \hyperref[TEI.binding]{<binding>} elements (\textit{\hyperref[msphbi]{10.7.3.1.\ Binding Descriptions}}) if it discusses only the condition of the binding or bindings concerned; or within the \hyperref[TEI.sealDesc]{<sealDesc>} element if it discusses the condition of any seal attached to the manuscript.\par\bgroup\index{supportDesc=<supportDesc>|exampleindex}\index{condition=<condition>|exampleindex}\index{p=<p>|exampleindex}\exampleFont \begin{shaded}\noindent\mbox{}{<\textbf{supportDesc}>}\mbox{}\newline 
\hspace*{1em}{<\textbf{condition}>}\mbox{}\newline 
\hspace*{1em}\hspace*{1em}{<\textbf{p}>}The manuscript shows signs of damage from water and mould on its\mbox{}\newline 
\hspace*{1em}\hspace*{1em}\hspace*{1em}\hspace*{1em} outermost leaves.{</\textbf{p}>}\mbox{}\newline 
\hspace*{1em}{</\textbf{condition}>}\mbox{}\newline 
{</\textbf{supportDesc}>}\end{shaded}\egroup\par \par\bgroup\index{condition=<condition>|exampleindex}\index{p=<p>|exampleindex}\exampleFont \begin{shaded}\noindent\mbox{}{<\textbf{condition}>}\mbox{}\newline 
\hspace*{1em}{<\textbf{p}>}Despite tears on many of the leaves the codex is reasonably well\mbox{}\newline 
\hspace*{1em}\hspace*{1em} preserved. The top and the bottom of f. 1 is damaged, and only a thin slip is\mbox{}\newline 
\hspace*{1em}\hspace*{1em} left of the original second leaf (now foliated as 1bis). The lower margin of f.\mbox{}\newline 
\hspace*{1em}\hspace*{1em} 92 has been cut away. There is a lacuna of one leaf between ff. 193 and 194. The\mbox{}\newline 
\hspace*{1em}\hspace*{1em} manuscript ends defectively (there are approximately six leaves\mbox{}\newline 
\hspace*{1em}\hspace*{1em} missing).{</\textbf{p}>}\mbox{}\newline 
{</\textbf{condition}>}\end{shaded}\egroup\par 
\paragraph[{Layout Description}]{Layout Description}\label{msphla}\par
The second part of the \hyperref[TEI.objectDesc]{<objectDesc>} element is the \hyperref[TEI.layoutDesc]{<layoutDesc>} element, which is used to describe and document the \textit{mise-en-page} of the manuscript, that is the way in which text and illumination are arranged on the page, specifying for example the number of written, ruled, or pricked lines and columns per page, size of margins, distinct textual streams such as glosses, commentaries, etc. This may be given as a simple series of paragraphs. Alternatively, one or more different layouts may be identified within a single manuscript, each described by its own \hyperref[TEI.layout]{<layout>} element. 
\begin{sansreflist}
  
\item [\textbf{<layoutDesc>}] (layout description) collects the set of layout descriptions applicable to a manuscript or other object.
\item [\textbf{<layout>}] (layout) describes how text is laid out on the page or surface of the object, including information about any ruling, pricking, or other evidence of page-preparation techniques.
\end{sansreflist}
\par
Where the \hyperref[TEI.layout]{<layout>} element is used, the layout will often be sufficiently regular for the attributes on this element to convey all that is necessary; more usually however a more detailed treatment will be required. The attributes are provided as a convenient shorthand for commonly occurring cases, and should not be used except where the layout is regular. The value \texttt{NA} (not-applicable) should be used for cases where the layout is either very irregular, or where it cannot be characterized simply in terms of lines and columns, for example, where blocks of commentary and text are arranged in a regular but complex pattern on each page\par
The following examples indicate the range of possibilities: \par\bgroup\index{layout=<layout>|exampleindex}\index{ruledLines=@ruledLines!<layout>|exampleindex}\index{p=<p>|exampleindex}\index{layout=<layout>|exampleindex}\index{columns=@columns!<layout>|exampleindex}\index{writtenLines=@writtenLines!<layout>|exampleindex}\index{p=<p>|exampleindex}\index{layout=<layout>|exampleindex}\index{p=<p>|exampleindex}\exampleFont \begin{shaded}\noindent\mbox{}{<\textbf{layout}\hspace*{1em}{ruledLines}="{25 32}">}\mbox{}\newline 
\hspace*{1em}{<\textbf{p}>}Most pages have between 25 and 32 long lines ruled in lead.{</\textbf{p}>}\mbox{}\newline 
{</\textbf{layout}>}\mbox{}\newline 
{<\textbf{layout}\hspace*{1em}{columns}="{1}"\hspace*{1em}{writtenLines}="{24}">}\mbox{}\newline 
\hspace*{1em}{<\textbf{p}>}Written in one column throughout; 24 lines per page.{</\textbf{p}>}\mbox{}\newline 
{</\textbf{layout}>}\mbox{}\newline 
{<\textbf{layout}>}\mbox{}\newline 
\hspace*{1em}{<\textbf{p}>}Written in 3 columns, with 8 lines of text and interlinear glosses in\mbox{}\newline 
\hspace*{1em}\hspace*{1em} the centre, and up to 26 lines of gloss in the outer two columns. Double\mbox{}\newline 
\hspace*{1em}\hspace*{1em} vertical bounding lines ruled in hard point on hair side. Text lines ruled\mbox{}\newline 
\hspace*{1em}\hspace*{1em} faintly in lead. Remains of prickings in upper, lower, and outer (for 8 lines of\mbox{}\newline 
\hspace*{1em}\hspace*{1em} text only) margins.{</\textbf{p}>}\mbox{}\newline 
{</\textbf{layout}>}\end{shaded}\egroup\par \par
Where multiple \hyperref[TEI.layout]{<layout>} elements are supplied, the scope for each specification can be indicated by means of \hyperref[TEI.locus]{<locus>} elements within the content of the element, as in the following example: \par\bgroup\index{layoutDesc=<layoutDesc>|exampleindex}\index{layout=<layout>|exampleindex}\index{ruledLines=@ruledLines!<layout>|exampleindex}\index{p=<p>|exampleindex}\index{locus=<locus>|exampleindex}\index{from=@from!<locus>|exampleindex}\index{to=@to!<locus>|exampleindex}\index{locus=<locus>|exampleindex}\index{from=@from!<locus>|exampleindex}\index{to=@to!<locus>|exampleindex}\index{layout=<layout>|exampleindex}\index{ruledLines=@ruledLines!<layout>|exampleindex}\index{p=<p>|exampleindex}\index{locus=<locus>|exampleindex}\index{from=@from!<locus>|exampleindex}\index{to=@to!<locus>|exampleindex}\exampleFont \begin{shaded}\noindent\mbox{}{<\textbf{layoutDesc}>}\mbox{}\newline 
\hspace*{1em}{<\textbf{layout}\hspace*{1em}{ruledLines}="{25 32}">}\mbox{}\newline 
\hspace*{1em}\hspace*{1em}{<\textbf{p}>}On {<\textbf{locus}\hspace*{1em}{from}="{1r}"\hspace*{1em}{to}="{202v}">}fols 1r-200v{</\textbf{locus}>} and {<\textbf{locus}\hspace*{1em}{from}="{210r}"\hspace*{1em}{to}="{212v}">}fols 210r-212v{</\textbf{locus}>} there are between 25 and 32 ruled lines.{</\textbf{p}>}\mbox{}\newline 
\hspace*{1em}{</\textbf{layout}>}\mbox{}\newline 
\hspace*{1em}{<\textbf{layout}\hspace*{1em}{ruledLines}="{34 50}">}\mbox{}\newline 
\hspace*{1em}\hspace*{1em}{<\textbf{p}>}On {<\textbf{locus}\hspace*{1em}{from}="{203r}"\hspace*{1em}{to}="{209v}">}fols 203r-209v{</\textbf{locus}>} there are between 34\mbox{}\newline 
\hspace*{1em}\hspace*{1em}\hspace*{1em}\hspace*{1em} and 50 ruled lines.{</\textbf{p}>}\mbox{}\newline 
\hspace*{1em}{</\textbf{layout}>}\mbox{}\newline 
{</\textbf{layoutDesc}>}\end{shaded}\egroup\par 
\subsubsection[{Writing, Decoration, and Other Notations}]{Writing, Decoration, and Other Notations}\label{msph2}\par
The second group of elements within a structured physical description concerns aspects of the writing, illumination, or other notation (notably, music) found in a manuscript, including additions made in later hands—the ‘text’, as it were, as opposed to the carrier. 
\begin{sansreflist}
  
\item [\textbf{<handDesc>}] (description of hands) contains a description of all the different hands used in a manuscript or other object.
\item [\textbf{<handNote>}] (note on hand) describes a particular style or hand distinguished within a manuscript.
\item [\textbf{<scriptDesc>}] contains a description of the scripts used in a manuscript or other object.
\item [\textbf{<scriptNote>}] describes a particular script distinguished within the description of a manuscript or similar resource.
\item [\textbf{<typeDesc>}] (typeface description) contains a description of the typefaces or other aspects of the printing of an incunable or other printed source.
\item [\textbf{<typeNote>}] (typographic note) describes a particular font or other significant typographic feature distinguished within the description of a printed resource.
\item [\textbf{<decoDesc>}] (decoration description) contains a description of the decoration of a manuscript or other object, either as in paragraphs, or as one or more \hyperref[TEI.decoNote]{<decoNote>} elements.
\item [\textbf{<decoNote>}] (note on decoration) contains a note describing either a decorative component of a manuscript or other object, or a fairly homogenous class of such components.
\item [\textbf{<musicNotation>}] (music notation) contains description of type of musical notation.
\item [\textbf{<additions>}] (additions) contains a description of any significant additions found within a manuscript or other object, such as marginalia or other annotations.
\end{sansreflist}

\paragraph[{Writing}]{Writing}\label{msphwr}\par
The \hyperref[TEI.handDesc]{<handDesc>} element can contain a short description of the general characteristics of the writing observed in a manuscript, as in the following example: \par\bgroup\index{handDesc=<handDesc>|exampleindex}\index{p=<p>|exampleindex}\index{term=<term>|exampleindex}\index{term=<term>|exampleindex}\exampleFont \begin{shaded}\noindent\mbox{}{<\textbf{handDesc}>}\mbox{}\newline 
\hspace*{1em}{<\textbf{p}>}Written in a {<\textbf{term}>}late Caroline minuscule{</\textbf{term}>}; versals in a form of\mbox{}\newline 
\hspace*{1em}{<\textbf{term}>}rustic capitals{</\textbf{term}>}; although the marginal and interlinear gloss is\mbox{}\newline 
\hspace*{1em}\hspace*{1em} written in varying shades of ink that are not those of the main text, text and\mbox{}\newline 
\hspace*{1em}\hspace*{1em} gloss appear to have been copied during approximately the same time span.{</\textbf{p}>}\mbox{}\newline 
{</\textbf{handDesc}>}\end{shaded}\egroup\par \par
Note the use of the \hyperref[TEI.term]{<term>} element to mark specific technical terms within the context of the \hyperref[TEI.handDesc]{<handDesc>} element.\par
Where several distinct hands have been identified, this fact can be registered by using the {\itshape hands} attribute, as in the following example: \par\bgroup\index{handDesc=<handDesc>|exampleindex}\index{hands=@hands!<handDesc>|exampleindex}\index{p=<p>|exampleindex}\exampleFont \begin{shaded}\noindent\mbox{}{<\textbf{handDesc}\hspace*{1em}{hands}="{2}">}\mbox{}\newline 
\hspace*{1em}{<\textbf{p}>}The manuscript is written in two contemporary hands, otherwise unknown, but\mbox{}\newline 
\hspace*{1em}\hspace*{1em} clearly those of practised scribes. Hand I writes ff. 1r-22v and hand II ff. 23\mbox{}\newline 
\hspace*{1em}\hspace*{1em} and 24. Some scholars, notably Verner Dahlerup and Hreinn Benediktsson, have\mbox{}\newline 
\hspace*{1em}\hspace*{1em} argued for a third hand on f. 24, but the evidence for this is\mbox{}\newline 
\hspace*{1em}\hspace*{1em} insubstantial.{</\textbf{p}>}\mbox{}\newline 
{</\textbf{handDesc}>}\end{shaded}\egroup\par \par
Alternatively, or in addition, where more specific information about one or more of the hands identified is to be recorded, the \hyperref[TEI.handNote]{<handNote>} element should be used, as in the following example: \par\bgroup\index{handDesc=<handDesc>|exampleindex}\index{hands=@hands!<handDesc>|exampleindex}\index{handNote=<handNote>|exampleindex}\index{scope=@scope!<handNote>|exampleindex}\index{p=<p>|exampleindex}\index{locus=<locus>|exampleindex}\index{from=@from!<locus>|exampleindex}\index{to=@to!<locus>|exampleindex}\index{handNote=<handNote>|exampleindex}\index{scope=@scope!<handNote>|exampleindex}\index{p=<p>|exampleindex}\index{locus=<locus>|exampleindex}\index{from=@from!<locus>|exampleindex}\index{to=@to!<locus>|exampleindex}\index{title=<title>|exampleindex}\index{ref=<ref>|exampleindex}\index{handNote=<handNote>|exampleindex}\index{scope=@scope!<handNote>|exampleindex}\index{p=<p>|exampleindex}\index{ref=<ref>|exampleindex}\exampleFont \begin{shaded}\noindent\mbox{}{<\textbf{handDesc}\hspace*{1em}{hands}="{3}">}\mbox{}\newline 
\hspace*{1em}{<\textbf{handNote}\hspace*{1em}{xml:id}="{Eirsp-1}"\hspace*{1em}{scope}="{minor}">}\mbox{}\newline 
\hspace*{1em}\hspace*{1em}{<\textbf{p}>}The first part of the manuscript, {<\textbf{locus}\hspace*{1em}{from}="{1v}"\hspace*{1em}{to}="{72v:4}">}fols\mbox{}\newline 
\hspace*{1em}\hspace*{1em}\hspace*{1em}\hspace*{1em}\hspace*{1em}\hspace*{1em} 1v-72v:4{</\textbf{locus}>}, is written in a practised Icelandic Gothic bookhand. This hand\mbox{}\newline 
\hspace*{1em}\hspace*{1em}\hspace*{1em}\hspace*{1em} is not found elsewhere.{</\textbf{p}>}\mbox{}\newline 
\hspace*{1em}{</\textbf{handNote}>}\mbox{}\newline 
\hspace*{1em}{<\textbf{handNote}\hspace*{1em}{xml:id}="{Eirsp-2}"\hspace*{1em}{scope}="{major}">}\mbox{}\newline 
\hspace*{1em}\hspace*{1em}{<\textbf{p}>}The second part of the manuscript, {<\textbf{locus}\hspace*{1em}{from}="{72v:4}"\hspace*{1em}{to}="{194v}">}fols\mbox{}\newline 
\hspace*{1em}\hspace*{1em}\hspace*{1em}\hspace*{1em}\hspace*{1em}\hspace*{1em} 72v:4-194{</\textbf{locus}>}, is written in a hand contemporary with the first; it can also\mbox{}\newline 
\hspace*{1em}\hspace*{1em}\hspace*{1em}\hspace*{1em} be found in a fragment of {<\textbf{title}>}Knýtlinga saga{</\textbf{title}>}, {<\textbf{ref}>}AM 20b II\mbox{}\newline 
\hspace*{1em}\hspace*{1em}\hspace*{1em}\hspace*{1em}\hspace*{1em}\hspace*{1em} fol.{</\textbf{ref}>}.{</\textbf{p}>}\mbox{}\newline 
\hspace*{1em}{</\textbf{handNote}>}\mbox{}\newline 
\hspace*{1em}{<\textbf{handNote}\hspace*{1em}{xml:id}="{Eirsp-3}"\hspace*{1em}{scope}="{minor}">}\mbox{}\newline 
\hspace*{1em}\hspace*{1em}{<\textbf{p}>}The third hand has written the majority of the chapter headings. This hand\mbox{}\newline 
\hspace*{1em}\hspace*{1em}\hspace*{1em}\hspace*{1em} has been identified as the one also found in {<\textbf{ref}>}AM 221\mbox{}\newline 
\hspace*{1em}\hspace*{1em}\hspace*{1em}\hspace*{1em}\hspace*{1em}\hspace*{1em} fol.{</\textbf{ref}>}.{</\textbf{p}>}\mbox{}\newline 
\hspace*{1em}{</\textbf{handNote}>}\mbox{}\newline 
{</\textbf{handDesc}>}\end{shaded}\egroup\par \noindent  Note here the use of the \hyperref[TEI.locus]{<locus>} element, discussed in section \textit{\hyperref[msloc]{10.3.5.\ References to Locations within a Manuscript}}, to specify exactly which parts of a manuscript are written by a given hand.\par
When a full or partial transcription of a manuscript is available in addition to the manuscript description, the \hyperref[TEI.handShift]{<handShift>} element described in \textit{\hyperref[PHDH]{11.3.2.1.\ Document Hands}} can be used to link the relevant parts of the transcription to the appropriate \hyperref[TEI.handNote]{<handNote>} element in the description: for example, at the point in the transcript where the second hand listed above starts (i.e. at folio 72v:4), we might insert <handShift new="\#Eirsp-2"/>.\par
Additions, notes, drawings etc. (e.g. \hyperref[TEI.add]{<add>}, \hyperref[TEI.note]{<note>} and \hyperref[TEI.figure]{<figure>}) made by other hands in the text, can be linked to the corresponding \hyperref[TEI.handNote]{<handNote>} element using the {\itshape hand} attribute.\par
The elements \hyperref[TEI.typeDesc]{<typeDesc>}, and \hyperref[TEI.typeNote]{<typeNote>} are used to provide information about the printing of a source, in exactly the same way as the \hyperref[TEI.handDesc]{<handDesc>} or \hyperref[TEI.handNote]{<handNote>} elements provide information about its writing. They are provided for the convenience of those using this module to provide information about early printed sources and incunables. The \hyperref[TEI.typeDesc]{<typeDesc>} element can simply provide a summary description: \par\bgroup\index{typeDesc=<typeDesc>|exampleindex}\index{p=<p>|exampleindex}\exampleFont \begin{shaded}\noindent\mbox{}{<\textbf{typeDesc}>}\mbox{}\newline 
\hspace*{1em}{<\textbf{p}>}Uses a mixture of Roman and Black Letter types.{</\textbf{p}>}\mbox{}\newline 
{</\textbf{typeDesc}>}\end{shaded}\egroup\par \par
Where detailed information about individual typefaces is to be recorded, this may be done using the \hyperref[TEI.typeNote]{<typeNote>} element: \par\bgroup\index{typeDesc=<typeDesc>|exampleindex}\index{summary=<summary>|exampleindex}\index{typeNote=<typeNote>|exampleindex}\index{typeNote=<typeNote>|exampleindex}\exampleFont \begin{shaded}\noindent\mbox{}{<\textbf{typeDesc}>}\mbox{}\newline 
\hspace*{1em}{<\textbf{summary}>}Uses a mixture of Roman and Black Letter types.{</\textbf{summary}>}\mbox{}\newline 
\hspace*{1em}{<\textbf{typeNote}>}Antiqua typeface, showing influence of Jenson's Venetian\mbox{}\newline 
\hspace*{1em}\hspace*{1em} fonts.{</\textbf{typeNote}>}\mbox{}\newline 
\hspace*{1em}{<\textbf{typeNote}>}The black letter face is a variant of Schwabacher.{</\textbf{typeNote}>}\mbox{}\newline 
{</\textbf{typeDesc}>}\end{shaded}\egroup\par \par
Where information is required about both typography and written script, for example where a printed book contains extensive handwritten annotation, both \hyperref[TEI.handDesc]{<handDesc>} and \hyperref[TEI.typeDesc]{<typeDesc>} elements should be supplied. Similarly, in the following example, the source text is a typescript with extensive handwritten annotation: \par\bgroup\index{typeDesc=<typeDesc>|exampleindex}\index{typeNote=<typeNote>|exampleindex}\index{handDesc=<handDesc>|exampleindex}\index{handNote=<handNote>|exampleindex}\index{medium=@medium!<handNote>|exampleindex}\index{handNote=<handNote>|exampleindex}\index{medium=@medium!<handNote>|exampleindex}\exampleFont \begin{shaded}\noindent\mbox{}{<\textbf{typeDesc}>}\mbox{}\newline 
\hspace*{1em}{<\textbf{typeNote}\hspace*{1em}{xml:id}="{TSET}">}Authorial typescript, probably produced on Eliot's own\mbox{}\newline 
\hspace*{1em}\hspace*{1em} Remington. {</\textbf{typeNote}>}\mbox{}\newline 
{</\textbf{typeDesc}>}\mbox{}\newline 
{<\textbf{handDesc}>}\mbox{}\newline 
\hspace*{1em}{<\textbf{handNote}\hspace*{1em}{xml:id}="{EP}"\hspace*{1em}{medium}="{red-ink}">}Ezra Pound's annotations.{</\textbf{handNote}>}\mbox{}\newline 
\hspace*{1em}{<\textbf{handNote}\hspace*{1em}{xml:id}="{TSE}"\hspace*{1em}{medium}="{black-ink}">}Commentary in Eliot's hand.{</\textbf{handNote}>}\mbox{}\newline 
{</\textbf{handDesc}>}\end{shaded}\egroup\par \par
The elements \hyperref[TEI.scriptNote]{<scriptNote>} and \hyperref[TEI.scriptDesc]{<scriptDesc>} may be used in exactly the same way to document a script used in this and other manuscripts, for example to record that this script was used mainly for the production of books or for charters; or that it is characteristic of some geographical area or scriptorium or date. Such information as the letter forms characteristic of this script may also be recorded. By contrast, the \hyperref[TEI.handNote]{<handNote>} element would be used to document the way that a particular scribe uses a script, for example with long or short descenders, or using a pen which is cut in a different way, or an ink of a given colour, and so forth.\par
As with \hyperref[TEI.typeNote]{<typeNote>}, the \hyperref[TEI.scriptNote]{<scriptNote>} element can be used in combination with \hyperref[TEI.handNote]{<handNote>}.
\paragraph[{Decoration}]{Decoration}\label{msphdec}\par
It can be difficult to draw a clear distinction between aspects of a manuscript which are purely physical and those which form part of its intellectual content. This is particularly true of illuminations and other forms of decoration in a manuscript. We propose the following elements for the purpose of delimiting discussion of these aspects within a manuscript description, and for convenience locate them all within the physical description, despite the fact that the illustrative features of a manuscript will in many cases also be seen as constituting part of its intellectual content.\par
The \hyperref[TEI.decoDesc]{<decoDesc>} element may contain simply one or more paragraphs summarizing the overall nature of the decorative features of the manuscript, as in the following example: \par\bgroup\index{decoDesc=<decoDesc>|exampleindex}\index{p=<p>|exampleindex}\exampleFont \begin{shaded}\noindent\mbox{}{<\textbf{decoDesc}>}\mbox{}\newline 
\hspace*{1em}{<\textbf{p}>}The decoration comprises two full page miniatures, perhaps added by the\mbox{}\newline 
\hspace*{1em}\hspace*{1em} original owner, or slightly later; the original major decoration consists of\mbox{}\newline 
\hspace*{1em}\hspace*{1em} twenty-three large miniatures, illustrating the divisions of the Passion\mbox{}\newline 
\hspace*{1em}\hspace*{1em} narrative and the start of the major texts, and the major divisions of the\mbox{}\newline 
\hspace*{1em}\hspace*{1em} Hours; seventeen smaller miniatures, illustrating the suffrages to saints; and\mbox{}\newline 
\hspace*{1em}\hspace*{1em} seven historiated initials, illustrating the pericopes and major prayers.{</\textbf{p}>}\mbox{}\newline 
{</\textbf{decoDesc}>}\end{shaded}\egroup\par \noindent  Alternatively, it may contain a series of more specific typed \hyperref[TEI.decoNote]{<decoNote>} elements, each summarizing a particular aspect or individual instance of the decoration present, for example the use of miniatures, initials (historiated or otherwise), borders, diagrams, etc., as in the following example: \par\bgroup\index{decoDesc=<decoDesc>|exampleindex}\index{decoNote=<decoNote>|exampleindex}\index{type=@type!<decoNote>|exampleindex}\index{p=<p>|exampleindex}\index{decoNote=<decoNote>|exampleindex}\index{type=@type!<decoNote>|exampleindex}\index{p=<p>|exampleindex}\index{decoNote=<decoNote>|exampleindex}\index{type=@type!<decoNote>|exampleindex}\index{p=<p>|exampleindex}\index{decoNote=<decoNote>|exampleindex}\index{type=@type!<decoNote>|exampleindex}\index{p=<p>|exampleindex}\index{decoNote=<decoNote>|exampleindex}\index{type=@type!<decoNote>|exampleindex}\index{p=<p>|exampleindex}\exampleFont \begin{shaded}\noindent\mbox{}{<\textbf{decoDesc}>}\mbox{}\newline 
\hspace*{1em}{<\textbf{decoNote}\hspace*{1em}{type}="{miniature}">}\mbox{}\newline 
\hspace*{1em}\hspace*{1em}{<\textbf{p}>}One full-page miniature, facing the beginning of the first Penitential\mbox{}\newline 
\hspace*{1em}\hspace*{1em}\hspace*{1em}\hspace*{1em} Psalm.{</\textbf{p}>}\mbox{}\newline 
\hspace*{1em}{</\textbf{decoNote}>}\mbox{}\newline 
\hspace*{1em}{<\textbf{decoNote}\hspace*{1em}{type}="{initial}">}\mbox{}\newline 
\hspace*{1em}\hspace*{1em}{<\textbf{p}>}One seven-line historiated initial, commencing the first Penitential\mbox{}\newline 
\hspace*{1em}\hspace*{1em}\hspace*{1em}\hspace*{1em} Psalm.{</\textbf{p}>}\mbox{}\newline 
\hspace*{1em}{</\textbf{decoNote}>}\mbox{}\newline 
\hspace*{1em}{<\textbf{decoNote}\hspace*{1em}{type}="{initial}">}\mbox{}\newline 
\hspace*{1em}\hspace*{1em}{<\textbf{p}>}Six four-line decorated initials, commencing the second through the seventh\mbox{}\newline 
\hspace*{1em}\hspace*{1em}\hspace*{1em}\hspace*{1em} Penitential Psalm.{</\textbf{p}>}\mbox{}\newline 
\hspace*{1em}{</\textbf{decoNote}>}\mbox{}\newline 
\hspace*{1em}{<\textbf{decoNote}\hspace*{1em}{type}="{initial}">}\mbox{}\newline 
\hspace*{1em}\hspace*{1em}{<\textbf{p}>}Some three hundred two-line versal initials with pen-flourishes, commencing\mbox{}\newline 
\hspace*{1em}\hspace*{1em}\hspace*{1em}\hspace*{1em} the psalm verses.{</\textbf{p}>}\mbox{}\newline 
\hspace*{1em}{</\textbf{decoNote}>}\mbox{}\newline 
\hspace*{1em}{<\textbf{decoNote}\hspace*{1em}{type}="{border}">}\mbox{}\newline 
\hspace*{1em}\hspace*{1em}{<\textbf{p}>}Four-sided border decoration surrounding the miniatures and three-sided\mbox{}\newline 
\hspace*{1em}\hspace*{1em}\hspace*{1em}\hspace*{1em} border decoration accompanying the historiated and decorated initials.{</\textbf{p}>}\mbox{}\newline 
\hspace*{1em}{</\textbf{decoNote}>}\mbox{}\newline 
{</\textbf{decoDesc}>}\end{shaded}\egroup\par \par
Where more exact indexing of the decorative content of a manuscript is required, the standard TEI elements \hyperref[TEI.term]{<term>} or \hyperref[TEI.index]{<index>} may be used within the prose description to supply or delimit appropriate iconographic terms, as in the following example: \par\bgroup\index{decoDesc=<decoDesc>|exampleindex}\index{decoNote=<decoNote>|exampleindex}\index{type=@type!<decoNote>|exampleindex}\index{p=<p>|exampleindex}\index{list=<list>|exampleindex}\index{item=<item>|exampleindex}\index{locus=<locus>|exampleindex}\index{term=<term>|exampleindex}\index{item=<item>|exampleindex}\index{locus=<locus>|exampleindex}\index{term=<term>|exampleindex}\index{item=<item>|exampleindex}\index{locus=<locus>|exampleindex}\index{term=<term>|exampleindex}\index{term=<term>|exampleindex}\index{item=<item>|exampleindex}\index{locus=<locus>|exampleindex}\index{term=<term>|exampleindex}\index{term=<term>|exampleindex}\exampleFont \begin{shaded}\noindent\mbox{}{<\textbf{decoDesc}>}\mbox{}\newline 
\hspace*{1em}{<\textbf{decoNote}\hspace*{1em}{type}="{miniatures}">}\mbox{}\newline 
\hspace*{1em}\hspace*{1em}{<\textbf{p}>}Fourteen large miniatures with arched tops, above five lines of text: {<\textbf{list}>}\mbox{}\newline 
\hspace*{1em}\hspace*{1em}\hspace*{1em}\hspace*{1em}{<\textbf{item}>}\mbox{}\newline 
\hspace*{1em}\hspace*{1em}\hspace*{1em}\hspace*{1em}\hspace*{1em}{<\textbf{locus}>}fol. 14r{</\textbf{locus}>}Pericopes. {<\textbf{term}>}St. John writing on Patmos{</\textbf{term}>},\mbox{}\newline 
\hspace*{1em}\hspace*{1em}\hspace*{1em}\hspace*{1em}\hspace*{1em}\hspace*{1em}\hspace*{1em}\hspace*{1em} with the Eagle holding his ink-pot and pen-case; some flaking of pigment,\mbox{}\newline 
\hspace*{1em}\hspace*{1em}\hspace*{1em}\hspace*{1em}\hspace*{1em}\hspace*{1em}\hspace*{1em}\hspace*{1em} especially in the sky{</\textbf{item}>}\mbox{}\newline 
\hspace*{1em}\hspace*{1em}\hspace*{1em}\hspace*{1em}{<\textbf{item}>}\mbox{}\newline 
\hspace*{1em}\hspace*{1em}\hspace*{1em}\hspace*{1em}\hspace*{1em}{<\textbf{locus}>}fol. 26r{</\textbf{locus}>}Hours of the Virgin, Matins.\mbox{}\newline 
\hspace*{1em}\hspace*{1em}\hspace*{1em}\hspace*{1em}{<\textbf{term}>}Annunciation{</\textbf{term}>}; Gabriel and the Dove to the right{</\textbf{item}>}\mbox{}\newline 
\hspace*{1em}\hspace*{1em}\hspace*{1em}\hspace*{1em}{<\textbf{item}>}\mbox{}\newline 
\hspace*{1em}\hspace*{1em}\hspace*{1em}\hspace*{1em}\hspace*{1em}{<\textbf{locus}>}fol. 60r{</\textbf{locus}>}Prime. {<\textbf{term}>}Nativity{</\textbf{term}>}; the {<\textbf{term}>}Virgin and\mbox{}\newline 
\hspace*{1em}\hspace*{1em}\hspace*{1em}\hspace*{1em}\hspace*{1em}\hspace*{1em}\hspace*{1em}\hspace*{1em}\hspace*{1em}\hspace*{1em} Joseph adoring the Child{</\textbf{term}>}\mbox{}\newline 
\hspace*{1em}\hspace*{1em}\hspace*{1em}\hspace*{1em}{</\textbf{item}>}\mbox{}\newline 
\hspace*{1em}\hspace*{1em}\hspace*{1em}\hspace*{1em}{<\textbf{item}>}\mbox{}\newline 
\hspace*{1em}\hspace*{1em}\hspace*{1em}\hspace*{1em}\hspace*{1em}{<\textbf{locus}>}fol. 66r{</\textbf{locus}>}Terce. {<\textbf{term}>}Annunciation to the Shepherds{</\textbf{term}>},\mbox{}\newline 
\hspace*{1em}\hspace*{1em}\hspace*{1em}\hspace*{1em}\hspace*{1em}\hspace*{1em}\hspace*{1em}\hspace*{1em} one with {<\textbf{term}>}bagpipes{</\textbf{term}>}\mbox{}\newline 
\hspace*{1em}\hspace*{1em}\hspace*{1em}\hspace*{1em}{</\textbf{item}>}\mbox{}\newline 
\textit{<!-- ... -->}\mbox{}\newline 
\hspace*{1em}\hspace*{1em}\hspace*{1em}{</\textbf{list}>}\mbox{}\newline 
\hspace*{1em}\hspace*{1em}{</\textbf{p}>}\mbox{}\newline 
\hspace*{1em}{</\textbf{decoNote}>}\mbox{}\newline 
{</\textbf{decoDesc}>}\end{shaded}\egroup\par 
\paragraph[{Musical Notation}]{Musical Notation}\label{msphmu}\par
Where a manuscript contains music, the \hyperref[TEI.musicNotation]{<musicNotation>} element may be used to describe the form of notation employed, as in the following examples: \par\bgroup\index{musicNotation=<musicNotation>|exampleindex}\index{p=<p>|exampleindex}\exampleFont \begin{shaded}\noindent\mbox{}{<\textbf{musicNotation}>}\mbox{}\newline 
\hspace*{1em}{<\textbf{p}>}Square notation on 4-line red staves.{</\textbf{p}>}\mbox{}\newline 
{</\textbf{musicNotation}>}\end{shaded}\egroup\par \noindent  \par\bgroup\index{musicNotation=<musicNotation>|exampleindex}\index{p=<p>|exampleindex}\exampleFont \begin{shaded}\noindent\mbox{}{<\textbf{musicNotation}>}\mbox{}\newline 
\hspace*{1em}{<\textbf{p}>}Neumes in campo aperto of the St. Gall type.{</\textbf{p}>}\mbox{}\newline 
{</\textbf{musicNotation}>}\end{shaded}\egroup\par \par
If a manuscript employs more than one notation, they must both be described within the same \hyperref[TEI.musicNotation]{<musicNotation>} element, for example as different list items. 
\paragraph[{Additions and Marginalia}]{Additions and Marginalia}\label{mspham}\par
The \hyperref[TEI.additions]{<additions>} element can be used to list or describe any additions to the manuscript, such as marginalia, scribblings, doodles, etc., which are considered to be of interest or importance. Such topics may also be discussed or referenced elsewhere in a description, for example in the \hyperref[TEI.history]{<history>} element, in cases where the marginalia provide evidence of ownership. Note that this element may not be repeated within a single manuscript description. If several different kinds of additional matter are discussed, the content may be structured as a labelled list or a series of paragraphs. Some examples follow: \par\bgroup\index{additions=<additions>|exampleindex}\index{p=<p>|exampleindex}\exampleFont \begin{shaded}\noindent\mbox{}{<\textbf{additions}>}\mbox{}\newline 
\hspace*{1em}{<\textbf{p}>}Doodles on most leaves, possibly by children, and often quite\mbox{}\newline 
\hspace*{1em}\hspace*{1em} amusing.{</\textbf{p}>}\mbox{}\newline 
{</\textbf{additions}>}\end{shaded}\egroup\par \noindent  \par\bgroup\index{additions=<additions>|exampleindex}\index{p=<p>|exampleindex}\exampleFont \begin{shaded}\noindent\mbox{}{<\textbf{additions}>}\mbox{}\newline 
\hspace*{1em}{<\textbf{p}\hspace*{1em}{xml:lang}="{fr}">}Quelques annotations marginales des XVIe et XVIIe s.{</\textbf{p}>}\mbox{}\newline 
{</\textbf{additions}>}\end{shaded}\egroup\par \noindent  \par\bgroup\index{additions=<additions>|exampleindex}\index{p=<p>|exampleindex}\index{p=<p>|exampleindex}\exampleFont \begin{shaded}\noindent\mbox{}{<\textbf{additions}>}\mbox{}\newline 
\hspace*{1em}{<\textbf{p}>}The text of this manuscript is not interpolated with sentences from Royal\mbox{}\newline 
\hspace*{1em}\hspace*{1em} decrees promulgated in 1294, 1305 and 1314. In the margins, however, another\mbox{}\newline 
\hspace*{1em}\hspace*{1em} somewhat later scribe has added the relevant paragraphs of these decrees, see\mbox{}\newline 
\hspace*{1em}\hspace*{1em} pp. 8, 24, 44, 47 etc.{</\textbf{p}>}\mbox{}\newline 
\hspace*{1em}{<\textbf{p}>}As a humorous gesture the scribe in one opening of the manuscript, pp. 36 and\mbox{}\newline 
\hspace*{1em}\hspace*{1em} 37, has prolonged the lower stems of one letter f and five letters þ and has\mbox{}\newline 
\hspace*{1em}\hspace*{1em} them drizzle down the margin.{</\textbf{p}>}\mbox{}\newline 
{</\textbf{additions}>}\end{shaded}\egroup\par \noindent  \par\bgroup\index{additions=<additions>|exampleindex}\index{p=<p>|exampleindex}\index{locus=<locus>|exampleindex}\index{locus=<locus>|exampleindex}\index{locus=<locus>|exampleindex}\index{locus=<locus>|exampleindex}\index{locus=<locus>|exampleindex}\index{locus=<locus>|exampleindex}\index{quote=<quote>|exampleindex}\index{ex=<ex>|exampleindex}\index{ex=<ex>|exampleindex}\index{locus=<locus>|exampleindex}\index{quote=<quote>|exampleindex}\index{ex=<ex>|exampleindex}\index{locus=<locus>|exampleindex}\index{quote=<quote>|exampleindex}\index{ex=<ex>|exampleindex}\index{ex=<ex>|exampleindex}\index{ex=<ex>|exampleindex}\index{p=<p>|exampleindex}\index{list=<list>|exampleindex}\index{item=<item>|exampleindex}\index{locus=<locus>|exampleindex}\index{quote=<quote>|exampleindex}\index{ex=<ex>|exampleindex}\index{lb=<lb>|exampleindex}\index{item=<item>|exampleindex}\index{locus=<locus>|exampleindex}\index{quote=<quote>|exampleindex}\index{ex=<ex>|exampleindex}\index{ex=<ex>|exampleindex}\index{ex=<ex>|exampleindex}\index{ex=<ex>|exampleindex}\index{ex=<ex>|exampleindex}\index{item=<item>|exampleindex}\index{locus=<locus>|exampleindex}\index{quote=<quote>|exampleindex}\index{sic=<sic>|exampleindex}\index{lb=<lb>|exampleindex}\index{lb=<lb>|exampleindex}\index{p=<p>|exampleindex}\index{locus=<locus>|exampleindex}\index{locus=<locus>|exampleindex}\index{locus=<locus>|exampleindex}\index{locus=<locus>|exampleindex}\exampleFont \begin{shaded}\noindent\mbox{}{<\textbf{additions}>}\mbox{}\newline 
\hspace*{1em}{<\textbf{p}>}Spaces for initials and chapter headings were left by the scribe but not\mbox{}\newline 
\hspace*{1em}\hspace*{1em} filled in. A later, probably fifteenth-century, hand has added initials and\mbox{}\newline 
\hspace*{1em}\hspace*{1em} chapter headings in greenish-coloured ink on fols {<\textbf{locus}>}8r{</\textbf{locus}>},\mbox{}\newline 
\hspace*{1em}{<\textbf{locus}>}8v{</\textbf{locus}>}, {<\textbf{locus}>}9r{</\textbf{locus}>}, {<\textbf{locus}>}10r{</\textbf{locus}>} and {<\textbf{locus}>}11r{</\textbf{locus}>}.\mbox{}\newline 
\hspace*{1em}\hspace*{1em} Although a few of these chapter headings are now rather difficult to read, most\mbox{}\newline 
\hspace*{1em}\hspace*{1em} can be made out, e.g. fol. {<\textbf{locus}>}8rb{</\textbf{locus}>}\mbox{}\newline 
\hspace*{1em}\hspace*{1em}{<\textbf{quote}\hspace*{1em}{xml:lang}="{is}">}floti ast{<\textbf{ex}>}ri{</\textbf{ex}>}d{<\textbf{ex}>}ar{</\textbf{ex}>}\mbox{}\newline 
\hspace*{1em}\hspace*{1em}{</\textbf{quote}>}; fol.\mbox{}\newline 
\hspace*{1em}{<\textbf{locus}>}9rb{</\textbf{locus}>}\mbox{}\newline 
\hspace*{1em}\hspace*{1em}{<\textbf{quote}\hspace*{1em}{xml:lang}="{is}">}v{<\textbf{ex}>}m{</\textbf{ex}>} olaf conung{</\textbf{quote}>}, and fol.\mbox{}\newline 
\hspace*{1em}{<\textbf{locus}>}10ra{</\textbf{locus}>}\mbox{}\newline 
\hspace*{1em}\hspace*{1em}{<\textbf{quote}\hspace*{1em}{xml:lang}="{is}">}Gipti{<\textbf{ex}>}n{</\textbf{ex}>}g ol{<\textbf{ex}>}a{</\textbf{ex}>}fs\mbox{}\newline 
\hspace*{1em}\hspace*{1em}\hspace*{1em}\hspace*{1em} k{<\textbf{ex}>}onun{</\textbf{ex}>}gs{</\textbf{quote}>}.{</\textbf{p}>}\mbox{}\newline 
\hspace*{1em}{<\textbf{p}>}The manuscript contains the following marginalia: {<\textbf{list}>}\mbox{}\newline 
\hspace*{1em}\hspace*{1em}\hspace*{1em}{<\textbf{item}>}Fol. {<\textbf{locus}>}4v{</\textbf{locus}>}, left margin: {<\textbf{quote}\hspace*{1em}{xml:lang}="{is}">}hialmadr\mbox{}\newline 
\hspace*{1em}\hspace*{1em}\hspace*{1em}\hspace*{1em}{<\textbf{ex}>}ok{</\textbf{ex}>}\mbox{}\newline 
\hspace*{1em}\hspace*{1em}\hspace*{1em}\hspace*{1em}\hspace*{1em}{<\textbf{lb}/>}brynjadr{</\textbf{quote}>}, in a fifteenth-century hand, imitating an addition made\mbox{}\newline 
\hspace*{1em}\hspace*{1em}\hspace*{1em}\hspace*{1em}\hspace*{1em}\hspace*{1em} to the text by the scribe at this point.{</\textbf{item}>}\mbox{}\newline 
\hspace*{1em}\hspace*{1em}\hspace*{1em}{<\textbf{item}>}Fol. {<\textbf{locus}>}5r{</\textbf{locus}>}, lower margin: {<\textbf{quote}\hspace*{1em}{xml:lang}="{is}">}þ{<\textbf{ex}>}e{</\textbf{ex}>}tta\mbox{}\newline 
\hspace*{1em}\hspace*{1em}\hspace*{1em}\hspace*{1em}\hspace*{1em}\hspace*{1em}\hspace*{1em}\hspace*{1em} þiki m{<\textbf{ex}>}er{</\textbf{ex}>} v{<\textbf{ex}>}er{</\textbf{ex}>}a gott blek en{<\textbf{ex}>}n{</\textbf{ex}>}da kan{<\textbf{ex}>}n{</\textbf{ex}>} ek icki\mbox{}\newline 
\hspace*{1em}\hspace*{1em}\hspace*{1em}\hspace*{1em}\hspace*{1em}\hspace*{1em}\hspace*{1em}\hspace*{1em} betr sia{</\textbf{quote}>}, in a fifteenth-century hand, probably the same as that on the\mbox{}\newline 
\hspace*{1em}\hspace*{1em}\hspace*{1em}\hspace*{1em}\hspace*{1em}\hspace*{1em} previous page.{</\textbf{item}>}\mbox{}\newline 
\hspace*{1em}\hspace*{1em}\hspace*{1em}{<\textbf{item}>}Fol. {<\textbf{locus}>}9v{</\textbf{locus}>}, bottom margin: {<\textbf{quote}\hspace*{1em}{xml:lang}="{is}">}þessa bok\mbox{}\newline 
\hspace*{1em}\hspace*{1em}\hspace*{1em}\hspace*{1em}\hspace*{1em}\hspace*{1em}\hspace*{1em}\hspace*{1em} uilda eg {<\textbf{sic}>}gæt{</\textbf{sic}>} lært med\mbox{}\newline 
\hspace*{1em}\hspace*{1em}\hspace*{1em}\hspace*{1em}{<\textbf{lb}/>}an Gud gefe myer Gott ad\mbox{}\newline 
\hspace*{1em}\hspace*{1em}\hspace*{1em}\hspace*{1em}{<\textbf{lb}/>}læra{</\textbf{quote}>}; seventeenth-century hand.{</\textbf{item}>}\mbox{}\newline 
\hspace*{1em}\hspace*{1em}{</\textbf{list}>}\mbox{}\newline 
\hspace*{1em}{</\textbf{p}>}\mbox{}\newline 
\hspace*{1em}{<\textbf{p}>}There are in addition a number of illegible scribbles in a later hand (or\mbox{}\newline 
\hspace*{1em}\hspace*{1em} hands) on fols {<\textbf{locus}>}2r{</\textbf{locus}>}, {<\textbf{locus}>}3r{</\textbf{locus}>}, {<\textbf{locus}>}5v{</\textbf{locus}>} and\mbox{}\newline 
\hspace*{1em}{<\textbf{locus}>}19r{</\textbf{locus}>}.{</\textbf{p}>}\mbox{}\newline 
{</\textbf{additions}>}\end{shaded}\egroup\par 
\subsubsection[{Bindings, Seals, and Additional Material}]{Bindings, Seals, and Additional Material}\label{msph3}\par
The third major component of the physical description relates to supporting but distinct physical components, such as bindings, seals and accompanying material. These may be described using the following specialist elements: 
\begin{sansreflist}
  
\item [\textbf{<bindingDesc>}] (binding description) describes the present and former bindings of a manuscript or other object, either as a series of paragraphs or as a series of distinct \hyperref[TEI.binding]{<binding>} elements, one for each binding of the manuscript.
\item [\textbf{<binding>}] (binding) contains a description of one binding, i.e. type of covering, boards, etc. applied to a manuscript or other object.
\item [\textbf{<sealDesc>}] (seal description) describes the seals or similar items related to the object described, either as a series of paragraphs or as a series of \hyperref[TEI.seal]{<seal>} elements.
\item [\textbf{<seal>}] (seal) contains a description of one seal or similar applied to the object described
\item [\textbf{<accMat>}] (accompanying material) contains details of any significant additional material which may be closely associated with the manuscript or object being described, such as non-contemporaneous documents or fragments bound in with it at some earlier historical period.
\end{sansreflist}

\paragraph[{Binding Descriptions}]{Binding Descriptions}\label{msphbi}\par
The \hyperref[TEI.bindingDesc]{<bindingDesc>} element contains a description of the state of the present and former bindings of a manuscript, including information about its material, any distinctive marks, and provenance information. This may be given as a series of paragraphs if only one binding is being described, or as a series of distinct \hyperref[TEI.binding]{<binding>} elements, each describing a distinct binding where these are separately described. For example: \par\bgroup\index{bindingDesc=<bindingDesc>|exampleindex}\index{p=<p>|exampleindex}\exampleFont \begin{shaded}\noindent\mbox{}{<\textbf{bindingDesc}>}\mbox{}\newline 
\hspace*{1em}{<\textbf{p}>}Sewing not visible; tightly rebound over 19th-century pasteboards, reusing\mbox{}\newline 
\hspace*{1em}\hspace*{1em} panels of 16th-century brown leather with gilt tooling à la fanfare, Paris c.\mbox{}\newline 
\hspace*{1em}\hspace*{1em} 1580-90, the centre of each cover inlaid with a 17th-century oval medallion of\mbox{}\newline 
\hspace*{1em}\hspace*{1em} red morocco tooled in gilt (perhaps replacing the identifying mark of a previous\mbox{}\newline 
\hspace*{1em}\hspace*{1em} owner); the spine similarly tooled, without raised bands or title-piece;\mbox{}\newline 
\hspace*{1em}\hspace*{1em} coloured endbands; the edges of the leaves and boards gilt. Boxed.{</\textbf{p}>}\mbox{}\newline 
{</\textbf{bindingDesc}>}\end{shaded}\egroup\par \par
Within a binding description, the elements \hyperref[TEI.decoNote]{<decoNote>} and \hyperref[TEI.condition]{<condition>} are available, as alternatives to \hyperref[TEI.p]{<p>}, for paragraphs dealing exclusively with information about decorative features of a binding, or about its condition, respectively. \par\bgroup\index{binding=<binding>|exampleindex}\index{p=<p>|exampleindex}\index{material=<material>|exampleindex}\index{decoNote=<decoNote>|exampleindex}\index{p=<p>|exampleindex}\index{decoNote=<decoNote>|exampleindex}\index{p=<p>|exampleindex}\index{condition=<condition>|exampleindex}\index{p=<p>|exampleindex}\exampleFont \begin{shaded}\noindent\mbox{}{<\textbf{binding}>}\mbox{}\newline 
\hspace*{1em}{<\textbf{p}>}Bound, s. XVIII (?), in {<\textbf{material}>}diced russia leather{</\textbf{material}>}\mbox{}\newline 
\hspace*{1em}\hspace*{1em} retaining most of the original 15th century metal ornaments (but with some\mbox{}\newline 
\hspace*{1em}\hspace*{1em} replacements) as well as the heavy wooden boards.{</\textbf{p}>}\mbox{}\newline 
\hspace*{1em}{<\textbf{decoNote}>}\mbox{}\newline 
\hspace*{1em}\hspace*{1em}{<\textbf{p}>}On each cover: alternating circular stamps of the Holy Monogram, a\mbox{}\newline 
\hspace*{1em}\hspace*{1em}\hspace*{1em}\hspace*{1em} sunburst, and a flower.{</\textbf{p}>}\mbox{}\newline 
\hspace*{1em}{</\textbf{decoNote}>}\mbox{}\newline 
\hspace*{1em}{<\textbf{decoNote}>}\mbox{}\newline 
\hspace*{1em}\hspace*{1em}{<\textbf{p}>}On the cornerpieces, one of which is missing, a rectangular stamp\mbox{}\newline 
\hspace*{1em}\hspace*{1em}\hspace*{1em}\hspace*{1em} of the Agnus Dei.{</\textbf{p}>}\mbox{}\newline 
\hspace*{1em}{</\textbf{decoNote}>}\mbox{}\newline 
\hspace*{1em}{<\textbf{condition}>}Front and back leather inlaid panels very badly worn.{</\textbf{condition}>}\mbox{}\newline 
\hspace*{1em}{<\textbf{p}>}Rebacked during the 19th century.{</\textbf{p}>}\mbox{}\newline 
{</\textbf{binding}>}\end{shaded}\egroup\par \par
As noted above, (\textit{\hyperref[msphco]{10.7.1.5.\ Condition}}) the element \hyperref[TEI.condition]{<condition>} may also be used as an alternative to \hyperref[TEI.p]{<p>} for paragraphs concerned exclusively with the condition of a binding, where this has not been supplied as part of the physical description.
\paragraph[{Seals}]{Seals}\label{msphse}\par
The \hyperref[TEI.sealDesc]{<sealDesc>} element supplies information about the seal(s) attached to documents to guarantee their integrity, or to show authentication of the issuer or consent of the participants. It may contain one or more paragraphs summarizing the overall nature of the seals, or may contain one or more \hyperref[TEI.seal]{<seal>} elements. \par\bgroup\index{sealDesc=<sealDesc>|exampleindex}\index{seal=<seal>|exampleindex}\index{n=@n!<seal>|exampleindex}\index{type=@type!<seal>|exampleindex}\index{subtype=@subtype!<seal>|exampleindex}\index{p=<p>|exampleindex}\index{name=<name>|exampleindex}\index{bibl=<bibl>|exampleindex}\index{ref=<ref>|exampleindex}\index{quote=<quote>|exampleindex}\index{seal=<seal>|exampleindex}\index{n=@n!<seal>|exampleindex}\index{type=@type!<seal>|exampleindex}\index{subtype=@subtype!<seal>|exampleindex}\index{p=<p>|exampleindex}\index{name=<name>|exampleindex}\index{bibl=<bibl>|exampleindex}\index{ref=<ref>|exampleindex}\index{quote=<quote>|exampleindex}\index{quote=<quote>|exampleindex}\exampleFont \begin{shaded}\noindent\mbox{}{<\textbf{sealDesc}>}\mbox{}\newline 
\hspace*{1em}{<\textbf{seal}\hspace*{1em}{n}="{1}"\hspace*{1em}{type}="{pendant}"\mbox{}\newline 
\hspace*{1em}\hspace*{1em}{subtype}="{cauda\textunderscore duplex}">}\mbox{}\newline 
\hspace*{1em}\hspace*{1em}{<\textbf{p}>}Round seal of {<\textbf{name}>}Anders Olufsen{</\textbf{name}>} in black wax: {<\textbf{bibl}>}\mbox{}\newline 
\hspace*{1em}\hspace*{1em}\hspace*{1em}\hspace*{1em}{<\textbf{ref}>}DAS\mbox{}\newline 
\hspace*{1em}\hspace*{1em}\hspace*{1em}\hspace*{1em}\hspace*{1em}\hspace*{1em}\hspace*{1em}\hspace*{1em} 930{</\textbf{ref}>}\mbox{}\newline 
\hspace*{1em}\hspace*{1em}\hspace*{1em}{</\textbf{bibl}>}. Parchment tag, on which is written: {<\textbf{quote}>}pertinere nos\mbox{}\newline 
\hspace*{1em}\hspace*{1em}\hspace*{1em}\hspace*{1em}\hspace*{1em}\hspace*{1em} predictorum placiti nostri iusticarii precessorum dif{</\textbf{quote}>}.{</\textbf{p}>}\mbox{}\newline 
\hspace*{1em}{</\textbf{seal}>}\mbox{}\newline 
\hspace*{1em}{<\textbf{seal}\hspace*{1em}{n}="{2}"\hspace*{1em}{type}="{pendant}"\mbox{}\newline 
\hspace*{1em}\hspace*{1em}{subtype}="{cauda\textunderscore duplex}">}\mbox{}\newline 
\hspace*{1em}\hspace*{1em}{<\textbf{p}>}The seal of {<\textbf{name}>}Jens Olufsen{</\textbf{name}>} in black wax: {<\textbf{bibl}>}\mbox{}\newline 
\hspace*{1em}\hspace*{1em}\hspace*{1em}\hspace*{1em}{<\textbf{ref}>}DAS\mbox{}\newline 
\hspace*{1em}\hspace*{1em}\hspace*{1em}\hspace*{1em}\hspace*{1em}\hspace*{1em}\hspace*{1em}\hspace*{1em} 1061{</\textbf{ref}>}\mbox{}\newline 
\hspace*{1em}\hspace*{1em}\hspace*{1em}{</\textbf{bibl}>}. Legend: {<\textbf{quote}>}S IOHANNES OLAVI{</\textbf{quote}>}. Parchment tag on\mbox{}\newline 
\hspace*{1em}\hspace*{1em}\hspace*{1em}\hspace*{1em} which is written: {<\textbf{quote}>}Woldorp Iohanne G{</\textbf{quote}>}.{</\textbf{p}>}\mbox{}\newline 
\hspace*{1em}{</\textbf{seal}>}\mbox{}\newline 
{</\textbf{sealDesc}>}\end{shaded}\egroup\par 
\paragraph[{Accompanying Material}]{Accompanying Material}\label{msadac}\par
The circumstance may arise where material not originally part of a manuscript is bound into or otherwise kept with a manuscript. In some cases this material would best be treated in a separate \hyperref[TEI.msPart]{<msPart>} element (see \textit{\hyperref[mspt]{10.10.\ Manuscript Parts}} below). There are, however, cases where the additional matter is not self-evidently a distinct manuscript: it might, for example, be a set of notes by a later scholar, or a file of correspondence relating to the manuscript. The \hyperref[TEI.accMat]{<accMat>} element is provided as a holder for this kind of information. 
\begin{sansreflist}
  
\item [\textbf{<accMat>}] (accompanying material) contains details of any significant additional material which may be closely associated with the manuscript or object being described, such as non-contemporaneous documents or fragments bound in with it at some earlier historical period.
\end{sansreflist}
\par
Here is an example of the use of this element, describing a note by the Icelandic manuscript collector Árni Magnússon which has been bound with the manuscript: \par\bgroup\index{accMat=<accMat>|exampleindex}\index{p=<p>|exampleindex}\index{quote=<quote>|exampleindex}\index{ex=<ex>|exampleindex}\index{ex=<ex>|exampleindex}\index{ex=<ex>|exampleindex}\exampleFont \begin{shaded}\noindent\mbox{}{<\textbf{accMat}>}\mbox{}\newline 
\hspace*{1em}{<\textbf{p}>}A slip in Árni Magnússon's hand has been stuck to the pastedown on the inside\mbox{}\newline 
\hspace*{1em}\hspace*{1em} front cover; the text reads: {<\textbf{quote}\hspace*{1em}{xml:lang}="{is}">}Þidreks Søgu þessa hefi eg\mbox{}\newline 
\hspace*{1em}\hspace*{1em}\hspace*{1em}\hspace*{1em} feiged af Sekreterer Wielandt Anno 1715 i Kaupmanna høfn. Hun er, sem eg sie,\mbox{}\newline 
\hspace*{1em}\hspace*{1em}\hspace*{1em}\hspace*{1em} Copia af Austfirda bókinni (Eidagás) en{<\textbf{ex}>}n{</\textbf{ex}>} ecki progenies Brædratungu\mbox{}\newline 
\hspace*{1em}\hspace*{1em}\hspace*{1em}\hspace*{1em} bokarinnar. Og er þar fyrer eigi i allan{<\textbf{ex}>}n{</\textbf{ex}>} máta samhlioda\mbox{}\newline 
\hspace*{1em}\hspace*{1em}\hspace*{1em}\hspace*{1em} þ{<\textbf{ex}>}eir{</\textbf{ex}>}re er Sr Jon Erlendz son hefer ritad fyrer Mag. Bryniolf. Þesse\mbox{}\newline 
\hspace*{1em}\hspace*{1em}\hspace*{1em}\hspace*{1em} Þidreks Saga mun vera komin fra Sr Vigfuse á Helgafelle.{</\textbf{quote}>}\mbox{}\newline 
\hspace*{1em}{</\textbf{p}>}\mbox{}\newline 
{</\textbf{accMat}>}\end{shaded}\egroup\par 
\subsection[{History}]{History}\label{mshy}\par
The following elements are used to record information about the history of a manuscript: 
\begin{sansreflist}
  
\item [\textbf{<history>}] (history) groups elements describing the full history of a manuscript, manuscript part, or other object.
\item [\textbf{<origin>}] (origin) contains any descriptive or other information concerning the origin of a manuscript, manuscript part, or other object.
\item [\textbf{<provenance>}] (provenance) contains any descriptive or other information concerning a single identifiable episode during the history of a manuscript, manuscript part, or other object after its creation but before its acquisition.
\item [\textbf{<acquisition>}] (acquisition) contains any descriptive or other information concerning the process by which a manuscript or manuscript part or other object entered the holding institution.
\end{sansreflist}
\par
The three components of the \hyperref[TEI.history]{<history>} element all have the same substructure, consisting of one or more paragraphs marked as \hyperref[TEI.p]{<p>} elements. Each of these three elements is also a member of the \textsf{att.datable} attribute class, itself a member of the \textsf{att.datable.w3c} class, and thus also carries the following optional attributes: 
\begin{sansreflist}
  
\item [\textbf{att.datable.w3c}] provides attributes for normalization of elements that contain datable events conforming to the W3C \textit{XML Schema Part 2: Datatypes Second Edition}.\hfil\\[-10pt]\begin{sansreflist}
    \item[@{\itshape notBefore}]
  specifies the earliest possible date for the event in standard form, e.g. yyyy-mm-dd.
    \item[@{\itshape notAfter}]
  specifies the latest possible date for the event in standard form, e.g. yyyy-mm-dd.
\end{sansreflist}  
\end{sansreflist}
\par
Information about the origins of the manuscript, its place and date of writing, should be given as one or more paragraphs contained by a single \hyperref[TEI.origin]{<origin>} element; following this, any available information on distinct stages in the history of the manuscript before its acquisition by its current holding institution should be included as paragraphs within one or more \hyperref[TEI.provenance]{<provenance>} elements. Finally, any information specific to the means by which the manuscript was acquired by its present owners should be given as paragraphs within the \hyperref[TEI.acquisition]{<acquisition>} element.\par
Here is a fairly simple example of the use of this element: \par\bgroup\index{history=<history>|exampleindex}\index{origin=<origin>|exampleindex}\index{p=<p>|exampleindex}\index{origPlace=<origPlace>|exampleindex}\index{origDate=<origDate>|exampleindex}\index{notBefore=@notBefore!<origDate>|exampleindex}\index{notAfter=@notAfter!<origDate>|exampleindex}\index{provenance=<provenance>|exampleindex}\index{p=<p>|exampleindex}\index{name=<name>|exampleindex}\index{type=@type!<name>|exampleindex}\index{date=<date>|exampleindex}\index{date=<date>|exampleindex}\index{provenance=<provenance>|exampleindex}\index{p=<p>|exampleindex}\index{name=<name>|exampleindex}\index{type=@type!<name>|exampleindex}\index{name=<name>|exampleindex}\index{type=@type!<name>|exampleindex}\index{name=<name>|exampleindex}\index{type=@type!<name>|exampleindex}\index{name=<name>|exampleindex}\index{type=@type!<name>|exampleindex}\index{acquisition=<acquisition>|exampleindex}\index{p=<p>|exampleindex}\index{name=<name>|exampleindex}\index{type=@type!<name>|exampleindex}\index{date=<date>|exampleindex}\index{name=<name>|exampleindex}\index{type=@type!<name>|exampleindex}\index{name=<name>|exampleindex}\index{type=@type!<name>|exampleindex}\exampleFont \begin{shaded}\noindent\mbox{}{<\textbf{history}>}\mbox{}\newline 
\hspace*{1em}{<\textbf{origin}>}\mbox{}\newline 
\hspace*{1em}\hspace*{1em}{<\textbf{p}>}Written in {<\textbf{origPlace}>}Durham{</\textbf{origPlace}>} during {<\textbf{origDate}\hspace*{1em}{notBefore}="{1125}"\mbox{}\newline 
\hspace*{1em}\hspace*{1em}\hspace*{1em}\hspace*{1em}{notAfter}="{1175}">}the mid-twelfth\mbox{}\newline 
\hspace*{1em}\hspace*{1em}\hspace*{1em}\hspace*{1em}\hspace*{1em}\hspace*{1em} century{</\textbf{origDate}>}.{</\textbf{p}>}\mbox{}\newline 
\hspace*{1em}{</\textbf{origin}>}\mbox{}\newline 
\hspace*{1em}{<\textbf{provenance}>}\mbox{}\newline 
\hspace*{1em}\hspace*{1em}{<\textbf{p}>}Recorded in two medieval catalogues of the books belonging to\mbox{}\newline 
\hspace*{1em}\hspace*{1em}{<\textbf{name}\hspace*{1em}{type}="{org}">}Durham Priory{</\textbf{name}>}, made in {<\textbf{date}>}1391{</\textbf{date}>} and\mbox{}\newline 
\hspace*{1em}\hspace*{1em}{<\textbf{date}>}1405{</\textbf{date}>}.{</\textbf{p}>}\mbox{}\newline 
\hspace*{1em}{</\textbf{provenance}>}\mbox{}\newline 
\hspace*{1em}{<\textbf{provenance}>}\mbox{}\newline 
\hspace*{1em}\hspace*{1em}{<\textbf{p}>}Given to {<\textbf{name}\hspace*{1em}{type}="{person}">}W. Olleyf{</\textbf{name}>} by {<\textbf{name}\hspace*{1em}{type}="{person}">}William Ebchester, Prior (1446-56){</\textbf{name}>} and later belonged to\mbox{}\newline 
\hspace*{1em}\hspace*{1em}{<\textbf{name}\hspace*{1em}{type}="{person}">}Henry Dalton{</\textbf{name}>}, Prior of Holy Island ({<\textbf{name}\hspace*{1em}{type}="{place}">}Lindisfarne{</\textbf{name}>}) according to inscriptions on ff. 4v and 5.{</\textbf{p}>}\mbox{}\newline 
\hspace*{1em}{</\textbf{provenance}>}\mbox{}\newline 
\hspace*{1em}{<\textbf{acquisition}>}\mbox{}\newline 
\hspace*{1em}\hspace*{1em}{<\textbf{p}>}Presented to {<\textbf{name}\hspace*{1em}{type}="{org}">}Trinity College{</\textbf{name}>} in\mbox{}\newline 
\hspace*{1em}\hspace*{1em}{<\textbf{date}>}1738{</\textbf{date}>} by {<\textbf{name}\hspace*{1em}{type}="{person}">}Thomas Gale{</\textbf{name}>} and his son {<\textbf{name}\hspace*{1em}{type}="{person}">}Roger{</\textbf{name}>}.{</\textbf{p}>}\mbox{}\newline 
\hspace*{1em}{</\textbf{acquisition}>}\mbox{}\newline 
{</\textbf{history}>}\end{shaded}\egroup\par \par
Here is a fuller example, demonstrating the use of multiple \hyperref[TEI.provenance]{<provenance>} elements where distinct periods of ownership for the manuscript have been identified: \par\bgroup\index{history=<history>|exampleindex}\index{origin=<origin>|exampleindex}\index{notBefore=@notBefore!<origin>|exampleindex}\index{notAfter=@notAfter!<origin>|exampleindex}\index{provenance=<provenance>|exampleindex}\index{name=<name>|exampleindex}\index{type=@type!<name>|exampleindex}\index{provenance=<provenance>|exampleindex}\index{foreign=<foreign>|exampleindex}\index{name=<name>|exampleindex}\index{type=@type!<name>|exampleindex}\index{date=<date>|exampleindex}\index{date=<date>|exampleindex}\index{provenance=<provenance>|exampleindex}\index{foreign=<foreign>|exampleindex}\index{name=<name>|exampleindex}\index{type=@type!<name>|exampleindex}\index{acquisition=<acquisition>|exampleindex}\index{notBefore=@notBefore!<acquisition>|exampleindex}\index{notAfter=@notAfter!<acquisition>|exampleindex}\exampleFont \begin{shaded}\noindent\mbox{}{<\textbf{history}>}\mbox{}\newline 
\hspace*{1em}{<\textbf{origin}\hspace*{1em}{notBefore}="{1225}"\hspace*{1em}{notAfter}="{1275}">} Written in Spain or Portugal in the\mbox{}\newline 
\hspace*{1em}\hspace*{1em} middle of the 13th century (the date 1042, given in a marginal note on f. 97v,\mbox{}\newline 
\hspace*{1em}\hspace*{1em} cannot be correct.){</\textbf{origin}>}\mbox{}\newline 
\hspace*{1em}{<\textbf{provenance}>}The Spanish scholar {<\textbf{name}\hspace*{1em}{type}="{person}">}Benito Arias Montano{</\textbf{name}>}\mbox{}\newline 
\hspace*{1em}\hspace*{1em} (1527-1598) has written his name on f. 97r, and may be presumed to have owned\mbox{}\newline 
\hspace*{1em}\hspace*{1em} the manuscript. {</\textbf{provenance}>}\mbox{}\newline 
\hspace*{1em}{<\textbf{provenance}>}It came somehow into the possession of {<\textbf{foreign}\hspace*{1em}{xml:lang}="{da}">}etatsråd{</\textbf{foreign}>}\mbox{}\newline 
\hspace*{1em}\hspace*{1em}{<\textbf{name}\hspace*{1em}{type}="{person}">}Holger Parsberg{</\textbf{name}>} (1636-1692), who has written his name\mbox{}\newline 
\hspace*{1em}\hspace*{1em} twice, once on the front pastedown and once on f. 1r, the former dated\mbox{}\newline 
\hspace*{1em}{<\textbf{date}>}1680{</\textbf{date}>} and the latter {<\textbf{date}>}1682{</\textbf{date}>}.{</\textbf{provenance}>}\mbox{}\newline 
\hspace*{1em}{<\textbf{provenance}>}Following Parsberg's death the manuscript was bought by\mbox{}\newline 
\hspace*{1em}{<\textbf{foreign}>}etatsråd{</\textbf{foreign}>}\mbox{}\newline 
\hspace*{1em}\hspace*{1em}{<\textbf{name}\hspace*{1em}{type}="{person}">}Jens Rosenkrantz{</\textbf{name}>} (1640-1695) when Parsberg's library\mbox{}\newline 
\hspace*{1em}\hspace*{1em} was auctioned off (23 October 1693).{</\textbf{provenance}>}\mbox{}\newline 
\hspace*{1em}{<\textbf{acquisition}\hspace*{1em}{notBefore}="{1696}"\mbox{}\newline 
\hspace*{1em}\hspace*{1em}{notAfter}="{1697}">}The manuscript was acquired by\mbox{}\newline 
\hspace*{1em}\hspace*{1em} Árni Magnússon from the estate of Jens Rosenkrantz, presumably at auction (the\mbox{}\newline 
\hspace*{1em}\hspace*{1em} auction lot number 468 is written in red chalk on the flyleaf), either in 1696\mbox{}\newline 
\hspace*{1em}\hspace*{1em} or 97.{</\textbf{acquisition}>}\mbox{}\newline 
{</\textbf{history}>}\end{shaded}\egroup\par 
\subsection[{Additional Information}]{Additional Information}\label{msad}\par
Three categories of additional information are provided for by the scheme described here, grouped together within the \hyperref[TEI.additional]{<additional>} element described in this section. 
\begin{sansreflist}
  
\item [\textbf{<additional>}] (additional) groups additional information, combining bibliographic information about a manuscript or other object, or surrogate copies of it, with curatorial or administrative information.
\item [\textbf{<adminInfo>}] (administrative information) contains information about the present custody and availability of the manuscript or other object, and also about the record description itself.
\item [\textbf{<surrogates>}] (surrogates) contains information about any representations of the manuscript or other object being described which may exist in the holding institution or elsewhere.
\item [\textbf{<listBibl>}] (citation list) contains a list of bibliographic citations of any kind.
\end{sansreflist}
\par
None of the constituent elements of \hyperref[TEI.additional]{<additional>} is required. If any is supplied, it may appear once only; furthermore, the order in which elements are supplied should be as specified above.
\subsubsection[{Administrative Information}]{Administrative Information}\label{msadad}\par
The \hyperref[TEI.adminInfo]{<adminInfo>} element is used to hold information relating to the curation and management of a manuscript. This may be supplied as a note using the global \hyperref[TEI.note]{<note>} element. Alternatively, different aspects of this information may be presented grouped within one of the following specialized elements: 
\begin{sansreflist}
  
\item [\textbf{<recordHist>}] (recorded history) provides information about the source and revision status of the parent manuscript or object description itself.
\item [\textbf{<availability>}] (availability) supplies information about the availability of a text, for example any restrictions on its use or distribution, its copyright status, any licence applying to it, etc.
\item [\textbf{<custodialHist>}] (custodial history) contains a description of a manuscript or other object's custodial history, either as running prose or as a series of dated custodial events.
\end{sansreflist}

\paragraph[{Record History}]{Record History}\label{msrh}\par
The \hyperref[TEI.recordHist]{<recordHist>} element may contain simply a series of paragraphs. Alternatively it may contain a \hyperref[TEI.source]{<source>} element, followed by an optional series of \hyperref[TEI.change]{<change>} elements. 
\begin{sansreflist}
  
\item [\textbf{<source>}] (source) describes the original source for the information contained with a manuscript or object description.
\item [\textbf{<change>}] (change) documents a change or set of changes made during the production of a source document, or during the revision of an electronic file.
\end{sansreflist}
\par
The \hyperref[TEI.source]{<source>} element is used to document the primary source of information for the record containing it, in a similar way to the standard TEI \hyperref[TEI.sourceDesc]{<sourceDesc>} element within a TEI Header. If the record is a new one, made without reference to anything other than the manuscript itself, then it may simply contain a \hyperref[TEI.p]{<p>} element, as in the following example: \par\bgroup\index{source=<source>|exampleindex}\index{p=<p>|exampleindex}\exampleFont \begin{shaded}\noindent\mbox{}{<\textbf{source}>}\mbox{}\newline 
\hspace*{1em}{<\textbf{p}>}Directly catalogued from the original manuscript.{</\textbf{p}>}\mbox{}\newline 
{</\textbf{source}>}\end{shaded}\egroup\par \par
Frequently, however, the record will be derived from some previously existing description, which may be specified using the standard TEI \hyperref[TEI.bibl]{<bibl>} element, as in the following example: \par\bgroup\index{recordHist=<recordHist>|exampleindex}\index{source=<source>|exampleindex}\index{p=<p>|exampleindex}\index{bibl=<bibl>|exampleindex}\index{title=<title>|exampleindex}\index{biblScope=<biblScope>|exampleindex}\exampleFont \begin{shaded}\noindent\mbox{}{<\textbf{recordHist}>}\mbox{}\newline 
\hspace*{1em}{<\textbf{source}>}\mbox{}\newline 
\hspace*{1em}\hspace*{1em}{<\textbf{p}>}Information transcribed from {<\textbf{bibl}>}\mbox{}\newline 
\hspace*{1em}\hspace*{1em}\hspace*{1em}\hspace*{1em}{<\textbf{title}>}The index of Middle English\mbox{}\newline 
\hspace*{1em}\hspace*{1em}\hspace*{1em}\hspace*{1em}\hspace*{1em}\hspace*{1em}\hspace*{1em}\hspace*{1em} verse{</\textbf{title}>}\mbox{}\newline 
\hspace*{1em}\hspace*{1em}\hspace*{1em}\hspace*{1em}{<\textbf{biblScope}>}123{</\textbf{biblScope}>}\mbox{}\newline 
\hspace*{1em}\hspace*{1em}\hspace*{1em}{</\textbf{bibl}>}.{</\textbf{p}>}\mbox{}\newline 
\hspace*{1em}{</\textbf{source}>}\mbox{}\newline 
{</\textbf{recordHist}>}\end{shaded}\egroup\par \par
If, as is likely, a full bibliographic description of the source from which cataloguing information was taken is included within the \hyperref[TEI.listBibl]{<listBibl>} element contained by the current \hyperref[TEI.additional]{<additional>} element, or elsewhere in the current document, then it need not be repeated here. Instead, it should be referenced using the standard TEI \hyperref[TEI.ref]{<ref>} element, as in the following example: \par\bgroup\index{additional=<additional>|exampleindex}\index{adminInfo=<adminInfo>|exampleindex}\index{recordHist=<recordHist>|exampleindex}\index{source=<source>|exampleindex}\index{p=<p>|exampleindex}\index{bibl=<bibl>|exampleindex}\index{ref=<ref>|exampleindex}\index{target=@target!<ref>|exampleindex}\index{listBibl=<listBibl>|exampleindex}\index{bibl=<bibl>|exampleindex}\index{author=<author>|exampleindex}\index{author=<author>|exampleindex}\index{title=<title>|exampleindex}\index{level=@level!<title>|exampleindex}\index{pubPlace=<pubPlace>|exampleindex}\index{date=<date>|exampleindex}\exampleFont \begin{shaded}\noindent\mbox{}{<\textbf{additional}>}\mbox{}\newline 
\hspace*{1em}{<\textbf{adminInfo}>}\mbox{}\newline 
\hspace*{1em}\hspace*{1em}{<\textbf{recordHist}>}\mbox{}\newline 
\hspace*{1em}\hspace*{1em}\hspace*{1em}{<\textbf{source}>}\mbox{}\newline 
\hspace*{1em}\hspace*{1em}\hspace*{1em}\hspace*{1em}{<\textbf{p}>}Information transcribed from {<\textbf{bibl}>}\mbox{}\newline 
\hspace*{1em}\hspace*{1em}\hspace*{1em}\hspace*{1em}\hspace*{1em}\hspace*{1em}{<\textbf{ref}\hspace*{1em}{target}="{\#IMEV}">}IMEV{</\textbf{ref}>}\mbox{}\newline 
\hspace*{1em}\hspace*{1em}\hspace*{1em}\hspace*{1em}\hspace*{1em}\hspace*{1em}\hspace*{1em}\hspace*{1em}\hspace*{1em}\hspace*{1em} 123{</\textbf{bibl}>}.{</\textbf{p}>}\mbox{}\newline 
\hspace*{1em}\hspace*{1em}\hspace*{1em}{</\textbf{source}>}\mbox{}\newline 
\hspace*{1em}\hspace*{1em}{</\textbf{recordHist}>}\mbox{}\newline 
\hspace*{1em}{</\textbf{adminInfo}>}\mbox{}\newline 
\hspace*{1em}{<\textbf{listBibl}>}\mbox{}\newline 
\hspace*{1em}\hspace*{1em}{<\textbf{bibl}\hspace*{1em}{xml:id}="{IMEV}">}\mbox{}\newline 
\hspace*{1em}\hspace*{1em}\hspace*{1em}{<\textbf{author}>}Carleton Brown{</\textbf{author}>} and {<\textbf{author}>}Rossell Hope Robbins{</\textbf{author}>}\mbox{}\newline 
\hspace*{1em}\hspace*{1em}\hspace*{1em}{<\textbf{title}\hspace*{1em}{level}="{m}">}The index of Middle English verse{</\textbf{title}>}\mbox{}\newline 
\hspace*{1em}\hspace*{1em}\hspace*{1em}{<\textbf{pubPlace}>}New York{</\textbf{pubPlace}>}\mbox{}\newline 
\hspace*{1em}\hspace*{1em}\hspace*{1em}{<\textbf{date}>}1943{</\textbf{date}>}\mbox{}\newline 
\hspace*{1em}\hspace*{1em}{</\textbf{bibl}>}\mbox{}\newline 
\textit{<!-- other bibliographic records relating to this manuscript here -->}\mbox{}\newline 
\hspace*{1em}{</\textbf{listBibl}>}\mbox{}\newline 
{</\textbf{additional}>}\end{shaded}\egroup\par \par
The \hyperref[TEI.change]{<change>} element may also appear within the \hyperref[TEI.revisionDesc]{<revisionDesc>} element of the standard TEI header; its use here is intended to signal the similarity of function between the two container elements. Where the TEI header should be used to document the revision history of the whole electronic file to which it is prefixed, the \hyperref[TEI.recordHist]{<recordHist>} element may be used to document changes at a lower level, relating to the individual description, as in the following example: \par\bgroup\index{change=<change>|exampleindex}\index{when=@when!<change>|exampleindex}\index{name=<name>|exampleindex}\exampleFont \begin{shaded}\noindent\mbox{}{<\textbf{change}\hspace*{1em}{when}="{2005-03-10}">}On 10 March 2005 {<\textbf{name}>}MJD{</\textbf{name}>} added provenance\mbox{}\newline 
 information{</\textbf{change}>}\end{shaded}\egroup\par 
\paragraph[{Availability and Custodial History}]{Availability and Custodial History}\label{msadch}\par
The \hyperref[TEI.availability]{<availability>} element is another element also available in the TEI header, which should be used here to supply any information concerning access to the current manuscript, such as its physical location (where this is not implicit in its identifier), any restrictions on access, information about copyright, etc. \par\bgroup\index{availability=<availability>|exampleindex}\index{p=<p>|exampleindex}\index{availability=<availability>|exampleindex}\index{p=<p>|exampleindex}\index{availability=<availability>|exampleindex}\index{p=<p>|exampleindex}\exampleFont \begin{shaded}\noindent\mbox{}{<\textbf{availability}>}\mbox{}\newline 
\hspace*{1em}{<\textbf{p}>}Viewed by appointment only, to be arranged with curator.{</\textbf{p}>}\mbox{}\newline 
{</\textbf{availability}>}\mbox{}\newline 
{<\textbf{availability}>}\mbox{}\newline 
\hspace*{1em}{<\textbf{p}>}In conservation, Jan. - Mar., 2002. On loan to the Bayerische\mbox{}\newline 
\hspace*{1em}\hspace*{1em} Staatsbibliothek, April - July, 2002.{</\textbf{p}>}\mbox{}\newline 
{</\textbf{availability}>}\mbox{}\newline 
{<\textbf{availability}>}\mbox{}\newline 
\hspace*{1em}{<\textbf{p}>}The manuscript is in poor condition, due to many of the leaves being brittle\mbox{}\newline 
\hspace*{1em}\hspace*{1em} and fragile and the poor quality of a number of earlier repairs; it should\mbox{}\newline 
\hspace*{1em}\hspace*{1em} therefore not be used or lent out until it has been conserved.{</\textbf{p}>}\mbox{}\newline 
{</\textbf{availability}>}\end{shaded}\egroup\par \par
The \hyperref[TEI.custodialHist]{<custodialHist>} record is used to describe the custodial history of a manuscript, recording any significant events noted during the period that it has been located within its holding institution. It may contain either a series of \hyperref[TEI.p]{<p>} elements, or a series of \hyperref[TEI.custEvent]{<custEvent>} elements, each describing a distinct incident or event, further specified by a {\itshape type} attribute, and carrying dating information by virtue of its membership in the \textsf{att.datable} class, as noted above. 
\begin{sansreflist}
  
\item [\textbf{<custEvent>}] (custodial event) describes a single event during the custodial history of a manuscript or other object.
\end{sansreflist}
\par
Here is an example of the use of this element: \par\bgroup\index{custodialHist=<custodialHist>|exampleindex}\index{custEvent=<custEvent>|exampleindex}\index{type=@type!<custEvent>|exampleindex}\index{notBefore=@notBefore!<custEvent>|exampleindex}\index{notAfter=@notAfter!<custEvent>|exampleindex}\index{p=<p>|exampleindex}\index{custEvent=<custEvent>|exampleindex}\index{type=@type!<custEvent>|exampleindex}\index{notBefore=@notBefore!<custEvent>|exampleindex}\index{notAfter=@notAfter!<custEvent>|exampleindex}\index{p=<p>|exampleindex}\index{custEvent=<custEvent>|exampleindex}\index{type=@type!<custEvent>|exampleindex}\index{notBefore=@notBefore!<custEvent>|exampleindex}\index{notAfter=@notAfter!<custEvent>|exampleindex}\index{p=<p>|exampleindex}\exampleFont \begin{shaded}\noindent\mbox{}{<\textbf{custodialHist}>}\mbox{}\newline 
\hspace*{1em}{<\textbf{custEvent}\hspace*{1em}{type}="{conservation}"\mbox{}\newline 
\hspace*{1em}\hspace*{1em}{notBefore}="{1961-03-01}"\hspace*{1em}{notAfter}="{1963-02-28}">}\mbox{}\newline 
\hspace*{1em}\hspace*{1em}{<\textbf{p}>}Conserved between March 1961 and February 1963 at Birgitte Dalls\mbox{}\newline 
\hspace*{1em}\hspace*{1em}\hspace*{1em}\hspace*{1em} Konserveringsværksted.{</\textbf{p}>}\mbox{}\newline 
\hspace*{1em}{</\textbf{custEvent}>}\mbox{}\newline 
\hspace*{1em}{<\textbf{custEvent}\hspace*{1em}{type}="{photography}"\mbox{}\newline 
\hspace*{1em}\hspace*{1em}{notBefore}="{1988-05-01}"\hspace*{1em}{notAfter}="{1988-05-30}">}\mbox{}\newline 
\hspace*{1em}\hspace*{1em}{<\textbf{p}>}Photographed in May 1988 by AMI/FA.{</\textbf{p}>}\mbox{}\newline 
\hspace*{1em}{</\textbf{custEvent}>}\mbox{}\newline 
\hspace*{1em}{<\textbf{custEvent}\hspace*{1em}{type}="{transfer}"\mbox{}\newline 
\hspace*{1em}\hspace*{1em}{notBefore}="{1989-11-13}"\hspace*{1em}{notAfter}="{1989-11-13}">}\mbox{}\newline 
\hspace*{1em}\hspace*{1em}{<\textbf{p}>}Dispatched to Iceland 13 November 1989.{</\textbf{p}>}\mbox{}\newline 
\hspace*{1em}{</\textbf{custEvent}>}\mbox{}\newline 
{</\textbf{custodialHist}>}\end{shaded}\egroup\par 
\subsubsection[{Surrogates}]{Surrogates}\label{msadsu}\par
The \hyperref[TEI.surrogates]{<surrogates>} element is used to provide information about representations such as photographs or other representations of the manuscript which may exist within the holding institution or elsewhere. 
\begin{sansreflist}
  
\item [\textbf{<surrogates>}] (surrogates) contains information about any representations of the manuscript or other object being described which may exist in the holding institution or elsewhere.
\end{sansreflist}
\par
The \hyperref[TEI.surrogates]{<surrogates>} element should not be used to repeat information about representations of the manuscript available within published works; this should normally be documented within the \hyperref[TEI.listBibl]{<listBibl>} element within the \hyperref[TEI.additional]{<additional>} element. However, it is often also convenient to record information such as negative numbers or digital identifiers for unpublished collections of manuscript images maintained within the holding institution, as well as to provide more detailed descriptive information about the surrogate itself. Such information may be provided as prose paragraphs, within which identifying information about particular surrogates may be presented using the standard TEI \hyperref[TEI.bibl]{<bibl>} element, as in the following example: \par\bgroup\index{surrogates=<surrogates>|exampleindex}\index{bibl=<bibl>|exampleindex}\index{title=<title>|exampleindex}\index{type=@type!<title>|exampleindex}\index{idno=<idno>|exampleindex}\index{bibl=<bibl>|exampleindex}\index{title=<title>|exampleindex}\index{type=@type!<title>|exampleindex}\index{idno=<idno>|exampleindex}\index{bibl=<bibl>|exampleindex}\index{title=<title>|exampleindex}\index{type=@type!<title>|exampleindex}\index{idno=<idno>|exampleindex}\index{date=<date>|exampleindex}\index{when=@when!<date>|exampleindex}\index{note=<note>|exampleindex}\index{bibl=<bibl>|exampleindex}\index{title=<title>|exampleindex}\index{type=@type!<title>|exampleindex}\index{idno=<idno>|exampleindex}\index{date=<date>|exampleindex}\index{when=@when!<date>|exampleindex}\index{note=<note>|exampleindex}\exampleFont \begin{shaded}\noindent\mbox{}{<\textbf{surrogates}>}\mbox{}\newline 
\hspace*{1em}{<\textbf{bibl}>}\mbox{}\newline 
\hspace*{1em}\hspace*{1em}{<\textbf{title}\hspace*{1em}{type}="{gmd}">}microfilm (master){</\textbf{title}>}\mbox{}\newline 
\hspace*{1em}\hspace*{1em}{<\textbf{idno}>}G.neg. 160{</\textbf{idno}>}\mbox{}\newline 
\hspace*{1em}\hspace*{1em} n.d.{</\textbf{bibl}>}\mbox{}\newline 
\hspace*{1em}{<\textbf{bibl}>}\mbox{}\newline 
\hspace*{1em}\hspace*{1em}{<\textbf{title}\hspace*{1em}{type}="{gmd}">}microfilm (archive){</\textbf{title}>}\mbox{}\newline 
\hspace*{1em}\hspace*{1em}{<\textbf{idno}>}G.pos. 186{</\textbf{idno}>}\mbox{}\newline 
\hspace*{1em}\hspace*{1em} n.d.{</\textbf{bibl}>}\mbox{}\newline 
\hspace*{1em}{<\textbf{bibl}>}\mbox{}\newline 
\hspace*{1em}\hspace*{1em}{<\textbf{title}\hspace*{1em}{type}="{gmd}">}b/w prints{</\textbf{title}>}\mbox{}\newline 
\hspace*{1em}\hspace*{1em}{<\textbf{idno}>}AM 795 4to{</\textbf{idno}>}\mbox{}\newline 
\hspace*{1em}\hspace*{1em}{<\textbf{date}\hspace*{1em}{when}="{1999-01-27}">}27 January 1999{</\textbf{date}>}\mbox{}\newline 
\hspace*{1em}\hspace*{1em}{<\textbf{note}>}copy of G.pos.\mbox{}\newline 
\hspace*{1em}\hspace*{1em}\hspace*{1em}\hspace*{1em} 186{</\textbf{note}>}\mbox{}\newline 
\hspace*{1em}{</\textbf{bibl}>}\mbox{}\newline 
\hspace*{1em}{<\textbf{bibl}>}\mbox{}\newline 
\hspace*{1em}\hspace*{1em}{<\textbf{title}\hspace*{1em}{type}="{gmd}">}b/w prints{</\textbf{title}>}\mbox{}\newline 
\hspace*{1em}\hspace*{1em}{<\textbf{idno}>}reg.nr. 75{</\textbf{idno}>}\mbox{}\newline 
\hspace*{1em}\hspace*{1em}{<\textbf{date}\hspace*{1em}{when}="{1999-01-25}">}25 January 1999{</\textbf{date}>}\mbox{}\newline 
\hspace*{1em}\hspace*{1em}{<\textbf{note}>}photographs of the spine, outside covers, stitching etc.{</\textbf{note}>}\mbox{}\newline 
\hspace*{1em}{</\textbf{bibl}>}\mbox{}\newline 
{</\textbf{surrogates}>}\end{shaded}\egroup\par \noindent  Note the use of the specialized form of title (\textit{general material designation}) to specify the kind of surrogate being documented.\par
At a later revision, the content of the \hyperref[TEI.surrogates]{<surrogates>} element is likely to be expanded to include elements more specifically intended to provide detailed information such as technical details of the process by which a digital or photographic image was made. For information about the inclusion of digital facsimile images within a TEI document, refer also to \textit{\hyperref[PHFAX]{11.1.\ Digital Facsimiles}}.
\subsection[{Manuscript Parts}]{Manuscript Parts}\label{mspt}\par
The \hyperref[TEI.msPart]{<msPart>} element may be used in cases where manuscripts or parts of manuscripts that were originally physically separate have been bound together and/or share the same call number. 
\begin{sansreflist}
  
\item [\textbf{<msPart>}] (manuscript part) contains information about an originally distinct manuscript or part of a manuscript, which is now part of a composite manuscript.
\end{sansreflist}
\par
Since each component of such a composite manuscript will in all likelihood have its own content, physical description, history, and so on, the structure of \hyperref[TEI.msPart]{<msPart>} is in the main identical to that of \hyperref[TEI.msDesc]{<msDesc>}, allowing one to retain the top level of identity (\hyperref[TEI.msIdentifier]{<msIdentifier>}), but to branch out thereafter into as many parts, or even subparts, as necessary.       \par\bgroup\index{msDesc=<msDesc>|exampleindex}\index{type=@type!<msDesc>|exampleindex}\index{msIdentifier=<msIdentifier>|exampleindex}\index{settlement=<settlement>|exampleindex}\index{key=@key!<settlement>|exampleindex}\index{repository=<repository>|exampleindex}\index{idno=<idno>|exampleindex}\index{msContents=<msContents>|exampleindex}\index{summary=<summary>|exampleindex}\index{textLang=<textLang>|exampleindex}\index{mainLang=@mainLang!<textLang>|exampleindex}\index{physDesc=<physDesc>|exampleindex}\index{objectDesc=<objectDesc>|exampleindex}\index{form=@form!<objectDesc>|exampleindex}\index{msPart=<msPart>|exampleindex}\index{msIdentifier=<msIdentifier>|exampleindex}\index{idno=<idno>|exampleindex}\index{msContents=<msContents>|exampleindex}\index{summary=<summary>|exampleindex}\index{textLang=<textLang>|exampleindex}\index{mainLang=@mainLang!<textLang>|exampleindex}\index{msPart=<msPart>|exampleindex}\index{msIdentifier=<msIdentifier>|exampleindex}\index{idno=<idno>|exampleindex}\index{msContents=<msContents>|exampleindex}\index{summary=<summary>|exampleindex}\index{textLang=<textLang>|exampleindex}\index{mainLang=@mainLang!<textLang>|exampleindex}\exampleFont \begin{shaded}\noindent\mbox{}{<\textbf{msDesc}\hspace*{1em}{xml:id}="{KBR\textunderscore ms\textunderscore 10066-77}"\mbox{}\newline 
\hspace*{1em}{xml:lang}="{en}"\hspace*{1em}{type}="{composite}">}\mbox{}\newline 
\hspace*{1em}{<\textbf{msIdentifier}>}\mbox{}\newline 
\hspace*{1em}\hspace*{1em}{<\textbf{settlement}\hspace*{1em}{key}="{tgn\textunderscore 7007868}">}Brussels{</\textbf{settlement}>}\mbox{}\newline 
\hspace*{1em}\hspace*{1em}{<\textbf{repository}>}Koninklijke Bibliotheek van België / Bibliothèque royale de\mbox{}\newline 
\hspace*{1em}\hspace*{1em}\hspace*{1em}\hspace*{1em} Belgique{</\textbf{repository}>}\mbox{}\newline 
\hspace*{1em}\hspace*{1em}{<\textbf{idno}>}ms. 10066-77{</\textbf{idno}>}\mbox{}\newline 
\hspace*{1em}{</\textbf{msIdentifier}>}\mbox{}\newline 
\hspace*{1em}{<\textbf{msContents}>}\mbox{}\newline 
\hspace*{1em}\hspace*{1em}{<\textbf{summary}\hspace*{1em}{xml:lang}="{la}">}Miscellany of various texts; Prudentius,\mbox{}\newline 
\hspace*{1em}\hspace*{1em}\hspace*{1em}\hspace*{1em} Psychomachia; Physiologus de natura animantium{</\textbf{summary}>}\mbox{}\newline 
\hspace*{1em}\hspace*{1em}{<\textbf{textLang}\hspace*{1em}{mainLang}="{la}">}Latin{</\textbf{textLang}>}\mbox{}\newline 
\hspace*{1em}{</\textbf{msContents}>}\mbox{}\newline 
\hspace*{1em}{<\textbf{physDesc}>}\mbox{}\newline 
\hspace*{1em}\hspace*{1em}{<\textbf{objectDesc}\hspace*{1em}{form}="{composite\textunderscore manuscript}"/>}\mbox{}\newline 
\hspace*{1em}{</\textbf{physDesc}>}\mbox{}\newline 
\hspace*{1em}{<\textbf{msPart}>}\mbox{}\newline 
\hspace*{1em}\hspace*{1em}{<\textbf{msIdentifier}>}\mbox{}\newline 
\hspace*{1em}\hspace*{1em}\hspace*{1em}{<\textbf{idno}>}ms. 10066-77 ff. 140r-156v{</\textbf{idno}>}\mbox{}\newline 
\hspace*{1em}\hspace*{1em}{</\textbf{msIdentifier}>}\mbox{}\newline 
\hspace*{1em}\hspace*{1em}{<\textbf{msContents}>}\mbox{}\newline 
\hspace*{1em}\hspace*{1em}\hspace*{1em}{<\textbf{summary}\hspace*{1em}{xml:lang}="{la}">}Physiologus{</\textbf{summary}>}\mbox{}\newline 
\hspace*{1em}\hspace*{1em}\hspace*{1em}{<\textbf{textLang}\hspace*{1em}{mainLang}="{la}">}Latin{</\textbf{textLang}>}\mbox{}\newline 
\hspace*{1em}\hspace*{1em}{</\textbf{msContents}>}\mbox{}\newline 
\hspace*{1em}{</\textbf{msPart}>}\mbox{}\newline 
\hspace*{1em}{<\textbf{msPart}>}\mbox{}\newline 
\hspace*{1em}\hspace*{1em}{<\textbf{msIdentifier}>}\mbox{}\newline 
\hspace*{1em}\hspace*{1em}\hspace*{1em}{<\textbf{idno}>}ms. 10066-77 ff. 112r-139r{</\textbf{idno}>}\mbox{}\newline 
\hspace*{1em}\hspace*{1em}{</\textbf{msIdentifier}>}\mbox{}\newline 
\hspace*{1em}\hspace*{1em}{<\textbf{msContents}>}\mbox{}\newline 
\hspace*{1em}\hspace*{1em}\hspace*{1em}{<\textbf{summary}\hspace*{1em}{xml:lang}="{la}">}Prudentius, Psychomachia{</\textbf{summary}>}\mbox{}\newline 
\hspace*{1em}\hspace*{1em}\hspace*{1em}{<\textbf{textLang}\hspace*{1em}{mainLang}="{la}">}Latin{</\textbf{textLang}>}\mbox{}\newline 
\hspace*{1em}\hspace*{1em}{</\textbf{msContents}>}\mbox{}\newline 
\hspace*{1em}{</\textbf{msPart}>}\mbox{}\newline 
{</\textbf{msDesc}>}\end{shaded}\egroup\par 
\subsection[{Manuscript Fragments}]{Manuscript Fragments}\label{msfg}\par
The \hyperref[TEI.msFrag]{<msFrag>} element may be used inside \hyperref[TEI.msDesc]{<msDesc>} when encoding one or more fragments of a scattered or fragmented manuscript. The fragment(s) described in a single \hyperref[TEI.msDesc]{<msDesc>} element may be held either at several institutions or at a single institution, so different call numbers may be attached to the fragments. Inside the \hyperref[TEI.msFrag]{<msFrag>} element, information about the single fragment or each dispersed part is provided: e.g. the current shelfmark or call number, the labels of the range of folios concerned if the fragment currently forms part of a larger manuscript, dimensions, extent, title, author, annotations, illuminations and so on. 
\begin{sansreflist}
  
\item [\textbf{<msFrag>}] (manuscript fragment) contains information about a fragment described in relation to a prior context, typically as a description of a virtual reconstruction of a manuscript or other object whose fragments were catalogued separately
\end{sansreflist}
\par
One well-known example of fragmentation is the Old Church Slavonic manuscript known as \textit{Codex Suprasliensis}, substantial parts of which are to be found in three separate repositories, in Ljubljana, Warsaw, and St. Petersburg. This manuscript should be represented using one single \hyperref[TEI.msDesc]{<msDesc>} element in which \hyperref[TEI.msName]{<msName>} is used to identify the fragmented manuscript, along with three distinct \hyperref[TEI.msFrag]{<msFrag>} elements, each of which contains the current identifier of one of the fragments, a physical description, and other related information: \par\bgroup\index{msDesc=<msDesc>|exampleindex}\index{msIdentifier=<msIdentifier>|exampleindex}\index{msName=<msName>|exampleindex}\index{msFrag=<msFrag>|exampleindex}\index{msIdentifier=<msIdentifier>|exampleindex}\index{settlement=<settlement>|exampleindex}\index{repository=<repository>|exampleindex}\index{idno=<idno>|exampleindex}\index{msContents=<msContents>|exampleindex}\index{summary=<summary>|exampleindex}\index{msFrag=<msFrag>|exampleindex}\index{msIdentifier=<msIdentifier>|exampleindex}\index{settlement=<settlement>|exampleindex}\index{repository=<repository>|exampleindex}\index{idno=<idno>|exampleindex}\index{msFrag=<msFrag>|exampleindex}\index{msIdentifier=<msIdentifier>|exampleindex}\index{settlement=<settlement>|exampleindex}\index{repository=<repository>|exampleindex}\index{idno=<idno>|exampleindex}\exampleFont \begin{shaded}\noindent\mbox{}{<\textbf{msDesc}>}\mbox{}\newline 
\hspace*{1em}{<\textbf{msIdentifier}>}\mbox{}\newline 
\hspace*{1em}\hspace*{1em}{<\textbf{msName}\hspace*{1em}{xml:lang}="{la}">}Codex Suprasliensis{</\textbf{msName}>}\mbox{}\newline 
\hspace*{1em}{</\textbf{msIdentifier}>}\mbox{}\newline 
\hspace*{1em}{<\textbf{msFrag}>}\mbox{}\newline 
\hspace*{1em}\hspace*{1em}{<\textbf{msIdentifier}>}\mbox{}\newline 
\hspace*{1em}\hspace*{1em}\hspace*{1em}{<\textbf{settlement}>}Ljubljana{</\textbf{settlement}>}\mbox{}\newline 
\hspace*{1em}\hspace*{1em}\hspace*{1em}{<\textbf{repository}>}Narodna in univerzitetna knjiznica{</\textbf{repository}>}\mbox{}\newline 
\hspace*{1em}\hspace*{1em}\hspace*{1em}{<\textbf{idno}>}MS Kopitar 2{</\textbf{idno}>}\mbox{}\newline 
\hspace*{1em}\hspace*{1em}{</\textbf{msIdentifier}>}\mbox{}\newline 
\hspace*{1em}\hspace*{1em}{<\textbf{msContents}>}\mbox{}\newline 
\hspace*{1em}\hspace*{1em}\hspace*{1em}{<\textbf{summary}>}Contains ff. 10 to 42 only{</\textbf{summary}>}\mbox{}\newline 
\hspace*{1em}\hspace*{1em}{</\textbf{msContents}>}\mbox{}\newline 
\hspace*{1em}{</\textbf{msFrag}>}\mbox{}\newline 
\hspace*{1em}{<\textbf{msFrag}>}\mbox{}\newline 
\hspace*{1em}\hspace*{1em}{<\textbf{msIdentifier}>}\mbox{}\newline 
\hspace*{1em}\hspace*{1em}\hspace*{1em}{<\textbf{settlement}>}Warszawa{</\textbf{settlement}>}\mbox{}\newline 
\hspace*{1em}\hspace*{1em}\hspace*{1em}{<\textbf{repository}>}Biblioteka Narodowa{</\textbf{repository}>}\mbox{}\newline 
\hspace*{1em}\hspace*{1em}\hspace*{1em}{<\textbf{idno}>}BO 3.201{</\textbf{idno}>}\mbox{}\newline 
\hspace*{1em}\hspace*{1em}{</\textbf{msIdentifier}>}\mbox{}\newline 
\hspace*{1em}{</\textbf{msFrag}>}\mbox{}\newline 
\hspace*{1em}{<\textbf{msFrag}>}\mbox{}\newline 
\hspace*{1em}\hspace*{1em}{<\textbf{msIdentifier}>}\mbox{}\newline 
\hspace*{1em}\hspace*{1em}\hspace*{1em}{<\textbf{settlement}>}Sankt-Peterburg{</\textbf{settlement}>}\mbox{}\newline 
\hspace*{1em}\hspace*{1em}\hspace*{1em}{<\textbf{repository}>}Rossiiskaia natsional'naia biblioteka{</\textbf{repository}>}\mbox{}\newline 
\hspace*{1em}\hspace*{1em}\hspace*{1em}{<\textbf{idno}>}Q.p.I.72{</\textbf{idno}>}\mbox{}\newline 
\hspace*{1em}\hspace*{1em}{</\textbf{msIdentifier}>}\mbox{}\newline 
\hspace*{1em}{</\textbf{msFrag}>}\mbox{}\newline 
{</\textbf{msDesc}>}\end{shaded}\egroup\par 
\subsection[{Module for Manuscript Description}]{Module for Manuscript Description}\label{MSref}\par
The module described in this chapter makes available the following components: \begin{description}

\item[{Module msdescription: Manuscript Description}]\hspace{1em}\hfill\linebreak
\mbox{}\\[-10pt] \begin{itemize}
\item {\itshape Elements defined}: \hyperref[TEI.accMat]{accMat} \hyperref[TEI.acquisition]{acquisition} \hyperref[TEI.additional]{additional} \hyperref[TEI.additions]{additions} \hyperref[TEI.adminInfo]{adminInfo} \hyperref[TEI.altIdentifier]{altIdentifier} \hyperref[TEI.binding]{binding} \hyperref[TEI.bindingDesc]{bindingDesc} \hyperref[TEI.catchwords]{catchwords} \hyperref[TEI.collation]{collation} \hyperref[TEI.collection]{collection} \hyperref[TEI.colophon]{colophon} \hyperref[TEI.condition]{condition} \hyperref[TEI.custEvent]{custEvent} \hyperref[TEI.custodialHist]{custodialHist} \hyperref[TEI.decoDesc]{decoDesc} \hyperref[TEI.decoNote]{decoNote} \hyperref[TEI.depth]{depth} \hyperref[TEI.dim]{dim} \hyperref[TEI.dimensions]{dimensions} \hyperref[TEI.explicit]{explicit} \hyperref[TEI.filiation]{filiation} \hyperref[TEI.finalRubric]{finalRubric} \hyperref[TEI.foliation]{foliation} \hyperref[TEI.handDesc]{handDesc} \hyperref[TEI.height]{height} \hyperref[TEI.heraldry]{heraldry} \hyperref[TEI.history]{history} \hyperref[TEI.incipit]{incipit} \hyperref[TEI.institution]{institution} \hyperref[TEI.layout]{layout} \hyperref[TEI.layoutDesc]{layoutDesc} \hyperref[TEI.locus]{locus} \hyperref[TEI.locusGrp]{locusGrp} \hyperref[TEI.material]{material} \hyperref[TEI.msContents]{msContents} \hyperref[TEI.msDesc]{msDesc} \hyperref[TEI.msFrag]{msFrag} \hyperref[TEI.msIdentifier]{msIdentifier} \hyperref[TEI.msItem]{msItem} \hyperref[TEI.msItemStruct]{msItemStruct} \hyperref[TEI.msName]{msName} \hyperref[TEI.msPart]{msPart} \hyperref[TEI.musicNotation]{musicNotation} \hyperref[TEI.objectDesc]{objectDesc} \hyperref[TEI.objectType]{objectType} \hyperref[TEI.origDate]{origDate} \hyperref[TEI.origPlace]{origPlace} \hyperref[TEI.origin]{origin} \hyperref[TEI.physDesc]{physDesc} \hyperref[TEI.provenance]{provenance} \hyperref[TEI.recordHist]{recordHist} \hyperref[TEI.repository]{repository} \hyperref[TEI.rubric]{rubric} \hyperref[TEI.scriptDesc]{scriptDesc} \hyperref[TEI.seal]{seal} \hyperref[TEI.sealDesc]{sealDesc} \hyperref[TEI.secFol]{secFol} \hyperref[TEI.signatures]{signatures} \hyperref[TEI.source]{source} \hyperref[TEI.stamp]{stamp} \hyperref[TEI.summary]{summary} \hyperref[TEI.support]{support} \hyperref[TEI.supportDesc]{supportDesc} \hyperref[TEI.surrogates]{surrogates} \hyperref[TEI.typeDesc]{typeDesc} \hyperref[TEI.typeNote]{typeNote} \hyperref[TEI.watermark]{watermark} \hyperref[TEI.width]{width}
\item {\itshape Classes defined}: \hyperref[TEI.att.msClass]{att.msClass} \hyperref[TEI.att.msExcerpt]{att.msExcerpt} \hyperref[TEI.model.physDescPart]{model.physDescPart}
\end{itemize} 
\end{description}  The selection and combination of modules to form a TEI schema is described in \textit{\hyperref[STIN]{1.2.\ Defining a TEI Schema}}.