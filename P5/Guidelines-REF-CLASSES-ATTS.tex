
\section[{Attribute Classes}]{Attribute Classes}\label{REF-CLASSES-ATTS}
\subsection[{About the Attribute Classes Appendix}]{About the Attribute Classes Appendix}\par
This appendix gives you a list of attribute classes and links to the reference pages for them. There are 80 distinctly-named attribute classes in revision \xref{https://github.com/TEIC/TEI/commit/27522dec1}{27522dec1} of TEI P5 \hyperref[ABTEI4]{Version} \xref{../../readme-4.3.0.html}{4.3.0a} of the TEI Guidelines.
\begin{reflist}
\item[]\begin{specHead}{TEI.att.anchoring}{att.anchoring} (anchoring) provides attributes for use on annotations, e.g. notes and groups of notes describing the existence and position of an anchor for annotations.\end{specHead} 
    \item[{Module}]
  tei — \hyperref[ST]{The TEI Infrastructure}
    \item[{Members}]
  \hyperref[TEI.note]{note} \hyperref[TEI.noteGrp]{noteGrp}
    \item[{Attributes}]
  Attributes\hfil\\[-10pt]\begin{sansreflist}
    \item[@anchored]
  (anchored) indicates whether the copy text shows the exact place of reference for the note.
\begin{reflist}
    \item[{Status}]
  Optional
    \item[{Datatype}]
  \hyperref[TEI.teidata.truthValue]{teidata.truthValue}
    \item[{Default}]
  true
    \item[{Note}]
  \par
In modern texts, notes are usually anchored by means of explicit footnote or endnote symbols. An explicit indication of the phrase or line annotated may however be used instead (e.g. ‘page 218, lines 3–4’). The {\itshape anchored} attribute indicates whether any explicit location is given, whether by symbol or by prose cross-reference. The value true indicates that such an explicit location is indicated in the copy text; the value false indicates that the copy text does not indicate a specific place of attachment for the note. If the specific symbols used in the copy text at the location the note is anchored are to be recorded, use the {\itshape n} attribute.
\end{reflist}  
    \item[@targetEnd]
  (target end) points to the end of the span to which the note is attached, if the note is not embedded in the text at that point.
\begin{reflist}
    \item[{Status}]
  Optional
    \item[{Datatype}]
  1–∞ occurrences of \hyperref[TEI.teidata.pointer]{teidata.pointer} separated by whitespace
    \item[{Note}]
  \par
This attribute is retained for backwards compatibility; it may be removed at a subsequent release of the Guidelines. The recommended way of pointing to a span of elements is by means of the \textsf{range} function of XPointer, as further described in \textit{\hyperref[SATSRN]{16.2.4.6.\ range()}}.
\end{reflist}  
\end{sansreflist}  
    \item[{Example}]
  \leavevmode\bgroup\index{p=<p>|exampleindex}\index{anchor=<anchor>|exampleindex}\index{noteGrp=<noteGrp>|exampleindex}\index{targetEnd=@targetEnd!<noteGrp>|exampleindex}\index{note=<note>|exampleindex}\index{note=<note>|exampleindex}\exampleFont \begin{shaded}\noindent\mbox{}{<\textbf{p}>}(...) tamen reuerendos dominos archiepiscopum et canonicos Leopolienses\mbox{}\newline 
 necnon episcopum in duplicibus Quatuortemporibus{<\textbf{anchor}\hspace*{1em}{xml:id}="{A55234}"/>} totaliter expediui...{</\textbf{p}>}\mbox{}\newline 
\textit{<!-- elsewhere in the document -->}\mbox{}\newline 
{<\textbf{noteGrp}\hspace*{1em}{targetEnd}="{\#A55234}">}\mbox{}\newline 
\hspace*{1em}{<\textbf{note}\hspace*{1em}{xml:lang}="{en}">} Quatuor Tempora, so called dry fast days.\mbox{}\newline 
\hspace*{1em}{</\textbf{note}>}\mbox{}\newline 
\hspace*{1em}{<\textbf{note}\hspace*{1em}{xml:lang}="{pl}">} Quatuor Tempora, tzw. Suche dni postne.\mbox{}\newline 
\hspace*{1em}{</\textbf{note}>}\mbox{}\newline 
{</\textbf{noteGrp}>}\end{shaded}\egroup 


\end{reflist}  
\begin{reflist}
\item[]\begin{specHead}{TEI.att.ascribed}{att.ascribed} provides attributes for elements representing speech or action that can be ascribed to a specific individual. [\textit{\hyperref[COHQQ]{3.3.3.\ Quotation}} \textit{\hyperref[TSBA]{8.3.\ Elements Unique to Spoken Texts}}]\end{specHead} 
    \item[{Module}]
  tei — \hyperref[ST]{The TEI Infrastructure}
    \item[{Members}]
  \hyperref[TEI.att.ascribed.directed]{att.ascribed.directed}[\hyperref[TEI.kinesic]{kinesic} \hyperref[TEI.move]{move} \hyperref[TEI.pause]{pause} \hyperref[TEI.q]{q} \hyperref[TEI.said]{said} \hyperref[TEI.sp]{sp} \hyperref[TEI.spGrp]{spGrp} \hyperref[TEI.stage]{stage} \hyperref[TEI.u]{u} \hyperref[TEI.vocal]{vocal} \hyperref[TEI.writing]{writing}] \hyperref[TEI.annotationBlock]{annotationBlock} \hyperref[TEI.change]{change} \hyperref[TEI.incident]{incident} \hyperref[TEI.setting]{setting} \hyperref[TEI.shift]{shift}
    \item[{Attributes}]
  Attributes\hfil\\[-10pt]\begin{sansreflist}
    \item[@who]
  indicates the person, or group of people, to whom the element content is ascribed.
\begin{reflist}
    \item[{Status}]
  Optional
    \item[{Datatype}]
  1–∞ occurrences of \hyperref[TEI.teidata.pointer]{teidata.pointer} separated by whitespace
    \item[]In the following example from Hamlet, speeches (\hyperref[TEI.sp]{<sp>}) in the body of the play are linked to \hyperref[TEI.castItem]{<castItem>} elements in the \hyperref[TEI.castList]{<castList>} using the {\itshape who} attribute.\index{castItem=<castItem>|exampleindex}\index{type=@type!<castItem>|exampleindex}\index{role=<role>|exampleindex}\index{castItem=<castItem>|exampleindex}\index{type=@type!<castItem>|exampleindex}\index{role=<role>|exampleindex}\index{roleDesc=<roleDesc>|exampleindex}\index{sp=<sp>|exampleindex}\index{who=@who!<sp>|exampleindex}\index{speaker=<speaker>|exampleindex}\index{l=<l>|exampleindex}\index{n=@n!<l>|exampleindex}\index{sp=<sp>|exampleindex}\index{who=@who!<sp>|exampleindex}\index{speaker=<speaker>|exampleindex}\index{l=<l>|exampleindex}\index{n=@n!<l>|exampleindex}\exampleFont {<\textbf{castItem}\hspace*{1em}{type}="{role}">}\mbox{}\newline 
\hspace*{1em}{<\textbf{role}\hspace*{1em}{xml:id}="{Barnardo}">}Bernardo{</\textbf{role}>}\mbox{}\newline 
{</\textbf{castItem}>}\mbox{}\newline 
{<\textbf{castItem}\hspace*{1em}{type}="{role}">}\mbox{}\newline 
\hspace*{1em}{<\textbf{role}\hspace*{1em}{xml:id}="{Francisco}">}Francisco{</\textbf{role}>}\mbox{}\newline 
\hspace*{1em}{<\textbf{roleDesc}>}a soldier{</\textbf{roleDesc}>}\mbox{}\newline 
{</\textbf{castItem}>}\mbox{}\newline 
\textit{<!-- ... -->}\mbox{}\newline 
{<\textbf{sp}\hspace*{1em}{who}="{\#Barnardo}">}\mbox{}\newline 
\hspace*{1em}{<\textbf{speaker}>}Bernardo{</\textbf{speaker}>}\mbox{}\newline 
\hspace*{1em}{<\textbf{l}\hspace*{1em}{n}="{1}">}Who's there?{</\textbf{l}>}\mbox{}\newline 
{</\textbf{sp}>}\mbox{}\newline 
{<\textbf{sp}\hspace*{1em}{who}="{\#Francisco}">}\mbox{}\newline 
\hspace*{1em}{<\textbf{speaker}>}Francisco{</\textbf{speaker}>}\mbox{}\newline 
\hspace*{1em}{<\textbf{l}\hspace*{1em}{n}="{2}">}Nay, answer me: stand, and unfold yourself.{</\textbf{l}>}\mbox{}\newline 
{</\textbf{sp}>}
    \item[{Note}]
  \par
For transcribed speech, this will typically identify a participant or participant group; in other contexts, it will point to any identified \hyperref[TEI.person]{<person>} element.
\end{reflist}  
\end{sansreflist}  
\end{reflist}  
\begin{reflist}
\item[]\begin{specHead}{TEI.att.ascribed.directed}{att.ascribed.directed} provides attributes for elements representing speech or action that can be directed at a group or individual. [\textit{\hyperref[COHQQ]{3.3.3.\ Quotation}} \textit{\hyperref[TSBA]{8.3.\ Elements Unique to Spoken Texts}}]\end{specHead} 
    \item[{Module}]
  tei — \hyperref[ST]{The TEI Infrastructure}
    \item[{Members}]
  \hyperref[TEI.kinesic]{kinesic} \hyperref[TEI.move]{move} \hyperref[TEI.pause]{pause} \hyperref[TEI.q]{q} \hyperref[TEI.said]{said} \hyperref[TEI.sp]{sp} \hyperref[TEI.spGrp]{spGrp} \hyperref[TEI.stage]{stage} \hyperref[TEI.u]{u} \hyperref[TEI.vocal]{vocal} \hyperref[TEI.writing]{writing}
    \item[{Attributes}]
  \hyperref[TEI.att.ascribed]{att.ascribed} (\textit{@who}) \hfil\\[-10pt]\begin{sansreflist}
    \item[@toWhom]
  indicates the person, or group of people, to whom a speech act or action is directed.
\begin{reflist}
    \item[{Status}]
  Optional
    \item[{Datatype}]
  1–∞ occurrences of \hyperref[TEI.teidata.pointer]{teidata.pointer} separated by whitespace
    \item[]In the following example from Mary Pix's The False Friend, speeches (\hyperref[TEI.sp]{<sp>}) in the body of the play are linked to \hyperref[TEI.castItem]{<castItem>} elements in the \hyperref[TEI.castList]{<castList>} using the {\itshape toWhom} attribute, which is used to specify who the speech is directed to. Additionally, the \hyperref[TEI.stage]{<stage>} includes {\itshape toWhom} to indicate the directionality of the action.\index{castItem=<castItem>|exampleindex}\index{type=@type!<castItem>|exampleindex}\index{role=<role>|exampleindex}\index{castItem=<castItem>|exampleindex}\index{type=@type!<castItem>|exampleindex}\index{role=<role>|exampleindex}\index{castItem=<castItem>|exampleindex}\index{type=@type!<castItem>|exampleindex}\index{role=<role>|exampleindex}\index{sp=<sp>|exampleindex}\index{who=@who!<sp>|exampleindex}\index{toWhom=@toWhom!<sp>|exampleindex}\index{speaker=<speaker>|exampleindex}\index{l=<l>|exampleindex}\index{n=@n!<l>|exampleindex}\index{sp=<sp>|exampleindex}\index{who=@who!<sp>|exampleindex}\index{toWhom=@toWhom!<sp>|exampleindex}\index{speaker=<speaker>|exampleindex}\index{l=<l>|exampleindex}\index{n=@n!<l>|exampleindex}\index{stage=<stage>|exampleindex}\index{who=@who!<stage>|exampleindex}\index{toWhom=@toWhom!<stage>|exampleindex}\index{l=<l>|exampleindex}\exampleFont {<\textbf{castItem}\hspace*{1em}{type}="{role}">}\mbox{}\newline 
\hspace*{1em}{<\textbf{role}\hspace*{1em}{xml:id}="{emil}">}Emilius.{</\textbf{role}>}\mbox{}\newline 
{</\textbf{castItem}>}\mbox{}\newline 
{<\textbf{castItem}\hspace*{1em}{type}="{role}">}\mbox{}\newline 
\hspace*{1em}{<\textbf{role}\hspace*{1em}{xml:id}="{lov}">}Lovisa{</\textbf{role}>}\mbox{}\newline 
{</\textbf{castItem}>}\mbox{}\newline 
{<\textbf{castItem}\hspace*{1em}{type}="{role}">}\mbox{}\newline 
\hspace*{1em}{<\textbf{role}\hspace*{1em}{xml:id}="{serv}">}A servant{</\textbf{role}>}\mbox{}\newline 
{</\textbf{castItem}>}\mbox{}\newline 
\textit{<!-- ... -->}\mbox{}\newline 
{<\textbf{sp}\hspace*{1em}{who}="{\#emil}"\mbox{}\newline 
\hspace*{1em}{toWhom}="{\#lov}">}\mbox{}\newline 
\hspace*{1em}{<\textbf{speaker}>}Emil.{</\textbf{speaker}>}\mbox{}\newline 
\hspace*{1em}{<\textbf{l}\hspace*{1em}{n}="{1}">}My love!{</\textbf{l}>}\mbox{}\newline 
{</\textbf{sp}>}\mbox{}\newline 
{<\textbf{sp}\hspace*{1em}{who}="{\#lov}"\mbox{}\newline 
\hspace*{1em}{toWhom}="{\#emil}">}\mbox{}\newline 
\hspace*{1em}{<\textbf{speaker}>}Lov.{</\textbf{speaker}>}\mbox{}\newline 
\hspace*{1em}{<\textbf{l}\hspace*{1em}{n}="{2}">}I have no Witness of my Noble Birth{</\textbf{l}>}\mbox{}\newline 
\hspace*{1em}{<\textbf{stage}\hspace*{1em}{who}="{emil}"\mbox{}\newline 
\hspace*{1em}\hspace*{1em}{toWhom}="{\#serv}">}Pointing to her Woman.{</\textbf{stage}>}\mbox{}\newline 
\hspace*{1em}{<\textbf{l}>}But that poor helpless wretch——{</\textbf{l}>}\mbox{}\newline 
{</\textbf{sp}>}
    \item[{Note}]
  \par
To indicate the recipient of written correspondence, use the elements used in section \textit{\hyperref[HD44CD]{2.4.6.\ Correspondence Description}}, rather than a {\itshape toWhom} attribute.
\end{reflist}  
\end{sansreflist}  
\end{reflist}  
\begin{reflist}
\item[]\begin{specHead}{TEI.att.breaking}{att.breaking} provides an attribute to indicate whether or not the element concerned is considered to mark the end of an orthographic token in the same way as whitespace. [\textit{\hyperref[CORS5]{3.11.3.\ Milestone Elements}}]\end{specHead} 
    \item[{Module}]
  tei — \hyperref[ST]{The TEI Infrastructure}
    \item[{Members}]
  \hyperref[TEI.cb]{cb} \hyperref[TEI.gb]{gb} \hyperref[TEI.lb]{lb} \hyperref[TEI.milestone]{milestone} \hyperref[TEI.pb]{pb}
    \item[{Attributes}]
  Attributes\hfil\\[-10pt]\begin{sansreflist}
    \item[@break]
  indicates whether or not the element bearing this attribute should be considered to mark the end of an orthographic token in the same way as whitespace.
\begin{reflist}
    \item[{Status}]
  Recommended
    \item[{Datatype}]
  \hyperref[TEI.teidata.enumerated]{teidata.enumerated}
    \item[{Sample values include}]
  \begin{description}

\item[{yes}]the element bearing this attribute is considered to mark the end of any adjacent orthographic token irrespective of the presence of any adjacent whitespace
\item[{no}]the element bearing this attribute is considered not to mark the end of any adjacent orthographic token irrespective of the presence of any adjacent whitespace
\item[{maybe}]the encoding does not take any position on this issue.
\end{description} 
    \item[]In the following lines from the ‘Dream of the Rood’, linebreaks occur in the middle of the words \textit{lāðost} and \textit{reord-berendum}.\index{ab=<ab>|exampleindex}\index{lb=<lb>|exampleindex}\index{break=@break!<lb>|exampleindex}\index{lb=<lb>|exampleindex}\index{break=@break!<lb>|exampleindex}\exampleFont {<\textbf{ab}>} ...eƿesa tome iu icƿæs ȝeƿorden ƿita heardoſt .\mbox{}\newline 
 leodum la{<\textbf{lb}\hspace*{1em}{break}="{no}"/>} ðost ærþan ichim lifes\mbox{}\newline 
 ƿeȝ rihtne ȝerymde reord be{<\textbf{lb}\hspace*{1em}{break}="{no}"/>}\mbox{}\newline 
 rendum hƿæt me þaȝeƿeorðode ƿuldres ealdor ofer...\mbox{}\newline 
{</\textbf{ab}>}
\end{reflist}  
\end{sansreflist}  
\end{reflist}  
\begin{reflist}
\item[]\begin{specHead}{TEI.att.canonical}{att.canonical} provides attributes which can be used to associate a representation such as a name or title with canonical information about the object being named or referenced. [\textit{\hyperref[NDATTSnr]{13.1.1.\ Linking Names and Their Referents}}]\end{specHead} 
    \item[{Module}]
  tei — \hyperref[ST]{The TEI Infrastructure}
    \item[{Members}]
  \hyperref[TEI.att.naming]{att.naming}[\hyperref[TEI.att.personal]{att.personal}[\hyperref[TEI.addName]{addName} \hyperref[TEI.forename]{forename} \hyperref[TEI.genName]{genName} \hyperref[TEI.name]{name} \hyperref[TEI.objectName]{objectName} \hyperref[TEI.orgName]{orgName} \hyperref[TEI.persName]{persName} \hyperref[TEI.placeName]{placeName} \hyperref[TEI.roleName]{roleName} \hyperref[TEI.surname]{surname}] \hyperref[TEI.affiliation]{affiliation} \hyperref[TEI.author]{author} \hyperref[TEI.birth]{birth} \hyperref[TEI.bloc]{bloc} \hyperref[TEI.climate]{climate} \hyperref[TEI.collection]{collection} \hyperref[TEI.country]{country} \hyperref[TEI.death]{death} \hyperref[TEI.district]{district} \hyperref[TEI.editor]{editor} \hyperref[TEI.education]{education} \hyperref[TEI.event]{event} \hyperref[TEI.geogFeat]{geogFeat} \hyperref[TEI.geogName]{geogName} \hyperref[TEI.institution]{institution} \hyperref[TEI.nationality]{nationality} \hyperref[TEI.occupation]{occupation} \hyperref[TEI.offset]{offset} \hyperref[TEI.origPlace]{origPlace} \hyperref[TEI.population]{population} \hyperref[TEI.pubPlace]{pubPlace} \hyperref[TEI.region]{region} \hyperref[TEI.repository]{repository} \hyperref[TEI.residence]{residence} \hyperref[TEI.rs]{rs} \hyperref[TEI.settlement]{settlement} \hyperref[TEI.socecStatus]{socecStatus} \hyperref[TEI.state]{state} \hyperref[TEI.terrain]{terrain} \hyperref[TEI.trait]{trait}] \hyperref[TEI.actor]{actor} \hyperref[TEI.authority]{authority} \hyperref[TEI.catDesc]{catDesc} \hyperref[TEI.correspDesc]{correspDesc} \hyperref[TEI.date]{date} \hyperref[TEI.distributor]{distributor} \hyperref[TEI.docAuthor]{docAuthor} \hyperref[TEI.docTitle]{docTitle} \hyperref[TEI.faith]{faith} \hyperref[TEI.funder]{funder} \hyperref[TEI.material]{material} \hyperref[TEI.meeting]{meeting} \hyperref[TEI.object]{object} \hyperref[TEI.objectType]{objectType} \hyperref[TEI.principal]{principal} \hyperref[TEI.publisher]{publisher} \hyperref[TEI.relation]{relation} \hyperref[TEI.resp]{resp} \hyperref[TEI.respStmt]{respStmt} \hyperref[TEI.sponsor]{sponsor} \hyperref[TEI.term]{term} \hyperref[TEI.time]{time} \hyperref[TEI.title]{title} \hyperref[TEI.unitDecl]{unitDecl} \hyperref[TEI.unitDef]{unitDef}
    \item[{Attributes}]
  Attributes\hfil\\[-10pt]\begin{sansreflist}
    \item[@key]
  provides an externally-defined means of identifying the entity (or entities) being named, using a coded value of some kind.
\begin{reflist}
    \item[{Status}]
  Optional
    \item[{Datatype}]
  \hyperref[TEI.teidata.text]{teidata.text}
    \item[]\index{author=<author>|exampleindex}\index{name=<name>|exampleindex}\index{key=@key!<name>|exampleindex}\index{type=@type!<name>|exampleindex}\exampleFont {<\textbf{author}>}\mbox{}\newline 
\hspace*{1em}{<\textbf{name}\hspace*{1em}{key}="{name 427308}"\mbox{}\newline 
\hspace*{1em}\hspace*{1em}{type}="{organisation}">}[New Zealand Parliament, Legislative Council]{</\textbf{name}>}\mbox{}\newline 
{</\textbf{author}>}
    \item[]\index{author=<author>|exampleindex}\index{name=<name>|exampleindex}\index{key=@key!<name>|exampleindex}\index{ref=@ref!<name>|exampleindex}\exampleFont {<\textbf{author}>}\mbox{}\newline 
\hspace*{1em}{<\textbf{name}\hspace*{1em}{key}="{Hugo, Victor (1802-1885)}"\mbox{}\newline 
\hspace*{1em}\hspace*{1em}{ref}="{http://www.idref.fr/026927608}">}Victor Hugo{</\textbf{name}>}\mbox{}\newline 
{</\textbf{author}>}
    \item[{Note}]
  \par
The value may be a unique identifier from a database, or any other externally-defined string identifying the referent. \par
No particular syntax is proposed for the values of the {\itshape key} attribute, since its form will depend entirely on practice within a given project. For the same reason, this attribute is not recommended in data interchange, since there is no way of ensuring that the values used by one project are distinct from those used by another. In such a situation, a preferable approach for magic tokens which follows standard practice on the Web is to use a {\itshape ref} attribute whose value is a tag URI as defined in \hyperref[RFC4151]{RFC 4151}.
\end{reflist}  
    \item[@ref]
  (reference) provides an explicit means of locating a full definition or identity for the entity being named by means of one or more URIs.
\begin{reflist}
    \item[{Status}]
  Optional
    \item[{Datatype}]
  1–∞ occurrences of \hyperref[TEI.teidata.pointer]{teidata.pointer} separated by whitespace
    \item[]\index{name=<name>|exampleindex}\index{ref=@ref!<name>|exampleindex}\index{type=@type!<name>|exampleindex}\exampleFont {<\textbf{name}\hspace*{1em}{ref}="{http://viaf.org/viaf/109557338}"\mbox{}\newline 
\hspace*{1em}{type}="{person}">}Seamus Heaney{</\textbf{name}>}
    \item[{Note}]
  \par
The value must point directly to one or more XML elements or other resources by means of one or more URIs, separated by whitespace. If more than one is supplied the implication is that the name identifies several distinct entities.
\end{reflist}  
\end{sansreflist}  
\end{reflist}  
\begin{reflist}
\item[]\begin{specHead}{TEI.att.citeStructurePart}{att.citeStructurePart} provides attributes for selecting particular elements within a document.\end{specHead} 
    \item[{Module}]
  tei — \hyperref[ST]{The TEI Infrastructure}
    \item[{Members}]
  \hyperref[TEI.citeData]{citeData} \hyperref[TEI.citeStructure]{citeStructure}
    \item[{Attributes}]
  Attributes\hfil\\[-10pt]\begin{sansreflist}
    \item[@use]
  (use) supplies an XPath selection pattern using the syntax defined in \cite{XSLT3}. The XPath pattern is relative to the context given in {\itshape match}, which will either be a sibling attribute in the case of <citeStructure> or on the parent <citeStructure> in the case of <citeData>.
\begin{reflist}
    \item[{Status}]
  Required
    \item[{Datatype}]
  \hyperref[TEI.teidata.xpath]{teidata.xpath}
\end{reflist}  
\end{sansreflist}  
\end{reflist}  
\begin{reflist}
\item[]\begin{specHead}{TEI.att.citing}{att.citing} provides attributes for specifying the specific part of a bibliographic item being cited. [\textit{\hyperref[STECAT]{1.3.1.\ Attribute Classes}}]\end{specHead} 
    \item[{Module}]
  tei — \hyperref[ST]{The TEI Infrastructure}
    \item[{Members}]
  \hyperref[TEI.biblScope]{biblScope} \hyperref[TEI.citedRange]{citedRange}
    \item[{Attributes}]
  Attributes\hfil\\[-10pt]\begin{sansreflist}
    \item[@unit]
  identifies the unit of information conveyed by the element, e.g. columns, pages, volume, entry.
\begin{reflist}
    \item[{Status}]
  Optional
    \item[{Datatype}]
  \hyperref[TEI.teidata.enumerated]{teidata.enumerated}
    \item[{Suggested values include:}]
  \begin{description}

\item[{volume}](volume) the element contains a volume number.
\item[{issue}]the element contains an issue number, or volume and issue numbers.
\item[{page}](page) the element contains a page number or page range.
\item[{line}]the element contains a line number or line range.
\item[{chapter}](chapter) the element contains a chapter indication (number and/or title)
\item[{part}]the element identifies a part of a book or collection.
\item[{column}]the element identifies a column.
\item[{entry}]the element identifies an entry number or label in a list of entries.
\end{description} 
\end{reflist}  
    \item[@from]
  specifies the starting point of the range of units indicated by the {\itshape unit} attribute.
\begin{reflist}
    \item[{Status}]
  Optional
    \item[{Datatype}]
  \hyperref[TEI.teidata.word]{teidata.word}
\end{reflist}  
    \item[@to]
  specifies the end-point of the range of units indicated by the {\itshape unit} attribute.
\begin{reflist}
    \item[{Status}]
  Optional
    \item[{Datatype}]
  \hyperref[TEI.teidata.word]{teidata.word}
\end{reflist}  
\end{sansreflist}  
\end{reflist}  
\begin{reflist}
\item[]\begin{specHead}{TEI.att.combinable}{att.combinable} provides attributes indicating how multiple references to the same object in a schema should be combined\end{specHead} 
    \item[{Module}]
  tagdocs — \hyperref[TD]{Documentation Elements}
    \item[{Members}]
  \hyperref[TEI.att.identified]{att.identified}[\hyperref[TEI.attDef]{attDef} \hyperref[TEI.classSpec]{classSpec} \hyperref[TEI.constraintSpec]{constraintSpec} \hyperref[TEI.dataSpec]{dataSpec} \hyperref[TEI.elementSpec]{elementSpec} \hyperref[TEI.macroSpec]{macroSpec} \hyperref[TEI.moduleSpec]{moduleSpec} \hyperref[TEI.paramSpec]{paramSpec} \hyperref[TEI.schemaSpec]{schemaSpec}] \hyperref[TEI.defaultVal]{defaultVal} \hyperref[TEI.remarks]{remarks} \hyperref[TEI.valDesc]{valDesc} \hyperref[TEI.valItem]{valItem} \hyperref[TEI.valList]{valList}
    \item[{Attributes}]
  \hyperref[TEI.att.deprecated]{att.deprecated} (\textit{@validUntil}) \hfil\\[-10pt]\begin{sansreflist}
    \item[@mode]
  specifies the effect of this declaration on its parent object.
\begin{reflist}
    \item[{Status}]
  Optional
    \item[{Datatype}]
  \hyperref[TEI.teidata.enumerated]{teidata.enumerated}
    \item[{Legal values are:}]
  \begin{description}

\item[{add}]this declaration is added to the current definitions{[Default] }
\item[{delete}]if present already, the whole of the declaration for this object is removed from the current setup
\item[{change}]this declaration changes the declaration of the same name in the current definition
\item[{replace}]this declaration replaces the declaration of the same name in the current definition
\end{description} 
\end{reflist}  
\end{sansreflist}  
    \item[{Note}]
  \par
An ODD processor should handle the values for {\itshape mode} as follows: \begin{description}

\item[{add}]the object should be created (processing any children in add mode); raise an error if an object with the same identifier already exists
\item[{replace}]use this object in preference to any existing object with the same identifier, and ignore any children of that object; process any new children in replace mode
\item[{delete}]do not process this object or any existing object with the same identifier; raise an error if any new children supplied 
\item[{change}]process this object, and process its children, and those of any existing object with the same identifier, in change mode
\end{description} 
\end{reflist}  
\begin{reflist}
\item[]\begin{specHead}{TEI.att.coordinated}{att.coordinated} provides attributes which can be used to position their parent element within a two dimensional coordinate system.\end{specHead} 
    \item[{Module}]
  transcr — \hyperref[PH]{Representation of Primary Sources}
    \item[{Members}]
  \hyperref[TEI.line]{line} \hyperref[TEI.path]{path} \hyperref[TEI.surface]{surface} \hyperref[TEI.zone]{zone}
    \item[{Attributes}]
  Attributes\hfil\\[-10pt]\begin{sansreflist}
    \item[@start]
  indicates the element within a transcription of the text containing at least the start of the writing represented by this zone or surface.
\begin{reflist}
    \item[{Status}]
  Optional
    \item[{Datatype}]
  \hyperref[TEI.teidata.pointer]{teidata.pointer}
\end{reflist}  
    \item[@ulx]
  gives the x coordinate value for the upper left corner of a rectangular space.
\begin{reflist}
    \item[{Status}]
  Optional
    \item[{Datatype}]
  \hyperref[TEI.teidata.numeric]{teidata.numeric}
\end{reflist}  
    \item[@uly]
  gives the y coordinate value for the upper left corner of a rectangular space.
\begin{reflist}
    \item[{Status}]
  Optional
    \item[{Datatype}]
  \hyperref[TEI.teidata.numeric]{teidata.numeric}
\end{reflist}  
    \item[@lrx]
  gives the x coordinate value for the lower right corner of a rectangular space.
\begin{reflist}
    \item[{Status}]
  Optional
    \item[{Datatype}]
  \hyperref[TEI.teidata.numeric]{teidata.numeric}
\end{reflist}  
    \item[@lry]
  gives the y coordinate value for the lower right corner of a rectangular space.
\begin{reflist}
    \item[{Status}]
  Optional
    \item[{Datatype}]
  \hyperref[TEI.teidata.numeric]{teidata.numeric}
\end{reflist}  
    \item[@points]
  identifies a two dimensional area by means of a series of pairs of numbers, each of which gives the x,y coordinates of a point on a line enclosing the area.
\begin{reflist}
    \item[{Status}]
  Optional
    \item[{Datatype}]
  3–∞ occurrences of \hyperref[TEI.teidata.point]{teidata.point} separated by whitespace
\end{reflist}  
\end{sansreflist}  
\end{reflist}  
\begin{reflist}
\item[]\begin{specHead}{TEI.att.cReferencing}{att.cReferencing} provides an attribute which may be used to supply a \textit{canonical reference} as a means of identifying the target of a pointer.\end{specHead} 
    \item[{Module}]
  tei — \hyperref[ST]{The TEI Infrastructure}
    \item[{Members}]
  \hyperref[TEI.gloss]{gloss} \hyperref[TEI.ptr]{ptr} \hyperref[TEI.ref]{ref} \hyperref[TEI.term]{term}
    \item[{Attributes}]
  Attributes\hfil\\[-10pt]\begin{sansreflist}
    \item[@cRef]
  (canonical reference) specifies the destination of the pointer by supplying a canonical reference expressed using the scheme defined in a \hyperref[TEI.refsDecl]{<refsDecl>} element in the TEI header
\begin{reflist}
    \item[{Status}]
  Optional
    \item[{Datatype}]
  \hyperref[TEI.teidata.text]{teidata.text}
    \item[{Note}]
  \par
The value of {\itshape cRef} should be constructed so that when the algorithm for the resolution of canonical references (described in section \textit{\hyperref[SACR]{16.2.5.\ Canonical References}}) is applied to it the result is a valid URI reference to the intended target.\par
The \hyperref[TEI.refsDecl]{<refsDecl>} to use may be indicated with the {\itshape decls} attribute.\par
Currently these Guidelines only provide for a single canonical reference to be encoded on any given \hyperref[TEI.ptr]{<ptr>} element.
\end{reflist}  
\end{sansreflist}  
\end{reflist}  
\begin{reflist}
\item[]\begin{specHead}{TEI.att.damaged}{att.damaged} provides attributes describing the nature of any physical damage affecting a reading. [\textit{\hyperref[PHDA]{11.3.3.1.\ Damage, Illegibility, and Supplied Text}} \textit{\hyperref[STECAT]{1.3.1.\ Attribute Classes}}]\end{specHead} 
    \item[{Module}]
  tei — \hyperref[ST]{The TEI Infrastructure}
    \item[{Members}]
  \hyperref[TEI.damage]{damage} \hyperref[TEI.damageSpan]{damageSpan}
    \item[{Attributes}]
  \hyperref[TEI.att.dimensions]{att.dimensions} (\textit{@unit}, \textit{@quantity}, \textit{@extent}, \textit{@precision}, \textit{@scope})  (\hyperref[TEI.att.ranging]{att.ranging} (\textit{@atLeast}, \textit{@atMost}, \textit{@min}, \textit{@max}, \textit{@confidence})) \hyperref[TEI.att.written]{att.written} (\textit{@hand}) \hfil\\[-10pt]\begin{sansreflist}
    \item[@agent]
  categorizes the cause of the damage, if it can be identified.
\begin{reflist}
    \item[{Status}]
  Optional
    \item[{Datatype}]
  \hyperref[TEI.teidata.enumerated]{teidata.enumerated}
    \item[{Sample values include:}]
  \begin{description}

\item[{rubbing}]damage results from rubbing of the leaf edges
\item[{mildew}]damage results from mildew on the leaf surface
\item[{smoke}]damage results from smoke
\end{description} 
\end{reflist}  
    \item[@degree]
  provides a coded representation of the degree of damage, either as a number between 0 (undamaged) and 1 (very extensively damaged), or as one of the codes high, medium, low, or unknown. The \hyperref[TEI.damage]{<damage>} element with the {\itshape degree} attribute should only be used where the text may be read with some confidence; text supplied from other sources should be tagged as \hyperref[TEI.supplied]{<supplied>}.
\begin{reflist}
    \item[{Status}]
  Optional
    \item[{Datatype}]
  \hyperref[TEI.teidata.probCert]{teidata.probCert}
    \item[{Note}]
  \par
The \hyperref[TEI.damage]{<damage>} element is appropriate where it is desired to record the fact of damage although this has not affected the readability of the text, for example a weathered inscription. Where the damage has rendered the text more or less illegible either the \hyperref[TEI.unclear]{<unclear>} tag (for partial illegibility) or the \hyperref[TEI.gap]{<gap>} tag (for complete illegibility, with no text supplied) should be used, with the information concerning the damage given in the attribute values of these tags. See section \textit{\hyperref[PHCOMB]{11.3.3.2.\ Use of the gap, del, damage, unclear, and supplied Elements in Combination}} for discussion of the use of these tags in particular circumstances.
\end{reflist}  
    \item[@group]
  assigns an arbitrary number to each stretch of damage regarded as forming part of the same physical phenomenon.
\begin{reflist}
    \item[{Status}]
  Optional
    \item[{Datatype}]
  \hyperref[TEI.teidata.count]{teidata.count}
\end{reflist}  
\end{sansreflist}  
\end{reflist}  
\begin{reflist}
\item[]\begin{specHead}{TEI.att.datable}{att.datable} provides attributes for normalization of elements that contain dates, times, or datable events. [\textit{\hyperref[CONADA]{3.6.4.\ Dates and Times}} \textit{\hyperref[NDDATE]{13.4.\ Dates}}]\end{specHead} 
    \item[{Module}]
  tei — \hyperref[ST]{The TEI Infrastructure}
    \item[{Members}]
  \hyperref[TEI.acquisition]{acquisition} \hyperref[TEI.affiliation]{affiliation} \hyperref[TEI.age]{age} \hyperref[TEI.altIdentifier]{altIdentifier} \hyperref[TEI.application]{application} \hyperref[TEI.author]{author} \hyperref[TEI.binding]{binding} \hyperref[TEI.birth]{birth} \hyperref[TEI.bloc]{bloc} \hyperref[TEI.change]{change} \hyperref[TEI.climate]{climate} \hyperref[TEI.conversion]{conversion} \hyperref[TEI.country]{country} \hyperref[TEI.creation]{creation} \hyperref[TEI.custEvent]{custEvent} \hyperref[TEI.date]{date} \hyperref[TEI.death]{death} \hyperref[TEI.district]{district} \hyperref[TEI.editor]{editor} \hyperref[TEI.education]{education} \hyperref[TEI.event]{event} \hyperref[TEI.faith]{faith} \hyperref[TEI.floruit]{floruit} \hyperref[TEI.funder]{funder} \hyperref[TEI.geogFeat]{geogFeat} \hyperref[TEI.geogName]{geogName} \hyperref[TEI.idno]{idno} \hyperref[TEI.langKnowledge]{langKnowledge} \hyperref[TEI.langKnown]{langKnown} \hyperref[TEI.licence]{licence} \hyperref[TEI.location]{location} \hyperref[TEI.meeting]{meeting} \hyperref[TEI.name]{name} \hyperref[TEI.nationality]{nationality} \hyperref[TEI.objectName]{objectName} \hyperref[TEI.occupation]{occupation} \hyperref[TEI.offset]{offset} \hyperref[TEI.orgName]{orgName} \hyperref[TEI.origDate]{origDate} \hyperref[TEI.origPlace]{origPlace} \hyperref[TEI.origin]{origin} \hyperref[TEI.persName]{persName} \hyperref[TEI.persPronouns]{persPronouns} \hyperref[TEI.placeName]{placeName} \hyperref[TEI.population]{population} \hyperref[TEI.precision]{precision} \hyperref[TEI.principal]{principal} \hyperref[TEI.provenance]{provenance} \hyperref[TEI.region]{region} \hyperref[TEI.relation]{relation} \hyperref[TEI.residence]{residence} \hyperref[TEI.resp]{resp} \hyperref[TEI.seal]{seal} \hyperref[TEI.settlement]{settlement} \hyperref[TEI.sex]{sex} \hyperref[TEI.socecStatus]{socecStatus} \hyperref[TEI.sponsor]{sponsor} \hyperref[TEI.stamp]{stamp} \hyperref[TEI.state]{state} \hyperref[TEI.terrain]{terrain} \hyperref[TEI.time]{time} \hyperref[TEI.title]{title} \hyperref[TEI.trait]{trait} \hyperref[TEI.unitDecl]{unitDecl} \hyperref[TEI.unitDef]{unitDef}
    \item[{Attributes}]
  \hyperref[TEI.att.datable.w3c]{att.datable.w3c} (\textit{@when}, \textit{@notBefore}, \textit{@notAfter}, \textit{@from}, \textit{@to}) \hyperref[TEI.att.datable.iso]{att.datable.iso} (\textit{@when-iso}, \textit{@notBefore-iso}, \textit{@notAfter-iso}, \textit{@from-iso}, \textit{@to-iso}) \hyperref[TEI.att.datable.custom]{att.datable.custom} (\textit{@when-custom}, \textit{@notBefore-custom}, \textit{@notAfter-custom}, \textit{@from-custom}, \textit{@to-custom}, \textit{@datingPoint}, \textit{@datingMethod}) \hfil\\[-10pt]\begin{sansreflist}
    \item[@calendar]
  indicates the system or calendar to which the date represented by the content of this element belongs.
\begin{reflist}
    \item[{Status}]
  Optional
    \item[{Datatype}]
  \hyperref[TEI.teidata.pointer]{teidata.pointer}
    \item[{Schematron}]
   <sch:rule context="tei:*[@calendar]"> <sch:assert test="string-length(.) gt 0"> @calendar indicates the system or calendar to  which the date represented by the content of this element belongs, but this <sch:name/> element has no textual content.</sch:assert> </sch:rule>
    \item[]\index{date=<date>|exampleindex}\index{calendar=@calendar!<date>|exampleindex}\index{date=<date>|exampleindex}\index{calendar=@calendar!<date>|exampleindex}\index{when=@when!<date>|exampleindex}\exampleFont He was born on {<\textbf{date}\hspace*{1em}{calendar}="{\#gregorian}">}Feb. 22, 1732{</\textbf{date}>} ({<\textbf{date}\hspace*{1em}{calendar}="{\#julian}"\mbox{}\newline 
\hspace*{1em}{when}="{1732-02-22}">} Feb. 11, 1731/32,\mbox{}\newline 
 O.S.{</\textbf{date}>}).
    \item[{Note}]
  \par
Note that the {\itshape calendar} attribute (unlike {\itshape datingMethod} defined in \textsf{att.datable.custom}) defines the calendar system of the date in the original material defined by the parent element, \textit{not} the calendar to which the date is normalized.
\end{reflist}  
    \item[@period]
  supplies a pointer to some location defining a named period of time within which the datable item is understood to have occurred.
\begin{reflist}
    \item[{Status}]
  Optional
    \item[{Datatype}]
  \hyperref[TEI.teidata.pointer]{teidata.pointer}
\end{reflist}  
\end{sansreflist}  
    \item[{Note}]
  \par
This ‘superclass’ provides attributes that can be used to provide normalized values of temporal information. By default, the attributes from the \textsf{att.datable.w3c} class are provided. If the module for names \& dates is loaded, this class also provides attributes from the \textsf{att.datable.iso} and \textsf{att.datable.custom} classes. In general, the possible values of attributes restricted to the W3C datatypes form a subset of those values available via the ISO 8601 standard. However, the greater expressiveness of the ISO datatypes may not be needed, and there exists much greater software support for the W3C datatypes.
\end{reflist}  
\begin{reflist}
\item[]\begin{specHead}{TEI.att.datable.custom}{att.datable.custom} provides attributes for normalization of elements that contain datable events to a custom dating system (i.e. other than the Gregorian used by W3 and ISO). [\textit{\hyperref[NDDATE]{13.4.\ Dates}}]\end{specHead} 
    \item[{Module}]
  namesdates — \hyperref[ND]{Names, Dates, People, and Places}
    \item[{Members}]
  \hyperref[TEI.att.datable]{att.datable}[\hyperref[TEI.acquisition]{acquisition} \hyperref[TEI.affiliation]{affiliation} \hyperref[TEI.age]{age} \hyperref[TEI.altIdentifier]{altIdentifier} \hyperref[TEI.application]{application} \hyperref[TEI.author]{author} \hyperref[TEI.binding]{binding} \hyperref[TEI.birth]{birth} \hyperref[TEI.bloc]{bloc} \hyperref[TEI.change]{change} \hyperref[TEI.climate]{climate} \hyperref[TEI.conversion]{conversion} \hyperref[TEI.country]{country} \hyperref[TEI.creation]{creation} \hyperref[TEI.custEvent]{custEvent} \hyperref[TEI.date]{date} \hyperref[TEI.death]{death} \hyperref[TEI.district]{district} \hyperref[TEI.editor]{editor} \hyperref[TEI.education]{education} \hyperref[TEI.event]{event} \hyperref[TEI.faith]{faith} \hyperref[TEI.floruit]{floruit} \hyperref[TEI.funder]{funder} \hyperref[TEI.geogFeat]{geogFeat} \hyperref[TEI.geogName]{geogName} \hyperref[TEI.idno]{idno} \hyperref[TEI.langKnowledge]{langKnowledge} \hyperref[TEI.langKnown]{langKnown} \hyperref[TEI.licence]{licence} \hyperref[TEI.location]{location} \hyperref[TEI.meeting]{meeting} \hyperref[TEI.name]{name} \hyperref[TEI.nationality]{nationality} \hyperref[TEI.objectName]{objectName} \hyperref[TEI.occupation]{occupation} \hyperref[TEI.offset]{offset} \hyperref[TEI.orgName]{orgName} \hyperref[TEI.origDate]{origDate} \hyperref[TEI.origPlace]{origPlace} \hyperref[TEI.origin]{origin} \hyperref[TEI.persName]{persName} \hyperref[TEI.persPronouns]{persPronouns} \hyperref[TEI.placeName]{placeName} \hyperref[TEI.population]{population} \hyperref[TEI.precision]{precision} \hyperref[TEI.principal]{principal} \hyperref[TEI.provenance]{provenance} \hyperref[TEI.region]{region} \hyperref[TEI.relation]{relation} \hyperref[TEI.residence]{residence} \hyperref[TEI.resp]{resp} \hyperref[TEI.seal]{seal} \hyperref[TEI.settlement]{settlement} \hyperref[TEI.sex]{sex} \hyperref[TEI.socecStatus]{socecStatus} \hyperref[TEI.sponsor]{sponsor} \hyperref[TEI.stamp]{stamp} \hyperref[TEI.state]{state} \hyperref[TEI.terrain]{terrain} \hyperref[TEI.time]{time} \hyperref[TEI.title]{title} \hyperref[TEI.trait]{trait} \hyperref[TEI.unitDecl]{unitDecl} \hyperref[TEI.unitDef]{unitDef}]
    \item[{Attributes}]
  Attributes\hfil\\[-10pt]\begin{sansreflist}
    \item[@when-custom]
  supplies the value of a date or time in some custom standard form.
\begin{reflist}
    \item[{Status}]
  Optional
    \item[{Datatype}]
  1–∞ occurrences of \hyperref[TEI.teidata.word]{teidata.word} separated by whitespace
    \item[]The following are examples of custom date or time formats that are \textit{not} valid ISO or W3C format normalizations, normalized to a different dating system\index{p=<p>|exampleindex}\index{date=<date>|exampleindex}\index{when=@when!<date>|exampleindex}\index{when-custom=@when-custom!<date>|exampleindex}\index{p=<p>|exampleindex}\index{date=<date>|exampleindex}\index{when=@when!<date>|exampleindex}\index{when-custom=@when-custom!<date>|exampleindex}\index{p=<p>|exampleindex}\index{date=<date>|exampleindex}\index{when-custom=@when-custom!<date>|exampleindex}\index{p=<p>|exampleindex}\index{date=<date>|exampleindex}\index{when-custom=@when-custom!<date>|exampleindex}\exampleFont {<\textbf{p}>}Alhazen died in Cairo on the\mbox{}\newline 
{<\textbf{date}\hspace*{1em}{when}="{1040-03-06}"\mbox{}\newline 
\hspace*{1em}\hspace*{1em}{when-custom}="{431-06-12}">} 12th day of Jumada t-Tania, 430 AH\mbox{}\newline 
\hspace*{1em}{</\textbf{date}>}.{</\textbf{p}>}\mbox{}\newline 
{<\textbf{p}>}The current world will end at the\mbox{}\newline 
{<\textbf{date}\hspace*{1em}{when}="{2012-12-21}"\mbox{}\newline 
\hspace*{1em}\hspace*{1em}{when-custom}="{13.0.0.0.0}">}end of B'ak'tun 13{</\textbf{date}>}.{</\textbf{p}>}\mbox{}\newline 
{<\textbf{p}>}The Battle of Meggidu\mbox{}\newline 
 ({<\textbf{date}\hspace*{1em}{when-custom}="{Thutmose\textunderscore III:23}">}23rd year of reign of Thutmose III{</\textbf{date}>}).{</\textbf{p}>}\mbox{}\newline 
{<\textbf{p}>}Esidorus bixit in pace annos LXX plus minus sub\mbox{}\newline 
{<\textbf{date}\hspace*{1em}{when-custom}="{Ind:4-10-11}">}die XI mensis Octobris indictione IIII{</\textbf{date}>}\mbox{}\newline 
{</\textbf{p}>}Not all custom date formulations will have Gregorian equivalents.The {\itshape when-custom} attribute and other custom dating are not constrained to a datatype by the TEI, but individual projects are recommended to regularize and document their dating formats.
\end{reflist}  
    \item[@notBefore-custom]
  specifies the earliest possible date for the event in some custom standard form.
\begin{reflist}
    \item[{Status}]
  Optional
    \item[{Datatype}]
  1–∞ occurrences of \hyperref[TEI.teidata.word]{teidata.word} separated by whitespace
\end{reflist}  
    \item[@notAfter-custom]
  specifies the latest possible date for the event in some custom standard form.
\begin{reflist}
    \item[{Status}]
  Optional
    \item[{Datatype}]
  1–∞ occurrences of \hyperref[TEI.teidata.word]{teidata.word} separated by whitespace
\end{reflist}  
    \item[@from-custom]
  indicates the starting point of the period in some custom standard form.
\begin{reflist}
    \item[{Status}]
  Optional
    \item[{Datatype}]
  1–∞ occurrences of \hyperref[TEI.teidata.word]{teidata.word} separated by whitespace
    \item[]\index{event=<event>|exampleindex}\index{datingMethod=@datingMethod!<event>|exampleindex}\index{from-custom=@from-custom!<event>|exampleindex}\index{to-custom=@to-custom!<event>|exampleindex}\index{head=<head>|exampleindex}\index{p=<p>|exampleindex}\exampleFont {<\textbf{event}\hspace*{1em}{xml:id}="{FIRE1}"\mbox{}\newline 
\hspace*{1em}{datingMethod}="{\#julian}"\mbox{}\newline 
\hspace*{1em}{from-custom}="{1666-09-02}"\mbox{}\newline 
\hspace*{1em}{to-custom}="{1666-09-05}">}\mbox{}\newline 
\hspace*{1em}{<\textbf{head}>}The Great Fire of London{</\textbf{head}>}\mbox{}\newline 
\hspace*{1em}{<\textbf{p}>}The Great Fire of London burned through a large part\mbox{}\newline 
\hspace*{1em}\hspace*{1em} of the city of London.{</\textbf{p}>}\mbox{}\newline 
{</\textbf{event}>}
\end{reflist}  
    \item[@to-custom]
  indicates the ending point of the period in some custom standard form.
\begin{reflist}
    \item[{Status}]
  Optional
    \item[{Datatype}]
  1–∞ occurrences of \hyperref[TEI.teidata.word]{teidata.word} separated by whitespace
\end{reflist}  
    \item[@datingPoint]
  supplies a pointer to some location defining a named point in time with reference to which the datable item is understood to have occurred
\begin{reflist}
    \item[{Status}]
  Optional
    \item[{Datatype}]
  \hyperref[TEI.teidata.pointer]{teidata.pointer}
\end{reflist}  
    \item[@datingMethod]
  supplies a pointer to a \hyperref[TEI.calendar]{<calendar>} element or other means of interpreting the values of the custom dating attributes.
\begin{reflist}
    \item[{Status}]
  Optional
    \item[{Datatype}]
  \hyperref[TEI.teidata.pointer]{teidata.pointer}
    \item[]\index{date=<date>|exampleindex}\index{when-custom=@when-custom!<date>|exampleindex}\index{calendar=@calendar!<date>|exampleindex}\index{datingMethod=@datingMethod!<date>|exampleindex}\exampleFont Contayning the Originall, Antiquity, Increaſe, Moderne\mbox{}\newline 
 eſtate, and deſcription of that Citie, written in the yeare\mbox{}\newline 
{<\textbf{date}\hspace*{1em}{when-custom}="{1598}"\mbox{}\newline 
\hspace*{1em}{calendar}="{\#julian}"\mbox{}\newline 
\hspace*{1em}{datingMethod}="{\#julian}">}1598{</\textbf{date}>}. by Iohn Stow\mbox{}\newline 
 Citizen of London.In this example, the {\itshape calendar} attribute points to a \hyperref[TEI.calendar]{<calendar>} element for the Julian calendar, specifying that the text content of the \hyperref[TEI.date]{<date>} element is a Julian date, and the {\itshape datingMethod} attribute also points to the Julian calendar to indicate that the content of the {\itshape when-custom} attribute value is Julian too.
    \item[]\index{date=<date>|exampleindex}\index{when=@when!<date>|exampleindex}\index{when-custom=@when-custom!<date>|exampleindex}\index{datingMethod=@datingMethod!<date>|exampleindex}\index{num=<num>|exampleindex}\index{num=<num>|exampleindex}\exampleFont {<\textbf{date}\hspace*{1em}{when}="{1382-06-28}"\mbox{}\newline 
\hspace*{1em}{when-custom}="{6890-06-20}"\mbox{}\newline 
\hspace*{1em}{datingMethod}="{\#creationOfWorld}">} μηνὶ Ἰουνίου εἰς {<\textbf{num}>}κ{</\textbf{num}>} ἔτους {<\textbf{num}>}ςωϞ{</\textbf{num}>}\mbox{}\newline 
{</\textbf{date}>}In this example, a date is given in a Mediaeval text measured "from the creation of the world", which is normalised (in {\itshape when}) to the Gregorian date, but is also normalized (in {\itshape when-custom}) to a machine-actionable, numeric version of the date from the Creation.
    \item[{Note}]
  \par
Note that the {\itshape datingMethod} attribute (unlike {\itshape calendar} defined in \textsf{att.datable}) defines the calendar or dating system to which the date described by the parent element is normalized (i.e. in the {\itshape when-custom} or other {\itshape X-custom} attributes), \textit{not} the calendar of the original date in the element.
\end{reflist}  
\end{sansreflist}  
\end{reflist}  
\begin{reflist}
\item[]\begin{specHead}{TEI.att.datable.iso}{att.datable.iso} provides attributes for normalization of elements that contain datable events using the ISO 8601 standard. [\textit{\hyperref[CONADA]{3.6.4.\ Dates and Times}} \textit{\hyperref[NDDATE]{13.4.\ Dates}}]\end{specHead} 
    \item[{Module}]
  namesdates — \hyperref[ND]{Names, Dates, People, and Places}
    \item[{Members}]
  \hyperref[TEI.att.datable]{att.datable}[\hyperref[TEI.acquisition]{acquisition} \hyperref[TEI.affiliation]{affiliation} \hyperref[TEI.age]{age} \hyperref[TEI.altIdentifier]{altIdentifier} \hyperref[TEI.application]{application} \hyperref[TEI.author]{author} \hyperref[TEI.binding]{binding} \hyperref[TEI.birth]{birth} \hyperref[TEI.bloc]{bloc} \hyperref[TEI.change]{change} \hyperref[TEI.climate]{climate} \hyperref[TEI.conversion]{conversion} \hyperref[TEI.country]{country} \hyperref[TEI.creation]{creation} \hyperref[TEI.custEvent]{custEvent} \hyperref[TEI.date]{date} \hyperref[TEI.death]{death} \hyperref[TEI.district]{district} \hyperref[TEI.editor]{editor} \hyperref[TEI.education]{education} \hyperref[TEI.event]{event} \hyperref[TEI.faith]{faith} \hyperref[TEI.floruit]{floruit} \hyperref[TEI.funder]{funder} \hyperref[TEI.geogFeat]{geogFeat} \hyperref[TEI.geogName]{geogName} \hyperref[TEI.idno]{idno} \hyperref[TEI.langKnowledge]{langKnowledge} \hyperref[TEI.langKnown]{langKnown} \hyperref[TEI.licence]{licence} \hyperref[TEI.location]{location} \hyperref[TEI.meeting]{meeting} \hyperref[TEI.name]{name} \hyperref[TEI.nationality]{nationality} \hyperref[TEI.objectName]{objectName} \hyperref[TEI.occupation]{occupation} \hyperref[TEI.offset]{offset} \hyperref[TEI.orgName]{orgName} \hyperref[TEI.origDate]{origDate} \hyperref[TEI.origPlace]{origPlace} \hyperref[TEI.origin]{origin} \hyperref[TEI.persName]{persName} \hyperref[TEI.persPronouns]{persPronouns} \hyperref[TEI.placeName]{placeName} \hyperref[TEI.population]{population} \hyperref[TEI.precision]{precision} \hyperref[TEI.principal]{principal} \hyperref[TEI.provenance]{provenance} \hyperref[TEI.region]{region} \hyperref[TEI.relation]{relation} \hyperref[TEI.residence]{residence} \hyperref[TEI.resp]{resp} \hyperref[TEI.seal]{seal} \hyperref[TEI.settlement]{settlement} \hyperref[TEI.sex]{sex} \hyperref[TEI.socecStatus]{socecStatus} \hyperref[TEI.sponsor]{sponsor} \hyperref[TEI.stamp]{stamp} \hyperref[TEI.state]{state} \hyperref[TEI.terrain]{terrain} \hyperref[TEI.time]{time} \hyperref[TEI.title]{title} \hyperref[TEI.trait]{trait} \hyperref[TEI.unitDecl]{unitDecl} \hyperref[TEI.unitDef]{unitDef}]
    \item[{Attributes}]
  Attributes\hfil\\[-10pt]\begin{sansreflist}
    \item[@when-iso]
  supplies the value of a date or time in a standard form.
\begin{reflist}
    \item[{Status}]
  Optional
    \item[{Datatype}]
  \hyperref[TEI.teidata.temporal.iso]{teidata.temporal.iso}
    \item[]The following are examples of ISO date, time, and date \& time formats that are \textit{not} valid W3C format normalizations.\index{date=<date>|exampleindex}\index{when-iso=@when-iso!<date>|exampleindex}\index{date=<date>|exampleindex}\index{when-iso=@when-iso!<date>|exampleindex}\index{time=<time>|exampleindex}\index{when-iso=@when-iso!<time>|exampleindex}\index{time=<time>|exampleindex}\index{when-iso=@when-iso!<time>|exampleindex}\index{date=<date>|exampleindex}\index{when-iso=@when-iso!<date>|exampleindex}\index{time=<time>|exampleindex}\index{when-iso=@when-iso!<time>|exampleindex}\index{time=<time>|exampleindex}\index{when-iso=@when-iso!<time>|exampleindex}\index{time=<time>|exampleindex}\index{when-iso=@when-iso!<time>|exampleindex}\exampleFont {<\textbf{date}\hspace*{1em}{when-iso}="{1996-09-24T07:25+00}">}Sept. 24th, 1996 at 3:25 in the morning{</\textbf{date}>}\mbox{}\newline 
{<\textbf{date}\hspace*{1em}{when-iso}="{1996-09-24T03:25-04}">}Sept. 24th, 1996 at 3:25 in the morning{</\textbf{date}>}\mbox{}\newline 
{<\textbf{time}\hspace*{1em}{when-iso}="{1999-01-04T20:42-05}">}4 Jan 1999 at 8:42 pm{</\textbf{time}>}\mbox{}\newline 
{<\textbf{time}\hspace*{1em}{when-iso}="{1999-W01-1T20,70-05}">}4 Jan 1999 at 8:42 pm{</\textbf{time}>}\mbox{}\newline 
{<\textbf{date}\hspace*{1em}{when-iso}="{2006-05-18T10:03}">}a few minutes after ten in the morning on Thu 18 May{</\textbf{date}>}\mbox{}\newline 
{<\textbf{time}\hspace*{1em}{when-iso}="{03:00}">}3 A.M.{</\textbf{time}>}\mbox{}\newline 
{<\textbf{time}\hspace*{1em}{when-iso}="{14}">}around two{</\textbf{time}>}\mbox{}\newline 
{<\textbf{time}\hspace*{1em}{when-iso}="{15,5}">}half past three{</\textbf{time}>}All of the examples of the {\itshape when} attribute in the \textsf{att.datable.w3c} class are also valid with respect to this attribute.
    \item[]\index{q=<q>|exampleindex}\index{time=<time>|exampleindex}\index{when-iso=@when-iso!<time>|exampleindex}\index{time=<time>|exampleindex}\index{when-iso=@when-iso!<time>|exampleindex}\exampleFont He likes to be punctual. I said {<\textbf{q}>}\mbox{}\newline 
\hspace*{1em}{<\textbf{time}\hspace*{1em}{when-iso}="{12}">}around noon{</\textbf{time}>}\mbox{}\newline 
{</\textbf{q}>}, and he showed up at {<\textbf{time}\hspace*{1em}{when-iso}="{12:00:00}">}12 O'clock{</\textbf{time}>} on the dot.The second occurence of \hyperref[TEI.time]{<time>} could have been encoded with the {\itshape when} attribute, as 12:00:00 is a valid time with respect to the W3C \textit{XML Schema Part 2: Datatypes Second Edition} specification. The first occurence could not.
\end{reflist}  
    \item[@notBefore-iso]
  specifies the earliest possible date for the event in standard form, e.g. yyyy-mm-dd.
\begin{reflist}
    \item[{Status}]
  Optional
    \item[{Datatype}]
  \hyperref[TEI.teidata.temporal.iso]{teidata.temporal.iso}
\end{reflist}  
    \item[@notAfter-iso]
  specifies the latest possible date for the event in standard form, e.g. yyyy-mm-dd.
\begin{reflist}
    \item[{Status}]
  Optional
    \item[{Datatype}]
  \hyperref[TEI.teidata.temporal.iso]{teidata.temporal.iso}
\end{reflist}  
    \item[@from-iso]
  indicates the starting point of the period in standard form.
\begin{reflist}
    \item[{Status}]
  Optional
    \item[{Datatype}]
  \hyperref[TEI.teidata.temporal.iso]{teidata.temporal.iso}
\end{reflist}  
    \item[@to-iso]
  indicates the ending point of the period in standard form.
\begin{reflist}
    \item[{Status}]
  Optional
    \item[{Datatype}]
  \hyperref[TEI.teidata.temporal.iso]{teidata.temporal.iso}
\end{reflist}  
\end{sansreflist}  
    \item[{Note}]
  \par
The value of these attributes should be a normalized representation of the date, time, or combined date \& time intended, in any of the standard formats specified by ISO 8601, using the Gregorian calendar.\par
If both {\itshape when-iso} and {\itshape dur-iso} are specified, the values should be interpreted as indicating a span of time by its starting time (or date) and duration. That is, \par\bgroup\index{date=<date>|exampleindex}\index{when-iso=@when-iso!<date>|exampleindex}\index{dur-iso=@dur-iso!<date>|exampleindex}\exampleFont \begin{shaded}\noindent\mbox{}{<\textbf{date}\hspace*{1em}{when-iso}="{2007-06-01}"\hspace*{1em}{dur-iso}="{P8D}"/>}\end{shaded}\egroup\par \noindent  indicates the same time period as \par\bgroup\index{date=<date>|exampleindex}\index{when-iso=@when-iso!<date>|exampleindex}\exampleFont \begin{shaded}\noindent\mbox{}{<\textbf{date}\hspace*{1em}{when-iso}="{2007-06-01/P8D}"/>}\end{shaded}\egroup\par \par
In providing a ‘regularized’ form, no claim is made that the form in the source text is incorrect; the regularized form is simply that chosen as the main form for purposes of unifying variant forms under a single heading.
\end{reflist}  
\begin{reflist}
\item[]\begin{specHead}{TEI.att.datable.w3c}{att.datable.w3c} provides attributes for normalization of elements that contain datable events conforming to the W3C \textit{XML Schema Part 2: Datatypes Second Edition}. [\textit{\hyperref[CONADA]{3.6.4.\ Dates and Times}} \textit{\hyperref[NDDATE]{13.4.\ Dates}}]\end{specHead} 
    \item[{Module}]
  tei — \hyperref[ST]{The TEI Infrastructure}
    \item[{Members}]
  \hyperref[TEI.att.datable]{att.datable}[\hyperref[TEI.acquisition]{acquisition} \hyperref[TEI.affiliation]{affiliation} \hyperref[TEI.age]{age} \hyperref[TEI.altIdentifier]{altIdentifier} \hyperref[TEI.application]{application} \hyperref[TEI.author]{author} \hyperref[TEI.binding]{binding} \hyperref[TEI.birth]{birth} \hyperref[TEI.bloc]{bloc} \hyperref[TEI.change]{change} \hyperref[TEI.climate]{climate} \hyperref[TEI.conversion]{conversion} \hyperref[TEI.country]{country} \hyperref[TEI.creation]{creation} \hyperref[TEI.custEvent]{custEvent} \hyperref[TEI.date]{date} \hyperref[TEI.death]{death} \hyperref[TEI.district]{district} \hyperref[TEI.editor]{editor} \hyperref[TEI.education]{education} \hyperref[TEI.event]{event} \hyperref[TEI.faith]{faith} \hyperref[TEI.floruit]{floruit} \hyperref[TEI.funder]{funder} \hyperref[TEI.geogFeat]{geogFeat} \hyperref[TEI.geogName]{geogName} \hyperref[TEI.idno]{idno} \hyperref[TEI.langKnowledge]{langKnowledge} \hyperref[TEI.langKnown]{langKnown} \hyperref[TEI.licence]{licence} \hyperref[TEI.location]{location} \hyperref[TEI.meeting]{meeting} \hyperref[TEI.name]{name} \hyperref[TEI.nationality]{nationality} \hyperref[TEI.objectName]{objectName} \hyperref[TEI.occupation]{occupation} \hyperref[TEI.offset]{offset} \hyperref[TEI.orgName]{orgName} \hyperref[TEI.origDate]{origDate} \hyperref[TEI.origPlace]{origPlace} \hyperref[TEI.origin]{origin} \hyperref[TEI.persName]{persName} \hyperref[TEI.persPronouns]{persPronouns} \hyperref[TEI.placeName]{placeName} \hyperref[TEI.population]{population} \hyperref[TEI.precision]{precision} \hyperref[TEI.principal]{principal} \hyperref[TEI.provenance]{provenance} \hyperref[TEI.region]{region} \hyperref[TEI.relation]{relation} \hyperref[TEI.residence]{residence} \hyperref[TEI.resp]{resp} \hyperref[TEI.seal]{seal} \hyperref[TEI.settlement]{settlement} \hyperref[TEI.sex]{sex} \hyperref[TEI.socecStatus]{socecStatus} \hyperref[TEI.sponsor]{sponsor} \hyperref[TEI.stamp]{stamp} \hyperref[TEI.state]{state} \hyperref[TEI.terrain]{terrain} \hyperref[TEI.time]{time} \hyperref[TEI.title]{title} \hyperref[TEI.trait]{trait} \hyperref[TEI.unitDecl]{unitDecl} \hyperref[TEI.unitDef]{unitDef}]
    \item[{Attributes}]
  Attributes\hfil\\[-10pt]\begin{sansreflist}
    \item[@when]
  supplies the value of the date or time in a standard form, e.g. yyyy-mm-dd.
\begin{reflist}
    \item[{Status}]
  Optional
    \item[{Datatype}]
  \hyperref[TEI.teidata.temporal.w3c]{teidata.temporal.w3c}
    \item[]Examples of W3C date, time, and date \& time formats.\index{p=<p>|exampleindex}\index{date=<date>|exampleindex}\index{when=@when!<date>|exampleindex}\index{date=<date>|exampleindex}\index{when=@when!<date>|exampleindex}\index{time=<time>|exampleindex}\index{when=@when!<time>|exampleindex}\index{time=<time>|exampleindex}\index{when=@when!<time>|exampleindex}\index{date=<date>|exampleindex}\index{when=@when!<date>|exampleindex}\index{date=<date>|exampleindex}\index{when=@when!<date>|exampleindex}\index{date=<date>|exampleindex}\index{when=@when!<date>|exampleindex}\index{date=<date>|exampleindex}\index{when=@when!<date>|exampleindex}\index{date=<date>|exampleindex}\index{when=@when!<date>|exampleindex}\index{date=<date>|exampleindex}\index{when=@when!<date>|exampleindex}\index{date=<date>|exampleindex}\index{when=@when!<date>|exampleindex}\exampleFont {<\textbf{p}>}\mbox{}\newline 
\hspace*{1em}{<\textbf{date}\hspace*{1em}{when}="{1945-10-24}">}24 Oct 45{</\textbf{date}>}\mbox{}\newline 
\hspace*{1em}{<\textbf{date}\hspace*{1em}{when}="{1996-09-24T07:25:00Z}">}September 24th, 1996 at 3:25 in the morning{</\textbf{date}>}\mbox{}\newline 
\hspace*{1em}{<\textbf{time}\hspace*{1em}{when}="{1999-01-04T20:42:00-05:00}">}Jan 4 1999 at 8 pm{</\textbf{time}>}\mbox{}\newline 
\hspace*{1em}{<\textbf{time}\hspace*{1em}{when}="{14:12:38}">}fourteen twelve and 38 seconds{</\textbf{time}>}\mbox{}\newline 
\hspace*{1em}{<\textbf{date}\hspace*{1em}{when}="{1962-10}">}October of 1962{</\textbf{date}>}\mbox{}\newline 
\hspace*{1em}{<\textbf{date}\hspace*{1em}{when}="{--06-12}">}June 12th{</\textbf{date}>}\mbox{}\newline 
\hspace*{1em}{<\textbf{date}\hspace*{1em}{when}="{---01}">}the first of the month{</\textbf{date}>}\mbox{}\newline 
\hspace*{1em}{<\textbf{date}\hspace*{1em}{when}="{--08}">}August{</\textbf{date}>}\mbox{}\newline 
\hspace*{1em}{<\textbf{date}\hspace*{1em}{when}="{2006}">}MMVI{</\textbf{date}>}\mbox{}\newline 
\hspace*{1em}{<\textbf{date}\hspace*{1em}{when}="{0056}">}AD 56{</\textbf{date}>}\mbox{}\newline 
\hspace*{1em}{<\textbf{date}\hspace*{1em}{when}="{-0056}">}56 BC{</\textbf{date}>}\mbox{}\newline 
{</\textbf{p}>}
    \item[]\index{date=<date>|exampleindex}\index{calendar=@calendar!<date>|exampleindex}\index{when=@when!<date>|exampleindex}\exampleFont This list begins in\mbox{}\newline 
 the year 1632, more precisely on Trinity Sunday, i.e. the Sunday after\mbox{}\newline 
 Pentecost, in that year the\mbox{}\newline 
{<\textbf{date}\hspace*{1em}{calendar}="{\#julian}"\mbox{}\newline 
\hspace*{1em}{when}="{1632-06-06}">}27th of May (old style){</\textbf{date}>}.
    \item[]\index{opener=<opener>|exampleindex}\index{dateline=<dateline>|exampleindex}\index{placeName=<placeName>|exampleindex}\index{date=<date>|exampleindex}\index{when=@when!<date>|exampleindex}\index{salute=<salute>|exampleindex}\index{time=<time>|exampleindex}\index{when=@when!<time>|exampleindex}\exampleFont {<\textbf{opener}>}\mbox{}\newline 
\hspace*{1em}{<\textbf{dateline}>}\mbox{}\newline 
\hspace*{1em}\hspace*{1em}{<\textbf{placeName}>}Dorchester, Village,{</\textbf{placeName}>}\mbox{}\newline 
\hspace*{1em}\hspace*{1em}{<\textbf{date}\hspace*{1em}{when}="{1828-03-02}">}March 2d. 1828.{</\textbf{date}>}\mbox{}\newline 
\hspace*{1em}{</\textbf{dateline}>}\mbox{}\newline 
\hspace*{1em}{<\textbf{salute}>}To\mbox{}\newline 
\hspace*{1em}\hspace*{1em} Mrs. Cornell,{</\textbf{salute}>} Sunday {<\textbf{time}\hspace*{1em}{when}="{12:00:00}">}noon.{</\textbf{time}>}\mbox{}\newline 
{</\textbf{opener}>}
\end{reflist}  
    \item[@notBefore]
  specifies the earliest possible date for the event in standard form, e.g. yyyy-mm-dd.
\begin{reflist}
    \item[{Status}]
  Optional
    \item[{Datatype}]
  \hyperref[TEI.teidata.temporal.w3c]{teidata.temporal.w3c}
\end{reflist}  
    \item[@notAfter]
  specifies the latest possible date for the event in standard form, e.g. yyyy-mm-dd.
\begin{reflist}
    \item[{Status}]
  Optional
    \item[{Datatype}]
  \hyperref[TEI.teidata.temporal.w3c]{teidata.temporal.w3c}
\end{reflist}  
    \item[@from]
  indicates the starting point of the period in standard form, e.g. yyyy-mm-dd.
\begin{reflist}
    \item[{Status}]
  Optional
    \item[{Datatype}]
  \hyperref[TEI.teidata.temporal.w3c]{teidata.temporal.w3c}
\end{reflist}  
    \item[@to]
  indicates the ending point of the period in standard form, e.g. yyyy-mm-dd.
\begin{reflist}
    \item[{Status}]
  Optional
    \item[{Datatype}]
  \hyperref[TEI.teidata.temporal.w3c]{teidata.temporal.w3c}
\end{reflist}  
\end{sansreflist}  
    \item[{Schematron}]
   <sch:rule context="tei:*[@when]"> <sch:report test="@notBefore|@notAfter|@from|@to"  role="nonfatal">The @when attribute cannot be used with any other att.datable.w3c attributes.</sch:report> </sch:rule>
    \item[{Schematron}]
   <sch:rule context="tei:*[@from]"> <sch:report test="@notBefore"  role="nonfatal">The @from and @notBefore attributes cannot be used together.</sch:report> </sch:rule>
    \item[{Schematron}]
   <sch:rule context="tei:*[@to]"> <sch:report test="@notAfter"  role="nonfatal">The @to and @notAfter attributes cannot be used together.</sch:report> </sch:rule>
    \item[{Example}]
  \leavevmode\bgroup\index{date=<date>|exampleindex}\index{from=@from!<date>|exampleindex}\index{to=@to!<date>|exampleindex}\exampleFont \begin{shaded}\noindent\mbox{}{<\textbf{date}\hspace*{1em}{from}="{1863-05-28}"\hspace*{1em}{to}="{1863-06-01}">}28 May through 1 June 1863{</\textbf{date}>}\end{shaded}\egroup 


    \item[{Note}]
  \par
The value of these attributes should be a normalized representation of the date, time, or combined date \& time intended, in any of the standard formats specified by \textit{XML Schema Part 2: Datatypes Second Edition}, using the Gregorian calendar.\par
The most commonly-encountered format for the date portion of a temporal attribute is \texttt{yyyy-mm-dd}, but \texttt{yyyy}, \texttt{--mm}, \texttt{---dd}, \texttt{yyyy-mm}, or \texttt{--mm-dd} may also be used. For the time part, the form \texttt{hh:mm:ss} is used.\par
Note that this format does not currently permit use of the value 0000 to represent the year 1 BCE; instead the value -0001 should be used.
\end{reflist}  
\begin{reflist}
\item[]\begin{specHead}{TEI.att.datcat}{att.datcat} provides the {\itshape dcr:datacat} and {\itshape dcr:ValueDatacat} attributes which are used to align XML elements or attributes with the appropriate Data Categories (DCs) defined by the ISO 12620:2009 standard and stored in the Web repository called ISOCat at \xref{http://www.isocat.org/}{http://www.isocat.org/}. [\textit{\hyperref[DIMVLV]{9.5.2.\ Lexical View}} \textit{\hyperref[FSSY]{18.3.\ Other Atomic Feature Values}}]\end{specHead} 
    \item[{Module}]
  tei — \hyperref[ST]{The TEI Infrastructure}
    \item[{Members}]
  \hyperref[TEI.att.lexicographic]{att.lexicographic}[\hyperref[TEI.case]{case} \hyperref[TEI.colloc]{colloc} \hyperref[TEI.def]{def} \hyperref[TEI.entryFree]{entryFree} \hyperref[TEI.etym]{etym} \hyperref[TEI.form]{form} \hyperref[TEI.gen]{gen} \hyperref[TEI.gram]{gram} \hyperref[TEI.gramGrp]{gramGrp} \hyperref[TEI.hom]{hom} \hyperref[TEI.hyph]{hyph} \hyperref[TEI.iType]{iType} \hyperref[TEI.lang]{lang} \hyperref[TEI.lbl]{lbl} \hyperref[TEI.mood]{mood} \hyperref[TEI.number]{number} \hyperref[TEI.oRef]{oRef} \hyperref[TEI.orth]{orth} \hyperref[TEI.pRef]{pRef} \hyperref[TEI.per]{per} \hyperref[TEI.pos]{pos} \hyperref[TEI.pron]{pron} \hyperref[TEI.re]{re} \hyperref[TEI.sense]{sense} \hyperref[TEI.subc]{subc} \hyperref[TEI.syll]{syll} \hyperref[TEI.tns]{tns} \hyperref[TEI.usg]{usg} \hyperref[TEI.xr]{xr}] \hyperref[TEI.att.segLike]{att.segLike}[\hyperref[TEI.c]{c} \hyperref[TEI.cl]{cl} \hyperref[TEI.m]{m} \hyperref[TEI.pc]{pc} \hyperref[TEI.phr]{phr} \hyperref[TEI.s]{s} \hyperref[TEI.seg]{seg} \hyperref[TEI.w]{w}] \hyperref[TEI.binary]{binary} \hyperref[TEI.f]{f} \hyperref[TEI.fs]{fs} \hyperref[TEI.numeric]{numeric} \hyperref[TEI.string]{string} \hyperref[TEI.symbol]{symbol}
    \item[{Attributes}]
  Attributes\hfil\\[-10pt]\begin{sansreflist}
    \item[@datcat]
  contains a PID (persistent identifier) that aligns the given element with the appropriate Data Category (or categories) in ISOcat.
\begin{reflist}
    \item[{Status}]
  Optional
    \item[{Datatype}]
  1–∞ occurrences of \hyperref[TEI.teidata.pointer]{teidata.pointer} separated by whitespace
\end{reflist}  
    \item[@valueDatcat]
  contains a PID (persistent identifier) that aligns the content of the given element or the value of the given attribute with the appropriate simple Data Category (or categories) in ISOcat.
\begin{reflist}
    \item[{Status}]
  Optional
    \item[{Datatype}]
  1–∞ occurrences of \hyperref[TEI.teidata.pointer]{teidata.pointer} separated by whitespace
\end{reflist}  
\end{sansreflist}  
    \item[{Example}]
  In this example {\itshape dcr:datcat} relates the feature name to the data category "partOfSpeech" and {\itshape dcr:valueDatcat} the feature value to the data category "commonNoun". Both these data categories reside in the ISOcat DCR at \xref{http://www.isocat.org}{www.isocat.org}, which is the DCR used by ISO TC37 and hosted by its registration authority, the MPI for Psycholinguistics in Nijmegen.\leavevmode\bgroup\index{fs=<fs>|exampleindex}\index{f=<f>|exampleindex}\index{name=@name!<f>|exampleindex}\index{dcr:datcat=@dcr:datcat!<f>|exampleindex}\index{fVal=@fVal!<f>|exampleindex}\index{dcr:valueDatcat=@dcr:valueDatcat!<f>|exampleindex}\exampleFont \begin{shaded}\noindent\mbox{}{<\textbf{fs}\mbox{}\newline 
   xmlns:dcr="http://www.isocat.org/ns/dcr">}\mbox{}\newline 
\hspace*{1em}{<\textbf{f}\hspace*{1em}{name}="{POS}"\mbox{}\newline 
\hspace*{1em}\hspace*{1em}{dcr:datcat}="{http://www.isocat.org/datcat/DC-1345}"\hspace*{1em}{fVal}="{\#commonNoun}"\mbox{}\newline 
\hspace*{1em}\hspace*{1em}{dcr:valueDatcat}="{http://www.isocat.org/datcat/DC-1256}"/>}\mbox{}\newline 
{</\textbf{fs}>}\end{shaded}\egroup 


    \item[{Note}]
  \par
ISO 12620:2009 is a standard describing the data model and procedures for a Data Category Registry (DCR). Data categories are defined as elementary descriptors in a linguistic structure. In the DCR data model each data category gets assigned a unique Peristent IDentifier (PID), i.e., an URI. Linguistic resources or preferably their schemas that make use of data categories from a DCR should refer to them using this PID. For XML-based resources, like TEI documents, ISO 12620:2009 normative Annex A gives a small Data Category Reference XML vocabulary (also available online at \xref{http://www.isocat.org/12620/}{http://www.isocat.org/12620/}), which provides two attributes, {\itshape dcr:datcat} and {\itshape dcr:valueDatcat}.
\end{reflist}  
\begin{reflist}
\item[]\begin{specHead}{TEI.att.declarable}{att.declarable} provides attributes for those elements in the TEI header which may be independently selected by means of the special purpose {\itshape decls} attribute. [\textit{\hyperref[CCAS]{15.3.\ Associating Contextual Information with a Text}}]\end{specHead} 
    \item[{Module}]
  tei — \hyperref[ST]{The TEI Infrastructure}
    \item[{Members}]
  \hyperref[TEI.availability]{availability} \hyperref[TEI.bibl]{bibl} \hyperref[TEI.biblFull]{biblFull} \hyperref[TEI.biblStruct]{biblStruct} \hyperref[TEI.broadcast]{broadcast} \hyperref[TEI.correction]{correction} \hyperref[TEI.correspDesc]{correspDesc} \hyperref[TEI.editorialDecl]{editorialDecl} \hyperref[TEI.equipment]{equipment} \hyperref[TEI.geoDecl]{geoDecl} \hyperref[TEI.hyphenation]{hyphenation} \hyperref[TEI.interpretation]{interpretation} \hyperref[TEI.langUsage]{langUsage} \hyperref[TEI.listApp]{listApp} \hyperref[TEI.listBibl]{listBibl} \hyperref[TEI.listEvent]{listEvent} \hyperref[TEI.listNym]{listNym} \hyperref[TEI.listObject]{listObject} \hyperref[TEI.listOrg]{listOrg} \hyperref[TEI.listPerson]{listPerson} \hyperref[TEI.listPlace]{listPlace} \hyperref[TEI.metDecl]{metDecl} \hyperref[TEI.normalization]{normalization} \hyperref[TEI.particDesc]{particDesc} \hyperref[TEI.projectDesc]{projectDesc} \hyperref[TEI.punctuation]{punctuation} \hyperref[TEI.quotation]{quotation} \hyperref[TEI.recording]{recording} \hyperref[TEI.refsDecl]{refsDecl} \hyperref[TEI.samplingDecl]{samplingDecl} \hyperref[TEI.scriptStmt]{scriptStmt} \hyperref[TEI.segmentation]{segmentation} \hyperref[TEI.seriesStmt]{seriesStmt} \hyperref[TEI.settingDesc]{settingDesc} \hyperref[TEI.sourceDesc]{sourceDesc} \hyperref[TEI.stdVals]{stdVals} \hyperref[TEI.styleDefDecl]{styleDefDecl} \hyperref[TEI.textClass]{textClass} \hyperref[TEI.textDesc]{textDesc} \hyperref[TEI.xenoData]{xenoData}
    \item[{Attributes}]
  Attributes\hfil\\[-10pt]\begin{sansreflist}
    \item[@default]
  indicates whether or not this element is selected by default when its parent is selected.
\begin{reflist}
    \item[{Status}]
  Optional
    \item[{Datatype}]
  \hyperref[TEI.teidata.truthValue]{teidata.truthValue}
    \item[{Legal values are:}]
  \begin{description}

\item[{true}]This element is selected if its parent is selected
\item[{false}]This element can only be selected explicitly, unless it is the only one of its kind, in which case it is selected if its parent is selected.{[Default] }
\end{description} 
\end{reflist}  
\end{sansreflist}  
    \item[{Note}]
  \par
The rules governing the association of declarable elements with individual parts of a TEI text are fully defined in chapter \textit{\hyperref[CCAS]{15.3.\ Associating Contextual Information with a Text}}. Only one element of a particular type may have a {\itshape default} attribute with a value of true.
\end{reflist}  
\begin{reflist}
\item[]\begin{specHead}{TEI.att.declaring}{att.declaring} provides attributes for elements which may be independently associated with a particular declarable element within the header, thus overriding the inherited default for that element. [\textit{\hyperref[CCAS]{15.3.\ Associating Contextual Information with a Text}}]\end{specHead} 
    \item[{Module}]
  tei — \hyperref[ST]{The TEI Infrastructure}
    \item[{Members}]
  \hyperref[TEI.ab]{ab} \hyperref[TEI.back]{back} \hyperref[TEI.body]{body} \hyperref[TEI.div]{div} \hyperref[TEI.div1]{div1} \hyperref[TEI.div2]{div2} \hyperref[TEI.div3]{div3} \hyperref[TEI.div4]{div4} \hyperref[TEI.div5]{div5} \hyperref[TEI.div6]{div6} \hyperref[TEI.div7]{div7} \hyperref[TEI.facsimile]{facsimile} \hyperref[TEI.floatingText]{floatingText} \hyperref[TEI.front]{front} \hyperref[TEI.geo]{geo} \hyperref[TEI.gloss]{gloss} \hyperref[TEI.graphic]{graphic} \hyperref[TEI.group]{group} \hyperref[TEI.lg]{lg} \hyperref[TEI.listAnnotation]{listAnnotation} \hyperref[TEI.media]{media} \hyperref[TEI.msDesc]{msDesc} \hyperref[TEI.object]{object} \hyperref[TEI.p]{p} \hyperref[TEI.ptr]{ptr} \hyperref[TEI.ref]{ref} \hyperref[TEI.sourceDoc]{sourceDoc} \hyperref[TEI.standOff]{standOff} \hyperref[TEI.surface]{surface} \hyperref[TEI.surfaceGrp]{surfaceGrp} \hyperref[TEI.term]{term} \hyperref[TEI.text]{text} \hyperref[TEI.u]{u}
    \item[{Attributes}]
  Attributes\hfil\\[-10pt]\begin{sansreflist}
    \item[@decls]
  identifies one or more \textit{declarable elements} within the header, which are understood to apply to the element bearing this attribute and its content.
\begin{reflist}
    \item[{Status}]
  Optional
    \item[{Datatype}]
  1–∞ occurrences of \hyperref[TEI.teidata.pointer]{teidata.pointer} separated by whitespace
\end{reflist}  
\end{sansreflist}  
    \item[{Note}]
  \par
The rules governing the association of declarable elements with individual parts of a TEI text are fully defined in chapter \textit{\hyperref[CCAS]{15.3.\ Associating Contextual Information with a Text}}.
\end{reflist}  
\begin{reflist}
\item[]\begin{specHead}{TEI.att.deprecated}{att.deprecated} provides attributes indicating how a deprecated feature will be treated in future releases.\end{specHead} 
    \item[{Module}]
  tagdocs — \hyperref[TD]{Documentation Elements}
    \item[{Members}]
  \hyperref[TEI.att.combinable]{att.combinable}[\hyperref[TEI.att.identified]{att.identified}[\hyperref[TEI.attDef]{attDef} \hyperref[TEI.classSpec]{classSpec} \hyperref[TEI.constraintSpec]{constraintSpec} \hyperref[TEI.dataSpec]{dataSpec} \hyperref[TEI.elementSpec]{elementSpec} \hyperref[TEI.macroSpec]{macroSpec} \hyperref[TEI.moduleSpec]{moduleSpec} \hyperref[TEI.paramSpec]{paramSpec} \hyperref[TEI.schemaSpec]{schemaSpec}] \hyperref[TEI.defaultVal]{defaultVal} \hyperref[TEI.remarks]{remarks} \hyperref[TEI.valDesc]{valDesc} \hyperref[TEI.valItem]{valItem} \hyperref[TEI.valList]{valList}]
    \item[{Attributes}]
  Attributes\hfil\\[-10pt]\begin{sansreflist}
    \item[@validUntil]
  provides a date before which the construct being defined will not be removed.
\begin{reflist}
    \item[{Status}]
  Optional
    \item[{Datatype}]
  \xref{https://www.w3.org/TR/xmlschema-2/\#date}{date}
    \item[{Schematron}]
   <sch:rule context="tei:*[@validUntil]"> <sch:let name="advance\textunderscore warning\textunderscore period"  value="current-date() + xs:dayTimeDuration('P60D')"/> <sch:let name="me\textunderscore phrase"  value="if (@ident) then concat('The ', @ident ) else concat('This ', local-name(.),   ' of ', ancestor::tei:*[@ident][1]/@ident )"/> <sch:assert test="@validUntil cast as xs:date ge current-date()">  <sch:value-of select=" concat( \$me\textunderscore phrase, ' construct is outdated (as of ', @validUntil,   '); ODD processors may ignore it, and its use is no longer supported'   )"/> </sch:assert> <sch:assert role="nonfatal"  test="@validUntil cast as xs:date ge \$advance\textunderscore warning\textunderscore period">  <sch:value-of select="concat( \$me\textunderscore phrase, ' construct becomes outdated on ', @validUntil   )"/> </sch:assert> </sch:rule>
    \item[{Schematron}]
   <sch:rule context="tei:*[@validUntil][ not( self::valDesc | self::valList | self::defaultVal   )]"> <sch:assert test="child::tei:desc[ @type eq 'deprecationInfo']"> A deprecated construct should include, whenever possible, an explanation, but this <sch:value-of select="name(.)"/> does not have a child <desc type="deprecationInfo"></sch:assert> </sch:rule>
    \item[{Note}]
  \par
The value of this attribute should represent a date (in standard \texttt{yyyy-mm-dd} format) which is later than the date on which the attribute is added to an ODD. Technically, this attribute asserts only the intent to leave a construct in future releases of the markup language being defined up to at least the specified date, and makes no assertion about what happens past that date. In practice, the expectation is that the construct will be removed from future releases of the markup language being defined sometime shortly after the {\itshape validUntil} date.\par
An ODD processor will typically not process a specification element which has a {\itshape validUntil} date that is in the past. An ODD processor will typically warn users about constructs which have a {\itshape validUntil} date that is in the future. E.g., the documentation for such a construct might include the phrase \textit{warning: deprecated} in red.
\end{reflist}  
\end{sansreflist}  
\end{reflist}  
\begin{reflist}
\item[]\begin{specHead}{TEI.att.dimensions}{att.dimensions} provides attributes for describing the size of physical objects.\end{specHead} 
    \item[{Module}]
  tei — \hyperref[ST]{The TEI Infrastructure}
    \item[{Members}]
  \hyperref[TEI.att.damaged]{att.damaged}[\hyperref[TEI.damage]{damage} \hyperref[TEI.damageSpan]{damageSpan}] \hyperref[TEI.add]{add} \hyperref[TEI.addSpan]{addSpan} \hyperref[TEI.age]{age} \hyperref[TEI.birth]{birth} \hyperref[TEI.date]{date} \hyperref[TEI.death]{death} \hyperref[TEI.del]{del} \hyperref[TEI.delSpan]{delSpan} \hyperref[TEI.depth]{depth} \hyperref[TEI.dim]{dim} \hyperref[TEI.dimensions]{dimensions} \hyperref[TEI.ex]{ex} \hyperref[TEI.floruit]{floruit} \hyperref[TEI.gap]{gap} \hyperref[TEI.geogFeat]{geogFeat} \hyperref[TEI.height]{height} \hyperref[TEI.mod]{mod} \hyperref[TEI.offset]{offset} \hyperref[TEI.origDate]{origDate} \hyperref[TEI.population]{population} \hyperref[TEI.redo]{redo} \hyperref[TEI.restore]{restore} \hyperref[TEI.retrace]{retrace} \hyperref[TEI.secl]{secl} \hyperref[TEI.space]{space} \hyperref[TEI.state]{state} \hyperref[TEI.subst]{subst} \hyperref[TEI.substJoin]{substJoin} \hyperref[TEI.supplied]{supplied} \hyperref[TEI.surplus]{surplus} \hyperref[TEI.time]{time} \hyperref[TEI.trait]{trait} \hyperref[TEI.unclear]{unclear} \hyperref[TEI.undo]{undo} \hyperref[TEI.width]{width}
    \item[{Attributes}]
  \hyperref[TEI.att.ranging]{att.ranging} (\textit{@atLeast}, \textit{@atMost}, \textit{@min}, \textit{@max}, \textit{@confidence}) \hfil\\[-10pt]\begin{sansreflist}
    \item[@unit]
  names the unit used for the measurement
\begin{reflist}
    \item[{Status}]
  Optional
    \item[{Datatype}]
  \hyperref[TEI.teidata.enumerated]{teidata.enumerated}
    \item[{Suggested values include:}]
  \begin{description}

\item[{cm}](centimetres)
\item[{mm}](millimetres)
\item[{in}](inches)
\item[{line}]lines of text
\item[{char}](characters) characters of text
\end{description} 
\end{reflist}  
    \item[@quantity]
  specifies the length in the units specified
\begin{reflist}
    \item[{Status}]
  Optional
    \item[{Datatype}]
  \hyperref[TEI.teidata.numeric]{teidata.numeric}
\end{reflist}  
    \item[@extent]
  indicates the size of the object concerned using a project-specific vocabulary combining quantity and units in a single string of words.
\begin{reflist}
    \item[{Status}]
  Optional
    \item[{Datatype}]
  \hyperref[TEI.teidata.text]{teidata.text}
    \item[]\index{gap=<gap>|exampleindex}\index{extent=@extent!<gap>|exampleindex}\exampleFont {<\textbf{gap}\hspace*{1em}{extent}="{5 words}"/>}
    \item[]\index{height=<height>|exampleindex}\index{extent=@extent!<height>|exampleindex}\exampleFont {<\textbf{height}\hspace*{1em}{extent}="{half the page}"/>}
\end{reflist}  
    \item[@precision]
  characterizes the precision of the values specified by the other attributes.
\begin{reflist}
    \item[{Status}]
  Optional
    \item[{Datatype}]
  \hyperref[TEI.teidata.certainty]{teidata.certainty}
\end{reflist}  
    \item[@scope]
  where the measurement summarizes more than one observation, specifies the applicability of this measurement.
\begin{reflist}
    \item[{Status}]
  Optional
    \item[{Datatype}]
  \hyperref[TEI.teidata.enumerated]{teidata.enumerated}
    \item[{Sample values include:}]
  \begin{description}

\item[{all}]measurement applies to all instances.
\item[{most}]measurement applies to most of the instances inspected.
\item[{range}]measurement applies to only the specified range of instances.
\end{description} 
\end{reflist}  
\end{sansreflist}  
\end{reflist}  
\begin{reflist}
\item[]\begin{specHead}{TEI.att.divLike}{att.divLike} provides attributes common to all elements which behave in the same way as divisions. [\textit{\hyperref[DS]{4.\ Default Text Structure}}]\end{specHead} 
    \item[{Module}]
  tei — \hyperref[ST]{The TEI Infrastructure}
    \item[{Members}]
  \hyperref[TEI.div]{div} \hyperref[TEI.div1]{div1} \hyperref[TEI.div2]{div2} \hyperref[TEI.div3]{div3} \hyperref[TEI.div4]{div4} \hyperref[TEI.div5]{div5} \hyperref[TEI.div6]{div6} \hyperref[TEI.div7]{div7} \hyperref[TEI.lg]{lg}
    \item[{Attributes}]
  \hyperref[TEI.att.metrical]{att.metrical} (\textit{@met}, \textit{@real}, \textit{@rhyme}) \hyperref[TEI.att.fragmentable]{att.fragmentable} (\textit{@part}) \hfil\\[-10pt]\begin{sansreflist}
    \item[@org]
  (organization) specifies how the content of the division is organized.
\begin{reflist}
    \item[{Status}]
  Optional
    \item[{Datatype}]
  \hyperref[TEI.teidata.enumerated]{teidata.enumerated}
    \item[{Legal values are:}]
  \begin{description}

\item[{composite}]no claim is made about the sequence in which the immediate contents of this division are to be processed, or their inter-relationships.
\item[{uniform}]the immediate contents of this element are regarded as forming a logical unit, to be processed in sequence.{[Default] }
\end{description} 
\end{reflist}  
    \item[@sample]
  indicates whether this division is a sample of the original source and if so, from which part.
\begin{reflist}
    \item[{Status}]
  Optional
    \item[{Datatype}]
  \hyperref[TEI.teidata.enumerated]{teidata.enumerated}
    \item[{Legal values are:}]
  \begin{description}

\item[{initial}]division lacks material present at end in source.
\item[{medial}]division lacks material at start and end.
\item[{final}]division lacks material at start.
\item[{unknown}]position of sampled material within original unknown.
\item[{complete}]division is not a sample.{[Default] }
\end{description} 
\end{reflist}  
\end{sansreflist}  
\end{reflist}  
\begin{reflist}
\item[]\begin{specHead}{TEI.att.docStatus}{att.docStatus} provides attributes for use on metadata elements describing the status of a document.\end{specHead} 
    \item[{Module}]
  tei — \hyperref[ST]{The TEI Infrastructure}
    \item[{Members}]
  \hyperref[TEI.bibl]{bibl} \hyperref[TEI.biblFull]{biblFull} \hyperref[TEI.biblStruct]{biblStruct} \hyperref[TEI.change]{change} \hyperref[TEI.msDesc]{msDesc} \hyperref[TEI.object]{object} \hyperref[TEI.revisionDesc]{revisionDesc} \hyperref[TEI.schemaSpec]{schemaSpec}
    \item[{Attributes}]
  Attributes\hfil\\[-10pt]\begin{sansreflist}
    \item[@status]
  describes the status of a document either currently or, when associated with a dated element, at the time indicated.
\begin{reflist}
    \item[{Status}]
  Optional
    \item[{Datatype}]
  \hyperref[TEI.teidata.enumerated]{teidata.enumerated}
    \item[{Sample values include:}]
  \begin{description}

\item[{approved}]
\item[{candidate}]
\item[{cleared}]
\item[{deprecated}]
\item[{draft}]{[Default] }
\item[{embargoed}]
\item[{expired}]
\item[{frozen}]
\item[{galley}]
\item[{proposed}]
\item[{published}]
\item[{recommendation}]
\item[{submitted}]
\item[{unfinished}]
\item[{withdrawn}]
\end{description} 
\end{reflist}  
\end{sansreflist}  
    \item[{Example}]
  \leavevmode\bgroup\index{revisionDesc=<revisionDesc>|exampleindex}\index{status=@status!<revisionDesc>|exampleindex}\index{change=<change>|exampleindex}\index{when=@when!<change>|exampleindex}\index{status=@status!<change>|exampleindex}\index{change=<change>|exampleindex}\index{when=@when!<change>|exampleindex}\index{status=@status!<change>|exampleindex}\index{change=<change>|exampleindex}\index{when=@when!<change>|exampleindex}\index{status=@status!<change>|exampleindex}\index{change=<change>|exampleindex}\index{when=@when!<change>|exampleindex}\index{status=@status!<change>|exampleindex}\index{who=@who!<change>|exampleindex}\index{change=<change>|exampleindex}\index{when=@when!<change>|exampleindex}\index{status=@status!<change>|exampleindex}\index{who=@who!<change>|exampleindex}\exampleFont \begin{shaded}\noindent\mbox{}{<\textbf{revisionDesc}\hspace*{1em}{status}="{published}">}\mbox{}\newline 
\hspace*{1em}{<\textbf{change}\hspace*{1em}{when}="{2010-10-21}"\mbox{}\newline 
\hspace*{1em}\hspace*{1em}{status}="{published}"/>}\mbox{}\newline 
\hspace*{1em}{<\textbf{change}\hspace*{1em}{when}="{2010-10-02}"\hspace*{1em}{status}="{cleared}"/>}\mbox{}\newline 
\hspace*{1em}{<\textbf{change}\hspace*{1em}{when}="{2010-08-02}"\mbox{}\newline 
\hspace*{1em}\hspace*{1em}{status}="{embargoed}"/>}\mbox{}\newline 
\hspace*{1em}{<\textbf{change}\hspace*{1em}{when}="{2010-05-01}"\hspace*{1em}{status}="{frozen}"\mbox{}\newline 
\hspace*{1em}\hspace*{1em}{who}="{\#MSM}"/>}\mbox{}\newline 
\hspace*{1em}{<\textbf{change}\hspace*{1em}{when}="{2010-03-01}"\hspace*{1em}{status}="{draft}"\mbox{}\newline 
\hspace*{1em}\hspace*{1em}{who}="{\#LB}"/>}\mbox{}\newline 
{</\textbf{revisionDesc}>}\end{shaded}\egroup 


\end{reflist}  
\begin{reflist}
\item[]\begin{specHead}{TEI.att.duration}{att.duration} provides attributes for normalization of elements that contain datable events.\end{specHead} 
    \item[{Module}]
  spoken — \hyperref[TS]{Transcriptions of Speech}
    \item[{Members}]
  \hyperref[TEI.att.timed]{att.timed}[\hyperref[TEI.annotationBlock]{annotationBlock} \hyperref[TEI.binaryObject]{binaryObject} \hyperref[TEI.gap]{gap} \hyperref[TEI.incident]{incident} \hyperref[TEI.kinesic]{kinesic} \hyperref[TEI.media]{media} \hyperref[TEI.pause]{pause} \hyperref[TEI.u]{u} \hyperref[TEI.vocal]{vocal} \hyperref[TEI.writing]{writing}] \hyperref[TEI.date]{date} \hyperref[TEI.recording]{recording} \hyperref[TEI.time]{time}
    \item[{Attributes}]
  \hyperref[TEI.att.duration.w3c]{att.duration.w3c} (\textit{@dur}) \hyperref[TEI.att.duration.iso]{att.duration.iso} (\textit{@dur-iso}) 
    \item[{Note}]
  \par
This ‘superclass’ provides attributes that can be used to provide normalized values of temporal information. By default, the attributes from the \textsf{att.duration.w3c} class are provided. If the module for names \& dates is loaded, this class also provides attributes from the \textsf{att.duration.iso} class. In general, the possible values of attributes restricted to the W3C datatypes form a subset of those values available via the ISO 8601 standard. However, the greater expressiveness of the ISO datatypes is rarely needed, and there exists much greater software support for the W3C datatypes.
\end{reflist}  
\begin{reflist}
\item[]\begin{specHead}{TEI.att.duration.iso}{att.duration.iso} provides attributes for recording normalized temporal durations. [\textit{\hyperref[CONADA]{3.6.4.\ Dates and Times}} \textit{\hyperref[NDDATE]{13.4.\ Dates}}]\end{specHead} 
    \item[{Module}]
  tei — \hyperref[ST]{The TEI Infrastructure}
    \item[{Members}]
  \hyperref[TEI.att.duration]{att.duration}[\hyperref[TEI.att.timed]{att.timed}[\hyperref[TEI.annotationBlock]{annotationBlock} \hyperref[TEI.binaryObject]{binaryObject} \hyperref[TEI.gap]{gap} \hyperref[TEI.incident]{incident} \hyperref[TEI.kinesic]{kinesic} \hyperref[TEI.media]{media} \hyperref[TEI.pause]{pause} \hyperref[TEI.u]{u} \hyperref[TEI.vocal]{vocal} \hyperref[TEI.writing]{writing}] \hyperref[TEI.date]{date} \hyperref[TEI.recording]{recording} \hyperref[TEI.time]{time}]
    \item[{Attributes}]
  Attributes\hfil\\[-10pt]\begin{sansreflist}
    \item[@dur-iso]
  (duration) indicates the length of this element in time.
\begin{reflist}
    \item[{Status}]
  Optional
    \item[{Datatype}]
  \hyperref[TEI.teidata.duration.iso]{teidata.duration.iso}
\end{reflist}  
\end{sansreflist}  
    \item[{Note}]
  \par
If both {\itshape when} and {\itshape dur} or {\itshape dur-iso} are specified, the values should be interpreted as indicating a span of time by its starting time (or date) and duration. In order to represent a time range by a duration and its ending time the {\itshape when-iso} attribute must be used.\par
In providing a ‘regularized’ form, no claim is made that the form in the source text is incorrect; the regularized form is simply that chosen as the main form for purposes of unifying variant forms under a single heading.
\end{reflist}  
\begin{reflist}
\item[]\begin{specHead}{TEI.att.duration.w3c}{att.duration.w3c} provides attributes for recording normalized temporal durations. [\textit{\hyperref[CONADA]{3.6.4.\ Dates and Times}} \textit{\hyperref[NDDATE]{13.4.\ Dates}}]\end{specHead} 
    \item[{Module}]
  tei — \hyperref[ST]{The TEI Infrastructure}
    \item[{Members}]
  \hyperref[TEI.att.duration]{att.duration}[\hyperref[TEI.att.timed]{att.timed}[\hyperref[TEI.annotationBlock]{annotationBlock} \hyperref[TEI.binaryObject]{binaryObject} \hyperref[TEI.gap]{gap} \hyperref[TEI.incident]{incident} \hyperref[TEI.kinesic]{kinesic} \hyperref[TEI.media]{media} \hyperref[TEI.pause]{pause} \hyperref[TEI.u]{u} \hyperref[TEI.vocal]{vocal} \hyperref[TEI.writing]{writing}] \hyperref[TEI.date]{date} \hyperref[TEI.recording]{recording} \hyperref[TEI.time]{time}]
    \item[{Attributes}]
  Attributes\hfil\\[-10pt]\begin{sansreflist}
    \item[@dur]
  (duration) indicates the length of this element in time.
\begin{reflist}
    \item[{Status}]
  Optional
    \item[{Datatype}]
  \hyperref[TEI.teidata.duration.w3c]{teidata.duration.w3c}
\end{reflist}  
\end{sansreflist}  
    \item[{Note}]
  \par
If both {\itshape when} and {\itshape dur} are specified, the values should be interpreted as indicating a span of time by its starting time (or date) and duration. In order to represent a time range by a duration and its ending time the {\itshape when-iso} attribute must be used.\par
In providing a ‘regularized’ form, no claim is made that the form in the source text is incorrect; the regularized form is simply that chosen as the main form for purposes of unifying variant forms under a single heading.
\end{reflist}  
\begin{reflist}
\item[]\begin{specHead}{TEI.att.edition}{att.edition} provides attributes identifying the source edition from which some encoded feature derives.\end{specHead} 
    \item[{Module}]
  tei — \hyperref[ST]{The TEI Infrastructure}
    \item[{Members}]
  \hyperref[TEI.cb]{cb} \hyperref[TEI.gb]{gb} \hyperref[TEI.lb]{lb} \hyperref[TEI.milestone]{milestone} \hyperref[TEI.pb]{pb} \hyperref[TEI.refState]{refState}
    \item[{Attributes}]
  Attributes\hfil\\[-10pt]\begin{sansreflist}
    \item[@ed]
  (edition) supplies a sigil or other arbitrary identifier for the source edition in which the associated feature (for example, a page, column, or line break) occurs at this point in the text.
\begin{reflist}
    \item[{Status}]
  Optional
    \item[{Datatype}]
  1–∞ occurrences of \hyperref[TEI.teidata.word]{teidata.word} separated by whitespace
\end{reflist}  
    \item[@edRef]
  (edition reference) provides a pointer to the source edition in which the associated feature (for example, a page, column, or line break) occurs at this point in the text.
\begin{reflist}
    \item[{Status}]
  Optional
    \item[{Datatype}]
  1–∞ occurrences of \hyperref[TEI.teidata.pointer]{teidata.pointer} separated by whitespace
\end{reflist}  
\end{sansreflist}  
    \item[{Example}]
  \leavevmode\bgroup\index{l=<l>|exampleindex}\index{lb=<lb>|exampleindex}\index{ed=@ed!<lb>|exampleindex}\index{lb=<lb>|exampleindex}\index{ed=@ed!<lb>|exampleindex}\index{l=<l>|exampleindex}\index{lb=<lb>|exampleindex}\index{ed=@ed!<lb>|exampleindex}\index{l=<l>|exampleindex}\index{lb=<lb>|exampleindex}\index{ed=@ed!<lb>|exampleindex}\index{lb=<lb>|exampleindex}\index{ed=@ed!<lb>|exampleindex}\exampleFont \begin{shaded}\noindent\mbox{}{<\textbf{l}>}Of Mans First Disobedience,{<\textbf{lb}\hspace*{1em}{ed}="{1674}"/>} and{<\textbf{lb}\hspace*{1em}{ed}="{1667}"/>} the Fruit{</\textbf{l}>}\mbox{}\newline 
{<\textbf{l}>}Of that Forbidden Tree, whose{<\textbf{lb}\hspace*{1em}{ed}="{1667 1674}"/>} mortal tast{</\textbf{l}>}\mbox{}\newline 
{<\textbf{l}>}Brought Death into the World,{<\textbf{lb}\hspace*{1em}{ed}="{1667}"/>} and all{<\textbf{lb}\hspace*{1em}{ed}="{1674}"/>} our woe,{</\textbf{l}>}\end{shaded}\egroup 


    \item[{Example}]
  \leavevmode\bgroup\index{listBibl=<listBibl>|exampleindex}\index{bibl=<bibl>|exampleindex}\index{author=<author>|exampleindex}\index{title=<title>|exampleindex}\index{publisher=<publisher>|exampleindex}\index{date=<date>|exampleindex}\index{bibl=<bibl>|exampleindex}\index{author=<author>|exampleindex}\index{title=<title>|exampleindex}\index{publisher=<publisher>|exampleindex}\index{date=<date>|exampleindex}\index{p=<p>|exampleindex}\index{pb=<pb>|exampleindex}\index{n=@n!<pb>|exampleindex}\index{edRef=@edRef!<pb>|exampleindex}\index{pb=<pb>|exampleindex}\index{n=@n!<pb>|exampleindex}\index{edRef=@edRef!<pb>|exampleindex}\exampleFont \begin{shaded}\noindent\mbox{}{<\textbf{listBibl}>}\mbox{}\newline 
\hspace*{1em}{<\textbf{bibl}\hspace*{1em}{xml:id}="{stapledon1937}">}\mbox{}\newline 
\hspace*{1em}\hspace*{1em}{<\textbf{author}>}Olaf Stapledon{</\textbf{author}>},\mbox{}\newline 
\hspace*{1em}{<\textbf{title}>}Starmaker{</\textbf{title}>}, {<\textbf{publisher}>}Methuen{</\textbf{publisher}>}, {<\textbf{date}>}1937{</\textbf{date}>}\mbox{}\newline 
\hspace*{1em}{</\textbf{bibl}>}\mbox{}\newline 
\hspace*{1em}{<\textbf{bibl}\hspace*{1em}{xml:id}="{stapledon1968}">}\mbox{}\newline 
\hspace*{1em}\hspace*{1em}{<\textbf{author}>}Olaf Stapledon{</\textbf{author}>},\mbox{}\newline 
\hspace*{1em}{<\textbf{title}>}Starmaker{</\textbf{title}>}, {<\textbf{publisher}>}Dover{</\textbf{publisher}>}, {<\textbf{date}>}1968{</\textbf{date}>}\mbox{}\newline 
\hspace*{1em}{</\textbf{bibl}>}\mbox{}\newline 
{</\textbf{listBibl}>}\mbox{}\newline 
\textit{<!-- ... -->}\mbox{}\newline 
{<\textbf{p}>}Looking into the future aeons from the supreme moment of\mbox{}\newline 
 the cosmos, I saw the populations still with all their\mbox{}\newline 
 strength maintaining the{<\textbf{pb}\hspace*{1em}{n}="{411}"\hspace*{1em}{edRef}="{\#stapledon1968}"/>}essentials of their ancient culture,\mbox{}\newline 
 still living their personal lives in zest and endless\mbox{}\newline 
 novelty of action, … I saw myself still\mbox{}\newline 
 preserving, though with increasing difficulty, my lucid\mbox{}\newline 
 con-{<\textbf{pb}\hspace*{1em}{n}="{291}"\hspace*{1em}{edRef}="{\#stapledon1937}"/>}sciousness;{</\textbf{p}>}\end{shaded}\egroup 


\end{reflist}  
\begin{reflist}
\item[]\begin{specHead}{TEI.att.editLike}{att.editLike} provides attributes describing the nature of an encoded scholarly intervention or interpretation of any kind. [\textit{\hyperref[COED]{3.5.\ Simple Editorial Changes}} \textit{\hyperref[msdates]{10.3.1.\ Origination}} \textit{\hyperref[NDPERSE]{13.3.2.\ The Person Element}} \textit{\hyperref[PHCO]{11.3.1.1.\ Core Elements for Transcriptional Work}}]\end{specHead} 
    \item[{Module}]
  tei — \hyperref[ST]{The TEI Infrastructure}
    \item[{Members}]
  \hyperref[TEI.att.transcriptional]{att.transcriptional}[\hyperref[TEI.add]{add} \hyperref[TEI.addSpan]{addSpan} \hyperref[TEI.del]{del} \hyperref[TEI.delSpan]{delSpan} \hyperref[TEI.mod]{mod} \hyperref[TEI.redo]{redo} \hyperref[TEI.restore]{restore} \hyperref[TEI.retrace]{retrace} \hyperref[TEI.rt]{rt} \hyperref[TEI.subst]{subst} \hyperref[TEI.substJoin]{substJoin} \hyperref[TEI.undo]{undo}] \hyperref[TEI.affiliation]{affiliation} \hyperref[TEI.age]{age} \hyperref[TEI.am]{am} \hyperref[TEI.birth]{birth} \hyperref[TEI.climate]{climate} \hyperref[TEI.corr]{corr} \hyperref[TEI.date]{date} \hyperref[TEI.death]{death} \hyperref[TEI.education]{education} \hyperref[TEI.event]{event} \hyperref[TEI.ex]{ex} \hyperref[TEI.expan]{expan} \hyperref[TEI.faith]{faith} \hyperref[TEI.floruit]{floruit} \hyperref[TEI.gap]{gap} \hyperref[TEI.geogFeat]{geogFeat} \hyperref[TEI.geogName]{geogName} \hyperref[TEI.langKnowledge]{langKnowledge} \hyperref[TEI.langKnown]{langKnown} \hyperref[TEI.location]{location} \hyperref[TEI.name]{name} \hyperref[TEI.nationality]{nationality} \hyperref[TEI.objectName]{objectName} \hyperref[TEI.occupation]{occupation} \hyperref[TEI.offset]{offset} \hyperref[TEI.org]{org} \hyperref[TEI.orgName]{orgName} \hyperref[TEI.origDate]{origDate} \hyperref[TEI.origPlace]{origPlace} \hyperref[TEI.origin]{origin} \hyperref[TEI.persName]{persName} \hyperref[TEI.persPronouns]{persPronouns} \hyperref[TEI.person]{person} \hyperref[TEI.persona]{persona} \hyperref[TEI.place]{place} \hyperref[TEI.placeName]{placeName} \hyperref[TEI.population]{population} \hyperref[TEI.reg]{reg} \hyperref[TEI.relation]{relation} \hyperref[TEI.residence]{residence} \hyperref[TEI.secl]{secl} \hyperref[TEI.sex]{sex} \hyperref[TEI.socecStatus]{socecStatus} \hyperref[TEI.state]{state} \hyperref[TEI.supplied]{supplied} \hyperref[TEI.surplus]{surplus} \hyperref[TEI.terrain]{terrain} \hyperref[TEI.time]{time} \hyperref[TEI.trait]{trait} \hyperref[TEI.unclear]{unclear}
    \item[{Attributes}]
  Attributes\hfil\\[-10pt]\begin{sansreflist}
    \item[@evidence]
  indicates the nature of the evidence supporting the reliability or accuracy of the intervention or interpretation.
\begin{reflist}
    \item[{Status}]
  Optional
    \item[{Datatype}]
  1–∞ occurrences of \hyperref[TEI.teidata.enumerated]{teidata.enumerated} separated by whitespace
    \item[{Suggested values include:}]
  \begin{description}

\item[{internal}]there is internal evidence to support the intervention.
\item[{external}]there is external evidence to support the intervention.
\item[{conjecture}]the intervention or interpretation has been made by the editor, cataloguer, or scholar on the basis of their expertise.
\end{description} 
\end{reflist}  
    \item[@instant]
  indicates whether this is an instant revision or not.
\begin{reflist}
    \item[{Status}]
  Optional
    \item[{Datatype}]
  \hyperref[TEI.teidata.xTruthValue]{teidata.xTruthValue}
    \item[{Default}]
  false
\end{reflist}  
\end{sansreflist}  
    \item[{Note}]
  \par
The members of this attribute class are typically used to represent any kind of editorial intervention in a text, for example a correction or interpretation, or to date or localize manuscripts etc.\par
Each pointer on the {\itshape source} (if present) corresponding to a witness or witness group should reference a bibliographic citation such as a \hyperref[TEI.witness]{<witness>}, \hyperref[TEI.msDesc]{<msDesc>}, or \hyperref[TEI.bibl]{<bibl>} element, or another external bibliographic citation, documenting the source concerned.
\end{reflist}  
\begin{reflist}
\item[]\begin{specHead}{TEI.att.enjamb}{att.enjamb} (enjambement) provides an attribute which may be used to indicate enjambement of the parent element. [\textit{\hyperref[VESE]{6.2.\ Components of the Verse Line}}]\end{specHead} 
    \item[{Module}]
  verse — \hyperref[VE]{Verse}
    \item[{Members}]
  \hyperref[TEI.l]{l}
    \item[{Attributes}]
  Attributes\hfil\\[-10pt]\begin{sansreflist}
    \item[@enjamb]
  (enjambement) indicates that the end of a verse line is marked by enjambement.
\begin{reflist}
    \item[{Status}]
  Optional
    \item[{Datatype}]
  \hyperref[TEI.teidata.enumerated]{teidata.enumerated}
    \item[{Sample values include:}]
  \begin{description}

\item[{no}]the line is end-stopped
\item[{yes}]the line in question runs on into the next
\item[{weak}]the line is weakly enjambed
\item[{strong}]the line is strongly enjambed
\end{description} 
    \item[{Note}]
  \par
The usual practice will be to give the value ‘yes’ to this attribute when enjambement is being marked, or the values ‘weak’ and ‘strong’ if degrees of enjambement are of interest; if no value is given, however, the attribute does not default to a value of ‘no’; this allows the attribute to be omitted entirely when enjambement is not of particular interest.
\end{reflist}  
\end{sansreflist}  
\end{reflist}  
\begin{reflist}
\item[]\begin{specHead}{TEI.att.entryLike}{att.entryLike} provides an attribute used to distinguish different styles of dictionary entries. [\textit{\hyperref[DIBO]{9.1.\ Dictionary Body and Overall Structure}} \textit{\hyperref[DIEN]{9.2.\ The Structure of Dictionary Entries}}]\end{specHead} 
    \item[{Module}]
  dictionaries — \hyperref[DI]{Dictionaries}
    \item[{Members}]
  \hyperref[TEI.entry]{entry} \hyperref[TEI.entryFree]{entryFree} \hyperref[TEI.superEntry]{superEntry}
    \item[{Attributes}]
  Attributes\hyperref[TEI.att.typed]{att.typed} (\unusedattribute{type}, @subtype) \hfil\\[-10pt]\begin{sansreflist}
    \item[@type]
  indicates type of entry, in dictionaries with multiple types.
\begin{reflist}
    \item[{Status}]
  Optional
    \item[{Datatype}]
  \hyperref[TEI.teidata.enumerated]{teidata.enumerated}
    \item[{Suggested values include:}]
  \begin{description}

\item[{main}]a main entry (default).{[Default] }
\item[{hom}](homograph) groups information relating to one homograph within an entry.
\item[{xref}](cross reference) a reduced entry whose only function is to point to another main entry (e.g. for forms of an irregular verb or for variant spellings: \textit{was} pointing to \textit{be}, or \textit{esthete} to \textit{aesthete}).
\item[{affix}]an entry for a prefix, infix, or suffix.
\item[{abbr}](abbreviation) an entry for an abbreviation.
\item[{supplemental}]a supplemental entry (for use in dictionaries which issue supplements to their main work in which they include updated information about entries).
\item[{foreign}]an entry for a foreign word in a monolingual dictionary.
\end{description} 
\end{reflist}  
\end{sansreflist}  
    \item[{Note}]
  \par
The global {\itshape n} attribute may be used to encode the homograph numbers attached to entries for homographs.
\end{reflist}  
\begin{reflist}
\item[]\begin{specHead}{TEI.att.formula}{att.formula} provides attributes for defining a mathematical formula. [\textit{\hyperref[HDUDECL]{2.3.9.\ The Unit Declaration}}]\end{specHead} 
    \item[{Module}]
  tei — \hyperref[ST]{The TEI Infrastructure}
    \item[{Members}]
  \hyperref[TEI.conversion]{conversion}
    \item[{Attributes}]
  Attributes\hfil\\[-10pt]\begin{sansreflist}
    \item[@formula]
  A {\itshape formula} is provided to describe a mathematical calculation such as a conversion between measurement systems.
\begin{reflist}
    \item[{Status}]
  Optional
    \item[{Datatype}]
  \hyperref[TEI.teidata.xpath]{teidata.xpath}
\end{reflist}  
\end{sansreflist}  
    \item[{Example}]
  \leavevmode\bgroup\index{encodingDesc=<encodingDesc>|exampleindex}\index{unitDecl=<unitDecl>|exampleindex}\index{unitDef=<unitDef>|exampleindex}\index{type=@type!<unitDef>|exampleindex}\index{label=<label>|exampleindex}\index{placeName=<placeName>|exampleindex}\index{ref=@ref!<placeName>|exampleindex}\index{conversion=<conversion>|exampleindex}\index{fromUnit=@fromUnit!<conversion>|exampleindex}\index{toUnit=@toUnit!<conversion>|exampleindex}\index{formula=@formula!<conversion>|exampleindex}\index{desc=<desc>|exampleindex}\exampleFont \begin{shaded}\noindent\mbox{}{<\textbf{encodingDesc}>}\mbox{}\newline 
\hspace*{1em}{<\textbf{unitDecl}>}\mbox{}\newline 
\hspace*{1em}\hspace*{1em}{<\textbf{unitDef}\hspace*{1em}{xml:id}="{stadium}"\hspace*{1em}{type}="{linear}">}\mbox{}\newline 
\hspace*{1em}\hspace*{1em}\hspace*{1em}{<\textbf{label}>}stadium{</\textbf{label}>}\mbox{}\newline 
\hspace*{1em}\hspace*{1em}\hspace*{1em}{<\textbf{placeName}\hspace*{1em}{ref}="{\#rome}"/>}\mbox{}\newline 
\hspace*{1em}\hspace*{1em}\hspace*{1em}{<\textbf{conversion}\hspace*{1em}{fromUnit}="{\#pes}"\mbox{}\newline 
\hspace*{1em}\hspace*{1em}\hspace*{1em}\hspace*{1em}{toUnit}="{\#stadium}"\hspace*{1em}{formula}="{\$fromUnit * 625}"/>}\mbox{}\newline 
\hspace*{1em}\hspace*{1em}\hspace*{1em}{<\textbf{desc}>}The stadium was a Roman unit of linear measurement equivalent to 625 pedes, or Roman feet.{</\textbf{desc}>}\mbox{}\newline 
\hspace*{1em}\hspace*{1em}{</\textbf{unitDef}>}\mbox{}\newline 
\hspace*{1em}{</\textbf{unitDecl}>}\mbox{}\newline 
{</\textbf{encodingDesc}>}\end{shaded}\egroup 


    \item[{Example}]
  \leavevmode\bgroup\index{encodingDesc=<encodingDesc>|exampleindex}\index{unitDecl=<unitDecl>|exampleindex}\index{unitDef=<unitDef>|exampleindex}\index{type=@type!<unitDef>|exampleindex}\index{label=<label>|exampleindex}\index{conversion=<conversion>|exampleindex}\index{fromUnit=@fromUnit!<conversion>|exampleindex}\index{toUnit=@toUnit!<conversion>|exampleindex}\index{formula=@formula!<conversion>|exampleindex}\index{desc=<desc>|exampleindex}\index{unitDef=<unitDef>|exampleindex}\index{type=@type!<unitDef>|exampleindex}\index{label=<label>|exampleindex}\index{conversion=<conversion>|exampleindex}\index{fromUnit=@fromUnit!<conversion>|exampleindex}\index{toUnit=@toUnit!<conversion>|exampleindex}\index{formula=@formula!<conversion>|exampleindex}\index{desc=<desc>|exampleindex}\index{unitDef=<unitDef>|exampleindex}\index{type=@type!<unitDef>|exampleindex}\index{label=<label>|exampleindex}\index{conversion=<conversion>|exampleindex}\index{fromUnit=@fromUnit!<conversion>|exampleindex}\index{toUnit=@toUnit!<conversion>|exampleindex}\index{formula=@formula!<conversion>|exampleindex}\index{desc=<desc>|exampleindex}\exampleFont \begin{shaded}\noindent\mbox{}{<\textbf{encodingDesc}>}\mbox{}\newline 
\hspace*{1em}{<\textbf{unitDecl}>}\mbox{}\newline 
\hspace*{1em}\hspace*{1em}{<\textbf{unitDef}\hspace*{1em}{xml:id}="{wmw}"\hspace*{1em}{type}="{power}">}\mbox{}\newline 
\hspace*{1em}\hspace*{1em}\hspace*{1em}{<\textbf{label}>}whatmeworry{</\textbf{label}>}\mbox{}\newline 
\hspace*{1em}\hspace*{1em}\hspace*{1em}{<\textbf{conversion}\hspace*{1em}{fromUnit}="{\#hpk}"\mbox{}\newline 
\hspace*{1em}\hspace*{1em}\hspace*{1em}\hspace*{1em}{toUnit}="{\#wmw}"\hspace*{1em}{formula}="{\$fromUnit * 1}"/>}\mbox{}\newline 
\hspace*{1em}\hspace*{1em}\hspace*{1em}{<\textbf{desc}>}In the Potrzebie system of measures as introduced by Donald Knuth, the whatmeworry unit of power is equivalent to one hah per kovac.{</\textbf{desc}>}\mbox{}\newline 
\hspace*{1em}\hspace*{1em}{</\textbf{unitDef}>}\mbox{}\newline 
\hspace*{1em}\hspace*{1em}{<\textbf{unitDef}\hspace*{1em}{xml:id}="{kwmw}"\hspace*{1em}{type}="{power}">}\mbox{}\newline 
\hspace*{1em}\hspace*{1em}\hspace*{1em}{<\textbf{label}>}kilowhatmeworry{</\textbf{label}>}\mbox{}\newline 
\hspace*{1em}\hspace*{1em}\hspace*{1em}{<\textbf{conversion}\hspace*{1em}{fromUnit}="{\#wmw}"\mbox{}\newline 
\hspace*{1em}\hspace*{1em}\hspace*{1em}\hspace*{1em}{toUnit}="{\#kwmw}"\hspace*{1em}{formula}="{\$fromUnit div 1000}"/>}\mbox{}\newline 
\hspace*{1em}\hspace*{1em}\hspace*{1em}{<\textbf{desc}>}The kilowhatmeworry is equivalent to 1000 whatmeworries.{</\textbf{desc}>}\mbox{}\newline 
\hspace*{1em}\hspace*{1em}{</\textbf{unitDef}>}\mbox{}\newline 
\hspace*{1em}\hspace*{1em}{<\textbf{unitDef}\hspace*{1em}{xml:id}="{ap}"\hspace*{1em}{type}="{power}">}\mbox{}\newline 
\hspace*{1em}\hspace*{1em}\hspace*{1em}{<\textbf{label}>}kilowhatmeworry{</\textbf{label}>}\mbox{}\newline 
\hspace*{1em}\hspace*{1em}\hspace*{1em}{<\textbf{conversion}\hspace*{1em}{fromUnit}="{\#kwmw}"\mbox{}\newline 
\hspace*{1em}\hspace*{1em}\hspace*{1em}\hspace*{1em}{toUnit}="{\#ap}"\hspace*{1em}{formula}="{\$fromUnit div 100}"/>}\mbox{}\newline 
\hspace*{1em}\hspace*{1em}\hspace*{1em}{<\textbf{desc}>}One unit of aeolipower (A.P.) is equivalent to 100 kilowhatmeworries.{</\textbf{desc}>}\mbox{}\newline 
\hspace*{1em}\hspace*{1em}{</\textbf{unitDef}>}\mbox{}\newline 
\hspace*{1em}{</\textbf{unitDecl}>}\mbox{}\newline 
{</\textbf{encodingDesc}>}\end{shaded}\egroup 


    \item[{Example}]
  \leavevmode\bgroup\index{conversion=<conversion>|exampleindex}\index{fromUnit=@fromUnit!<conversion>|exampleindex}\index{toUnit=@toUnit!<conversion>|exampleindex}\index{formula=@formula!<conversion>|exampleindex}\exampleFont \begin{shaded}\noindent\mbox{}{<\textbf{conversion}\hspace*{1em}{fromUnit}="{\#furlongsPerFortnight}"\mbox{}\newline 
\hspace*{1em}{toUnit}="{\#milesPerHour}"\mbox{}\newline 
\hspace*{1em}{formula}="{\$fromUnit cast as xs:decimal * 0.000372}"/>}\end{shaded}\egroup 


    \item[{Example}]
  \leavevmode\bgroup\index{conversion=<conversion>|exampleindex}\index{fromUnit=@fromUnit!<conversion>|exampleindex}\index{toUnit=@toUnit!<conversion>|exampleindex}\index{formula=@formula!<conversion>|exampleindex}\exampleFont \begin{shaded}\noindent\mbox{}{<\textbf{conversion}\hspace*{1em}{fromUnit}="{\#deciday}"\mbox{}\newline 
\hspace*{1em}{toUnit}="{hour}"\mbox{}\newline 
\hspace*{1em}{formula}="{\$fromUnit cast as xs:decimal * 144 div 60}"/>}\end{shaded}\egroup 


    \item[{Note}]
  \par
This attribute class provides {\itshape formula} for use in defining a value used in mathematical calculation. It can be used to store a mathematical operation needed to convert from one system of measurement to another. We use the teidata.xpath datatype to express this value in order to communicate mathematical operations on an XML node or nodes. The \$fromUnit variable notation simplifies referencing of the {\itshape fromUnit} attribute on the parent \hyperref[TEI.conversion]{<conversion>} element. Note that ‘div’ is required to express the division operator in XPath.
\end{reflist}  
\begin{reflist}
\item[]\begin{specHead}{TEI.att.fragmentable}{att.fragmentable} provides an attribute for representing fragmentation of a structural element, typically as a consequence of some overlapping hierarchy.\end{specHead} 
    \item[{Module}]
  tei — \hyperref[ST]{The TEI Infrastructure}
    \item[{Members}]
  \hyperref[TEI.att.divLike]{att.divLike}[\hyperref[TEI.div]{div} \hyperref[TEI.div1]{div1} \hyperref[TEI.div2]{div2} \hyperref[TEI.div3]{div3} \hyperref[TEI.div4]{div4} \hyperref[TEI.div5]{div5} \hyperref[TEI.div6]{div6} \hyperref[TEI.div7]{div7} \hyperref[TEI.lg]{lg}] \hyperref[TEI.att.segLike]{att.segLike}[\hyperref[TEI.c]{c} \hyperref[TEI.cl]{cl} \hyperref[TEI.m]{m} \hyperref[TEI.pc]{pc} \hyperref[TEI.phr]{phr} \hyperref[TEI.s]{s} \hyperref[TEI.seg]{seg} \hyperref[TEI.w]{w}] \hyperref[TEI.ab]{ab} \hyperref[TEI.l]{l} \hyperref[TEI.p]{p}
    \item[{Attributes}]
  Attributes\hfil\\[-10pt]\begin{sansreflist}
    \item[@part]
  specifies whether or not its parent element is fragmented in some way, typically by some other overlapping structure: for example a speech which is divided between two or more verse stanzas, a paragraph which is split across a page division, a verse line which is divided between two speakers.
\begin{reflist}
    \item[{Status}]
  Optional
    \item[{Datatype}]
  \hyperref[TEI.teidata.enumerated]{teidata.enumerated}
    \item[{Legal values are:}]
  \begin{description}

\item[{Y}](yes) the element is fragmented in some (unspecified) respect
\item[{N}](no) the element is not fragmented, or no claim is made as to its completeness{[Default] }
\item[{I}](initial) this is the initial part of a fragmented element
\item[{M}](medial) this is a medial part of a fragmented element
\item[{F}](final) this is the final part of a fragmented element
\end{description} 
    \item[{Note}]
  \par
The values I, M, or F should be used only where it is clear how the element may be reconstituted.
\end{reflist}  
\end{sansreflist}  
\end{reflist}  
\begin{reflist}
\item[]\begin{specHead}{TEI.att.gaijiProp}{att.gaijiProp} provides attributes for defining the properties of non-standard characters or glyphs.  [\textit{\hyperref[WD]{5.\ Characters, Glyphs, and Writing Modes}}]\end{specHead} 
    \item[{Module}]
  gaiji — \hyperref[WD]{Characters, Glyphs, and Writing Modes}
    \item[{Members}]
  \hyperref[TEI.localProp]{localProp} \hyperref[TEI.unicodeProp]{unicodeProp} \hyperref[TEI.unihanProp]{unihanProp}
    \item[{Attributes}]
  Attributes\hfil\\[-10pt]\begin{sansreflist}
    \item[@name]
  provides the name of the character or glyph property being defined.
\begin{reflist}
    \item[{Status}]
  Required
    \item[{Datatype}]
  \hyperref[TEI.teidata.xmlName]{teidata.xmlName}
\end{reflist}  
    \item[@value]
  provides the value of the character or glyph property being defined.
\begin{reflist}
    \item[{Status}]
  Required
    \item[{Datatype}]
  \hyperref[TEI.teidata.text]{teidata.text}
\end{reflist}  
    \item[@version]
  specifies the version number of the Unicode Standard in which this property name is defined.
\begin{reflist}
    \item[{Status}]
  Optional
    \item[{Datatype}]
  \hyperref[TEI.teidata.enumerated]{teidata.enumerated}
    \item[{Suggested values include:}]
  \begin{description}

\item[{1.0.1}]
\item[{1.1}]
\item[{2.0}]
\item[{2.1}]
\item[{3.0}]
\item[{3.1}]
\item[{3.2}]
\item[{4.0}]
\item[{4.1}]
\item[{5.0}]
\item[{5.1}]
\item[{5.2}]
\item[{6.0}]
\item[{6.1}]
\item[{6.2}]
\item[{6.3}]
\item[{7.0}]
\item[{8.0}]
\item[{9.0}]
\item[{10.0}]
\item[{11.0}]
\item[{12.0}]
\item[{12.1}]
\item[{unassigned}]
\end{description} 
\end{reflist}  
\end{sansreflist}  
    \item[{Example}]
  In this example a definition for the Unicode property  {\name Decomposition Mapping} is provided.\leavevmode\bgroup\index{unicodeProp=<unicodeProp>|exampleindex}\index{name=@name!<unicodeProp>|exampleindex}\index{value=@value!<unicodeProp>|exampleindex}\exampleFont \begin{shaded}\noindent\mbox{}{<\textbf{unicodeProp}\hspace*{1em}{name}="{Decomposition\textunderscore Mapping}"\mbox{}\newline 
\hspace*{1em}{value}="{circle}"/>}\end{shaded}\egroup 


    \item[{Note}]
  \par
All name-only attributes need an xs:boolean attribute value inside {\itshape value}.
\end{reflist}  
\begin{reflist}
\item[]\begin{specHead}{TEI.att.global}{att.global} provides attributes common to all elements in the TEI encoding scheme. [\textit{\hyperref[STGA]{1.3.1.1.\ Global Attributes}}]\end{specHead} 
    \item[{Module}]
  tei — \hyperref[ST]{The TEI Infrastructure}
    \item[{Members}]
  \hyperref[TEI.TEI]{TEI} \hyperref[TEI.ab]{ab} \hyperref[TEI.abbr]{abbr} \hyperref[TEI.abstract]{abstract} \hyperref[TEI.accMat]{accMat} \hyperref[TEI.acquisition]{acquisition} \hyperref[TEI.activity]{activity} \hyperref[TEI.actor]{actor} \hyperref[TEI.add]{add} \hyperref[TEI.addName]{addName} \hyperref[TEI.addSpan]{addSpan} \hyperref[TEI.additional]{additional} \hyperref[TEI.additions]{additions} \hyperref[TEI.addrLine]{addrLine} \hyperref[TEI.address]{address} \hyperref[TEI.adminInfo]{adminInfo} \hyperref[TEI.affiliation]{affiliation} \hyperref[TEI.age]{age} \hyperref[TEI.alt]{alt} \hyperref[TEI.altGrp]{altGrp} \hyperref[TEI.altIdent]{altIdent} \hyperref[TEI.altIdentifier]{altIdentifier} \hyperref[TEI.alternate]{alternate} \hyperref[TEI.am]{am} \hyperref[TEI.analytic]{analytic} \hyperref[TEI.anchor]{anchor} \hyperref[TEI.annotation]{annotation} \hyperref[TEI.annotationBlock]{annotationBlock} \hyperref[TEI.anyElement]{anyElement} \hyperref[TEI.app]{app} \hyperref[TEI.appInfo]{appInfo} \hyperref[TEI.application]{application} \hyperref[TEI.arc]{arc} \hyperref[TEI.argument]{argument} \hyperref[TEI.att]{att} \hyperref[TEI.attDef]{attDef} \hyperref[TEI.attList]{attList} \hyperref[TEI.attRef]{attRef} \hyperref[TEI.author]{author} \hyperref[TEI.authority]{authority} \hyperref[TEI.availability]{availability} \hyperref[TEI.back]{back} \hyperref[TEI.bibl]{bibl} \hyperref[TEI.biblFull]{biblFull} \hyperref[TEI.biblScope]{biblScope} \hyperref[TEI.biblStruct]{biblStruct} \hyperref[TEI.bicond]{bicond} \hyperref[TEI.binary]{binary} \hyperref[TEI.binaryObject]{binaryObject} \hyperref[TEI.binding]{binding} \hyperref[TEI.bindingDesc]{bindingDesc} \hyperref[TEI.birth]{birth} \hyperref[TEI.bloc]{bloc} \hyperref[TEI.body]{body} \hyperref[TEI.broadcast]{broadcast} \hyperref[TEI.byline]{byline} \hyperref[TEI.c]{c} \hyperref[TEI.cRefPattern]{cRefPattern} \hyperref[TEI.caesura]{caesura} \hyperref[TEI.calendar]{calendar} \hyperref[TEI.calendarDesc]{calendarDesc} \hyperref[TEI.camera]{camera} \hyperref[TEI.caption]{caption} \hyperref[TEI.case]{case} \hyperref[TEI.castGroup]{castGroup} \hyperref[TEI.castItem]{castItem} \hyperref[TEI.castList]{castList} \hyperref[TEI.catDesc]{catDesc} \hyperref[TEI.catRef]{catRef} \hyperref[TEI.catchwords]{catchwords} \hyperref[TEI.category]{category} \hyperref[TEI.cb]{cb} \hyperref[TEI.cell]{cell} \hyperref[TEI.certainty]{certainty} \hyperref[TEI.change]{change} \hyperref[TEI.channel]{channel} \hyperref[TEI.char]{char} \hyperref[TEI.charDecl]{charDecl} \hyperref[TEI.charName]{charName} \hyperref[TEI.charProp]{charProp} \hyperref[TEI.choice]{choice} \hyperref[TEI.cit]{cit} \hyperref[TEI.citeData]{citeData} \hyperref[TEI.citeStructure]{citeStructure} \hyperref[TEI.citedRange]{citedRange} \hyperref[TEI.cl]{cl} \hyperref[TEI.classCode]{classCode} \hyperref[TEI.classDecl]{classDecl} \hyperref[TEI.classRef]{classRef} \hyperref[TEI.classSpec]{classSpec} \hyperref[TEI.classes]{classes} \hyperref[TEI.climate]{climate} \hyperref[TEI.closer]{closer} \hyperref[TEI.code]{code} \hyperref[TEI.collation]{collation} \hyperref[TEI.collection]{collection} \hyperref[TEI.colloc]{colloc} \hyperref[TEI.colophon]{colophon} \hyperref[TEI.cond]{cond} \hyperref[TEI.condition]{condition} \hyperref[TEI.constitution]{constitution} \hyperref[TEI.constraint]{constraint} \hyperref[TEI.constraintSpec]{constraintSpec} \hyperref[TEI.content]{content} \hyperref[TEI.conversion]{conversion} \hyperref[TEI.corr]{corr} \hyperref[TEI.correction]{correction} \hyperref[TEI.correspAction]{correspAction} \hyperref[TEI.correspContext]{correspContext} \hyperref[TEI.correspDesc]{correspDesc} \hyperref[TEI.country]{country} \hyperref[TEI.creation]{creation} \hyperref[TEI.custEvent]{custEvent} \hyperref[TEI.custodialHist]{custodialHist} \hyperref[TEI.damage]{damage} \hyperref[TEI.damageSpan]{damageSpan} \hyperref[TEI.dataFacet]{dataFacet} \hyperref[TEI.dataRef]{dataRef} \hyperref[TEI.dataSpec]{dataSpec} \hyperref[TEI.datatype]{datatype} \hyperref[TEI.date]{date} \hyperref[TEI.dateline]{dateline} \hyperref[TEI.death]{death} \hyperref[TEI.decoDesc]{decoDesc} \hyperref[TEI.decoNote]{decoNote} \hyperref[TEI.def]{def} \hyperref[TEI.default]{default} \hyperref[TEI.defaultVal]{defaultVal} \hyperref[TEI.del]{del} \hyperref[TEI.delSpan]{delSpan} \hyperref[TEI.depth]{depth} \hyperref[TEI.derivation]{derivation} \hyperref[TEI.desc]{desc} \hyperref[TEI.dictScrap]{dictScrap} \hyperref[TEI.dim]{dim} \hyperref[TEI.dimensions]{dimensions} \hyperref[TEI.distinct]{distinct} \hyperref[TEI.distributor]{distributor} \hyperref[TEI.district]{district} \hyperref[TEI.div]{div} \hyperref[TEI.div1]{div1} \hyperref[TEI.div2]{div2} \hyperref[TEI.div3]{div3} \hyperref[TEI.div4]{div4} \hyperref[TEI.div5]{div5} \hyperref[TEI.div6]{div6} \hyperref[TEI.div7]{div7} \hyperref[TEI.divGen]{divGen} \hyperref[TEI.docAuthor]{docAuthor} \hyperref[TEI.docDate]{docDate} \hyperref[TEI.docEdition]{docEdition} \hyperref[TEI.docImprint]{docImprint} \hyperref[TEI.docTitle]{docTitle} \hyperref[TEI.domain]{domain} \hyperref[TEI.eLeaf]{eLeaf} \hyperref[TEI.eTree]{eTree} \hyperref[TEI.edition]{edition} \hyperref[TEI.editionStmt]{editionStmt} \hyperref[TEI.editor]{editor} \hyperref[TEI.editorialDecl]{editorialDecl} \hyperref[TEI.education]{education} \hyperref[TEI.eg]{eg} \hyperref[TEI.egXML]{egXML} \hyperref[TEI.elementRef]{elementRef} \hyperref[TEI.elementSpec]{elementSpec} \hyperref[TEI.email]{email} \hyperref[TEI.emph]{emph} \hyperref[TEI.empty]{empty} \hyperref[TEI.encodingDesc]{encodingDesc} \hyperref[TEI.entry]{entry} \hyperref[TEI.entryFree]{entryFree} \hyperref[TEI.epigraph]{epigraph} \hyperref[TEI.epilogue]{epilogue} \hyperref[TEI.equipment]{equipment} \hyperref[TEI.equiv]{equiv} \hyperref[TEI.etym]{etym} \hyperref[TEI.event]{event} \hyperref[TEI.ex]{ex} \hyperref[TEI.exemplum]{exemplum} \hyperref[TEI.expan]{expan} \hyperref[TEI.explicit]{explicit} \hyperref[TEI.extent]{extent} \hyperref[TEI.f]{f} \hyperref[TEI.fDecl]{fDecl} \hyperref[TEI.fDescr]{fDescr} \hyperref[TEI.fLib]{fLib} \hyperref[TEI.facsimile]{facsimile} \hyperref[TEI.factuality]{factuality} \hyperref[TEI.faith]{faith} \hyperref[TEI.figDesc]{figDesc} \hyperref[TEI.figure]{figure} \hyperref[TEI.fileDesc]{fileDesc} \hyperref[TEI.filiation]{filiation} \hyperref[TEI.finalRubric]{finalRubric} \hyperref[TEI.floatingText]{floatingText} \hyperref[TEI.floruit]{floruit} \hyperref[TEI.foliation]{foliation} \hyperref[TEI.foreign]{foreign} \hyperref[TEI.forename]{forename} \hyperref[TEI.forest]{forest} \hyperref[TEI.form]{form} \hyperref[TEI.formula]{formula} \hyperref[TEI.front]{front} \hyperref[TEI.fs]{fs} \hyperref[TEI.fsConstraints]{fsConstraints} \hyperref[TEI.fsDecl]{fsDecl} \hyperref[TEI.fsDescr]{fsDescr} \hyperref[TEI.fsdDecl]{fsdDecl} \hyperref[TEI.fsdLink]{fsdLink} \hyperref[TEI.funder]{funder} \hyperref[TEI.fvLib]{fvLib} \hyperref[TEI.fw]{fw} \hyperref[TEI.g]{g} \hyperref[TEI.gap]{gap} \hyperref[TEI.gb]{gb} \hyperref[TEI.gen]{gen} \hyperref[TEI.genName]{genName} \hyperref[TEI.geo]{geo} \hyperref[TEI.geoDecl]{geoDecl} \hyperref[TEI.geogFeat]{geogFeat} \hyperref[TEI.geogName]{geogName} \hyperref[TEI.gi]{gi} \hyperref[TEI.gloss]{gloss} \hyperref[TEI.glyph]{glyph} \hyperref[TEI.glyphName]{glyphName} \hyperref[TEI.gram]{gram} \hyperref[TEI.gramGrp]{gramGrp} \hyperref[TEI.graph]{graph} \hyperref[TEI.graphic]{graphic} \hyperref[TEI.group]{group} \hyperref[TEI.handDesc]{handDesc} \hyperref[TEI.handNote]{handNote} \hyperref[TEI.handNotes]{handNotes} \hyperref[TEI.handShift]{handShift} \hyperref[TEI.head]{head} \hyperref[TEI.headItem]{headItem} \hyperref[TEI.headLabel]{headLabel} \hyperref[TEI.height]{height} \hyperref[TEI.heraldry]{heraldry} \hyperref[TEI.hi]{hi} \hyperref[TEI.history]{history} \hyperref[TEI.hom]{hom} \hyperref[TEI.hyph]{hyph} \hyperref[TEI.hyphenation]{hyphenation} \hyperref[TEI.iNode]{iNode} \hyperref[TEI.iType]{iType} \hyperref[TEI.ident]{ident} \hyperref[TEI.idno]{idno} \hyperref[TEI.if]{if} \hyperref[TEI.iff]{iff} \hyperref[TEI.imprimatur]{imprimatur} \hyperref[TEI.imprint]{imprint} \hyperref[TEI.incident]{incident} \hyperref[TEI.incipit]{incipit} \hyperref[TEI.index]{index} \hyperref[TEI.institution]{institution} \hyperref[TEI.interaction]{interaction} \hyperref[TEI.interp]{interp} \hyperref[TEI.interpGrp]{interpGrp} \hyperref[TEI.interpretation]{interpretation} \hyperref[TEI.item]{item} \hyperref[TEI.join]{join} \hyperref[TEI.joinGrp]{joinGrp} \hyperref[TEI.keywords]{keywords} \hyperref[TEI.kinesic]{kinesic} \hyperref[TEI.l]{l} \hyperref[TEI.label]{label} \hyperref[TEI.lacunaEnd]{lacunaEnd} \hyperref[TEI.lacunaStart]{lacunaStart} \hyperref[TEI.lang]{lang} \hyperref[TEI.langKnowledge]{langKnowledge} \hyperref[TEI.langKnown]{langKnown} \hyperref[TEI.langUsage]{langUsage} \hyperref[TEI.language]{language} \hyperref[TEI.layout]{layout} \hyperref[TEI.layoutDesc]{layoutDesc} \hyperref[TEI.lb]{lb} \hyperref[TEI.lbl]{lbl} \hyperref[TEI.leaf]{leaf} \hyperref[TEI.lem]{lem} \hyperref[TEI.lg]{lg} \hyperref[TEI.licence]{licence} \hyperref[TEI.line]{line} \hyperref[TEI.link]{link} \hyperref[TEI.linkGrp]{linkGrp} \hyperref[TEI.list]{list} \hyperref[TEI.listAnnotation]{listAnnotation} \hyperref[TEI.listApp]{listApp} \hyperref[TEI.listBibl]{listBibl} \hyperref[TEI.listChange]{listChange} \hyperref[TEI.listEvent]{listEvent} \hyperref[TEI.listForest]{listForest} \hyperref[TEI.listNym]{listNym} \hyperref[TEI.listObject]{listObject} \hyperref[TEI.listOrg]{listOrg} \hyperref[TEI.listPerson]{listPerson} \hyperref[TEI.listPlace]{listPlace} \hyperref[TEI.listPrefixDef]{listPrefixDef} \hyperref[TEI.listRef]{listRef} \hyperref[TEI.listRelation]{listRelation} \hyperref[TEI.listTranspose]{listTranspose} \hyperref[TEI.listWit]{listWit} \hyperref[TEI.localName]{localName} \hyperref[TEI.localProp]{localProp} \hyperref[TEI.locale]{locale} \hyperref[TEI.location]{location} \hyperref[TEI.locus]{locus} \hyperref[TEI.locusGrp]{locusGrp} \hyperref[TEI.m]{m} \hyperref[TEI.macroRef]{macroRef} \hyperref[TEI.macroSpec]{macroSpec} \hyperref[TEI.mapping]{mapping} \hyperref[TEI.material]{material} \hyperref[TEI.measure]{measure} \hyperref[TEI.measureGrp]{measureGrp} \hyperref[TEI.media]{media} \hyperref[TEI.meeting]{meeting} \hyperref[TEI.memberOf]{memberOf} \hyperref[TEI.mentioned]{mentioned} \hyperref[TEI.metDecl]{metDecl} \hyperref[TEI.metSym]{metSym} \hyperref[TEI.metamark]{metamark} \hyperref[TEI.milestone]{milestone} \hyperref[TEI.mod]{mod} \hyperref[TEI.model]{model} \hyperref[TEI.modelGrp]{modelGrp} \hyperref[TEI.modelSequence]{modelSequence} \hyperref[TEI.moduleRef]{moduleRef} \hyperref[TEI.moduleSpec]{moduleSpec} \hyperref[TEI.monogr]{monogr} \hyperref[TEI.mood]{mood} \hyperref[TEI.move]{move} \hyperref[TEI.msContents]{msContents} \hyperref[TEI.msDesc]{msDesc} \hyperref[TEI.msFrag]{msFrag} \hyperref[TEI.msIdentifier]{msIdentifier} \hyperref[TEI.msItem]{msItem} \hyperref[TEI.msItemStruct]{msItemStruct} \hyperref[TEI.msName]{msName} \hyperref[TEI.msPart]{msPart} \hyperref[TEI.musicNotation]{musicNotation} \hyperref[TEI.name]{name} \hyperref[TEI.nameLink]{nameLink} \hyperref[TEI.namespace]{namespace} \hyperref[TEI.nationality]{nationality} \hyperref[TEI.node]{node} \hyperref[TEI.normalization]{normalization} \hyperref[TEI.notatedMusic]{notatedMusic} \hyperref[TEI.note]{note} \hyperref[TEI.noteGrp]{noteGrp} \hyperref[TEI.notesStmt]{notesStmt} \hyperref[TEI.num]{num} \hyperref[TEI.number]{number} \hyperref[TEI.numeric]{numeric} \hyperref[TEI.nym]{nym} \hyperref[TEI.oRef]{oRef} \hyperref[TEI.object]{object} \hyperref[TEI.objectDesc]{objectDesc} \hyperref[TEI.objectIdentifier]{objectIdentifier} \hyperref[TEI.objectName]{objectName} \hyperref[TEI.objectType]{objectType} \hyperref[TEI.occupation]{occupation} \hyperref[TEI.offset]{offset} \hyperref[TEI.opener]{opener} \hyperref[TEI.org]{org} \hyperref[TEI.orgName]{orgName} \hyperref[TEI.orig]{orig} \hyperref[TEI.origDate]{origDate} \hyperref[TEI.origPlace]{origPlace} \hyperref[TEI.origin]{origin} \hyperref[TEI.orth]{orth} \hyperref[TEI.outputRendition]{outputRendition} \hyperref[TEI.p]{p} \hyperref[TEI.pRef]{pRef} \hyperref[TEI.param]{param} \hyperref[TEI.paramList]{paramList} \hyperref[TEI.paramSpec]{paramSpec} \hyperref[TEI.particDesc]{particDesc} \hyperref[TEI.path]{path} \hyperref[TEI.pause]{pause} \hyperref[TEI.pb]{pb} \hyperref[TEI.pc]{pc} \hyperref[TEI.per]{per} \hyperref[TEI.performance]{performance} \hyperref[TEI.persName]{persName} \hyperref[TEI.persPronouns]{persPronouns} \hyperref[TEI.person]{person} \hyperref[TEI.personGrp]{personGrp} \hyperref[TEI.persona]{persona} \hyperref[TEI.phr]{phr} \hyperref[TEI.physDesc]{physDesc} \hyperref[TEI.place]{place} \hyperref[TEI.placeName]{placeName} \hyperref[TEI.population]{population} \hyperref[TEI.pos]{pos} \hyperref[TEI.postBox]{postBox} \hyperref[TEI.postCode]{postCode} \hyperref[TEI.postscript]{postscript} \hyperref[TEI.precision]{precision} \hyperref[TEI.prefixDef]{prefixDef} \hyperref[TEI.preparedness]{preparedness} \hyperref[TEI.principal]{principal} \hyperref[TEI.profileDesc]{profileDesc} \hyperref[TEI.projectDesc]{projectDesc} \hyperref[TEI.prologue]{prologue} \hyperref[TEI.pron]{pron} \hyperref[TEI.provenance]{provenance} \hyperref[TEI.ptr]{ptr} \hyperref[TEI.pubPlace]{pubPlace} \hyperref[TEI.publicationStmt]{publicationStmt} \hyperref[TEI.publisher]{publisher} \hyperref[TEI.punctuation]{punctuation} \hyperref[TEI.purpose]{purpose} \hyperref[TEI.q]{q} \hyperref[TEI.quotation]{quotation} \hyperref[TEI.quote]{quote} \hyperref[TEI.rb]{rb} \hyperref[TEI.rdg]{rdg} \hyperref[TEI.rdgGrp]{rdgGrp} \hyperref[TEI.re]{re} \hyperref[TEI.recordHist]{recordHist} \hyperref[TEI.recording]{recording} \hyperref[TEI.recordingStmt]{recordingStmt} \hyperref[TEI.redo]{redo} \hyperref[TEI.ref]{ref} \hyperref[TEI.refState]{refState} \hyperref[TEI.refsDecl]{refsDecl} \hyperref[TEI.reg]{reg} \hyperref[TEI.region]{region} \hyperref[TEI.relatedItem]{relatedItem} \hyperref[TEI.relation]{relation} \hyperref[TEI.remarks]{remarks} \hyperref[TEI.rendition]{rendition} \hyperref[TEI.repository]{repository} \hyperref[TEI.residence]{residence} \hyperref[TEI.resp]{resp} \hyperref[TEI.respStmt]{respStmt} \hyperref[TEI.respons]{respons} \hyperref[TEI.restore]{restore} \hyperref[TEI.retrace]{retrace} \hyperref[TEI.revisionDesc]{revisionDesc} \hyperref[TEI.rhyme]{rhyme} \hyperref[TEI.role]{role} \hyperref[TEI.roleDesc]{roleDesc} \hyperref[TEI.roleName]{roleName} \hyperref[TEI.root]{root} \hyperref[TEI.row]{row} \hyperref[TEI.rs]{rs} \hyperref[TEI.rt]{rt} \hyperref[TEI.rubric]{rubric} \hyperref[TEI.ruby]{ruby} \hyperref[TEI.s]{s} \hyperref[TEI.said]{said} \hyperref[TEI.salute]{salute} \hyperref[TEI.samplingDecl]{samplingDecl} \hyperref[TEI.schemaRef]{schemaRef} \hyperref[TEI.schemaSpec]{schemaSpec} \hyperref[TEI.scriptDesc]{scriptDesc} \hyperref[TEI.scriptNote]{scriptNote} \hyperref[TEI.scriptStmt]{scriptStmt} \hyperref[TEI.seal]{seal} \hyperref[TEI.sealDesc]{sealDesc} \hyperref[TEI.secFol]{secFol} \hyperref[TEI.secl]{secl} \hyperref[TEI.seg]{seg} \hyperref[TEI.segmentation]{segmentation} \hyperref[TEI.sense]{sense} \hyperref[TEI.sequence]{sequence} \hyperref[TEI.series]{series} \hyperref[TEI.seriesStmt]{seriesStmt} \hyperref[TEI.set]{set} \hyperref[TEI.setting]{setting} \hyperref[TEI.settingDesc]{settingDesc} \hyperref[TEI.settlement]{settlement} \hyperref[TEI.sex]{sex} \hyperref[TEI.shift]{shift} \hyperref[TEI.sic]{sic} \hyperref[TEI.signatures]{signatures} \hyperref[TEI.signed]{signed} \hyperref[TEI.soCalled]{soCalled} \hyperref[TEI.socecStatus]{socecStatus} \hyperref[TEI.sound]{sound} \hyperref[TEI.source]{source} \hyperref[TEI.sourceDesc]{sourceDesc} \hyperref[TEI.sourceDoc]{sourceDoc} \hyperref[TEI.sp]{sp} \hyperref[TEI.spGrp]{spGrp} \hyperref[TEI.space]{space} \hyperref[TEI.span]{span} \hyperref[TEI.spanGrp]{spanGrp} \hyperref[TEI.speaker]{speaker} \hyperref[TEI.specDesc]{specDesc} \hyperref[TEI.specGrp]{specGrp} \hyperref[TEI.specGrpRef]{specGrpRef} \hyperref[TEI.specList]{specList} \hyperref[TEI.sponsor]{sponsor} \hyperref[TEI.stage]{stage} \hyperref[TEI.stamp]{stamp} \hyperref[TEI.standOff]{standOff} \hyperref[TEI.state]{state} \hyperref[TEI.stdVals]{stdVals} \hyperref[TEI.street]{street} \hyperref[TEI.stress]{stress} \hyperref[TEI.string]{string} \hyperref[TEI.styleDefDecl]{styleDefDecl} \hyperref[TEI.subc]{subc} \hyperref[TEI.subst]{subst} \hyperref[TEI.substJoin]{substJoin} \hyperref[TEI.summary]{summary} \hyperref[TEI.superEntry]{superEntry} \hyperref[TEI.supplied]{supplied} \hyperref[TEI.support]{support} \hyperref[TEI.supportDesc]{supportDesc} \hyperref[TEI.surface]{surface} \hyperref[TEI.surfaceGrp]{surfaceGrp} \hyperref[TEI.surname]{surname} \hyperref[TEI.surplus]{surplus} \hyperref[TEI.surrogates]{surrogates} \hyperref[TEI.syll]{syll} \hyperref[TEI.symbol]{symbol} \hyperref[TEI.table]{table} \hyperref[TEI.tag]{tag} \hyperref[TEI.tagUsage]{tagUsage} \hyperref[TEI.tagsDecl]{tagsDecl} \hyperref[TEI.taxonomy]{taxonomy} \hyperref[TEI.tech]{tech} \hyperref[TEI.teiCorpus]{teiCorpus} \hyperref[TEI.teiHeader]{teiHeader} \hyperref[TEI.term]{term} \hyperref[TEI.terrain]{terrain} \hyperref[TEI.text]{text} \hyperref[TEI.textClass]{textClass} \hyperref[TEI.textDesc]{textDesc} \hyperref[TEI.textLang]{textLang} \hyperref[TEI.textNode]{textNode} \hyperref[TEI.then]{then} \hyperref[TEI.time]{time} \hyperref[TEI.timeline]{timeline} \hyperref[TEI.title]{title} \hyperref[TEI.titlePage]{titlePage} \hyperref[TEI.titlePart]{titlePart} \hyperref[TEI.titleStmt]{titleStmt} \hyperref[TEI.tns]{tns} \hyperref[TEI.trailer]{trailer} \hyperref[TEI.trait]{trait} \hyperref[TEI.transcriptionDesc]{transcriptionDesc} \hyperref[TEI.transpose]{transpose} \hyperref[TEI.tree]{tree} \hyperref[TEI.triangle]{triangle} \hyperref[TEI.typeDesc]{typeDesc} \hyperref[TEI.typeNote]{typeNote} \hyperref[TEI.u]{u} \hyperref[TEI.unclear]{unclear} \hyperref[TEI.undo]{undo} \hyperref[TEI.unicodeName]{unicodeName} \hyperref[TEI.unicodeProp]{unicodeProp} \hyperref[TEI.unihanProp]{unihanProp} \hyperref[TEI.unit]{unit} \hyperref[TEI.unitDecl]{unitDecl} \hyperref[TEI.unitDef]{unitDef} \hyperref[TEI.usg]{usg} \hyperref[TEI.vAlt]{vAlt} \hyperref[TEI.vColl]{vColl} \hyperref[TEI.vDefault]{vDefault} \hyperref[TEI.vLabel]{vLabel} \hyperref[TEI.vMerge]{vMerge} \hyperref[TEI.vNot]{vNot} \hyperref[TEI.vRange]{vRange} \hyperref[TEI.val]{val} \hyperref[TEI.valDesc]{valDesc} \hyperref[TEI.valItem]{valItem} \hyperref[TEI.valList]{valList} \hyperref[TEI.value]{value} \hyperref[TEI.variantEncoding]{variantEncoding} \hyperref[TEI.view]{view} \hyperref[TEI.vocal]{vocal} \hyperref[TEI.w]{w} \hyperref[TEI.watermark]{watermark} \hyperref[TEI.when]{when} \hyperref[TEI.width]{width} \hyperref[TEI.wit]{wit} \hyperref[TEI.witDetail]{witDetail} \hyperref[TEI.witEnd]{witEnd} \hyperref[TEI.witStart]{witStart} \hyperref[TEI.witness]{witness} \hyperref[TEI.writing]{writing} \hyperref[TEI.xenoData]{xenoData} \hyperref[TEI.xr]{xr} \hyperref[TEI.zone]{zone}
    \item[{Attributes}]
  \hyperref[TEI.att.global.rendition]{att.global.rendition} (\textit{@rend}, \textit{@style}, \textit{@rendition}) \hyperref[TEI.att.global.linking]{att.global.linking} (\textit{@corresp}, \textit{@synch}, \textit{@sameAs}, \textit{@copyOf}, \textit{@next}, \textit{@prev}, \textit{@exclude}, \textit{@select}) \hyperref[TEI.att.global.analytic]{att.global.analytic} (\textit{@ana}) \hyperref[TEI.att.global.facs]{att.global.facs} (\textit{@facs}) \hyperref[TEI.att.global.change]{att.global.change} (\textit{@change}) \hyperref[TEI.att.global.responsibility]{att.global.responsibility} (\textit{@cert}, \textit{@resp}) \hyperref[TEI.att.global.source]{att.global.source} (\textit{@source}) \hfil\\[-10pt]\begin{sansreflist}
    \item[@xml:id]
  (identifier) provides a unique identifier for the element bearing the attribute.
\begin{reflist}
    \item[{Status}]
  Optional
    \item[{Datatype}]
  \xref{https://www.w3.org/TR/xmlschema-2/\#ID}{ID}
    \item[{Note}]
  \par
The {\itshape xml:id} attribute may be used to specify a canonical reference for an element; see section \textit{\hyperref[CORS]{3.11.\ Reference Systems}}.
\end{reflist}  
    \item[@n]
  (number) gives a number (or other label) for an element, which is not necessarily unique within the document.
\begin{reflist}
    \item[{Status}]
  Optional
    \item[{Datatype}]
  \hyperref[TEI.teidata.text]{teidata.text}
    \item[{Note}]
  \par
The value of this attribute is always understood to be a single token, even if it contains space or other punctuation characters, and need not be composed of numbers only. It is typically used to specify the numbering of chapters, sections, list items, etc.; it may also be used in the specification of a standard reference system for the text.
\end{reflist}  
    \item[@xml:lang]
  (language) indicates the language of the element content using a ‘tag’ generated according to \xref{http://www.rfc-editor.org/rfc/bcp/bcp47.txt}{BCP 47}.
\begin{reflist}
    \item[{Status}]
  Optional
    \item[{Datatype}]
  \hyperref[TEI.teidata.language]{teidata.language}
    \item[]\index{p=<p>|exampleindex}\index{foreign=<foreign>|exampleindex}\exampleFont {<\textbf{p}>} … The consequences of\mbox{}\newline 
 this rapid depopulation were the loss of the last\mbox{}\newline 
{<\textbf{foreign}\hspace*{1em}{xml:lang}="{rap}">}ariki{</\textbf{foreign}>} or chief\mbox{}\newline 
 (Routledge 1920:205,210) and their connections to\mbox{}\newline 
 ancestral territorial organization.{</\textbf{p}>}
    \item[{Note}]
  \par
The {\itshape xml:lang} value will be inherited from the immediately enclosing element, or from its parent, and so on up the document hierarchy. It is generally good practice to specify {\itshape xml:lang} at the highest appropriate level, noticing that a different default may be needed for the \hyperref[TEI.teiHeader]{<teiHeader>} from that needed for the associated resource element or elements, and that a single TEI document may contain texts in many languages.\par
The authoritative list of registered language subtags is maintained by IANA and is available at \url{http://www.iana.org/assignments/language-subtag-registry}. For a good general overview of the construction of language tags, see \url{http://www.w3.org/International/articles/language-tags/}, and for a practical step-by-step guide, see \url{https://www.w3.org/International/questions/qa-choosing-language-tags.en.php}.\par
The value used must conform with BCP 47. If the value is a private use code (i.e., starts with x- or contains -x-), a \hyperref[TEI.language]{<language>} element with a matching value for its {\itshape ident} attribute should be supplied in the TEI header to document this value. Such documentation may also optionally be supplied for non-private-use codes, though these must remain consistent with their  ( {\abbr IETF}) {\expan Internet Engineering Task Force} definitions.
\end{reflist}  
    \item[@xml:base]
  provides a base URI reference with which applications can resolve relative URI references into absolute URI references.
\begin{reflist}
    \item[{Status}]
  Optional
    \item[{Datatype}]
  \hyperref[TEI.teidata.pointer]{teidata.pointer}
    \item[]\index{div=<div>|exampleindex}\index{type=@type!<div>|exampleindex}\index{head=<head>|exampleindex}\index{listBibl=<listBibl>|exampleindex}\index{bibl=<bibl>|exampleindex}\index{author=<author>|exampleindex}\index{name=<name>|exampleindex}\index{ref=<ref>|exampleindex}\index{target=@target!<ref>|exampleindex}\index{title=<title>|exampleindex}\index{bibl=<bibl>|exampleindex}\index{author=<author>|exampleindex}\index{name=<name>|exampleindex}\index{ref=<ref>|exampleindex}\index{target=@target!<ref>|exampleindex}\index{title=<title>|exampleindex}\index{bibl=<bibl>|exampleindex}\index{author=<author>|exampleindex}\index{name=<name>|exampleindex}\index{ref=<ref>|exampleindex}\index{target=@target!<ref>|exampleindex}\index{title=<title>|exampleindex}\exampleFont {<\textbf{div}\hspace*{1em}{type}="{bibl}">}\mbox{}\newline 
\hspace*{1em}{<\textbf{head}>}Bibliography{</\textbf{head}>}\mbox{}\newline 
\hspace*{1em}{<\textbf{listBibl}\hspace*{1em}{xml:base}="{http://www.lib.ucdavis.edu/BWRP/Works/}">}\mbox{}\newline 
\hspace*{1em}\hspace*{1em}{<\textbf{bibl}>}\mbox{}\newline 
\hspace*{1em}\hspace*{1em}\hspace*{1em}{<\textbf{author}>}\mbox{}\newline 
\hspace*{1em}\hspace*{1em}\hspace*{1em}\hspace*{1em}{<\textbf{name}>}Landon, Letitia Elizabeth{</\textbf{name}>}\mbox{}\newline 
\hspace*{1em}\hspace*{1em}\hspace*{1em}{</\textbf{author}>}\mbox{}\newline 
\hspace*{1em}\hspace*{1em}\hspace*{1em}{<\textbf{ref}\hspace*{1em}{target}="{LandLVowOf.sgm}">}\mbox{}\newline 
\hspace*{1em}\hspace*{1em}\hspace*{1em}\hspace*{1em}{<\textbf{title}>}The Vow of the Peacock{</\textbf{title}>}\mbox{}\newline 
\hspace*{1em}\hspace*{1em}\hspace*{1em}{</\textbf{ref}>}\mbox{}\newline 
\hspace*{1em}\hspace*{1em}{</\textbf{bibl}>}\mbox{}\newline 
\hspace*{1em}\hspace*{1em}{<\textbf{bibl}>}\mbox{}\newline 
\hspace*{1em}\hspace*{1em}\hspace*{1em}{<\textbf{author}>}\mbox{}\newline 
\hspace*{1em}\hspace*{1em}\hspace*{1em}\hspace*{1em}{<\textbf{name}>}Compton, Margaret Clephane{</\textbf{name}>}\mbox{}\newline 
\hspace*{1em}\hspace*{1em}\hspace*{1em}{</\textbf{author}>}\mbox{}\newline 
\hspace*{1em}\hspace*{1em}\hspace*{1em}{<\textbf{ref}\hspace*{1em}{target}="{NortMIrene.sgm}">}\mbox{}\newline 
\hspace*{1em}\hspace*{1em}\hspace*{1em}\hspace*{1em}{<\textbf{title}>}Irene, a Poem in Six Cantos{</\textbf{title}>}\mbox{}\newline 
\hspace*{1em}\hspace*{1em}\hspace*{1em}{</\textbf{ref}>}\mbox{}\newline 
\hspace*{1em}\hspace*{1em}{</\textbf{bibl}>}\mbox{}\newline 
\hspace*{1em}\hspace*{1em}{<\textbf{bibl}>}\mbox{}\newline 
\hspace*{1em}\hspace*{1em}\hspace*{1em}{<\textbf{author}>}\mbox{}\newline 
\hspace*{1em}\hspace*{1em}\hspace*{1em}\hspace*{1em}{<\textbf{name}>}Taylor, Jane{</\textbf{name}>}\mbox{}\newline 
\hspace*{1em}\hspace*{1em}\hspace*{1em}{</\textbf{author}>}\mbox{}\newline 
\hspace*{1em}\hspace*{1em}\hspace*{1em}{<\textbf{ref}\hspace*{1em}{target}="{TaylJEssay.sgm}">}\mbox{}\newline 
\hspace*{1em}\hspace*{1em}\hspace*{1em}\hspace*{1em}{<\textbf{title}>}Essays in Rhyme on Morals and Manners{</\textbf{title}>}\mbox{}\newline 
\hspace*{1em}\hspace*{1em}\hspace*{1em}{</\textbf{ref}>}\mbox{}\newline 
\hspace*{1em}\hspace*{1em}{</\textbf{bibl}>}\mbox{}\newline 
\hspace*{1em}{</\textbf{listBibl}>}\mbox{}\newline 
{</\textbf{div}>}
\end{reflist}  
    \item[@xml:space]
  signals an intention about how white space should be managed by applications.
\begin{reflist}
    \item[{Status}]
  Optional
    \item[{Datatype}]
  \hyperref[TEI.teidata.enumerated]{teidata.enumerated}
    \item[{Legal values are:}]
  \begin{description}

\item[{default}]signals that the application's default white-space processing modes are acceptable
\item[{preserve}]indicates the intent that applications preserve all white space
\end{description} 
    \item[{Note}]
  \par
The \xref{http://www.w3.org/TR/REC-xml/\#sec-white-space}{XML specification} provides further guidance on the use of this attribute. Note that many parsers may not handle xml:space correctly.
\end{reflist}  
\end{sansreflist}  
\end{reflist}  
\begin{reflist}
\item[]\begin{specHead}{TEI.att.global.analytic}{att.global.analytic} provides additional global attributes for associating specific analyses or interpretations with appropriate portions of a text. [\textit{\hyperref[AIATTS]{17.2.\ Global Attributes for Simple Analyses}} \textit{\hyperref[AISP]{17.3.\ Spans and Interpretations}}]\end{specHead} 
    \item[{Module}]
  analysis — \hyperref[AI]{Simple Analytic Mechanisms}
    \item[{Members}]
  \hyperref[TEI.att.global]{att.global}[\hyperref[TEI.TEI]{TEI} \hyperref[TEI.ab]{ab} \hyperref[TEI.abbr]{abbr} \hyperref[TEI.abstract]{abstract} \hyperref[TEI.accMat]{accMat} \hyperref[TEI.acquisition]{acquisition} \hyperref[TEI.activity]{activity} \hyperref[TEI.actor]{actor} \hyperref[TEI.add]{add} \hyperref[TEI.addName]{addName} \hyperref[TEI.addSpan]{addSpan} \hyperref[TEI.additional]{additional} \hyperref[TEI.additions]{additions} \hyperref[TEI.addrLine]{addrLine} \hyperref[TEI.address]{address} \hyperref[TEI.adminInfo]{adminInfo} \hyperref[TEI.affiliation]{affiliation} \hyperref[TEI.age]{age} \hyperref[TEI.alt]{alt} \hyperref[TEI.altGrp]{altGrp} \hyperref[TEI.altIdent]{altIdent} \hyperref[TEI.altIdentifier]{altIdentifier} \hyperref[TEI.alternate]{alternate} \hyperref[TEI.am]{am} \hyperref[TEI.analytic]{analytic} \hyperref[TEI.anchor]{anchor} \hyperref[TEI.annotation]{annotation} \hyperref[TEI.annotationBlock]{annotationBlock} \hyperref[TEI.anyElement]{anyElement} \hyperref[TEI.app]{app} \hyperref[TEI.appInfo]{appInfo} \hyperref[TEI.application]{application} \hyperref[TEI.arc]{arc} \hyperref[TEI.argument]{argument} \hyperref[TEI.att]{att} \hyperref[TEI.attDef]{attDef} \hyperref[TEI.attList]{attList} \hyperref[TEI.attRef]{attRef} \hyperref[TEI.author]{author} \hyperref[TEI.authority]{authority} \hyperref[TEI.availability]{availability} \hyperref[TEI.back]{back} \hyperref[TEI.bibl]{bibl} \hyperref[TEI.biblFull]{biblFull} \hyperref[TEI.biblScope]{biblScope} \hyperref[TEI.biblStruct]{biblStruct} \hyperref[TEI.bicond]{bicond} \hyperref[TEI.binary]{binary} \hyperref[TEI.binaryObject]{binaryObject} \hyperref[TEI.binding]{binding} \hyperref[TEI.bindingDesc]{bindingDesc} \hyperref[TEI.birth]{birth} \hyperref[TEI.bloc]{bloc} \hyperref[TEI.body]{body} \hyperref[TEI.broadcast]{broadcast} \hyperref[TEI.byline]{byline} \hyperref[TEI.c]{c} \hyperref[TEI.cRefPattern]{cRefPattern} \hyperref[TEI.caesura]{caesura} \hyperref[TEI.calendar]{calendar} \hyperref[TEI.calendarDesc]{calendarDesc} \hyperref[TEI.camera]{camera} \hyperref[TEI.caption]{caption} \hyperref[TEI.case]{case} \hyperref[TEI.castGroup]{castGroup} \hyperref[TEI.castItem]{castItem} \hyperref[TEI.castList]{castList} \hyperref[TEI.catDesc]{catDesc} \hyperref[TEI.catRef]{catRef} \hyperref[TEI.catchwords]{catchwords} \hyperref[TEI.category]{category} \hyperref[TEI.cb]{cb} \hyperref[TEI.cell]{cell} \hyperref[TEI.certainty]{certainty} \hyperref[TEI.change]{change} \hyperref[TEI.channel]{channel} \hyperref[TEI.char]{char} \hyperref[TEI.charDecl]{charDecl} \hyperref[TEI.charName]{charName} \hyperref[TEI.charProp]{charProp} \hyperref[TEI.choice]{choice} \hyperref[TEI.cit]{cit} \hyperref[TEI.citeData]{citeData} \hyperref[TEI.citeStructure]{citeStructure} \hyperref[TEI.citedRange]{citedRange} \hyperref[TEI.cl]{cl} \hyperref[TEI.classCode]{classCode} \hyperref[TEI.classDecl]{classDecl} \hyperref[TEI.classRef]{classRef} \hyperref[TEI.classSpec]{classSpec} \hyperref[TEI.classes]{classes} \hyperref[TEI.climate]{climate} \hyperref[TEI.closer]{closer} \hyperref[TEI.code]{code} \hyperref[TEI.collation]{collation} \hyperref[TEI.collection]{collection} \hyperref[TEI.colloc]{colloc} \hyperref[TEI.colophon]{colophon} \hyperref[TEI.cond]{cond} \hyperref[TEI.condition]{condition} \hyperref[TEI.constitution]{constitution} \hyperref[TEI.constraint]{constraint} \hyperref[TEI.constraintSpec]{constraintSpec} \hyperref[TEI.content]{content} \hyperref[TEI.conversion]{conversion} \hyperref[TEI.corr]{corr} \hyperref[TEI.correction]{correction} \hyperref[TEI.correspAction]{correspAction} \hyperref[TEI.correspContext]{correspContext} \hyperref[TEI.correspDesc]{correspDesc} \hyperref[TEI.country]{country} \hyperref[TEI.creation]{creation} \hyperref[TEI.custEvent]{custEvent} \hyperref[TEI.custodialHist]{custodialHist} \hyperref[TEI.damage]{damage} \hyperref[TEI.damageSpan]{damageSpan} \hyperref[TEI.dataFacet]{dataFacet} \hyperref[TEI.dataRef]{dataRef} \hyperref[TEI.dataSpec]{dataSpec} \hyperref[TEI.datatype]{datatype} \hyperref[TEI.date]{date} \hyperref[TEI.dateline]{dateline} \hyperref[TEI.death]{death} \hyperref[TEI.decoDesc]{decoDesc} \hyperref[TEI.decoNote]{decoNote} \hyperref[TEI.def]{def} \hyperref[TEI.default]{default} \hyperref[TEI.defaultVal]{defaultVal} \hyperref[TEI.del]{del} \hyperref[TEI.delSpan]{delSpan} \hyperref[TEI.depth]{depth} \hyperref[TEI.derivation]{derivation} \hyperref[TEI.desc]{desc} \hyperref[TEI.dictScrap]{dictScrap} \hyperref[TEI.dim]{dim} \hyperref[TEI.dimensions]{dimensions} \hyperref[TEI.distinct]{distinct} \hyperref[TEI.distributor]{distributor} \hyperref[TEI.district]{district} \hyperref[TEI.div]{div} \hyperref[TEI.div1]{div1} \hyperref[TEI.div2]{div2} \hyperref[TEI.div3]{div3} \hyperref[TEI.div4]{div4} \hyperref[TEI.div5]{div5} \hyperref[TEI.div6]{div6} \hyperref[TEI.div7]{div7} \hyperref[TEI.divGen]{divGen} \hyperref[TEI.docAuthor]{docAuthor} \hyperref[TEI.docDate]{docDate} \hyperref[TEI.docEdition]{docEdition} \hyperref[TEI.docImprint]{docImprint} \hyperref[TEI.docTitle]{docTitle} \hyperref[TEI.domain]{domain} \hyperref[TEI.eLeaf]{eLeaf} \hyperref[TEI.eTree]{eTree} \hyperref[TEI.edition]{edition} \hyperref[TEI.editionStmt]{editionStmt} \hyperref[TEI.editor]{editor} \hyperref[TEI.editorialDecl]{editorialDecl} \hyperref[TEI.education]{education} \hyperref[TEI.eg]{eg} \hyperref[TEI.egXML]{egXML} \hyperref[TEI.elementRef]{elementRef} \hyperref[TEI.elementSpec]{elementSpec} \hyperref[TEI.email]{email} \hyperref[TEI.emph]{emph} \hyperref[TEI.empty]{empty} \hyperref[TEI.encodingDesc]{encodingDesc} \hyperref[TEI.entry]{entry} \hyperref[TEI.entryFree]{entryFree} \hyperref[TEI.epigraph]{epigraph} \hyperref[TEI.epilogue]{epilogue} \hyperref[TEI.equipment]{equipment} \hyperref[TEI.equiv]{equiv} \hyperref[TEI.etym]{etym} \hyperref[TEI.event]{event} \hyperref[TEI.ex]{ex} \hyperref[TEI.exemplum]{exemplum} \hyperref[TEI.expan]{expan} \hyperref[TEI.explicit]{explicit} \hyperref[TEI.extent]{extent} \hyperref[TEI.f]{f} \hyperref[TEI.fDecl]{fDecl} \hyperref[TEI.fDescr]{fDescr} \hyperref[TEI.fLib]{fLib} \hyperref[TEI.facsimile]{facsimile} \hyperref[TEI.factuality]{factuality} \hyperref[TEI.faith]{faith} \hyperref[TEI.figDesc]{figDesc} \hyperref[TEI.figure]{figure} \hyperref[TEI.fileDesc]{fileDesc} \hyperref[TEI.filiation]{filiation} \hyperref[TEI.finalRubric]{finalRubric} \hyperref[TEI.floatingText]{floatingText} \hyperref[TEI.floruit]{floruit} \hyperref[TEI.foliation]{foliation} \hyperref[TEI.foreign]{foreign} \hyperref[TEI.forename]{forename} \hyperref[TEI.forest]{forest} \hyperref[TEI.form]{form} \hyperref[TEI.formula]{formula} \hyperref[TEI.front]{front} \hyperref[TEI.fs]{fs} \hyperref[TEI.fsConstraints]{fsConstraints} \hyperref[TEI.fsDecl]{fsDecl} \hyperref[TEI.fsDescr]{fsDescr} \hyperref[TEI.fsdDecl]{fsdDecl} \hyperref[TEI.fsdLink]{fsdLink} \hyperref[TEI.funder]{funder} \hyperref[TEI.fvLib]{fvLib} \hyperref[TEI.fw]{fw} \hyperref[TEI.g]{g} \hyperref[TEI.gap]{gap} \hyperref[TEI.gb]{gb} \hyperref[TEI.gen]{gen} \hyperref[TEI.genName]{genName} \hyperref[TEI.geo]{geo} \hyperref[TEI.geoDecl]{geoDecl} \hyperref[TEI.geogFeat]{geogFeat} \hyperref[TEI.geogName]{geogName} \hyperref[TEI.gi]{gi} \hyperref[TEI.gloss]{gloss} \hyperref[TEI.glyph]{glyph} \hyperref[TEI.glyphName]{glyphName} \hyperref[TEI.gram]{gram} \hyperref[TEI.gramGrp]{gramGrp} \hyperref[TEI.graph]{graph} \hyperref[TEI.graphic]{graphic} \hyperref[TEI.group]{group} \hyperref[TEI.handDesc]{handDesc} \hyperref[TEI.handNote]{handNote} \hyperref[TEI.handNotes]{handNotes} \hyperref[TEI.handShift]{handShift} \hyperref[TEI.head]{head} \hyperref[TEI.headItem]{headItem} \hyperref[TEI.headLabel]{headLabel} \hyperref[TEI.height]{height} \hyperref[TEI.heraldry]{heraldry} \hyperref[TEI.hi]{hi} \hyperref[TEI.history]{history} \hyperref[TEI.hom]{hom} \hyperref[TEI.hyph]{hyph} \hyperref[TEI.hyphenation]{hyphenation} \hyperref[TEI.iNode]{iNode} \hyperref[TEI.iType]{iType} \hyperref[TEI.ident]{ident} \hyperref[TEI.idno]{idno} \hyperref[TEI.if]{if} \hyperref[TEI.iff]{iff} \hyperref[TEI.imprimatur]{imprimatur} \hyperref[TEI.imprint]{imprint} \hyperref[TEI.incident]{incident} \hyperref[TEI.incipit]{incipit} \hyperref[TEI.index]{index} \hyperref[TEI.institution]{institution} \hyperref[TEI.interaction]{interaction} \hyperref[TEI.interp]{interp} \hyperref[TEI.interpGrp]{interpGrp} \hyperref[TEI.interpretation]{interpretation} \hyperref[TEI.item]{item} \hyperref[TEI.join]{join} \hyperref[TEI.joinGrp]{joinGrp} \hyperref[TEI.keywords]{keywords} \hyperref[TEI.kinesic]{kinesic} \hyperref[TEI.l]{l} \hyperref[TEI.label]{label} \hyperref[TEI.lacunaEnd]{lacunaEnd} \hyperref[TEI.lacunaStart]{lacunaStart} \hyperref[TEI.lang]{lang} \hyperref[TEI.langKnowledge]{langKnowledge} \hyperref[TEI.langKnown]{langKnown} \hyperref[TEI.langUsage]{langUsage} \hyperref[TEI.language]{language} \hyperref[TEI.layout]{layout} \hyperref[TEI.layoutDesc]{layoutDesc} \hyperref[TEI.lb]{lb} \hyperref[TEI.lbl]{lbl} \hyperref[TEI.leaf]{leaf} \hyperref[TEI.lem]{lem} \hyperref[TEI.lg]{lg} \hyperref[TEI.licence]{licence} \hyperref[TEI.line]{line} \hyperref[TEI.link]{link} \hyperref[TEI.linkGrp]{linkGrp} \hyperref[TEI.list]{list} \hyperref[TEI.listAnnotation]{listAnnotation} \hyperref[TEI.listApp]{listApp} \hyperref[TEI.listBibl]{listBibl} \hyperref[TEI.listChange]{listChange} \hyperref[TEI.listEvent]{listEvent} \hyperref[TEI.listForest]{listForest} \hyperref[TEI.listNym]{listNym} \hyperref[TEI.listObject]{listObject} \hyperref[TEI.listOrg]{listOrg} \hyperref[TEI.listPerson]{listPerson} \hyperref[TEI.listPlace]{listPlace} \hyperref[TEI.listPrefixDef]{listPrefixDef} \hyperref[TEI.listRef]{listRef} \hyperref[TEI.listRelation]{listRelation} \hyperref[TEI.listTranspose]{listTranspose} \hyperref[TEI.listWit]{listWit} \hyperref[TEI.localName]{localName} \hyperref[TEI.localProp]{localProp} \hyperref[TEI.locale]{locale} \hyperref[TEI.location]{location} \hyperref[TEI.locus]{locus} \hyperref[TEI.locusGrp]{locusGrp} \hyperref[TEI.m]{m} \hyperref[TEI.macroRef]{macroRef} \hyperref[TEI.macroSpec]{macroSpec} \hyperref[TEI.mapping]{mapping} \hyperref[TEI.material]{material} \hyperref[TEI.measure]{measure} \hyperref[TEI.measureGrp]{measureGrp} \hyperref[TEI.media]{media} \hyperref[TEI.meeting]{meeting} \hyperref[TEI.memberOf]{memberOf} \hyperref[TEI.mentioned]{mentioned} \hyperref[TEI.metDecl]{metDecl} \hyperref[TEI.metSym]{metSym} \hyperref[TEI.metamark]{metamark} \hyperref[TEI.milestone]{milestone} \hyperref[TEI.mod]{mod} \hyperref[TEI.model]{model} \hyperref[TEI.modelGrp]{modelGrp} \hyperref[TEI.modelSequence]{modelSequence} \hyperref[TEI.moduleRef]{moduleRef} \hyperref[TEI.moduleSpec]{moduleSpec} \hyperref[TEI.monogr]{monogr} \hyperref[TEI.mood]{mood} \hyperref[TEI.move]{move} \hyperref[TEI.msContents]{msContents} \hyperref[TEI.msDesc]{msDesc} \hyperref[TEI.msFrag]{msFrag} \hyperref[TEI.msIdentifier]{msIdentifier} \hyperref[TEI.msItem]{msItem} \hyperref[TEI.msItemStruct]{msItemStruct} \hyperref[TEI.msName]{msName} \hyperref[TEI.msPart]{msPart} \hyperref[TEI.musicNotation]{musicNotation} \hyperref[TEI.name]{name} \hyperref[TEI.nameLink]{nameLink} \hyperref[TEI.namespace]{namespace} \hyperref[TEI.nationality]{nationality} \hyperref[TEI.node]{node} \hyperref[TEI.normalization]{normalization} \hyperref[TEI.notatedMusic]{notatedMusic} \hyperref[TEI.note]{note} \hyperref[TEI.noteGrp]{noteGrp} \hyperref[TEI.notesStmt]{notesStmt} \hyperref[TEI.num]{num} \hyperref[TEI.number]{number} \hyperref[TEI.numeric]{numeric} \hyperref[TEI.nym]{nym} \hyperref[TEI.oRef]{oRef} \hyperref[TEI.object]{object} \hyperref[TEI.objectDesc]{objectDesc} \hyperref[TEI.objectIdentifier]{objectIdentifier} \hyperref[TEI.objectName]{objectName} \hyperref[TEI.objectType]{objectType} \hyperref[TEI.occupation]{occupation} \hyperref[TEI.offset]{offset} \hyperref[TEI.opener]{opener} \hyperref[TEI.org]{org} \hyperref[TEI.orgName]{orgName} \hyperref[TEI.orig]{orig} \hyperref[TEI.origDate]{origDate} \hyperref[TEI.origPlace]{origPlace} \hyperref[TEI.origin]{origin} \hyperref[TEI.orth]{orth} \hyperref[TEI.outputRendition]{outputRendition} \hyperref[TEI.p]{p} \hyperref[TEI.pRef]{pRef} \hyperref[TEI.param]{param} \hyperref[TEI.paramList]{paramList} \hyperref[TEI.paramSpec]{paramSpec} \hyperref[TEI.particDesc]{particDesc} \hyperref[TEI.path]{path} \hyperref[TEI.pause]{pause} \hyperref[TEI.pb]{pb} \hyperref[TEI.pc]{pc} \hyperref[TEI.per]{per} \hyperref[TEI.performance]{performance} \hyperref[TEI.persName]{persName} \hyperref[TEI.persPronouns]{persPronouns} \hyperref[TEI.person]{person} \hyperref[TEI.personGrp]{personGrp} \hyperref[TEI.persona]{persona} \hyperref[TEI.phr]{phr} \hyperref[TEI.physDesc]{physDesc} \hyperref[TEI.place]{place} \hyperref[TEI.placeName]{placeName} \hyperref[TEI.population]{population} \hyperref[TEI.pos]{pos} \hyperref[TEI.postBox]{postBox} \hyperref[TEI.postCode]{postCode} \hyperref[TEI.postscript]{postscript} \hyperref[TEI.precision]{precision} \hyperref[TEI.prefixDef]{prefixDef} \hyperref[TEI.preparedness]{preparedness} \hyperref[TEI.principal]{principal} \hyperref[TEI.profileDesc]{profileDesc} \hyperref[TEI.projectDesc]{projectDesc} \hyperref[TEI.prologue]{prologue} \hyperref[TEI.pron]{pron} \hyperref[TEI.provenance]{provenance} \hyperref[TEI.ptr]{ptr} \hyperref[TEI.pubPlace]{pubPlace} \hyperref[TEI.publicationStmt]{publicationStmt} \hyperref[TEI.publisher]{publisher} \hyperref[TEI.punctuation]{punctuation} \hyperref[TEI.purpose]{purpose} \hyperref[TEI.q]{q} \hyperref[TEI.quotation]{quotation} \hyperref[TEI.quote]{quote} \hyperref[TEI.rb]{rb} \hyperref[TEI.rdg]{rdg} \hyperref[TEI.rdgGrp]{rdgGrp} \hyperref[TEI.re]{re} \hyperref[TEI.recordHist]{recordHist} \hyperref[TEI.recording]{recording} \hyperref[TEI.recordingStmt]{recordingStmt} \hyperref[TEI.redo]{redo} \hyperref[TEI.ref]{ref} \hyperref[TEI.refState]{refState} \hyperref[TEI.refsDecl]{refsDecl} \hyperref[TEI.reg]{reg} \hyperref[TEI.region]{region} \hyperref[TEI.relatedItem]{relatedItem} \hyperref[TEI.relation]{relation} \hyperref[TEI.remarks]{remarks} \hyperref[TEI.rendition]{rendition} \hyperref[TEI.repository]{repository} \hyperref[TEI.residence]{residence} \hyperref[TEI.resp]{resp} \hyperref[TEI.respStmt]{respStmt} \hyperref[TEI.respons]{respons} \hyperref[TEI.restore]{restore} \hyperref[TEI.retrace]{retrace} \hyperref[TEI.revisionDesc]{revisionDesc} \hyperref[TEI.rhyme]{rhyme} \hyperref[TEI.role]{role} \hyperref[TEI.roleDesc]{roleDesc} \hyperref[TEI.roleName]{roleName} \hyperref[TEI.root]{root} \hyperref[TEI.row]{row} \hyperref[TEI.rs]{rs} \hyperref[TEI.rt]{rt} \hyperref[TEI.rubric]{rubric} \hyperref[TEI.ruby]{ruby} \hyperref[TEI.s]{s} \hyperref[TEI.said]{said} \hyperref[TEI.salute]{salute} \hyperref[TEI.samplingDecl]{samplingDecl} \hyperref[TEI.schemaRef]{schemaRef} \hyperref[TEI.schemaSpec]{schemaSpec} \hyperref[TEI.scriptDesc]{scriptDesc} \hyperref[TEI.scriptNote]{scriptNote} \hyperref[TEI.scriptStmt]{scriptStmt} \hyperref[TEI.seal]{seal} \hyperref[TEI.sealDesc]{sealDesc} \hyperref[TEI.secFol]{secFol} \hyperref[TEI.secl]{secl} \hyperref[TEI.seg]{seg} \hyperref[TEI.segmentation]{segmentation} \hyperref[TEI.sense]{sense} \hyperref[TEI.sequence]{sequence} \hyperref[TEI.series]{series} \hyperref[TEI.seriesStmt]{seriesStmt} \hyperref[TEI.set]{set} \hyperref[TEI.setting]{setting} \hyperref[TEI.settingDesc]{settingDesc} \hyperref[TEI.settlement]{settlement} \hyperref[TEI.sex]{sex} \hyperref[TEI.shift]{shift} \hyperref[TEI.sic]{sic} \hyperref[TEI.signatures]{signatures} \hyperref[TEI.signed]{signed} \hyperref[TEI.soCalled]{soCalled} \hyperref[TEI.socecStatus]{socecStatus} \hyperref[TEI.sound]{sound} \hyperref[TEI.source]{source} \hyperref[TEI.sourceDesc]{sourceDesc} \hyperref[TEI.sourceDoc]{sourceDoc} \hyperref[TEI.sp]{sp} \hyperref[TEI.spGrp]{spGrp} \hyperref[TEI.space]{space} \hyperref[TEI.span]{span} \hyperref[TEI.spanGrp]{spanGrp} \hyperref[TEI.speaker]{speaker} \hyperref[TEI.specDesc]{specDesc} \hyperref[TEI.specGrp]{specGrp} \hyperref[TEI.specGrpRef]{specGrpRef} \hyperref[TEI.specList]{specList} \hyperref[TEI.sponsor]{sponsor} \hyperref[TEI.stage]{stage} \hyperref[TEI.stamp]{stamp} \hyperref[TEI.standOff]{standOff} \hyperref[TEI.state]{state} \hyperref[TEI.stdVals]{stdVals} \hyperref[TEI.street]{street} \hyperref[TEI.stress]{stress} \hyperref[TEI.string]{string} \hyperref[TEI.styleDefDecl]{styleDefDecl} \hyperref[TEI.subc]{subc} \hyperref[TEI.subst]{subst} \hyperref[TEI.substJoin]{substJoin} \hyperref[TEI.summary]{summary} \hyperref[TEI.superEntry]{superEntry} \hyperref[TEI.supplied]{supplied} \hyperref[TEI.support]{support} \hyperref[TEI.supportDesc]{supportDesc} \hyperref[TEI.surface]{surface} \hyperref[TEI.surfaceGrp]{surfaceGrp} \hyperref[TEI.surname]{surname} \hyperref[TEI.surplus]{surplus} \hyperref[TEI.surrogates]{surrogates} \hyperref[TEI.syll]{syll} \hyperref[TEI.symbol]{symbol} \hyperref[TEI.table]{table} \hyperref[TEI.tag]{tag} \hyperref[TEI.tagUsage]{tagUsage} \hyperref[TEI.tagsDecl]{tagsDecl} \hyperref[TEI.taxonomy]{taxonomy} \hyperref[TEI.tech]{tech} \hyperref[TEI.teiCorpus]{teiCorpus} \hyperref[TEI.teiHeader]{teiHeader} \hyperref[TEI.term]{term} \hyperref[TEI.terrain]{terrain} \hyperref[TEI.text]{text} \hyperref[TEI.textClass]{textClass} \hyperref[TEI.textDesc]{textDesc} \hyperref[TEI.textLang]{textLang} \hyperref[TEI.textNode]{textNode} \hyperref[TEI.then]{then} \hyperref[TEI.time]{time} \hyperref[TEI.timeline]{timeline} \hyperref[TEI.title]{title} \hyperref[TEI.titlePage]{titlePage} \hyperref[TEI.titlePart]{titlePart} \hyperref[TEI.titleStmt]{titleStmt} \hyperref[TEI.tns]{tns} \hyperref[TEI.trailer]{trailer} \hyperref[TEI.trait]{trait} \hyperref[TEI.transcriptionDesc]{transcriptionDesc} \hyperref[TEI.transpose]{transpose} \hyperref[TEI.tree]{tree} \hyperref[TEI.triangle]{triangle} \hyperref[TEI.typeDesc]{typeDesc} \hyperref[TEI.typeNote]{typeNote} \hyperref[TEI.u]{u} \hyperref[TEI.unclear]{unclear} \hyperref[TEI.undo]{undo} \hyperref[TEI.unicodeName]{unicodeName} \hyperref[TEI.unicodeProp]{unicodeProp} \hyperref[TEI.unihanProp]{unihanProp} \hyperref[TEI.unit]{unit} \hyperref[TEI.unitDecl]{unitDecl} \hyperref[TEI.unitDef]{unitDef} \hyperref[TEI.usg]{usg} \hyperref[TEI.vAlt]{vAlt} \hyperref[TEI.vColl]{vColl} \hyperref[TEI.vDefault]{vDefault} \hyperref[TEI.vLabel]{vLabel} \hyperref[TEI.vMerge]{vMerge} \hyperref[TEI.vNot]{vNot} \hyperref[TEI.vRange]{vRange} \hyperref[TEI.val]{val} \hyperref[TEI.valDesc]{valDesc} \hyperref[TEI.valItem]{valItem} \hyperref[TEI.valList]{valList} \hyperref[TEI.value]{value} \hyperref[TEI.variantEncoding]{variantEncoding} \hyperref[TEI.view]{view} \hyperref[TEI.vocal]{vocal} \hyperref[TEI.w]{w} \hyperref[TEI.watermark]{watermark} \hyperref[TEI.when]{when} \hyperref[TEI.width]{width} \hyperref[TEI.wit]{wit} \hyperref[TEI.witDetail]{witDetail} \hyperref[TEI.witEnd]{witEnd} \hyperref[TEI.witStart]{witStart} \hyperref[TEI.witness]{witness} \hyperref[TEI.writing]{writing} \hyperref[TEI.xenoData]{xenoData} \hyperref[TEI.xr]{xr} \hyperref[TEI.zone]{zone}]
    \item[{Attributes}]
  Attributes\hfil\\[-10pt]\begin{sansreflist}
    \item[@ana]
  (analysis) indicates one or more elements containing interpretations of the element on which the {\itshape ana} attribute appears.
\begin{reflist}
    \item[{Status}]
  Optional
    \item[{Datatype}]
  1–∞ occurrences of \hyperref[TEI.teidata.pointer]{teidata.pointer} separated by whitespace
    \item[{Note}]
  \par
When multiple values are given, they may reflect either multiple divergent interpretations of an ambiguous text, or multiple mutually consistent interpretations of the same passage in different contexts.
\end{reflist}  
\end{sansreflist}  
\end{reflist}  
\begin{reflist}
\item[]\begin{specHead}{TEI.att.global.change}{att.global.change} supplies the {\itshape change} attribute, allowing its member elements to specify one or more states or revision campaigns with which they are associated.\end{specHead} 
    \item[{Module}]
  transcr — \hyperref[PH]{Representation of Primary Sources}
    \item[{Members}]
  \hyperref[TEI.att.global]{att.global}[\hyperref[TEI.TEI]{TEI} \hyperref[TEI.ab]{ab} \hyperref[TEI.abbr]{abbr} \hyperref[TEI.abstract]{abstract} \hyperref[TEI.accMat]{accMat} \hyperref[TEI.acquisition]{acquisition} \hyperref[TEI.activity]{activity} \hyperref[TEI.actor]{actor} \hyperref[TEI.add]{add} \hyperref[TEI.addName]{addName} \hyperref[TEI.addSpan]{addSpan} \hyperref[TEI.additional]{additional} \hyperref[TEI.additions]{additions} \hyperref[TEI.addrLine]{addrLine} \hyperref[TEI.address]{address} \hyperref[TEI.adminInfo]{adminInfo} \hyperref[TEI.affiliation]{affiliation} \hyperref[TEI.age]{age} \hyperref[TEI.alt]{alt} \hyperref[TEI.altGrp]{altGrp} \hyperref[TEI.altIdent]{altIdent} \hyperref[TEI.altIdentifier]{altIdentifier} \hyperref[TEI.alternate]{alternate} \hyperref[TEI.am]{am} \hyperref[TEI.analytic]{analytic} \hyperref[TEI.anchor]{anchor} \hyperref[TEI.annotation]{annotation} \hyperref[TEI.annotationBlock]{annotationBlock} \hyperref[TEI.anyElement]{anyElement} \hyperref[TEI.app]{app} \hyperref[TEI.appInfo]{appInfo} \hyperref[TEI.application]{application} \hyperref[TEI.arc]{arc} \hyperref[TEI.argument]{argument} \hyperref[TEI.att]{att} \hyperref[TEI.attDef]{attDef} \hyperref[TEI.attList]{attList} \hyperref[TEI.attRef]{attRef} \hyperref[TEI.author]{author} \hyperref[TEI.authority]{authority} \hyperref[TEI.availability]{availability} \hyperref[TEI.back]{back} \hyperref[TEI.bibl]{bibl} \hyperref[TEI.biblFull]{biblFull} \hyperref[TEI.biblScope]{biblScope} \hyperref[TEI.biblStruct]{biblStruct} \hyperref[TEI.bicond]{bicond} \hyperref[TEI.binary]{binary} \hyperref[TEI.binaryObject]{binaryObject} \hyperref[TEI.binding]{binding} \hyperref[TEI.bindingDesc]{bindingDesc} \hyperref[TEI.birth]{birth} \hyperref[TEI.bloc]{bloc} \hyperref[TEI.body]{body} \hyperref[TEI.broadcast]{broadcast} \hyperref[TEI.byline]{byline} \hyperref[TEI.c]{c} \hyperref[TEI.cRefPattern]{cRefPattern} \hyperref[TEI.caesura]{caesura} \hyperref[TEI.calendar]{calendar} \hyperref[TEI.calendarDesc]{calendarDesc} \hyperref[TEI.camera]{camera} \hyperref[TEI.caption]{caption} \hyperref[TEI.case]{case} \hyperref[TEI.castGroup]{castGroup} \hyperref[TEI.castItem]{castItem} \hyperref[TEI.castList]{castList} \hyperref[TEI.catDesc]{catDesc} \hyperref[TEI.catRef]{catRef} \hyperref[TEI.catchwords]{catchwords} \hyperref[TEI.category]{category} \hyperref[TEI.cb]{cb} \hyperref[TEI.cell]{cell} \hyperref[TEI.certainty]{certainty} \hyperref[TEI.change]{change} \hyperref[TEI.channel]{channel} \hyperref[TEI.char]{char} \hyperref[TEI.charDecl]{charDecl} \hyperref[TEI.charName]{charName} \hyperref[TEI.charProp]{charProp} \hyperref[TEI.choice]{choice} \hyperref[TEI.cit]{cit} \hyperref[TEI.citeData]{citeData} \hyperref[TEI.citeStructure]{citeStructure} \hyperref[TEI.citedRange]{citedRange} \hyperref[TEI.cl]{cl} \hyperref[TEI.classCode]{classCode} \hyperref[TEI.classDecl]{classDecl} \hyperref[TEI.classRef]{classRef} \hyperref[TEI.classSpec]{classSpec} \hyperref[TEI.classes]{classes} \hyperref[TEI.climate]{climate} \hyperref[TEI.closer]{closer} \hyperref[TEI.code]{code} \hyperref[TEI.collation]{collation} \hyperref[TEI.collection]{collection} \hyperref[TEI.colloc]{colloc} \hyperref[TEI.colophon]{colophon} \hyperref[TEI.cond]{cond} \hyperref[TEI.condition]{condition} \hyperref[TEI.constitution]{constitution} \hyperref[TEI.constraint]{constraint} \hyperref[TEI.constraintSpec]{constraintSpec} \hyperref[TEI.content]{content} \hyperref[TEI.conversion]{conversion} \hyperref[TEI.corr]{corr} \hyperref[TEI.correction]{correction} \hyperref[TEI.correspAction]{correspAction} \hyperref[TEI.correspContext]{correspContext} \hyperref[TEI.correspDesc]{correspDesc} \hyperref[TEI.country]{country} \hyperref[TEI.creation]{creation} \hyperref[TEI.custEvent]{custEvent} \hyperref[TEI.custodialHist]{custodialHist} \hyperref[TEI.damage]{damage} \hyperref[TEI.damageSpan]{damageSpan} \hyperref[TEI.dataFacet]{dataFacet} \hyperref[TEI.dataRef]{dataRef} \hyperref[TEI.dataSpec]{dataSpec} \hyperref[TEI.datatype]{datatype} \hyperref[TEI.date]{date} \hyperref[TEI.dateline]{dateline} \hyperref[TEI.death]{death} \hyperref[TEI.decoDesc]{decoDesc} \hyperref[TEI.decoNote]{decoNote} \hyperref[TEI.def]{def} \hyperref[TEI.default]{default} \hyperref[TEI.defaultVal]{defaultVal} \hyperref[TEI.del]{del} \hyperref[TEI.delSpan]{delSpan} \hyperref[TEI.depth]{depth} \hyperref[TEI.derivation]{derivation} \hyperref[TEI.desc]{desc} \hyperref[TEI.dictScrap]{dictScrap} \hyperref[TEI.dim]{dim} \hyperref[TEI.dimensions]{dimensions} \hyperref[TEI.distinct]{distinct} \hyperref[TEI.distributor]{distributor} \hyperref[TEI.district]{district} \hyperref[TEI.div]{div} \hyperref[TEI.div1]{div1} \hyperref[TEI.div2]{div2} \hyperref[TEI.div3]{div3} \hyperref[TEI.div4]{div4} \hyperref[TEI.div5]{div5} \hyperref[TEI.div6]{div6} \hyperref[TEI.div7]{div7} \hyperref[TEI.divGen]{divGen} \hyperref[TEI.docAuthor]{docAuthor} \hyperref[TEI.docDate]{docDate} \hyperref[TEI.docEdition]{docEdition} \hyperref[TEI.docImprint]{docImprint} \hyperref[TEI.docTitle]{docTitle} \hyperref[TEI.domain]{domain} \hyperref[TEI.eLeaf]{eLeaf} \hyperref[TEI.eTree]{eTree} \hyperref[TEI.edition]{edition} \hyperref[TEI.editionStmt]{editionStmt} \hyperref[TEI.editor]{editor} \hyperref[TEI.editorialDecl]{editorialDecl} \hyperref[TEI.education]{education} \hyperref[TEI.eg]{eg} \hyperref[TEI.egXML]{egXML} \hyperref[TEI.elementRef]{elementRef} \hyperref[TEI.elementSpec]{elementSpec} \hyperref[TEI.email]{email} \hyperref[TEI.emph]{emph} \hyperref[TEI.empty]{empty} \hyperref[TEI.encodingDesc]{encodingDesc} \hyperref[TEI.entry]{entry} \hyperref[TEI.entryFree]{entryFree} \hyperref[TEI.epigraph]{epigraph} \hyperref[TEI.epilogue]{epilogue} \hyperref[TEI.equipment]{equipment} \hyperref[TEI.equiv]{equiv} \hyperref[TEI.etym]{etym} \hyperref[TEI.event]{event} \hyperref[TEI.ex]{ex} \hyperref[TEI.exemplum]{exemplum} \hyperref[TEI.expan]{expan} \hyperref[TEI.explicit]{explicit} \hyperref[TEI.extent]{extent} \hyperref[TEI.f]{f} \hyperref[TEI.fDecl]{fDecl} \hyperref[TEI.fDescr]{fDescr} \hyperref[TEI.fLib]{fLib} \hyperref[TEI.facsimile]{facsimile} \hyperref[TEI.factuality]{factuality} \hyperref[TEI.faith]{faith} \hyperref[TEI.figDesc]{figDesc} \hyperref[TEI.figure]{figure} \hyperref[TEI.fileDesc]{fileDesc} \hyperref[TEI.filiation]{filiation} \hyperref[TEI.finalRubric]{finalRubric} \hyperref[TEI.floatingText]{floatingText} \hyperref[TEI.floruit]{floruit} \hyperref[TEI.foliation]{foliation} \hyperref[TEI.foreign]{foreign} \hyperref[TEI.forename]{forename} \hyperref[TEI.forest]{forest} \hyperref[TEI.form]{form} \hyperref[TEI.formula]{formula} \hyperref[TEI.front]{front} \hyperref[TEI.fs]{fs} \hyperref[TEI.fsConstraints]{fsConstraints} \hyperref[TEI.fsDecl]{fsDecl} \hyperref[TEI.fsDescr]{fsDescr} \hyperref[TEI.fsdDecl]{fsdDecl} \hyperref[TEI.fsdLink]{fsdLink} \hyperref[TEI.funder]{funder} \hyperref[TEI.fvLib]{fvLib} \hyperref[TEI.fw]{fw} \hyperref[TEI.g]{g} \hyperref[TEI.gap]{gap} \hyperref[TEI.gb]{gb} \hyperref[TEI.gen]{gen} \hyperref[TEI.genName]{genName} \hyperref[TEI.geo]{geo} \hyperref[TEI.geoDecl]{geoDecl} \hyperref[TEI.geogFeat]{geogFeat} \hyperref[TEI.geogName]{geogName} \hyperref[TEI.gi]{gi} \hyperref[TEI.gloss]{gloss} \hyperref[TEI.glyph]{glyph} \hyperref[TEI.glyphName]{glyphName} \hyperref[TEI.gram]{gram} \hyperref[TEI.gramGrp]{gramGrp} \hyperref[TEI.graph]{graph} \hyperref[TEI.graphic]{graphic} \hyperref[TEI.group]{group} \hyperref[TEI.handDesc]{handDesc} \hyperref[TEI.handNote]{handNote} \hyperref[TEI.handNotes]{handNotes} \hyperref[TEI.handShift]{handShift} \hyperref[TEI.head]{head} \hyperref[TEI.headItem]{headItem} \hyperref[TEI.headLabel]{headLabel} \hyperref[TEI.height]{height} \hyperref[TEI.heraldry]{heraldry} \hyperref[TEI.hi]{hi} \hyperref[TEI.history]{history} \hyperref[TEI.hom]{hom} \hyperref[TEI.hyph]{hyph} \hyperref[TEI.hyphenation]{hyphenation} \hyperref[TEI.iNode]{iNode} \hyperref[TEI.iType]{iType} \hyperref[TEI.ident]{ident} \hyperref[TEI.idno]{idno} \hyperref[TEI.if]{if} \hyperref[TEI.iff]{iff} \hyperref[TEI.imprimatur]{imprimatur} \hyperref[TEI.imprint]{imprint} \hyperref[TEI.incident]{incident} \hyperref[TEI.incipit]{incipit} \hyperref[TEI.index]{index} \hyperref[TEI.institution]{institution} \hyperref[TEI.interaction]{interaction} \hyperref[TEI.interp]{interp} \hyperref[TEI.interpGrp]{interpGrp} \hyperref[TEI.interpretation]{interpretation} \hyperref[TEI.item]{item} \hyperref[TEI.join]{join} \hyperref[TEI.joinGrp]{joinGrp} \hyperref[TEI.keywords]{keywords} \hyperref[TEI.kinesic]{kinesic} \hyperref[TEI.l]{l} \hyperref[TEI.label]{label} \hyperref[TEI.lacunaEnd]{lacunaEnd} \hyperref[TEI.lacunaStart]{lacunaStart} \hyperref[TEI.lang]{lang} \hyperref[TEI.langKnowledge]{langKnowledge} \hyperref[TEI.langKnown]{langKnown} \hyperref[TEI.langUsage]{langUsage} \hyperref[TEI.language]{language} \hyperref[TEI.layout]{layout} \hyperref[TEI.layoutDesc]{layoutDesc} \hyperref[TEI.lb]{lb} \hyperref[TEI.lbl]{lbl} \hyperref[TEI.leaf]{leaf} \hyperref[TEI.lem]{lem} \hyperref[TEI.lg]{lg} \hyperref[TEI.licence]{licence} \hyperref[TEI.line]{line} \hyperref[TEI.link]{link} \hyperref[TEI.linkGrp]{linkGrp} \hyperref[TEI.list]{list} \hyperref[TEI.listAnnotation]{listAnnotation} \hyperref[TEI.listApp]{listApp} \hyperref[TEI.listBibl]{listBibl} \hyperref[TEI.listChange]{listChange} \hyperref[TEI.listEvent]{listEvent} \hyperref[TEI.listForest]{listForest} \hyperref[TEI.listNym]{listNym} \hyperref[TEI.listObject]{listObject} \hyperref[TEI.listOrg]{listOrg} \hyperref[TEI.listPerson]{listPerson} \hyperref[TEI.listPlace]{listPlace} \hyperref[TEI.listPrefixDef]{listPrefixDef} \hyperref[TEI.listRef]{listRef} \hyperref[TEI.listRelation]{listRelation} \hyperref[TEI.listTranspose]{listTranspose} \hyperref[TEI.listWit]{listWit} \hyperref[TEI.localName]{localName} \hyperref[TEI.localProp]{localProp} \hyperref[TEI.locale]{locale} \hyperref[TEI.location]{location} \hyperref[TEI.locus]{locus} \hyperref[TEI.locusGrp]{locusGrp} \hyperref[TEI.m]{m} \hyperref[TEI.macroRef]{macroRef} \hyperref[TEI.macroSpec]{macroSpec} \hyperref[TEI.mapping]{mapping} \hyperref[TEI.material]{material} \hyperref[TEI.measure]{measure} \hyperref[TEI.measureGrp]{measureGrp} \hyperref[TEI.media]{media} \hyperref[TEI.meeting]{meeting} \hyperref[TEI.memberOf]{memberOf} \hyperref[TEI.mentioned]{mentioned} \hyperref[TEI.metDecl]{metDecl} \hyperref[TEI.metSym]{metSym} \hyperref[TEI.metamark]{metamark} \hyperref[TEI.milestone]{milestone} \hyperref[TEI.mod]{mod} \hyperref[TEI.model]{model} \hyperref[TEI.modelGrp]{modelGrp} \hyperref[TEI.modelSequence]{modelSequence} \hyperref[TEI.moduleRef]{moduleRef} \hyperref[TEI.moduleSpec]{moduleSpec} \hyperref[TEI.monogr]{monogr} \hyperref[TEI.mood]{mood} \hyperref[TEI.move]{move} \hyperref[TEI.msContents]{msContents} \hyperref[TEI.msDesc]{msDesc} \hyperref[TEI.msFrag]{msFrag} \hyperref[TEI.msIdentifier]{msIdentifier} \hyperref[TEI.msItem]{msItem} \hyperref[TEI.msItemStruct]{msItemStruct} \hyperref[TEI.msName]{msName} \hyperref[TEI.msPart]{msPart} \hyperref[TEI.musicNotation]{musicNotation} \hyperref[TEI.name]{name} \hyperref[TEI.nameLink]{nameLink} \hyperref[TEI.namespace]{namespace} \hyperref[TEI.nationality]{nationality} \hyperref[TEI.node]{node} \hyperref[TEI.normalization]{normalization} \hyperref[TEI.notatedMusic]{notatedMusic} \hyperref[TEI.note]{note} \hyperref[TEI.noteGrp]{noteGrp} \hyperref[TEI.notesStmt]{notesStmt} \hyperref[TEI.num]{num} \hyperref[TEI.number]{number} \hyperref[TEI.numeric]{numeric} \hyperref[TEI.nym]{nym} \hyperref[TEI.oRef]{oRef} \hyperref[TEI.object]{object} \hyperref[TEI.objectDesc]{objectDesc} \hyperref[TEI.objectIdentifier]{objectIdentifier} \hyperref[TEI.objectName]{objectName} \hyperref[TEI.objectType]{objectType} \hyperref[TEI.occupation]{occupation} \hyperref[TEI.offset]{offset} \hyperref[TEI.opener]{opener} \hyperref[TEI.org]{org} \hyperref[TEI.orgName]{orgName} \hyperref[TEI.orig]{orig} \hyperref[TEI.origDate]{origDate} \hyperref[TEI.origPlace]{origPlace} \hyperref[TEI.origin]{origin} \hyperref[TEI.orth]{orth} \hyperref[TEI.outputRendition]{outputRendition} \hyperref[TEI.p]{p} \hyperref[TEI.pRef]{pRef} \hyperref[TEI.param]{param} \hyperref[TEI.paramList]{paramList} \hyperref[TEI.paramSpec]{paramSpec} \hyperref[TEI.particDesc]{particDesc} \hyperref[TEI.path]{path} \hyperref[TEI.pause]{pause} \hyperref[TEI.pb]{pb} \hyperref[TEI.pc]{pc} \hyperref[TEI.per]{per} \hyperref[TEI.performance]{performance} \hyperref[TEI.persName]{persName} \hyperref[TEI.persPronouns]{persPronouns} \hyperref[TEI.person]{person} \hyperref[TEI.personGrp]{personGrp} \hyperref[TEI.persona]{persona} \hyperref[TEI.phr]{phr} \hyperref[TEI.physDesc]{physDesc} \hyperref[TEI.place]{place} \hyperref[TEI.placeName]{placeName} \hyperref[TEI.population]{population} \hyperref[TEI.pos]{pos} \hyperref[TEI.postBox]{postBox} \hyperref[TEI.postCode]{postCode} \hyperref[TEI.postscript]{postscript} \hyperref[TEI.precision]{precision} \hyperref[TEI.prefixDef]{prefixDef} \hyperref[TEI.preparedness]{preparedness} \hyperref[TEI.principal]{principal} \hyperref[TEI.profileDesc]{profileDesc} \hyperref[TEI.projectDesc]{projectDesc} \hyperref[TEI.prologue]{prologue} \hyperref[TEI.pron]{pron} \hyperref[TEI.provenance]{provenance} \hyperref[TEI.ptr]{ptr} \hyperref[TEI.pubPlace]{pubPlace} \hyperref[TEI.publicationStmt]{publicationStmt} \hyperref[TEI.publisher]{publisher} \hyperref[TEI.punctuation]{punctuation} \hyperref[TEI.purpose]{purpose} \hyperref[TEI.q]{q} \hyperref[TEI.quotation]{quotation} \hyperref[TEI.quote]{quote} \hyperref[TEI.rb]{rb} \hyperref[TEI.rdg]{rdg} \hyperref[TEI.rdgGrp]{rdgGrp} \hyperref[TEI.re]{re} \hyperref[TEI.recordHist]{recordHist} \hyperref[TEI.recording]{recording} \hyperref[TEI.recordingStmt]{recordingStmt} \hyperref[TEI.redo]{redo} \hyperref[TEI.ref]{ref} \hyperref[TEI.refState]{refState} \hyperref[TEI.refsDecl]{refsDecl} \hyperref[TEI.reg]{reg} \hyperref[TEI.region]{region} \hyperref[TEI.relatedItem]{relatedItem} \hyperref[TEI.relation]{relation} \hyperref[TEI.remarks]{remarks} \hyperref[TEI.rendition]{rendition} \hyperref[TEI.repository]{repository} \hyperref[TEI.residence]{residence} \hyperref[TEI.resp]{resp} \hyperref[TEI.respStmt]{respStmt} \hyperref[TEI.respons]{respons} \hyperref[TEI.restore]{restore} \hyperref[TEI.retrace]{retrace} \hyperref[TEI.revisionDesc]{revisionDesc} \hyperref[TEI.rhyme]{rhyme} \hyperref[TEI.role]{role} \hyperref[TEI.roleDesc]{roleDesc} \hyperref[TEI.roleName]{roleName} \hyperref[TEI.root]{root} \hyperref[TEI.row]{row} \hyperref[TEI.rs]{rs} \hyperref[TEI.rt]{rt} \hyperref[TEI.rubric]{rubric} \hyperref[TEI.ruby]{ruby} \hyperref[TEI.s]{s} \hyperref[TEI.said]{said} \hyperref[TEI.salute]{salute} \hyperref[TEI.samplingDecl]{samplingDecl} \hyperref[TEI.schemaRef]{schemaRef} \hyperref[TEI.schemaSpec]{schemaSpec} \hyperref[TEI.scriptDesc]{scriptDesc} \hyperref[TEI.scriptNote]{scriptNote} \hyperref[TEI.scriptStmt]{scriptStmt} \hyperref[TEI.seal]{seal} \hyperref[TEI.sealDesc]{sealDesc} \hyperref[TEI.secFol]{secFol} \hyperref[TEI.secl]{secl} \hyperref[TEI.seg]{seg} \hyperref[TEI.segmentation]{segmentation} \hyperref[TEI.sense]{sense} \hyperref[TEI.sequence]{sequence} \hyperref[TEI.series]{series} \hyperref[TEI.seriesStmt]{seriesStmt} \hyperref[TEI.set]{set} \hyperref[TEI.setting]{setting} \hyperref[TEI.settingDesc]{settingDesc} \hyperref[TEI.settlement]{settlement} \hyperref[TEI.sex]{sex} \hyperref[TEI.shift]{shift} \hyperref[TEI.sic]{sic} \hyperref[TEI.signatures]{signatures} \hyperref[TEI.signed]{signed} \hyperref[TEI.soCalled]{soCalled} \hyperref[TEI.socecStatus]{socecStatus} \hyperref[TEI.sound]{sound} \hyperref[TEI.source]{source} \hyperref[TEI.sourceDesc]{sourceDesc} \hyperref[TEI.sourceDoc]{sourceDoc} \hyperref[TEI.sp]{sp} \hyperref[TEI.spGrp]{spGrp} \hyperref[TEI.space]{space} \hyperref[TEI.span]{span} \hyperref[TEI.spanGrp]{spanGrp} \hyperref[TEI.speaker]{speaker} \hyperref[TEI.specDesc]{specDesc} \hyperref[TEI.specGrp]{specGrp} \hyperref[TEI.specGrpRef]{specGrpRef} \hyperref[TEI.specList]{specList} \hyperref[TEI.sponsor]{sponsor} \hyperref[TEI.stage]{stage} \hyperref[TEI.stamp]{stamp} \hyperref[TEI.standOff]{standOff} \hyperref[TEI.state]{state} \hyperref[TEI.stdVals]{stdVals} \hyperref[TEI.street]{street} \hyperref[TEI.stress]{stress} \hyperref[TEI.string]{string} \hyperref[TEI.styleDefDecl]{styleDefDecl} \hyperref[TEI.subc]{subc} \hyperref[TEI.subst]{subst} \hyperref[TEI.substJoin]{substJoin} \hyperref[TEI.summary]{summary} \hyperref[TEI.superEntry]{superEntry} \hyperref[TEI.supplied]{supplied} \hyperref[TEI.support]{support} \hyperref[TEI.supportDesc]{supportDesc} \hyperref[TEI.surface]{surface} \hyperref[TEI.surfaceGrp]{surfaceGrp} \hyperref[TEI.surname]{surname} \hyperref[TEI.surplus]{surplus} \hyperref[TEI.surrogates]{surrogates} \hyperref[TEI.syll]{syll} \hyperref[TEI.symbol]{symbol} \hyperref[TEI.table]{table} \hyperref[TEI.tag]{tag} \hyperref[TEI.tagUsage]{tagUsage} \hyperref[TEI.tagsDecl]{tagsDecl} \hyperref[TEI.taxonomy]{taxonomy} \hyperref[TEI.tech]{tech} \hyperref[TEI.teiCorpus]{teiCorpus} \hyperref[TEI.teiHeader]{teiHeader} \hyperref[TEI.term]{term} \hyperref[TEI.terrain]{terrain} \hyperref[TEI.text]{text} \hyperref[TEI.textClass]{textClass} \hyperref[TEI.textDesc]{textDesc} \hyperref[TEI.textLang]{textLang} \hyperref[TEI.textNode]{textNode} \hyperref[TEI.then]{then} \hyperref[TEI.time]{time} \hyperref[TEI.timeline]{timeline} \hyperref[TEI.title]{title} \hyperref[TEI.titlePage]{titlePage} \hyperref[TEI.titlePart]{titlePart} \hyperref[TEI.titleStmt]{titleStmt} \hyperref[TEI.tns]{tns} \hyperref[TEI.trailer]{trailer} \hyperref[TEI.trait]{trait} \hyperref[TEI.transcriptionDesc]{transcriptionDesc} \hyperref[TEI.transpose]{transpose} \hyperref[TEI.tree]{tree} \hyperref[TEI.triangle]{triangle} \hyperref[TEI.typeDesc]{typeDesc} \hyperref[TEI.typeNote]{typeNote} \hyperref[TEI.u]{u} \hyperref[TEI.unclear]{unclear} \hyperref[TEI.undo]{undo} \hyperref[TEI.unicodeName]{unicodeName} \hyperref[TEI.unicodeProp]{unicodeProp} \hyperref[TEI.unihanProp]{unihanProp} \hyperref[TEI.unit]{unit} \hyperref[TEI.unitDecl]{unitDecl} \hyperref[TEI.unitDef]{unitDef} \hyperref[TEI.usg]{usg} \hyperref[TEI.vAlt]{vAlt} \hyperref[TEI.vColl]{vColl} \hyperref[TEI.vDefault]{vDefault} \hyperref[TEI.vLabel]{vLabel} \hyperref[TEI.vMerge]{vMerge} \hyperref[TEI.vNot]{vNot} \hyperref[TEI.vRange]{vRange} \hyperref[TEI.val]{val} \hyperref[TEI.valDesc]{valDesc} \hyperref[TEI.valItem]{valItem} \hyperref[TEI.valList]{valList} \hyperref[TEI.value]{value} \hyperref[TEI.variantEncoding]{variantEncoding} \hyperref[TEI.view]{view} \hyperref[TEI.vocal]{vocal} \hyperref[TEI.w]{w} \hyperref[TEI.watermark]{watermark} \hyperref[TEI.when]{when} \hyperref[TEI.width]{width} \hyperref[TEI.wit]{wit} \hyperref[TEI.witDetail]{witDetail} \hyperref[TEI.witEnd]{witEnd} \hyperref[TEI.witStart]{witStart} \hyperref[TEI.witness]{witness} \hyperref[TEI.writing]{writing} \hyperref[TEI.xenoData]{xenoData} \hyperref[TEI.xr]{xr} \hyperref[TEI.zone]{zone}]
    \item[{Attributes}]
  Attributes\hfil\\[-10pt]\begin{sansreflist}
    \item[@change]
  points to one or more \hyperref[TEI.change]{<change>} elements documenting a state or revision campaign to which the element bearing this attribute and its children have been assigned by the encoder.
\begin{reflist}
    \item[{Status}]
  Optional
    \item[{Datatype}]
  1–∞ occurrences of \hyperref[TEI.teidata.pointer]{teidata.pointer} separated by whitespace
\end{reflist}  
\end{sansreflist}  
\end{reflist}  
\begin{reflist}
\item[]\begin{specHead}{TEI.att.global.facs}{att.global.facs} provides an attribute used to express correspondence between an element containing transcribed text and all or part of an image representing that text. [\textit{\hyperref[PHFAX]{11.1.\ Digital Facsimiles}}]\end{specHead} 
    \item[{Module}]
  transcr — \hyperref[PH]{Representation of Primary Sources}
    \item[{Members}]
  \hyperref[TEI.att.global]{att.global}[\hyperref[TEI.TEI]{TEI} \hyperref[TEI.ab]{ab} \hyperref[TEI.abbr]{abbr} \hyperref[TEI.abstract]{abstract} \hyperref[TEI.accMat]{accMat} \hyperref[TEI.acquisition]{acquisition} \hyperref[TEI.activity]{activity} \hyperref[TEI.actor]{actor} \hyperref[TEI.add]{add} \hyperref[TEI.addName]{addName} \hyperref[TEI.addSpan]{addSpan} \hyperref[TEI.additional]{additional} \hyperref[TEI.additions]{additions} \hyperref[TEI.addrLine]{addrLine} \hyperref[TEI.address]{address} \hyperref[TEI.adminInfo]{adminInfo} \hyperref[TEI.affiliation]{affiliation} \hyperref[TEI.age]{age} \hyperref[TEI.alt]{alt} \hyperref[TEI.altGrp]{altGrp} \hyperref[TEI.altIdent]{altIdent} \hyperref[TEI.altIdentifier]{altIdentifier} \hyperref[TEI.alternate]{alternate} \hyperref[TEI.am]{am} \hyperref[TEI.analytic]{analytic} \hyperref[TEI.anchor]{anchor} \hyperref[TEI.annotation]{annotation} \hyperref[TEI.annotationBlock]{annotationBlock} \hyperref[TEI.anyElement]{anyElement} \hyperref[TEI.app]{app} \hyperref[TEI.appInfo]{appInfo} \hyperref[TEI.application]{application} \hyperref[TEI.arc]{arc} \hyperref[TEI.argument]{argument} \hyperref[TEI.att]{att} \hyperref[TEI.attDef]{attDef} \hyperref[TEI.attList]{attList} \hyperref[TEI.attRef]{attRef} \hyperref[TEI.author]{author} \hyperref[TEI.authority]{authority} \hyperref[TEI.availability]{availability} \hyperref[TEI.back]{back} \hyperref[TEI.bibl]{bibl} \hyperref[TEI.biblFull]{biblFull} \hyperref[TEI.biblScope]{biblScope} \hyperref[TEI.biblStruct]{biblStruct} \hyperref[TEI.bicond]{bicond} \hyperref[TEI.binary]{binary} \hyperref[TEI.binaryObject]{binaryObject} \hyperref[TEI.binding]{binding} \hyperref[TEI.bindingDesc]{bindingDesc} \hyperref[TEI.birth]{birth} \hyperref[TEI.bloc]{bloc} \hyperref[TEI.body]{body} \hyperref[TEI.broadcast]{broadcast} \hyperref[TEI.byline]{byline} \hyperref[TEI.c]{c} \hyperref[TEI.cRefPattern]{cRefPattern} \hyperref[TEI.caesura]{caesura} \hyperref[TEI.calendar]{calendar} \hyperref[TEI.calendarDesc]{calendarDesc} \hyperref[TEI.camera]{camera} \hyperref[TEI.caption]{caption} \hyperref[TEI.case]{case} \hyperref[TEI.castGroup]{castGroup} \hyperref[TEI.castItem]{castItem} \hyperref[TEI.castList]{castList} \hyperref[TEI.catDesc]{catDesc} \hyperref[TEI.catRef]{catRef} \hyperref[TEI.catchwords]{catchwords} \hyperref[TEI.category]{category} \hyperref[TEI.cb]{cb} \hyperref[TEI.cell]{cell} \hyperref[TEI.certainty]{certainty} \hyperref[TEI.change]{change} \hyperref[TEI.channel]{channel} \hyperref[TEI.char]{char} \hyperref[TEI.charDecl]{charDecl} \hyperref[TEI.charName]{charName} \hyperref[TEI.charProp]{charProp} \hyperref[TEI.choice]{choice} \hyperref[TEI.cit]{cit} \hyperref[TEI.citeData]{citeData} \hyperref[TEI.citeStructure]{citeStructure} \hyperref[TEI.citedRange]{citedRange} \hyperref[TEI.cl]{cl} \hyperref[TEI.classCode]{classCode} \hyperref[TEI.classDecl]{classDecl} \hyperref[TEI.classRef]{classRef} \hyperref[TEI.classSpec]{classSpec} \hyperref[TEI.classes]{classes} \hyperref[TEI.climate]{climate} \hyperref[TEI.closer]{closer} \hyperref[TEI.code]{code} \hyperref[TEI.collation]{collation} \hyperref[TEI.collection]{collection} \hyperref[TEI.colloc]{colloc} \hyperref[TEI.colophon]{colophon} \hyperref[TEI.cond]{cond} \hyperref[TEI.condition]{condition} \hyperref[TEI.constitution]{constitution} \hyperref[TEI.constraint]{constraint} \hyperref[TEI.constraintSpec]{constraintSpec} \hyperref[TEI.content]{content} \hyperref[TEI.conversion]{conversion} \hyperref[TEI.corr]{corr} \hyperref[TEI.correction]{correction} \hyperref[TEI.correspAction]{correspAction} \hyperref[TEI.correspContext]{correspContext} \hyperref[TEI.correspDesc]{correspDesc} \hyperref[TEI.country]{country} \hyperref[TEI.creation]{creation} \hyperref[TEI.custEvent]{custEvent} \hyperref[TEI.custodialHist]{custodialHist} \hyperref[TEI.damage]{damage} \hyperref[TEI.damageSpan]{damageSpan} \hyperref[TEI.dataFacet]{dataFacet} \hyperref[TEI.dataRef]{dataRef} \hyperref[TEI.dataSpec]{dataSpec} \hyperref[TEI.datatype]{datatype} \hyperref[TEI.date]{date} \hyperref[TEI.dateline]{dateline} \hyperref[TEI.death]{death} \hyperref[TEI.decoDesc]{decoDesc} \hyperref[TEI.decoNote]{decoNote} \hyperref[TEI.def]{def} \hyperref[TEI.default]{default} \hyperref[TEI.defaultVal]{defaultVal} \hyperref[TEI.del]{del} \hyperref[TEI.delSpan]{delSpan} \hyperref[TEI.depth]{depth} \hyperref[TEI.derivation]{derivation} \hyperref[TEI.desc]{desc} \hyperref[TEI.dictScrap]{dictScrap} \hyperref[TEI.dim]{dim} \hyperref[TEI.dimensions]{dimensions} \hyperref[TEI.distinct]{distinct} \hyperref[TEI.distributor]{distributor} \hyperref[TEI.district]{district} \hyperref[TEI.div]{div} \hyperref[TEI.div1]{div1} \hyperref[TEI.div2]{div2} \hyperref[TEI.div3]{div3} \hyperref[TEI.div4]{div4} \hyperref[TEI.div5]{div5} \hyperref[TEI.div6]{div6} \hyperref[TEI.div7]{div7} \hyperref[TEI.divGen]{divGen} \hyperref[TEI.docAuthor]{docAuthor} \hyperref[TEI.docDate]{docDate} \hyperref[TEI.docEdition]{docEdition} \hyperref[TEI.docImprint]{docImprint} \hyperref[TEI.docTitle]{docTitle} \hyperref[TEI.domain]{domain} \hyperref[TEI.eLeaf]{eLeaf} \hyperref[TEI.eTree]{eTree} \hyperref[TEI.edition]{edition} \hyperref[TEI.editionStmt]{editionStmt} \hyperref[TEI.editor]{editor} \hyperref[TEI.editorialDecl]{editorialDecl} \hyperref[TEI.education]{education} \hyperref[TEI.eg]{eg} \hyperref[TEI.egXML]{egXML} \hyperref[TEI.elementRef]{elementRef} \hyperref[TEI.elementSpec]{elementSpec} \hyperref[TEI.email]{email} \hyperref[TEI.emph]{emph} \hyperref[TEI.empty]{empty} \hyperref[TEI.encodingDesc]{encodingDesc} \hyperref[TEI.entry]{entry} \hyperref[TEI.entryFree]{entryFree} \hyperref[TEI.epigraph]{epigraph} \hyperref[TEI.epilogue]{epilogue} \hyperref[TEI.equipment]{equipment} \hyperref[TEI.equiv]{equiv} \hyperref[TEI.etym]{etym} \hyperref[TEI.event]{event} \hyperref[TEI.ex]{ex} \hyperref[TEI.exemplum]{exemplum} \hyperref[TEI.expan]{expan} \hyperref[TEI.explicit]{explicit} \hyperref[TEI.extent]{extent} \hyperref[TEI.f]{f} \hyperref[TEI.fDecl]{fDecl} \hyperref[TEI.fDescr]{fDescr} \hyperref[TEI.fLib]{fLib} \hyperref[TEI.facsimile]{facsimile} \hyperref[TEI.factuality]{factuality} \hyperref[TEI.faith]{faith} \hyperref[TEI.figDesc]{figDesc} \hyperref[TEI.figure]{figure} \hyperref[TEI.fileDesc]{fileDesc} \hyperref[TEI.filiation]{filiation} \hyperref[TEI.finalRubric]{finalRubric} \hyperref[TEI.floatingText]{floatingText} \hyperref[TEI.floruit]{floruit} \hyperref[TEI.foliation]{foliation} \hyperref[TEI.foreign]{foreign} \hyperref[TEI.forename]{forename} \hyperref[TEI.forest]{forest} \hyperref[TEI.form]{form} \hyperref[TEI.formula]{formula} \hyperref[TEI.front]{front} \hyperref[TEI.fs]{fs} \hyperref[TEI.fsConstraints]{fsConstraints} \hyperref[TEI.fsDecl]{fsDecl} \hyperref[TEI.fsDescr]{fsDescr} \hyperref[TEI.fsdDecl]{fsdDecl} \hyperref[TEI.fsdLink]{fsdLink} \hyperref[TEI.funder]{funder} \hyperref[TEI.fvLib]{fvLib} \hyperref[TEI.fw]{fw} \hyperref[TEI.g]{g} \hyperref[TEI.gap]{gap} \hyperref[TEI.gb]{gb} \hyperref[TEI.gen]{gen} \hyperref[TEI.genName]{genName} \hyperref[TEI.geo]{geo} \hyperref[TEI.geoDecl]{geoDecl} \hyperref[TEI.geogFeat]{geogFeat} \hyperref[TEI.geogName]{geogName} \hyperref[TEI.gi]{gi} \hyperref[TEI.gloss]{gloss} \hyperref[TEI.glyph]{glyph} \hyperref[TEI.glyphName]{glyphName} \hyperref[TEI.gram]{gram} \hyperref[TEI.gramGrp]{gramGrp} \hyperref[TEI.graph]{graph} \hyperref[TEI.graphic]{graphic} \hyperref[TEI.group]{group} \hyperref[TEI.handDesc]{handDesc} \hyperref[TEI.handNote]{handNote} \hyperref[TEI.handNotes]{handNotes} \hyperref[TEI.handShift]{handShift} \hyperref[TEI.head]{head} \hyperref[TEI.headItem]{headItem} \hyperref[TEI.headLabel]{headLabel} \hyperref[TEI.height]{height} \hyperref[TEI.heraldry]{heraldry} \hyperref[TEI.hi]{hi} \hyperref[TEI.history]{history} \hyperref[TEI.hom]{hom} \hyperref[TEI.hyph]{hyph} \hyperref[TEI.hyphenation]{hyphenation} \hyperref[TEI.iNode]{iNode} \hyperref[TEI.iType]{iType} \hyperref[TEI.ident]{ident} \hyperref[TEI.idno]{idno} \hyperref[TEI.if]{if} \hyperref[TEI.iff]{iff} \hyperref[TEI.imprimatur]{imprimatur} \hyperref[TEI.imprint]{imprint} \hyperref[TEI.incident]{incident} \hyperref[TEI.incipit]{incipit} \hyperref[TEI.index]{index} \hyperref[TEI.institution]{institution} \hyperref[TEI.interaction]{interaction} \hyperref[TEI.interp]{interp} \hyperref[TEI.interpGrp]{interpGrp} \hyperref[TEI.interpretation]{interpretation} \hyperref[TEI.item]{item} \hyperref[TEI.join]{join} \hyperref[TEI.joinGrp]{joinGrp} \hyperref[TEI.keywords]{keywords} \hyperref[TEI.kinesic]{kinesic} \hyperref[TEI.l]{l} \hyperref[TEI.label]{label} \hyperref[TEI.lacunaEnd]{lacunaEnd} \hyperref[TEI.lacunaStart]{lacunaStart} \hyperref[TEI.lang]{lang} \hyperref[TEI.langKnowledge]{langKnowledge} \hyperref[TEI.langKnown]{langKnown} \hyperref[TEI.langUsage]{langUsage} \hyperref[TEI.language]{language} \hyperref[TEI.layout]{layout} \hyperref[TEI.layoutDesc]{layoutDesc} \hyperref[TEI.lb]{lb} \hyperref[TEI.lbl]{lbl} \hyperref[TEI.leaf]{leaf} \hyperref[TEI.lem]{lem} \hyperref[TEI.lg]{lg} \hyperref[TEI.licence]{licence} \hyperref[TEI.line]{line} \hyperref[TEI.link]{link} \hyperref[TEI.linkGrp]{linkGrp} \hyperref[TEI.list]{list} \hyperref[TEI.listAnnotation]{listAnnotation} \hyperref[TEI.listApp]{listApp} \hyperref[TEI.listBibl]{listBibl} \hyperref[TEI.listChange]{listChange} \hyperref[TEI.listEvent]{listEvent} \hyperref[TEI.listForest]{listForest} \hyperref[TEI.listNym]{listNym} \hyperref[TEI.listObject]{listObject} \hyperref[TEI.listOrg]{listOrg} \hyperref[TEI.listPerson]{listPerson} \hyperref[TEI.listPlace]{listPlace} \hyperref[TEI.listPrefixDef]{listPrefixDef} \hyperref[TEI.listRef]{listRef} \hyperref[TEI.listRelation]{listRelation} \hyperref[TEI.listTranspose]{listTranspose} \hyperref[TEI.listWit]{listWit} \hyperref[TEI.localName]{localName} \hyperref[TEI.localProp]{localProp} \hyperref[TEI.locale]{locale} \hyperref[TEI.location]{location} \hyperref[TEI.locus]{locus} \hyperref[TEI.locusGrp]{locusGrp} \hyperref[TEI.m]{m} \hyperref[TEI.macroRef]{macroRef} \hyperref[TEI.macroSpec]{macroSpec} \hyperref[TEI.mapping]{mapping} \hyperref[TEI.material]{material} \hyperref[TEI.measure]{measure} \hyperref[TEI.measureGrp]{measureGrp} \hyperref[TEI.media]{media} \hyperref[TEI.meeting]{meeting} \hyperref[TEI.memberOf]{memberOf} \hyperref[TEI.mentioned]{mentioned} \hyperref[TEI.metDecl]{metDecl} \hyperref[TEI.metSym]{metSym} \hyperref[TEI.metamark]{metamark} \hyperref[TEI.milestone]{milestone} \hyperref[TEI.mod]{mod} \hyperref[TEI.model]{model} \hyperref[TEI.modelGrp]{modelGrp} \hyperref[TEI.modelSequence]{modelSequence} \hyperref[TEI.moduleRef]{moduleRef} \hyperref[TEI.moduleSpec]{moduleSpec} \hyperref[TEI.monogr]{monogr} \hyperref[TEI.mood]{mood} \hyperref[TEI.move]{move} \hyperref[TEI.msContents]{msContents} \hyperref[TEI.msDesc]{msDesc} \hyperref[TEI.msFrag]{msFrag} \hyperref[TEI.msIdentifier]{msIdentifier} \hyperref[TEI.msItem]{msItem} \hyperref[TEI.msItemStruct]{msItemStruct} \hyperref[TEI.msName]{msName} \hyperref[TEI.msPart]{msPart} \hyperref[TEI.musicNotation]{musicNotation} \hyperref[TEI.name]{name} \hyperref[TEI.nameLink]{nameLink} \hyperref[TEI.namespace]{namespace} \hyperref[TEI.nationality]{nationality} \hyperref[TEI.node]{node} \hyperref[TEI.normalization]{normalization} \hyperref[TEI.notatedMusic]{notatedMusic} \hyperref[TEI.note]{note} \hyperref[TEI.noteGrp]{noteGrp} \hyperref[TEI.notesStmt]{notesStmt} \hyperref[TEI.num]{num} \hyperref[TEI.number]{number} \hyperref[TEI.numeric]{numeric} \hyperref[TEI.nym]{nym} \hyperref[TEI.oRef]{oRef} \hyperref[TEI.object]{object} \hyperref[TEI.objectDesc]{objectDesc} \hyperref[TEI.objectIdentifier]{objectIdentifier} \hyperref[TEI.objectName]{objectName} \hyperref[TEI.objectType]{objectType} \hyperref[TEI.occupation]{occupation} \hyperref[TEI.offset]{offset} \hyperref[TEI.opener]{opener} \hyperref[TEI.org]{org} \hyperref[TEI.orgName]{orgName} \hyperref[TEI.orig]{orig} \hyperref[TEI.origDate]{origDate} \hyperref[TEI.origPlace]{origPlace} \hyperref[TEI.origin]{origin} \hyperref[TEI.orth]{orth} \hyperref[TEI.outputRendition]{outputRendition} \hyperref[TEI.p]{p} \hyperref[TEI.pRef]{pRef} \hyperref[TEI.param]{param} \hyperref[TEI.paramList]{paramList} \hyperref[TEI.paramSpec]{paramSpec} \hyperref[TEI.particDesc]{particDesc} \hyperref[TEI.path]{path} \hyperref[TEI.pause]{pause} \hyperref[TEI.pb]{pb} \hyperref[TEI.pc]{pc} \hyperref[TEI.per]{per} \hyperref[TEI.performance]{performance} \hyperref[TEI.persName]{persName} \hyperref[TEI.persPronouns]{persPronouns} \hyperref[TEI.person]{person} \hyperref[TEI.personGrp]{personGrp} \hyperref[TEI.persona]{persona} \hyperref[TEI.phr]{phr} \hyperref[TEI.physDesc]{physDesc} \hyperref[TEI.place]{place} \hyperref[TEI.placeName]{placeName} \hyperref[TEI.population]{population} \hyperref[TEI.pos]{pos} \hyperref[TEI.postBox]{postBox} \hyperref[TEI.postCode]{postCode} \hyperref[TEI.postscript]{postscript} \hyperref[TEI.precision]{precision} \hyperref[TEI.prefixDef]{prefixDef} \hyperref[TEI.preparedness]{preparedness} \hyperref[TEI.principal]{principal} \hyperref[TEI.profileDesc]{profileDesc} \hyperref[TEI.projectDesc]{projectDesc} \hyperref[TEI.prologue]{prologue} \hyperref[TEI.pron]{pron} \hyperref[TEI.provenance]{provenance} \hyperref[TEI.ptr]{ptr} \hyperref[TEI.pubPlace]{pubPlace} \hyperref[TEI.publicationStmt]{publicationStmt} \hyperref[TEI.publisher]{publisher} \hyperref[TEI.punctuation]{punctuation} \hyperref[TEI.purpose]{purpose} \hyperref[TEI.q]{q} \hyperref[TEI.quotation]{quotation} \hyperref[TEI.quote]{quote} \hyperref[TEI.rb]{rb} \hyperref[TEI.rdg]{rdg} \hyperref[TEI.rdgGrp]{rdgGrp} \hyperref[TEI.re]{re} \hyperref[TEI.recordHist]{recordHist} \hyperref[TEI.recording]{recording} \hyperref[TEI.recordingStmt]{recordingStmt} \hyperref[TEI.redo]{redo} \hyperref[TEI.ref]{ref} \hyperref[TEI.refState]{refState} \hyperref[TEI.refsDecl]{refsDecl} \hyperref[TEI.reg]{reg} \hyperref[TEI.region]{region} \hyperref[TEI.relatedItem]{relatedItem} \hyperref[TEI.relation]{relation} \hyperref[TEI.remarks]{remarks} \hyperref[TEI.rendition]{rendition} \hyperref[TEI.repository]{repository} \hyperref[TEI.residence]{residence} \hyperref[TEI.resp]{resp} \hyperref[TEI.respStmt]{respStmt} \hyperref[TEI.respons]{respons} \hyperref[TEI.restore]{restore} \hyperref[TEI.retrace]{retrace} \hyperref[TEI.revisionDesc]{revisionDesc} \hyperref[TEI.rhyme]{rhyme} \hyperref[TEI.role]{role} \hyperref[TEI.roleDesc]{roleDesc} \hyperref[TEI.roleName]{roleName} \hyperref[TEI.root]{root} \hyperref[TEI.row]{row} \hyperref[TEI.rs]{rs} \hyperref[TEI.rt]{rt} \hyperref[TEI.rubric]{rubric} \hyperref[TEI.ruby]{ruby} \hyperref[TEI.s]{s} \hyperref[TEI.said]{said} \hyperref[TEI.salute]{salute} \hyperref[TEI.samplingDecl]{samplingDecl} \hyperref[TEI.schemaRef]{schemaRef} \hyperref[TEI.schemaSpec]{schemaSpec} \hyperref[TEI.scriptDesc]{scriptDesc} \hyperref[TEI.scriptNote]{scriptNote} \hyperref[TEI.scriptStmt]{scriptStmt} \hyperref[TEI.seal]{seal} \hyperref[TEI.sealDesc]{sealDesc} \hyperref[TEI.secFol]{secFol} \hyperref[TEI.secl]{secl} \hyperref[TEI.seg]{seg} \hyperref[TEI.segmentation]{segmentation} \hyperref[TEI.sense]{sense} \hyperref[TEI.sequence]{sequence} \hyperref[TEI.series]{series} \hyperref[TEI.seriesStmt]{seriesStmt} \hyperref[TEI.set]{set} \hyperref[TEI.setting]{setting} \hyperref[TEI.settingDesc]{settingDesc} \hyperref[TEI.settlement]{settlement} \hyperref[TEI.sex]{sex} \hyperref[TEI.shift]{shift} \hyperref[TEI.sic]{sic} \hyperref[TEI.signatures]{signatures} \hyperref[TEI.signed]{signed} \hyperref[TEI.soCalled]{soCalled} \hyperref[TEI.socecStatus]{socecStatus} \hyperref[TEI.sound]{sound} \hyperref[TEI.source]{source} \hyperref[TEI.sourceDesc]{sourceDesc} \hyperref[TEI.sourceDoc]{sourceDoc} \hyperref[TEI.sp]{sp} \hyperref[TEI.spGrp]{spGrp} \hyperref[TEI.space]{space} \hyperref[TEI.span]{span} \hyperref[TEI.spanGrp]{spanGrp} \hyperref[TEI.speaker]{speaker} \hyperref[TEI.specDesc]{specDesc} \hyperref[TEI.specGrp]{specGrp} \hyperref[TEI.specGrpRef]{specGrpRef} \hyperref[TEI.specList]{specList} \hyperref[TEI.sponsor]{sponsor} \hyperref[TEI.stage]{stage} \hyperref[TEI.stamp]{stamp} \hyperref[TEI.standOff]{standOff} \hyperref[TEI.state]{state} \hyperref[TEI.stdVals]{stdVals} \hyperref[TEI.street]{street} \hyperref[TEI.stress]{stress} \hyperref[TEI.string]{string} \hyperref[TEI.styleDefDecl]{styleDefDecl} \hyperref[TEI.subc]{subc} \hyperref[TEI.subst]{subst} \hyperref[TEI.substJoin]{substJoin} \hyperref[TEI.summary]{summary} \hyperref[TEI.superEntry]{superEntry} \hyperref[TEI.supplied]{supplied} \hyperref[TEI.support]{support} \hyperref[TEI.supportDesc]{supportDesc} \hyperref[TEI.surface]{surface} \hyperref[TEI.surfaceGrp]{surfaceGrp} \hyperref[TEI.surname]{surname} \hyperref[TEI.surplus]{surplus} \hyperref[TEI.surrogates]{surrogates} \hyperref[TEI.syll]{syll} \hyperref[TEI.symbol]{symbol} \hyperref[TEI.table]{table} \hyperref[TEI.tag]{tag} \hyperref[TEI.tagUsage]{tagUsage} \hyperref[TEI.tagsDecl]{tagsDecl} \hyperref[TEI.taxonomy]{taxonomy} \hyperref[TEI.tech]{tech} \hyperref[TEI.teiCorpus]{teiCorpus} \hyperref[TEI.teiHeader]{teiHeader} \hyperref[TEI.term]{term} \hyperref[TEI.terrain]{terrain} \hyperref[TEI.text]{text} \hyperref[TEI.textClass]{textClass} \hyperref[TEI.textDesc]{textDesc} \hyperref[TEI.textLang]{textLang} \hyperref[TEI.textNode]{textNode} \hyperref[TEI.then]{then} \hyperref[TEI.time]{time} \hyperref[TEI.timeline]{timeline} \hyperref[TEI.title]{title} \hyperref[TEI.titlePage]{titlePage} \hyperref[TEI.titlePart]{titlePart} \hyperref[TEI.titleStmt]{titleStmt} \hyperref[TEI.tns]{tns} \hyperref[TEI.trailer]{trailer} \hyperref[TEI.trait]{trait} \hyperref[TEI.transcriptionDesc]{transcriptionDesc} \hyperref[TEI.transpose]{transpose} \hyperref[TEI.tree]{tree} \hyperref[TEI.triangle]{triangle} \hyperref[TEI.typeDesc]{typeDesc} \hyperref[TEI.typeNote]{typeNote} \hyperref[TEI.u]{u} \hyperref[TEI.unclear]{unclear} \hyperref[TEI.undo]{undo} \hyperref[TEI.unicodeName]{unicodeName} \hyperref[TEI.unicodeProp]{unicodeProp} \hyperref[TEI.unihanProp]{unihanProp} \hyperref[TEI.unit]{unit} \hyperref[TEI.unitDecl]{unitDecl} \hyperref[TEI.unitDef]{unitDef} \hyperref[TEI.usg]{usg} \hyperref[TEI.vAlt]{vAlt} \hyperref[TEI.vColl]{vColl} \hyperref[TEI.vDefault]{vDefault} \hyperref[TEI.vLabel]{vLabel} \hyperref[TEI.vMerge]{vMerge} \hyperref[TEI.vNot]{vNot} \hyperref[TEI.vRange]{vRange} \hyperref[TEI.val]{val} \hyperref[TEI.valDesc]{valDesc} \hyperref[TEI.valItem]{valItem} \hyperref[TEI.valList]{valList} \hyperref[TEI.value]{value} \hyperref[TEI.variantEncoding]{variantEncoding} \hyperref[TEI.view]{view} \hyperref[TEI.vocal]{vocal} \hyperref[TEI.w]{w} \hyperref[TEI.watermark]{watermark} \hyperref[TEI.when]{when} \hyperref[TEI.width]{width} \hyperref[TEI.wit]{wit} \hyperref[TEI.witDetail]{witDetail} \hyperref[TEI.witEnd]{witEnd} \hyperref[TEI.witStart]{witStart} \hyperref[TEI.witness]{witness} \hyperref[TEI.writing]{writing} \hyperref[TEI.xenoData]{xenoData} \hyperref[TEI.xr]{xr} \hyperref[TEI.zone]{zone}]
    \item[{Attributes}]
  Attributes\hfil\\[-10pt]\begin{sansreflist}
    \item[@facs]
  (facsimile) points to all or part of an image which corresponds with the content of the element.
\begin{reflist}
    \item[{Status}]
  Optional
    \item[{Datatype}]
  1–∞ occurrences of \hyperref[TEI.teidata.pointer]{teidata.pointer} separated by whitespace
\end{reflist}  
\end{sansreflist}  
\end{reflist}  
\begin{reflist}
\item[]\begin{specHead}{TEI.att.global.linking}{att.global.linking} provides a set of attributes for hypertextual linking. [\textit{\hyperref[SA]{16.\ Linking, Segmentation, and Alignment}}]\end{specHead} 
    \item[{Module}]
  linking — \hyperref[SA]{Linking, Segmentation, and Alignment}
    \item[{Members}]
  \hyperref[TEI.att.global]{att.global}[\hyperref[TEI.TEI]{TEI} \hyperref[TEI.ab]{ab} \hyperref[TEI.abbr]{abbr} \hyperref[TEI.abstract]{abstract} \hyperref[TEI.accMat]{accMat} \hyperref[TEI.acquisition]{acquisition} \hyperref[TEI.activity]{activity} \hyperref[TEI.actor]{actor} \hyperref[TEI.add]{add} \hyperref[TEI.addName]{addName} \hyperref[TEI.addSpan]{addSpan} \hyperref[TEI.additional]{additional} \hyperref[TEI.additions]{additions} \hyperref[TEI.addrLine]{addrLine} \hyperref[TEI.address]{address} \hyperref[TEI.adminInfo]{adminInfo} \hyperref[TEI.affiliation]{affiliation} \hyperref[TEI.age]{age} \hyperref[TEI.alt]{alt} \hyperref[TEI.altGrp]{altGrp} \hyperref[TEI.altIdent]{altIdent} \hyperref[TEI.altIdentifier]{altIdentifier} \hyperref[TEI.alternate]{alternate} \hyperref[TEI.am]{am} \hyperref[TEI.analytic]{analytic} \hyperref[TEI.anchor]{anchor} \hyperref[TEI.annotation]{annotation} \hyperref[TEI.annotationBlock]{annotationBlock} \hyperref[TEI.anyElement]{anyElement} \hyperref[TEI.app]{app} \hyperref[TEI.appInfo]{appInfo} \hyperref[TEI.application]{application} \hyperref[TEI.arc]{arc} \hyperref[TEI.argument]{argument} \hyperref[TEI.att]{att} \hyperref[TEI.attDef]{attDef} \hyperref[TEI.attList]{attList} \hyperref[TEI.attRef]{attRef} \hyperref[TEI.author]{author} \hyperref[TEI.authority]{authority} \hyperref[TEI.availability]{availability} \hyperref[TEI.back]{back} \hyperref[TEI.bibl]{bibl} \hyperref[TEI.biblFull]{biblFull} \hyperref[TEI.biblScope]{biblScope} \hyperref[TEI.biblStruct]{biblStruct} \hyperref[TEI.bicond]{bicond} \hyperref[TEI.binary]{binary} \hyperref[TEI.binaryObject]{binaryObject} \hyperref[TEI.binding]{binding} \hyperref[TEI.bindingDesc]{bindingDesc} \hyperref[TEI.birth]{birth} \hyperref[TEI.bloc]{bloc} \hyperref[TEI.body]{body} \hyperref[TEI.broadcast]{broadcast} \hyperref[TEI.byline]{byline} \hyperref[TEI.c]{c} \hyperref[TEI.cRefPattern]{cRefPattern} \hyperref[TEI.caesura]{caesura} \hyperref[TEI.calendar]{calendar} \hyperref[TEI.calendarDesc]{calendarDesc} \hyperref[TEI.camera]{camera} \hyperref[TEI.caption]{caption} \hyperref[TEI.case]{case} \hyperref[TEI.castGroup]{castGroup} \hyperref[TEI.castItem]{castItem} \hyperref[TEI.castList]{castList} \hyperref[TEI.catDesc]{catDesc} \hyperref[TEI.catRef]{catRef} \hyperref[TEI.catchwords]{catchwords} \hyperref[TEI.category]{category} \hyperref[TEI.cb]{cb} \hyperref[TEI.cell]{cell} \hyperref[TEI.certainty]{certainty} \hyperref[TEI.change]{change} \hyperref[TEI.channel]{channel} \hyperref[TEI.char]{char} \hyperref[TEI.charDecl]{charDecl} \hyperref[TEI.charName]{charName} \hyperref[TEI.charProp]{charProp} \hyperref[TEI.choice]{choice} \hyperref[TEI.cit]{cit} \hyperref[TEI.citeData]{citeData} \hyperref[TEI.citeStructure]{citeStructure} \hyperref[TEI.citedRange]{citedRange} \hyperref[TEI.cl]{cl} \hyperref[TEI.classCode]{classCode} \hyperref[TEI.classDecl]{classDecl} \hyperref[TEI.classRef]{classRef} \hyperref[TEI.classSpec]{classSpec} \hyperref[TEI.classes]{classes} \hyperref[TEI.climate]{climate} \hyperref[TEI.closer]{closer} \hyperref[TEI.code]{code} \hyperref[TEI.collation]{collation} \hyperref[TEI.collection]{collection} \hyperref[TEI.colloc]{colloc} \hyperref[TEI.colophon]{colophon} \hyperref[TEI.cond]{cond} \hyperref[TEI.condition]{condition} \hyperref[TEI.constitution]{constitution} \hyperref[TEI.constraint]{constraint} \hyperref[TEI.constraintSpec]{constraintSpec} \hyperref[TEI.content]{content} \hyperref[TEI.conversion]{conversion} \hyperref[TEI.corr]{corr} \hyperref[TEI.correction]{correction} \hyperref[TEI.correspAction]{correspAction} \hyperref[TEI.correspContext]{correspContext} \hyperref[TEI.correspDesc]{correspDesc} \hyperref[TEI.country]{country} \hyperref[TEI.creation]{creation} \hyperref[TEI.custEvent]{custEvent} \hyperref[TEI.custodialHist]{custodialHist} \hyperref[TEI.damage]{damage} \hyperref[TEI.damageSpan]{damageSpan} \hyperref[TEI.dataFacet]{dataFacet} \hyperref[TEI.dataRef]{dataRef} \hyperref[TEI.dataSpec]{dataSpec} \hyperref[TEI.datatype]{datatype} \hyperref[TEI.date]{date} \hyperref[TEI.dateline]{dateline} \hyperref[TEI.death]{death} \hyperref[TEI.decoDesc]{decoDesc} \hyperref[TEI.decoNote]{decoNote} \hyperref[TEI.def]{def} \hyperref[TEI.default]{default} \hyperref[TEI.defaultVal]{defaultVal} \hyperref[TEI.del]{del} \hyperref[TEI.delSpan]{delSpan} \hyperref[TEI.depth]{depth} \hyperref[TEI.derivation]{derivation} \hyperref[TEI.desc]{desc} \hyperref[TEI.dictScrap]{dictScrap} \hyperref[TEI.dim]{dim} \hyperref[TEI.dimensions]{dimensions} \hyperref[TEI.distinct]{distinct} \hyperref[TEI.distributor]{distributor} \hyperref[TEI.district]{district} \hyperref[TEI.div]{div} \hyperref[TEI.div1]{div1} \hyperref[TEI.div2]{div2} \hyperref[TEI.div3]{div3} \hyperref[TEI.div4]{div4} \hyperref[TEI.div5]{div5} \hyperref[TEI.div6]{div6} \hyperref[TEI.div7]{div7} \hyperref[TEI.divGen]{divGen} \hyperref[TEI.docAuthor]{docAuthor} \hyperref[TEI.docDate]{docDate} \hyperref[TEI.docEdition]{docEdition} \hyperref[TEI.docImprint]{docImprint} \hyperref[TEI.docTitle]{docTitle} \hyperref[TEI.domain]{domain} \hyperref[TEI.eLeaf]{eLeaf} \hyperref[TEI.eTree]{eTree} \hyperref[TEI.edition]{edition} \hyperref[TEI.editionStmt]{editionStmt} \hyperref[TEI.editor]{editor} \hyperref[TEI.editorialDecl]{editorialDecl} \hyperref[TEI.education]{education} \hyperref[TEI.eg]{eg} \hyperref[TEI.egXML]{egXML} \hyperref[TEI.elementRef]{elementRef} \hyperref[TEI.elementSpec]{elementSpec} \hyperref[TEI.email]{email} \hyperref[TEI.emph]{emph} \hyperref[TEI.empty]{empty} \hyperref[TEI.encodingDesc]{encodingDesc} \hyperref[TEI.entry]{entry} \hyperref[TEI.entryFree]{entryFree} \hyperref[TEI.epigraph]{epigraph} \hyperref[TEI.epilogue]{epilogue} \hyperref[TEI.equipment]{equipment} \hyperref[TEI.equiv]{equiv} \hyperref[TEI.etym]{etym} \hyperref[TEI.event]{event} \hyperref[TEI.ex]{ex} \hyperref[TEI.exemplum]{exemplum} \hyperref[TEI.expan]{expan} \hyperref[TEI.explicit]{explicit} \hyperref[TEI.extent]{extent} \hyperref[TEI.f]{f} \hyperref[TEI.fDecl]{fDecl} \hyperref[TEI.fDescr]{fDescr} \hyperref[TEI.fLib]{fLib} \hyperref[TEI.facsimile]{facsimile} \hyperref[TEI.factuality]{factuality} \hyperref[TEI.faith]{faith} \hyperref[TEI.figDesc]{figDesc} \hyperref[TEI.figure]{figure} \hyperref[TEI.fileDesc]{fileDesc} \hyperref[TEI.filiation]{filiation} \hyperref[TEI.finalRubric]{finalRubric} \hyperref[TEI.floatingText]{floatingText} \hyperref[TEI.floruit]{floruit} \hyperref[TEI.foliation]{foliation} \hyperref[TEI.foreign]{foreign} \hyperref[TEI.forename]{forename} \hyperref[TEI.forest]{forest} \hyperref[TEI.form]{form} \hyperref[TEI.formula]{formula} \hyperref[TEI.front]{front} \hyperref[TEI.fs]{fs} \hyperref[TEI.fsConstraints]{fsConstraints} \hyperref[TEI.fsDecl]{fsDecl} \hyperref[TEI.fsDescr]{fsDescr} \hyperref[TEI.fsdDecl]{fsdDecl} \hyperref[TEI.fsdLink]{fsdLink} \hyperref[TEI.funder]{funder} \hyperref[TEI.fvLib]{fvLib} \hyperref[TEI.fw]{fw} \hyperref[TEI.g]{g} \hyperref[TEI.gap]{gap} \hyperref[TEI.gb]{gb} \hyperref[TEI.gen]{gen} \hyperref[TEI.genName]{genName} \hyperref[TEI.geo]{geo} \hyperref[TEI.geoDecl]{geoDecl} \hyperref[TEI.geogFeat]{geogFeat} \hyperref[TEI.geogName]{geogName} \hyperref[TEI.gi]{gi} \hyperref[TEI.gloss]{gloss} \hyperref[TEI.glyph]{glyph} \hyperref[TEI.glyphName]{glyphName} \hyperref[TEI.gram]{gram} \hyperref[TEI.gramGrp]{gramGrp} \hyperref[TEI.graph]{graph} \hyperref[TEI.graphic]{graphic} \hyperref[TEI.group]{group} \hyperref[TEI.handDesc]{handDesc} \hyperref[TEI.handNote]{handNote} \hyperref[TEI.handNotes]{handNotes} \hyperref[TEI.handShift]{handShift} \hyperref[TEI.head]{head} \hyperref[TEI.headItem]{headItem} \hyperref[TEI.headLabel]{headLabel} \hyperref[TEI.height]{height} \hyperref[TEI.heraldry]{heraldry} \hyperref[TEI.hi]{hi} \hyperref[TEI.history]{history} \hyperref[TEI.hom]{hom} \hyperref[TEI.hyph]{hyph} \hyperref[TEI.hyphenation]{hyphenation} \hyperref[TEI.iNode]{iNode} \hyperref[TEI.iType]{iType} \hyperref[TEI.ident]{ident} \hyperref[TEI.idno]{idno} \hyperref[TEI.if]{if} \hyperref[TEI.iff]{iff} \hyperref[TEI.imprimatur]{imprimatur} \hyperref[TEI.imprint]{imprint} \hyperref[TEI.incident]{incident} \hyperref[TEI.incipit]{incipit} \hyperref[TEI.index]{index} \hyperref[TEI.institution]{institution} \hyperref[TEI.interaction]{interaction} \hyperref[TEI.interp]{interp} \hyperref[TEI.interpGrp]{interpGrp} \hyperref[TEI.interpretation]{interpretation} \hyperref[TEI.item]{item} \hyperref[TEI.join]{join} \hyperref[TEI.joinGrp]{joinGrp} \hyperref[TEI.keywords]{keywords} \hyperref[TEI.kinesic]{kinesic} \hyperref[TEI.l]{l} \hyperref[TEI.label]{label} \hyperref[TEI.lacunaEnd]{lacunaEnd} \hyperref[TEI.lacunaStart]{lacunaStart} \hyperref[TEI.lang]{lang} \hyperref[TEI.langKnowledge]{langKnowledge} \hyperref[TEI.langKnown]{langKnown} \hyperref[TEI.langUsage]{langUsage} \hyperref[TEI.language]{language} \hyperref[TEI.layout]{layout} \hyperref[TEI.layoutDesc]{layoutDesc} \hyperref[TEI.lb]{lb} \hyperref[TEI.lbl]{lbl} \hyperref[TEI.leaf]{leaf} \hyperref[TEI.lem]{lem} \hyperref[TEI.lg]{lg} \hyperref[TEI.licence]{licence} \hyperref[TEI.line]{line} \hyperref[TEI.link]{link} \hyperref[TEI.linkGrp]{linkGrp} \hyperref[TEI.list]{list} \hyperref[TEI.listAnnotation]{listAnnotation} \hyperref[TEI.listApp]{listApp} \hyperref[TEI.listBibl]{listBibl} \hyperref[TEI.listChange]{listChange} \hyperref[TEI.listEvent]{listEvent} \hyperref[TEI.listForest]{listForest} \hyperref[TEI.listNym]{listNym} \hyperref[TEI.listObject]{listObject} \hyperref[TEI.listOrg]{listOrg} \hyperref[TEI.listPerson]{listPerson} \hyperref[TEI.listPlace]{listPlace} \hyperref[TEI.listPrefixDef]{listPrefixDef} \hyperref[TEI.listRef]{listRef} \hyperref[TEI.listRelation]{listRelation} \hyperref[TEI.listTranspose]{listTranspose} \hyperref[TEI.listWit]{listWit} \hyperref[TEI.localName]{localName} \hyperref[TEI.localProp]{localProp} \hyperref[TEI.locale]{locale} \hyperref[TEI.location]{location} \hyperref[TEI.locus]{locus} \hyperref[TEI.locusGrp]{locusGrp} \hyperref[TEI.m]{m} \hyperref[TEI.macroRef]{macroRef} \hyperref[TEI.macroSpec]{macroSpec} \hyperref[TEI.mapping]{mapping} \hyperref[TEI.material]{material} \hyperref[TEI.measure]{measure} \hyperref[TEI.measureGrp]{measureGrp} \hyperref[TEI.media]{media} \hyperref[TEI.meeting]{meeting} \hyperref[TEI.memberOf]{memberOf} \hyperref[TEI.mentioned]{mentioned} \hyperref[TEI.metDecl]{metDecl} \hyperref[TEI.metSym]{metSym} \hyperref[TEI.metamark]{metamark} \hyperref[TEI.milestone]{milestone} \hyperref[TEI.mod]{mod} \hyperref[TEI.model]{model} \hyperref[TEI.modelGrp]{modelGrp} \hyperref[TEI.modelSequence]{modelSequence} \hyperref[TEI.moduleRef]{moduleRef} \hyperref[TEI.moduleSpec]{moduleSpec} \hyperref[TEI.monogr]{monogr} \hyperref[TEI.mood]{mood} \hyperref[TEI.move]{move} \hyperref[TEI.msContents]{msContents} \hyperref[TEI.msDesc]{msDesc} \hyperref[TEI.msFrag]{msFrag} \hyperref[TEI.msIdentifier]{msIdentifier} \hyperref[TEI.msItem]{msItem} \hyperref[TEI.msItemStruct]{msItemStruct} \hyperref[TEI.msName]{msName} \hyperref[TEI.msPart]{msPart} \hyperref[TEI.musicNotation]{musicNotation} \hyperref[TEI.name]{name} \hyperref[TEI.nameLink]{nameLink} \hyperref[TEI.namespace]{namespace} \hyperref[TEI.nationality]{nationality} \hyperref[TEI.node]{node} \hyperref[TEI.normalization]{normalization} \hyperref[TEI.notatedMusic]{notatedMusic} \hyperref[TEI.note]{note} \hyperref[TEI.noteGrp]{noteGrp} \hyperref[TEI.notesStmt]{notesStmt} \hyperref[TEI.num]{num} \hyperref[TEI.number]{number} \hyperref[TEI.numeric]{numeric} \hyperref[TEI.nym]{nym} \hyperref[TEI.oRef]{oRef} \hyperref[TEI.object]{object} \hyperref[TEI.objectDesc]{objectDesc} \hyperref[TEI.objectIdentifier]{objectIdentifier} \hyperref[TEI.objectName]{objectName} \hyperref[TEI.objectType]{objectType} \hyperref[TEI.occupation]{occupation} \hyperref[TEI.offset]{offset} \hyperref[TEI.opener]{opener} \hyperref[TEI.org]{org} \hyperref[TEI.orgName]{orgName} \hyperref[TEI.orig]{orig} \hyperref[TEI.origDate]{origDate} \hyperref[TEI.origPlace]{origPlace} \hyperref[TEI.origin]{origin} \hyperref[TEI.orth]{orth} \hyperref[TEI.outputRendition]{outputRendition} \hyperref[TEI.p]{p} \hyperref[TEI.pRef]{pRef} \hyperref[TEI.param]{param} \hyperref[TEI.paramList]{paramList} \hyperref[TEI.paramSpec]{paramSpec} \hyperref[TEI.particDesc]{particDesc} \hyperref[TEI.path]{path} \hyperref[TEI.pause]{pause} \hyperref[TEI.pb]{pb} \hyperref[TEI.pc]{pc} \hyperref[TEI.per]{per} \hyperref[TEI.performance]{performance} \hyperref[TEI.persName]{persName} \hyperref[TEI.persPronouns]{persPronouns} \hyperref[TEI.person]{person} \hyperref[TEI.personGrp]{personGrp} \hyperref[TEI.persona]{persona} \hyperref[TEI.phr]{phr} \hyperref[TEI.physDesc]{physDesc} \hyperref[TEI.place]{place} \hyperref[TEI.placeName]{placeName} \hyperref[TEI.population]{population} \hyperref[TEI.pos]{pos} \hyperref[TEI.postBox]{postBox} \hyperref[TEI.postCode]{postCode} \hyperref[TEI.postscript]{postscript} \hyperref[TEI.precision]{precision} \hyperref[TEI.prefixDef]{prefixDef} \hyperref[TEI.preparedness]{preparedness} \hyperref[TEI.principal]{principal} \hyperref[TEI.profileDesc]{profileDesc} \hyperref[TEI.projectDesc]{projectDesc} \hyperref[TEI.prologue]{prologue} \hyperref[TEI.pron]{pron} \hyperref[TEI.provenance]{provenance} \hyperref[TEI.ptr]{ptr} \hyperref[TEI.pubPlace]{pubPlace} \hyperref[TEI.publicationStmt]{publicationStmt} \hyperref[TEI.publisher]{publisher} \hyperref[TEI.punctuation]{punctuation} \hyperref[TEI.purpose]{purpose} \hyperref[TEI.q]{q} \hyperref[TEI.quotation]{quotation} \hyperref[TEI.quote]{quote} \hyperref[TEI.rb]{rb} \hyperref[TEI.rdg]{rdg} \hyperref[TEI.rdgGrp]{rdgGrp} \hyperref[TEI.re]{re} \hyperref[TEI.recordHist]{recordHist} \hyperref[TEI.recording]{recording} \hyperref[TEI.recordingStmt]{recordingStmt} \hyperref[TEI.redo]{redo} \hyperref[TEI.ref]{ref} \hyperref[TEI.refState]{refState} \hyperref[TEI.refsDecl]{refsDecl} \hyperref[TEI.reg]{reg} \hyperref[TEI.region]{region} \hyperref[TEI.relatedItem]{relatedItem} \hyperref[TEI.relation]{relation} \hyperref[TEI.remarks]{remarks} \hyperref[TEI.rendition]{rendition} \hyperref[TEI.repository]{repository} \hyperref[TEI.residence]{residence} \hyperref[TEI.resp]{resp} \hyperref[TEI.respStmt]{respStmt} \hyperref[TEI.respons]{respons} \hyperref[TEI.restore]{restore} \hyperref[TEI.retrace]{retrace} \hyperref[TEI.revisionDesc]{revisionDesc} \hyperref[TEI.rhyme]{rhyme} \hyperref[TEI.role]{role} \hyperref[TEI.roleDesc]{roleDesc} \hyperref[TEI.roleName]{roleName} \hyperref[TEI.root]{root} \hyperref[TEI.row]{row} \hyperref[TEI.rs]{rs} \hyperref[TEI.rt]{rt} \hyperref[TEI.rubric]{rubric} \hyperref[TEI.ruby]{ruby} \hyperref[TEI.s]{s} \hyperref[TEI.said]{said} \hyperref[TEI.salute]{salute} \hyperref[TEI.samplingDecl]{samplingDecl} \hyperref[TEI.schemaRef]{schemaRef} \hyperref[TEI.schemaSpec]{schemaSpec} \hyperref[TEI.scriptDesc]{scriptDesc} \hyperref[TEI.scriptNote]{scriptNote} \hyperref[TEI.scriptStmt]{scriptStmt} \hyperref[TEI.seal]{seal} \hyperref[TEI.sealDesc]{sealDesc} \hyperref[TEI.secFol]{secFol} \hyperref[TEI.secl]{secl} \hyperref[TEI.seg]{seg} \hyperref[TEI.segmentation]{segmentation} \hyperref[TEI.sense]{sense} \hyperref[TEI.sequence]{sequence} \hyperref[TEI.series]{series} \hyperref[TEI.seriesStmt]{seriesStmt} \hyperref[TEI.set]{set} \hyperref[TEI.setting]{setting} \hyperref[TEI.settingDesc]{settingDesc} \hyperref[TEI.settlement]{settlement} \hyperref[TEI.sex]{sex} \hyperref[TEI.shift]{shift} \hyperref[TEI.sic]{sic} \hyperref[TEI.signatures]{signatures} \hyperref[TEI.signed]{signed} \hyperref[TEI.soCalled]{soCalled} \hyperref[TEI.socecStatus]{socecStatus} \hyperref[TEI.sound]{sound} \hyperref[TEI.source]{source} \hyperref[TEI.sourceDesc]{sourceDesc} \hyperref[TEI.sourceDoc]{sourceDoc} \hyperref[TEI.sp]{sp} \hyperref[TEI.spGrp]{spGrp} \hyperref[TEI.space]{space} \hyperref[TEI.span]{span} \hyperref[TEI.spanGrp]{spanGrp} \hyperref[TEI.speaker]{speaker} \hyperref[TEI.specDesc]{specDesc} \hyperref[TEI.specGrp]{specGrp} \hyperref[TEI.specGrpRef]{specGrpRef} \hyperref[TEI.specList]{specList} \hyperref[TEI.sponsor]{sponsor} \hyperref[TEI.stage]{stage} \hyperref[TEI.stamp]{stamp} \hyperref[TEI.standOff]{standOff} \hyperref[TEI.state]{state} \hyperref[TEI.stdVals]{stdVals} \hyperref[TEI.street]{street} \hyperref[TEI.stress]{stress} \hyperref[TEI.string]{string} \hyperref[TEI.styleDefDecl]{styleDefDecl} \hyperref[TEI.subc]{subc} \hyperref[TEI.subst]{subst} \hyperref[TEI.substJoin]{substJoin} \hyperref[TEI.summary]{summary} \hyperref[TEI.superEntry]{superEntry} \hyperref[TEI.supplied]{supplied} \hyperref[TEI.support]{support} \hyperref[TEI.supportDesc]{supportDesc} \hyperref[TEI.surface]{surface} \hyperref[TEI.surfaceGrp]{surfaceGrp} \hyperref[TEI.surname]{surname} \hyperref[TEI.surplus]{surplus} \hyperref[TEI.surrogates]{surrogates} \hyperref[TEI.syll]{syll} \hyperref[TEI.symbol]{symbol} \hyperref[TEI.table]{table} \hyperref[TEI.tag]{tag} \hyperref[TEI.tagUsage]{tagUsage} \hyperref[TEI.tagsDecl]{tagsDecl} \hyperref[TEI.taxonomy]{taxonomy} \hyperref[TEI.tech]{tech} \hyperref[TEI.teiCorpus]{teiCorpus} \hyperref[TEI.teiHeader]{teiHeader} \hyperref[TEI.term]{term} \hyperref[TEI.terrain]{terrain} \hyperref[TEI.text]{text} \hyperref[TEI.textClass]{textClass} \hyperref[TEI.textDesc]{textDesc} \hyperref[TEI.textLang]{textLang} \hyperref[TEI.textNode]{textNode} \hyperref[TEI.then]{then} \hyperref[TEI.time]{time} \hyperref[TEI.timeline]{timeline} \hyperref[TEI.title]{title} \hyperref[TEI.titlePage]{titlePage} \hyperref[TEI.titlePart]{titlePart} \hyperref[TEI.titleStmt]{titleStmt} \hyperref[TEI.tns]{tns} \hyperref[TEI.trailer]{trailer} \hyperref[TEI.trait]{trait} \hyperref[TEI.transcriptionDesc]{transcriptionDesc} \hyperref[TEI.transpose]{transpose} \hyperref[TEI.tree]{tree} \hyperref[TEI.triangle]{triangle} \hyperref[TEI.typeDesc]{typeDesc} \hyperref[TEI.typeNote]{typeNote} \hyperref[TEI.u]{u} \hyperref[TEI.unclear]{unclear} \hyperref[TEI.undo]{undo} \hyperref[TEI.unicodeName]{unicodeName} \hyperref[TEI.unicodeProp]{unicodeProp} \hyperref[TEI.unihanProp]{unihanProp} \hyperref[TEI.unit]{unit} \hyperref[TEI.unitDecl]{unitDecl} \hyperref[TEI.unitDef]{unitDef} \hyperref[TEI.usg]{usg} \hyperref[TEI.vAlt]{vAlt} \hyperref[TEI.vColl]{vColl} \hyperref[TEI.vDefault]{vDefault} \hyperref[TEI.vLabel]{vLabel} \hyperref[TEI.vMerge]{vMerge} \hyperref[TEI.vNot]{vNot} \hyperref[TEI.vRange]{vRange} \hyperref[TEI.val]{val} \hyperref[TEI.valDesc]{valDesc} \hyperref[TEI.valItem]{valItem} \hyperref[TEI.valList]{valList} \hyperref[TEI.value]{value} \hyperref[TEI.variantEncoding]{variantEncoding} \hyperref[TEI.view]{view} \hyperref[TEI.vocal]{vocal} \hyperref[TEI.w]{w} \hyperref[TEI.watermark]{watermark} \hyperref[TEI.when]{when} \hyperref[TEI.width]{width} \hyperref[TEI.wit]{wit} \hyperref[TEI.witDetail]{witDetail} \hyperref[TEI.witEnd]{witEnd} \hyperref[TEI.witStart]{witStart} \hyperref[TEI.witness]{witness} \hyperref[TEI.writing]{writing} \hyperref[TEI.xenoData]{xenoData} \hyperref[TEI.xr]{xr} \hyperref[TEI.zone]{zone}]
    \item[{Attributes}]
  Attributes\hfil\\[-10pt]\begin{sansreflist}
    \item[@corresp]
  (corresponds) points to elements that correspond to the current element in some way.
\begin{reflist}
    \item[{Status}]
  Optional
    \item[{Datatype}]
  1–∞ occurrences of \hyperref[TEI.teidata.pointer]{teidata.pointer} separated by whitespace
    \item[]\index{group=<group>|exampleindex}\index{text=<text>|exampleindex}\index{body=<body>|exampleindex}\index{div=<div>|exampleindex}\index{type=@type!<div>|exampleindex}\index{head=<head>|exampleindex}\index{p=<p>|exampleindex}\index{text=<text>|exampleindex}\index{body=<body>|exampleindex}\index{corresp=@corresp!<body>|exampleindex}\index{div=<div>|exampleindex}\index{type=@type!<div>|exampleindex}\index{head=<head>|exampleindex}\index{p=<p>|exampleindex}\exampleFont {<\textbf{group}>}\mbox{}\newline 
\hspace*{1em}{<\textbf{text}\hspace*{1em}{xml:id}="{t1-g1-t1}"\mbox{}\newline 
\hspace*{1em}\hspace*{1em}{xml:lang}="{mi}">}\mbox{}\newline 
\hspace*{1em}\hspace*{1em}{<\textbf{body}\hspace*{1em}{xml:id}="{t1-g1-t1-body1}">}\mbox{}\newline 
\hspace*{1em}\hspace*{1em}\hspace*{1em}{<\textbf{div}\hspace*{1em}{type}="{chapter}">}\mbox{}\newline 
\hspace*{1em}\hspace*{1em}\hspace*{1em}\hspace*{1em}{<\textbf{head}>}He Whakamaramatanga mo te Ture Hoko, Riihi hoki, i nga Whenua Maori, 1876.{</\textbf{head}>}\mbox{}\newline 
\hspace*{1em}\hspace*{1em}\hspace*{1em}\hspace*{1em}{<\textbf{p}>}…{</\textbf{p}>}\mbox{}\newline 
\hspace*{1em}\hspace*{1em}\hspace*{1em}{</\textbf{div}>}\mbox{}\newline 
\hspace*{1em}\hspace*{1em}{</\textbf{body}>}\mbox{}\newline 
\hspace*{1em}{</\textbf{text}>}\mbox{}\newline 
\hspace*{1em}{<\textbf{text}\hspace*{1em}{xml:id}="{t1-g1-t2}"\mbox{}\newline 
\hspace*{1em}\hspace*{1em}{xml:lang}="{en}">}\mbox{}\newline 
\hspace*{1em}\hspace*{1em}{<\textbf{body}\hspace*{1em}{xml:id}="{t1-g1-t2-body1}"\mbox{}\newline 
\hspace*{1em}\hspace*{1em}\hspace*{1em}{corresp}="{\#t1-g1-t1-body1}">}\mbox{}\newline 
\hspace*{1em}\hspace*{1em}\hspace*{1em}{<\textbf{div}\hspace*{1em}{type}="{chapter}">}\mbox{}\newline 
\hspace*{1em}\hspace*{1em}\hspace*{1em}\hspace*{1em}{<\textbf{head}>}An Act to regulate the Sale, Letting, and Disposal of Native Lands, 1876.{</\textbf{head}>}\mbox{}\newline 
\hspace*{1em}\hspace*{1em}\hspace*{1em}\hspace*{1em}{<\textbf{p}>}…{</\textbf{p}>}\mbox{}\newline 
\hspace*{1em}\hspace*{1em}\hspace*{1em}{</\textbf{div}>}\mbox{}\newline 
\hspace*{1em}\hspace*{1em}{</\textbf{body}>}\mbox{}\newline 
\hspace*{1em}{</\textbf{text}>}\mbox{}\newline 
{</\textbf{group}>}In this example a \hyperref[TEI.group]{<group>} contains two \hyperref[TEI.text]{<text>}s, each containing the same document in a different language. The correspondence is indicated using {\itshape corresp}. The language is indicated using {\itshape xml:lang}, whose value is inherited; both the tag with the {\itshape corresp} and the tag pointed to by the {\itshape corresp} inherit the value from their immediate parent.
    \item[]\index{place=<place>|exampleindex}\index{corresp=@corresp!<place>|exampleindex}\index{placeName=<placeName>|exampleindex}\index{desc=<desc>|exampleindex}\index{person=<person>|exampleindex}\index{corresp=@corresp!<person>|exampleindex}\index{persName=<persName>|exampleindex}\index{type=@type!<persName>|exampleindex}\index{note=<note>|exampleindex}\index{p=<p>|exampleindex}\index{placeName=<placeName>|exampleindex}\index{ref=@ref!<placeName>|exampleindex}\index{person=<person>|exampleindex}\index{corresp=@corresp!<person>|exampleindex}\index{persName=<persName>|exampleindex}\index{type=@type!<persName>|exampleindex}\index{note=<note>|exampleindex}\index{p=<p>|exampleindex}\exampleFont \mbox{}\newline 
\textit{<!-- In a placeography called "places.xml" -->}{<\textbf{place}\hspace*{1em}{xml:id}="{LOND1}"\mbox{}\newline 
\hspace*{1em}{corresp}="{people.xml\#LOND2 people.xml\#GENI1}">}\mbox{}\newline 
\hspace*{1em}{<\textbf{placeName}>}London{</\textbf{placeName}>}\mbox{}\newline 
\hspace*{1em}{<\textbf{desc}>}The city of London...{</\textbf{desc}>}\mbox{}\newline 
{</\textbf{place}>}\mbox{}\newline 
\textit{<!-- In a literary personography called "people.xml" -->}\mbox{}\newline 
{<\textbf{person}\hspace*{1em}{xml:id}="{LOND2}"\mbox{}\newline 
\hspace*{1em}{corresp}="{places.xml\#LOND1 \#GENI1}">}\mbox{}\newline 
\hspace*{1em}{<\textbf{persName}\hspace*{1em}{type}="{lit}">}London{</\textbf{persName}>}\mbox{}\newline 
\hspace*{1em}{<\textbf{note}>}\mbox{}\newline 
\hspace*{1em}\hspace*{1em}{<\textbf{p}>}Allegorical character representing the city of {<\textbf{placeName}\hspace*{1em}{ref}="{places.xml\#LOND1}">}London{</\textbf{placeName}>}.{</\textbf{p}>}\mbox{}\newline 
\hspace*{1em}{</\textbf{note}>}\mbox{}\newline 
{</\textbf{person}>}\mbox{}\newline 
{<\textbf{person}\hspace*{1em}{xml:id}="{GENI1}"\mbox{}\newline 
\hspace*{1em}{corresp}="{places.xml\#LOND1 \#LOND2}">}\mbox{}\newline 
\hspace*{1em}{<\textbf{persName}\hspace*{1em}{type}="{lit}">}London’s Genius{</\textbf{persName}>}\mbox{}\newline 
\hspace*{1em}{<\textbf{note}>}\mbox{}\newline 
\hspace*{1em}\hspace*{1em}{<\textbf{p}>}Personification of London’s genius. Appears as an\mbox{}\newline 
\hspace*{1em}\hspace*{1em}\hspace*{1em}\hspace*{1em} allegorical character in mayoral shows.\mbox{}\newline 
\hspace*{1em}\hspace*{1em}{</\textbf{p}>}\mbox{}\newline 
\hspace*{1em}{</\textbf{note}>}\mbox{}\newline 
{</\textbf{person}>}In this example, a \hyperref[TEI.place]{<place>} element containing information about the city of London is linked with two \hyperref[TEI.person]{<person>} elements in a literary personography. This correspondence represents a slightly looser relationship than the one in the preceding example; there is no sense in which an allegorical character could be substituted for the physical city, or vice versa, but there is obviously a correspondence between them.
\end{reflist}  
    \item[@synch]
  (synchronous) points to elements that are synchronous with the current element.
\begin{reflist}
    \item[{Status}]
  Optional
    \item[{Datatype}]
  1–∞ occurrences of \hyperref[TEI.teidata.pointer]{teidata.pointer} separated by whitespace
\end{reflist}  
    \item[@sameAs]
  points to an element that is the same as the current element.
\begin{reflist}
    \item[{Status}]
  Optional
    \item[{Datatype}]
  \hyperref[TEI.teidata.pointer]{teidata.pointer}
\end{reflist}  
    \item[@copyOf]
  points to an element of which the current element is a copy.
\begin{reflist}
    \item[{Status}]
  Optional
    \item[{Datatype}]
  \hyperref[TEI.teidata.pointer]{teidata.pointer}
    \item[{Note}]
  \par
Any content of the current element should be ignored. Its true content is that of the element being pointed at.
\end{reflist}  
    \item[@next]
  points to the next element of a virtual aggregate of which the current element is part.
\begin{reflist}
    \item[{Status}]
  Optional
    \item[{Datatype}]
  \hyperref[TEI.teidata.pointer]{teidata.pointer}
    \item[{Note}]
  \par
It is recommended that the element indicated be of the same type as the element bearing this attribute.
\end{reflist}  
    \item[@prev]
  (previous) points to the previous element of a virtual aggregate of which the current element is part.
\begin{reflist}
    \item[{Status}]
  Optional
    \item[{Datatype}]
  \hyperref[TEI.teidata.pointer]{teidata.pointer}
    \item[{Note}]
  \par
It is recommended that the element indicated be of the same type as the element bearing this attribute.
\end{reflist}  
    \item[@exclude]
  points to elements that are in exclusive alternation with the current element.
\begin{reflist}
    \item[{Status}]
  Optional
    \item[{Datatype}]
  1–∞ occurrences of \hyperref[TEI.teidata.pointer]{teidata.pointer} separated by whitespace
\end{reflist}  
    \item[@select]
  selects one or more alternants; if one alternant is selected, the ambiguity or uncertainty is marked as resolved. If more than one alternant is selected, the degree of ambiguity or uncertainty is marked as reduced by the number of alternants not selected.
\begin{reflist}
    \item[{Status}]
  Optional
    \item[{Datatype}]
  1–∞ occurrences of \hyperref[TEI.teidata.pointer]{teidata.pointer} separated by whitespace
    \item[{Note}]
  \par
This attribute should be placed on an element which is superordinate to all of the alternants from which the selection is being made.
\end{reflist}  
\end{sansreflist}  
\end{reflist}  
\begin{reflist}
\item[]\begin{specHead}{TEI.att.global.rendition}{att.global.rendition} provides rendering attributes common to all elements in the TEI encoding scheme. [\textit{\hyperref[STGAre]{1.3.1.1.3.\ Rendition Indicators}}]\end{specHead} 
    \item[{Module}]
  tei — \hyperref[ST]{The TEI Infrastructure}
    \item[{Members}]
  \hyperref[TEI.att.global]{att.global}[\hyperref[TEI.TEI]{TEI} \hyperref[TEI.ab]{ab} \hyperref[TEI.abbr]{abbr} \hyperref[TEI.abstract]{abstract} \hyperref[TEI.accMat]{accMat} \hyperref[TEI.acquisition]{acquisition} \hyperref[TEI.activity]{activity} \hyperref[TEI.actor]{actor} \hyperref[TEI.add]{add} \hyperref[TEI.addName]{addName} \hyperref[TEI.addSpan]{addSpan} \hyperref[TEI.additional]{additional} \hyperref[TEI.additions]{additions} \hyperref[TEI.addrLine]{addrLine} \hyperref[TEI.address]{address} \hyperref[TEI.adminInfo]{adminInfo} \hyperref[TEI.affiliation]{affiliation} \hyperref[TEI.age]{age} \hyperref[TEI.alt]{alt} \hyperref[TEI.altGrp]{altGrp} \hyperref[TEI.altIdent]{altIdent} \hyperref[TEI.altIdentifier]{altIdentifier} \hyperref[TEI.alternate]{alternate} \hyperref[TEI.am]{am} \hyperref[TEI.analytic]{analytic} \hyperref[TEI.anchor]{anchor} \hyperref[TEI.annotation]{annotation} \hyperref[TEI.annotationBlock]{annotationBlock} \hyperref[TEI.anyElement]{anyElement} \hyperref[TEI.app]{app} \hyperref[TEI.appInfo]{appInfo} \hyperref[TEI.application]{application} \hyperref[TEI.arc]{arc} \hyperref[TEI.argument]{argument} \hyperref[TEI.att]{att} \hyperref[TEI.attDef]{attDef} \hyperref[TEI.attList]{attList} \hyperref[TEI.attRef]{attRef} \hyperref[TEI.author]{author} \hyperref[TEI.authority]{authority} \hyperref[TEI.availability]{availability} \hyperref[TEI.back]{back} \hyperref[TEI.bibl]{bibl} \hyperref[TEI.biblFull]{biblFull} \hyperref[TEI.biblScope]{biblScope} \hyperref[TEI.biblStruct]{biblStruct} \hyperref[TEI.bicond]{bicond} \hyperref[TEI.binary]{binary} \hyperref[TEI.binaryObject]{binaryObject} \hyperref[TEI.binding]{binding} \hyperref[TEI.bindingDesc]{bindingDesc} \hyperref[TEI.birth]{birth} \hyperref[TEI.bloc]{bloc} \hyperref[TEI.body]{body} \hyperref[TEI.broadcast]{broadcast} \hyperref[TEI.byline]{byline} \hyperref[TEI.c]{c} \hyperref[TEI.cRefPattern]{cRefPattern} \hyperref[TEI.caesura]{caesura} \hyperref[TEI.calendar]{calendar} \hyperref[TEI.calendarDesc]{calendarDesc} \hyperref[TEI.camera]{camera} \hyperref[TEI.caption]{caption} \hyperref[TEI.case]{case} \hyperref[TEI.castGroup]{castGroup} \hyperref[TEI.castItem]{castItem} \hyperref[TEI.castList]{castList} \hyperref[TEI.catDesc]{catDesc} \hyperref[TEI.catRef]{catRef} \hyperref[TEI.catchwords]{catchwords} \hyperref[TEI.category]{category} \hyperref[TEI.cb]{cb} \hyperref[TEI.cell]{cell} \hyperref[TEI.certainty]{certainty} \hyperref[TEI.change]{change} \hyperref[TEI.channel]{channel} \hyperref[TEI.char]{char} \hyperref[TEI.charDecl]{charDecl} \hyperref[TEI.charName]{charName} \hyperref[TEI.charProp]{charProp} \hyperref[TEI.choice]{choice} \hyperref[TEI.cit]{cit} \hyperref[TEI.citeData]{citeData} \hyperref[TEI.citeStructure]{citeStructure} \hyperref[TEI.citedRange]{citedRange} \hyperref[TEI.cl]{cl} \hyperref[TEI.classCode]{classCode} \hyperref[TEI.classDecl]{classDecl} \hyperref[TEI.classRef]{classRef} \hyperref[TEI.classSpec]{classSpec} \hyperref[TEI.classes]{classes} \hyperref[TEI.climate]{climate} \hyperref[TEI.closer]{closer} \hyperref[TEI.code]{code} \hyperref[TEI.collation]{collation} \hyperref[TEI.collection]{collection} \hyperref[TEI.colloc]{colloc} \hyperref[TEI.colophon]{colophon} \hyperref[TEI.cond]{cond} \hyperref[TEI.condition]{condition} \hyperref[TEI.constitution]{constitution} \hyperref[TEI.constraint]{constraint} \hyperref[TEI.constraintSpec]{constraintSpec} \hyperref[TEI.content]{content} \hyperref[TEI.conversion]{conversion} \hyperref[TEI.corr]{corr} \hyperref[TEI.correction]{correction} \hyperref[TEI.correspAction]{correspAction} \hyperref[TEI.correspContext]{correspContext} \hyperref[TEI.correspDesc]{correspDesc} \hyperref[TEI.country]{country} \hyperref[TEI.creation]{creation} \hyperref[TEI.custEvent]{custEvent} \hyperref[TEI.custodialHist]{custodialHist} \hyperref[TEI.damage]{damage} \hyperref[TEI.damageSpan]{damageSpan} \hyperref[TEI.dataFacet]{dataFacet} \hyperref[TEI.dataRef]{dataRef} \hyperref[TEI.dataSpec]{dataSpec} \hyperref[TEI.datatype]{datatype} \hyperref[TEI.date]{date} \hyperref[TEI.dateline]{dateline} \hyperref[TEI.death]{death} \hyperref[TEI.decoDesc]{decoDesc} \hyperref[TEI.decoNote]{decoNote} \hyperref[TEI.def]{def} \hyperref[TEI.default]{default} \hyperref[TEI.defaultVal]{defaultVal} \hyperref[TEI.del]{del} \hyperref[TEI.delSpan]{delSpan} \hyperref[TEI.depth]{depth} \hyperref[TEI.derivation]{derivation} \hyperref[TEI.desc]{desc} \hyperref[TEI.dictScrap]{dictScrap} \hyperref[TEI.dim]{dim} \hyperref[TEI.dimensions]{dimensions} \hyperref[TEI.distinct]{distinct} \hyperref[TEI.distributor]{distributor} \hyperref[TEI.district]{district} \hyperref[TEI.div]{div} \hyperref[TEI.div1]{div1} \hyperref[TEI.div2]{div2} \hyperref[TEI.div3]{div3} \hyperref[TEI.div4]{div4} \hyperref[TEI.div5]{div5} \hyperref[TEI.div6]{div6} \hyperref[TEI.div7]{div7} \hyperref[TEI.divGen]{divGen} \hyperref[TEI.docAuthor]{docAuthor} \hyperref[TEI.docDate]{docDate} \hyperref[TEI.docEdition]{docEdition} \hyperref[TEI.docImprint]{docImprint} \hyperref[TEI.docTitle]{docTitle} \hyperref[TEI.domain]{domain} \hyperref[TEI.eLeaf]{eLeaf} \hyperref[TEI.eTree]{eTree} \hyperref[TEI.edition]{edition} \hyperref[TEI.editionStmt]{editionStmt} \hyperref[TEI.editor]{editor} \hyperref[TEI.editorialDecl]{editorialDecl} \hyperref[TEI.education]{education} \hyperref[TEI.eg]{eg} \hyperref[TEI.egXML]{egXML} \hyperref[TEI.elementRef]{elementRef} \hyperref[TEI.elementSpec]{elementSpec} \hyperref[TEI.email]{email} \hyperref[TEI.emph]{emph} \hyperref[TEI.empty]{empty} \hyperref[TEI.encodingDesc]{encodingDesc} \hyperref[TEI.entry]{entry} \hyperref[TEI.entryFree]{entryFree} \hyperref[TEI.epigraph]{epigraph} \hyperref[TEI.epilogue]{epilogue} \hyperref[TEI.equipment]{equipment} \hyperref[TEI.equiv]{equiv} \hyperref[TEI.etym]{etym} \hyperref[TEI.event]{event} \hyperref[TEI.ex]{ex} \hyperref[TEI.exemplum]{exemplum} \hyperref[TEI.expan]{expan} \hyperref[TEI.explicit]{explicit} \hyperref[TEI.extent]{extent} \hyperref[TEI.f]{f} \hyperref[TEI.fDecl]{fDecl} \hyperref[TEI.fDescr]{fDescr} \hyperref[TEI.fLib]{fLib} \hyperref[TEI.facsimile]{facsimile} \hyperref[TEI.factuality]{factuality} \hyperref[TEI.faith]{faith} \hyperref[TEI.figDesc]{figDesc} \hyperref[TEI.figure]{figure} \hyperref[TEI.fileDesc]{fileDesc} \hyperref[TEI.filiation]{filiation} \hyperref[TEI.finalRubric]{finalRubric} \hyperref[TEI.floatingText]{floatingText} \hyperref[TEI.floruit]{floruit} \hyperref[TEI.foliation]{foliation} \hyperref[TEI.foreign]{foreign} \hyperref[TEI.forename]{forename} \hyperref[TEI.forest]{forest} \hyperref[TEI.form]{form} \hyperref[TEI.formula]{formula} \hyperref[TEI.front]{front} \hyperref[TEI.fs]{fs} \hyperref[TEI.fsConstraints]{fsConstraints} \hyperref[TEI.fsDecl]{fsDecl} \hyperref[TEI.fsDescr]{fsDescr} \hyperref[TEI.fsdDecl]{fsdDecl} \hyperref[TEI.fsdLink]{fsdLink} \hyperref[TEI.funder]{funder} \hyperref[TEI.fvLib]{fvLib} \hyperref[TEI.fw]{fw} \hyperref[TEI.g]{g} \hyperref[TEI.gap]{gap} \hyperref[TEI.gb]{gb} \hyperref[TEI.gen]{gen} \hyperref[TEI.genName]{genName} \hyperref[TEI.geo]{geo} \hyperref[TEI.geoDecl]{geoDecl} \hyperref[TEI.geogFeat]{geogFeat} \hyperref[TEI.geogName]{geogName} \hyperref[TEI.gi]{gi} \hyperref[TEI.gloss]{gloss} \hyperref[TEI.glyph]{glyph} \hyperref[TEI.glyphName]{glyphName} \hyperref[TEI.gram]{gram} \hyperref[TEI.gramGrp]{gramGrp} \hyperref[TEI.graph]{graph} \hyperref[TEI.graphic]{graphic} \hyperref[TEI.group]{group} \hyperref[TEI.handDesc]{handDesc} \hyperref[TEI.handNote]{handNote} \hyperref[TEI.handNotes]{handNotes} \hyperref[TEI.handShift]{handShift} \hyperref[TEI.head]{head} \hyperref[TEI.headItem]{headItem} \hyperref[TEI.headLabel]{headLabel} \hyperref[TEI.height]{height} \hyperref[TEI.heraldry]{heraldry} \hyperref[TEI.hi]{hi} \hyperref[TEI.history]{history} \hyperref[TEI.hom]{hom} \hyperref[TEI.hyph]{hyph} \hyperref[TEI.hyphenation]{hyphenation} \hyperref[TEI.iNode]{iNode} \hyperref[TEI.iType]{iType} \hyperref[TEI.ident]{ident} \hyperref[TEI.idno]{idno} \hyperref[TEI.if]{if} \hyperref[TEI.iff]{iff} \hyperref[TEI.imprimatur]{imprimatur} \hyperref[TEI.imprint]{imprint} \hyperref[TEI.incident]{incident} \hyperref[TEI.incipit]{incipit} \hyperref[TEI.index]{index} \hyperref[TEI.institution]{institution} \hyperref[TEI.interaction]{interaction} \hyperref[TEI.interp]{interp} \hyperref[TEI.interpGrp]{interpGrp} \hyperref[TEI.interpretation]{interpretation} \hyperref[TEI.item]{item} \hyperref[TEI.join]{join} \hyperref[TEI.joinGrp]{joinGrp} \hyperref[TEI.keywords]{keywords} \hyperref[TEI.kinesic]{kinesic} \hyperref[TEI.l]{l} \hyperref[TEI.label]{label} \hyperref[TEI.lacunaEnd]{lacunaEnd} \hyperref[TEI.lacunaStart]{lacunaStart} \hyperref[TEI.lang]{lang} \hyperref[TEI.langKnowledge]{langKnowledge} \hyperref[TEI.langKnown]{langKnown} \hyperref[TEI.langUsage]{langUsage} \hyperref[TEI.language]{language} \hyperref[TEI.layout]{layout} \hyperref[TEI.layoutDesc]{layoutDesc} \hyperref[TEI.lb]{lb} \hyperref[TEI.lbl]{lbl} \hyperref[TEI.leaf]{leaf} \hyperref[TEI.lem]{lem} \hyperref[TEI.lg]{lg} \hyperref[TEI.licence]{licence} \hyperref[TEI.line]{line} \hyperref[TEI.link]{link} \hyperref[TEI.linkGrp]{linkGrp} \hyperref[TEI.list]{list} \hyperref[TEI.listAnnotation]{listAnnotation} \hyperref[TEI.listApp]{listApp} \hyperref[TEI.listBibl]{listBibl} \hyperref[TEI.listChange]{listChange} \hyperref[TEI.listEvent]{listEvent} \hyperref[TEI.listForest]{listForest} \hyperref[TEI.listNym]{listNym} \hyperref[TEI.listObject]{listObject} \hyperref[TEI.listOrg]{listOrg} \hyperref[TEI.listPerson]{listPerson} \hyperref[TEI.listPlace]{listPlace} \hyperref[TEI.listPrefixDef]{listPrefixDef} \hyperref[TEI.listRef]{listRef} \hyperref[TEI.listRelation]{listRelation} \hyperref[TEI.listTranspose]{listTranspose} \hyperref[TEI.listWit]{listWit} \hyperref[TEI.localName]{localName} \hyperref[TEI.localProp]{localProp} \hyperref[TEI.locale]{locale} \hyperref[TEI.location]{location} \hyperref[TEI.locus]{locus} \hyperref[TEI.locusGrp]{locusGrp} \hyperref[TEI.m]{m} \hyperref[TEI.macroRef]{macroRef} \hyperref[TEI.macroSpec]{macroSpec} \hyperref[TEI.mapping]{mapping} \hyperref[TEI.material]{material} \hyperref[TEI.measure]{measure} \hyperref[TEI.measureGrp]{measureGrp} \hyperref[TEI.media]{media} \hyperref[TEI.meeting]{meeting} \hyperref[TEI.memberOf]{memberOf} \hyperref[TEI.mentioned]{mentioned} \hyperref[TEI.metDecl]{metDecl} \hyperref[TEI.metSym]{metSym} \hyperref[TEI.metamark]{metamark} \hyperref[TEI.milestone]{milestone} \hyperref[TEI.mod]{mod} \hyperref[TEI.model]{model} \hyperref[TEI.modelGrp]{modelGrp} \hyperref[TEI.modelSequence]{modelSequence} \hyperref[TEI.moduleRef]{moduleRef} \hyperref[TEI.moduleSpec]{moduleSpec} \hyperref[TEI.monogr]{monogr} \hyperref[TEI.mood]{mood} \hyperref[TEI.move]{move} \hyperref[TEI.msContents]{msContents} \hyperref[TEI.msDesc]{msDesc} \hyperref[TEI.msFrag]{msFrag} \hyperref[TEI.msIdentifier]{msIdentifier} \hyperref[TEI.msItem]{msItem} \hyperref[TEI.msItemStruct]{msItemStruct} \hyperref[TEI.msName]{msName} \hyperref[TEI.msPart]{msPart} \hyperref[TEI.musicNotation]{musicNotation} \hyperref[TEI.name]{name} \hyperref[TEI.nameLink]{nameLink} \hyperref[TEI.namespace]{namespace} \hyperref[TEI.nationality]{nationality} \hyperref[TEI.node]{node} \hyperref[TEI.normalization]{normalization} \hyperref[TEI.notatedMusic]{notatedMusic} \hyperref[TEI.note]{note} \hyperref[TEI.noteGrp]{noteGrp} \hyperref[TEI.notesStmt]{notesStmt} \hyperref[TEI.num]{num} \hyperref[TEI.number]{number} \hyperref[TEI.numeric]{numeric} \hyperref[TEI.nym]{nym} \hyperref[TEI.oRef]{oRef} \hyperref[TEI.object]{object} \hyperref[TEI.objectDesc]{objectDesc} \hyperref[TEI.objectIdentifier]{objectIdentifier} \hyperref[TEI.objectName]{objectName} \hyperref[TEI.objectType]{objectType} \hyperref[TEI.occupation]{occupation} \hyperref[TEI.offset]{offset} \hyperref[TEI.opener]{opener} \hyperref[TEI.org]{org} \hyperref[TEI.orgName]{orgName} \hyperref[TEI.orig]{orig} \hyperref[TEI.origDate]{origDate} \hyperref[TEI.origPlace]{origPlace} \hyperref[TEI.origin]{origin} \hyperref[TEI.orth]{orth} \hyperref[TEI.outputRendition]{outputRendition} \hyperref[TEI.p]{p} \hyperref[TEI.pRef]{pRef} \hyperref[TEI.param]{param} \hyperref[TEI.paramList]{paramList} \hyperref[TEI.paramSpec]{paramSpec} \hyperref[TEI.particDesc]{particDesc} \hyperref[TEI.path]{path} \hyperref[TEI.pause]{pause} \hyperref[TEI.pb]{pb} \hyperref[TEI.pc]{pc} \hyperref[TEI.per]{per} \hyperref[TEI.performance]{performance} \hyperref[TEI.persName]{persName} \hyperref[TEI.persPronouns]{persPronouns} \hyperref[TEI.person]{person} \hyperref[TEI.personGrp]{personGrp} \hyperref[TEI.persona]{persona} \hyperref[TEI.phr]{phr} \hyperref[TEI.physDesc]{physDesc} \hyperref[TEI.place]{place} \hyperref[TEI.placeName]{placeName} \hyperref[TEI.population]{population} \hyperref[TEI.pos]{pos} \hyperref[TEI.postBox]{postBox} \hyperref[TEI.postCode]{postCode} \hyperref[TEI.postscript]{postscript} \hyperref[TEI.precision]{precision} \hyperref[TEI.prefixDef]{prefixDef} \hyperref[TEI.preparedness]{preparedness} \hyperref[TEI.principal]{principal} \hyperref[TEI.profileDesc]{profileDesc} \hyperref[TEI.projectDesc]{projectDesc} \hyperref[TEI.prologue]{prologue} \hyperref[TEI.pron]{pron} \hyperref[TEI.provenance]{provenance} \hyperref[TEI.ptr]{ptr} \hyperref[TEI.pubPlace]{pubPlace} \hyperref[TEI.publicationStmt]{publicationStmt} \hyperref[TEI.publisher]{publisher} \hyperref[TEI.punctuation]{punctuation} \hyperref[TEI.purpose]{purpose} \hyperref[TEI.q]{q} \hyperref[TEI.quotation]{quotation} \hyperref[TEI.quote]{quote} \hyperref[TEI.rb]{rb} \hyperref[TEI.rdg]{rdg} \hyperref[TEI.rdgGrp]{rdgGrp} \hyperref[TEI.re]{re} \hyperref[TEI.recordHist]{recordHist} \hyperref[TEI.recording]{recording} \hyperref[TEI.recordingStmt]{recordingStmt} \hyperref[TEI.redo]{redo} \hyperref[TEI.ref]{ref} \hyperref[TEI.refState]{refState} \hyperref[TEI.refsDecl]{refsDecl} \hyperref[TEI.reg]{reg} \hyperref[TEI.region]{region} \hyperref[TEI.relatedItem]{relatedItem} \hyperref[TEI.relation]{relation} \hyperref[TEI.remarks]{remarks} \hyperref[TEI.rendition]{rendition} \hyperref[TEI.repository]{repository} \hyperref[TEI.residence]{residence} \hyperref[TEI.resp]{resp} \hyperref[TEI.respStmt]{respStmt} \hyperref[TEI.respons]{respons} \hyperref[TEI.restore]{restore} \hyperref[TEI.retrace]{retrace} \hyperref[TEI.revisionDesc]{revisionDesc} \hyperref[TEI.rhyme]{rhyme} \hyperref[TEI.role]{role} \hyperref[TEI.roleDesc]{roleDesc} \hyperref[TEI.roleName]{roleName} \hyperref[TEI.root]{root} \hyperref[TEI.row]{row} \hyperref[TEI.rs]{rs} \hyperref[TEI.rt]{rt} \hyperref[TEI.rubric]{rubric} \hyperref[TEI.ruby]{ruby} \hyperref[TEI.s]{s} \hyperref[TEI.said]{said} \hyperref[TEI.salute]{salute} \hyperref[TEI.samplingDecl]{samplingDecl} \hyperref[TEI.schemaRef]{schemaRef} \hyperref[TEI.schemaSpec]{schemaSpec} \hyperref[TEI.scriptDesc]{scriptDesc} \hyperref[TEI.scriptNote]{scriptNote} \hyperref[TEI.scriptStmt]{scriptStmt} \hyperref[TEI.seal]{seal} \hyperref[TEI.sealDesc]{sealDesc} \hyperref[TEI.secFol]{secFol} \hyperref[TEI.secl]{secl} \hyperref[TEI.seg]{seg} \hyperref[TEI.segmentation]{segmentation} \hyperref[TEI.sense]{sense} \hyperref[TEI.sequence]{sequence} \hyperref[TEI.series]{series} \hyperref[TEI.seriesStmt]{seriesStmt} \hyperref[TEI.set]{set} \hyperref[TEI.setting]{setting} \hyperref[TEI.settingDesc]{settingDesc} \hyperref[TEI.settlement]{settlement} \hyperref[TEI.sex]{sex} \hyperref[TEI.shift]{shift} \hyperref[TEI.sic]{sic} \hyperref[TEI.signatures]{signatures} \hyperref[TEI.signed]{signed} \hyperref[TEI.soCalled]{soCalled} \hyperref[TEI.socecStatus]{socecStatus} \hyperref[TEI.sound]{sound} \hyperref[TEI.source]{source} \hyperref[TEI.sourceDesc]{sourceDesc} \hyperref[TEI.sourceDoc]{sourceDoc} \hyperref[TEI.sp]{sp} \hyperref[TEI.spGrp]{spGrp} \hyperref[TEI.space]{space} \hyperref[TEI.span]{span} \hyperref[TEI.spanGrp]{spanGrp} \hyperref[TEI.speaker]{speaker} \hyperref[TEI.specDesc]{specDesc} \hyperref[TEI.specGrp]{specGrp} \hyperref[TEI.specGrpRef]{specGrpRef} \hyperref[TEI.specList]{specList} \hyperref[TEI.sponsor]{sponsor} \hyperref[TEI.stage]{stage} \hyperref[TEI.stamp]{stamp} \hyperref[TEI.standOff]{standOff} \hyperref[TEI.state]{state} \hyperref[TEI.stdVals]{stdVals} \hyperref[TEI.street]{street} \hyperref[TEI.stress]{stress} \hyperref[TEI.string]{string} \hyperref[TEI.styleDefDecl]{styleDefDecl} \hyperref[TEI.subc]{subc} \hyperref[TEI.subst]{subst} \hyperref[TEI.substJoin]{substJoin} \hyperref[TEI.summary]{summary} \hyperref[TEI.superEntry]{superEntry} \hyperref[TEI.supplied]{supplied} \hyperref[TEI.support]{support} \hyperref[TEI.supportDesc]{supportDesc} \hyperref[TEI.surface]{surface} \hyperref[TEI.surfaceGrp]{surfaceGrp} \hyperref[TEI.surname]{surname} \hyperref[TEI.surplus]{surplus} \hyperref[TEI.surrogates]{surrogates} \hyperref[TEI.syll]{syll} \hyperref[TEI.symbol]{symbol} \hyperref[TEI.table]{table} \hyperref[TEI.tag]{tag} \hyperref[TEI.tagUsage]{tagUsage} \hyperref[TEI.tagsDecl]{tagsDecl} \hyperref[TEI.taxonomy]{taxonomy} \hyperref[TEI.tech]{tech} \hyperref[TEI.teiCorpus]{teiCorpus} \hyperref[TEI.teiHeader]{teiHeader} \hyperref[TEI.term]{term} \hyperref[TEI.terrain]{terrain} \hyperref[TEI.text]{text} \hyperref[TEI.textClass]{textClass} \hyperref[TEI.textDesc]{textDesc} \hyperref[TEI.textLang]{textLang} \hyperref[TEI.textNode]{textNode} \hyperref[TEI.then]{then} \hyperref[TEI.time]{time} \hyperref[TEI.timeline]{timeline} \hyperref[TEI.title]{title} \hyperref[TEI.titlePage]{titlePage} \hyperref[TEI.titlePart]{titlePart} \hyperref[TEI.titleStmt]{titleStmt} \hyperref[TEI.tns]{tns} \hyperref[TEI.trailer]{trailer} \hyperref[TEI.trait]{trait} \hyperref[TEI.transcriptionDesc]{transcriptionDesc} \hyperref[TEI.transpose]{transpose} \hyperref[TEI.tree]{tree} \hyperref[TEI.triangle]{triangle} \hyperref[TEI.typeDesc]{typeDesc} \hyperref[TEI.typeNote]{typeNote} \hyperref[TEI.u]{u} \hyperref[TEI.unclear]{unclear} \hyperref[TEI.undo]{undo} \hyperref[TEI.unicodeName]{unicodeName} \hyperref[TEI.unicodeProp]{unicodeProp} \hyperref[TEI.unihanProp]{unihanProp} \hyperref[TEI.unit]{unit} \hyperref[TEI.unitDecl]{unitDecl} \hyperref[TEI.unitDef]{unitDef} \hyperref[TEI.usg]{usg} \hyperref[TEI.vAlt]{vAlt} \hyperref[TEI.vColl]{vColl} \hyperref[TEI.vDefault]{vDefault} \hyperref[TEI.vLabel]{vLabel} \hyperref[TEI.vMerge]{vMerge} \hyperref[TEI.vNot]{vNot} \hyperref[TEI.vRange]{vRange} \hyperref[TEI.val]{val} \hyperref[TEI.valDesc]{valDesc} \hyperref[TEI.valItem]{valItem} \hyperref[TEI.valList]{valList} \hyperref[TEI.value]{value} \hyperref[TEI.variantEncoding]{variantEncoding} \hyperref[TEI.view]{view} \hyperref[TEI.vocal]{vocal} \hyperref[TEI.w]{w} \hyperref[TEI.watermark]{watermark} \hyperref[TEI.when]{when} \hyperref[TEI.width]{width} \hyperref[TEI.wit]{wit} \hyperref[TEI.witDetail]{witDetail} \hyperref[TEI.witEnd]{witEnd} \hyperref[TEI.witStart]{witStart} \hyperref[TEI.witness]{witness} \hyperref[TEI.writing]{writing} \hyperref[TEI.xenoData]{xenoData} \hyperref[TEI.xr]{xr} \hyperref[TEI.zone]{zone}]
    \item[{Attributes}]
  Attributes\hfil\\[-10pt]\begin{sansreflist}
    \item[@rend]
  (rendition) indicates how the element in question was rendered or presented in the source text.
\begin{reflist}
    \item[{Status}]
  Optional
    \item[{Datatype}]
  1–∞ occurrences of \hyperref[TEI.teidata.word]{teidata.word} separated by whitespace
    \item[]\index{head=<head>|exampleindex}\index{rend=@rend!<head>|exampleindex}\index{lb=<lb>|exampleindex}\index{lb=<lb>|exampleindex}\index{lb=<lb>|exampleindex}\index{lb=<lb>|exampleindex}\index{lb=<lb>|exampleindex}\index{lb=<lb>|exampleindex}\index{hi=<hi>|exampleindex}\index{rend=@rend!<hi>|exampleindex}\exampleFont {<\textbf{head}\hspace*{1em}{rend}="{align(center) case(allcaps)}">}\mbox{}\newline 
\hspace*{1em}{<\textbf{lb}/>}To The {<\textbf{lb}/>}Duchesse {<\textbf{lb}/>}of {<\textbf{lb}/>}Newcastle,\mbox{}\newline 
{<\textbf{lb}/>}On Her {<\textbf{lb}/>}\mbox{}\newline 
\hspace*{1em}{<\textbf{hi}\hspace*{1em}{rend}="{case(mixed)}">}New Blazing-World{</\textbf{hi}>}. \mbox{}\newline 
{</\textbf{head}>}
    \item[{Note}]
  \par
These Guidelines make no binding recommendations for the values of the {\itshape rend} attribute; the characteristics of visual presentation vary too much from text to text and the decision to record or ignore individual characteristics varies too much from project to project. Some potentially useful conventions are noted from time to time at appropriate points in the Guidelines. The values of the {\itshape rend} attribute are a set of sequence-indeterminate individual tokens separated by whitespace.
\end{reflist}  
    \item[@style]
  contains an expression in some formal style definition language which defines the rendering or presentation used for this element in the source text
\begin{reflist}
    \item[{Status}]
  Optional
    \item[{Datatype}]
  \hyperref[TEI.teidata.text]{teidata.text}
    \item[]\index{head=<head>|exampleindex}\index{style=@style!<head>|exampleindex}\index{lb=<lb>|exampleindex}\index{lb=<lb>|exampleindex}\index{lb=<lb>|exampleindex}\index{lb=<lb>|exampleindex}\index{lb=<lb>|exampleindex}\index{lb=<lb>|exampleindex}\index{hi=<hi>|exampleindex}\index{style=@style!<hi>|exampleindex}\exampleFont {<\textbf{head}\hspace*{1em}{style}="{text-align: center; font-variant: small-caps}">}\mbox{}\newline 
\hspace*{1em}{<\textbf{lb}/>}To The {<\textbf{lb}/>}Duchesse {<\textbf{lb}/>}of {<\textbf{lb}/>}Newcastle, {<\textbf{lb}/>}On Her\mbox{}\newline 
{<\textbf{lb}/>}\mbox{}\newline 
\hspace*{1em}{<\textbf{hi}\hspace*{1em}{style}="{font-variant: normal}">}New Blazing-World{</\textbf{hi}>}. \mbox{}\newline 
{</\textbf{head}>}
    \item[{Note}]
  \par
Unlike the attribute values of {\itshape rend}, which uses whitespace as a separator, the {\itshape style} attribute may contain whitespace. This attribute is intended for recording inline stylistic information concerning the source, not any particular output.\par
The formal language in which values for this attribute are expressed may be specified using the \hyperref[TEI.styleDefDecl]{<styleDefDecl>} element in the TEI header.\par
If {\itshape style} and {\itshape rendition} are both present on an element, then {\itshape style} overrides or complements {\itshape rendition}. {\itshape style} should not be used in conjunction with {\itshape rend}, because the latter does not employ a formal style definition language.
\end{reflist}  
    \item[@rendition]
  points to a description of the rendering or presentation used for this element in the source text.
\begin{reflist}
    \item[{Status}]
  Optional
    \item[{Datatype}]
  1–∞ occurrences of \hyperref[TEI.teidata.pointer]{teidata.pointer} separated by whitespace
    \item[]\index{head=<head>|exampleindex}\index{rendition=@rendition!<head>|exampleindex}\index{lb=<lb>|exampleindex}\index{lb=<lb>|exampleindex}\index{lb=<lb>|exampleindex}\index{lb=<lb>|exampleindex}\index{lb=<lb>|exampleindex}\index{lb=<lb>|exampleindex}\index{hi=<hi>|exampleindex}\index{rendition=@rendition!<hi>|exampleindex}\index{rendition=<rendition>|exampleindex}\index{scheme=@scheme!<rendition>|exampleindex}\index{rendition=<rendition>|exampleindex}\index{scheme=@scheme!<rendition>|exampleindex}\index{rendition=<rendition>|exampleindex}\index{scheme=@scheme!<rendition>|exampleindex}\exampleFont {<\textbf{head}\hspace*{1em}{rendition}="{\#ac \#sc}">}\mbox{}\newline 
\hspace*{1em}{<\textbf{lb}/>}To The {<\textbf{lb}/>}Duchesse {<\textbf{lb}/>}of {<\textbf{lb}/>}Newcastle, {<\textbf{lb}/>}On Her\mbox{}\newline 
{<\textbf{lb}/>}\mbox{}\newline 
\hspace*{1em}{<\textbf{hi}\hspace*{1em}{rendition}="{\#normal}">}New Blazing-World{</\textbf{hi}>}. \mbox{}\newline 
{</\textbf{head}>}\mbox{}\newline 
\textit{<!-- elsewhere... -->}\mbox{}\newline 
{<\textbf{rendition}\hspace*{1em}{xml:id}="{sc}"\mbox{}\newline 
\hspace*{1em}{scheme}="{css}">}font-variant: small-caps{</\textbf{rendition}>}\mbox{}\newline 
{<\textbf{rendition}\hspace*{1em}{xml:id}="{normal}"\mbox{}\newline 
\hspace*{1em}{scheme}="{css}">}font-variant: normal{</\textbf{rendition}>}\mbox{}\newline 
{<\textbf{rendition}\hspace*{1em}{xml:id}="{ac}"\mbox{}\newline 
\hspace*{1em}{scheme}="{css}">}text-align: center{</\textbf{rendition}>}
    \item[{Note}]
  \par
The {\itshape rendition} attribute is used in a very similar way to the {\itshape class} attribute defined for XHTML but with the important distinction that its function is to describe the appearance of the source text, not necessarily to determine how that text should be presented on screen or paper.\par
If {\itshape rendition} is used to refer to a style definition in a formal language like CSS, it is recommended that it not be used in conjunction with {\itshape rend}. Where both {\itshape rendition} and {\itshape rend} are supplied, the latter is understood to override or complement the former.\par
Each URI provided should indicate a \hyperref[TEI.rendition]{<rendition>} element defining the intended rendition in terms of some appropriate style language, as indicated by the {\itshape scheme} attribute.
\end{reflist}  
\end{sansreflist}  
\end{reflist}  
\begin{reflist}
\item[]\begin{specHead}{TEI.att.global.responsibility}{att.global.responsibility} provides attributes indicating the agent responsible for some aspect of the text, the markup or something asserted by the markup, and the degree of certainty associated with it. [\textit{\hyperref[STGAso]{1.3.1.1.4.\ Sources, certainty, and responsibility}} \textit{\hyperref[COED]{3.5.\ Simple Editorial Changes}} \textit{\hyperref[PHHR]{11.3.2.2.\ Hand, Responsibility, and Certainty Attributes}} \textit{\hyperref[AISP]{17.3.\ Spans and Interpretations}} \textit{\hyperref[NDATTSnr]{13.1.1.\ Linking Names and Their Referents}}]\end{specHead} 
    \item[{Module}]
  tei — \hyperref[ST]{The TEI Infrastructure}
    \item[{Members}]
  \hyperref[TEI.att.global]{att.global}[\hyperref[TEI.TEI]{TEI} \hyperref[TEI.ab]{ab} \hyperref[TEI.abbr]{abbr} \hyperref[TEI.abstract]{abstract} \hyperref[TEI.accMat]{accMat} \hyperref[TEI.acquisition]{acquisition} \hyperref[TEI.activity]{activity} \hyperref[TEI.actor]{actor} \hyperref[TEI.add]{add} \hyperref[TEI.addName]{addName} \hyperref[TEI.addSpan]{addSpan} \hyperref[TEI.additional]{additional} \hyperref[TEI.additions]{additions} \hyperref[TEI.addrLine]{addrLine} \hyperref[TEI.address]{address} \hyperref[TEI.adminInfo]{adminInfo} \hyperref[TEI.affiliation]{affiliation} \hyperref[TEI.age]{age} \hyperref[TEI.alt]{alt} \hyperref[TEI.altGrp]{altGrp} \hyperref[TEI.altIdent]{altIdent} \hyperref[TEI.altIdentifier]{altIdentifier} \hyperref[TEI.alternate]{alternate} \hyperref[TEI.am]{am} \hyperref[TEI.analytic]{analytic} \hyperref[TEI.anchor]{anchor} \hyperref[TEI.annotation]{annotation} \hyperref[TEI.annotationBlock]{annotationBlock} \hyperref[TEI.anyElement]{anyElement} \hyperref[TEI.app]{app} \hyperref[TEI.appInfo]{appInfo} \hyperref[TEI.application]{application} \hyperref[TEI.arc]{arc} \hyperref[TEI.argument]{argument} \hyperref[TEI.att]{att} \hyperref[TEI.attDef]{attDef} \hyperref[TEI.attList]{attList} \hyperref[TEI.attRef]{attRef} \hyperref[TEI.author]{author} \hyperref[TEI.authority]{authority} \hyperref[TEI.availability]{availability} \hyperref[TEI.back]{back} \hyperref[TEI.bibl]{bibl} \hyperref[TEI.biblFull]{biblFull} \hyperref[TEI.biblScope]{biblScope} \hyperref[TEI.biblStruct]{biblStruct} \hyperref[TEI.bicond]{bicond} \hyperref[TEI.binary]{binary} \hyperref[TEI.binaryObject]{binaryObject} \hyperref[TEI.binding]{binding} \hyperref[TEI.bindingDesc]{bindingDesc} \hyperref[TEI.birth]{birth} \hyperref[TEI.bloc]{bloc} \hyperref[TEI.body]{body} \hyperref[TEI.broadcast]{broadcast} \hyperref[TEI.byline]{byline} \hyperref[TEI.c]{c} \hyperref[TEI.cRefPattern]{cRefPattern} \hyperref[TEI.caesura]{caesura} \hyperref[TEI.calendar]{calendar} \hyperref[TEI.calendarDesc]{calendarDesc} \hyperref[TEI.camera]{camera} \hyperref[TEI.caption]{caption} \hyperref[TEI.case]{case} \hyperref[TEI.castGroup]{castGroup} \hyperref[TEI.castItem]{castItem} \hyperref[TEI.castList]{castList} \hyperref[TEI.catDesc]{catDesc} \hyperref[TEI.catRef]{catRef} \hyperref[TEI.catchwords]{catchwords} \hyperref[TEI.category]{category} \hyperref[TEI.cb]{cb} \hyperref[TEI.cell]{cell} \hyperref[TEI.certainty]{certainty} \hyperref[TEI.change]{change} \hyperref[TEI.channel]{channel} \hyperref[TEI.char]{char} \hyperref[TEI.charDecl]{charDecl} \hyperref[TEI.charName]{charName} \hyperref[TEI.charProp]{charProp} \hyperref[TEI.choice]{choice} \hyperref[TEI.cit]{cit} \hyperref[TEI.citeData]{citeData} \hyperref[TEI.citeStructure]{citeStructure} \hyperref[TEI.citedRange]{citedRange} \hyperref[TEI.cl]{cl} \hyperref[TEI.classCode]{classCode} \hyperref[TEI.classDecl]{classDecl} \hyperref[TEI.classRef]{classRef} \hyperref[TEI.classSpec]{classSpec} \hyperref[TEI.classes]{classes} \hyperref[TEI.climate]{climate} \hyperref[TEI.closer]{closer} \hyperref[TEI.code]{code} \hyperref[TEI.collation]{collation} \hyperref[TEI.collection]{collection} \hyperref[TEI.colloc]{colloc} \hyperref[TEI.colophon]{colophon} \hyperref[TEI.cond]{cond} \hyperref[TEI.condition]{condition} \hyperref[TEI.constitution]{constitution} \hyperref[TEI.constraint]{constraint} \hyperref[TEI.constraintSpec]{constraintSpec} \hyperref[TEI.content]{content} \hyperref[TEI.conversion]{conversion} \hyperref[TEI.corr]{corr} \hyperref[TEI.correction]{correction} \hyperref[TEI.correspAction]{correspAction} \hyperref[TEI.correspContext]{correspContext} \hyperref[TEI.correspDesc]{correspDesc} \hyperref[TEI.country]{country} \hyperref[TEI.creation]{creation} \hyperref[TEI.custEvent]{custEvent} \hyperref[TEI.custodialHist]{custodialHist} \hyperref[TEI.damage]{damage} \hyperref[TEI.damageSpan]{damageSpan} \hyperref[TEI.dataFacet]{dataFacet} \hyperref[TEI.dataRef]{dataRef} \hyperref[TEI.dataSpec]{dataSpec} \hyperref[TEI.datatype]{datatype} \hyperref[TEI.date]{date} \hyperref[TEI.dateline]{dateline} \hyperref[TEI.death]{death} \hyperref[TEI.decoDesc]{decoDesc} \hyperref[TEI.decoNote]{decoNote} \hyperref[TEI.def]{def} \hyperref[TEI.default]{default} \hyperref[TEI.defaultVal]{defaultVal} \hyperref[TEI.del]{del} \hyperref[TEI.delSpan]{delSpan} \hyperref[TEI.depth]{depth} \hyperref[TEI.derivation]{derivation} \hyperref[TEI.desc]{desc} \hyperref[TEI.dictScrap]{dictScrap} \hyperref[TEI.dim]{dim} \hyperref[TEI.dimensions]{dimensions} \hyperref[TEI.distinct]{distinct} \hyperref[TEI.distributor]{distributor} \hyperref[TEI.district]{district} \hyperref[TEI.div]{div} \hyperref[TEI.div1]{div1} \hyperref[TEI.div2]{div2} \hyperref[TEI.div3]{div3} \hyperref[TEI.div4]{div4} \hyperref[TEI.div5]{div5} \hyperref[TEI.div6]{div6} \hyperref[TEI.div7]{div7} \hyperref[TEI.divGen]{divGen} \hyperref[TEI.docAuthor]{docAuthor} \hyperref[TEI.docDate]{docDate} \hyperref[TEI.docEdition]{docEdition} \hyperref[TEI.docImprint]{docImprint} \hyperref[TEI.docTitle]{docTitle} \hyperref[TEI.domain]{domain} \hyperref[TEI.eLeaf]{eLeaf} \hyperref[TEI.eTree]{eTree} \hyperref[TEI.edition]{edition} \hyperref[TEI.editionStmt]{editionStmt} \hyperref[TEI.editor]{editor} \hyperref[TEI.editorialDecl]{editorialDecl} \hyperref[TEI.education]{education} \hyperref[TEI.eg]{eg} \hyperref[TEI.egXML]{egXML} \hyperref[TEI.elementRef]{elementRef} \hyperref[TEI.elementSpec]{elementSpec} \hyperref[TEI.email]{email} \hyperref[TEI.emph]{emph} \hyperref[TEI.empty]{empty} \hyperref[TEI.encodingDesc]{encodingDesc} \hyperref[TEI.entry]{entry} \hyperref[TEI.entryFree]{entryFree} \hyperref[TEI.epigraph]{epigraph} \hyperref[TEI.epilogue]{epilogue} \hyperref[TEI.equipment]{equipment} \hyperref[TEI.equiv]{equiv} \hyperref[TEI.etym]{etym} \hyperref[TEI.event]{event} \hyperref[TEI.ex]{ex} \hyperref[TEI.exemplum]{exemplum} \hyperref[TEI.expan]{expan} \hyperref[TEI.explicit]{explicit} \hyperref[TEI.extent]{extent} \hyperref[TEI.f]{f} \hyperref[TEI.fDecl]{fDecl} \hyperref[TEI.fDescr]{fDescr} \hyperref[TEI.fLib]{fLib} \hyperref[TEI.facsimile]{facsimile} \hyperref[TEI.factuality]{factuality} \hyperref[TEI.faith]{faith} \hyperref[TEI.figDesc]{figDesc} \hyperref[TEI.figure]{figure} \hyperref[TEI.fileDesc]{fileDesc} \hyperref[TEI.filiation]{filiation} \hyperref[TEI.finalRubric]{finalRubric} \hyperref[TEI.floatingText]{floatingText} \hyperref[TEI.floruit]{floruit} \hyperref[TEI.foliation]{foliation} \hyperref[TEI.foreign]{foreign} \hyperref[TEI.forename]{forename} \hyperref[TEI.forest]{forest} \hyperref[TEI.form]{form} \hyperref[TEI.formula]{formula} \hyperref[TEI.front]{front} \hyperref[TEI.fs]{fs} \hyperref[TEI.fsConstraints]{fsConstraints} \hyperref[TEI.fsDecl]{fsDecl} \hyperref[TEI.fsDescr]{fsDescr} \hyperref[TEI.fsdDecl]{fsdDecl} \hyperref[TEI.fsdLink]{fsdLink} \hyperref[TEI.funder]{funder} \hyperref[TEI.fvLib]{fvLib} \hyperref[TEI.fw]{fw} \hyperref[TEI.g]{g} \hyperref[TEI.gap]{gap} \hyperref[TEI.gb]{gb} \hyperref[TEI.gen]{gen} \hyperref[TEI.genName]{genName} \hyperref[TEI.geo]{geo} \hyperref[TEI.geoDecl]{geoDecl} \hyperref[TEI.geogFeat]{geogFeat} \hyperref[TEI.geogName]{geogName} \hyperref[TEI.gi]{gi} \hyperref[TEI.gloss]{gloss} \hyperref[TEI.glyph]{glyph} \hyperref[TEI.glyphName]{glyphName} \hyperref[TEI.gram]{gram} \hyperref[TEI.gramGrp]{gramGrp} \hyperref[TEI.graph]{graph} \hyperref[TEI.graphic]{graphic} \hyperref[TEI.group]{group} \hyperref[TEI.handDesc]{handDesc} \hyperref[TEI.handNote]{handNote} \hyperref[TEI.handNotes]{handNotes} \hyperref[TEI.handShift]{handShift} \hyperref[TEI.head]{head} \hyperref[TEI.headItem]{headItem} \hyperref[TEI.headLabel]{headLabel} \hyperref[TEI.height]{height} \hyperref[TEI.heraldry]{heraldry} \hyperref[TEI.hi]{hi} \hyperref[TEI.history]{history} \hyperref[TEI.hom]{hom} \hyperref[TEI.hyph]{hyph} \hyperref[TEI.hyphenation]{hyphenation} \hyperref[TEI.iNode]{iNode} \hyperref[TEI.iType]{iType} \hyperref[TEI.ident]{ident} \hyperref[TEI.idno]{idno} \hyperref[TEI.if]{if} \hyperref[TEI.iff]{iff} \hyperref[TEI.imprimatur]{imprimatur} \hyperref[TEI.imprint]{imprint} \hyperref[TEI.incident]{incident} \hyperref[TEI.incipit]{incipit} \hyperref[TEI.index]{index} \hyperref[TEI.institution]{institution} \hyperref[TEI.interaction]{interaction} \hyperref[TEI.interp]{interp} \hyperref[TEI.interpGrp]{interpGrp} \hyperref[TEI.interpretation]{interpretation} \hyperref[TEI.item]{item} \hyperref[TEI.join]{join} \hyperref[TEI.joinGrp]{joinGrp} \hyperref[TEI.keywords]{keywords} \hyperref[TEI.kinesic]{kinesic} \hyperref[TEI.l]{l} \hyperref[TEI.label]{label} \hyperref[TEI.lacunaEnd]{lacunaEnd} \hyperref[TEI.lacunaStart]{lacunaStart} \hyperref[TEI.lang]{lang} \hyperref[TEI.langKnowledge]{langKnowledge} \hyperref[TEI.langKnown]{langKnown} \hyperref[TEI.langUsage]{langUsage} \hyperref[TEI.language]{language} \hyperref[TEI.layout]{layout} \hyperref[TEI.layoutDesc]{layoutDesc} \hyperref[TEI.lb]{lb} \hyperref[TEI.lbl]{lbl} \hyperref[TEI.leaf]{leaf} \hyperref[TEI.lem]{lem} \hyperref[TEI.lg]{lg} \hyperref[TEI.licence]{licence} \hyperref[TEI.line]{line} \hyperref[TEI.link]{link} \hyperref[TEI.linkGrp]{linkGrp} \hyperref[TEI.list]{list} \hyperref[TEI.listAnnotation]{listAnnotation} \hyperref[TEI.listApp]{listApp} \hyperref[TEI.listBibl]{listBibl} \hyperref[TEI.listChange]{listChange} \hyperref[TEI.listEvent]{listEvent} \hyperref[TEI.listForest]{listForest} \hyperref[TEI.listNym]{listNym} \hyperref[TEI.listObject]{listObject} \hyperref[TEI.listOrg]{listOrg} \hyperref[TEI.listPerson]{listPerson} \hyperref[TEI.listPlace]{listPlace} \hyperref[TEI.listPrefixDef]{listPrefixDef} \hyperref[TEI.listRef]{listRef} \hyperref[TEI.listRelation]{listRelation} \hyperref[TEI.listTranspose]{listTranspose} \hyperref[TEI.listWit]{listWit} \hyperref[TEI.localName]{localName} \hyperref[TEI.localProp]{localProp} \hyperref[TEI.locale]{locale} \hyperref[TEI.location]{location} \hyperref[TEI.locus]{locus} \hyperref[TEI.locusGrp]{locusGrp} \hyperref[TEI.m]{m} \hyperref[TEI.macroRef]{macroRef} \hyperref[TEI.macroSpec]{macroSpec} \hyperref[TEI.mapping]{mapping} \hyperref[TEI.material]{material} \hyperref[TEI.measure]{measure} \hyperref[TEI.measureGrp]{measureGrp} \hyperref[TEI.media]{media} \hyperref[TEI.meeting]{meeting} \hyperref[TEI.memberOf]{memberOf} \hyperref[TEI.mentioned]{mentioned} \hyperref[TEI.metDecl]{metDecl} \hyperref[TEI.metSym]{metSym} \hyperref[TEI.metamark]{metamark} \hyperref[TEI.milestone]{milestone} \hyperref[TEI.mod]{mod} \hyperref[TEI.model]{model} \hyperref[TEI.modelGrp]{modelGrp} \hyperref[TEI.modelSequence]{modelSequence} \hyperref[TEI.moduleRef]{moduleRef} \hyperref[TEI.moduleSpec]{moduleSpec} \hyperref[TEI.monogr]{monogr} \hyperref[TEI.mood]{mood} \hyperref[TEI.move]{move} \hyperref[TEI.msContents]{msContents} \hyperref[TEI.msDesc]{msDesc} \hyperref[TEI.msFrag]{msFrag} \hyperref[TEI.msIdentifier]{msIdentifier} \hyperref[TEI.msItem]{msItem} \hyperref[TEI.msItemStruct]{msItemStruct} \hyperref[TEI.msName]{msName} \hyperref[TEI.msPart]{msPart} \hyperref[TEI.musicNotation]{musicNotation} \hyperref[TEI.name]{name} \hyperref[TEI.nameLink]{nameLink} \hyperref[TEI.namespace]{namespace} \hyperref[TEI.nationality]{nationality} \hyperref[TEI.node]{node} \hyperref[TEI.normalization]{normalization} \hyperref[TEI.notatedMusic]{notatedMusic} \hyperref[TEI.note]{note} \hyperref[TEI.noteGrp]{noteGrp} \hyperref[TEI.notesStmt]{notesStmt} \hyperref[TEI.num]{num} \hyperref[TEI.number]{number} \hyperref[TEI.numeric]{numeric} \hyperref[TEI.nym]{nym} \hyperref[TEI.oRef]{oRef} \hyperref[TEI.object]{object} \hyperref[TEI.objectDesc]{objectDesc} \hyperref[TEI.objectIdentifier]{objectIdentifier} \hyperref[TEI.objectName]{objectName} \hyperref[TEI.objectType]{objectType} \hyperref[TEI.occupation]{occupation} \hyperref[TEI.offset]{offset} \hyperref[TEI.opener]{opener} \hyperref[TEI.org]{org} \hyperref[TEI.orgName]{orgName} \hyperref[TEI.orig]{orig} \hyperref[TEI.origDate]{origDate} \hyperref[TEI.origPlace]{origPlace} \hyperref[TEI.origin]{origin} \hyperref[TEI.orth]{orth} \hyperref[TEI.outputRendition]{outputRendition} \hyperref[TEI.p]{p} \hyperref[TEI.pRef]{pRef} \hyperref[TEI.param]{param} \hyperref[TEI.paramList]{paramList} \hyperref[TEI.paramSpec]{paramSpec} \hyperref[TEI.particDesc]{particDesc} \hyperref[TEI.path]{path} \hyperref[TEI.pause]{pause} \hyperref[TEI.pb]{pb} \hyperref[TEI.pc]{pc} \hyperref[TEI.per]{per} \hyperref[TEI.performance]{performance} \hyperref[TEI.persName]{persName} \hyperref[TEI.persPronouns]{persPronouns} \hyperref[TEI.person]{person} \hyperref[TEI.personGrp]{personGrp} \hyperref[TEI.persona]{persona} \hyperref[TEI.phr]{phr} \hyperref[TEI.physDesc]{physDesc} \hyperref[TEI.place]{place} \hyperref[TEI.placeName]{placeName} \hyperref[TEI.population]{population} \hyperref[TEI.pos]{pos} \hyperref[TEI.postBox]{postBox} \hyperref[TEI.postCode]{postCode} \hyperref[TEI.postscript]{postscript} \hyperref[TEI.precision]{precision} \hyperref[TEI.prefixDef]{prefixDef} \hyperref[TEI.preparedness]{preparedness} \hyperref[TEI.principal]{principal} \hyperref[TEI.profileDesc]{profileDesc} \hyperref[TEI.projectDesc]{projectDesc} \hyperref[TEI.prologue]{prologue} \hyperref[TEI.pron]{pron} \hyperref[TEI.provenance]{provenance} \hyperref[TEI.ptr]{ptr} \hyperref[TEI.pubPlace]{pubPlace} \hyperref[TEI.publicationStmt]{publicationStmt} \hyperref[TEI.publisher]{publisher} \hyperref[TEI.punctuation]{punctuation} \hyperref[TEI.purpose]{purpose} \hyperref[TEI.q]{q} \hyperref[TEI.quotation]{quotation} \hyperref[TEI.quote]{quote} \hyperref[TEI.rb]{rb} \hyperref[TEI.rdg]{rdg} \hyperref[TEI.rdgGrp]{rdgGrp} \hyperref[TEI.re]{re} \hyperref[TEI.recordHist]{recordHist} \hyperref[TEI.recording]{recording} \hyperref[TEI.recordingStmt]{recordingStmt} \hyperref[TEI.redo]{redo} \hyperref[TEI.ref]{ref} \hyperref[TEI.refState]{refState} \hyperref[TEI.refsDecl]{refsDecl} \hyperref[TEI.reg]{reg} \hyperref[TEI.region]{region} \hyperref[TEI.relatedItem]{relatedItem} \hyperref[TEI.relation]{relation} \hyperref[TEI.remarks]{remarks} \hyperref[TEI.rendition]{rendition} \hyperref[TEI.repository]{repository} \hyperref[TEI.residence]{residence} \hyperref[TEI.resp]{resp} \hyperref[TEI.respStmt]{respStmt} \hyperref[TEI.respons]{respons} \hyperref[TEI.restore]{restore} \hyperref[TEI.retrace]{retrace} \hyperref[TEI.revisionDesc]{revisionDesc} \hyperref[TEI.rhyme]{rhyme} \hyperref[TEI.role]{role} \hyperref[TEI.roleDesc]{roleDesc} \hyperref[TEI.roleName]{roleName} \hyperref[TEI.root]{root} \hyperref[TEI.row]{row} \hyperref[TEI.rs]{rs} \hyperref[TEI.rt]{rt} \hyperref[TEI.rubric]{rubric} \hyperref[TEI.ruby]{ruby} \hyperref[TEI.s]{s} \hyperref[TEI.said]{said} \hyperref[TEI.salute]{salute} \hyperref[TEI.samplingDecl]{samplingDecl} \hyperref[TEI.schemaRef]{schemaRef} \hyperref[TEI.schemaSpec]{schemaSpec} \hyperref[TEI.scriptDesc]{scriptDesc} \hyperref[TEI.scriptNote]{scriptNote} \hyperref[TEI.scriptStmt]{scriptStmt} \hyperref[TEI.seal]{seal} \hyperref[TEI.sealDesc]{sealDesc} \hyperref[TEI.secFol]{secFol} \hyperref[TEI.secl]{secl} \hyperref[TEI.seg]{seg} \hyperref[TEI.segmentation]{segmentation} \hyperref[TEI.sense]{sense} \hyperref[TEI.sequence]{sequence} \hyperref[TEI.series]{series} \hyperref[TEI.seriesStmt]{seriesStmt} \hyperref[TEI.set]{set} \hyperref[TEI.setting]{setting} \hyperref[TEI.settingDesc]{settingDesc} \hyperref[TEI.settlement]{settlement} \hyperref[TEI.sex]{sex} \hyperref[TEI.shift]{shift} \hyperref[TEI.sic]{sic} \hyperref[TEI.signatures]{signatures} \hyperref[TEI.signed]{signed} \hyperref[TEI.soCalled]{soCalled} \hyperref[TEI.socecStatus]{socecStatus} \hyperref[TEI.sound]{sound} \hyperref[TEI.source]{source} \hyperref[TEI.sourceDesc]{sourceDesc} \hyperref[TEI.sourceDoc]{sourceDoc} \hyperref[TEI.sp]{sp} \hyperref[TEI.spGrp]{spGrp} \hyperref[TEI.space]{space} \hyperref[TEI.span]{span} \hyperref[TEI.spanGrp]{spanGrp} \hyperref[TEI.speaker]{speaker} \hyperref[TEI.specDesc]{specDesc} \hyperref[TEI.specGrp]{specGrp} \hyperref[TEI.specGrpRef]{specGrpRef} \hyperref[TEI.specList]{specList} \hyperref[TEI.sponsor]{sponsor} \hyperref[TEI.stage]{stage} \hyperref[TEI.stamp]{stamp} \hyperref[TEI.standOff]{standOff} \hyperref[TEI.state]{state} \hyperref[TEI.stdVals]{stdVals} \hyperref[TEI.street]{street} \hyperref[TEI.stress]{stress} \hyperref[TEI.string]{string} \hyperref[TEI.styleDefDecl]{styleDefDecl} \hyperref[TEI.subc]{subc} \hyperref[TEI.subst]{subst} \hyperref[TEI.substJoin]{substJoin} \hyperref[TEI.summary]{summary} \hyperref[TEI.superEntry]{superEntry} \hyperref[TEI.supplied]{supplied} \hyperref[TEI.support]{support} \hyperref[TEI.supportDesc]{supportDesc} \hyperref[TEI.surface]{surface} \hyperref[TEI.surfaceGrp]{surfaceGrp} \hyperref[TEI.surname]{surname} \hyperref[TEI.surplus]{surplus} \hyperref[TEI.surrogates]{surrogates} \hyperref[TEI.syll]{syll} \hyperref[TEI.symbol]{symbol} \hyperref[TEI.table]{table} \hyperref[TEI.tag]{tag} \hyperref[TEI.tagUsage]{tagUsage} \hyperref[TEI.tagsDecl]{tagsDecl} \hyperref[TEI.taxonomy]{taxonomy} \hyperref[TEI.tech]{tech} \hyperref[TEI.teiCorpus]{teiCorpus} \hyperref[TEI.teiHeader]{teiHeader} \hyperref[TEI.term]{term} \hyperref[TEI.terrain]{terrain} \hyperref[TEI.text]{text} \hyperref[TEI.textClass]{textClass} \hyperref[TEI.textDesc]{textDesc} \hyperref[TEI.textLang]{textLang} \hyperref[TEI.textNode]{textNode} \hyperref[TEI.then]{then} \hyperref[TEI.time]{time} \hyperref[TEI.timeline]{timeline} \hyperref[TEI.title]{title} \hyperref[TEI.titlePage]{titlePage} \hyperref[TEI.titlePart]{titlePart} \hyperref[TEI.titleStmt]{titleStmt} \hyperref[TEI.tns]{tns} \hyperref[TEI.trailer]{trailer} \hyperref[TEI.trait]{trait} \hyperref[TEI.transcriptionDesc]{transcriptionDesc} \hyperref[TEI.transpose]{transpose} \hyperref[TEI.tree]{tree} \hyperref[TEI.triangle]{triangle} \hyperref[TEI.typeDesc]{typeDesc} \hyperref[TEI.typeNote]{typeNote} \hyperref[TEI.u]{u} \hyperref[TEI.unclear]{unclear} \hyperref[TEI.undo]{undo} \hyperref[TEI.unicodeName]{unicodeName} \hyperref[TEI.unicodeProp]{unicodeProp} \hyperref[TEI.unihanProp]{unihanProp} \hyperref[TEI.unit]{unit} \hyperref[TEI.unitDecl]{unitDecl} \hyperref[TEI.unitDef]{unitDef} \hyperref[TEI.usg]{usg} \hyperref[TEI.vAlt]{vAlt} \hyperref[TEI.vColl]{vColl} \hyperref[TEI.vDefault]{vDefault} \hyperref[TEI.vLabel]{vLabel} \hyperref[TEI.vMerge]{vMerge} \hyperref[TEI.vNot]{vNot} \hyperref[TEI.vRange]{vRange} \hyperref[TEI.val]{val} \hyperref[TEI.valDesc]{valDesc} \hyperref[TEI.valItem]{valItem} \hyperref[TEI.valList]{valList} \hyperref[TEI.value]{value} \hyperref[TEI.variantEncoding]{variantEncoding} \hyperref[TEI.view]{view} \hyperref[TEI.vocal]{vocal} \hyperref[TEI.w]{w} \hyperref[TEI.watermark]{watermark} \hyperref[TEI.when]{when} \hyperref[TEI.width]{width} \hyperref[TEI.wit]{wit} \hyperref[TEI.witDetail]{witDetail} \hyperref[TEI.witEnd]{witEnd} \hyperref[TEI.witStart]{witStart} \hyperref[TEI.witness]{witness} \hyperref[TEI.writing]{writing} \hyperref[TEI.xenoData]{xenoData} \hyperref[TEI.xr]{xr} \hyperref[TEI.zone]{zone}]
    \item[{Attributes}]
  Attributes\hfil\\[-10pt]\begin{sansreflist}
    \item[@cert]
  (certainty) signifies the degree of certainty associated with the intervention or interpretation.
\begin{reflist}
    \item[{Status}]
  Optional
    \item[{Datatype}]
  \hyperref[TEI.teidata.probCert]{teidata.probCert}
\end{reflist}  
    \item[@resp]
  (responsible party) indicates the agency responsible for the intervention or interpretation, for example an editor or transcriber.
\begin{reflist}
    \item[{Status}]
  Optional
    \item[{Datatype}]
  1–∞ occurrences of \hyperref[TEI.teidata.pointer]{teidata.pointer} separated by whitespace
    \item[{Note}]
  \par
To reduce the ambiguity of a {\itshape resp} pointing directly to a person or organization, we recommend that {\itshape resp} be used to point not to an agent (\hyperref[TEI.person]{<person>} or \hyperref[TEI.org]{<org>}) but to a \hyperref[TEI.respStmt]{<respStmt>}, \hyperref[TEI.author]{<author>}, \hyperref[TEI.editor]{<editor>} or similar element which clarifies the exact role played by the agent. Pointing to multiple \hyperref[TEI.respStmt]{<respStmt>}s allows the encoder to specify clearly each of the roles played in part of a TEI file (creating, transcribing, encoding, editing, proofing etc.).
\end{reflist}  
\end{sansreflist}  
    \item[{Example}]
  \leavevmode\bgroup\index{choice=<choice>|exampleindex}\index{sic=<sic>|exampleindex}\index{corr=<corr>|exampleindex}\index{resp=@resp!<corr>|exampleindex}\index{cert=@cert!<corr>|exampleindex}\exampleFont \begin{shaded}\noindent\mbox{}Blessed are the\mbox{}\newline 
{<\textbf{choice}>}\mbox{}\newline 
\hspace*{1em}{<\textbf{sic}>}cheesemakers{</\textbf{sic}>}\mbox{}\newline 
\hspace*{1em}{<\textbf{corr}\hspace*{1em}{resp}="{\#editor}"\hspace*{1em}{cert}="{high}">}peacemakers{</\textbf{corr}>}\mbox{}\newline 
{</\textbf{choice}>}: for they shall be called the children of God.\end{shaded}\egroup 


    \item[{Example}]
  \leavevmode\bgroup\index{lg=<lg>|exampleindex}\index{l=<l>|exampleindex}\index{choice=<choice>|exampleindex}\index{sic=<sic>|exampleindex}\index{corr=<corr>|exampleindex}\index{resp=@resp!<corr>|exampleindex}\index{respStmt=<respStmt>|exampleindex}\index{resp=<resp>|exampleindex}\index{when=@when!<resp>|exampleindex}\index{name=<name>|exampleindex}\exampleFont \begin{shaded}\noindent\mbox{}\mbox{}\newline 
\textit{<!-- in the <text> ... -->}{<\textbf{lg}>}\mbox{}\newline 
\textit{<!-- ... -->}\mbox{}\newline 
\hspace*{1em}{<\textbf{l}>}Punkes, Panders, baſe extortionizing\mbox{}\newline 
\hspace*{1em}\hspace*{1em} sla{<\textbf{choice}>}\mbox{}\newline 
\hspace*{1em}\hspace*{1em}\hspace*{1em}{<\textbf{sic}>}n{</\textbf{sic}>}\mbox{}\newline 
\hspace*{1em}\hspace*{1em}\hspace*{1em}{<\textbf{corr}\hspace*{1em}{resp}="{\#JENS1\textunderscore transcriber}">}u{</\textbf{corr}>}\mbox{}\newline 
\hspace*{1em}\hspace*{1em}{</\textbf{choice}>}es,{</\textbf{l}>}\mbox{}\newline 
\textit{<!-- ... -->}\mbox{}\newline 
{</\textbf{lg}>}\mbox{}\newline 
\textit{<!-- in the <teiHeader> ... -->}\mbox{}\newline 
\textit{<!-- ... -->}\mbox{}\newline 
{<\textbf{respStmt}\hspace*{1em}{xml:id}="{JENS1\textunderscore transcriber}">}\mbox{}\newline 
\hspace*{1em}{<\textbf{resp}\hspace*{1em}{when}="{2014}">}Transcriber{</\textbf{resp}>}\mbox{}\newline 
\hspace*{1em}{<\textbf{name}>}Janelle Jenstad{</\textbf{name}>}\mbox{}\newline 
{</\textbf{respStmt}>}\end{shaded}\egroup 


\end{reflist}  
\begin{reflist}
\item[]\begin{specHead}{TEI.att.global.source}{att.global.source} provides an attribute used by elements to point to an external source. [\textit{\hyperref[STGAso]{1.3.1.1.4.\ Sources, certainty, and responsibility}} \textit{\hyperref[COHQQ]{3.3.3.\ Quotation}} \textit{\hyperref[TSBAWR]{8.3.4.\ Writing}}]\end{specHead} 
    \item[{Module}]
  tei — \hyperref[ST]{The TEI Infrastructure}
    \item[{Members}]
  \hyperref[TEI.att.global]{att.global}[\hyperref[TEI.TEI]{TEI} \hyperref[TEI.ab]{ab} \hyperref[TEI.abbr]{abbr} \hyperref[TEI.abstract]{abstract} \hyperref[TEI.accMat]{accMat} \hyperref[TEI.acquisition]{acquisition} \hyperref[TEI.activity]{activity} \hyperref[TEI.actor]{actor} \hyperref[TEI.add]{add} \hyperref[TEI.addName]{addName} \hyperref[TEI.addSpan]{addSpan} \hyperref[TEI.additional]{additional} \hyperref[TEI.additions]{additions} \hyperref[TEI.addrLine]{addrLine} \hyperref[TEI.address]{address} \hyperref[TEI.adminInfo]{adminInfo} \hyperref[TEI.affiliation]{affiliation} \hyperref[TEI.age]{age} \hyperref[TEI.alt]{alt} \hyperref[TEI.altGrp]{altGrp} \hyperref[TEI.altIdent]{altIdent} \hyperref[TEI.altIdentifier]{altIdentifier} \hyperref[TEI.alternate]{alternate} \hyperref[TEI.am]{am} \hyperref[TEI.analytic]{analytic} \hyperref[TEI.anchor]{anchor} \hyperref[TEI.annotation]{annotation} \hyperref[TEI.annotationBlock]{annotationBlock} \hyperref[TEI.anyElement]{anyElement} \hyperref[TEI.app]{app} \hyperref[TEI.appInfo]{appInfo} \hyperref[TEI.application]{application} \hyperref[TEI.arc]{arc} \hyperref[TEI.argument]{argument} \hyperref[TEI.att]{att} \hyperref[TEI.attDef]{attDef} \hyperref[TEI.attList]{attList} \hyperref[TEI.attRef]{attRef} \hyperref[TEI.author]{author} \hyperref[TEI.authority]{authority} \hyperref[TEI.availability]{availability} \hyperref[TEI.back]{back} \hyperref[TEI.bibl]{bibl} \hyperref[TEI.biblFull]{biblFull} \hyperref[TEI.biblScope]{biblScope} \hyperref[TEI.biblStruct]{biblStruct} \hyperref[TEI.bicond]{bicond} \hyperref[TEI.binary]{binary} \hyperref[TEI.binaryObject]{binaryObject} \hyperref[TEI.binding]{binding} \hyperref[TEI.bindingDesc]{bindingDesc} \hyperref[TEI.birth]{birth} \hyperref[TEI.bloc]{bloc} \hyperref[TEI.body]{body} \hyperref[TEI.broadcast]{broadcast} \hyperref[TEI.byline]{byline} \hyperref[TEI.c]{c} \hyperref[TEI.cRefPattern]{cRefPattern} \hyperref[TEI.caesura]{caesura} \hyperref[TEI.calendar]{calendar} \hyperref[TEI.calendarDesc]{calendarDesc} \hyperref[TEI.camera]{camera} \hyperref[TEI.caption]{caption} \hyperref[TEI.case]{case} \hyperref[TEI.castGroup]{castGroup} \hyperref[TEI.castItem]{castItem} \hyperref[TEI.castList]{castList} \hyperref[TEI.catDesc]{catDesc} \hyperref[TEI.catRef]{catRef} \hyperref[TEI.catchwords]{catchwords} \hyperref[TEI.category]{category} \hyperref[TEI.cb]{cb} \hyperref[TEI.cell]{cell} \hyperref[TEI.certainty]{certainty} \hyperref[TEI.change]{change} \hyperref[TEI.channel]{channel} \hyperref[TEI.char]{char} \hyperref[TEI.charDecl]{charDecl} \hyperref[TEI.charName]{charName} \hyperref[TEI.charProp]{charProp} \hyperref[TEI.choice]{choice} \hyperref[TEI.cit]{cit} \hyperref[TEI.citeData]{citeData} \hyperref[TEI.citeStructure]{citeStructure} \hyperref[TEI.citedRange]{citedRange} \hyperref[TEI.cl]{cl} \hyperref[TEI.classCode]{classCode} \hyperref[TEI.classDecl]{classDecl} \hyperref[TEI.classRef]{classRef} \hyperref[TEI.classSpec]{classSpec} \hyperref[TEI.classes]{classes} \hyperref[TEI.climate]{climate} \hyperref[TEI.closer]{closer} \hyperref[TEI.code]{code} \hyperref[TEI.collation]{collation} \hyperref[TEI.collection]{collection} \hyperref[TEI.colloc]{colloc} \hyperref[TEI.colophon]{colophon} \hyperref[TEI.cond]{cond} \hyperref[TEI.condition]{condition} \hyperref[TEI.constitution]{constitution} \hyperref[TEI.constraint]{constraint} \hyperref[TEI.constraintSpec]{constraintSpec} \hyperref[TEI.content]{content} \hyperref[TEI.conversion]{conversion} \hyperref[TEI.corr]{corr} \hyperref[TEI.correction]{correction} \hyperref[TEI.correspAction]{correspAction} \hyperref[TEI.correspContext]{correspContext} \hyperref[TEI.correspDesc]{correspDesc} \hyperref[TEI.country]{country} \hyperref[TEI.creation]{creation} \hyperref[TEI.custEvent]{custEvent} \hyperref[TEI.custodialHist]{custodialHist} \hyperref[TEI.damage]{damage} \hyperref[TEI.damageSpan]{damageSpan} \hyperref[TEI.dataFacet]{dataFacet} \hyperref[TEI.dataRef]{dataRef} \hyperref[TEI.dataSpec]{dataSpec} \hyperref[TEI.datatype]{datatype} \hyperref[TEI.date]{date} \hyperref[TEI.dateline]{dateline} \hyperref[TEI.death]{death} \hyperref[TEI.decoDesc]{decoDesc} \hyperref[TEI.decoNote]{decoNote} \hyperref[TEI.def]{def} \hyperref[TEI.default]{default} \hyperref[TEI.defaultVal]{defaultVal} \hyperref[TEI.del]{del} \hyperref[TEI.delSpan]{delSpan} \hyperref[TEI.depth]{depth} \hyperref[TEI.derivation]{derivation} \hyperref[TEI.desc]{desc} \hyperref[TEI.dictScrap]{dictScrap} \hyperref[TEI.dim]{dim} \hyperref[TEI.dimensions]{dimensions} \hyperref[TEI.distinct]{distinct} \hyperref[TEI.distributor]{distributor} \hyperref[TEI.district]{district} \hyperref[TEI.div]{div} \hyperref[TEI.div1]{div1} \hyperref[TEI.div2]{div2} \hyperref[TEI.div3]{div3} \hyperref[TEI.div4]{div4} \hyperref[TEI.div5]{div5} \hyperref[TEI.div6]{div6} \hyperref[TEI.div7]{div7} \hyperref[TEI.divGen]{divGen} \hyperref[TEI.docAuthor]{docAuthor} \hyperref[TEI.docDate]{docDate} \hyperref[TEI.docEdition]{docEdition} \hyperref[TEI.docImprint]{docImprint} \hyperref[TEI.docTitle]{docTitle} \hyperref[TEI.domain]{domain} \hyperref[TEI.eLeaf]{eLeaf} \hyperref[TEI.eTree]{eTree} \hyperref[TEI.edition]{edition} \hyperref[TEI.editionStmt]{editionStmt} \hyperref[TEI.editor]{editor} \hyperref[TEI.editorialDecl]{editorialDecl} \hyperref[TEI.education]{education} \hyperref[TEI.eg]{eg} \hyperref[TEI.egXML]{egXML} \hyperref[TEI.elementRef]{elementRef} \hyperref[TEI.elementSpec]{elementSpec} \hyperref[TEI.email]{email} \hyperref[TEI.emph]{emph} \hyperref[TEI.empty]{empty} \hyperref[TEI.encodingDesc]{encodingDesc} \hyperref[TEI.entry]{entry} \hyperref[TEI.entryFree]{entryFree} \hyperref[TEI.epigraph]{epigraph} \hyperref[TEI.epilogue]{epilogue} \hyperref[TEI.equipment]{equipment} \hyperref[TEI.equiv]{equiv} \hyperref[TEI.etym]{etym} \hyperref[TEI.event]{event} \hyperref[TEI.ex]{ex} \hyperref[TEI.exemplum]{exemplum} \hyperref[TEI.expan]{expan} \hyperref[TEI.explicit]{explicit} \hyperref[TEI.extent]{extent} \hyperref[TEI.f]{f} \hyperref[TEI.fDecl]{fDecl} \hyperref[TEI.fDescr]{fDescr} \hyperref[TEI.fLib]{fLib} \hyperref[TEI.facsimile]{facsimile} \hyperref[TEI.factuality]{factuality} \hyperref[TEI.faith]{faith} \hyperref[TEI.figDesc]{figDesc} \hyperref[TEI.figure]{figure} \hyperref[TEI.fileDesc]{fileDesc} \hyperref[TEI.filiation]{filiation} \hyperref[TEI.finalRubric]{finalRubric} \hyperref[TEI.floatingText]{floatingText} \hyperref[TEI.floruit]{floruit} \hyperref[TEI.foliation]{foliation} \hyperref[TEI.foreign]{foreign} \hyperref[TEI.forename]{forename} \hyperref[TEI.forest]{forest} \hyperref[TEI.form]{form} \hyperref[TEI.formula]{formula} \hyperref[TEI.front]{front} \hyperref[TEI.fs]{fs} \hyperref[TEI.fsConstraints]{fsConstraints} \hyperref[TEI.fsDecl]{fsDecl} \hyperref[TEI.fsDescr]{fsDescr} \hyperref[TEI.fsdDecl]{fsdDecl} \hyperref[TEI.fsdLink]{fsdLink} \hyperref[TEI.funder]{funder} \hyperref[TEI.fvLib]{fvLib} \hyperref[TEI.fw]{fw} \hyperref[TEI.g]{g} \hyperref[TEI.gap]{gap} \hyperref[TEI.gb]{gb} \hyperref[TEI.gen]{gen} \hyperref[TEI.genName]{genName} \hyperref[TEI.geo]{geo} \hyperref[TEI.geoDecl]{geoDecl} \hyperref[TEI.geogFeat]{geogFeat} \hyperref[TEI.geogName]{geogName} \hyperref[TEI.gi]{gi} \hyperref[TEI.gloss]{gloss} \hyperref[TEI.glyph]{glyph} \hyperref[TEI.glyphName]{glyphName} \hyperref[TEI.gram]{gram} \hyperref[TEI.gramGrp]{gramGrp} \hyperref[TEI.graph]{graph} \hyperref[TEI.graphic]{graphic} \hyperref[TEI.group]{group} \hyperref[TEI.handDesc]{handDesc} \hyperref[TEI.handNote]{handNote} \hyperref[TEI.handNotes]{handNotes} \hyperref[TEI.handShift]{handShift} \hyperref[TEI.head]{head} \hyperref[TEI.headItem]{headItem} \hyperref[TEI.headLabel]{headLabel} \hyperref[TEI.height]{height} \hyperref[TEI.heraldry]{heraldry} \hyperref[TEI.hi]{hi} \hyperref[TEI.history]{history} \hyperref[TEI.hom]{hom} \hyperref[TEI.hyph]{hyph} \hyperref[TEI.hyphenation]{hyphenation} \hyperref[TEI.iNode]{iNode} \hyperref[TEI.iType]{iType} \hyperref[TEI.ident]{ident} \hyperref[TEI.idno]{idno} \hyperref[TEI.if]{if} \hyperref[TEI.iff]{iff} \hyperref[TEI.imprimatur]{imprimatur} \hyperref[TEI.imprint]{imprint} \hyperref[TEI.incident]{incident} \hyperref[TEI.incipit]{incipit} \hyperref[TEI.index]{index} \hyperref[TEI.institution]{institution} \hyperref[TEI.interaction]{interaction} \hyperref[TEI.interp]{interp} \hyperref[TEI.interpGrp]{interpGrp} \hyperref[TEI.interpretation]{interpretation} \hyperref[TEI.item]{item} \hyperref[TEI.join]{join} \hyperref[TEI.joinGrp]{joinGrp} \hyperref[TEI.keywords]{keywords} \hyperref[TEI.kinesic]{kinesic} \hyperref[TEI.l]{l} \hyperref[TEI.label]{label} \hyperref[TEI.lacunaEnd]{lacunaEnd} \hyperref[TEI.lacunaStart]{lacunaStart} \hyperref[TEI.lang]{lang} \hyperref[TEI.langKnowledge]{langKnowledge} \hyperref[TEI.langKnown]{langKnown} \hyperref[TEI.langUsage]{langUsage} \hyperref[TEI.language]{language} \hyperref[TEI.layout]{layout} \hyperref[TEI.layoutDesc]{layoutDesc} \hyperref[TEI.lb]{lb} \hyperref[TEI.lbl]{lbl} \hyperref[TEI.leaf]{leaf} \hyperref[TEI.lem]{lem} \hyperref[TEI.lg]{lg} \hyperref[TEI.licence]{licence} \hyperref[TEI.line]{line} \hyperref[TEI.link]{link} \hyperref[TEI.linkGrp]{linkGrp} \hyperref[TEI.list]{list} \hyperref[TEI.listAnnotation]{listAnnotation} \hyperref[TEI.listApp]{listApp} \hyperref[TEI.listBibl]{listBibl} \hyperref[TEI.listChange]{listChange} \hyperref[TEI.listEvent]{listEvent} \hyperref[TEI.listForest]{listForest} \hyperref[TEI.listNym]{listNym} \hyperref[TEI.listObject]{listObject} \hyperref[TEI.listOrg]{listOrg} \hyperref[TEI.listPerson]{listPerson} \hyperref[TEI.listPlace]{listPlace} \hyperref[TEI.listPrefixDef]{listPrefixDef} \hyperref[TEI.listRef]{listRef} \hyperref[TEI.listRelation]{listRelation} \hyperref[TEI.listTranspose]{listTranspose} \hyperref[TEI.listWit]{listWit} \hyperref[TEI.localName]{localName} \hyperref[TEI.localProp]{localProp} \hyperref[TEI.locale]{locale} \hyperref[TEI.location]{location} \hyperref[TEI.locus]{locus} \hyperref[TEI.locusGrp]{locusGrp} \hyperref[TEI.m]{m} \hyperref[TEI.macroRef]{macroRef} \hyperref[TEI.macroSpec]{macroSpec} \hyperref[TEI.mapping]{mapping} \hyperref[TEI.material]{material} \hyperref[TEI.measure]{measure} \hyperref[TEI.measureGrp]{measureGrp} \hyperref[TEI.media]{media} \hyperref[TEI.meeting]{meeting} \hyperref[TEI.memberOf]{memberOf} \hyperref[TEI.mentioned]{mentioned} \hyperref[TEI.metDecl]{metDecl} \hyperref[TEI.metSym]{metSym} \hyperref[TEI.metamark]{metamark} \hyperref[TEI.milestone]{milestone} \hyperref[TEI.mod]{mod} \hyperref[TEI.model]{model} \hyperref[TEI.modelGrp]{modelGrp} \hyperref[TEI.modelSequence]{modelSequence} \hyperref[TEI.moduleRef]{moduleRef} \hyperref[TEI.moduleSpec]{moduleSpec} \hyperref[TEI.monogr]{monogr} \hyperref[TEI.mood]{mood} \hyperref[TEI.move]{move} \hyperref[TEI.msContents]{msContents} \hyperref[TEI.msDesc]{msDesc} \hyperref[TEI.msFrag]{msFrag} \hyperref[TEI.msIdentifier]{msIdentifier} \hyperref[TEI.msItem]{msItem} \hyperref[TEI.msItemStruct]{msItemStruct} \hyperref[TEI.msName]{msName} \hyperref[TEI.msPart]{msPart} \hyperref[TEI.musicNotation]{musicNotation} \hyperref[TEI.name]{name} \hyperref[TEI.nameLink]{nameLink} \hyperref[TEI.namespace]{namespace} \hyperref[TEI.nationality]{nationality} \hyperref[TEI.node]{node} \hyperref[TEI.normalization]{normalization} \hyperref[TEI.notatedMusic]{notatedMusic} \hyperref[TEI.note]{note} \hyperref[TEI.noteGrp]{noteGrp} \hyperref[TEI.notesStmt]{notesStmt} \hyperref[TEI.num]{num} \hyperref[TEI.number]{number} \hyperref[TEI.numeric]{numeric} \hyperref[TEI.nym]{nym} \hyperref[TEI.oRef]{oRef} \hyperref[TEI.object]{object} \hyperref[TEI.objectDesc]{objectDesc} \hyperref[TEI.objectIdentifier]{objectIdentifier} \hyperref[TEI.objectName]{objectName} \hyperref[TEI.objectType]{objectType} \hyperref[TEI.occupation]{occupation} \hyperref[TEI.offset]{offset} \hyperref[TEI.opener]{opener} \hyperref[TEI.org]{org} \hyperref[TEI.orgName]{orgName} \hyperref[TEI.orig]{orig} \hyperref[TEI.origDate]{origDate} \hyperref[TEI.origPlace]{origPlace} \hyperref[TEI.origin]{origin} \hyperref[TEI.orth]{orth} \hyperref[TEI.outputRendition]{outputRendition} \hyperref[TEI.p]{p} \hyperref[TEI.pRef]{pRef} \hyperref[TEI.param]{param} \hyperref[TEI.paramList]{paramList} \hyperref[TEI.paramSpec]{paramSpec} \hyperref[TEI.particDesc]{particDesc} \hyperref[TEI.path]{path} \hyperref[TEI.pause]{pause} \hyperref[TEI.pb]{pb} \hyperref[TEI.pc]{pc} \hyperref[TEI.per]{per} \hyperref[TEI.performance]{performance} \hyperref[TEI.persName]{persName} \hyperref[TEI.persPronouns]{persPronouns} \hyperref[TEI.person]{person} \hyperref[TEI.personGrp]{personGrp} \hyperref[TEI.persona]{persona} \hyperref[TEI.phr]{phr} \hyperref[TEI.physDesc]{physDesc} \hyperref[TEI.place]{place} \hyperref[TEI.placeName]{placeName} \hyperref[TEI.population]{population} \hyperref[TEI.pos]{pos} \hyperref[TEI.postBox]{postBox} \hyperref[TEI.postCode]{postCode} \hyperref[TEI.postscript]{postscript} \hyperref[TEI.precision]{precision} \hyperref[TEI.prefixDef]{prefixDef} \hyperref[TEI.preparedness]{preparedness} \hyperref[TEI.principal]{principal} \hyperref[TEI.profileDesc]{profileDesc} \hyperref[TEI.projectDesc]{projectDesc} \hyperref[TEI.prologue]{prologue} \hyperref[TEI.pron]{pron} \hyperref[TEI.provenance]{provenance} \hyperref[TEI.ptr]{ptr} \hyperref[TEI.pubPlace]{pubPlace} \hyperref[TEI.publicationStmt]{publicationStmt} \hyperref[TEI.publisher]{publisher} \hyperref[TEI.punctuation]{punctuation} \hyperref[TEI.purpose]{purpose} \hyperref[TEI.q]{q} \hyperref[TEI.quotation]{quotation} \hyperref[TEI.quote]{quote} \hyperref[TEI.rb]{rb} \hyperref[TEI.rdg]{rdg} \hyperref[TEI.rdgGrp]{rdgGrp} \hyperref[TEI.re]{re} \hyperref[TEI.recordHist]{recordHist} \hyperref[TEI.recording]{recording} \hyperref[TEI.recordingStmt]{recordingStmt} \hyperref[TEI.redo]{redo} \hyperref[TEI.ref]{ref} \hyperref[TEI.refState]{refState} \hyperref[TEI.refsDecl]{refsDecl} \hyperref[TEI.reg]{reg} \hyperref[TEI.region]{region} \hyperref[TEI.relatedItem]{relatedItem} \hyperref[TEI.relation]{relation} \hyperref[TEI.remarks]{remarks} \hyperref[TEI.rendition]{rendition} \hyperref[TEI.repository]{repository} \hyperref[TEI.residence]{residence} \hyperref[TEI.resp]{resp} \hyperref[TEI.respStmt]{respStmt} \hyperref[TEI.respons]{respons} \hyperref[TEI.restore]{restore} \hyperref[TEI.retrace]{retrace} \hyperref[TEI.revisionDesc]{revisionDesc} \hyperref[TEI.rhyme]{rhyme} \hyperref[TEI.role]{role} \hyperref[TEI.roleDesc]{roleDesc} \hyperref[TEI.roleName]{roleName} \hyperref[TEI.root]{root} \hyperref[TEI.row]{row} \hyperref[TEI.rs]{rs} \hyperref[TEI.rt]{rt} \hyperref[TEI.rubric]{rubric} \hyperref[TEI.ruby]{ruby} \hyperref[TEI.s]{s} \hyperref[TEI.said]{said} \hyperref[TEI.salute]{salute} \hyperref[TEI.samplingDecl]{samplingDecl} \hyperref[TEI.schemaRef]{schemaRef} \hyperref[TEI.schemaSpec]{schemaSpec} \hyperref[TEI.scriptDesc]{scriptDesc} \hyperref[TEI.scriptNote]{scriptNote} \hyperref[TEI.scriptStmt]{scriptStmt} \hyperref[TEI.seal]{seal} \hyperref[TEI.sealDesc]{sealDesc} \hyperref[TEI.secFol]{secFol} \hyperref[TEI.secl]{secl} \hyperref[TEI.seg]{seg} \hyperref[TEI.segmentation]{segmentation} \hyperref[TEI.sense]{sense} \hyperref[TEI.sequence]{sequence} \hyperref[TEI.series]{series} \hyperref[TEI.seriesStmt]{seriesStmt} \hyperref[TEI.set]{set} \hyperref[TEI.setting]{setting} \hyperref[TEI.settingDesc]{settingDesc} \hyperref[TEI.settlement]{settlement} \hyperref[TEI.sex]{sex} \hyperref[TEI.shift]{shift} \hyperref[TEI.sic]{sic} \hyperref[TEI.signatures]{signatures} \hyperref[TEI.signed]{signed} \hyperref[TEI.soCalled]{soCalled} \hyperref[TEI.socecStatus]{socecStatus} \hyperref[TEI.sound]{sound} \hyperref[TEI.source]{source} \hyperref[TEI.sourceDesc]{sourceDesc} \hyperref[TEI.sourceDoc]{sourceDoc} \hyperref[TEI.sp]{sp} \hyperref[TEI.spGrp]{spGrp} \hyperref[TEI.space]{space} \hyperref[TEI.span]{span} \hyperref[TEI.spanGrp]{spanGrp} \hyperref[TEI.speaker]{speaker} \hyperref[TEI.specDesc]{specDesc} \hyperref[TEI.specGrp]{specGrp} \hyperref[TEI.specGrpRef]{specGrpRef} \hyperref[TEI.specList]{specList} \hyperref[TEI.sponsor]{sponsor} \hyperref[TEI.stage]{stage} \hyperref[TEI.stamp]{stamp} \hyperref[TEI.standOff]{standOff} \hyperref[TEI.state]{state} \hyperref[TEI.stdVals]{stdVals} \hyperref[TEI.street]{street} \hyperref[TEI.stress]{stress} \hyperref[TEI.string]{string} \hyperref[TEI.styleDefDecl]{styleDefDecl} \hyperref[TEI.subc]{subc} \hyperref[TEI.subst]{subst} \hyperref[TEI.substJoin]{substJoin} \hyperref[TEI.summary]{summary} \hyperref[TEI.superEntry]{superEntry} \hyperref[TEI.supplied]{supplied} \hyperref[TEI.support]{support} \hyperref[TEI.supportDesc]{supportDesc} \hyperref[TEI.surface]{surface} \hyperref[TEI.surfaceGrp]{surfaceGrp} \hyperref[TEI.surname]{surname} \hyperref[TEI.surplus]{surplus} \hyperref[TEI.surrogates]{surrogates} \hyperref[TEI.syll]{syll} \hyperref[TEI.symbol]{symbol} \hyperref[TEI.table]{table} \hyperref[TEI.tag]{tag} \hyperref[TEI.tagUsage]{tagUsage} \hyperref[TEI.tagsDecl]{tagsDecl} \hyperref[TEI.taxonomy]{taxonomy} \hyperref[TEI.tech]{tech} \hyperref[TEI.teiCorpus]{teiCorpus} \hyperref[TEI.teiHeader]{teiHeader} \hyperref[TEI.term]{term} \hyperref[TEI.terrain]{terrain} \hyperref[TEI.text]{text} \hyperref[TEI.textClass]{textClass} \hyperref[TEI.textDesc]{textDesc} \hyperref[TEI.textLang]{textLang} \hyperref[TEI.textNode]{textNode} \hyperref[TEI.then]{then} \hyperref[TEI.time]{time} \hyperref[TEI.timeline]{timeline} \hyperref[TEI.title]{title} \hyperref[TEI.titlePage]{titlePage} \hyperref[TEI.titlePart]{titlePart} \hyperref[TEI.titleStmt]{titleStmt} \hyperref[TEI.tns]{tns} \hyperref[TEI.trailer]{trailer} \hyperref[TEI.trait]{trait} \hyperref[TEI.transcriptionDesc]{transcriptionDesc} \hyperref[TEI.transpose]{transpose} \hyperref[TEI.tree]{tree} \hyperref[TEI.triangle]{triangle} \hyperref[TEI.typeDesc]{typeDesc} \hyperref[TEI.typeNote]{typeNote} \hyperref[TEI.u]{u} \hyperref[TEI.unclear]{unclear} \hyperref[TEI.undo]{undo} \hyperref[TEI.unicodeName]{unicodeName} \hyperref[TEI.unicodeProp]{unicodeProp} \hyperref[TEI.unihanProp]{unihanProp} \hyperref[TEI.unit]{unit} \hyperref[TEI.unitDecl]{unitDecl} \hyperref[TEI.unitDef]{unitDef} \hyperref[TEI.usg]{usg} \hyperref[TEI.vAlt]{vAlt} \hyperref[TEI.vColl]{vColl} \hyperref[TEI.vDefault]{vDefault} \hyperref[TEI.vLabel]{vLabel} \hyperref[TEI.vMerge]{vMerge} \hyperref[TEI.vNot]{vNot} \hyperref[TEI.vRange]{vRange} \hyperref[TEI.val]{val} \hyperref[TEI.valDesc]{valDesc} \hyperref[TEI.valItem]{valItem} \hyperref[TEI.valList]{valList} \hyperref[TEI.value]{value} \hyperref[TEI.variantEncoding]{variantEncoding} \hyperref[TEI.view]{view} \hyperref[TEI.vocal]{vocal} \hyperref[TEI.w]{w} \hyperref[TEI.watermark]{watermark} \hyperref[TEI.when]{when} \hyperref[TEI.width]{width} \hyperref[TEI.wit]{wit} \hyperref[TEI.witDetail]{witDetail} \hyperref[TEI.witEnd]{witEnd} \hyperref[TEI.witStart]{witStart} \hyperref[TEI.witness]{witness} \hyperref[TEI.writing]{writing} \hyperref[TEI.xenoData]{xenoData} \hyperref[TEI.xr]{xr} \hyperref[TEI.zone]{zone}]
    \item[{Attributes}]
  Attributes\hfil\\[-10pt]\begin{sansreflist}
    \item[@source]
  specifies the source from which some aspect of this element is drawn.
\begin{reflist}
    \item[{Status}]
  Optional
    \item[{Datatype}]
  1–∞ occurrences of \hyperref[TEI.teidata.pointer]{teidata.pointer} separated by whitespace
    \item[{Note}]
  \par
The {\itshape source} attribute points to an external source. When used on elements describing schema components such as \hyperref[TEI.schemaSpec]{<schemaSpec>} or \hyperref[TEI.moduleRef]{<moduleRef>} it identifies the source from which declarations for the components of the object being defined may be obtained.\par
On other elements it provides a pointer to the bibliographical source from which a quotation or citation is drawn. \par
In either case, the location may be provided using any form of URI, for example an absolute URI, a relative URI, or private scheme URI that is expanded to an absolute URI as documented in a \hyperref[TEI.prefixDef]{<prefixDef>}.\par
If more than one location is specified, the default assumption is that the required source should be obtained by combining the resources indicated. 
\end{reflist}  
\end{sansreflist}  
    \item[{Example}]
  \leavevmode\bgroup\index{p=<p>|exampleindex}\index{bibl=<bibl>|exampleindex}\index{quote=<quote>|exampleindex}\index{source=@source!<quote>|exampleindex}\exampleFont \begin{shaded}\noindent\mbox{}{<\textbf{p}>}\mbox{}\newline 
\textit{<!-- ... -->} As Willard McCarty ({<\textbf{bibl}\hspace*{1em}{xml:id}="{mcc\textunderscore 2012}">}2012, p.2{</\textbf{bibl}>}) tells us, {<\textbf{quote}\hspace*{1em}{source}="{\#mcc\textunderscore 2012}">}‘Collaboration’ is a problematic and should be a contested\mbox{}\newline 
\hspace*{1em}\hspace*{1em} term.{</\textbf{quote}>}\mbox{}\newline 
\textit{<!-- ... -->}\mbox{}\newline 
{</\textbf{p}>}\end{shaded}\egroup 


    \item[{Example}]
  \leavevmode\bgroup\index{p=<p>|exampleindex}\index{quote=<quote>|exampleindex}\index{source=@source!<quote>|exampleindex}\index{bibl=<bibl>|exampleindex}\index{title=<title>|exampleindex}\index{level=@level!<title>|exampleindex}\index{edition=<edition>|exampleindex}\index{pubPlace=<pubPlace>|exampleindex}\index{publisher=<publisher>|exampleindex}\index{date=<date>|exampleindex}\index{biblScope=<biblScope>|exampleindex}\index{unit=@unit!<biblScope>|exampleindex}\exampleFont \begin{shaded}\noindent\mbox{}{<\textbf{p}>}\mbox{}\newline 
\textit{<!-- ... -->}\mbox{}\newline 
\hspace*{1em}{<\textbf{quote}\hspace*{1em}{source}="{\#chicago\textunderscore 15\textunderscore ed}">}Grammatical theories are in flux, and the more we learn, the\mbox{}\newline 
\hspace*{1em}\hspace*{1em} less we seem to know.{</\textbf{quote}>}\mbox{}\newline 
\textit{<!-- ... -->}\mbox{}\newline 
{</\textbf{p}>}\mbox{}\newline 
\textit{<!-- ... -->}\mbox{}\newline 
{<\textbf{bibl}\hspace*{1em}{xml:id}="{chicago\textunderscore 15\textunderscore ed}">}\mbox{}\newline 
\hspace*{1em}{<\textbf{title}\hspace*{1em}{level}="{m}">}The Chicago Manual of Style{</\textbf{title}>},\mbox{}\newline 
{<\textbf{edition}>}15th edition{</\textbf{edition}>}. {<\textbf{pubPlace}>}Chicago{</\textbf{pubPlace}>}: {<\textbf{publisher}>}University of\mbox{}\newline 
\hspace*{1em}\hspace*{1em} Chicago Press{</\textbf{publisher}>} ({<\textbf{date}>}2003{</\textbf{date}>}), {<\textbf{biblScope}\hspace*{1em}{unit}="{page}">}p.147{</\textbf{biblScope}>}.\mbox{}\newline 
\mbox{}\newline 
{</\textbf{bibl}>}\end{shaded}\egroup 


    \item[{Example}]
  \leavevmode\bgroup\index{elementRef=<elementRef>|exampleindex}\index{key=@key!<elementRef>|exampleindex}\index{source=@source!<elementRef>|exampleindex}\exampleFont \begin{shaded}\noindent\mbox{}{<\textbf{elementRef}\hspace*{1em}{key}="{p}"\hspace*{1em}{source}="{tei:2.0.1}"/>}\end{shaded}\egroup 

Include in the schema an element named \hyperref[TEI.p]{<p>} available from the TEI P5 2.0.1 release.
    \item[{Example}]
  \leavevmode\bgroup\index{schemaSpec=<schemaSpec>|exampleindex}\index{ident=@ident!<schemaSpec>|exampleindex}\index{source=@source!<schemaSpec>|exampleindex}\exampleFont \begin{shaded}\noindent\mbox{}{<\textbf{schemaSpec}\hspace*{1em}{ident}="{myODD}"\mbox{}\newline 
\hspace*{1em}{source}="{mycompiledODD.xml}">}\mbox{}\newline 
\textit{<!-- further declarations specifying the components required -->}\mbox{}\newline 
{</\textbf{schemaSpec}>}\end{shaded}\egroup 

Create a schema using components taken from the file \textsf{mycompiledODD.xml}.
\end{reflist}  
\begin{reflist}
\item[]\begin{specHead}{TEI.att.handFeatures}{att.handFeatures} provides attributes describing aspects of the hand in which a manuscript is written. [\textit{\hyperref[PHDH]{11.3.2.1.\ Document Hands}}]\end{specHead} 
    \item[{Module}]
  tei — \hyperref[ST]{The TEI Infrastructure}
    \item[{Members}]
  \hyperref[TEI.handNote]{handNote} \hyperref[TEI.handShift]{handShift} \hyperref[TEI.scriptNote]{scriptNote} \hyperref[TEI.typeNote]{typeNote}
    \item[{Attributes}]
  Attributes\hfil\\[-10pt]\begin{sansreflist}
    \item[@scribe]
  gives a name or other identifier for the scribe believed to be responsible for this hand.
\begin{reflist}
    \item[{Status}]
  Optional
    \item[{Datatype}]
  \hyperref[TEI.teidata.name]{teidata.name}
\end{reflist}  
    \item[@scribeRef]
  points to a full description of the scribe concerned, typically supplied by a \hyperref[TEI.person]{<person>} element elsewhere in the description.
\begin{reflist}
    \item[{Status}]
  Optional
    \item[{Datatype}]
  1–∞ occurrences of \hyperref[TEI.teidata.pointer]{teidata.pointer} separated by whitespace
\end{reflist}  
    \item[@script]
  characterizes the particular script or writing style used by this hand, for example \textit{secretary}, \textit{copperplate}, \textit{Chancery}, \textit{Italian}, etc.
\begin{reflist}
    \item[{Status}]
  Optional
    \item[{Datatype}]
  1–∞ occurrences of \hyperref[TEI.teidata.name]{teidata.name} separated by whitespace
\end{reflist}  
    \item[@scriptRef]
  points to a full description of the script or writing style used by this hand, typically supplied by a \hyperref[TEI.scriptNote]{<scriptNote>} element elsewhere in the description.
\begin{reflist}
    \item[{Status}]
  Optional
    \item[{Datatype}]
  1–∞ occurrences of \hyperref[TEI.teidata.pointer]{teidata.pointer} separated by whitespace
\end{reflist}  
    \item[@medium]
  describes the tint or type of ink, e.g. \textit{brown}, or other writing medium, e.g. \textit{pencil}
\begin{reflist}
    \item[{Status}]
  Optional
    \item[{Datatype}]
  1–∞ occurrences of \hyperref[TEI.teidata.enumerated]{teidata.enumerated} separated by whitespace
\end{reflist}  
    \item[@scope]
  specifies how widely this hand is used in the manuscript.
\begin{reflist}
    \item[{Status}]
  Optional
    \item[{Datatype}]
  \hyperref[TEI.teidata.enumerated]{teidata.enumerated}
    \item[{Legal values are:}]
  \begin{description}

\item[{sole}]only this hand is used throughout the manuscript
\item[{major}]this hand is used through most of the manuscript
\item[{minor}]this hand is used occasionally in the manuscript
\end{description} 
\end{reflist}  
\end{sansreflist}  
    \item[{Note}]
  \par
Usually either {\itshape script} or {\itshape scriptRef}, and similarly, either {\itshape scribe} or {\itshape scribeRef}, will be supplied.
\end{reflist}  
\begin{reflist}
\item[]\begin{specHead}{TEI.att.identified}{att.identified} provides the identifying attribute for elements which can be subsequently referenced by means of a {\itshape key} attribute.\end{specHead} 
    \item[{Module}]
  tagdocs — \hyperref[TD]{Documentation Elements}
    \item[{Members}]
  \hyperref[TEI.attDef]{attDef} \hyperref[TEI.classSpec]{classSpec} \hyperref[TEI.constraintSpec]{constraintSpec} \hyperref[TEI.dataSpec]{dataSpec} \hyperref[TEI.elementSpec]{elementSpec} \hyperref[TEI.macroSpec]{macroSpec} \hyperref[TEI.moduleSpec]{moduleSpec} \hyperref[TEI.paramSpec]{paramSpec} \hyperref[TEI.schemaSpec]{schemaSpec}
    \item[{Attributes}]
  \hyperref[TEI.att.combinable]{att.combinable} (\textit{@mode})  (\hyperref[TEI.att.deprecated]{att.deprecated} (\textit{@validUntil})) \hfil\\[-10pt]\begin{sansreflist}
    \item[@ident]
  supplies the identifier by which this element may be referenced.
\begin{reflist}
    \item[{Status}]
  Required
    \item[{Datatype}]
  \hyperref[TEI.teidata.name]{teidata.name}
\end{reflist}  
    \item[@predeclare]
  says whether this object should be predeclared in the \textsf{tei} infrastructure module.
\begin{reflist}
    \item[{Status}]
  Optional
    \item[{Datatype}]
  \hyperref[TEI.teidata.truthValue]{teidata.truthValue}
    \item[{Default}]
  false
\end{reflist}  
    \item[@module]
  supplies a name for the module in which this object is to be declared.
\begin{reflist}
    \item[{Status}]
  Optional
    \item[{Datatype}]
  \hyperref[TEI.teidata.xmlName]{teidata.xmlName}
\end{reflist}  
\end{sansreflist}  
    \item[{Schematron}]
   <s:rule context="tei:elementSpec[@module]|tei:classSpec[@module]|tei:macroSpec[@module]"> <s:assert test=" (not(ancestor::tei:schemaSpec | ancestor::tei:TEI | ancestor::tei:teiCorpus))   or (not(@module) or (not(//tei:moduleSpec) and not(//tei:moduleRef))   or (//tei:moduleSpec[@ident = current()/@module]) or (//tei:moduleRef[@key   = current()/@module])) "> Specification <s:value-of select="@ident"/>: the value of the module attribute ("<s:value-of select="@module"/>")  should correspond to an existing module, via a moduleSpec or  moduleRef</s:assert> </s:rule>
\end{reflist}  
\begin{reflist}
\item[]\begin{specHead}{TEI.att.internetMedia}{att.internetMedia} provides attributes for specifying the type of a computer resource using a standard taxonomy.\end{specHead} 
    \item[{Module}]
  tei — \hyperref[ST]{The TEI Infrastructure}
    \item[{Members}]
  \hyperref[TEI.att.media]{att.media}[\hyperref[TEI.binaryObject]{binaryObject} \hyperref[TEI.graphic]{graphic} \hyperref[TEI.media]{media}] \hyperref[TEI.equiv]{equiv} \hyperref[TEI.ptr]{ptr} \hyperref[TEI.ref]{ref}
    \item[{Attributes}]
  Attributes\hfil\\[-10pt]\begin{sansreflist}
    \item[@mimeType]
  (MIME media type) specifies the applicable multimedia internet mail extension (MIME) media type
\begin{reflist}
    \item[{Status}]
  Optional
    \item[{Datatype}]
  1–∞ occurrences of \hyperref[TEI.teidata.word]{teidata.word} separated by whitespace
\end{reflist}  
\end{sansreflist}  
    \item[{Example}]
  In this example {\itshape mimeType} is used to indicate that the URL points to a TEI XML file encoded in UTF-8.\leavevmode\bgroup\index{ref=<ref>|exampleindex}\index{mimeType=@mimeType!<ref>|exampleindex}\index{target=@target!<ref>|exampleindex}\exampleFont \begin{shaded}\noindent\mbox{}{<\textbf{ref}\hspace*{1em}{mimeType}="{application/tei+xml; charset=UTF-8}"\mbox{}\newline 
\hspace*{1em}{target}="{http://sourceforge.net/p/tei/code/HEAD/tree/trunk/P5/Source/guidelines-en.xml}"/>}\end{shaded}\egroup 


    \item[{Note}]
  \par
This attribute class provides an attribute for describing a computer resource, typically available over the internet, using a value taken from a standard taxonomy. At present only a single taxonomy is supported, the Multipurpose Internet Mail Extensions (MIME) Media Type system. This typology of media types is defined by the Internet Engineering Task Force in \xref{http://www.ietf.org/rfc/rfc2046.txt}{RFC 2046}. The \xref{http://www.iana.org/assignments/media-types/}{list of types} is maintained by the Internet Assigned Numbers Authority (IANA). The {\itshape mimeType} attribute must have a value taken from this list.
\end{reflist}  
\begin{reflist}
\item[]\begin{specHead}{TEI.att.lexicographic}{att.lexicographic} provides a set of attributes for specifying standard and normalized values, grammatical functions, alternate or equivalent forms, and information about composite parts. [\textit{\hyperref[DIEN]{9.2.\ The Structure of Dictionary Entries}}]\end{specHead} 
    \item[{Module}]
  dictionaries — \hyperref[DI]{Dictionaries}
    \item[{Members}]
  \hyperref[TEI.case]{case} \hyperref[TEI.colloc]{colloc} \hyperref[TEI.def]{def} \hyperref[TEI.entryFree]{entryFree} \hyperref[TEI.etym]{etym} \hyperref[TEI.form]{form} \hyperref[TEI.gen]{gen} \hyperref[TEI.gram]{gram} \hyperref[TEI.gramGrp]{gramGrp} \hyperref[TEI.hom]{hom} \hyperref[TEI.hyph]{hyph} \hyperref[TEI.iType]{iType} \hyperref[TEI.lang]{lang} \hyperref[TEI.lbl]{lbl} \hyperref[TEI.mood]{mood} \hyperref[TEI.number]{number} \hyperref[TEI.oRef]{oRef} \hyperref[TEI.orth]{orth} \hyperref[TEI.pRef]{pRef} \hyperref[TEI.per]{per} \hyperref[TEI.pos]{pos} \hyperref[TEI.pron]{pron} \hyperref[TEI.re]{re} \hyperref[TEI.sense]{sense} \hyperref[TEI.subc]{subc} \hyperref[TEI.syll]{syll} \hyperref[TEI.tns]{tns} \hyperref[TEI.usg]{usg} \hyperref[TEI.xr]{xr}
    \item[{Attributes}]
  \hyperref[TEI.att.datcat]{att.datcat} (\textit{@datcat}, \textit{@valueDatcat}) \hyperref[TEI.att.lexicographic.normalized]{att.lexicographic.normalized} (\textit{@norm}, \textit{@orig}) \hfil\\[-10pt]\begin{sansreflist}
    \item[@expand]
  (expand) gives an expanded form of information presented more concisely in the dictionary
\begin{reflist}
    \item[{Status}]
  Optional
    \item[{Datatype}]
  \hyperref[TEI.teidata.text]{teidata.text}
    \item[]\index{gramGrp=<gramGrp>|exampleindex}\index{pos=<pos>|exampleindex}\index{expand=@expand!<pos>|exampleindex}\exampleFont {<\textbf{gramGrp}>}\mbox{}\newline 
\hspace*{1em}{<\textbf{pos}\hspace*{1em}{expand}="{noun}">}n{</\textbf{pos}>}\mbox{}\newline 
{</\textbf{gramGrp}>}
\end{reflist}  
    \item[@split]
  (split) gives the list of split values for a merged form
\begin{reflist}
    \item[{Status}]
  Optional
    \item[{Datatype}]
  \hyperref[TEI.teidata.text]{teidata.text}
\end{reflist}  
    \item[@value]
  (value) gives a value which lacks any realization in the printed source text.
\begin{reflist}
    \item[{Status}]
  Optional
    \item[{Datatype}]
  \hyperref[TEI.teidata.text]{teidata.text}
\end{reflist}  
    \item[@location]
  (location) indicates an \hyperref[TEI.anchor]{<anchor>} element typically elsewhere in the document, but possibly in another document, which is the original location of this component.
\begin{reflist}
    \item[{Status}]
  Optional
    \item[{Datatype}]
  \hyperref[TEI.teidata.pointer]{teidata.pointer}
\end{reflist}  
    \item[@mergedIn]
  (merged into) gives a reference to another element, where the original appears as a merged form.
\begin{reflist}
    \item[{Status}]
  Optional
    \item[{Datatype}]
  \hyperref[TEI.teidata.pointer]{teidata.pointer}
\end{reflist}  
    \item[@opt]
  (optional) indicates whether the element is optional or not
\begin{reflist}
    \item[{Status}]
  Optional
    \item[{Datatype}]
  \hyperref[TEI.teidata.truthValue]{teidata.truthValue}
    \item[{Default}]
  false
\end{reflist}  
\end{sansreflist}  
\end{reflist}  
\begin{reflist}
\item[]\begin{specHead}{TEI.att.lexicographic.normalized}{att.lexicographic.normalized} provides the {\itshape norm} and {\itshape orig} attributes for usage within word-level elements in the analysis module and within lexicographic microstructure in the dictionaries module.\end{specHead} 
    \item[{Module}]
  analysis — \hyperref[AI]{Simple Analytic Mechanisms}
    \item[{Members}]
  \hyperref[TEI.att.lexicographic]{att.lexicographic}[\hyperref[TEI.case]{case} \hyperref[TEI.colloc]{colloc} \hyperref[TEI.def]{def} \hyperref[TEI.entryFree]{entryFree} \hyperref[TEI.etym]{etym} \hyperref[TEI.form]{form} \hyperref[TEI.gen]{gen} \hyperref[TEI.gram]{gram} \hyperref[TEI.gramGrp]{gramGrp} \hyperref[TEI.hom]{hom} \hyperref[TEI.hyph]{hyph} \hyperref[TEI.iType]{iType} \hyperref[TEI.lang]{lang} \hyperref[TEI.lbl]{lbl} \hyperref[TEI.mood]{mood} \hyperref[TEI.number]{number} \hyperref[TEI.oRef]{oRef} \hyperref[TEI.orth]{orth} \hyperref[TEI.pRef]{pRef} \hyperref[TEI.per]{per} \hyperref[TEI.pos]{pos} \hyperref[TEI.pron]{pron} \hyperref[TEI.re]{re} \hyperref[TEI.sense]{sense} \hyperref[TEI.subc]{subc} \hyperref[TEI.syll]{syll} \hyperref[TEI.tns]{tns} \hyperref[TEI.usg]{usg} \hyperref[TEI.xr]{xr}] \hyperref[TEI.att.linguistic]{att.linguistic}[\hyperref[TEI.pc]{pc} \hyperref[TEI.w]{w}]
    \item[{Attributes}]
  Attributes\hfil\\[-10pt]\begin{sansreflist}
    \item[@norm]
  (normalized) provides the normalized/standardized form of information present in the source text in a non-normalized form
\begin{reflist}
    \item[{Status}]
  Optional
    \item[{Datatype}]
  \hyperref[TEI.teidata.text]{teidata.text}
    \item[]Normalization of part-of-speech information within a dictionary entry.\index{gramGrp=<gramGrp>|exampleindex}\index{pos=<pos>|exampleindex}\index{norm=@norm!<pos>|exampleindex}\exampleFont {<\textbf{gramGrp}>}\mbox{}\newline 
\hspace*{1em}{<\textbf{pos}\hspace*{1em}{norm}="{noun}">}n{</\textbf{pos}>}\mbox{}\newline 
{</\textbf{gramGrp}>}
    \item[]Normalization of a source form in a tokenized historical corpus.\exampleFont {<\textbf{s}>}\mbox{}\newline 
\hspace*{1em}{<\textbf{w}>}for{</\textbf{w}>}\mbox{}\newline 
\hspace*{1em}{<\textbf{w}\hspace*{1em}{norm}="{virtue's}">}vertues{</\textbf{w}>}\mbox{}\newline 
\hspace*{1em}{<\textbf{w}>}sake{</\textbf{w}>}\mbox{}\newline 
{</\textbf{s}>}
    \item[]\exampleFont {<\textbf{s}>}\mbox{}\newline 
\hspace*{1em}{<\textbf{w}\hspace*{1em}{norm}="{persuasion}">}perswasion{</\textbf{w}>}\mbox{}\newline 
\hspace*{1em}{<\textbf{w}>}of{</\textbf{w}>}\mbox{}\newline 
\hspace*{1em}{<\textbf{w}\hspace*{1em}{norm}="{Unity}">}Vnitie{</\textbf{w}>}\mbox{}\newline 
{</\textbf{s}>}
    \item[]Example of normalization from \xref{http://www.deutschestextarchiv.de/anonym_aviso_1609/258}{Aviso. Relation oder Zeitung. Wolfenbüttel, 1609. In: Deutsches Textarchiv}.\exampleFont {<\textbf{s}>}\mbox{}\newline 
\hspace*{1em}{<\textbf{w}\hspace*{1em}{norm}="{freiwillig}">}freywillig{</\textbf{w}>}\mbox{}\newline 
\hspace*{1em}{<\textbf{pc}\hspace*{1em}{norm}="{,}"\mbox{}\newline 
\hspace*{1em}\hspace*{1em}{join}="{left}">}/{</\textbf{pc}>}\mbox{}\newline 
\hspace*{1em}{<\textbf{w}\hspace*{1em}{norm}="{unbedrängt}">}vnbedraͤngt{</\textbf{w}>}\mbox{}\newline 
\hspace*{1em}{<\textbf{w}\hspace*{1em}{norm}="{und}">}vnd{</\textbf{w}>}\mbox{}\newline 
\hspace*{1em}{<\textbf{w}\hspace*{1em}{norm}="{unverhindert}">}vnuerhindert{</\textbf{w}>}\mbox{}\newline 
{</\textbf{s}>}
    \item[]\exampleFont {<\textbf{w}\hspace*{1em}{norm}="{Teil}">}Theyll{</\textbf{w}>}
    \item[]\exampleFont {<\textbf{w}\hspace*{1em}{norm}="{Freude}">}Frewde{</\textbf{w}>}
\end{reflist}  
    \item[@orig]
  (original) gives the original string or is the empty string when the element does not appear in the source text.
\begin{reflist}
    \item[{Status}]
  Optional
    \item[{Datatype}]
  \hyperref[TEI.teidata.text]{teidata.text}
    \item[]Example from a language documentation project of the Mixtepec-Mixtec language (ISO 639-3: 'mix'). This is a use case where speakers spell something incorrectly but we would like to preserve it for any number of reasons, the use of {\itshape orig} is essential and could have uses for both the speaker to see past mistakes, researchers to get insight into how untrained speakers write their language instinctually (in contrast to prescribed convention), etc.:\index{w=<w>|exampleindex}\index{orig=@orig!<w>|exampleindex}\exampleFont {<\textbf{w}\hspace*{1em}{orig}="{ntsa sia'i}">}ntsasia'i{</\textbf{w}>}
    \item[]Example from the \xref{https://earlyprint.org}{EarlyPrint} project. Fragment of text where obvious errors have been corrected but the original forms remain recorded:\index{w=<w>|exampleindex}\index{lemma=@lemma!<w>|exampleindex}\index{pos=@pos!<w>|exampleindex}\index{w=<w>|exampleindex}\index{lemma=@lemma!<w>|exampleindex}\index{pos=@pos!<w>|exampleindex}\index{w=<w>|exampleindex}\index{lemma=@lemma!<w>|exampleindex}\index{pos=@pos!<w>|exampleindex}\index{w=<w>|exampleindex}\index{lemma=@lemma!<w>|exampleindex}\index{pos=@pos!<w>|exampleindex}\index{orig=@orig!<w>|exampleindex}\exampleFont {<\textbf{w}\hspace*{1em}{lemma}="{he}"\mbox{}\newline 
\hspace*{1em}{pos}="{pns}"\mbox{}\newline 
\hspace*{1em}{xml:id}="{b1afj-003-a-0950}">}he{</\textbf{w}>}\mbox{}\newline 
{<\textbf{w}\hspace*{1em}{lemma}="{have}"\mbox{}\newline 
\hspace*{1em}{pos}="{vvz}"\mbox{}\newline 
\hspace*{1em}{xml:id}="{b1afj-003-a-0960}">}hath{</\textbf{w}>}\mbox{}\newline 
{<\textbf{w}\hspace*{1em}{lemma}="{bring}"\mbox{}\newline 
\hspace*{1em}{pos}="{vvn}"\mbox{}\newline 
\hspace*{1em}{xml:id}="{b1afj-003-a-0970}">}brought{</\textbf{w}>}\mbox{}\newline 
{<\textbf{w}\hspace*{1em}{lemma}="{forth}"\mbox{}\newline 
\hspace*{1em}{pos}="{av}"\mbox{}\newline 
\hspace*{1em}{xml:id}="{b1afj-003-a-0980}"\mbox{}\newline 
\hspace*{1em}{orig}="{sorth}">}forth{</\textbf{w}>}
    \item[]An example from the EarlyPrint project showing the use of both {\itshape norm} and {\itshape orig}. The {\itshape orig} attribute preserves the original version (sometimes with spelling errors, often with printer abbreviations), the element content resolves printer abbreviations but retains the original orthography, and the {\itshape norm} attribute holds normalized values:\index{w=<w>|exampleindex}\index{lemma=@lemma!<w>|exampleindex}\index{pos=@pos!<w>|exampleindex}\index{norm=@norm!<w>|exampleindex}\index{orig=@orig!<w>|exampleindex}\exampleFont {<\textbf{w}\hspace*{1em}{lemma}="{commandment}"\mbox{}\newline 
\hspace*{1em}{pos}="{n1}"\mbox{}\newline 
\hspace*{1em}{norm}="{commandment}"\mbox{}\newline 
\hspace*{1em}{xml:id}="{b9avr-018-a-7720}"\mbox{}\newline 
\hspace*{1em}{orig}="{commandemēt}">}commandement{</\textbf{w}>}
\end{reflist}  
\end{sansreflist}  
    \item[{Note}]
  \par
It needs to be stressed that the two attributes in this class are meant for strictly lexicographic and linguistic uses, and not for editorial interventions. For the latter, the mechanism based on \hyperref[TEI.choice]{<choice>}, \hyperref[TEI.orig]{<orig>}, and \hyperref[TEI.reg]{<reg>} needs to be employed.
\end{reflist}  
\begin{reflist}
\item[]\begin{specHead}{TEI.att.linguistic}{att.linguistic} provides a set of attributes concerning linguistic features of tokens, for usage within token-level elements, specifically \hyperref[TEI.w]{<w>} and \hyperref[TEI.pc]{<pc>} in the analysis module. [\textit{\hyperref[AILALW]{17.4.2.\ Lightweight Linguistic Annotation}}]\end{specHead} 
    \item[{Module}]
  analysis — \hyperref[AI]{Simple Analytic Mechanisms}
    \item[{Members}]
  \hyperref[TEI.pc]{pc} \hyperref[TEI.w]{w}
    \item[{Attributes}]
  \hyperref[TEI.att.lexicographic.normalized]{att.lexicographic.normalized} (\textit{@norm}, \textit{@orig}) \hfil\\[-10pt]\begin{sansreflist}
    \item[@lemma]
  provides a lemma (base form) for the word, typically uninflected and serving both as an identifier (e.g. in dictionary contexts, as a headword), and as a basis for potential inflections.
\begin{reflist}
    \item[{Status}]
  Optional
    \item[{Datatype}]
  \hyperref[TEI.teidata.text]{teidata.text}
    \item[]\index{w=<w>|exampleindex}\index{lemma=@lemma!<w>|exampleindex}\exampleFont {<\textbf{w}\hspace*{1em}{lemma}="{wife}">}wives{</\textbf{w}>}
    \item[]\index{w=<w>|exampleindex}\index{lemma=@lemma!<w>|exampleindex}\exampleFont {<\textbf{w}\hspace*{1em}{lemma}="{Arznei}">}Artzeneyen{</\textbf{w}>}
\end{reflist}  
    \item[@lemmaRef]
  provides a pointer to a definition of the lemma for the word, for example in an online lexicon.
\begin{reflist}
    \item[{Status}]
  Optional
    \item[{Datatype}]
  \hyperref[TEI.teidata.pointer]{teidata.pointer}
    \item[]\index{w=<w>|exampleindex}\index{type=@type!<w>|exampleindex}\index{lemma=@lemma!<w>|exampleindex}\index{lemmaRef=@lemmaRef!<w>|exampleindex}\index{m=<m>|exampleindex}\index{type=@type!<m>|exampleindex}\exampleFont {<\textbf{w}\hspace*{1em}{type}="{verb}"\mbox{}\newline 
\hspace*{1em}{lemma}="{hit}"\mbox{}\newline 
\hspace*{1em}{lemmaRef}="{http://www.example.com/lexicon/hitvb.xml}">}hitt{<\textbf{m}\hspace*{1em}{type}="{suffix}">}ing{</\textbf{m}>}\mbox{}\newline 
{</\textbf{w}>}
\end{reflist}  
    \item[@pos]
  (part of speech) indicates the part of speech assigned to a token (i.e. information on whether it is a noun, adjective, or verb), usually according to some official reference vocabulary (e.g. for German: STTS, for English: CLAWS, for Polish: NKJP, etc.).
\begin{reflist}
    \item[{Status}]
  Optional
    \item[{Datatype}]
  \hyperref[TEI.teidata.text]{teidata.text}
    \item[]The German sentence ‘Wir fahren in den Urlaub.’ tagged with the Stuttgart-Tuebingen-Tagset (STTS).\index{s=<s>|exampleindex}\index{w=<w>|exampleindex}\index{pos=@pos!<w>|exampleindex}\index{w=<w>|exampleindex}\index{pos=@pos!<w>|exampleindex}\index{w=<w>|exampleindex}\index{pos=@pos!<w>|exampleindex}\index{w=<w>|exampleindex}\index{pos=@pos!<w>|exampleindex}\index{w=<w>|exampleindex}\index{pos=@pos!<w>|exampleindex}\index{w=<w>|exampleindex}\index{pos=@pos!<w>|exampleindex}\exampleFont {<\textbf{s}>}\mbox{}\newline 
\hspace*{1em}{<\textbf{w}\hspace*{1em}{pos}="{PPER}">}Wir{</\textbf{w}>}\mbox{}\newline 
\hspace*{1em}{<\textbf{w}\hspace*{1em}{pos}="{VVFIN}">}fahren{</\textbf{w}>}\mbox{}\newline 
\hspace*{1em}{<\textbf{w}\hspace*{1em}{pos}="{APPR}">}in{</\textbf{w}>}\mbox{}\newline 
\hspace*{1em}{<\textbf{w}\hspace*{1em}{pos}="{ART}">}den{</\textbf{w}>}\mbox{}\newline 
\hspace*{1em}{<\textbf{w}\hspace*{1em}{pos}="{NN}">}Urlaub{</\textbf{w}>}\mbox{}\newline 
\hspace*{1em}{<\textbf{w}\hspace*{1em}{pos}="{\$.}">}.{</\textbf{w}>}\mbox{}\newline 
{</\textbf{s}>}
    \item[]The English sentence ‘We're going to Brazil.’ tagged with the \xref{http://ucrel.lancs.ac.uk/claws5tags.html}{CLAWS-5} tagset, arranged inline (with significant whitespace).\exampleFont {<\textbf{p}>}{<\textbf{w}\hspace*{1em}{pos}="{PNP}">}We{</\textbf{w}>}{<\textbf{w}\hspace*{1em}{pos}="{VBB}">}'re{</\textbf{w}>} {<\textbf{w}\hspace*{1em}{pos}="{VVG}">}going{</\textbf{w}>} {<\textbf{w}\hspace*{1em}{pos}="{PRP}">}to{</\textbf{w}>} {<\textbf{w}\hspace*{1em}{pos}="{NP0}">}Brazil{</\textbf{w}>}{<\textbf{pc}\hspace*{1em}{pos}="{PUN}">}.{</\textbf{pc}>}{</\textbf{p}>}\newline
        
    \item[]The English sentence ‘We're going on vacation to Brazil for a month!’ tagged with the \xref{http://ucrel.lancs.ac.uk/claws7tags.html}{CLAWS-7} tagset and arranged sequentially.\index{p=<p>|exampleindex}\index{w=<w>|exampleindex}\index{pos=@pos!<w>|exampleindex}\index{w=<w>|exampleindex}\index{pos=@pos!<w>|exampleindex}\index{w=<w>|exampleindex}\index{pos=@pos!<w>|exampleindex}\index{w=<w>|exampleindex}\index{pos=@pos!<w>|exampleindex}\index{w=<w>|exampleindex}\index{pos=@pos!<w>|exampleindex}\index{w=<w>|exampleindex}\index{pos=@pos!<w>|exampleindex}\index{w=<w>|exampleindex}\index{pos=@pos!<w>|exampleindex}\index{w=<w>|exampleindex}\index{pos=@pos!<w>|exampleindex}\index{w=<w>|exampleindex}\index{pos=@pos!<w>|exampleindex}\index{w=<w>|exampleindex}\index{pos=@pos!<w>|exampleindex}\index{pc=<pc>|exampleindex}\index{pos=@pos!<pc>|exampleindex}\exampleFont {<\textbf{p}>}\mbox{}\newline 
\hspace*{1em}{<\textbf{w}\hspace*{1em}{pos}="{PPIS2}">}We{</\textbf{w}>}\mbox{}\newline 
\hspace*{1em}{<\textbf{w}\hspace*{1em}{pos}="{VBR}">}'re{</\textbf{w}>}\mbox{}\newline 
\hspace*{1em}{<\textbf{w}\hspace*{1em}{pos}="{VVG}">}going{</\textbf{w}>}\mbox{}\newline 
\hspace*{1em}{<\textbf{w}\hspace*{1em}{pos}="{II}">}on{</\textbf{w}>}\mbox{}\newline 
\hspace*{1em}{<\textbf{w}\hspace*{1em}{pos}="{NN1}">}vacation{</\textbf{w}>}\mbox{}\newline 
\hspace*{1em}{<\textbf{w}\hspace*{1em}{pos}="{II}">}to{</\textbf{w}>}\mbox{}\newline 
\hspace*{1em}{<\textbf{w}\hspace*{1em}{pos}="{NP1}">}Brazil{</\textbf{w}>}\mbox{}\newline 
\hspace*{1em}{<\textbf{w}\hspace*{1em}{pos}="{IF}">}for{</\textbf{w}>}\mbox{}\newline 
\hspace*{1em}{<\textbf{w}\hspace*{1em}{pos}="{AT1}">}a{</\textbf{w}>}\mbox{}\newline 
\hspace*{1em}{<\textbf{w}\hspace*{1em}{pos}="{NNT1}">}month{</\textbf{w}>}\mbox{}\newline 
\hspace*{1em}{<\textbf{pc}\hspace*{1em}{pos}="{!}">}!{</\textbf{pc}>}\mbox{}\newline 
{</\textbf{p}>}
\end{reflist}  
    \item[@msd]
  (morphosyntactic description) supplies morphosyntactic information for a token, usually according to some official reference vocabulary (e.g. for German: \xref{http://www.ims.uni-stuttgart.de/forschung/ressourcen/lexika/TagSets/stts-1999.pdf}{STTS-large tagset}; for a feature description system designed as (pragmatically) universal, see \xref{http://universaldependencies.org/u/feat/index.html}{Universal Features}).
\begin{reflist}
    \item[{Status}]
  Optional
    \item[{Datatype}]
  \hyperref[TEI.teidata.text]{teidata.text}
    \item[]\index{ab=<ab>|exampleindex}\index{w=<w>|exampleindex}\index{pos=@pos!<w>|exampleindex}\index{msd=@msd!<w>|exampleindex}\index{w=<w>|exampleindex}\index{pos=@pos!<w>|exampleindex}\index{msd=@msd!<w>|exampleindex}\index{w=<w>|exampleindex}\index{pos=@pos!<w>|exampleindex}\index{msd=@msd!<w>|exampleindex}\index{w=<w>|exampleindex}\index{pos=@pos!<w>|exampleindex}\index{msd=@msd!<w>|exampleindex}\index{w=<w>|exampleindex}\index{pos=@pos!<w>|exampleindex}\index{msd=@msd!<w>|exampleindex}\index{pc=<pc>|exampleindex}\index{pos=@pos!<pc>|exampleindex}\index{msd=@msd!<pc>|exampleindex}\exampleFont {<\textbf{ab}>}\mbox{}\newline 
\hspace*{1em}{<\textbf{w}\hspace*{1em}{pos}="{PPER}"\mbox{}\newline 
\hspace*{1em}\hspace*{1em}{msd}="{1.Pl.*.Nom}">}Wir{</\textbf{w}>}\mbox{}\newline 
\hspace*{1em}{<\textbf{w}\hspace*{1em}{pos}="{VVFIN}"\mbox{}\newline 
\hspace*{1em}\hspace*{1em}{msd}="{1.Pl.Pres.Ind}">}fahren{</\textbf{w}>}\mbox{}\newline 
\hspace*{1em}{<\textbf{w}\hspace*{1em}{pos}="{APPR}"\mbox{}\newline 
\hspace*{1em}\hspace*{1em}{msd}="{--}">}in{</\textbf{w}>}\mbox{}\newline 
\hspace*{1em}{<\textbf{w}\hspace*{1em}{pos}="{ART}"\mbox{}\newline 
\hspace*{1em}\hspace*{1em}{msd}="{Def.Masc.Akk.Sg}">}den{</\textbf{w}>}\mbox{}\newline 
\hspace*{1em}{<\textbf{w}\hspace*{1em}{pos}="{NN}"\mbox{}\newline 
\hspace*{1em}\hspace*{1em}{msd}="{Masc.Akk.Sg}">}Urlaub{</\textbf{w}>}\mbox{}\newline 
\hspace*{1em}{<\textbf{pc}\hspace*{1em}{pos}="{\$.}"\mbox{}\newline 
\hspace*{1em}\hspace*{1em}{msd}="{--}">}.{</\textbf{pc}>}\mbox{}\newline 
{</\textbf{ab}>}
\end{reflist}  
    \item[@join]
  when present, it provides information on whether the token in question is adjacent to another, and if so, on which side. The definition of this attribute is adapted from ISO MAF (Morpho-syntactic Annotation Framework), ISO 24611:2012.
\begin{reflist}
    \item[{Status}]
  Optional
    \item[{Datatype}]
  \hyperref[TEI.teidata.text]{teidata.text}
    \item[{Legal values are:}]
  \begin{description}

\item[{no}](the token is not adjacent to another)
\item[{left}](there is no whitespace on the left side of the token)
\item[{right}](there is no whitespace on the right side of the token)
\item[{both}](there is no whitespace on either side of the token)
\item[{overlap}](the token overlaps with another; other devices (specifying the extent and the area of overlap) are needed to more precisely locate this token in the character stream)
\end{description} 
    \item[]The example below assumes that the lack of whitespace is marked redundantly, by using the appropriate values of {\itshape join}.\index{s=<s>|exampleindex}\index{pc=<pc>|exampleindex}\index{join=@join!<pc>|exampleindex}\index{w=<w>|exampleindex}\index{join=@join!<w>|exampleindex}\index{w=<w>|exampleindex}\index{w=<w>|exampleindex}\index{w=<w>|exampleindex}\index{join=@join!<w>|exampleindex}\index{pc=<pc>|exampleindex}\index{join=@join!<pc>|exampleindex}\index{pc=<pc>|exampleindex}\index{join=@join!<pc>|exampleindex}\exampleFont {<\textbf{s}>}\mbox{}\newline 
\hspace*{1em}{<\textbf{pc}\hspace*{1em}{join}="{right}">}"{</\textbf{pc}>}\mbox{}\newline 
\hspace*{1em}{<\textbf{w}\hspace*{1em}{join}="{left}">}Friends{</\textbf{w}>}\mbox{}\newline 
\hspace*{1em}{<\textbf{w}>}will{</\textbf{w}>}\mbox{}\newline 
\hspace*{1em}{<\textbf{w}>}be{</\textbf{w}>}\mbox{}\newline 
\hspace*{1em}{<\textbf{w}\hspace*{1em}{join}="{right}">}friends{</\textbf{w}>}\mbox{}\newline 
\hspace*{1em}{<\textbf{pc}\hspace*{1em}{join}="{both}">}.{</\textbf{pc}>}\mbox{}\newline 
\hspace*{1em}{<\textbf{pc}\hspace*{1em}{join}="{left}">}"{</\textbf{pc}>}\mbox{}\newline 
{</\textbf{s}>}Note that a project may make a decision to only indicate lack of whitespace in one direction, or do that non-redundantly. The existing proposal is the broadest possible, on the assumption that we adopt the "streamable view", where all the information on the current element needs to be represented locally.
    \item[]The English sentence ‘We're going on vacation.’ tagged with the CLAWS-5 tagset, arranged sequentially, tagged on the assumption that only the lack of the preceding whitespace is indicated.\index{p=<p>|exampleindex}\index{w=<w>|exampleindex}\index{pos=@pos!<w>|exampleindex}\index{w=<w>|exampleindex}\index{pos=@pos!<w>|exampleindex}\index{join=@join!<w>|exampleindex}\index{w=<w>|exampleindex}\index{pos=@pos!<w>|exampleindex}\index{w=<w>|exampleindex}\index{pos=@pos!<w>|exampleindex}\index{w=<w>|exampleindex}\index{pos=@pos!<w>|exampleindex}\index{pc=<pc>|exampleindex}\index{pos=@pos!<pc>|exampleindex}\index{join=@join!<pc>|exampleindex}\exampleFont {<\textbf{p}>}\mbox{}\newline 
\hspace*{1em}{<\textbf{w}\hspace*{1em}{pos}="{PNP}">}We{</\textbf{w}>}\mbox{}\newline 
\hspace*{1em}{<\textbf{w}\hspace*{1em}{pos}="{VBB}"\mbox{}\newline 
\hspace*{1em}\hspace*{1em}{join}="{left}">}'re{</\textbf{w}>}\mbox{}\newline 
\hspace*{1em}{<\textbf{w}\hspace*{1em}{pos}="{VVG}">}going{</\textbf{w}>}\mbox{}\newline 
\hspace*{1em}{<\textbf{w}\hspace*{1em}{pos}="{PRP}">}on{</\textbf{w}>}\mbox{}\newline 
\hspace*{1em}{<\textbf{w}\hspace*{1em}{pos}="{NN1}">}vacation{</\textbf{w}>}\mbox{}\newline 
\hspace*{1em}{<\textbf{pc}\hspace*{1em}{pos}="{PUN}"\mbox{}\newline 
\hspace*{1em}\hspace*{1em}{join}="{left}">}.{</\textbf{pc}>}\mbox{}\newline 
{</\textbf{p}>}
\end{reflist}  
\end{sansreflist}  
    \item[{Note}]
  \par
These attributes make it possible to encode simple language corpora and to add a layer of linguistic information to any tokenized resource. See section \textit{\hyperref[AILALW]{17.4.2.\ Lightweight Linguistic Annotation}} for discussion.
\end{reflist}  
\begin{reflist}
\item[]\begin{specHead}{TEI.att.measurement}{att.measurement} provides attributes to represent a regularized or normalized measurement.\end{specHead} 
    \item[{Module}]
  tei — \hyperref[ST]{The TEI Infrastructure}
    \item[{Members}]
  \hyperref[TEI.measure]{measure} \hyperref[TEI.measureGrp]{measureGrp} \hyperref[TEI.unit]{unit}
    \item[{Attributes}]
  Attributes\hfil\\[-10pt]\begin{sansreflist}
    \item[@unit]
  (unit) indicates the units used for the measurement, usually using the standard symbol for the desired units.
\begin{reflist}
    \item[{Status}]
  Optional
    \item[{Datatype}]
  \hyperref[TEI.teidata.enumerated]{teidata.enumerated}
    \item[{Suggested values include:}]
  \begin{description}

\item[{m}](metre) SI base unit of length
\item[{kg}](kilogram) SI base unit of mass
\item[{s}](second) SI base unit of time
\item[{Hz}](hertz) SI unit of frequency
\item[{Pa}](pascal) SI unit of pressure or stress
\item[{Ω}](ohm) SI unit of electric resistance
\item[{L}](litre) 1 dm³
\item[{t}](tonne) 10³ kg
\item[{ha}](hectare) 1 hm²
\item[{Å}](ångström) 10⁻¹⁰ m
\item[{mL}](millilitre)
\item[{cm}](centimetre)
\item[{dB}](decibel) see remarks, below
\item[{kbit}](kilobit) 10³ or 1000 bits
\item[{Kibit}](kibibit) 2¹⁰ or 1024 bits
\item[{kB}](kilobyte) 10³ or 1000 bytes
\item[{KiB}](kibibyte) 2¹⁰ or 1024 bytes
\item[{MB}](megabyte) 10⁶ or 1 000 000 bytes
\item[{MiB}](mebibyte) 2²⁰ or 1 048 576 bytes
\end{description} 
    \item[{Note}]
  \par
If the measurement being represented is not expressed in a particular unit, but rather is a number of discrete items, the unit count should be used, or the {\itshape unit} attribute may be left unspecified.\par
Wherever appropriate, a recognized SI unit name should be used (see further \url{http://www.bipm.org/en/publications/si-brochure/}; \url{http://physics.nist.gov/cuu/Units/}). The list above is indicative rather than exhaustive.
\end{reflist}  
    \item[@unitRef]
  points to a unique identifier stored in the {\itshape xml:id} of a \hyperref[TEI.unitDef]{<unitDef>} element that defines a unit of measure.
\begin{reflist}
    \item[{Status}]
  Optional
    \item[{Datatype}]
  \hyperref[TEI.teidata.pointer]{teidata.pointer}
\end{reflist}  
    \item[@quantity]
  (quantity) specifies the number of the specified units that comprise the measurement
\begin{reflist}
    \item[{Status}]
  Optional
    \item[{Datatype}]
  \hyperref[TEI.teidata.numeric]{teidata.numeric}
\end{reflist}  
    \item[@commodity]
  (commodity) indicates the substance that is being measured
\begin{reflist}
    \item[{Status}]
  Optional
    \item[{Datatype}]
  1–∞ occurrences of \hyperref[TEI.teidata.word]{teidata.word} separated by whitespace
    \item[{Note}]
  \par
In general, when the commodity is made of discrete entities, the plural form should be used, even when the measurement is of only one of them.
\end{reflist}  
\end{sansreflist}  
    \item[{Schematron}]
   <sch:rule context="tei:*[@unitRef]"> <sch:report test="@unit" role="info">The @unit attribute may be unnecessary when @unitRef is present.</sch:report> </sch:rule>
    \item[{Note}]
  \par
This attribute class provides a triplet of attributes that may be used either to regularize the values of the measurement being encoded, or to normalize them with respect to a standard measurement system. \par\bgroup\index{l=<l>|exampleindex}\index{measure=<measure>|exampleindex}\index{quantity=@quantity!<measure>|exampleindex}\index{unit=@unit!<measure>|exampleindex}\index{commodity=@commodity!<measure>|exampleindex}\index{l=<l>|exampleindex}\index{measure=<measure>|exampleindex}\index{quantity=@quantity!<measure>|exampleindex}\index{unit=@unit!<measure>|exampleindex}\index{commodity=@commodity!<measure>|exampleindex}\exampleFont \begin{shaded}\noindent\mbox{}{<\textbf{l}>}\mbox{}\newline 
\textit{<!-- regularization:-->}\mbox{}\newline 
 So weren't you gonna buy {<\textbf{measure}\hspace*{1em}{quantity}="{0.5}"\hspace*{1em}{unit}="{gal}"\mbox{}\newline 
\hspace*{1em}\hspace*{1em}{commodity}="{ice cream}">}half\mbox{}\newline 
\hspace*{1em}\hspace*{1em} a gallon{</\textbf{measure}>}, baby\mbox{}\newline 
{</\textbf{l}>}\mbox{}\newline 
{<\textbf{l}>}\mbox{}\newline 
\textit{<!-- normalization: -->}\mbox{}\newline 
 So won't you go and buy {<\textbf{measure}\hspace*{1em}{quantity}="{1.893}"\hspace*{1em}{unit}="{L}"\mbox{}\newline 
\hspace*{1em}\hspace*{1em}{commodity}="{ice cream}">}half\mbox{}\newline 
\hspace*{1em}\hspace*{1em} a gallon{</\textbf{measure}>}, baby?\mbox{}\newline 
{</\textbf{l}>}\end{shaded}\egroup\par \noindent    \par
The unit should normally be named using the standard symbol for an SI unit (see further \url{http://www.bipm.org/en/publications/si-brochure/}; \url{http://physics.nist.gov/cuu/Units/}). However, encoders may also specify measurements using informally defined units such as lines or characters.
\end{reflist}  
\begin{reflist}
\item[]\begin{specHead}{TEI.att.media}{att.media} provides attributes for specifying display and related properties of external media.\end{specHead} 
    \item[{Module}]
  tei — \hyperref[ST]{The TEI Infrastructure}
    \item[{Members}]
  \hyperref[TEI.binaryObject]{binaryObject} \hyperref[TEI.graphic]{graphic} \hyperref[TEI.media]{media}
    \item[{Attributes}]
  \hyperref[TEI.att.internetMedia]{att.internetMedia} (\textit{@mimeType}) \hfil\\[-10pt]\begin{sansreflist}
    \item[@width]
  Where the media are displayed, indicates the display width
\begin{reflist}
    \item[{Status}]
  Optional
    \item[{Datatype}]
  \hyperref[TEI.teidata.outputMeasurement]{teidata.outputMeasurement}
\end{reflist}  
    \item[@height]
  Where the media are displayed, indicates the display height
\begin{reflist}
    \item[{Status}]
  Optional
    \item[{Datatype}]
  \hyperref[TEI.teidata.outputMeasurement]{teidata.outputMeasurement}
\end{reflist}  
    \item[@scale]
  Where the media are displayed, indicates a scale factor to be applied when generating the desired display size
\begin{reflist}
    \item[{Status}]
  Optional
    \item[{Datatype}]
  \hyperref[TEI.teidata.numeric]{teidata.numeric}
\end{reflist}  
\end{sansreflist}  
\end{reflist}  
\begin{reflist}
\item[]\begin{specHead}{TEI.att.metrical}{att.metrical} defines a set of attributes which certain elements may use to represent metrical information. [\textit{\hyperref[VEME]{6.4.\ Rhyme and Metrical Analysis}}]\end{specHead} 
    \item[{Module}]
  verse — \hyperref[VE]{Verse}
    \item[{Members}]
  \hyperref[TEI.att.divLike]{att.divLike}[\hyperref[TEI.div]{div} \hyperref[TEI.div1]{div1} \hyperref[TEI.div2]{div2} \hyperref[TEI.div3]{div3} \hyperref[TEI.div4]{div4} \hyperref[TEI.div5]{div5} \hyperref[TEI.div6]{div6} \hyperref[TEI.div7]{div7} \hyperref[TEI.lg]{lg}] \hyperref[TEI.att.segLike]{att.segLike}[\hyperref[TEI.c]{c} \hyperref[TEI.cl]{cl} \hyperref[TEI.m]{m} \hyperref[TEI.pc]{pc} \hyperref[TEI.phr]{phr} \hyperref[TEI.s]{s} \hyperref[TEI.seg]{seg} \hyperref[TEI.w]{w}] \hyperref[TEI.l]{l}
    \item[{Attributes}]
  Attributes\hfil\\[-10pt]\begin{sansreflist}
    \item[@met]
  (metrical structure, conventional) contains a user-specified encoding for the conventional metrical structure of the element.
\begin{reflist}
    \item[{Status}]
  Recommended
    \item[{Datatype}]
  \xref{https://www.w3.org/TR/xmlschema-2/\#token}{token}
    \item[{Note}]
  \par
The pattern may be specified by means of either a standard term for the kind of metrical unit (e.g. \textit{hexameter}) or an encoded representation for the metrical pattern (e.g. \textit{+--+-+-+-+-}). In either case, the notation used should be documented by a \hyperref[TEI.metDecl]{<metDecl>} element within the \hyperref[TEI.encodingDesc]{<encodingDesc>} of the associated header.\par
Where this attribute is not specified, the metrical pattern for the element concerned is understood to be inherited from its parent.
\end{reflist}  
    \item[@real]
  (metrical structure, realized) contains a user-specified encoding for the actual realization of the conventional metrical structure applicable to the element.
\begin{reflist}
    \item[{Status}]
  Optional
    \item[{Datatype}]
  \xref{https://www.w3.org/TR/xmlschema-2/\#token}{token}
    \item[{Note}]
  \par
The pattern may be specified by means of either a standard term for the kind of metrical unit (e.g. \textit{hexameter}) or an encoded representation for the metrical pattern (e.g. \textit{+--+-+-+-+-}). In either case, the notation used should be documented by a \hyperref[TEI.metDecl]{<metDecl>} element within the \hyperref[TEI.encodingDesc]{<encodingDesc>} of the associated header.\par
Where this attribute is not specified, the metrical realization for the element concerned is understood to be identical to that specified or implied for the {\itshape met} attribute.
\end{reflist}  
    \item[@rhyme]
  (rhyme scheme) specifies the rhyme scheme applicable to a group of verse lines.
\begin{reflist}
    \item[{Status}]
  Recommended
    \item[{Datatype}]
  \xref{https://www.w3.org/TR/xmlschema-2/\#token}{token}
    \item[{Note}]
  \par
By default, the rhyme scheme is expressed as a string of alphabetic characters each corresponding with a rhyming line. Any non-rhyming lines should be represented by a hyphen or an X. Alternative notations may be defined as for {\itshape met} by use of the \hyperref[TEI.metDecl]{<metDecl>} element in the TEI header.\par
When the default notation is used, it does not make sense to specify this attribute on any unit smaller than a line. Nor does the default notation provide any way to record internal rhyme, or to specify non-conventional rhyming practice. These extensions would require user-defined alternative notations.
\end{reflist}  
\end{sansreflist}  
\end{reflist}  
\begin{reflist}
\item[]\begin{specHead}{TEI.att.milestoneUnit}{att.milestoneUnit} provides an attribute to indicate the type of section which is changing at a specific milestone. [\textit{\hyperref[CORS5]{3.11.3.\ Milestone Elements}} \textit{\hyperref[HD54M]{2.3.6.3.\ Milestone Method}} \textit{\hyperref[HD54]{2.3.6.\ The Reference System Declaration}}]\end{specHead} 
    \item[{Module}]
  core — \hyperref[CO]{Elements Available in All TEI Documents}
    \item[{Members}]
  \hyperref[TEI.milestone]{milestone} \hyperref[TEI.refState]{refState}
    \item[{Attributes}]
  Attributes\hfil\\[-10pt]\begin{sansreflist}
    \item[@unit]
  provides a conventional name for the kind of section changing at this milestone.
\begin{reflist}
    \item[{Status}]
  Required
    \item[{Datatype}]
  \hyperref[TEI.teidata.enumerated]{teidata.enumerated}
    \item[{Suggested values include:}]
  \begin{description}

\item[{page}]physical page breaks (synonymous with the \hyperref[TEI.pb]{<pb>} element).
\item[{column}]column breaks.
\item[{line}]line breaks (synonymous with the \hyperref[TEI.lb]{<lb>} element).
\item[{book}]any units termed book, liber, etc.
\item[{poem}]individual poems in a collection.
\item[{canto}]cantos or other major sections of a poem.
\item[{speaker}]changes of speaker or narrator.
\item[{stanza}]stanzas within a poem, book, or canto.
\item[{act}]acts within a play.
\item[{scene}]scenes within a play or act.
\item[{section}]sections of any kind.
\item[{absent}]passages not present in the reference edition.
\item[{unnumbered}]passages present in the text, but not to be included as part of the reference.
\end{description} 
    \item[]\index{milestone=<milestone>|exampleindex}\index{n=@n!<milestone>|exampleindex}\index{ed=@ed!<milestone>|exampleindex}\index{unit=@unit!<milestone>|exampleindex}\index{milestone=<milestone>|exampleindex}\index{n=@n!<milestone>|exampleindex}\index{ed=@ed!<milestone>|exampleindex}\index{unit=@unit!<milestone>|exampleindex}\exampleFont {<\textbf{milestone}\hspace*{1em}{n}="{23}"\mbox{}\newline 
\hspace*{1em}{ed}="{La}"\mbox{}\newline 
\hspace*{1em}{unit}="{Dreissiger}"/>}\mbox{}\newline 
 ... {<\textbf{milestone}\hspace*{1em}{n}="{24}"\mbox{}\newline 
\hspace*{1em}{ed}="{AV}"\mbox{}\newline 
\hspace*{1em}{unit}="{verse}"/>} ...
    \item[{Note}]
  \par
If the milestone marks the beginning of a piece of text not present in the reference edition, the special value \textit{absent} may be used as the value of {\itshape unit}. The normal interpretation is that the reference edition does not contain the text which follows, until the next \hyperref[TEI.milestone]{<milestone>} tag for the edition in question is encountered.\par
In addition to the values suggested, other terms may be appropriate (e.g. \textit{Stephanus} for the Stephanus numbers in Plato).\par
The {\itshape type} attribute may be used to characterize the unit boundary in any respect other than simply identifying the type of unit, for example as word-breaking or not.
\end{reflist}  
\end{sansreflist}  
\end{reflist}  
\begin{reflist}
\item[]\begin{specHead}{TEI.att.msClass}{att.msClass} provides an attribute to indicate text type or classification. [\textit{\hyperref[msco]{10.6.\ Intellectual Content}} \textit{\hyperref[mscoit]{10.6.1.\ The msItem and msItemStruct Elements}}]\end{specHead} 
    \item[{Module}]
  msdescription — \hyperref[MS]{Manuscript Description}
    \item[{Members}]
  \hyperref[TEI.msContents]{msContents} \hyperref[TEI.msItem]{msItem} \hyperref[TEI.msItemStruct]{msItemStruct}
    \item[{Attributes}]
  Attributes\hfil\\[-10pt]\begin{sansreflist}
    \item[@class]
  identifies the text types or classifications applicable to this item by pointing to other elements or resources defining the classification concerned. 
\begin{reflist}
    \item[{Status}]
  Optional
    \item[{Datatype}]
  1–∞ occurrences of \hyperref[TEI.teidata.pointer]{teidata.pointer} separated by whitespace
\end{reflist}  
\end{sansreflist}  
\end{reflist}  
\begin{reflist}
\item[]\begin{specHead}{TEI.att.msExcerpt}{att.msExcerpt} (manuscript excerpt) provides attributes used to describe excerpts from a manuscript placed in a description thereof. [\textit{\hyperref[msco]{10.6.\ Intellectual Content}}]\end{specHead} 
    \item[{Module}]
  msdescription — \hyperref[MS]{Manuscript Description}
    \item[{Members}]
  \hyperref[TEI.colophon]{colophon} \hyperref[TEI.explicit]{explicit} \hyperref[TEI.finalRubric]{finalRubric} \hyperref[TEI.incipit]{incipit} \hyperref[TEI.msContents]{msContents} \hyperref[TEI.msItem]{msItem} \hyperref[TEI.msItemStruct]{msItemStruct} \hyperref[TEI.quote]{quote} \hyperref[TEI.rubric]{rubric}
    \item[{Attributes}]
  Attributes\hfil\\[-10pt]\begin{sansreflist}
    \item[@defective]
  indicates whether the passage being quoted is defective, i.e. incomplete through loss or damage.
\begin{reflist}
    \item[{Status}]
  Optional
    \item[{Datatype}]
  \hyperref[TEI.teidata.xTruthValue]{teidata.xTruthValue}
\end{reflist}  
\end{sansreflist}  
    \item[{Note}]
  \par
In the case of an incipit, indicates whether the incipit as given is defective, i.e. the first words of the text as preserved, as opposed to the first words of the work itself. In the case of an explicit, indicates whether the explicit as given is defective, i.e. the final words of the text as preserved, as opposed to what the closing words would have been had the text of the work been whole.
\end{reflist}  
\begin{reflist}
\item[]\begin{specHead}{TEI.att.namespaceable}{att.namespaceable} provides an attribute indicating the target namespace for an object being created\end{specHead} 
    \item[{Module}]
  tagdocs — \hyperref[TD]{Documentation Elements}
    \item[{Members}]
  \hyperref[TEI.elementSpec]{elementSpec} \hyperref[TEI.schemaSpec]{schemaSpec}
    \item[{Attributes}]
  Attributes\hfil\\[-10pt]\begin{sansreflist}
    \item[@ns]
  (namespace) specifies the namespace to which this element belongs
\begin{reflist}
    \item[{Status}]
  Optional
    \item[{Datatype}]
  \hyperref[TEI.teidata.namespace]{teidata.namespace}
    \item[{Default}]
  http://www.tei-c.org/ns/1.0
\end{reflist}  
\end{sansreflist}  
\end{reflist}  
\begin{reflist}
\item[]\begin{specHead}{TEI.att.naming}{att.naming} provides attributes common to elements which refer to named persons, places, organizations etc. [\textit{\hyperref[CONARS]{3.6.1.\ Referring Strings}} \textit{\hyperref[NDNYM]{13.3.6.\ Names and Nyms}}]\end{specHead} 
    \item[{Module}]
  tei — \hyperref[ST]{The TEI Infrastructure}
    \item[{Members}]
  \hyperref[TEI.att.personal]{att.personal}[\hyperref[TEI.addName]{addName} \hyperref[TEI.forename]{forename} \hyperref[TEI.genName]{genName} \hyperref[TEI.name]{name} \hyperref[TEI.objectName]{objectName} \hyperref[TEI.orgName]{orgName} \hyperref[TEI.persName]{persName} \hyperref[TEI.placeName]{placeName} \hyperref[TEI.roleName]{roleName} \hyperref[TEI.surname]{surname}] \hyperref[TEI.affiliation]{affiliation} \hyperref[TEI.author]{author} \hyperref[TEI.birth]{birth} \hyperref[TEI.bloc]{bloc} \hyperref[TEI.climate]{climate} \hyperref[TEI.collection]{collection} \hyperref[TEI.country]{country} \hyperref[TEI.death]{death} \hyperref[TEI.district]{district} \hyperref[TEI.editor]{editor} \hyperref[TEI.education]{education} \hyperref[TEI.event]{event} \hyperref[TEI.geogFeat]{geogFeat} \hyperref[TEI.geogName]{geogName} \hyperref[TEI.institution]{institution} \hyperref[TEI.nationality]{nationality} \hyperref[TEI.occupation]{occupation} \hyperref[TEI.offset]{offset} \hyperref[TEI.origPlace]{origPlace} \hyperref[TEI.population]{population} \hyperref[TEI.pubPlace]{pubPlace} \hyperref[TEI.region]{region} \hyperref[TEI.repository]{repository} \hyperref[TEI.residence]{residence} \hyperref[TEI.rs]{rs} \hyperref[TEI.settlement]{settlement} \hyperref[TEI.socecStatus]{socecStatus} \hyperref[TEI.state]{state} \hyperref[TEI.terrain]{terrain} \hyperref[TEI.trait]{trait}
    \item[{Attributes}]
  \hyperref[TEI.att.canonical]{att.canonical} (\textit{@key}, \textit{@ref}) \hfil\\[-10pt]\begin{sansreflist}
    \item[@role]
  may be used to specify further information about the entity referenced by this name in the form of a set of whitespace-separated values, for example the occupation of a person, or the status of a place.
\begin{reflist}
    \item[{Status}]
  Optional
    \item[{Datatype}]
  1–∞ occurrences of \hyperref[TEI.teidata.enumerated]{teidata.enumerated} separated by whitespace
\end{reflist}  
    \item[@nymRef]
  (reference to the canonical name) provides a means of locating the canonical form (\textit{nym}) of the names associated with the object named by the element bearing it.
\begin{reflist}
    \item[{Status}]
  Optional
    \item[{Datatype}]
  1–∞ occurrences of \hyperref[TEI.teidata.pointer]{teidata.pointer} separated by whitespace
    \item[{Note}]
  \par
The value must point directly to one or more XML elements by means of one or more URIs, separated by whitespace. If more than one is supplied, the implication is that the name is associated with several distinct canonical names.
\end{reflist}  
\end{sansreflist}  
\end{reflist}  
\begin{reflist}
\item[]\begin{specHead}{TEI.att.notated}{att.notated} provides an attribute to indicate any specialised notation used for element content.\end{specHead} 
    \item[{Module}]
  tei — \hyperref[ST]{The TEI Infrastructure}
    \item[{Members}]
  \hyperref[TEI.c]{c} \hyperref[TEI.cl]{cl} \hyperref[TEI.formula]{formula} \hyperref[TEI.hyph]{hyph} \hyperref[TEI.listAnnotation]{listAnnotation} \hyperref[TEI.m]{m} \hyperref[TEI.oRef]{oRef} \hyperref[TEI.orth]{orth} \hyperref[TEI.pRef]{pRef} \hyperref[TEI.phr]{phr} \hyperref[TEI.pron]{pron} \hyperref[TEI.quote]{quote} \hyperref[TEI.s]{s} \hyperref[TEI.seg]{seg} \hyperref[TEI.stress]{stress} \hyperref[TEI.syll]{syll} \hyperref[TEI.u]{u} \hyperref[TEI.w]{w}
    \item[{Attributes}]
  Attributes\hfil\\[-10pt]\begin{sansreflist}
    \item[@notation]
  names the notation used for the content of the element.
\begin{reflist}
    \item[{Status}]
  Optional
    \item[{Datatype}]
  \hyperref[TEI.teidata.enumerated]{teidata.enumerated}
\end{reflist}  
\end{sansreflist}  
\end{reflist}  
\begin{reflist}
\item[]\begin{specHead}{TEI.att.partials}{att.partials} provides attributes for describing the extent of lexical references for a dictionary term.\end{specHead} 
    \item[{Module}]
  tei — \hyperref[ST]{The TEI Infrastructure}
    \item[{Members}]
  \hyperref[TEI.orth]{orth} \hyperref[TEI.pron]{pron}
    \item[{Attributes}]
  Attributes\hfil\\[-10pt]\begin{sansreflist}
    \item[@extent]
  indicates whether the pronunciation or orthography applies to all or part of a word.
\begin{reflist}
    \item[{Status}]
  Optional
    \item[{Datatype}]
  \hyperref[TEI.teidata.enumerated]{teidata.enumerated}
    \item[{Suggested values include:}]
  \begin{description}

\item[{full}](full form)
\item[{pref}](prefix)
\item[{suff}](suffix)
\item[{inf}](infix)
\item[{part}](partial)
\end{description} 
    \item[{Note}]
  \par
This attribute is optional, and no default value is specified, so it can be omitted if this information is not necessary. 
\end{reflist}  
\end{sansreflist}  
\end{reflist}  
\begin{reflist}
\item[]\begin{specHead}{TEI.att.patternReplacement}{att.patternReplacement} provides attributes for regular-expression matching and replacement. [\textit{\hyperref[SAPU]{16.2.3.\ Using Abbreviated Pointers}} \textit{\hyperref[HD54M]{2.3.6.3.\ Milestone Method}} \textit{\hyperref[HD54]{2.3.6.\ The Reference System Declaration}} \textit{\hyperref[HD54S]{2.3.6.2.\ Search-and-Replace Method}}]\end{specHead} 
    \item[{Module}]
  header — \hyperref[HD]{The TEI Header}
    \item[{Members}]
  \hyperref[TEI.cRefPattern]{cRefPattern} \hyperref[TEI.prefixDef]{prefixDef}
    \item[{Attributes}]
  Attributes\hfil\\[-10pt]\begin{sansreflist}
    \item[@matchPattern]
  specifies a regular expression against which the values of other attributes can be matched.
\begin{reflist}
    \item[{Status}]
  Required
    \item[{Datatype}]
  \hyperref[TEI.teidata.pattern]{teidata.pattern}
    \item[{Note}]
  \par
The syntax used should follow that defined by \xref{http://www.w3.org/TR/xpath-functions/\#regex-syntax}{W3C XPath syntax}. Note that parenthesized groups are used not only for establishing order of precedence and atoms for quantification, but also for creating subpatterns to be referenced by the {\itshape replacementPattern} attribute.
\end{reflist}  
    \item[@replacementPattern]
  specifies a ‘replacement pattern’, that is, the skeleton of a relative or absolute URI containing references to groups in the {\itshape matchPattern} which, once subpattern substitution has been performed, complete the URI.
\begin{reflist}
    \item[{Status}]
  Required
    \item[{Datatype}]
  \hyperref[TEI.teidata.replacement]{teidata.replacement}
    \item[{Note}]
  \par
The strings \textit{\$1}, \textit{\$2} etc. are references to the corresponding group in the regular expression specified by {\itshape matchPattern} (counting open parenthesis, left to right). Processors are expected to replace them with whatever matched the corresponding group in the regular expression.\par
If a digit preceded by a dollar sign is needed in the actual replacement pattern (as opposed to being used as a back reference), the dollar sign must be written as \texttt{\%24}.
\end{reflist}  
\end{sansreflist}  
\end{reflist}  
\begin{reflist}
\item[]\begin{specHead}{TEI.att.personal}{att.personal} (attributes for components of names usually, but not necessarily, personal names) common attributes for those elements which form part of a name usually, but not necessarily, a personal name. [\textit{\hyperref[NDPER]{13.2.1.\ Personal Names}}]\end{specHead} 
    \item[{Module}]
  tei — \hyperref[ST]{The TEI Infrastructure}
    \item[{Members}]
  \hyperref[TEI.addName]{addName} \hyperref[TEI.forename]{forename} \hyperref[TEI.genName]{genName} \hyperref[TEI.name]{name} \hyperref[TEI.objectName]{objectName} \hyperref[TEI.orgName]{orgName} \hyperref[TEI.persName]{persName} \hyperref[TEI.placeName]{placeName} \hyperref[TEI.roleName]{roleName} \hyperref[TEI.surname]{surname}
    \item[{Attributes}]
  \hyperref[TEI.att.naming]{att.naming} (\textit{@role}, \textit{@nymRef})  (\hyperref[TEI.att.canonical]{att.canonical} (\textit{@key}, \textit{@ref})) \hfil\\[-10pt]\begin{sansreflist}
    \item[@full]
  indicates whether the name component is given in full, as an abbreviation or simply as an initial.
\begin{reflist}
    \item[{Status}]
  Optional
    \item[{Datatype}]
  \hyperref[TEI.teidata.enumerated]{teidata.enumerated}
    \item[{Legal values are:}]
  \begin{description}

\item[{yes}](yes) the name component is spelled out in full.{[Default] }
\item[{abb}](abbreviated) the name component is given in an abbreviated form.
\item[{init}](initial letter) the name component is indicated only by one initial.
\end{description} 
\end{reflist}  
    \item[@sort]
  (sort) specifies the sort order of the name component in relation to others within the name.
\begin{reflist}
    \item[{Status}]
  Optional
    \item[{Datatype}]
  \hyperref[TEI.teidata.count]{teidata.count}
\end{reflist}  
\end{sansreflist}  
\end{reflist}  
\begin{reflist}
\item[]\begin{specHead}{TEI.att.placement}{att.placement} provides attributes for describing where on the source page or object a textual element appears. [\textit{\hyperref[COEDADD]{3.5.3.\ Additions, Deletions, and Omissions}} \textit{\hyperref[PHAD]{11.3.1.4.\ Additions and Deletions}}]\end{specHead} 
    \item[{Module}]
  tei — \hyperref[ST]{The TEI Infrastructure}
    \item[{Members}]
  \hyperref[TEI.add]{add} \hyperref[TEI.addSpan]{addSpan} \hyperref[TEI.figure]{figure} \hyperref[TEI.fw]{fw} \hyperref[TEI.head]{head} \hyperref[TEI.label]{label} \hyperref[TEI.metamark]{metamark} \hyperref[TEI.notatedMusic]{notatedMusic} \hyperref[TEI.note]{note} \hyperref[TEI.noteGrp]{noteGrp} \hyperref[TEI.rt]{rt} \hyperref[TEI.stage]{stage} \hyperref[TEI.trailer]{trailer} \hyperref[TEI.witDetail]{witDetail}
    \item[{Attributes}]
  Attributes\hfil\\[-10pt]\begin{sansreflist}
    \item[@place]
  specifies where this item is placed.
\begin{reflist}
    \item[{Status}]
  Recommended
    \item[{Datatype}]
  1–∞ occurrences of \hyperref[TEI.teidata.enumerated]{teidata.enumerated} separated by whitespace
    \item[{Suggested values include:}]
  \begin{description}

\item[{top}]at the top of the page
\item[{bottom}]at the foot of the page
\item[{margin}]in the margin (left, right, or both)
\item[{opposite}]on the opposite, i.e. facing, page
\item[{overleaf}]on the other side of the leaf
\item[{above}]above the line
\item[{right}]to the right, e.g. to the right of a vertical line of text, or to the right of a figure
\item[{below}]below the line
\item[{left}]to the left, e.g. to the left of a vertical line of text, or to the left of a figure
\item[{end}]at the end of e.g. chapter or volume.
\item[{inline}]within the body of the text.
\item[{inspace}]in a predefined space, for example left by an earlier scribe.
\end{description} 
    \item[]\index{add=<add>|exampleindex}\index{place=@place!<add>|exampleindex}\index{add=<add>|exampleindex}\index{place=@place!<add>|exampleindex}\exampleFont {<\textbf{add}\hspace*{1em}{place}="{margin}">}[An addition written in the margin]{</\textbf{add}>}\mbox{}\newline 
{<\textbf{add}\hspace*{1em}{place}="{bottom opposite}">}[An addition written at the\mbox{}\newline 
 foot of the current page and also on the facing page]{</\textbf{add}>}
    \item[]\index{note=<note>|exampleindex}\index{place=@place!<note>|exampleindex}\exampleFont {<\textbf{note}\hspace*{1em}{place}="{bottom}">}Ibid, p.7{</\textbf{note}>}
\end{reflist}  
\end{sansreflist}  
\end{reflist}  
\begin{reflist}
\item[]\begin{specHead}{TEI.att.pointing}{att.pointing} provides a set of attributes used by all elements which point to other elements by means of one or more URI references. [\textit{\hyperref[STGAla]{1.3.1.1.2.\ Language Indicators}} \textit{\hyperref[COXR]{3.7.\ Simple Links and Cross-References}}]\end{specHead} 
    \item[{Module}]
  tei — \hyperref[ST]{The TEI Infrastructure}
    \item[{Members}]
  \hyperref[TEI.att.pointing.group]{att.pointing.group}[\hyperref[TEI.altGrp]{altGrp} \hyperref[TEI.joinGrp]{joinGrp} \hyperref[TEI.linkGrp]{linkGrp}] \hyperref[TEI.alt]{alt} \hyperref[TEI.annotation]{annotation} \hyperref[TEI.calendar]{calendar} \hyperref[TEI.catRef]{catRef} \hyperref[TEI.citedRange]{citedRange} \hyperref[TEI.gloss]{gloss} \hyperref[TEI.join]{join} \hyperref[TEI.licence]{licence} \hyperref[TEI.link]{link} \hyperref[TEI.locus]{locus} \hyperref[TEI.note]{note} \hyperref[TEI.noteGrp]{noteGrp} \hyperref[TEI.oRef]{oRef} \hyperref[TEI.pRef]{pRef} \hyperref[TEI.ptr]{ptr} \hyperref[TEI.ref]{ref} \hyperref[TEI.span]{span} \hyperref[TEI.substJoin]{substJoin} \hyperref[TEI.term]{term} \hyperref[TEI.witDetail]{witDetail}
    \item[{Attributes}]
  Attributes\hfil\\[-10pt]\begin{sansreflist}
    \item[@targetLang]
  specifies the language of the content to be found at the destination referenced by {\itshape target}, using a ‘language tag’ generated according to \xref{http://www.rfc-editor.org/rfc/bcp/bcp47.txt}{BCP 47}.
\begin{reflist}
    \item[{Status}]
  Optional
    \item[{Datatype}]
  \hyperref[TEI.teidata.language]{teidata.language}
    \item[{Schematron}]
   <sch:rule context="tei:*[not(self::tei:schemaSpec)][@targetLang]"> <sch:assert test="@target">@targetLang should only be used on <sch:name/> if @target is specified.</sch:assert> </sch:rule>
    \item[]\index{linkGrp=<linkGrp>|exampleindex}\index{ptr=<ptr>|exampleindex}\index{target=@target!<ptr>|exampleindex}\index{type=@type!<ptr>|exampleindex}\index{targetLang=@targetLang!<ptr>|exampleindex}\index{ptr=<ptr>|exampleindex}\index{target=@target!<ptr>|exampleindex}\index{type=@type!<ptr>|exampleindex}\index{targetLang=@targetLang!<ptr>|exampleindex}\exampleFont {<\textbf{linkGrp}\hspace*{1em}{xml:id}="{pol-swh\textunderscore aln\textunderscore 2.1-linkGrp}">}\mbox{}\newline 
\hspace*{1em}{<\textbf{ptr}\hspace*{1em}{xml:id}="{pol-swh\textunderscore aln\textunderscore 2.1.1-ptr}"\mbox{}\newline 
\hspace*{1em}\hspace*{1em}{target}="{pol/UDHR/text.xml\#pol\textunderscore txt\textunderscore 1-head}"\mbox{}\newline 
\hspace*{1em}\hspace*{1em}{type}="{tuv}"\mbox{}\newline 
\hspace*{1em}\hspace*{1em}{targetLang}="{pl}"/>}\mbox{}\newline 
\hspace*{1em}{<\textbf{ptr}\hspace*{1em}{xml:id}="{pol-swh\textunderscore aln\textunderscore 2.1.2-ptr}"\mbox{}\newline 
\hspace*{1em}\hspace*{1em}{target}="{swh/UDHR/text.xml\#swh\textunderscore txt\textunderscore 1-head}"\mbox{}\newline 
\hspace*{1em}\hspace*{1em}{type}="{tuv}"\mbox{}\newline 
\hspace*{1em}\hspace*{1em}{targetLang}="{sw}"/>}\mbox{}\newline 
{</\textbf{linkGrp}>}In the example above, the \hyperref[TEI.linkGrp]{<linkGrp>} combines pointers at parallel fragments of the \textit{Universal Declaration of Human Rights}: one of them is in Polish, the other in Swahili.
    \item[{Note}]
  \par
The value must conform to BCP 47. If the value is a private use code (i.e., starts with x- or contains -x-), a \hyperref[TEI.language]{<language>} element with a matching value for its {\itshape ident} attribute should be supplied in the TEI header to document this value. Such documentation may also optionally be supplied for non-private-use codes, though these must remain consistent with their  ( {\abbr IETF}) {\expan Internet Engineering Task Force} definitions.
\end{reflist}  
    \item[@target]
  specifies the destination of the reference by supplying one or more URI References
\begin{reflist}
    \item[{Status}]
  Optional
    \item[{Datatype}]
  1–∞ occurrences of \hyperref[TEI.teidata.pointer]{teidata.pointer} separated by whitespace
    \item[{Note}]
  \par
One or more syntactically valid URI references, separated by whitespace. Because whitespace is used to separate URIs, no whitespace is permitted inside a single URI. If a whitespace character is required in a URI, it should be escaped with the normal mechanism, e.g. \texttt{TEI\%20Consortium}.
\end{reflist}  
    \item[@evaluate]
  (evaluate) specifies the intended meaning when the target of a pointer is itself a pointer.
\begin{reflist}
    \item[{Status}]
  Optional
    \item[{Datatype}]
  \hyperref[TEI.teidata.enumerated]{teidata.enumerated}
    \item[{Legal values are:}]
  \begin{description}

\item[{all}]if the element pointed to is itself a pointer, then the target of that pointer will be taken, and so on, until an element is found which is not a pointer.
\item[{one}]if the element pointed to is itself a pointer, then its target (whether a pointer or not) is taken as the target of this pointer.
\item[{none}]no further evaluation of targets is carried out beyond that needed to find the element specified in the pointer's target.
\end{description} 
    \item[{Note}]
  \par
If no value is given, the application program is responsible for deciding (possibly on the basis of user input) how far to trace a chain of pointers.
\end{reflist}  
\end{sansreflist}  
\end{reflist}  
\begin{reflist}
\item[]\begin{specHead}{TEI.att.pointing.group}{att.pointing.group} provides a set of attributes common to all elements which enclose groups of pointer elements. [\textit{\hyperref[SA]{16.\ Linking, Segmentation, and Alignment}}]\end{specHead} 
    \item[{Module}]
  tei — \hyperref[ST]{The TEI Infrastructure}
    \item[{Members}]
  \hyperref[TEI.altGrp]{altGrp} \hyperref[TEI.joinGrp]{joinGrp} \hyperref[TEI.linkGrp]{linkGrp}
    \item[{Attributes}]
  \hyperref[TEI.att.pointing]{att.pointing} (\textit{@targetLang}, \textit{@target}, \textit{@evaluate}) \hyperref[TEI.att.typed]{att.typed} (\textit{@type}, \textit{@subtype}) \hfil\\[-10pt]\begin{sansreflist}
    \item[@domains]
  optionally specifies the identifiers of the elements within which all elements indicated by the contents of this element lie.
\begin{reflist}
    \item[{Status}]
  Optional
    \item[{Datatype}]
  2–∞ occurrences of \hyperref[TEI.teidata.pointer]{teidata.pointer} separated by whitespace
    \item[{Note}]
  \par
If this attribute is supplied every element specified as a target must be contained within the element or elements named by it. An application may choose whether or not to report failures to satisfy this constraint as errors, but may not access an element of the right identifier but in the wrong context. If this attribute is not supplied, then target elements may appear anywhere within the target document.
\end{reflist}  
    \item[@targFunc]
  (target function) describes the function of each of the values of the {\itshape target} attribute of the enclosed \hyperref[TEI.link]{<link>}, \hyperref[TEI.join]{<join>}, or \hyperref[TEI.alt]{<alt>} tags.
\begin{reflist}
    \item[{Status}]
  Optional
    \item[{Datatype}]
  2–∞ occurrences of \hyperref[TEI.teidata.word]{teidata.word} separated by whitespace
    \item[{Note}]
  \par
The number of separate values must match the number of values in the {\itshape target} attribute in the enclosed \hyperref[TEI.link]{<link>}, \hyperref[TEI.join]{<join>}, or \hyperref[TEI.alt]{<alt>} tags (an intermediate \hyperref[TEI.ptr]{<ptr>} element may be needed to accomplish this). It should also match the number of values in the {\itshape domains} attribute, of the current element, if one has been specified.
\end{reflist}  
\end{sansreflist}  
\end{reflist}  
\begin{reflist}
\item[]\begin{specHead}{TEI.att.predicate}{att.predicate} provides attributes for filtering by an XPath predicate expression. [\textit{\hyperref[TDcrystalsCE]{22.4.\ Common Elements}} \textit{\hyperref[TDPMPM]{22.5.4.1.\ The TEI processing model}} \textit{\hyperref[TDPMIP]{22.5.4.9.\ Implementation of Processing Models}}]\end{specHead} 
    \item[{Module}]
  tagdocs — \hyperref[TD]{Documentation Elements}
    \item[{Members}]
  \hyperref[TEI.equiv]{equiv} \hyperref[TEI.model]{model}
    \item[{Attributes}]
  Attributes\hfil\\[-10pt]\begin{sansreflist}
    \item[@predicate]
  the condition under which the element bearing this attribute applies, given as an XPath predicate expression.
\begin{reflist}
    \item[{Status}]
  Optional
    \item[{Datatype}]
  \hyperref[TEI.teidata.xpath]{teidata.xpath}
    \item[{Note}]
  \par
The XPath predicate expression given as the value of the {\itshape predicate} attribute has to be provided \textit{without} wrapping square brackets.
\end{reflist}  
\end{sansreflist}  
    \item[{Example}]
  \leavevmode\bgroup\index{model=<model>|exampleindex}\index{predicate=@predicate!<model>|exampleindex}\index{behaviour=@behaviour!<model>|exampleindex}\index{desc=<desc>|exampleindex}\index{versionDate=@versionDate!<desc>|exampleindex}\exampleFont \begin{shaded}\noindent\mbox{}{<\textbf{model}\hspace*{1em}{predicate}="{parent::person}"\mbox{}\newline 
\hspace*{1em}{behaviour}="{inline}">}\mbox{}\newline 
\hspace*{1em}{<\textbf{desc}\hspace*{1em}{versionDate}="{2015-08-21}"\mbox{}\newline 
\hspace*{1em}\hspace*{1em}{xml:lang}="{en}">}If it is a child of a person element, treat as\mbox{}\newline 
\hspace*{1em}\hspace*{1em} inline{</\textbf{desc}>}\mbox{}\newline 
{</\textbf{model}>}\end{shaded}\egroup 


    \item[{Example}]
  The following example declares that the \hyperref[TEI.name]{<name>} element can be mapped to, or is equivalent to, the external concepts of ‘PERSON’ and ‘PLACE’ depending on the ‘XPath’ expression given in {\itshape predicate}\leavevmode\bgroup\index{elementSpec=<elementSpec>|exampleindex}\index{ident=@ident!<elementSpec>|exampleindex}\index{mode=@mode!<elementSpec>|exampleindex}\index{equiv=<equiv>|exampleindex}\index{name=@name!<equiv>|exampleindex}\index{predicate=@predicate!<equiv>|exampleindex}\index{uri=@uri!<equiv>|exampleindex}\index{equiv=<equiv>|exampleindex}\index{name=@name!<equiv>|exampleindex}\index{predicate=@predicate!<equiv>|exampleindex}\index{uri=@uri!<equiv>|exampleindex}\exampleFont \begin{shaded}\noindent\mbox{}{<\textbf{elementSpec}\hspace*{1em}{ident}="{name}"\hspace*{1em}{mode}="{change}">}\mbox{}\newline 
\hspace*{1em}{<\textbf{equiv}\hspace*{1em}{name}="{PERSON}"\mbox{}\newline 
\hspace*{1em}\hspace*{1em}{predicate}="{@type eq 'person'}"\mbox{}\newline 
\hspace*{1em}\hspace*{1em}{uri}="{http://www.example.com/entities/person}"/>}\mbox{}\newline 
\hspace*{1em}{<\textbf{equiv}\hspace*{1em}{name}="{PLACE}"\mbox{}\newline 
\hspace*{1em}\hspace*{1em}{predicate}="{@type eq 'place'}"\mbox{}\newline 
\hspace*{1em}\hspace*{1em}{uri}="{http://www.example.com/entities/place}"/>}\mbox{}\newline 
{</\textbf{elementSpec}>}\end{shaded}\egroup 


\end{reflist}  
\begin{reflist}
\item[]\begin{specHead}{TEI.att.ranging}{att.ranging} provides attributes for describing numerical ranges.\end{specHead} 
    \item[{Module}]
  tei — \hyperref[ST]{The TEI Infrastructure}
    \item[{Members}]
  \hyperref[TEI.att.dimensions]{att.dimensions}[\hyperref[TEI.att.damaged]{att.damaged}[\hyperref[TEI.damage]{damage} \hyperref[TEI.damageSpan]{damageSpan}] \hyperref[TEI.add]{add} \hyperref[TEI.addSpan]{addSpan} \hyperref[TEI.age]{age} \hyperref[TEI.birth]{birth} \hyperref[TEI.date]{date} \hyperref[TEI.death]{death} \hyperref[TEI.del]{del} \hyperref[TEI.delSpan]{delSpan} \hyperref[TEI.depth]{depth} \hyperref[TEI.dim]{dim} \hyperref[TEI.dimensions]{dimensions} \hyperref[TEI.ex]{ex} \hyperref[TEI.floruit]{floruit} \hyperref[TEI.gap]{gap} \hyperref[TEI.geogFeat]{geogFeat} \hyperref[TEI.height]{height} \hyperref[TEI.mod]{mod} \hyperref[TEI.offset]{offset} \hyperref[TEI.origDate]{origDate} \hyperref[TEI.population]{population} \hyperref[TEI.redo]{redo} \hyperref[TEI.restore]{restore} \hyperref[TEI.retrace]{retrace} \hyperref[TEI.secl]{secl} \hyperref[TEI.space]{space} \hyperref[TEI.state]{state} \hyperref[TEI.subst]{subst} \hyperref[TEI.substJoin]{substJoin} \hyperref[TEI.supplied]{supplied} \hyperref[TEI.surplus]{surplus} \hyperref[TEI.time]{time} \hyperref[TEI.trait]{trait} \hyperref[TEI.unclear]{unclear} \hyperref[TEI.undo]{undo} \hyperref[TEI.width]{width}] \hyperref[TEI.num]{num} \hyperref[TEI.precision]{precision}
    \item[{Attributes}]
  Attributes\hfil\\[-10pt]\begin{sansreflist}
    \item[@atLeast]
  gives a minimum estimated value for the approximate measurement.
\begin{reflist}
    \item[{Status}]
  Optional
    \item[{Datatype}]
  \hyperref[TEI.teidata.numeric]{teidata.numeric}
\end{reflist}  
    \item[@atMost]
  gives a maximum estimated value for the approximate measurement.
\begin{reflist}
    \item[{Status}]
  Optional
    \item[{Datatype}]
  \hyperref[TEI.teidata.numeric]{teidata.numeric}
\end{reflist}  
    \item[@min]
  where the measurement summarizes more than one observation or a range, supplies the minimum value observed.
\begin{reflist}
    \item[{Status}]
  Optional
    \item[{Datatype}]
  \hyperref[TEI.teidata.numeric]{teidata.numeric}
\end{reflist}  
    \item[@max]
  where the measurement summarizes more than one observation or a range, supplies the maximum value observed.
\begin{reflist}
    \item[{Status}]
  Optional
    \item[{Datatype}]
  \hyperref[TEI.teidata.numeric]{teidata.numeric}
\end{reflist}  
    \item[@confidence]
  specifies the degree of statistical confidence (between zero and one) that a value falls within the range specified by {\itshape min} and {\itshape max}, or the proportion of observed values that fall within that range.
\begin{reflist}
    \item[{Status}]
  Optional
    \item[{Datatype}]
  \hyperref[TEI.teidata.probability]{teidata.probability}
\end{reflist}  
\end{sansreflist}  
    \item[{Example}]
  \leavevmode\bgroup\index{del=<del>|exampleindex}\index{rend=@rend!<del>|exampleindex}\index{gap=<gap>|exampleindex}\index{reason=@reason!<gap>|exampleindex}\index{extent=@extent!<gap>|exampleindex}\index{atLeast=@atLeast!<gap>|exampleindex}\index{atMost=@atMost!<gap>|exampleindex}\index{unit=@unit!<gap>|exampleindex}\exampleFont \begin{shaded}\noindent\mbox{}The MS. was lost in transmission by mail from {<\textbf{del}\hspace*{1em}{rend}="{overstrike}">}\mbox{}\newline 
\hspace*{1em}{<\textbf{gap}\hspace*{1em}{reason}="{illegible}"\mbox{}\newline 
\hspace*{1em}\hspace*{1em}{extent}="{one or two letters}"\hspace*{1em}{atLeast}="{1}"\hspace*{1em}{atMost}="{2}"\hspace*{1em}{unit}="{chars}"/>}\mbox{}\newline 
{</\textbf{del}>} Philadelphia to the Graphic office, New York.\mbox{}\newline 
\end{shaded}\egroup 


\end{reflist}  
\begin{reflist}
\item[]\begin{specHead}{TEI.att.rdgPart}{att.rdgPart} provides attributes to mark the beginning or ending of a fragmentary manuscript or other witness. [\textit{\hyperref[TCAPMI]{12.1.5.\ Fragmentary Witnesses}}]\end{specHead} 
    \item[{Module}]
  textcrit — \hyperref[TC]{Critical Apparatus}
    \item[{Members}]
  \hyperref[TEI.lacunaEnd]{lacunaEnd} \hyperref[TEI.lacunaStart]{lacunaStart} \hyperref[TEI.wit]{wit} \hyperref[TEI.witEnd]{witEnd} \hyperref[TEI.witStart]{witStart}
    \item[{Attributes}]
  Attributes\hfil\\[-10pt]\begin{sansreflist}
    \item[@wit]
  (witness or witnesses) contains a space-delimited list of one or more sigla indicating the witnesses to this reading beginning or ending at this point.
\begin{reflist}
    \item[{Status}]
  Optional
    \item[{Datatype}]
  1–∞ occurrences of \hyperref[TEI.teidata.pointer]{teidata.pointer} separated by whitespace
\end{reflist}  
\end{sansreflist}  
    \item[{Note}]
  \par
These elements may appear anywhere within the elements \hyperref[TEI.lem]{<lem>} and \hyperref[TEI.rdg]{<rdg>}, and also within any of their constituent elements.
\end{reflist}  
\begin{reflist}
\item[]\begin{specHead}{TEI.att.repeatable}{att.repeatable} supplies attributes for the elements which define component parts of a content model.\end{specHead} 
    \item[{Module}]
  tagdocs — \hyperref[TD]{Documentation Elements}
    \item[{Members}]
  \hyperref[TEI.alternate]{alternate} \hyperref[TEI.anyElement]{anyElement} \hyperref[TEI.classRef]{classRef} \hyperref[TEI.datatype]{datatype} \hyperref[TEI.elementRef]{elementRef} \hyperref[TEI.sequence]{sequence}
    \item[{Attributes}]
  Attributes\hfil\\[-10pt]\begin{sansreflist}
    \item[@minOccurs]
  (minimum number of occurences) indicates the smallest number of times this component may occur.
\begin{reflist}
    \item[{Status}]
  Optional
    \item[{Datatype}]
  \hyperref[TEI.teidata.count]{teidata.count}
    \item[{Default}]
  1
\end{reflist}  
    \item[@maxOccurs]
  (maximum number of occurences) indicates the largest number of times this component may occur.
\begin{reflist}
    \item[{Status}]
  Optional
    \item[{Datatype}]
  \hyperref[TEI.teidata.unboundedInt]{teidata.unboundedInt}
    \item[{Default}]
  1
\end{reflist}  
\end{sansreflist}  
    \item[{Schematron}]
   <sch:rule context="*[ @minOccurs and @maxOccurs ]"> <sch:let name="min"  value="@minOccurs cast as xs:integer"/> <sch:let name="max"  value="if ( normalize-space( @maxOccurs ) eq 'unbounded') then -1 else @maxOccurs   cast as xs:integer"/> <sch:assert test="\$max eq -1 or \$max ge \$min">@maxOccurs should be greater than or equal to @minOccurs</sch:assert> </sch:rule> <sch:rule context="*[ @minOccurs and not( @maxOccurs ) ]"> <sch:assert test="@minOccurs cast as xs:integer lt 2">When @maxOccurs is not specified, @minOccurs must be 0 or 1</sch:assert> </sch:rule>
    \item[{Note}]
  \par
The value of {\itshape minOccurs} must always be less than or equal to that of {\itshape maxOccurs}. Since the default value of {\itshape maxOccurs} is 1, when {\itshape maxOccurs} is not specified {\itshape minOccurs} must always be less than or equal to 1. The default value of {\itshape minOccurs} is also 1, and therefore, when {\itshape minOccurs} is not specified the value of {\itshape maxOccurs} must always be greater than or equal to 1. An ODD processor should raise an error if either of these conditions is not met.
\end{reflist}  
\begin{reflist}
\item[]\begin{specHead}{TEI.att.resourced}{att.resourced} provides attributes by which a resource (such as an externally held media file) may be located.\end{specHead} 
    \item[{Module}]
  tei — \hyperref[ST]{The TEI Infrastructure}
    \item[{Members}]
  \hyperref[TEI.graphic]{graphic} \hyperref[TEI.media]{media} \hyperref[TEI.schemaRef]{schemaRef}
    \item[{Attributes}]
  Attributes\hfil\\[-10pt]\begin{sansreflist}
    \item[@url]
  (uniform resource locator) specifies the URL from which the media concerned may be obtained.
\begin{reflist}
    \item[{Status}]
  Required
    \item[{Datatype}]
  \hyperref[TEI.teidata.pointer]{teidata.pointer}
\end{reflist}  
\end{sansreflist}  
\end{reflist}  
\begin{reflist}
\item[]\begin{specHead}{TEI.att.scoping}{att.scoping} provides attributes for selecting particular elements within a document.\end{specHead} 
    \item[{Module}]
  tei — \hyperref[ST]{The TEI Infrastructure}
    \item[{Members}]
  \hyperref[TEI.certainty]{certainty} \hyperref[TEI.precision]{precision} \hyperref[TEI.respons]{respons}
    \item[{Attributes}]
  Attributes\hfil\\[-10pt]\begin{sansreflist}
    \item[@target]
  points at one or more sets of zero or more elements each.
\begin{reflist}
    \item[{Status}]
  Optional
    \item[{Datatype}]
  1–∞ occurrences of \hyperref[TEI.teidata.pointer]{teidata.pointer} separated by whitespace
    \item[]\index{persName=<persName>|exampleindex}\index{certainty=<certainty>|exampleindex}\index{target=@target!<certainty>|exampleindex}\index{locus=@locus!<certainty>|exampleindex}\index{degree=@degree!<certainty>|exampleindex}\exampleFont Elizabeth went to {<\textbf{persName}\hspace*{1em}{xml:id}="{ESSEX}">}Essex{</\textbf{persName}>}\mbox{}\newline 
{<\textbf{certainty}\hspace*{1em}{target}="{\#ESSEX}"\mbox{}\newline 
\hspace*{1em}{locus}="{name}"\mbox{}\newline 
\hspace*{1em}{degree}="{0.6}"/>}
\end{reflist}  
    \item[@match]
  supplies an XPath selection pattern using the syntax defined in \cite{XSLT3} which identifies a set of nodes, selected within the context identified by the {\itshape target} attribute if this is supplied, or within the context of the parent element if it is not.
\begin{reflist}
    \item[{Status}]
  Optional
    \item[{Datatype}]
  \hyperref[TEI.teidata.xpath]{teidata.xpath}
    \item[]\index{gap=<gap>|exampleindex}\index{reason=@reason!<gap>|exampleindex}\index{certainty=<certainty>|exampleindex}\index{match=@match!<certainty>|exampleindex}\index{locus=@locus!<certainty>|exampleindex}\index{cert=@cert!<certainty>|exampleindex}\exampleFont {<\textbf{gap}\hspace*{1em}{reason}="{cancelled}">}\mbox{}\newline 
\hspace*{1em}{<\textbf{certainty}\hspace*{1em}{match}="{@reason}"\mbox{}\newline 
\hspace*{1em}\hspace*{1em}{locus}="{value}"\mbox{}\newline 
\hspace*{1em}\hspace*{1em}{cert}="{low}"/>}\mbox{}\newline 
{</\textbf{gap}>}
\end{reflist}  
\end{sansreflist}  
    \item[{Note}]
  \par
The semantics of this element apply to the nodeset identified by the value of the {\itshape target} attribute, possibly modified by the value of the {\itshape match} attribute. If more than one identifier is given, the implication is that all elements (or nodesets) are intended. The {\itshape match} attribute may also be used as a means of identifying groups of elements.\par
If {\itshape target} and {\itshape match} are present, {\itshape target} selects an element and the XPath expression in {\itshape match} is evaluated in the context of that element. If neither attribute is present, the expression applies to its parent element. If only {\itshape target} is given, the expression refers to the selected element or nodeset. If only {\itshape match} is given, the XPath expression is evaluated in the context of the parent element of the bearing element.\par
Note that the value of the {\itshape target} attribute may include an XPointer expression including an XPath expression (see \textit{\hyperref[SATS]{16.2.4.\ TEI XPointer Schemes}}).
\end{reflist}  
\begin{reflist}
\item[]\begin{specHead}{TEI.att.segLike}{att.segLike} provides attributes for elements used for arbitrary segmentation. [\textit{\hyperref[SASE]{16.3.\ Blocks, Segments, and Anchors}} \textit{\hyperref[AILC]{17.1.\ Linguistic Segment Categories}}]\end{specHead} 
    \item[{Module}]
  tei — \hyperref[ST]{The TEI Infrastructure}
    \item[{Members}]
  \hyperref[TEI.c]{c} \hyperref[TEI.cl]{cl} \hyperref[TEI.m]{m} \hyperref[TEI.pc]{pc} \hyperref[TEI.phr]{phr} \hyperref[TEI.s]{s} \hyperref[TEI.seg]{seg} \hyperref[TEI.w]{w}
    \item[{Attributes}]
  \hyperref[TEI.att.metrical]{att.metrical} (\textit{@met}, \textit{@real}, \textit{@rhyme}) \hyperref[TEI.att.datcat]{att.datcat} (\textit{@datcat}, \textit{@valueDatcat}) \hyperref[TEI.att.fragmentable]{att.fragmentable} (\textit{@part}) \hfil\\[-10pt]\begin{sansreflist}
    \item[@function]
  (function) characterizes the function of the segment.
\begin{reflist}
    \item[{Status}]
  Optional
    \item[{Datatype}]
  \hyperref[TEI.teidata.enumerated]{teidata.enumerated}
    \item[{Note}]
  \par
Attribute values will often vary depending on the type of element to which they are attached. For example, a \hyperref[TEI.cl]{<cl>}, may take values such as coordinate, subject, adverbial etc. For a \hyperref[TEI.phr]{<phr>}, such values as subject, predicate etc. may be more appropriate. Such constraints will typically be implemented by a project-defined customization.
\end{reflist}  
\end{sansreflist}  
\end{reflist}  
\begin{reflist}
\item[]\begin{specHead}{TEI.att.sortable}{att.sortable} provides attributes for elements in lists or groups that are sortable, but whose sorting key cannot be derived mechanically from the element content. [\textit{\hyperref[DIBO]{9.1.\ Dictionary Body and Overall Structure}}]\end{specHead} 
    \item[{Module}]
  tei — \hyperref[ST]{The TEI Infrastructure}
    \item[{Members}]
  \hyperref[TEI.bibl]{bibl} \hyperref[TEI.biblFull]{biblFull} \hyperref[TEI.biblStruct]{biblStruct} \hyperref[TEI.correspAction]{correspAction} \hyperref[TEI.entry]{entry} \hyperref[TEI.entryFree]{entryFree} \hyperref[TEI.event]{event} \hyperref[TEI.idno]{idno} \hyperref[TEI.item]{item} \hyperref[TEI.list]{list} \hyperref[TEI.listApp]{listApp} \hyperref[TEI.listBibl]{listBibl} \hyperref[TEI.listChange]{listChange} \hyperref[TEI.listEvent]{listEvent} \hyperref[TEI.listNym]{listNym} \hyperref[TEI.listObject]{listObject} \hyperref[TEI.listOrg]{listOrg} \hyperref[TEI.listPerson]{listPerson} \hyperref[TEI.listPlace]{listPlace} \hyperref[TEI.listRelation]{listRelation} \hyperref[TEI.listWit]{listWit} \hyperref[TEI.msDesc]{msDesc} \hyperref[TEI.nym]{nym} \hyperref[TEI.object]{object} \hyperref[TEI.org]{org} \hyperref[TEI.person]{person} \hyperref[TEI.personGrp]{personGrp} \hyperref[TEI.persona]{persona} \hyperref[TEI.place]{place} \hyperref[TEI.relation]{relation} \hyperref[TEI.superEntry]{superEntry} \hyperref[TEI.term]{term} \hyperref[TEI.witness]{witness}
    \item[{Attributes}]
  Attributes\hfil\\[-10pt]\begin{sansreflist}
    \item[@sortKey]
  supplies the sort key for this element in an index, list or group which contains it.
\begin{reflist}
    \item[{Status}]
  Optional
    \item[{Datatype}]
  \hyperref[TEI.teidata.word]{teidata.word}
    \item[]\index{index=<index>|exampleindex}\index{indexName=@indexName!<index>|exampleindex}\index{term=<term>|exampleindex}\index{sortKey=@sortKey!<term>|exampleindex}\exampleFont David's other principal backer, Josiah\mbox{}\newline 
 ha-Kohen {<\textbf{index}\hspace*{1em}{indexName}="{NAMES}">}\mbox{}\newline 
\hspace*{1em}{<\textbf{term}\hspace*{1em}{sortKey}="{Azarya\textunderscore Josiah\textunderscore Kohen}">}Josiah ha-Kohen b. Azarya{</\textbf{term}>}\mbox{}\newline 
{</\textbf{index}>} b. Azarya, son of one of the last gaons of Sura was David's own first\mbox{}\newline 
 cousin.
    \item[{Note}]
  \par
The sort key is used to determine the sequence and grouping of entries in an index. It provides a sequence of characters which, when sorted with the other values, will produced the desired order; specifics of sort key construction are application-dependent\par
Dictionary order often differs from the collation sequence of machine-readable character sets; in English-language dictionaries, an entry for \textit{4-H} will often appear alphabetized under ‘fourh’, and \textit{McCoy} may be alphabetized under ‘maccoy’, while \textit{A1}, \textit{A4}, and \textit{A5} may all appear in numeric order ‘alphabetized’ between ‘a-’ and ‘AA’. The sort key is required if the orthography of the dictionary entry does not suffice to determine its location.
\end{reflist}  
\end{sansreflist}  
\end{reflist}  
\begin{reflist}
\item[]\begin{specHead}{TEI.att.spanning}{att.spanning} provides attributes for elements which delimit a span of text by pointing mechanisms rather than by enclosing it. [\textit{\hyperref[PHAD]{11.3.1.4.\ Additions and Deletions}} \textit{\hyperref[STECAT]{1.3.1.\ Attribute Classes}}]\end{specHead} 
    \item[{Module}]
  tei — \hyperref[ST]{The TEI Infrastructure}
    \item[{Members}]
  \hyperref[TEI.addSpan]{addSpan} \hyperref[TEI.cb]{cb} \hyperref[TEI.damageSpan]{damageSpan} \hyperref[TEI.delSpan]{delSpan} \hyperref[TEI.gb]{gb} \hyperref[TEI.index]{index} \hyperref[TEI.lb]{lb} \hyperref[TEI.metamark]{metamark} \hyperref[TEI.milestone]{milestone} \hyperref[TEI.mod]{mod} \hyperref[TEI.pb]{pb} \hyperref[TEI.redo]{redo} \hyperref[TEI.retrace]{retrace} \hyperref[TEI.undo]{undo}
    \item[{Attributes}]
  Attributes\hfil\\[-10pt]\begin{sansreflist}
    \item[@spanTo]
  indicates the end of a span initiated by the element bearing this attribute.
\begin{reflist}
    \item[{Status}]
  Optional
    \item[{Datatype}]
  \hyperref[TEI.teidata.pointer]{teidata.pointer}
    \item[{Schematron}]
  The @spanTo attribute must point to an element following the current element <sch:rule context="tei:*[@spanTo]"> <sch:assert test="id(substring(@spanTo,2)) and following::*[@xml:id=substring(current()/@spanTo,2)]">The element indicated by @spanTo (<sch:value-of select="@spanTo"/>) must follow the current element <sch:name/> </sch:assert> </sch:rule>
\end{reflist}  
\end{sansreflist}  
    \item[{Note}]
  \par
The span is defined as running in document order from the start of the content of the pointing element to the end of the content of the element pointed to by the {\itshape spanTo} attribute (if any). If no value is supplied for the attribute, the assumption is that the span is coextensive with the pointing element. If no content is present, the assumption is that the starting point of the span is immediately following the element itself.
\end{reflist}  
\begin{reflist}
\item[]\begin{specHead}{TEI.att.styleDef}{att.styleDef} provides attributes to specify the name of a formal definition language used to provide formatting or rendition information.\end{specHead} 
    \item[{Module}]
  tei — \hyperref[ST]{The TEI Infrastructure}
    \item[{Members}]
  \hyperref[TEI.rendition]{rendition} \hyperref[TEI.styleDefDecl]{styleDefDecl}
    \item[{Attributes}]
  Attributes\hfil\\[-10pt]\begin{sansreflist}
    \item[@scheme]
  identifies the language used to describe the rendition.
\begin{reflist}
    \item[{Status}]
  Optional
    \item[{Datatype}]
  \hyperref[TEI.teidata.enumerated]{teidata.enumerated}
    \item[{Legal values are:}]
  \begin{description}

\item[{css}]Cascading Stylesheet Language
\item[{xslfo}]Extensible Stylesheet Language Formatting Objects
\item[{free}]Informal free text description
\item[{other}]A user-defined rendition description language
\end{description} 
    \item[{Note}]
  \par
If no value for the @scheme attribute is provided, then the default assumption should be that CSS is in use. 
\end{reflist}  
    \item[@schemeVersion]
  supplies a version number for the style language provided in {\itshape scheme}.
\begin{reflist}
    \item[{Status}]
  Optional
    \item[{Datatype}]
  \hyperref[TEI.teidata.versionNumber]{teidata.versionNumber}
    \item[{Schematron}]
   <sch:rule context="tei:*[@schemeVersion]"> <sch:assert test="@scheme and not(@scheme = 'free')"> @schemeVersion can only be used if @scheme is specified. </sch:assert> </sch:rule>
    \item[{Note}]
  \par
If {\itshape schemeVersion} is used, then {\itshape scheme} should also appear, with a value other than free.
\end{reflist}  
\end{sansreflist}  
\end{reflist}  
\begin{reflist}
\item[]\begin{specHead}{TEI.att.tableDecoration}{att.tableDecoration} provides attributes used to decorate rows or cells of a table. [\textit{\hyperref[FT]{14.\ Tables, Formulæ, Graphics and Notated Music}}]\end{specHead} 
    \item[{Module}]
  figures — \hyperref[FT]{Tables, Formulæ, Graphics and Notated Music}
    \item[{Members}]
  \hyperref[TEI.cell]{cell} \hyperref[TEI.row]{row}
    \item[{Attributes}]
  Attributes\hfil\\[-10pt]\begin{sansreflist}
    \item[@role]
  (role) indicates the kind of information held in this cell or in each cell of this row.
\begin{reflist}
    \item[{Status}]
  Optional
    \item[{Datatype}]
  \hyperref[TEI.teidata.enumerated]{teidata.enumerated}
    \item[{Suggested values include:}]
  \begin{description}

\item[{label}]labelling or descriptive information only.
\item[{data}]data values.{[Default] }
\end{description} 
    \item[{Note}]
  \par
When this attribute is specified on a row, its value is the default for all cells in this row. When specified on a cell, its value overrides any default specified by the {\itshape role} attribute of the parent \hyperref[TEI.row]{<row>} element.
\end{reflist}  
    \item[@rows]
  (rows) indicates the number of rows occupied by this cell or row.
\begin{reflist}
    \item[{Status}]
  Optional
    \item[{Datatype}]
  \hyperref[TEI.teidata.count]{teidata.count}
    \item[{Default}]
  1
    \item[{Note}]
  \par
A value greater than one indicates that this cell  spans several rows. Where several cells span multiple rows, it may be more convenient to use nested tables.
\end{reflist}  
    \item[@cols]
  (columns) indicates the number of columns occupied by this cell or row.
\begin{reflist}
    \item[{Status}]
  Optional
    \item[{Datatype}]
  \hyperref[TEI.teidata.count]{teidata.count}
    \item[{Default}]
  1
    \item[{Note}]
  \par
A value greater than one indicates that this cell or row spans several columns. Where an initial cell spans an entire row, it may be better treated as a heading.
\end{reflist}  
\end{sansreflist}  
\end{reflist}  
\begin{reflist}
\item[]\begin{specHead}{TEI.att.textCritical}{att.textCritical} defines a set of attributes common to all elements representing variant readings in text critical work. [\textit{\hyperref[TCAPLL]{12.1.\ The Apparatus Entry, Readings, and Witnesses}}]\end{specHead} 
    \item[{Module}]
  textcrit — \hyperref[TC]{Critical Apparatus}
    \item[{Members}]
  \hyperref[TEI.lem]{lem} \hyperref[TEI.rdg]{rdg} \hyperref[TEI.rdgGrp]{rdgGrp}
    \item[{Attributes}]
  \hyperref[TEI.att.written]{att.written} (\textit{@hand}) \hyperref[TEI.att.typed]{att.typed} (\unusedattribute{type}, @subtype) \hfil\\[-10pt]\begin{sansreflist}
    \item[@type]
  classifies the reading according to some useful typology.
\begin{reflist}
    \item[{Status}]
  Optional
    \item[{Datatype}]
  \hyperref[TEI.teidata.enumerated]{teidata.enumerated}
    \item[{Sample values include:}]
  \begin{description}

\item[{substantive}](substantive) the reading offers a substantive variant.
\item[{orthographic}](orthographic) the reading differs only orthographically, not in substance, from other readings.
\end{description} 
\end{reflist}  
    \item[@cause]
  classifies the cause for the variant reading, according to any appropriate typology of possible origins.
\begin{reflist}
    \item[{Status}]
  Optional
    \item[{Datatype}]
  \hyperref[TEI.teidata.enumerated]{teidata.enumerated}
    \item[{Sample values include:}]
  \begin{description}

\item[{homeoteleuton}]
\item[{homeoarchy}]
\item[{paleographicConfusion}]
\item[{haplography}]
\item[{dittography}]
\item[{falseEmendation}]
\end{description} 
\end{reflist}  
    \item[@varSeq]
  (variant sequence) provides a number indicating the position of this reading in a sequence, when there is reason to presume a sequence to the variants. 
\begin{reflist}
    \item[{Status}]
  Optional
    \item[{Datatype}]
  \hyperref[TEI.teidata.count]{teidata.count}
    \item[{Note}]
  \par
Different variant sequences could be coded with distinct number trails: 1-2-3 for one sequence, 5-6-7 for another. More complex variant sequences, with (for example) multiple branchings from single readings, may be expressed through the \hyperref[TEI.join]{<join>} element.
\end{reflist}  
    \item[@require]
  points to other readings that are required when adopting the current reading or lemma.
\begin{reflist}
    \item[{Status}]
  Optional
    \item[{Datatype}]
  1–∞ occurrences of \hyperref[TEI.teidata.pointer]{teidata.pointer} separated by whitespace
\end{reflist}  
\end{sansreflist}  
    \item[{Note}]
  \par
This element class defines attributes inherited by \hyperref[TEI.rdg]{<rdg>}, \hyperref[TEI.lem]{<lem>}, and \hyperref[TEI.rdgGrp]{<rdgGrp>}.
\end{reflist}  
\begin{reflist}
\item[]\begin{specHead}{TEI.att.timed}{att.timed} provides attributes common to those elements which have a duration in time, expressed either absolutely or by reference to an alignment map. [\textit{\hyperref[TSBATI]{8.3.5.\ Temporal Information}}]\end{specHead} 
    \item[{Module}]
  tei — \hyperref[ST]{The TEI Infrastructure}
    \item[{Members}]
  \hyperref[TEI.annotationBlock]{annotationBlock} \hyperref[TEI.binaryObject]{binaryObject} \hyperref[TEI.gap]{gap} \hyperref[TEI.incident]{incident} \hyperref[TEI.kinesic]{kinesic} \hyperref[TEI.media]{media} \hyperref[TEI.pause]{pause} \hyperref[TEI.u]{u} \hyperref[TEI.vocal]{vocal} \hyperref[TEI.writing]{writing}
    \item[{Attributes}]
  \hyperref[TEI.att.duration]{att.duration} (\hyperref[TEI.att.duration.w3c]{att.duration.w3c} (\textit{@dur})) (\hyperref[TEI.att.duration.iso]{att.duration.iso} (\textit{@dur-iso})) \hfil\\[-10pt]\begin{sansreflist}
    \item[@start]
  indicates the location within a temporal alignment at which this element begins.
\begin{reflist}
    \item[{Status}]
  Optional
    \item[{Datatype}]
  \hyperref[TEI.teidata.pointer]{teidata.pointer}
    \item[{Note}]
  \par
If no value is supplied, the element is assumed to follow the immediately preceding element at the same hierarchic level.
\end{reflist}  
    \item[@end]
  indicates the location within a temporal alignment at which this element ends.
\begin{reflist}
    \item[{Status}]
  Optional
    \item[{Datatype}]
  \hyperref[TEI.teidata.pointer]{teidata.pointer}
    \item[{Note}]
  \par
If no value is supplied, the element is assumed to precede the immediately following element at the same hierarchic level.
\end{reflist}  
\end{sansreflist}  
\end{reflist}  
\begin{reflist}
\item[]\begin{specHead}{TEI.att.transcriptional}{att.transcriptional} provides attributes specific to elements encoding authorial or scribal intervention in a text when transcribing manuscript or similar sources. [\textit{\hyperref[PHAD]{11.3.1.4.\ Additions and Deletions}}]\end{specHead} 
    \item[{Module}]
  tei — \hyperref[ST]{The TEI Infrastructure}
    \item[{Members}]
  \hyperref[TEI.add]{add} \hyperref[TEI.addSpan]{addSpan} \hyperref[TEI.del]{del} \hyperref[TEI.delSpan]{delSpan} \hyperref[TEI.mod]{mod} \hyperref[TEI.redo]{redo} \hyperref[TEI.restore]{restore} \hyperref[TEI.retrace]{retrace} \hyperref[TEI.rt]{rt} \hyperref[TEI.subst]{subst} \hyperref[TEI.substJoin]{substJoin} \hyperref[TEI.undo]{undo}
    \item[{Attributes}]
  \hyperref[TEI.att.editLike]{att.editLike} (\textit{@evidence}, \textit{@instant}) \hyperref[TEI.att.written]{att.written} (\textit{@hand}) \hfil\\[-10pt]\begin{sansreflist}
    \item[@status]
  indicates the effect of the intervention, for example in the case of a deletion, strikeouts which include too much or too little text, or in the case of an addition, an insertion which duplicates some of the text already present.
\begin{reflist}
    \item[{Status}]
  Optional
    \item[{Datatype}]
  \hyperref[TEI.teidata.enumerated]{teidata.enumerated}
    \item[{Sample values include:}]
  \begin{description}

\item[{duplicate}]all of the text indicated as an addition duplicates some text that is in the original, whether the duplication is word-for-word or less exact.
\item[{duplicate-partial}]part of the text indicated as an addition duplicates some text that is in the original
\item[{excessStart}]some text at the beginning of the deletion is marked as deleted even though it clearly should not be deleted.
\item[{excessEnd}]some text at the end of the deletion is marked as deleted even though it clearly should not be deleted.
\item[{shortStart}]some text at the beginning of the deletion is not marked as deleted even though it clearly should be.
\item[{shortEnd}]some text at the end of the deletion is not marked as deleted even though it clearly should be.
\item[{partial}]some text in the deletion is not marked as deleted even though it clearly should be.
\item[{unremarkable}]the deletion is not faulty.{[Default] }
\end{description} 
    \item[{Note}]
  \par
Status information on each deletion is needed rather rarely except in critical editions from authorial manuscripts; status information on additions is even less common.\par
Marking a deletion or addition as faulty is inescapably an interpretive act; the usual test applied in practice is the linguistic acceptability of the text with and without the letters or words in question.
\end{reflist}  
    \item[@cause]
  documents the presumed cause for the intervention.
\begin{reflist}
    \item[{Status}]
  Optional
    \item[{Datatype}]
  \hyperref[TEI.teidata.enumerated]{teidata.enumerated}
\end{reflist}  
    \item[@seq]
  (sequence) assigns a sequence number related to the order in which the encoded features carrying this attribute are believed to have occurred.
\begin{reflist}
    \item[{Status}]
  Optional
    \item[{Datatype}]
  \hyperref[TEI.teidata.count]{teidata.count}
\end{reflist}  
\end{sansreflist}  
\end{reflist}  
\begin{reflist}
\item[]\begin{specHead}{TEI.att.translatable}{att.translatable} provides attributes used to indicate the status of a translatable portion of an ODD document.\end{specHead} 
    \item[{Module}]
  tagdocs — \hyperref[TD]{Documentation Elements}
    \item[{Members}]
  \hyperref[TEI.desc]{desc} \hyperref[TEI.exemplum]{exemplum} \hyperref[TEI.gloss]{gloss} \hyperref[TEI.remarks]{remarks} \hyperref[TEI.valDesc]{valDesc}
    \item[{Attributes}]
  Attributes\hfil\\[-10pt]\begin{sansreflist}
    \item[@versionDate]
  specifies the date on which the source text was extracted and sent to the translator
\begin{reflist}
    \item[{Status}]
  Optional
    \item[{Datatype}]
  \hyperref[TEI.teidata.temporal.working]{teidata.temporal.working}
    \item[{Note}]
  \par
The {\itshape versionDate} attribute can be used to determine whether a translation might need to be revisited, by comparing the modification date on the containing file with the {\itshape versionDate} value on the translation. If the file has changed, changelogs can be checked to see whether the source text has been modified since the translation was made.
\end{reflist}  
\end{sansreflist}  
\end{reflist}  
\begin{reflist}
\item[]\begin{specHead}{TEI.att.typed}{att.typed} provides attributes which can be used to classify or subclassify elements in any way. [\textit{\hyperref[STECAT]{1.3.1.\ Attribute Classes}} \textit{\hyperref[AILCW]{17.1.1.\ Words and Above}} \textit{\hyperref[CONARS]{3.6.1.\ Referring Strings}} \textit{\hyperref[COXR]{3.7.\ Simple Links and Cross-References}} \textit{\hyperref[CONAAB]{3.6.5.\ Abbreviations and Their Expansions}} \textit{\hyperref[COVE]{3.13.1.\ Core Tags for Verse}} \textit{\hyperref[DRPAL]{7.2.5.\ Speech Contents}} \textit{\hyperref[DSDIV1]{4.1.1.\ Un-numbered Divisions}} \textit{\hyperref[DSDIV2]{4.1.2.\ Numbered Divisions}} \textit{\hyperref[DSHD]{4.2.1.\ Headings and Trailers}} \textit{\hyperref[DSVIRT]{4.4.\ Virtual Divisions}} \textit{\hyperref[NDPERSREL]{13.3.2.3.\ Personal Relationships}} \textit{\hyperref[PHCO]{11.3.1.1.\ Core Elements for Transcriptional Work}} \textit{\hyperref[SAPTL]{16.1.1.\ Pointers and Links}} \textit{\hyperref[SASE]{16.3.\ Blocks, Segments, and Anchors}} \textit{\hyperref[TCAPLK]{12.2.\ Linking the Apparatus to the Text}} \textit{\hyperref[TDTAGCONT]{22.5.1.2.\ Defining Content Models: RELAX NG}} \textit{\hyperref[TSBA]{8.3.\ Elements Unique to Spoken Texts}} \textit{\hyperref[MDMDAL]{23.3.1.3.\ Modification of Attribute and Attribute Value Lists}}]\end{specHead} 
    \item[{Module}]
  tei — \hyperref[ST]{The TEI Infrastructure}
    \item[{Members}]
  att.interpLike[\hyperref[TEI.interp]{interp} \hyperref[TEI.interpGrp]{interpGrp} \hyperref[TEI.span]{span} \hyperref[TEI.spanGrp]{spanGrp}] \hyperref[TEI.att.pointing.group]{att.pointing.group}[\hyperref[TEI.altGrp]{altGrp} \hyperref[TEI.joinGrp]{joinGrp} \hyperref[TEI.linkGrp]{linkGrp}] \hyperref[TEI.TEI]{TEI} \hyperref[TEI.ab]{ab} \hyperref[TEI.abbr]{abbr} \hyperref[TEI.accMat]{accMat} \hyperref[TEI.add]{add} \hyperref[TEI.addName]{addName} \hyperref[TEI.addSpan]{addSpan} \hyperref[TEI.affiliation]{affiliation} \hyperref[TEI.age]{age} \hyperref[TEI.alt]{alt} \hyperref[TEI.altIdent]{altIdent} \hyperref[TEI.altIdentifier]{altIdentifier} \hyperref[TEI.am]{am} \hyperref[TEI.anchor]{anchor} \hyperref[TEI.app]{app} \hyperref[TEI.application]{application} \hyperref[TEI.bibl]{bibl} \hyperref[TEI.biblStruct]{biblStruct} \hyperref[TEI.binaryObject]{binaryObject} \hyperref[TEI.birth]{birth} \hyperref[TEI.bloc]{bloc} \hyperref[TEI.c]{c} \hyperref[TEI.camera]{camera} \hyperref[TEI.castItem]{castItem} \hyperref[TEI.cb]{cb} \hyperref[TEI.certainty]{certainty} \hyperref[TEI.change]{change} \hyperref[TEI.charProp]{charProp} \hyperref[TEI.cit]{cit} \hyperref[TEI.cl]{cl} \hyperref[TEI.classSpec]{classSpec} \hyperref[TEI.climate]{climate} \hyperref[TEI.collection]{collection} \hyperref[TEI.colloc]{colloc} \hyperref[TEI.constitution]{constitution} \hyperref[TEI.constraintSpec]{constraintSpec} \hyperref[TEI.corr]{corr} \hyperref[TEI.correspAction]{correspAction} \hyperref[TEI.correspDesc]{correspDesc} \hyperref[TEI.country]{country} \hyperref[TEI.custEvent]{custEvent} \hyperref[TEI.damage]{damage} \hyperref[TEI.damageSpan]{damageSpan} \hyperref[TEI.date]{date} \hyperref[TEI.death]{death} \hyperref[TEI.decoNote]{decoNote} \hyperref[TEI.del]{del} \hyperref[TEI.delSpan]{delSpan} \hyperref[TEI.derivation]{derivation} \hyperref[TEI.desc]{desc} \hyperref[TEI.dim]{dim} \hyperref[TEI.dimensions]{dimensions} \hyperref[TEI.distinct]{distinct} \hyperref[TEI.district]{district} \hyperref[TEI.div]{div} \hyperref[TEI.div1]{div1} \hyperref[TEI.div2]{div2} \hyperref[TEI.div3]{div3} \hyperref[TEI.div4]{div4} \hyperref[TEI.div5]{div5} \hyperref[TEI.div6]{div6} \hyperref[TEI.div7]{div7} \hyperref[TEI.divGen]{divGen} \hyperref[TEI.domain]{domain} \hyperref[TEI.eLeaf]{eLeaf} \hyperref[TEI.eTree]{eTree} \hyperref[TEI.education]{education} \hyperref[TEI.etym]{etym} \hyperref[TEI.event]{event} \hyperref[TEI.exemplum]{exemplum} \hyperref[TEI.explicit]{explicit} \hyperref[TEI.factuality]{factuality} \hyperref[TEI.faith]{faith} \hyperref[TEI.figure]{figure} \hyperref[TEI.filiation]{filiation} \hyperref[TEI.finalRubric]{finalRubric} \hyperref[TEI.floatingText]{floatingText} \hyperref[TEI.forename]{forename} \hyperref[TEI.forest]{forest} \hyperref[TEI.form]{form} \hyperref[TEI.fw]{fw} \hyperref[TEI.g]{g} \hyperref[TEI.gb]{gb} \hyperref[TEI.genName]{genName} \hyperref[TEI.geogFeat]{geogFeat} \hyperref[TEI.geogName]{geogName} \hyperref[TEI.gloss]{gloss} \hyperref[TEI.gram]{gram} \hyperref[TEI.gramGrp]{gramGrp} \hyperref[TEI.graph]{graph} \hyperref[TEI.group]{group} \hyperref[TEI.head]{head} \hyperref[TEI.iType]{iType} \hyperref[TEI.ident]{ident} \hyperref[TEI.idno]{idno} \hyperref[TEI.incident]{incident} \hyperref[TEI.incipit]{incipit} \hyperref[TEI.interaction]{interaction} \hyperref[TEI.join]{join} \hyperref[TEI.kinesic]{kinesic} \hyperref[TEI.label]{label} \hyperref[TEI.langKnowledge]{langKnowledge} \hyperref[TEI.lb]{lb} \hyperref[TEI.lbl]{lbl} \hyperref[TEI.lg]{lg} \hyperref[TEI.line]{line} \hyperref[TEI.link]{link} \hyperref[TEI.list]{list} \hyperref[TEI.listAnnotation]{listAnnotation} \hyperref[TEI.listApp]{listApp} \hyperref[TEI.listBibl]{listBibl} \hyperref[TEI.listChange]{listChange} \hyperref[TEI.listEvent]{listEvent} \hyperref[TEI.listForest]{listForest} \hyperref[TEI.listNym]{listNym} \hyperref[TEI.listObject]{listObject} \hyperref[TEI.listOrg]{listOrg} \hyperref[TEI.listPerson]{listPerson} \hyperref[TEI.listPlace]{listPlace} \hyperref[TEI.listRef]{listRef} \hyperref[TEI.listRelation]{listRelation} \hyperref[TEI.location]{location} \hyperref[TEI.locus]{locus} \hyperref[TEI.m]{m} \hyperref[TEI.mapping]{mapping} \hyperref[TEI.material]{material} \hyperref[TEI.measure]{measure} \hyperref[TEI.measureGrp]{measureGrp} \hyperref[TEI.media]{media} \hyperref[TEI.milestone]{milestone} \hyperref[TEI.mod]{mod} \hyperref[TEI.moduleSpec]{moduleSpec} \hyperref[TEI.move]{move} \hyperref[TEI.msDesc]{msDesc} \hyperref[TEI.msFrag]{msFrag} \hyperref[TEI.msName]{msName} \hyperref[TEI.msPart]{msPart} \hyperref[TEI.name]{name} \hyperref[TEI.nameLink]{nameLink} \hyperref[TEI.nationality]{nationality} \hyperref[TEI.node]{node} \hyperref[TEI.notatedMusic]{notatedMusic} \hyperref[TEI.note]{note} \hyperref[TEI.noteGrp]{noteGrp} \hyperref[TEI.num]{num} \hyperref[TEI.nym]{nym} \hyperref[TEI.oRef]{oRef} \hyperref[TEI.object]{object} \hyperref[TEI.objectName]{objectName} \hyperref[TEI.occupation]{occupation} \hyperref[TEI.offset]{offset} \hyperref[TEI.org]{org} \hyperref[TEI.orgName]{orgName} \hyperref[TEI.origDate]{origDate} \hyperref[TEI.origPlace]{origPlace} \hyperref[TEI.orth]{orth} \hyperref[TEI.path]{path} \hyperref[TEI.pause]{pause} \hyperref[TEI.pb]{pb} \hyperref[TEI.pc]{pc} \hyperref[TEI.persName]{persName} \hyperref[TEI.persPronouns]{persPronouns} \hyperref[TEI.phr]{phr} \hyperref[TEI.place]{place} \hyperref[TEI.placeName]{placeName} \hyperref[TEI.population]{population} \hyperref[TEI.preparedness]{preparedness} \hyperref[TEI.pron]{pron} \hyperref[TEI.provenance]{provenance} \hyperref[TEI.ptr]{ptr} \hyperref[TEI.purpose]{purpose} \hyperref[TEI.quote]{quote} \hyperref[TEI.rb]{rb} \hyperref[TEI.re]{re} \hyperref[TEI.recording]{recording} \hyperref[TEI.ref]{ref} \hyperref[TEI.reg]{reg} \hyperref[TEI.region]{region} \hyperref[TEI.relatedItem]{relatedItem} \hyperref[TEI.relation]{relation} \hyperref[TEI.residence]{residence} \hyperref[TEI.restore]{restore} \hyperref[TEI.rhyme]{rhyme} \hyperref[TEI.roleName]{roleName} \hyperref[TEI.rs]{rs} \hyperref[TEI.rt]{rt} \hyperref[TEI.rubric]{rubric} \hyperref[TEI.ruby]{ruby} \hyperref[TEI.s]{s} \hyperref[TEI.schemaRef]{schemaRef} \hyperref[TEI.seal]{seal} \hyperref[TEI.seg]{seg} \hyperref[TEI.settlement]{settlement} \hyperref[TEI.sex]{sex} \hyperref[TEI.socecStatus]{socecStatus} \hyperref[TEI.sound]{sound} \hyperref[TEI.spGrp]{spGrp} \hyperref[TEI.space]{space} \hyperref[TEI.stamp]{stamp} \hyperref[TEI.standOff]{standOff} \hyperref[TEI.state]{state} \hyperref[TEI.surface]{surface} \hyperref[TEI.surfaceGrp]{surfaceGrp} \hyperref[TEI.surname]{surname} \hyperref[TEI.table]{table} \hyperref[TEI.tag]{tag} \hyperref[TEI.tech]{tech} \hyperref[TEI.teiCorpus]{teiCorpus} \hyperref[TEI.term]{term} \hyperref[TEI.terrain]{terrain} \hyperref[TEI.text]{text} \hyperref[TEI.time]{time} \hyperref[TEI.title]{title} \hyperref[TEI.titlePage]{titlePage} \hyperref[TEI.titlePart]{titlePart} \hyperref[TEI.trailer]{trailer} \hyperref[TEI.trait]{trait} \hyperref[TEI.unit]{unit} \hyperref[TEI.unitDef]{unitDef} \hyperref[TEI.usg]{usg} \hyperref[TEI.vocal]{vocal} \hyperref[TEI.w]{w} \hyperref[TEI.witDetail]{witDetail} \hyperref[TEI.writing]{writing} \hyperref[TEI.xenoData]{xenoData} \hyperref[TEI.xr]{xr} \hyperref[TEI.zone]{zone}
    \item[{Attributes}]
  Attributes\hfil\\[-10pt]\begin{sansreflist}
    \item[@type]
  characterizes the element in some sense, using any convenient classification scheme or typology.
\begin{reflist}
    \item[{Status}]
  Optional
    \item[{Datatype}]
  \hyperref[TEI.teidata.enumerated]{teidata.enumerated}
    \item[]\index{div=<div>|exampleindex}\index{type=@type!<div>|exampleindex}\index{head=<head>|exampleindex}\index{lg=<lg>|exampleindex}\index{type=@type!<lg>|exampleindex}\index{l=<l>|exampleindex}\index{l=<l>|exampleindex}\index{lg=<lg>|exampleindex}\index{type=@type!<lg>|exampleindex}\index{l=<l>|exampleindex}\index{l=<l>|exampleindex}\exampleFont {<\textbf{div}\hspace*{1em}{type}="{verse}">}\mbox{}\newline 
\hspace*{1em}{<\textbf{head}>}Night in Tarras{</\textbf{head}>}\mbox{}\newline 
\hspace*{1em}{<\textbf{lg}\hspace*{1em}{type}="{stanza}">}\mbox{}\newline 
\hspace*{1em}\hspace*{1em}{<\textbf{l}>}At evening tramping on the hot white road{</\textbf{l}>}\mbox{}\newline 
\hspace*{1em}\hspace*{1em}{<\textbf{l}>}…{</\textbf{l}>}\mbox{}\newline 
\hspace*{1em}{</\textbf{lg}>}\mbox{}\newline 
\hspace*{1em}{<\textbf{lg}\hspace*{1em}{type}="{stanza}">}\mbox{}\newline 
\hspace*{1em}\hspace*{1em}{<\textbf{l}>}A wind sprang up from nowhere as the sky{</\textbf{l}>}\mbox{}\newline 
\hspace*{1em}\hspace*{1em}{<\textbf{l}>}…{</\textbf{l}>}\mbox{}\newline 
\hspace*{1em}{</\textbf{lg}>}\mbox{}\newline 
{</\textbf{div}>}
    \item[{Note}]
  \par
The {\itshape type} attribute is present on a number of elements, not all of which are members of \textsf{att.typed}, usually because these elements restrict the possible values for the attribute in a specific way.
\end{reflist}  
    \item[@subtype]
  (subtype) provides a sub-categorization of the element, if needed
\begin{reflist}
    \item[{Status}]
  Optional
    \item[{Datatype}]
  \hyperref[TEI.teidata.enumerated]{teidata.enumerated}
    \item[{Note}]
  \par
The {\itshape subtype} attribute may be used to provide any sub-classification for the element additional to that provided by its {\itshape type} attribute.
\end{reflist}  
\end{sansreflist}  
    \item[{Schematron}]
   <sch:rule context="tei:*[@subtype]"> <sch:assert test="@type">The <sch:name/> element should not be categorized in detail with @subtype unless also categorized in general with @type</sch:assert> </sch:rule>
    \item[{Note}]
  \par
When appropriate, values from an established typology should be used. Alternatively a typology may be defined in the associated TEI header. If values are to be taken from a project-specific list, this should be defined using the \hyperref[TEI.valList]{<valList>} element in the project-specific schema description, as described in \textit{\hyperref[MDMDAL]{23.3.1.3.\ Modification of Attribute and Attribute Value Lists}} .
\end{reflist}  
\begin{reflist}
\item[]\begin{specHead}{TEI.att.witnessed}{att.witnessed} supplies the attribute used to identify the witnesses supporting a particular reading in a critical apparatus. [\textit{\hyperref[TCAPLL]{12.1.\ The Apparatus Entry, Readings, and Witnesses}}]\end{specHead} 
    \item[{Module}]
  textcrit — \hyperref[TC]{Critical Apparatus}
    \item[{Members}]
  \hyperref[TEI.lem]{lem} \hyperref[TEI.rdg]{rdg}
    \item[{Attributes}]
  Attributes\hfil\\[-10pt]\begin{sansreflist}
    \item[@wit]
  (witness or witnesses) contains a space-delimited list of one or more pointers indicating the witnesses which attest to a given reading.
\begin{reflist}
    \item[{Status}]
  Optional
    \item[{Datatype}]
  1–∞ occurrences of \hyperref[TEI.teidata.pointer]{teidata.pointer} separated by whitespace
    \item[{Note}]
  \par
If the apparatus contains readings only for a single witness, this attribute may be consistently omitted.\par
This attribute may occur both within an apparatus gathering variant readings in the transcription of an individual witness and within an apparatus gathering readings from different witnesses.\par
Additional descriptions or alternative versions of the sigla referenced may be supplied as the content of a child \hyperref[TEI.wit]{<wit>} element.
\end{reflist}  
\end{sansreflist}  
\end{reflist}  
\begin{reflist}
\item[]\begin{specHead}{TEI.att.written}{att.written} provides an attribute to indicate the hand in which the content of an element was written in the source being transcribed. [\textit{\hyperref[STECAT]{1.3.1.\ Attribute Classes}}]\end{specHead} 
    \item[{Module}]
  tei — \hyperref[ST]{The TEI Infrastructure}
    \item[{Members}]
  \hyperref[TEI.att.damaged]{att.damaged}[\hyperref[TEI.damage]{damage} \hyperref[TEI.damageSpan]{damageSpan}] \hyperref[TEI.att.textCritical]{att.textCritical}[\hyperref[TEI.lem]{lem} \hyperref[TEI.rdg]{rdg} \hyperref[TEI.rdgGrp]{rdgGrp}] \hyperref[TEI.att.transcriptional]{att.transcriptional}[\hyperref[TEI.add]{add} \hyperref[TEI.addSpan]{addSpan} \hyperref[TEI.del]{del} \hyperref[TEI.delSpan]{delSpan} \hyperref[TEI.mod]{mod} \hyperref[TEI.redo]{redo} \hyperref[TEI.restore]{restore} \hyperref[TEI.retrace]{retrace} \hyperref[TEI.rt]{rt} \hyperref[TEI.subst]{subst} \hyperref[TEI.substJoin]{substJoin} \hyperref[TEI.undo]{undo}] \hyperref[TEI.ab]{ab} \hyperref[TEI.closer]{closer} \hyperref[TEI.div]{div} \hyperref[TEI.figure]{figure} \hyperref[TEI.fw]{fw} \hyperref[TEI.head]{head} \hyperref[TEI.hi]{hi} \hyperref[TEI.label]{label} \hyperref[TEI.line]{line} \hyperref[TEI.note]{note} \hyperref[TEI.noteGrp]{noteGrp} \hyperref[TEI.opener]{opener} \hyperref[TEI.p]{p} \hyperref[TEI.path]{path} \hyperref[TEI.postscript]{postscript} \hyperref[TEI.salute]{salute} \hyperref[TEI.seg]{seg} \hyperref[TEI.signed]{signed} \hyperref[TEI.text]{text} \hyperref[TEI.trailer]{trailer} \hyperref[TEI.zone]{zone}
    \item[{Attributes}]
  Attributes\hfil\\[-10pt]\begin{sansreflist}
    \item[@hand]
  points to a \hyperref[TEI.handNote]{<handNote>} element describing the hand considered responsible for the content of the element concerned.
\begin{reflist}
    \item[{Status}]
  Optional
    \item[{Datatype}]
  \hyperref[TEI.teidata.pointer]{teidata.pointer}
\end{reflist}  
\end{sansreflist}  
\end{reflist}  