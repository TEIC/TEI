
\section[{Datatypes and Other Macros}]{Datatypes and Other Macros}\label{REF-MACROS}
\subsection[{About the Datatypes and Macros Appendix}]{About the Datatypes and Macros Appendix}\par
This appendix gives you a list of datypes and links to the reference pages for them. There are 34 distinctly-named data specifications in revision \xref{https://github.com/TEIC/TEI/commit/27522dec1}{27522dec1} of TEI P5 \hyperref[ABTEI4]{Version} \xref{../../readme-4.3.0.html}{4.3.0a} of the TEI Guidelines.
\begin{reflist}
\item[]\begin{specHead}{TEI.macro.limitedContent}{macro.limitedContent} (paragraph content) defines the content of prose elements that are not used for transcription of extant materials. [\textit{\hyperref[STEC]{1.3.\ The TEI Class System}}]\end{specHead} 
    \item[{Module}]
  tei — \hyperref[ST]{The TEI Infrastructure}
    \item[{Used by}]
  \hyperref[TEI.desc]{desc} \hyperref[TEI.fDescr]{fDescr} \hyperref[TEI.figDesc]{figDesc} \hyperref[TEI.fsDescr]{fsDescr} \hyperref[TEI.meeting]{meeting} \hyperref[TEI.rendition]{rendition} \hyperref[TEI.tagUsage]{tagUsage}
    \item[{Content model}]
  \mbox{}\hfill\\[-10pt]\begin{Verbatim}[fontsize=\small]
<content>
 <alternate minOccurs="0"
  maxOccurs="unbounded">
  <textNode/>
  <classRef key="model.limitedPhrase"/>
  <classRef key="model.inter"/>
 </alternate>
</content>
    
\end{Verbatim}

    \item[{Declaration}]
  \mbox{}\hfill\\[-10pt]\begin{Verbatim}[fontsize=\small]
macro.limitedContent = ( text | model.limitedPhrase | model.inter )*
\end{Verbatim}

\end{reflist}  
\begin{reflist}
\item[]\begin{specHead}{TEI.macro.paraContent}{macro.paraContent} (paragraph content) defines the content of paragraphs and similar elements. [\textit{\hyperref[STEC]{1.3.\ The TEI Class System}}]\end{specHead} 
    \item[{Module}]
  tei — \hyperref[ST]{The TEI Infrastructure}
    \item[{Used by}]
  \hyperref[TEI.ab]{ab} \hyperref[TEI.add]{add} \hyperref[TEI.camera]{camera} \hyperref[TEI.caption]{caption} \hyperref[TEI.case]{case} \hyperref[TEI.colloc]{colloc} \hyperref[TEI.corr]{corr} \hyperref[TEI.damage]{damage} \hyperref[TEI.def]{def} \hyperref[TEI.del]{del} \hyperref[TEI.docEdition]{docEdition} \hyperref[TEI.emph]{emph} \hyperref[TEI.gen]{gen} \hyperref[TEI.gram]{gram} \hyperref[TEI.hi]{hi} \hyperref[TEI.hyph]{hyph} \hyperref[TEI.iType]{iType} \hyperref[TEI.imprimatur]{imprimatur} \hyperref[TEI.lang]{lang} \hyperref[TEI.lbl]{lbl} \hyperref[TEI.mod]{mod} \hyperref[TEI.mood]{mood} \hyperref[TEI.number]{number} \hyperref[TEI.orig]{orig} \hyperref[TEI.orth]{orth} \hyperref[TEI.p]{p} \hyperref[TEI.per]{per} \hyperref[TEI.pos]{pos} \hyperref[TEI.pron]{pron} \hyperref[TEI.ref]{ref} \hyperref[TEI.reg]{reg} \hyperref[TEI.restore]{restore} \hyperref[TEI.retrace]{retrace} \hyperref[TEI.rhyme]{rhyme} \hyperref[TEI.salute]{salute} \hyperref[TEI.secl]{secl} \hyperref[TEI.seg]{seg} \hyperref[TEI.sic]{sic} \hyperref[TEI.signed]{signed} \hyperref[TEI.sound]{sound} \hyperref[TEI.stress]{stress} \hyperref[TEI.subc]{subc} \hyperref[TEI.supplied]{supplied} \hyperref[TEI.surplus]{surplus} \hyperref[TEI.syll]{syll} \hyperref[TEI.tech]{tech} \hyperref[TEI.title]{title} \hyperref[TEI.titlePart]{titlePart} \hyperref[TEI.tns]{tns} \hyperref[TEI.u]{u} \hyperref[TEI.unclear]{unclear} \hyperref[TEI.usg]{usg} \hyperref[TEI.writing]{writing}
    \item[{Content model}]
  \mbox{}\hfill\\[-10pt]\begin{Verbatim}[fontsize=\small]
<content>
 <alternate minOccurs="0"
  maxOccurs="unbounded">
  <textNode/>
  <classRef key="model.gLike"/>
  <classRef key="model.phrase"/>
  <classRef key="model.inter"/>
  <classRef key="model.global"/>
  <elementRef key="lg"/>
  <classRef key="model.lLike"/>
 </alternate>
</content>
    
\end{Verbatim}

    \item[{Declaration}]
  \mbox{}\hfill\\[-10pt]\begin{Verbatim}[fontsize=\small]
macro.paraContent =
   (
      text
    | model.gLike    | model.phrase    | model.inter    | model.global    | lg    | model.lLike   )*
\end{Verbatim}

\end{reflist}  
\begin{reflist}
\item[]\begin{specHead}{TEI.macro.phraseSeq}{macro.phraseSeq} (phrase sequence) defines a sequence of character data and phrase-level elements. [\textit{\hyperref[STECST]{1.4.1.\ Standard Content Models}}]\end{specHead} 
    \item[{Module}]
  tei — \hyperref[ST]{The TEI Infrastructure}
    \item[{Used by}]
  \hyperref[TEI.abbr]{abbr} \hyperref[TEI.actor]{actor} \hyperref[TEI.addName]{addName} \hyperref[TEI.addrLine]{addrLine} \hyperref[TEI.affiliation]{affiliation} \hyperref[TEI.author]{author} \hyperref[TEI.biblScope]{biblScope} \hyperref[TEI.birth]{birth} \hyperref[TEI.bloc]{bloc} \hyperref[TEI.catchwords]{catchwords} \hyperref[TEI.citedRange]{citedRange} \hyperref[TEI.cl]{cl} \hyperref[TEI.colophon]{colophon} \hyperref[TEI.country]{country} \hyperref[TEI.death]{death} \hyperref[TEI.distinct]{distinct} \hyperref[TEI.distributor]{distributor} \hyperref[TEI.district]{district} \hyperref[TEI.docAuthor]{docAuthor} \hyperref[TEI.docDate]{docDate} \hyperref[TEI.edition]{edition} \hyperref[TEI.editor]{editor} \hyperref[TEI.education]{education} \hyperref[TEI.eg]{eg} \hyperref[TEI.email]{email} \hyperref[TEI.expan]{expan} \hyperref[TEI.explicit]{explicit} \hyperref[TEI.extent]{extent} \hyperref[TEI.faith]{faith} \hyperref[TEI.finalRubric]{finalRubric} \hyperref[TEI.floruit]{floruit} \hyperref[TEI.foreign]{foreign} \hyperref[TEI.forename]{forename} \hyperref[TEI.fw]{fw} \hyperref[TEI.genName]{genName} \hyperref[TEI.geoDecl]{geoDecl} \hyperref[TEI.geogFeat]{geogFeat} \hyperref[TEI.geogName]{geogName} \hyperref[TEI.gloss]{gloss} \hyperref[TEI.headItem]{headItem} \hyperref[TEI.headLabel]{headLabel} \hyperref[TEI.heraldry]{heraldry} \hyperref[TEI.incipit]{incipit} \hyperref[TEI.label]{label} \hyperref[TEI.material]{material} \hyperref[TEI.measure]{measure} \hyperref[TEI.mentioned]{mentioned} \hyperref[TEI.name]{name} \hyperref[TEI.nameLink]{nameLink} \hyperref[TEI.nationality]{nationality} \hyperref[TEI.num]{num} \hyperref[TEI.objectName]{objectName} \hyperref[TEI.objectType]{objectType} \hyperref[TEI.offset]{offset} \hyperref[TEI.orgName]{orgName} \hyperref[TEI.origPlace]{origPlace} \hyperref[TEI.persName]{persName} \hyperref[TEI.persPronouns]{persPronouns} \hyperref[TEI.phr]{phr} \hyperref[TEI.placeName]{placeName} \hyperref[TEI.pubPlace]{pubPlace} \hyperref[TEI.publisher]{publisher} \hyperref[TEI.rb]{rb} \hyperref[TEI.region]{region} \hyperref[TEI.residence]{residence} \hyperref[TEI.role]{role} \hyperref[TEI.roleDesc]{roleDesc} \hyperref[TEI.roleName]{roleName} \hyperref[TEI.rs]{rs} \hyperref[TEI.rt]{rt} \hyperref[TEI.rubric]{rubric} \hyperref[TEI.s]{s} \hyperref[TEI.secFol]{secFol} \hyperref[TEI.settlement]{settlement} \hyperref[TEI.sex]{sex} \hyperref[TEI.soCalled]{soCalled} \hyperref[TEI.socecStatus]{socecStatus} \hyperref[TEI.speaker]{speaker} \hyperref[TEI.stamp]{stamp} \hyperref[TEI.street]{street} \hyperref[TEI.surname]{surname} \hyperref[TEI.term]{term} \hyperref[TEI.textLang]{textLang} \hyperref[TEI.unit]{unit} \hyperref[TEI.watermark]{watermark} \hyperref[TEI.wit]{wit}
    \item[{Content model}]
  \mbox{}\hfill\\[-10pt]\begin{Verbatim}[fontsize=\small]
<content>
 <alternate minOccurs="0"
  maxOccurs="unbounded">
  <textNode/>
  <classRef key="model.gLike"/>
  <classRef key="model.attributable"/>
  <classRef key="model.phrase"/>
  <classRef key="model.global"/>
 </alternate>
</content>
    
\end{Verbatim}

    \item[{Declaration}]
  \mbox{}\hfill\\[-10pt]\begin{Verbatim}[fontsize=\small]
macro.phraseSeq =
   ( text | model.gLike | model.attributable | model.phrase | model.global )*
\end{Verbatim}

\end{reflist}  
\begin{reflist}
\item[]\begin{specHead}{TEI.macro.phraseSeq.limited}{macro.phraseSeq.limited} (limited phrase sequence) defines a sequence of character data and those phrase-level elements that are not typically used for transcribing extant documents. [\textit{\hyperref[STECST]{1.4.1.\ Standard Content Models}}]\end{specHead} 
    \item[{Module}]
  tei — \hyperref[ST]{The TEI Infrastructure}
    \item[{Used by}]
  \hyperref[TEI.activity]{activity} \hyperref[TEI.age]{age} \hyperref[TEI.authority]{authority} \hyperref[TEI.channel]{channel} \hyperref[TEI.classCode]{classCode} \hyperref[TEI.collection]{collection} \hyperref[TEI.constitution]{constitution} \hyperref[TEI.derivation]{derivation} \hyperref[TEI.domain]{domain} \hyperref[TEI.factuality]{factuality} \hyperref[TEI.funder]{funder} \hyperref[TEI.institution]{institution} \hyperref[TEI.interaction]{interaction} \hyperref[TEI.langKnown]{langKnown} \hyperref[TEI.language]{language} \hyperref[TEI.locale]{locale} \hyperref[TEI.metSym]{metSym} \hyperref[TEI.preparedness]{preparedness} \hyperref[TEI.principal]{principal} \hyperref[TEI.purpose]{purpose} \hyperref[TEI.repository]{repository} \hyperref[TEI.resp]{resp} \hyperref[TEI.span]{span} \hyperref[TEI.sponsor]{sponsor} \hyperref[TEI.valDesc]{valDesc}
    \item[{Content model}]
  \mbox{}\hfill\\[-10pt]\begin{Verbatim}[fontsize=\small]
<content>
 <alternate minOccurs="0"
  maxOccurs="unbounded">
  <textNode/>
  <classRef key="model.limitedPhrase"/>
  <classRef key="model.global"/>
 </alternate>
</content>
    
\end{Verbatim}

    \item[{Declaration}]
  \mbox{}\hfill\\[-10pt]\begin{Verbatim}[fontsize=\small]
macro.phraseSeq.limited = ( text | model.limitedPhrase | model.global )*
\end{Verbatim}

\end{reflist}  
\begin{reflist}
\item[]\begin{specHead}{TEI.macro.specialPara}{macro.specialPara} ('special' paragraph content) defines the content model of elements such as notes or list items, which either contain a series of component-level elements or else have the same structure as a paragraph, containing a series of phrase-level and inter-level elements. [\textit{\hyperref[STEC]{1.3.\ The TEI Class System}}]\end{specHead} 
    \item[{Module}]
  tei — \hyperref[ST]{The TEI Infrastructure}
    \item[{Used by}]
  \hyperref[TEI.accMat]{accMat} \hyperref[TEI.acquisition]{acquisition} \hyperref[TEI.additions]{additions} \hyperref[TEI.cell]{cell} \hyperref[TEI.change]{change} \hyperref[TEI.collation]{collation} \hyperref[TEI.condition]{condition} \hyperref[TEI.custEvent]{custEvent} \hyperref[TEI.decoNote]{decoNote} \hyperref[TEI.filiation]{filiation} \hyperref[TEI.foliation]{foliation} \hyperref[TEI.handNote]{handNote} \hyperref[TEI.item]{item} \hyperref[TEI.layout]{layout} \hyperref[TEI.licence]{licence} \hyperref[TEI.metamark]{metamark} \hyperref[TEI.musicNotation]{musicNotation} \hyperref[TEI.note]{note} \hyperref[TEI.occupation]{occupation} \hyperref[TEI.origin]{origin} \hyperref[TEI.provenance]{provenance} \hyperref[TEI.q]{q} \hyperref[TEI.quote]{quote} \hyperref[TEI.said]{said} \hyperref[TEI.scriptNote]{scriptNote} \hyperref[TEI.signatures]{signatures} \hyperref[TEI.source]{source} \hyperref[TEI.stage]{stage} \hyperref[TEI.summary]{summary} \hyperref[TEI.support]{support} \hyperref[TEI.surrogates]{surrogates} \hyperref[TEI.typeNote]{typeNote} \hyperref[TEI.view]{view}
    \item[{Content model}]
  \mbox{}\hfill\\[-10pt]\begin{Verbatim}[fontsize=\small]
<content>
 <alternate minOccurs="0"
  maxOccurs="unbounded">
  <textNode/>
  <classRef key="model.gLike"/>
  <classRef key="model.phrase"/>
  <classRef key="model.inter"/>
  <classRef key="model.divPart"/>
  <classRef key="model.global"/>
 </alternate>
</content>
    
\end{Verbatim}

    \item[{Declaration}]
  \mbox{}\hfill\\[-10pt]\begin{Verbatim}[fontsize=\small]
macro.specialPara =
   (
      text
    | model.gLike    | model.phrase    | model.inter    | model.divPart    | model.global   )*
\end{Verbatim}

\end{reflist}  
\begin{reflist}
\item[]\begin{specHead}{TEI.macro.xtext}{macro.xtext} (extended text) defines a sequence of character data and gaiji elements.\end{specHead} 
    \item[{Module}]
  tei — \hyperref[ST]{The TEI Infrastructure}
    \item[{Used by}]
  \hyperref[TEI.c]{c} \hyperref[TEI.depth]{depth} \hyperref[TEI.dim]{dim} \hyperref[TEI.ex]{ex} \hyperref[TEI.height]{height} \hyperref[TEI.mapping]{mapping} \hyperref[TEI.memberOf]{memberOf} \hyperref[TEI.string]{string} \hyperref[TEI.value]{value} \hyperref[TEI.width]{width}
    \item[{Content model}]
  \mbox{}\hfill\\[-10pt]\begin{Verbatim}[fontsize=\small]
<content>
 <alternate minOccurs="0"
  maxOccurs="unbounded">
  <textNode/>
  <classRef key="model.gLike"/>
 </alternate>
</content>
    
\end{Verbatim}

    \item[{Declaration}]
  \fbox{\ttfamily macro.xtext = ( text | model.gLike )*} 
\end{reflist}  
\begin{reflist}
\item[]\begin{specHead}{TEI.teidata.authority}{teidata.authority} defines attribute values which derive from an authority list, which may be an enumerated list defined in the document's schema, a list or taxonomy elsewhere in the document, or an online taxonomy, gazetteer, or other authority.\end{specHead} 
    \item[{Module}]
  tei — \hyperref[ST]{The TEI Infrastructure}
    \item[{Used by}]
  Element: \begin{itemize}
\item \hyperref[TEI.move]{move}/@where
\end{itemize} 
    \item[{Content model}]
  \mbox{}\hfill\\[-10pt]\begin{Verbatim}[fontsize=\small]
<content>
 <alternate>
  <dataRef key="teidata.enumerated"/>
  <dataRef key="teidata.pointer">  
  </dataRef>
 </alternate>
</content>
    
\end{Verbatim}

    \item[{Declaration}]
  \mbox{}\hfill\\[-10pt]\begin{Verbatim}[fontsize=\small]
teidata.authority = teidata.enumerated | teidata.pointer
\end{Verbatim}

    \item[{Note}]
  \par
Attribute values with this datatype should either come from a value list in the attribute specification (\textsf{teidata.enumerated}) or be a valid URI (\textsf{teidata.pointer}).
\end{reflist}  
\begin{reflist}
\item[]\begin{specHead}{TEI.teidata.certainty}{teidata.certainty} defines the range of attribute values expressing a degree of certainty.\end{specHead} 
    \item[{Module}]
  tei — \hyperref[ST]{The TEI Infrastructure}
    \item[{Used by}]
  \hyperref[TEI.teidata.probCert]{teidata.probCert}Element: \begin{itemize}
\item \hyperref[TEI.certainty]{certainty}/@cert
\item \hyperref[TEI.precision]{precision}/@precision
\item \hyperref[TEI.purpose]{purpose}/@degree
\end{itemize} 
    \item[{Content model}]
  \mbox{}\hfill\\[-10pt]\begin{Verbatim}[fontsize=\small]
<content>
 <valList type="closed">
  <valItem ident="high"/>
  <valItem ident="medium"/>
  <valItem ident="low"/>
  <valItem ident="unknown"/>
 </valList>
</content>
    
\end{Verbatim}

    \item[{Declaration}]
  \fbox{\ttfamily teidata.certainty = "high" | "medium" | "low" | "unknown"} 
    \item[{Note}]
  \par
Certainty may be expressed by one of the predefined symbolic values high, medium, or low. The value unknown should be used in cases where the encoder does not wish to assert an opinion about the matter. 
\end{reflist}  
\begin{reflist}
\item[]\begin{specHead}{TEI.teidata.count}{teidata.count} defines the range of attribute values used for a non-negative integer value used as a count.\end{specHead} 
    \item[{Module}]
  tei — \hyperref[ST]{The TEI Infrastructure}
    \item[{Used by}]
  Element: \begin{itemize}
\item \hyperref[TEI.age]{age}/@value
\item \hyperref[TEI.datatype]{datatype}/@minOccurs
\item \hyperref[TEI.graph]{graph}/@order
\item \hyperref[TEI.graph]{graph}/@size
\item \hyperref[TEI.handDesc]{handDesc}/@hands
\item \hyperref[TEI.iNode]{iNode}/@outDegree
\item \hyperref[TEI.layout]{layout}/@columns
\item \hyperref[TEI.layout]{layout}/@streams
\item \hyperref[TEI.layout]{layout}/@ruledLines
\item \hyperref[TEI.layout]{layout}/@writtenLines
\item \hyperref[TEI.node]{node}/@inDegree
\item \hyperref[TEI.node]{node}/@outDegree
\item \hyperref[TEI.node]{node}/@degree
\item \hyperref[TEI.refState]{refState}/@length
\item \hyperref[TEI.root]{root}/@outDegree
\item \hyperref[TEI.sense]{sense}/@level
\item \hyperref[TEI.table]{table}/@rows
\item \hyperref[TEI.table]{table}/@cols
\item \hyperref[TEI.tagUsage]{tagUsage}/@occurs
\item \hyperref[TEI.tagUsage]{tagUsage}/@withId
\item \hyperref[TEI.tree]{tree}/@arity
\item \hyperref[TEI.tree]{tree}/@order
\item \hyperref[TEI.zone]{zone}/@rotate
\end{itemize} 
    \item[{Content model}]
  \mbox{}\hfill\\[-10pt]\begin{Verbatim}[fontsize=\small]
<content>
 <dataRef name="nonNegativeInteger"/>
</content>
    
\end{Verbatim}

    \item[{Declaration}]
  \fbox{\ttfamily teidata.count = xsd:nonNegativeInteger} 
    \item[{Note}]
  \par
Any positive integer value or zero is permitted
\end{reflist}  
\begin{reflist}
\item[]\begin{specHead}{TEI.teidata.duration.iso}{teidata.duration.iso} defines the range of attribute values available for representation of a duration in time using ISO 8601 standard formats\end{specHead} 
    \item[{Module}]
  tei — \hyperref[ST]{The TEI Infrastructure}
    \item[{Used by}]
  
    \item[{Content model}]
  \mbox{}\hfill\\[-10pt]\begin{Verbatim}[fontsize=\small]
<content>
 <dataRef name="token"
  restriction="[0-9.,DHMPRSTWYZ/:+\-]+"/>
</content>
    
\end{Verbatim}

    \item[{Declaration}]
  \mbox{}\hfill\\[-10pt]\begin{Verbatim}[fontsize=\small]
teidata.duration.iso = token { pattern = "[0-9.,DHMPRSTWYZ/:+\-]+" }
\end{Verbatim}

    \item[{Example}]
  \leavevmode\bgroup\index{time=<time>|exampleindex}\index{dur-iso=@dur-iso!<time>|exampleindex}\exampleFont \begin{shaded}\noindent\mbox{}{<\textbf{time}\hspace*{1em}{dur-iso}="{PT0,75H}">}three-quarters of an hour{</\textbf{time}>}\end{shaded}\egroup 


    \item[{Example}]
  \leavevmode\bgroup\index{date=<date>|exampleindex}\index{dur-iso=@dur-iso!<date>|exampleindex}\exampleFont \begin{shaded}\noindent\mbox{}{<\textbf{date}\hspace*{1em}{dur-iso}="{P1,5D}">}a day and a half{</\textbf{date}>}\end{shaded}\egroup 


    \item[{Example}]
  \leavevmode\bgroup\index{date=<date>|exampleindex}\index{dur-iso=@dur-iso!<date>|exampleindex}\exampleFont \begin{shaded}\noindent\mbox{}{<\textbf{date}\hspace*{1em}{dur-iso}="{P14D}">}a fortnight{</\textbf{date}>}\end{shaded}\egroup 


    \item[{Example}]
  \leavevmode\bgroup\index{time=<time>|exampleindex}\index{dur-iso=@dur-iso!<time>|exampleindex}\exampleFont \begin{shaded}\noindent\mbox{}{<\textbf{time}\hspace*{1em}{dur-iso}="{PT0.02S}">}20 ms{</\textbf{time}>}\end{shaded}\egroup 


    \item[{Note}]
  \par
A duration is expressed as a sequence of number-letter pairs, preceded by the letter P; the letter gives the unit and may be Y (year), M (month), D (day), H (hour), M (minute), or S (second), in that order. The numbers are all unsigned integers, except for the last, which may have a decimal component (using either \texttt{.} or \texttt{,} as the decimal point; the latter is preferred). If any number is \textit{0}, then that number-letter pair may be omitted. If any of the H (hour), M (minute), or S (second) number-letter pairs are present, then the separator \texttt{T} must precede the first ‘time’ number-letter pair.\par
For complete details, see ISO 8601 \textit{Data elements and interchange formats — Information interchange — Representation of dates and times}.
\end{reflist}  
\begin{reflist}
\item[]\begin{specHead}{TEI.teidata.duration.w3c}{teidata.duration.w3c} defines the range of attribute values available for representation of a duration in time using W3C datatypes.\end{specHead} 
    \item[{Module}]
  tei — \hyperref[ST]{The TEI Infrastructure}
    \item[{Used by}]
  
    \item[{Content model}]
  \fbox{\ttfamily <content>\newline
 <dataRef name="duration"/>\newline
</content>\newline
    } 
    \item[{Declaration}]
  \fbox{\ttfamily teidata.duration.w3c = xsd:duration} 
    \item[{Example}]
  \leavevmode\bgroup\index{time=<time>|exampleindex}\index{dur=@dur!<time>|exampleindex}\exampleFont \begin{shaded}\noindent\mbox{}{<\textbf{time}\hspace*{1em}{dur}="{PT45M}">}forty-five minutes{</\textbf{time}>}\end{shaded}\egroup 


    \item[{Example}]
  \leavevmode\bgroup\index{date=<date>|exampleindex}\index{dur=@dur!<date>|exampleindex}\exampleFont \begin{shaded}\noindent\mbox{}{<\textbf{date}\hspace*{1em}{dur}="{P1DT12H}">}a day and a half{</\textbf{date}>}\end{shaded}\egroup 


    \item[{Example}]
  \leavevmode\bgroup\index{date=<date>|exampleindex}\index{dur=@dur!<date>|exampleindex}\exampleFont \begin{shaded}\noindent\mbox{}{<\textbf{date}\hspace*{1em}{dur}="{P7D}">}a week{</\textbf{date}>}\end{shaded}\egroup 


    \item[{Example}]
  \leavevmode\bgroup\index{time=<time>|exampleindex}\index{dur=@dur!<time>|exampleindex}\exampleFont \begin{shaded}\noindent\mbox{}{<\textbf{time}\hspace*{1em}{dur}="{PT0.02S}">}20 ms{</\textbf{time}>}\end{shaded}\egroup 


    \item[{Note}]
  \par
A duration is expressed as a sequence of number-letter pairs, preceded by the letter P; the letter gives the unit and may be Y (year), M (month), D (day), H (hour), M (minute), or S (second), in that order. The numbers are all unsigned integers, except for the \texttt{S} number, which may have a decimal component (using \texttt{.} as the decimal point). If any number is \textit{0}, then that number-letter pair may be omitted. If any of the H (hour), M (minute), or S (second) number-letter pairs are present, then the separator \texttt{T} must precede the first ‘time’ number-letter pair.\par
For complete details, see the \xref{http://www.w3.org/TR/2004/REC-xmlschema-2-20041028/\#duration}{W3C specification}.
\end{reflist}  
\begin{reflist}
\item[]\begin{specHead}{TEI.teidata.enumerated}{teidata.enumerated} defines the range of attribute values expressed as a single XML name taken from a list of documented possibilities.\end{specHead} 
    \item[{Module}]
  tei — \hyperref[ST]{The TEI Infrastructure}
    \item[{Used by}]
  \hyperref[TEI.teidata.authority]{teidata.authority}Element: \begin{itemize}
\item \hyperref[TEI.abbr]{abbr}/@type
\item \hyperref[TEI.affiliation]{affiliation}/@type
\item \hyperref[TEI.age]{age}/@type
\item \hyperref[TEI.alt]{alt}/@mode
\item \hyperref[TEI.altGrp]{altGrp}/@mode
\item \hyperref[TEI.annotation]{annotation}/@motivation
\item \hyperref[TEI.app]{app}/@type
\item \hyperref[TEI.att]{att}/@scheme
\item \hyperref[TEI.attDef]{attDef}/@usage
\item \hyperref[TEI.attList]{attList}/@org
\item \hyperref[TEI.availability]{availability}/@status
\item \hyperref[TEI.birth]{birth}/@type
\item \hyperref[TEI.castItem]{castItem}/@type
\item \hyperref[TEI.certainty]{certainty}/@type
\item \hyperref[TEI.certainty]{certainty}/@locus
\item \hyperref[TEI.channel]{channel}/@mode
\item \hyperref[TEI.citeStructure]{citeStructure}/@unit
\item \hyperref[TEI.classSpec]{classSpec}/@type
\item \hyperref[TEI.classSpec]{classSpec}/@generate
\item \hyperref[TEI.classes]{classes}/@mode
\item \hyperref[TEI.constitution]{constitution}/@type
\item \hyperref[TEI.constraintSpec]{constraintSpec}/@type
\item \hyperref[TEI.constraintSpec]{constraintSpec}/@scheme
\item \hyperref[TEI.correction]{correction}/@status
\item \hyperref[TEI.correction]{correction}/@method
\item \hyperref[TEI.correspAction]{correspAction}/@type
\item \hyperref[TEI.death]{death}/@type
\item \hyperref[TEI.derivation]{derivation}/@type
\item \hyperref[TEI.desc]{desc}/@type
\item \hyperref[TEI.dimensions]{dimensions}/@type
\item \hyperref[TEI.distinct]{distinct}/@type
\item \hyperref[TEI.divGen]{divGen}/@type
\item \hyperref[TEI.domain]{domain}/@type
\item \hyperref[TEI.education]{education}/@type
\item \hyperref[TEI.egXML]{egXML}/@valid
\item \hyperref[TEI.factuality]{factuality}/@type
\item \hyperref[TEI.faith]{faith}/@type
\item \hyperref[TEI.form]{form}/@type
\item \hyperref[TEI.fs]{fs}/@type
\item \hyperref[TEI.fsDecl]{fsDecl}/@type
\item \hyperref[TEI.fsdLink]{fsdLink}/@type
\item \hyperref[TEI.fw]{fw}/@type
\item \hyperref[TEI.gap]{gap}/@reason
\item \hyperref[TEI.gap]{gap}/@agent
\item \hyperref[TEI.geoDecl]{geoDecl}/@datum
\item \hyperref[TEI.gi]{gi}/@scheme
\item \hyperref[TEI.gram]{gram}/@type
\item \hyperref[TEI.graph]{graph}/@type
\item \hyperref[TEI.hyphenation]{hyphenation}/@eol
\item \hyperref[TEI.iType]{iType}/@type
\item \hyperref[TEI.idno]{idno}/@type
\item \hyperref[TEI.interaction]{interaction}/@type
\item \hyperref[TEI.interaction]{interaction}/@active
\item \hyperref[TEI.interaction]{interaction}/@passive
\item \hyperref[TEI.interp]{interp}/@type
\item \hyperref[TEI.interpGrp]{interpGrp}/@type
\item \hyperref[TEI.join]{join}/@scope
\item \hyperref[TEI.langKnowledge]{langKnowledge}/@type
\item \hyperref[TEI.lbl]{lbl}/@type
\item \hyperref[TEI.list]{list}/@type
\item \hyperref[TEI.listForest]{listForest}/@type
\item \hyperref[TEI.material]{material}/@function
\item \hyperref[TEI.measure]{measure}/@type
\item \hyperref[TEI.memberOf]{memberOf}/@mode
\item \hyperref[TEI.metDecl]{metDecl}/@type
\item \hyperref[TEI.model]{model}/@behaviour
\item \hyperref[TEI.model]{model}/@output
\item \hyperref[TEI.modelGrp]{modelGrp}/@output
\item \hyperref[TEI.modelSequence]{modelSequence}/@output
\item \hyperref[TEI.move]{move}/@type
\item \hyperref[TEI.nationality]{nationality}/@type
\item \hyperref[TEI.node]{node}/@type
\item \hyperref[TEI.normalization]{normalization}/@method
\item \hyperref[TEI.num]{num}/@type
\item \hyperref[TEI.oRef]{oRef}/@type
\item \hyperref[TEI.objectDesc]{objectDesc}/@form
\item \hyperref[TEI.occupation]{occupation}/@type
\item \hyperref[TEI.org]{org}/@role
\item \hyperref[TEI.orth]{orth}/@type
\item \hyperref[TEI.outputRendition]{outputRendition}/@scope
\item \hyperref[TEI.param]{param}/@name
\item \hyperref[TEI.pc]{pc}/@force
\item \hyperref[TEI.pc]{pc}/@unit
\item \hyperref[TEI.persPronouns]{persPronouns}/@evidence
\item \hyperref[TEI.persPronouns]{persPronouns}/@value
\item \hyperref[TEI.person]{person}/@role
\item \hyperref[TEI.person]{person}/@age
\item \hyperref[TEI.personGrp]{personGrp}/@role
\item \hyperref[TEI.personGrp]{personGrp}/@age
\item \hyperref[TEI.persona]{persona}/@role
\item \hyperref[TEI.persona]{persona}/@age
\item \hyperref[TEI.preparedness]{preparedness}/@type
\item \hyperref[TEI.punctuation]{punctuation}/@marks
\item \hyperref[TEI.punctuation]{punctuation}/@placement
\item \hyperref[TEI.purpose]{purpose}/@type
\item \hyperref[TEI.q]{q}/@type
\item \hyperref[TEI.quotation]{quotation}/@marks
\item \hyperref[TEI.recording]{recording}/@type
\item \hyperref[TEI.relation]{relation}/@name
\item \hyperref[TEI.rendition]{rendition}/@scope
\item \hyperref[TEI.residence]{residence}/@type
\item \hyperref[TEI.respons]{respons}/@locus
\item \hyperref[TEI.sex]{sex}/@type
\item \hyperref[TEI.shift]{shift}/@feature
\item \hyperref[TEI.shift]{shift}/@new
\item \hyperref[TEI.socecStatus]{socecStatus}/@type
\item \hyperref[TEI.sound]{sound}/@type
\item \hyperref[TEI.space]{space}/@dim
\item \hyperref[TEI.span]{span}/@type
\item \hyperref[TEI.spanGrp]{spanGrp}/@type
\item \hyperref[TEI.stage]{stage}/@type
\item \hyperref[TEI.supportDesc]{supportDesc}/@material
\item \hyperref[TEI.surface]{surface}/@attachment
\item \hyperref[TEI.tag]{tag}/@type
\item \hyperref[TEI.tag]{tag}/@scheme
\item \hyperref[TEI.tech]{tech}/@type
\item \hyperref[TEI.timeline]{timeline}/@unit
\item \hyperref[TEI.title]{title}/@type
\item \hyperref[TEI.title]{title}/@level
\item \hyperref[TEI.titlePage]{titlePage}/@type
\item \hyperref[TEI.titlePart]{titlePart}/@type
\item \hyperref[TEI.tree]{tree}/@ord
\item \hyperref[TEI.u]{u}/@trans
\item \hyperref[TEI.unclear]{unclear}/@reason
\item \hyperref[TEI.unclear]{unclear}/@agent
\item \hyperref[TEI.usg]{usg}/@type
\item \hyperref[TEI.vColl]{vColl}/@org
\item \hyperref[TEI.vMerge]{vMerge}/@org
\item \hyperref[TEI.valList]{valList}/@type
\item \hyperref[TEI.variantEncoding]{variantEncoding}/@method
\item \hyperref[TEI.variantEncoding]{variantEncoding}/@location
\item \hyperref[TEI.when]{when}/@unit
\item \hyperref[TEI.witDetail]{witDetail}/@type
\item \hyperref[TEI.xr]{xr}/@type
\end{itemize} 
    \item[{Content model}]
  \fbox{\ttfamily <content>\newline
 <dataRef key="teidata.word"/>\newline
</content>\newline
    } 
    \item[{Declaration}]
  \fbox{\ttfamily teidata.enumerated = teidata.word} 
    \item[{Note}]
  \par
Attributes using this datatype must contain a single ‘word’ which contains only letters, digits, punctuation characters, or symbols: thus it cannot include whitespace.\par
Typically, the list of documented possibilities will be provided (or exemplified) by a value list in the associated attribute specification, expressed with a \hyperref[TEI.valList]{<valList>} element.
\end{reflist}  
\begin{reflist}
\item[]\begin{specHead}{TEI.teidata.interval}{teidata.interval} defines attribute values used to express an interval value.\end{specHead} 
    \item[{Module}]
  tei — \hyperref[ST]{The TEI Infrastructure}
    \item[{Used by}]
  Element: \begin{itemize}
\item \hyperref[TEI.timeline]{timeline}/@interval
\item \hyperref[TEI.when]{when}/@interval
\end{itemize} 
    \item[{Content model}]
  \mbox{}\hfill\\[-10pt]\begin{Verbatim}[fontsize=\small]
<content>
 <alternate>
  <dataRef name="float"/>
  <valList>
   <valItem ident="regular"/>
   <valItem ident="irregular"/>
   <valItem ident="unknown"/>
  </valList>
 </alternate>
</content>
    
\end{Verbatim}

    \item[{Declaration}]
  \mbox{}\hfill\\[-10pt]\begin{Verbatim}[fontsize=\small]
teidata.interval = xsd:float | ( "regular" | "irregular" | "unknown" )
\end{Verbatim}

    \item[{Note}]
  \par
Any value greater than zero or any one of the values regular, irregular, unknown.
\end{reflist}  
\begin{reflist}
\item[]\begin{specHead}{TEI.teidata.language}{teidata.language} defines the range of attribute values used to identify a particular combination of human language and writing system. [\textit{\hyperref[CHSH]{vi.1\ Language Identification}}]\end{specHead} 
    \item[{Module}]
  tei — \hyperref[ST]{The TEI Infrastructure}
    \item[{Used by}]
  Element: \begin{itemize}
\item \hyperref[TEI.langKnowledge]{langKnowledge}/@tags
\item \hyperref[TEI.langKnown]{langKnown}/@tag
\item \hyperref[TEI.language]{language}/@ident
\item \hyperref[TEI.schemaSpec]{schemaSpec}/@targetLang
\item \hyperref[TEI.schemaSpec]{schemaSpec}/@docLang
\item \hyperref[TEI.textLang]{textLang}/@mainLang
\item \hyperref[TEI.textLang]{textLang}/@otherLangs
\end{itemize} 
    \item[{Content model}]
  \mbox{}\hfill\\[-10pt]\begin{Verbatim}[fontsize=\small]
<content>
 <alternate>
  <dataRef name="language"/>
  <valList>
   <valItem ident=""/>
  </valList>
 </alternate>
</content>
    
\end{Verbatim}

    \item[{Declaration}]
  \fbox{\ttfamily teidata.language = xsd:language | ( "" )} 
    \item[{Note}]
  \par
The values for this attribute are language ‘tags’ as defined in \xref{https://tools.ietf.org/html/bcp47}{BCP 47}. Currently BCP 47 comprises RFC 5646 and RFC 4647; over time, other IETF documents may succeed these as the best current practice.\par
A ‘language tag’, per BCP 47, is assembled from a sequence of components or \textit{subtags} separated by the hyphen character (\textit{-}, U+002D). The tag is made of the following subtags, in the following order. Every subtag except the first is optional. If present, each occurs only once, except the fourth and fifth components (variant and extension), which are repeatable. \begin{description}

\item[{language}]The IANA-registered code for the language. This is almost always the same as the ISO 639 2-letter language code if there is one. The list of available registered language subtags can be found at \url{http://www.iana.org/assignments/language-subtag-registry}. It is recommended that this code be written in lower case.
\item[{script}]The ISO 15924 code for the script. These codes consist of 4 letters, and it is recommended they be written with an initial capital, the other three letters in lower case. The canonical list of codes is maintained by the Unicode Consortium, and is available at \url{http://unicode.org/iso15924/iso15924-codes.html}. The IETF recommends this code be omitted unless it is necessary to make a distinction you need.
\item[{region}]Either an ISO 3166 country code or a UN M.49 region code that is registered with IANA (not all such codes are registered, e.g. UN codes for economic groupings or codes for countries for which there is already an ISO 3166 2-letter code are not registered). The former consist of 2 letters, and it is recommended they be written in upper case; the list of codes can be searched or browsed at \url{https://www.iso.org/obp/ui/\#search/code/}. The latter consist of 3 digits; the list of codes can be found at \url{http://unstats.un.org/unsd/methods/m49/m49.htm}.
\item[{variant}]An IANA-registered variation. These codes ‘are used to indicate additional, well-recognized variations that define a language or its dialects that are not covered by other available subtags’.
\item[{extension}]An extension has the format of a single letter followed by a hyphen followed by additional subtags. These exist to allow for future extension to BCP 47, but as of this writing no such extensions are in use.
\item[{private use}]An extension that uses the initial subtag of the single letter \textit{x} (i.e., starts with \texttt{x-}) has no meaning except as negotiated among the parties involved. These should be used with great care, since they interfere with the interoperability that use of RFC 4646 is intended to promote. In order for a document that makes use of these subtags to be TEI-conformant, a corresponding \hyperref[TEI.language]{<language>} element must be present in the TEI header.
\end{description} \par
There are two exceptions to the above format. First, there are language tags in the \xref{http://www.iana.org/assignments/language-subtag-registry}{IANA registry} that do not match the above syntax, but are present because they have been ‘grandfathered’ from previous specifications.\par
Second, an entire language tag can consist of only a private use subtag. These tags start with \texttt{x-}, and do not need to follow any further rules established by the IETF and endorsed by these Guidelines. Like all language tags that make use of private use subtags, the language in question must be documented in a corresponding \hyperref[TEI.language]{<language>} element in the TEI header.\par
Examples include \begin{description}

\item[{sn}]Shona
\item[{zh-TW}]Taiwanese
\item[{zh-Hant-HK}]Chinese written in traditional script as used in Hong Kong
\item[{en-SL}]English as spoken in Sierra Leone
\item[{pl}]Polish
\item[{es-MX}]Spanish as spoken in Mexico
\item[{es-419}]Spanish as spoken in Latin America
\end{description} \par
The W3C Internationalization Activity has published a useful introduction to BCP 47, \xref{http://www.w3.org/International/articles/language-tags/Overview.en.php}{Language tags in HTML and XML}.
\end{reflist}  
\begin{reflist}
\item[]\begin{specHead}{TEI.teidata.name}{teidata.name} defines the range of attribute values expressed as an XML Name.\end{specHead} 
    \item[{Module}]
  tei — \hyperref[ST]{The TEI Infrastructure}
    \item[{Used by}]
  \hyperref[TEI.att]{att} \hyperref[TEI.gi]{gi}Element: \begin{itemize}
\item \hyperref[TEI.application]{application}/@ident
\item \hyperref[TEI.attRef]{attRef}/@name
\item \hyperref[TEI.equiv]{equiv}/@name
\item \hyperref[TEI.f]{f}/@name
\item \hyperref[TEI.fDecl]{fDecl}/@name
\item \hyperref[TEI.fsDecl]{fsDecl}/@baseTypes
\item \hyperref[TEI.index]{index}/@indexName
\item \hyperref[TEI.join]{join}/@result
\item \hyperref[TEI.joinGrp]{joinGrp}/@result
\item \hyperref[TEI.memberOf]{memberOf}/@key
\item \hyperref[TEI.model]{model}/@cssClass
\item \hyperref[TEI.schemaSpec]{schemaSpec}/@start
\item \hyperref[TEI.specDesc]{specDesc}/@key
\item \hyperref[TEI.specDesc]{specDesc}/@atts
\item \hyperref[TEI.tagUsage]{tagUsage}/@gi
\item \hyperref[TEI.transcriptionDesc]{transcriptionDesc}/@ident
\end{itemize} 
    \item[{Content model}]
  \fbox{\ttfamily <content>\newline
 <dataRef name="Name"/>\newline
</content>\newline
    } 
    \item[{Declaration}]
  \fbox{\ttfamily teidata.name = xsd:Name} 
    \item[{Note}]
  \par
Attributes using this datatype must contain a single word which follows the rules defining a legal XML name (see \url{http://www.w3.org/TR/REC-xml/\#dt-name}): for example they cannot include whitespace or begin with digits.
\end{reflist}  
\begin{reflist}
\item[]\begin{specHead}{TEI.teidata.namespace}{teidata.namespace} defines the range of attribute values used to indicate XML namespaces as defined by the W3C \xref{http://www.w3.org/TR/1999/REC-xml-names-19990114/}{Namespaces in XML} Technical Recommendation.\end{specHead} 
    \item[{Module}]
  tei — \hyperref[ST]{The TEI Infrastructure}
    \item[{Used by}]
  Element: \begin{itemize}
\item \hyperref[TEI.anyElement]{anyElement}/@require
\item \hyperref[TEI.attDef]{attDef}/@ns
\item \hyperref[TEI.namespace]{namespace}/@name
\end{itemize} 
    \item[{Content model}]
  \fbox{\ttfamily <content>\newline
 <dataRef name="anyURI"/>\newline
</content>\newline
    } 
    \item[{Declaration}]
  \fbox{\ttfamily teidata.namespace = xsd:anyURI} 
    \item[{Note}]
  \par
The range of syntactically valid values is defined by \xref{http://www.ietf.org/rfc/rfc3986.txt}{RFC 3986 \textit{Uniform Resource Identifier (URI): Generic Syntax}}
\end{reflist}  
\begin{reflist}
\item[]\begin{specHead}{TEI.teidata.namespaceOrName}{teidata.namespaceOrName} defines attribute values which contain either an absolute namespace URI or a qualified XML name.\end{specHead} 
    \item[{Module}]
  tei — \hyperref[ST]{The TEI Infrastructure}
    \item[{Used by}]
  Element: \begin{itemize}
\item \hyperref[TEI.anyElement]{anyElement}/@except
\item \hyperref[TEI.schemaSpec]{schemaSpec}/@defaultExceptions
\end{itemize} 
    \item[{Content model}]
  \mbox{}\hfill\\[-10pt]\begin{Verbatim}[fontsize=\small]
<content>
 <alternate>
  <dataRef name="anyURI">
   <dataFacet name="pattern"
    value="[^/]+:.*"/>
  </dataRef>
  <dataRef name="Name">
   <dataFacet name="pattern" value=".+:.+"/>
  </dataRef>
 </alternate>
</content>
    
\end{Verbatim}

    \item[{Declaration}]
  \fbox{\ttfamily teidata.namespaceOrName = xsd:anyURI | xsd:Name} 
\end{reflist}  
\begin{reflist}
\item[]\begin{specHead}{TEI.teidata.nullOrName}{teidata.nullOrName} defines attribute values which contain either the null string or an XML name.\end{specHead} 
    \item[{Module}]
  tei — \hyperref[ST]{The TEI Infrastructure}
    \item[{Used by}]
  
    \item[{Content model}]
  \mbox{}\hfill\\[-10pt]\begin{Verbatim}[fontsize=\small]
<content>
 <alternate>
  <valList>
   <valItem ident=""/>
  </valList>
  <dataRef name="NCName"/>
 </alternate>
</content>
    
\end{Verbatim}

    \item[{Declaration}]
  \fbox{\ttfamily teidata.nullOrName = ( "" ) | xsd:NCName} 
    \item[{Note}]
  \par
The rules defining an XML name form a part of the XML Specification.
\end{reflist}  
\begin{reflist}
\item[]\begin{specHead}{TEI.teidata.numeric}{teidata.numeric} defines the range of attribute values used for numeric values.\end{specHead} 
    \item[{Module}]
  tei — \hyperref[ST]{The TEI Infrastructure}
    \item[{Used by}]
  Element: \begin{itemize}
\item \hyperref[TEI.memberOf]{memberOf}/@max
\item \hyperref[TEI.memberOf]{memberOf}/@min
\item \hyperref[TEI.num]{num}/@value
\item \hyperref[TEI.numeric]{numeric}/@value
\item \hyperref[TEI.numeric]{numeric}/@max
\item \hyperref[TEI.precision]{precision}/@stdDeviation
\end{itemize} 
    \item[{Content model}]
  \mbox{}\hfill\\[-10pt]\begin{Verbatim}[fontsize=\small]
<content>
 <alternate>
  <dataRef name="double"/>
  <dataRef name="token"
   restriction="(\-?[\d]+/\-?[\d]+)"/>
  <dataRef name="decimal"/>
 </alternate>
</content>
    
\end{Verbatim}

    \item[{Declaration}]
  \mbox{}\hfill\\[-10pt]\begin{Verbatim}[fontsize=\small]
teidata.numeric =
   xsd:double | token { pattern = "(\-?[\d]+/\-?[\d]+)" } | xsd:decimal
\end{Verbatim}

    \item[{Note}]
  \par
Any numeric value, represented as a decimal number, in floating point format, or as a ratio.\par
To represent a floating point number, expressed in scientific notation, ‘E notation’, a variant of ‘exponential notation’, may be used. In this format, the value is expressed as two numbers separated by the letter E. The first number, the significand (sometimes called the mantissa) is given in decimal format, while the second is an integer. The value is obtained by multiplying the mantissa by 10 the number of times indicated by the integer. Thus the value represented in decimal notation as 1000.0 might be represented in scientific notation as 10E3.\par
A value expressed as a ratio is represented by two integer values separated by a solidus (/) character. Thus, the value represented in decimal notation as 0.5 might be represented as a ratio by the string 1/2.
\end{reflist}  
\begin{reflist}
\item[]\begin{specHead}{TEI.teidata.outputMeasurement}{teidata.outputMeasurement} defines a range of values for use in specifying the size of an object that is intended for display.\end{specHead} 
    \item[{Module}]
  tei — \hyperref[ST]{The TEI Infrastructure}
    \item[{Used by}]
  
    \item[{Content model}]
  \mbox{}\hfill\\[-10pt]\begin{Verbatim}[fontsize=\small]
<content>
 <dataRef name="token"
  restriction="[\-+]?\d+(\.\d+)?(%|cm|mm|in|pt|pc|px|em|ex|gd|rem|vw|vh|vm)"/>
</content>
    
\end{Verbatim}

    \item[{Declaration}]
  \mbox{}\hfill\\[-10pt]\begin{Verbatim}[fontsize=\small]
teidata.outputMeasurement =
   token
   {
      pattern = "[\-+]?\d+(\.\d+)?(%|cm|mm|in|pt|pc|px|em|ex|gd|rem|vw|vh|vm)"
   }
\end{Verbatim}

    \item[{Example}]
  \leavevmode\bgroup\index{figure=<figure>|exampleindex}\index{head=<head>|exampleindex}\index{figDesc=<figDesc>|exampleindex}\index{mentioned=<mentioned>|exampleindex}\index{mentioned=<mentioned>|exampleindex}\index{graphic=<graphic>|exampleindex}\index{height=@height!<graphic>|exampleindex}\index{width=@width!<graphic>|exampleindex}\index{url=@url!<graphic>|exampleindex}\exampleFont \begin{shaded}\noindent\mbox{}{<\textbf{figure}>}\mbox{}\newline 
\hspace*{1em}{<\textbf{head}>}The TEI Logo{</\textbf{head}>}\mbox{}\newline 
\hspace*{1em}{<\textbf{figDesc}>}Stylized yellow angle brackets with the letters {<\textbf{mentioned}>}TEI{</\textbf{mentioned}>} in\mbox{}\newline 
\hspace*{1em}\hspace*{1em} between and {<\textbf{mentioned}>}text encoding initiative{</\textbf{mentioned}>} underneath, all on a white\mbox{}\newline 
\hspace*{1em}\hspace*{1em} background.{</\textbf{figDesc}>}\mbox{}\newline 
\hspace*{1em}{<\textbf{graphic}\hspace*{1em}{height}="{600px}"\hspace*{1em}{width}="{600px}"\mbox{}\newline 
\hspace*{1em}\hspace*{1em}{url}="{http://www.tei-c.org/logos/TEI-600.jpg}"/>}\mbox{}\newline 
{</\textbf{figure}>}\end{shaded}\egroup 


    \item[{Note}]
  \par
These values map directly onto the values used by XSL-FO and CSS. For definitions of the units see those specifications; at the time of this writing the most complete list is in the \xref{http://www.w3.org/TR/2005/WD-css3-values-20050726/\#numbers0}{CSS3 working draft}.
\end{reflist}  
\begin{reflist}
\item[]\begin{specHead}{TEI.teidata.pattern}{teidata.pattern} defines attribute values which are expressed as a regular expression.\end{specHead} 
    \item[{Module}]
  tei — \hyperref[ST]{The TEI Infrastructure}
    \item[{Used by}]
  Element: \begin{itemize}
\item \hyperref[TEI.dataRef]{dataRef}/@restriction
\item \hyperref[TEI.metDecl]{metDecl}/@pattern
\end{itemize} 
    \item[{Content model}]
  \fbox{\ttfamily <content>\newline
 <dataRef name="token"/>\newline
</content>\newline
    } 
    \item[{Declaration}]
  \fbox{\ttfamily teidata.pattern = token} 
    \item[{Note}]
  \par

\begin{quote}
 A regular expression, often called a \textit{pattern}, is an expression that describes a set of strings. They are usually used to give a concise description of a set, without having to list all elements. For example, the set containing the three strings \textit{Handel}, \textit{Händel}, and \textit{Haendel} can be described by the pattern \texttt{H(ä|ae?)ndel} (or alternatively, it is said that the pattern \texttt{H(ä|ae?)ndel} \textit{matches} each of the three strings)\xref{http://en.wikipedia.org/wiki/Regular_expression\#Basic_concepts}{Wikipedia}\par

\end{quote}
\par
This TEI datatype is mapped to the XSD token datatype, and may therefore contain any string of characters. However, it is recommended that the value used conform to the particular flavour of regular expression syntax supported by XSD Schema. 
\end{reflist}  
\begin{reflist}
\item[]\begin{specHead}{TEI.teidata.point}{teidata.point} defines the data type used to express a point in cartesian space.\end{specHead} 
    \item[{Module}]
  tei — \hyperref[ST]{The TEI Infrastructure}
    \item[{Used by}]
  Element: \begin{itemize}
\item \hyperref[TEI.path]{path}/@points
\end{itemize} 
    \item[{Content model}]
  \mbox{}\hfill\\[-10pt]\begin{Verbatim}[fontsize=\small]
<content>
 <dataRef name="token"
  restriction="(-?[0-9]+(\.[0-9]+)?,-?[0-9]+(\.[0-9]+)?)"/>
</content>
    
\end{Verbatim}

    \item[{Declaration}]
  \mbox{}\hfill\\[-10pt]\begin{Verbatim}[fontsize=\small]
teidata.point = token { pattern = "(-?[0-9]+(\.[0-9]+)?,-?[0-9]+(\.[0-9]+)?)" }
\end{Verbatim}

    \item[{Example}]
  \leavevmode\bgroup\index{facsimile=<facsimile>|exampleindex}\index{surface=<surface>|exampleindex}\index{ulx=@ulx!<surface>|exampleindex}\index{uly=@uly!<surface>|exampleindex}\index{lrx=@lrx!<surface>|exampleindex}\index{lry=@lry!<surface>|exampleindex}\index{zone=<zone>|exampleindex}\index{points=@points!<zone>|exampleindex}\index{graphic=<graphic>|exampleindex}\index{url=@url!<graphic>|exampleindex}\exampleFont \begin{shaded}\noindent\mbox{}{<\textbf{facsimile}>}\mbox{}\newline 
\hspace*{1em}{<\textbf{surface}\hspace*{1em}{ulx}="{0}"\hspace*{1em}{uly}="{0}"\hspace*{1em}{lrx}="{400}"\hspace*{1em}{lry}="{280}">}\mbox{}\newline 
\hspace*{1em}\hspace*{1em}{<\textbf{zone}\hspace*{1em}{points}="{220,100 300,210 170,250 123,234}">}\mbox{}\newline 
\hspace*{1em}\hspace*{1em}\hspace*{1em}{<\textbf{graphic}\hspace*{1em}{url}="{handwriting.png }"/>}\mbox{}\newline 
\hspace*{1em}\hspace*{1em}{</\textbf{zone}>}\mbox{}\newline 
\hspace*{1em}{</\textbf{surface}>}\mbox{}\newline 
{</\textbf{facsimile}>}\end{shaded}\egroup 


    \item[{Note}]
  \par
A point is defined by two numeric values, which should be expressed as decimal numbers. Neither number can end in a decimal point. E.g., both 0.0,84.2 and 0,84 are allowed, but 0.,84. is not. 
\end{reflist}  
\begin{reflist}
\item[]\begin{specHead}{TEI.teidata.pointer}{teidata.pointer} defines the range of attribute values used to provide a single URI, absolute or relative, pointing to some other resource, either within the current document or elsewhere.\end{specHead} 
    \item[{Module}]
  tei — \hyperref[ST]{The TEI Infrastructure}
    \item[{Used by}]
  \hyperref[TEI.teidata.authority]{teidata.authority}Element: \begin{itemize}
\item \hyperref[TEI.alt]{alt}/@target
\item \hyperref[TEI.annotation]{annotation}/@target
\item \hyperref[TEI.app]{app}/@from
\item \hyperref[TEI.app]{app}/@to
\item \hyperref[TEI.arc]{arc}/@from
\item \hyperref[TEI.arc]{arc}/@to
\item \hyperref[TEI.catRef]{catRef}/@scheme
\item \hyperref[TEI.certainty]{certainty}/@given
\item \hyperref[TEI.change]{change}/@target
\item \hyperref[TEI.classCode]{classCode}/@scheme
\item \hyperref[TEI.conversion]{conversion}/@fromUnit
\item \hyperref[TEI.conversion]{conversion}/@toUnit
\item \hyperref[TEI.dataRef]{dataRef}/@ref
\item \hyperref[TEI.eLeaf]{eLeaf}/@value
\item \hyperref[TEI.eTree]{eTree}/@value
\item \hyperref[TEI.equiv]{equiv}/@uri
\item \hyperref[TEI.equiv]{equiv}/@filter
\item \hyperref[TEI.event]{event}/@where
\item \hyperref[TEI.f]{f}/@fVal
\item \hyperref[TEI.fs]{fs}/@feats
\item \hyperref[TEI.fsdLink]{fsdLink}/@target
\item \hyperref[TEI.g]{g}/@ref
\item \hyperref[TEI.handShift]{handShift}/@new
\item \hyperref[TEI.iNode]{iNode}/@value
\item \hyperref[TEI.iNode]{iNode}/@children
\item \hyperref[TEI.iNode]{iNode}/@parent
\item \hyperref[TEI.iNode]{iNode}/@follow
\item \hyperref[TEI.keywords]{keywords}/@scheme
\item \hyperref[TEI.leaf]{leaf}/@value
\item \hyperref[TEI.leaf]{leaf}/@parent
\item \hyperref[TEI.leaf]{leaf}/@follow
\item \hyperref[TEI.locus]{locus}/@scheme
\item \hyperref[TEI.locusGrp]{locusGrp}/@scheme
\item \hyperref[TEI.material]{material}/@target
\item \hyperref[TEI.metamark]{metamark}/@target
\item \hyperref[TEI.moduleRef]{moduleRef}/@url
\item \hyperref[TEI.move]{move}/@perf
\item \hyperref[TEI.node]{node}/@value
\item \hyperref[TEI.node]{node}/@adjTo
\item \hyperref[TEI.node]{node}/@adjFrom
\item \hyperref[TEI.node]{node}/@adj
\item \hyperref[TEI.nym]{nym}/@parts
\item \hyperref[TEI.occupation]{occupation}/@scheme
\item \hyperref[TEI.occupation]{occupation}/@code
\item \hyperref[TEI.redo]{redo}/@target
\item \hyperref[TEI.relatedItem]{relatedItem}/@target
\item \hyperref[TEI.relation]{relation}/@active
\item \hyperref[TEI.relation]{relation}/@mutual
\item \hyperref[TEI.relation]{relation}/@passive
\item \hyperref[TEI.root]{root}/@value
\item \hyperref[TEI.root]{root}/@children
\item \hyperref[TEI.rt]{rt}/@target
\item \hyperref[TEI.rt]{rt}/@from
\item \hyperref[TEI.rt]{rt}/@to
\item \hyperref[TEI.socecStatus]{socecStatus}/@scheme
\item \hyperref[TEI.socecStatus]{socecStatus}/@code
\item \hyperref[TEI.space]{space}/@resp
\item \hyperref[TEI.span]{span}/@from
\item \hyperref[TEI.span]{span}/@to
\item \hyperref[TEI.specGrpRef]{specGrpRef}/@target
\item \hyperref[TEI.tech]{tech}/@perf
\item \hyperref[TEI.timeline]{timeline}/@origin
\item \hyperref[TEI.triangle]{triangle}/@value
\item \hyperref[TEI.undo]{undo}/@target
\item \hyperref[TEI.when]{when}/@since
\item \hyperref[TEI.witDetail]{witDetail}/@wit
\end{itemize} 
    \item[{Content model}]
  \fbox{\ttfamily <content>\newline
 <dataRef name="anyURI"/>\newline
</content>\newline
    } 
    \item[{Declaration}]
  \fbox{\ttfamily teidata.pointer = xsd:anyURI} 
    \item[{Note}]
  \par
The range of syntactically valid values is defined by \xref{http://www.ietf.org/rfc/rfc3986.txt}{RFC 3986} \textit{Uniform Resource Identifier (URI): Generic Syntax}. Note that the values themselves are encoded using \xref{http://www.ietf.org/rfc/rfc3987.txt}{RFC 3987} \textit{Internationalized Resource Identifiers} (IRIs) mapping to URIs. For example, \texttt{https://secure.wikimedia.org/wikipedia/en/wiki/\%} is encoded as \texttt{https://secure.wikimedia.org/wikipedia/en/wiki/\%25} while \texttt{http://موقع.وزارة-الاتصالات.مصر/} is encoded as \texttt{http://xn--4gbrim.xn----rmckbbajlc6dj7bxne2c.xn--wgbh1c/}
\end{reflist}  
\begin{reflist}
\item[]\begin{specHead}{TEI.teidata.prefix}{teidata.prefix} defines a range of values that may function as a URI scheme name.\end{specHead} 
    \item[{Module}]
  tei — \hyperref[ST]{The TEI Infrastructure}
    \item[{Used by}]
  Element: \begin{itemize}
\item \hyperref[TEI.prefixDef]{prefixDef}/@ident
\end{itemize} 
    \item[{Content model}]
  \mbox{}\hfill\\[-10pt]\begin{Verbatim}[fontsize=\small]
<content>
 <dataRef name="token"
  restriction="[a-z][a-z0-9\+\.\-]*"/>
</content>
    
\end{Verbatim}

    \item[{Declaration}]
  \mbox{}\hfill\\[-10pt]\begin{Verbatim}[fontsize=\small]
teidata.prefix = token { pattern = "[a-z][a-z0-9\+\.\-]*" }
\end{Verbatim}

    \item[{Note}]
  \par
This datatype is used to constrain a string of characters to one that can be used as a URI scheme name according to \xref{http://www.ietf.org/rfc/rfc3986.txt}{RFC 3986}, \xref{https://tools.ietf.org/html/rfc3986\#section-3.1}{section 3.1}. Thus only the 26 lowercase letters a–z, the 10 digits 0–9, the plus sign, the period, and the hyphen are permitted, and the value must start with a letter.
\end{reflist}  
\begin{reflist}
\item[]\begin{specHead}{TEI.teidata.probability}{teidata.probability} defines the range of attribute values expressing a probability.\end{specHead} 
    \item[{Module}]
  tei — \hyperref[ST]{The TEI Infrastructure}
    \item[{Used by}]
  \hyperref[TEI.teidata.probCert]{teidata.probCert}Element: \begin{itemize}
\item \hyperref[TEI.alt]{alt}/@weights
\item \hyperref[TEI.certainty]{certainty}/@degree
\end{itemize} 
    \item[{Content model}]
  \fbox{\ttfamily <content>\newline
 <dataRef name="double"/>\newline
</content>\newline
    } 
    \item[{Declaration}]
  \fbox{\ttfamily teidata.probability = xsd:double} 
    \item[{Note}]
  \par
Probability is expressed as a real number between 0 and 1; 0 representing \textit{certainly false} and 1 representing \textit{certainly true}.
\end{reflist}  
\begin{reflist}
\item[]\begin{specHead}{TEI.teidata.probCert}{teidata.probCert} defines a range of attribute values which can be expressed either as a numeric probability or as a coded certainty value.\end{specHead} 
    \item[{Module}]
  tei — \hyperref[ST]{The TEI Infrastructure}
    \item[{Used by}]
  
    \item[{Content model}]
  \mbox{}\hfill\\[-10pt]\begin{Verbatim}[fontsize=\small]
<content>
 <alternate>
  <dataRef key="teidata.probability"/>
  <dataRef key="teidata.certainty"/>
 </alternate>
</content>
    
\end{Verbatim}

    \item[{Declaration}]
  \mbox{}\hfill\\[-10pt]\begin{Verbatim}[fontsize=\small]
teidata.probCert = teidata.probability | teidata.certainty
\end{Verbatim}

\end{reflist}  
\begin{reflist}
\item[]\begin{specHead}{TEI.teidata.replacement}{teidata.replacement} defines attribute values which contain a replacement template.\end{specHead} 
    \item[{Module}]
  tei — \hyperref[ST]{The TEI Infrastructure}
    \item[{Used by}]
  
    \item[{Content model}]
  \fbox{\ttfamily <content>\newline
 <textNode/>\newline
</content>\newline
    } 
    \item[{Declaration}]
  \fbox{\ttfamily teidata.replacement = text} 
\end{reflist}  
\begin{reflist}
\item[]\begin{specHead}{TEI.teidata.sex}{teidata.sex} defines the range of attribute values used to identify human or animal sex.\end{specHead} 
    \item[{Module}]
  tei — \hyperref[ST]{The TEI Infrastructure}
    \item[{Used by}]
  Element: \begin{itemize}
\item \hyperref[TEI.person]{person}/@sex
\item \hyperref[TEI.personGrp]{personGrp}/@sex
\item \hyperref[TEI.persona]{persona}/@sex
\item \hyperref[TEI.sex]{sex}/@value
\end{itemize} 
    \item[{Content model}]
  \fbox{\ttfamily <content>\newline
 <dataRef key="teidata.word"/>\newline
</content>\newline
    } 
    \item[{Declaration}]
  \fbox{\ttfamily teidata.sex = teidata.word} 
    \item[{Note}]
  \par
Values for attributes using this datatype may be locally defined by a project, or may refer to an external standard, such as vCard's sex property \url{http://microformats.org/wiki/gender-formats} (in which M indicates male, F female, O other, N none or not applicable, U unknown), or the often used ISO 5218:2004 \textit{Representation of Human Sexes} \url{http://standards.iso.org/ittf/PubliclyAvailableStandards/c036266\textunderscore ISO\textunderscore IEC\textunderscore 5218\textunderscore 2004(E\textunderscore F).zip} (in which 0 indicates unknown; 1 male; 2 female; and 9 not applicable, although the ISO standard is widely considered inadequate); cf. CETH's \textit{Recommendations for Inclusive Data Collection of Trans People} \url{http://transhealth.ucsf.edu/trans?page=lib-data-collection}.
\end{reflist}  
\begin{reflist}
\item[]\begin{specHead}{TEI.teidata.temporal.iso}{teidata.temporal.iso} defines the range of attribute values expressing a temporal expression such as a date, a time, or a combination of them, that conform to the international standard \textit{Data elements and interchange formats – Information interchange – Representation of dates and times}.\end{specHead} 
    \item[{Module}]
  tei — \hyperref[ST]{The TEI Infrastructure}
    \item[{Used by}]
  
    \item[{Content model}]
  \mbox{}\hfill\\[-10pt]\begin{Verbatim}[fontsize=\small]
<content>
 <alternate>
  <dataRef name="date"/>
  <dataRef name="gYear"/>
  <dataRef name="gMonth"/>
  <dataRef name="gDay"/>
  <dataRef name="gYearMonth"/>
  <dataRef name="gMonthDay"/>
  <dataRef name="time"/>
  <dataRef name="dateTime"/>
  <dataRef name="token"
   restriction="[0-9.,DHMPRSTWYZ/:+\-]+"/>
 </alternate>
</content>
    
\end{Verbatim}

    \item[{Declaration}]
  \mbox{}\hfill\\[-10pt]\begin{Verbatim}[fontsize=\small]
teidata.temporal.iso =
   xsd:date
 | xsd:gYear
 | xsd:gMonth
 | xsd:gDay
 | xsd:gYearMonth
 | xsd:gMonthDay
 | xsd:time
 | xsd:dateTime
 | token { pattern = "[0-9.,DHMPRSTWYZ/:+\-]+" }
\end{Verbatim}

    \item[{Note}]
  \par
If it is likely that the value used is to be compared with another, then a time zone indicator should always be included, and only the dateTime representation should be used.\par
For all representations for which ISO 8601 describes both a \textit{basic} and an \textit{extended} format, these Guidelines recommend use of the extended format.\par
While ISO 8601 permits the use of both \texttt{00:00} and \texttt{24:00} to represent midnight, these Guidelines strongly recommend against the use of \texttt{24:00}. 
\end{reflist}  
\begin{reflist}
\item[]\begin{specHead}{TEI.teidata.temporal.w3c}{teidata.temporal.w3c} defines the range of attribute values expressing a temporal expression such as a date, a time, or a combination of them, that conform to the W3C \textit{XML Schema Part 2: Datatypes Second Edition} specification.\end{specHead} 
    \item[{Module}]
  tei — \hyperref[ST]{The TEI Infrastructure}
    \item[{Used by}]
  Element: \begin{itemize}
\item \hyperref[TEI.docDate]{docDate}/@when
\item \hyperref[TEI.when]{when}/@absolute
\end{itemize} 
    \item[{Content model}]
  \mbox{}\hfill\\[-10pt]\begin{Verbatim}[fontsize=\small]
<content>
 <alternate>
  <dataRef name="date"/>
  <dataRef name="gYear"/>
  <dataRef name="gMonth"/>
  <dataRef name="gDay"/>
  <dataRef name="gYearMonth"/>
  <dataRef name="gMonthDay"/>
  <dataRef name="time"/>
  <dataRef name="dateTime"/>
 </alternate>
</content>
    
\end{Verbatim}

    \item[{Declaration}]
  \mbox{}\hfill\\[-10pt]\begin{Verbatim}[fontsize=\small]
teidata.temporal.w3c =
   xsd:date
 | xsd:gYear
 | xsd:gMonth
 | xsd:gDay
 | xsd:gYearMonth
 | xsd:gMonthDay
 | xsd:time
 | xsd:dateTime
\end{Verbatim}

    \item[{Note}]
  \par
If it is likely that the value used is to be compared with another, then a time zone indicator should always be included, and only the dateTime representation should be used.
\end{reflist}  
\begin{reflist}
\item[]\begin{specHead}{TEI.teidata.temporal.working}{teidata.temporal.working} defines the range of values, conforming to the W3C \textit{XML Schema Part 2: Datatypes Second Edition} specification, expressing a date or a date and a time within the working life of the document.\end{specHead} 
    \item[{Module}]
  tei — \hyperref[ST]{The TEI Infrastructure}
    \item[{Used by}]
  
    \item[{Content model}]
  \mbox{}\hfill\\[-10pt]\begin{Verbatim}[fontsize=\small]
<content>
 <alternate>
  <dataRef name="date"
   restriction="(19[789][0-9]|[2-9][0-9]{3}).*"/>
  <dataRef name="dateTime"
   restriction="(19[789][0-9]|[2-9][0-9]{3}).*"/>
 </alternate>
</content>
    
\end{Verbatim}

    \item[{Declaration}]
  \mbox{}\hfill\\[-10pt]\begin{Verbatim}[fontsize=\small]
teidata.temporal.working =
   xsd:date { pattern = "(19[789][0-9]|[2-9][0-9]{3}).*" }
 | xsd:dateTime { pattern = "(19[789][0-9]|[2-9][0-9]{3}).*" }
\end{Verbatim}

    \item[{Note}]
  \par
If it is likely that the value used is to be compared with another, then a time zone indicator should always be included, and only the dateTime representation should be used.\par
The earliest time expressable with this datatype is 01 January 1970 (the Unix Epoch), which could be written as either 1970-01-01 or 1970-01-01T00:00:00Z.
\end{reflist}  
\begin{reflist}
\item[]\begin{specHead}{TEI.teidata.text}{teidata.text} defines the range of attribute values used to express some kind of identifying string as a single sequence of Unicode characters possibly including whitespace.\end{specHead} 
    \item[{Module}]
  tei — \hyperref[ST]{The TEI Infrastructure}
    \item[{Used by}]
  Element: \begin{itemize}
\item \hyperref[TEI.distinct]{distinct}/@time
\item \hyperref[TEI.distinct]{distinct}/@space
\item \hyperref[TEI.distinct]{distinct}/@social
\item \hyperref[TEI.refState]{refState}/@delim
\item \hyperref[TEI.remarks]{remarks}/@ident
\item \hyperref[TEI.rendition]{rendition}/@selector
\item \hyperref[TEI.unicodeProp]{unicodeProp}/@value
\item \hyperref[TEI.valItem]{valItem}/@ident
\end{itemize} 
    \item[{Content model}]
  \fbox{\ttfamily <content>\newline
 <dataRef name="string"/>\newline
</content>\newline
    } 
    \item[{Declaration}]
  \fbox{\ttfamily teidata.text = string} 
    \item[{Note}]
  \par
Attributes using this datatype must contain a single ‘token’ in which whitespace and other punctuation characters are permitted.
\end{reflist}  
\begin{reflist}
\item[]\begin{specHead}{TEI.teidata.truthValue}{teidata.truthValue} defines the range of attribute values used to express a truth value.\end{specHead} 
    \item[{Module}]
  tei — \hyperref[ST]{The TEI Infrastructure}
    \item[{Used by}]
  Element: \begin{itemize}
\item \hyperref[TEI.binary]{binary}/@value
\item \hyperref[TEI.content]{content}/@autoPrefix
\item \hyperref[TEI.fDecl]{fDecl}/@optional
\item \hyperref[TEI.listChange]{listChange}/@ordered
\item \hyperref[TEI.metSym]{metSym}/@terminal
\item \hyperref[TEI.model]{model}/@useSourceRendition
\item \hyperref[TEI.modelGrp]{modelGrp}/@useSourceRendition
\item \hyperref[TEI.modelSequence]{modelSequence}/@useSourceRendition
\item \hyperref[TEI.numeric]{numeric}/@trunc
\item \hyperref[TEI.pc]{pc}/@pre
\item \hyperref[TEI.sequence]{sequence}/@preserveOrder
\item \hyperref[TEI.surface]{surface}/@flipping
\item \hyperref[TEI.tagsDecl]{tagsDecl}/@partial
\end{itemize} 
    \item[{Content model}]
  \fbox{\ttfamily <content>\newline
 <dataRef name="boolean"/>\newline
</content>\newline
    } 
    \item[{Declaration}]
  \fbox{\ttfamily teidata.truthValue = xsd:boolean} 
    \item[{Note}]
  \par
The possible values of this datatype are 1 or true, or 0 or false.\par
This datatype applies only for cases where uncertainty is inappropriate; if the attribute concerned may have a value other than true or false, e.g. unknown, or inapplicable, it should have the extended version of this datatype: \textsf{teidata.xTruthValue}.
\end{reflist}  
\begin{reflist}
\item[]\begin{specHead}{TEI.teidata.unboundedInt}{teidata.unboundedInt} defines an attribute value which can be either any non-negative integer or the string "unbounded".\end{specHead} 
    \item[{Module}]
  tei — \hyperref[ST]{The TEI Infrastructure}
    \item[{Used by}]
  Element: \begin{itemize}
\item \hyperref[TEI.datatype]{datatype}/@maxOccurs
\end{itemize} 
    \item[{Content model}]
  \mbox{}\hfill\\[-10pt]\begin{Verbatim}[fontsize=\small]
<content>  <alternate>
  <dataRef name="nonNegativeInteger"/>
  <valList type="closed">
   <valItem ident="unbounded"/>
  </valList>
 </alternate>
</content>
    
\end{Verbatim}

    \item[{Declaration}]
  \mbox{}\hfill\\[-10pt]\begin{Verbatim}[fontsize=\small]
teidata.unboundedInt = xsd:nonNegativeInteger | ( "unbounded" )
\end{Verbatim}

\end{reflist}  
\begin{reflist}
\item[]\begin{specHead}{TEI.teidata.version}{teidata.version} defines the range of attribute values which may be used to specify a TEI or Unicode version number.\end{specHead} 
    \item[{Module}]
  tei — \hyperref[ST]{The TEI Infrastructure}
    \item[{Used by}]
  Element: \begin{itemize}
\item \hyperref[TEI.TEI]{TEI}/@version
\item \hyperref[TEI.teiCorpus]{teiCorpus}/@version
\item \hyperref[TEI.unicodeName]{unicodeName}/@version
\end{itemize} 
    \item[{Content model}]
  \mbox{}\hfill\\[-10pt]\begin{Verbatim}[fontsize=\small]
<content>
 <dataRef name="token"
  restriction="[\d]+(\.[\d]+){0,2}"/>
</content>
    
\end{Verbatim}

    \item[{Declaration}]
  \mbox{}\hfill\\[-10pt]\begin{Verbatim}[fontsize=\small]
teidata.version = token { pattern = "[\d]+(\.[\d]+){0,2}" }
\end{Verbatim}

    \item[{Note}]
  \par
The value of this attribute follows the pattern specified by the Unicode consortium for its version number (\url{http://unicode.org/versions/}). A version number contains digits and fullstop characters only. The first number supplied identifies the major version number. A second and third number, for minor and sub-minor version numbers, may also be supplied.
\end{reflist}  
\begin{reflist}
\item[]\begin{specHead}{TEI.teidata.versionNumber}{teidata.versionNumber} defines the range of attribute values used for version numbers.\end{specHead} 
    \item[{Module}]
  tei — \hyperref[ST]{The TEI Infrastructure}
    \item[{Used by}]
  Element: \begin{itemize}
\item \hyperref[TEI.application]{application}/@version
\item \hyperref[TEI.transcriptionDesc]{transcriptionDesc}/@version
\end{itemize} 
    \item[{Content model}]
  \mbox{}\hfill\\[-10pt]\begin{Verbatim}[fontsize=\small]
<content>
 <dataRef name="token"
  restriction="[\d]+[a-z]*[\d]*(\.[\d]+[a-z]*[\d]*){0,3}"/>
</content>
    
\end{Verbatim}

    \item[{Declaration}]
  \mbox{}\hfill\\[-10pt]\begin{Verbatim}[fontsize=\small]
teidata.versionNumber =
   token { pattern = "[\d]+[a-z]*[\d]*(\.[\d]+[a-z]*[\d]*){0,3}" }
\end{Verbatim}

\end{reflist}  
\begin{reflist}
\item[]\begin{specHead}{TEI.teidata.word}{teidata.word} defines the range of attribute values expressed as a single word or token.\end{specHead} 
    \item[{Module}]
  tei — \hyperref[ST]{The TEI Infrastructure}
    \item[{Used by}]
  \hyperref[TEI.teidata.enumerated]{teidata.enumerated} \hyperref[TEI.teidata.sex]{teidata.sex}Element: \begin{itemize}
\item \hyperref[TEI.app]{app}/@loc
\item \hyperref[TEI.attRef]{attRef}/@class
\item \hyperref[TEI.binaryObject]{binaryObject}/@encoding
\item \hyperref[TEI.certainty]{certainty}/@assertedValue
\item \hyperref[TEI.code]{code}/@lang
\item \hyperref[TEI.dataFacet]{dataFacet}/@name
\item \hyperref[TEI.langKnown]{langKnown}/@level
\item \hyperref[TEI.locus]{locus}/@from
\item \hyperref[TEI.locus]{locus}/@to
\item \hyperref[TEI.m]{m}/@baseForm
\item \hyperref[TEI.media]{media}/@mimeType
\item \hyperref[TEI.metSym]{metSym}/@value
\item \hyperref[TEI.metamark]{metamark}/@function
\item \hyperref[TEI.personGrp]{personGrp}/@size
\item \hyperref[TEI.rhyme]{rhyme}/@label
\item \hyperref[TEI.secl]{secl}/@reason
\item \hyperref[TEI.supplied]{supplied}/@reason
\item \hyperref[TEI.surplus]{surplus}/@reason
\item \hyperref[TEI.symbol]{symbol}/@value
\item \hyperref[TEI.unihanProp]{unihanProp}/@value
\item \hyperref[TEI.vLabel]{vLabel}/@name
\end{itemize} 
    \item[{Content model}]
  \mbox{}\hfill\\[-10pt]\begin{Verbatim}[fontsize=\small]
<content>
 <dataRef name="token"
  restriction="[^\p{C}\p{Z}]+"/>
</content>
    
\end{Verbatim}

    \item[{Declaration}]
  \fbox{\ttfamily teidata.word = token ❴ pattern = "[\textasciicircum ⃥p❴C❵⃥p❴Z❵]+" ❵} 
    \item[{Note}]
  \par
Attributes using this datatype must contain a single ‘word’ which contains only letters, digits, punctuation characters, or symbols: thus it cannot include whitespace.
\end{reflist}  
\begin{reflist}
\item[]\begin{specHead}{TEI.teidata.xmlName}{teidata.xmlName} defines attribute values which contain an XML name.\end{specHead} 
    \item[{Module}]
  tei — \hyperref[ST]{The TEI Infrastructure}
    \item[{Used by}]
  Element: \begin{itemize}
\item \hyperref[TEI.classRef]{classRef}/@key
\item \hyperref[TEI.classRef]{classRef}/@include
\item \hyperref[TEI.classRef]{classRef}/@except
\item \hyperref[TEI.dataRef]{dataRef}/@key
\item \hyperref[TEI.dataRef]{dataRef}/@name
\item \hyperref[TEI.elementRef]{elementRef}/@key
\item \hyperref[TEI.elementSpec]{elementSpec}/@prefix
\item \hyperref[TEI.macroRef]{macroRef}/@key
\item \hyperref[TEI.moduleRef]{moduleRef}/@prefix
\item \hyperref[TEI.moduleRef]{moduleRef}/@include
\item \hyperref[TEI.moduleRef]{moduleRef}/@except
\item \hyperref[TEI.moduleRef]{moduleRef}/@key
\item \hyperref[TEI.schemaRef]{schemaRef}/@key
\item \hyperref[TEI.schemaSpec]{schemaSpec}/@prefix
\item \hyperref[TEI.unicodeProp]{unicodeProp}/@name
\item \hyperref[TEI.unihanProp]{unihanProp}/@name
\end{itemize} 
    \item[{Content model}]
  \fbox{\ttfamily <content>\newline
 <dataRef name="NCName"/>\newline
</content>\newline
    } 
    \item[{Declaration}]
  \fbox{\ttfamily teidata.xmlName = xsd:NCName} 
    \item[{Note}]
  \par
The rules defining an XML name form a part of the XML Specification.
\end{reflist}  
\begin{reflist}
\item[]\begin{specHead}{TEI.teidata.xpath}{teidata.xpath} defines attribute values which contain an XPath expression.\end{specHead} 
    \item[{Module}]
  tei — \hyperref[ST]{The TEI Infrastructure}
    \item[{Used by}]
  Element: \begin{itemize}
\item \hyperref[TEI.citeStructure]{citeStructure}/@match
\item \hyperref[TEI.modelSequence]{modelSequence}/@predicate
\item \hyperref[TEI.param]{param}/@value
\end{itemize} 
    \item[{Content model}]
  \fbox{\ttfamily <content>\newline
 <textNode/>\newline
</content>\newline
    } 
    \item[{Declaration}]
  \fbox{\ttfamily teidata.xpath = text} 
    \item[{Note}]
  \par
Any XPath expression using the syntax defined in \cite{XSLT2}.\par
When writing programs that evaluate XPath expressions, programmers should be mindful of the possibility of malicious code injection attacks. For further information about XPath injection attacks, see the \xref{https://owasp.org/www-community/attacks/XPATH_Injection}{article at OWASP}.
\end{reflist}  
\begin{reflist}
\item[]\begin{specHead}{TEI.teidata.xTruthValue}{teidata.xTruthValue} (extended truth value) defines the range of attribute values used to express a truth value which may be unknown.\end{specHead} 
    \item[{Module}]
  tei — \hyperref[ST]{The TEI Infrastructure}
    \item[{Used by}]
  Element: \begin{itemize}
\item \hyperref[TEI.binding]{binding}/@contemporary
\item \hyperref[TEI.iNode]{iNode}/@ord
\item \hyperref[TEI.kinesic]{kinesic}/@iterated
\item \hyperref[TEI.root]{root}/@ord
\item \hyperref[TEI.said]{said}/@aloud
\item \hyperref[TEI.said]{said}/@direct
\item \hyperref[TEI.seal]{seal}/@contemporary
\item \hyperref[TEI.sound]{sound}/@discrete
\item \hyperref[TEI.vocal]{vocal}/@iterated
\item \hyperref[TEI.writing]{writing}/@gradual
\end{itemize} 
    \item[{Content model}]
  \mbox{}\hfill\\[-10pt]\begin{Verbatim}[fontsize=\small]
<content>
 <alternate>
  <dataRef name="boolean"/>
  <valList>
   <valItem ident="unknown"/>
   <valItem ident="inapplicable"/>
  </valList>
 </alternate>
</content>
    
\end{Verbatim}

    \item[{Declaration}]
  \mbox{}\hfill\\[-10pt]\begin{Verbatim}[fontsize=\small]
teidata.xTruthValue = xsd:boolean | ( "unknown" | "inapplicable" )
\end{Verbatim}

    \item[{Note}]
  \par
In cases where where uncertainty is inappropriate, use the datatype \textsf{teidata.TruthValue}.
\end{reflist}  