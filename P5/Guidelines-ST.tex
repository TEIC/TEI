
\section[{The TEI Infrastructure}]{The TEI Infrastructure}\label{ST}\par
This chapter describes the infrastructure for the encoding scheme defined by these Guidelines. It introduces the conceptual framework within which the following chapters are to be understood, and the means by which that conceptual framework is implemented. It assumes some familiarity with XML and XML schemas (see chapter \textit{\hyperref[SG]{v\ A Gentle Introduction to XML}}) but is intended to be accessible to any user of these Guidelines. Other chapters supply further technical details, in particular chapter \textit{\hyperref[TD]{22.\ Documentation Elements}} which describes the XML schema used to express these Guidelines themselves, and chapter \textit{\hyperref[USE]{23.\ Using the TEI}} which combines a discussion of modification and conformance issues with a description of the intended behaviour of an ODD processor; these chapters should be read by anyone intending to implement a new TEI-based system.\par
The TEI encoding scheme consists of a number of \textit{modules}, each of which declares particular XML elements and their attributes. Part of an element's declaration includes its assignment to one or more element \textit{classes}. Another part defines its possible content and attributes with reference to these classes. This indirection gives the TEI system much of its strength and its flexibility. Elements may be combined more or less freely to form a \textit{schema} appropriate to a particular set of requirements. It is also easy to add new elements which reference existing classes or elements to a schema, as it is to exclude some of the elements provided by any module included in a schema.\par
In principle, a TEI schema may be constructed using any combination of modules. However, certain TEI modules are of particular importance, and should always be included in all but exceptional circumstances: the module \textsf{tei} described in the present chapter is of this kind because it defines classes, macros, and datatypes which are used by all other modules. The \textsf{core} module, defined in chapter \textit{\hyperref[CO]{3.\ Elements Available in All TEI Documents}} contains declarations for elements and attributes which are likely to be needed in almost any kind of document, and is therefore recommended for global use. The \textsf{header} module defined in chapter \textit{\hyperref[HD]{2.\ The TEI Header}} provides declarations for the metadata elements and attributes constituting the TEI header, a component which is required for TEI conformance, while the \textsf{textstructure} module defined in chapter \textit{\hyperref[DS]{4.\ Default Text Structure}} declares basic structural elements needed for the encoding of most book-like objects. Most schemas will therefore need to include these four modules.\par
The specification for a TEI schema is itself a TEI document, using elements from the module described in chapter \textit{\hyperref[TD]{22.\ Documentation Elements}}: we refer to such a document informally as an \textit{ODD} document, from the design goal originally formulated for the system: ‘One Document Does it all’. Stylesheets for maintaining and processing ODD documents are maintained by the TEI, and these Guidelines are also maintained as such a document. As further discussed in \textit{\hyperref[IM]{23.5.\ Implementation of an ODD System}}, an ODD document can be processed to generate a schema expressed using any of the three schema languages currently in wide use: the XML DTD language, the ISO RELAX NG language, or the W3C Schema language, as well as to generate documentation such as the \textit{Guidelines} and their associated web site.\par
The bulk of this chapter describes the TEI infrastructure module itself. Although it may be skipped at a first reading, an understanding of the topics addressed here is essential for anyone planning to take full advantage of the TEI customization techniques described in chapter \textit{\hyperref[MD]{23.3.\ Customization}}.\par
The chapter begins by briefly characterizing each of the modules available in the TEI scheme. Section \textit{\hyperref[STIN]{1.2.\ Defining a TEI Schema}} describes in general terms the method of constructing a TEI schema in a specific schema language such as XML DTD language, RELAX NG, or W3C Schema.\par
The next and largest part of the chapter introduces the attribute and element classes used to define groups of elements and their characteristics (section \textit{\hyperref[STEC]{1.3.\ The TEI Class System}}).\par
Finally, section \textit{\hyperref[STmacros]{1.4.\ Macros}} introduces the concept of \textit{macros}, which are used to express some commonly used content models, and lists the \textit{datatypes} used to constrain the range of legal values for TEI attributes (section \textit{\hyperref[DTYPES]{1.4.2.\ Datatype Specifications}}).
\subsection[{TEI Modules}]{TEI Modules}\label{STMA}\par
These Guidelines define several hundred elements and attributes for marking up documents of any kind. Each definition has the following components: \begin{itemize}
\item a prose description
\item a formal declaration, expressed using a special-purpose XML vocabulary defined by these Guidelines in combination with elements taken from the ISO schema language RELAX NG
\item usage examples
\end{itemize} \par
Each chapter of these Guidelines presents a group of related elements, and also defines a corresponding set of declarations, which we call a \textit{module}. All the definitions are collected together in the reference sections provided as an appendix. Formal declarations for a given chapter are collected together within the corresponding module. For convenience, each element is assigned to a single module, typically for use in some specific application area, or to support a particular kind of usage. A module is thus simply a convenient way of grouping together a number of associated element declarations. In the simple case, a TEI schema is made by combining together a small number of modules, as further described in section \textit{\hyperref[STIN]{1.2.\ Defining a TEI Schema}} below.\par
The following table lists the modules defined by the current release of these Guidelines:  \label{tab-mods} \par 
\begin{longtable}{L{.15\textwidth}P{.4\textwidth}L{.35\textwidth}}
\rowcolor{label}Module name\tabcellsep Formal public identifier\tabcellsep Where defined\\\hline 
analysis\tabcellsep Analysis and Interpretation\tabcellsep \textit{\hyperref[AI]{17.\ Simple Analytic Mechanisms}}\\
certainty\tabcellsep Certainty and Uncertainty\tabcellsep \textit{\hyperref[CE]{21.\ Certainty, Precision, and Responsibility}}\\
core\tabcellsep Common Core\tabcellsep \textit{\hyperref[CO]{3.\ Elements Available in All TEI Documents}}\\
corpus\tabcellsep Metadata for Language Corpora\tabcellsep \textit{\hyperref[CC]{15.\ Language Corpora}}\\
dictionaries\tabcellsep Print Dictionaries\tabcellsep \textit{\hyperref[DI]{9.\ Dictionaries}}\\
drama\tabcellsep Performance Texts\tabcellsep \textit{\hyperref[DR]{7.\ Performance Texts}}\\
figures\tabcellsep Tables, Formulae, Figures\tabcellsep \textit{\hyperref[FT]{14.\ Tables, Formulæ, Graphics and Notated Music}}\\
gaiji\tabcellsep Character and Glyph Documentation\tabcellsep \textit{\hyperref[WD]{5.\ Characters, Glyphs, and Writing Modes}}\\
header\tabcellsep Common Metadata\tabcellsep \textit{\hyperref[HD]{2.\ The TEI Header}}\\
iso-fs\tabcellsep Feature Structures\tabcellsep \textit{\hyperref[FS]{18.\ Feature Structures}}\\
linking\tabcellsep Linking, Segmentation, and Alignment\tabcellsep \textit{\hyperref[SA]{16.\ Linking, Segmentation, and Alignment}}\\
msdescription\tabcellsep Manuscript Description\tabcellsep \textit{\hyperref[MS]{10.\ Manuscript Description}}\\
namesdates\tabcellsep Names, Dates, People, and Places\tabcellsep \textit{\hyperref[ND]{13.\ Names, Dates, People, and Places}}\\
nets\tabcellsep Graphs, Networks, and Trees\tabcellsep \textit{\hyperref[GD]{19.\ Graphs, Networks, and Trees}}\\
spoken\tabcellsep Transcribed Speech\tabcellsep \textit{\hyperref[TS]{8.\ Transcriptions of Speech}}\\
tagdocs\tabcellsep Documentation Elements\tabcellsep \textit{\hyperref[TD]{22.\ Documentation Elements}}\\
tei\tabcellsep TEI Infrastructure\tabcellsep \textit{\hyperref[ST]{1.\ The TEI Infrastructure}}\\
textcrit\tabcellsep Text Criticism\tabcellsep \textit{\hyperref[TC]{12.\ Critical Apparatus}}\\
textstructure\tabcellsep Default Text Structure\tabcellsep \textit{\hyperref[DS]{4.\ Default Text Structure}}\\
transcr\tabcellsep Transcription of Primary Sources\tabcellsep \textit{\hyperref[PH]{11.\ Representation of Primary Sources}}\\
verse\tabcellsep Verse\tabcellsep \textit{\hyperref[VE]{6.\ Verse}}\end{longtable} \par
 \par
For each module listed above, the corresponding chapter gives a full description of the classes, elements, and macros which it makes available when it is included in a schema. Other chapters of these Guidelines explore other aspects of using the TEI scheme.
\subsection[{Defining a TEI Schema}]{Defining a TEI Schema}\label{STIN}\par
To determine that an XML document is valid (as opposed to merely well-formed), its structure must be checked against a schema, as discussed in chapter \textit{\hyperref[SG]{v\ A Gentle Introduction to XML}}. For a valid TEI document, this schema must be a conformant TEI schema, as further defined in chapter \textit{\hyperref[CF]{23.4.\ Conformance}}. Local systems may allow their schema to be implicit, but for interchange purposes the schema associated with a document \textit{must} be made explicit. The method of doing this recommended by these Guidelines is to provide explicitly or by reference a TEI schema specification against which the document may be validated.\par
A TEI-conformant schema is a specific combination of TEI modules, possibly also including additional declarations that modify the element and attribute declarations contained by each module, for example to suppress or rename some elements. The TEI provides an application-independent way of specifying a TEI schema by means of the \hyperref[TEI.schemaSpec]{<schemaSpec>} element defined in chapter \textit{\hyperref[TD]{22.\ Documentation Elements}}. The same system may also be used to specify a schema which extends the TEI by adding new elements explicitly, or by reference to other XML vocabularies. In either case, the specification may be processed to generate a formal schema, expressed in a variety of specific schema languages, such as XML DTD language, RELAX NG, or W3C Schema. These output schemas can then be used by an XML processor such as a validator or editor to validate or otherwise process documents. Further information about the processing of a TEI formal specification is given in chapter \textit{\hyperref[USE]{23.\ Using the TEI}}.
\subsubsection[{A Simple Customization}]{A Simple Customization}\label{STINsimpleExample}\par
The simplest customization of the TEI scheme combines just the four recommended modules mentioned above. In ODD format, this schema specification takes this form: \par\bgroup\index{schemaSpec=<schemaSpec>|exampleindex}\index{ident=@ident!<schemaSpec>|exampleindex}\index{start=@start!<schemaSpec>|exampleindex}\index{moduleRef=<moduleRef>|exampleindex}\index{key=@key!<moduleRef>|exampleindex}\index{moduleRef=<moduleRef>|exampleindex}\index{key=@key!<moduleRef>|exampleindex}\index{moduleRef=<moduleRef>|exampleindex}\index{key=@key!<moduleRef>|exampleindex}\index{moduleRef=<moduleRef>|exampleindex}\index{key=@key!<moduleRef>|exampleindex}\exampleFont \begin{shaded}\noindent\mbox{}{<\textbf{schemaSpec}\hspace*{1em}{ident}="{TEI-minimal}"\hspace*{1em}{start}="{TEI}">}\mbox{}\newline 
\hspace*{1em}{<\textbf{moduleRef}\hspace*{1em}{key}="{tei}"/>}\mbox{}\newline 
\hspace*{1em}{<\textbf{moduleRef}\hspace*{1em}{key}="{header}"/>}\mbox{}\newline 
\hspace*{1em}{<\textbf{moduleRef}\hspace*{1em}{key}="{core}"/>}\mbox{}\newline 
\hspace*{1em}{<\textbf{moduleRef}\hspace*{1em}{key}="{textstructure}"/>}\mbox{}\newline 
{</\textbf{schemaSpec}>}\end{shaded}\egroup\par \par
This schema specification contains references to each of four modules, identified by the {\itshape key} attribute on the \hyperref[TEI.moduleRef]{<moduleRef>} element. The schema specification itself is also given an identifier (\textsf{TEI-minimal}). An ODD processor will generate an appropriate schema from this set of declarations, expressed using the XML DTD language, the ISO RELAX NG language, the W3C Schema language, or in principle any other adequately powerful schema language. The resulting schema may then be associated with the document instance by one of a number of different mechanisms, as further described in chapter \textit{\hyperref[SG]{v\ A Gentle Introduction to XML}}. The start point (or root element) of document instances to be validated against the schema is specified by means of the {\itshape start} attribute. Further information about the processing of an ODD specification is given in \textit{\hyperref[IM]{23.5.\ Implementation of an ODD System}}.
\subsubsection[{A Larger Customization}]{A Larger Customization}\label{STINlargerExample}\par
These Guidelines introduce each of the modules making up the TEI scheme one by one, and therefore, for clarity of exposition, each chapter focusses on elements drawn from a single module. In reality, of course, the markup of a text will draw on elements taken from many different modules, partly because texts are heterogeneous objects, and partly because encoders have different goals. Some examples of this heterogeneity include: \begin{itemize}
\item a text may be a collection of other texts of different types: for example, an anthology of prose, verse, and drama;
\item a text may contain other smaller, embedded texts: for example, a poem or song included in a prose narrative;
\item some sections of a text may be written in one form, and others in a different form: for example, a novel where some chapters are in prose, others take the form of dictionary entries, and still others the form of scenes in a play; 
\item an encoded text may include detailed analytic annotation, for example of rhetorical or linguistic features; 
\item an encoded text may combine a literal transcription with a diplomatic edition of the same or different sources; 
\item the description of a text may require additional specialized metadata elements, for example when describing manuscript material in detail.
\end{itemize} \par
The TEI provides mechanisms to support all of these and many other use cases. The architecture permits elements and attributes from any combination of modules to co-exist within a single schema. Within particular modules, elements and attributes are provided to support differing views of the ‘granularity’ of a text, for example: \begin{itemize}
\item a definition of a corpus or collection as a series of \hyperref[TEI.TEI]{<TEI>} documents, sharing a common TEI header (see chapter \textit{\hyperref[CC]{15.\ Language Corpora}})
\item a definition of composite texts which combine optional front- and back-matter with a group of collected texts, themselves possibly composite (see section \textit{\hyperref[DSGRP]{4.3.1.\ Grouped Texts}})
\item an element for the representation of \textit{embedded texts}, where one narrative appears to ‘float’ within another (see section \textit{\hyperref[DSFLT]{4.3.2.\ Floating Texts}})
\end{itemize} \par
Subsequent chapters of these Guidelines describe in detail markup constructs appropriate for these and many other possible features of interest. The markup constructs can be combined as needed for any given set of applications or project.\par
For example, a project aiming to produce an ambitious digital edition of a collection of manuscript materials, to include detailed metadata about each source, digital images of the content, along with a detailed transcription of each source, and a supporting biographical and geographical database might need a schema combining several modules, as follows: \par\bgroup\index{schemaSpec=<schemaSpec>|exampleindex}\index{ident=@ident!<schemaSpec>|exampleindex}\index{start=@start!<schemaSpec>|exampleindex}\index{moduleRef=<moduleRef>|exampleindex}\index{key=@key!<moduleRef>|exampleindex}\index{moduleRef=<moduleRef>|exampleindex}\index{key=@key!<moduleRef>|exampleindex}\index{moduleRef=<moduleRef>|exampleindex}\index{key=@key!<moduleRef>|exampleindex}\index{moduleRef=<moduleRef>|exampleindex}\index{key=@key!<moduleRef>|exampleindex}\index{moduleRef=<moduleRef>|exampleindex}\index{key=@key!<moduleRef>|exampleindex}\index{moduleRef=<moduleRef>|exampleindex}\index{key=@key!<moduleRef>|exampleindex}\index{moduleRef=<moduleRef>|exampleindex}\index{key=@key!<moduleRef>|exampleindex}\index{moduleRef=<moduleRef>|exampleindex}\index{key=@key!<moduleRef>|exampleindex}\exampleFont \begin{shaded}\noindent\mbox{}{<\textbf{schemaSpec}\hspace*{1em}{ident}="{TEI-PROJECT}"\hspace*{1em}{start}="{TEI}">}\mbox{}\newline 
\hspace*{1em}{<\textbf{moduleRef}\hspace*{1em}{key}="{tei}"/>}\mbox{}\newline 
\hspace*{1em}{<\textbf{moduleRef}\hspace*{1em}{key}="{header}"/>}\mbox{}\newline 
\hspace*{1em}{<\textbf{moduleRef}\hspace*{1em}{key}="{core}"/>}\mbox{}\newline 
\hspace*{1em}{<\textbf{moduleRef}\hspace*{1em}{key}="{textstructure}"/>}\mbox{}\newline 
\hspace*{1em}{<\textbf{moduleRef}\hspace*{1em}{key}="{msdescription}"/>}\mbox{}\newline 
\textit{<!-- manuscript description -->}\mbox{}\newline 
\hspace*{1em}{<\textbf{moduleRef}\hspace*{1em}{key}="{transcr}"/>}\mbox{}\newline 
\textit{<!-- transcription of primary sources -->}\mbox{}\newline 
\hspace*{1em}{<\textbf{moduleRef}\hspace*{1em}{key}="{figures}"/>}\mbox{}\newline 
\textit{<!-- figures and tables -->}\mbox{}\newline 
\hspace*{1em}{<\textbf{moduleRef}\hspace*{1em}{key}="{namesdates}"/>}\mbox{}\newline 
\textit{<!-- names, dates, people, and places -->}\mbox{}\newline 
{</\textbf{schemaSpec}>}\end{shaded}\egroup\par \par
Alternatively, a simpler schema might be used for a part of such a project: those preparing the transcriptions, for example, might need only elements from the \textsf{core}, \textsf{textstructure}, and \textsf{transcr} modules, and might therefore prefer to use a simpler schema such as that generated by the following: \par\bgroup\index{schemaSpec=<schemaSpec>|exampleindex}\index{ident=@ident!<schemaSpec>|exampleindex}\index{start=@start!<schemaSpec>|exampleindex}\index{moduleRef=<moduleRef>|exampleindex}\index{key=@key!<moduleRef>|exampleindex}\index{moduleRef=<moduleRef>|exampleindex}\index{key=@key!<moduleRef>|exampleindex}\index{moduleRef=<moduleRef>|exampleindex}\index{key=@key!<moduleRef>|exampleindex}\index{moduleRef=<moduleRef>|exampleindex}\index{key=@key!<moduleRef>|exampleindex}\exampleFont \begin{shaded}\noindent\mbox{}{<\textbf{schemaSpec}\hspace*{1em}{ident}="{TEI-TRANSCR}"\hspace*{1em}{start}="{TEI}">}\mbox{}\newline 
\hspace*{1em}{<\textbf{moduleRef}\hspace*{1em}{key}="{tei}"/>}\mbox{}\newline 
\hspace*{1em}{<\textbf{moduleRef}\hspace*{1em}{key}="{core}"/>}\mbox{}\newline 
\hspace*{1em}{<\textbf{moduleRef}\hspace*{1em}{key}="{textstructure}"/>}\mbox{}\newline 
\hspace*{1em}{<\textbf{moduleRef}\hspace*{1em}{key}="{transcr}"/>}\mbox{}\newline 
{</\textbf{schemaSpec}>}\end{shaded}\egroup\par \par
The TEI architecture also supports more detailed customization beyond the simple selection of modules. A schema may suppress elements from a module, suppress some of their attributes, change their names, or even add new elements and attributes. Detailed discussion of the kind of modification possible in this way is provided in \textit{\hyperref[MD]{23.3.\ Customization}} and conformance rules relating to their application are discussed in \textit{\hyperref[CF]{23.4.\ Conformance}}. These facilities are available for any schema language (though some features may not be available in all languages). The ODD language also makes it possible to combine TEI and non-TEI modules into a single schema, provided that the non-TEI module is expressed using the RELAX NG schema language (see further \textit{\hyperref[ST-aliens]{22.8.2.\ Combining TEI and Non-TEI Modules}}).
\subsection[{The TEI Class System}]{The TEI Class System}\label{STEC}\par
The TEI scheme distinguishes about five hundred different elements. To aid comprehension, modularity, and modification, the majority of these elements are formally classified in some way. Classes are used to express two distinct kinds of commonality among elements. The elements of a class may share some set of attributes, or they may appear in the same locations in a content model. A class is known as an \textit{attribute class} if its members share attributes, and as a \textit{model class} if its members appear in the same locations. In either case, an element is said to \textit{inherit} properties from any classes of which it is a member.\par
Classes (and therefore elements which are members of those classes) may also inherit properties from other classes. For example, supposing that class A is a member (or a \textit{subclass}) of class B, any element which is a member of class A will inherit not only the properties defined by class A, but also those defined by class B. In such a situation, we also say that class B is a \textit{superclass} of class A. The properties of a superclass are inherited by all members of its subclasses.\par
A basic understanding of the classes into which the TEI scheme is organized is strongly recommended and is essential for any successful customization of the system.
\subsubsection[{Attribute Classes}]{Attribute Classes}\label{STECAT}\par
An attribute class groups together elements which share some set of common attributes. Attribute classes are given names composed of the prefix \texttt{att.}, often followed by an adjective. For example, the members of the class \textsf{att.canonical} have in common a {\itshape key} and a {\itshape ref} attribute, both of which are inherited from their membership in the class rather than individually defined for each element. These attributes are said to be defined by (or inherited from) the \textsf{att.canonical} class. If another element were to be added to the TEI scheme for which these attributes were considered useful, the simplest way to provide them would be to make the new element a member of the \textsf{att.canonical} class. Note also that this method ensures that the attributes in question are always defined in the same way, taking the same default values etc., no matter which element they are attached to.\par
Some attribute classes are defined within the \textsf{tei} infrastructural module and are thus globally available. Other attribute classes are specific to particular modules and thus defined in other chapters. Attributes defined by such classes will not be available unless the module concerned is included in a schema.\par
The attributes provided by an attribute class are those specified by the class itself, either directly, or by inheritance from another class. For example, the attribute class \textsf{att.pointing.group} provides attributes {\itshape domains} and {\itshape targFunc} to all of its members. This class is however a subclass of the \textsf{att.pointing} class, from which its members also inherit the attributes {\itshape target}, {\itshape targetLang} and {\itshape evaluate}. Members of the class \textsf{att.pointing} will thus have these three attributes, while members of the class \textsf{att.pointing.group} will have all five.\par
Note that some modules define superclasses of an existing infrastructural class. For example, the global attribute class \textsf{att.divLike} makes attributes {\itshape org} and {\itshape sample} available, while the \textsf{att.metrical} class, which is specific to the \textsf{verse} module, provides attributes {\itshape met}, {\itshape real}, and {\itshape rhyme}. Because \textsf{att.metrical} is defined as a superclass of \textsf{att.divLike}, all five of these attributes are available to elements; the declaration for \textsf{att.metrical} adds its three attributes to the three already defined by \textsf{att.divLike} when the \textsf{verse} module is included in a schema. If, however, this module is not included in a schema, then the \textsf{att.divLike} class supplies only the two attributes first mentioned.\par
Attributes specific to particular modules are documented along with the relevant module rather than in the present chapter. One particular attribute class, known as \textsf{att.global}, is common to all modules, and is therefore described in some detail in the next section. A full list of all attribute classes is given in \textit{\hyperref[REF-CLASSES-ATTS]{Appendix B\ Attribute Classes}} below.
\paragraph[{Global Attributes}]{Global Attributes}\label{STGA}\par
The following attributes are defined in the infrastructure module for every TEI element. 
\begin{sansreflist}
  
\item [\textbf{att.global}] provides attributes common to all elements in the TEI encoding scheme.\hfil\\[-10pt]\begin{sansreflist}
    \item[@{\itshape xml:id}]
  (identifier) provides a unique identifier for the element bearing the attribute.
    \item[@{\itshape n}]
  (number) gives a number (or other label) for an element, which is not necessarily unique within the document.
    \item[@{\itshape xml:lang}]
  (language) indicates the language of the element content using a ‘tag’ generated according to \xref{http://www.rfc-editor.org/rfc/bcp/bcp47.txt}{BCP 47}.
    \item[@{\itshape rend [att.global.rendition]}]
  (rendition) indicates how the element in question was rendered or presented in the source text.
    \item[@{\itshape style [att.global.rendition]}]
  contains an expression in some formal style definition language which defines the rendering or presentation used for this element in the source text
    \item[@{\itshape rendition [att.global.rendition]}]
  points to a description of the rendering or presentation used for this element in the source text.
    \item[@{\itshape xml:base}]
  provides a base URI reference with which applications can resolve relative URI references into absolute URI references.
    \item[@{\itshape xml:space}]
  signals an intention about how white space should be managed by applications.
    \item[@{\itshape source [att.global.source]}]
  specifies the source from which some aspect of this element is drawn.
    \item[@{\itshape cert [att.global.responsibility]}]
  (certainty) signifies the degree of certainty associated with the intervention or interpretation.
    \item[@{\itshape resp [att.global.responsibility]}]
  (responsible party) indicates the agency responsible for the intervention or interpretation, for example an editor or transcriber.
\end{sansreflist}  
\end{sansreflist}
\par
Some of these attributes (specifically {\itshape xml:id}, {\itshape n}, {\itshape xml:lang}, {\itshape xml:base} and {\itshape xml:space}) are provided by the \textsf{att.global} attribute class itself. The others are provided by one its subclasses \textsf{att.global.rendition}, \textsf{att.global.responsibility}, or \textsf{att.global.source}. Their usage is discussed in the following subsections.\par
Several other globally-available attributes are defined by other subclasses of the \textsf{att.global} class. These are provided by other modules, and are therefore discussed in the chapter discussing that module. A brief summary table is provided in section \textit{\hyperref[STGAothers]{1.3.1.1.7.\ Other Globally Available Attributes}} below.
\subparagraph[{Element Identifiers and Labels}]{Element Identifiers and Labels}\label{STGAid}\par
The value supplied for the {\itshape xml:id} attribute must be a legal \textit{name}, as defined in the World Wide Web Consortium's \xref{http://www.w3.org/TR/REC-xml/}{XML Recommendation}. This means that it must begin with a letter, or the underscore character (‘\textunderscore ’), and contain no characters other than letters, digits, hyphens, underscores, full stops, and certain combining and extension characters.\footnote{The colon is also by default a valid name character; however, it has a specific purpose in XML (to indicate namespace prefixes), and may not therefore be used in any other way within a name.}\par
In XML names (and thus the values of {\itshape xml:id} in an XML TEI document) uppercase and lowercase letters are distinguished, and thus \textit{partTime} and \textit{parttime} are two distinctly different names, and could (though perhaps unwisely) be used to denote two different element occurrences.\par
If two elements are given the same identifier, a validating XML parser will signal a syntax error. The following example, therefore, is \textit{not} valid: \par\bgroup\exampleFont \begin{shaded}\noindent\mbox{}\newline
<p xml:id="PAGE1"><q>What's it going to be then, eh?</q></p> \newline
<p xml:id="PAGE1">There was me, that is Alex, and my three droogs, that is Pete,\newline
Georgie, and Dim, ... </p>\end{shaded}\egroup\par \par
For a discussion of methods of providing unique identifiers for elements, see section \textit{\hyperref[CORS2]{3.11.2.\ Creating New Reference Systems}}.\par
The {\itshape n} attribute also provides an identifying name or number for an element, but in this case the information need not be a legal {\itshape xml:id} value. Its value may be any string of characters; typically it is a number or other similar enumerator or label. For example, the numbers given to the items of a numbered list may be recorded with the {\itshape n} attribute; this would make it possible to record errors in the numeration of the original, as in this list of chapters, transcribed from a faulty original in which the number 10 is used twice, and 11 is omitted: \par\bgroup\index{list=<list>|exampleindex}\index{rend=@rend!<list>|exampleindex}\index{item=<item>|exampleindex}\index{n=@n!<item>|exampleindex}\index{item=<item>|exampleindex}\index{n=@n!<item>|exampleindex}\index{item=<item>|exampleindex}\index{n=@n!<item>|exampleindex}\index{item=<item>|exampleindex}\index{n=@n!<item>|exampleindex}\index{item=<item>|exampleindex}\index{n=@n!<item>|exampleindex}\index{item=<item>|exampleindex}\index{n=@n!<item>|exampleindex}\exampleFont \begin{shaded}\noindent\mbox{}{<\textbf{list}\hspace*{1em}{rend}="{numbered}">}\mbox{}\newline 
\hspace*{1em}{<\textbf{item}\hspace*{1em}{n}="{1}">}About These Guidelines{</\textbf{item}>}\mbox{}\newline 
\hspace*{1em}{<\textbf{item}\hspace*{1em}{n}="{2}">}A Gentle Introduction to XML{</\textbf{item}>}\mbox{}\newline 
\hspace*{1em}{<\textbf{item}\hspace*{1em}{n}="{9}">}Verse{</\textbf{item}>}\mbox{}\newline 
\hspace*{1em}{<\textbf{item}\hspace*{1em}{n}="{10}">}Drama{</\textbf{item}>}\mbox{}\newline 
\hspace*{1em}{<\textbf{item}\hspace*{1em}{n}="{10}">}Spoken Materials {</\textbf{item}>}\mbox{}\newline 
\hspace*{1em}{<\textbf{item}\hspace*{1em}{n}="{12}">}Dictionaries{</\textbf{item}>}\mbox{}\newline 
{</\textbf{list}>}\end{shaded}\egroup\par \noindent  The {\itshape n} attribute may also be used to record non-unique names associated with elements in a text, possibly together with a unique identifier as in the following example: \par\bgroup\index{div=<div>|exampleindex}\index{type=@type!<div>|exampleindex}\index{n=@n!<div>|exampleindex}\index{div=<div>|exampleindex}\index{type=@type!<div>|exampleindex}\index{n=@n!<div>|exampleindex}\exampleFont \begin{shaded}\noindent\mbox{}{<\textbf{div}\hspace*{1em}{type}="{book}"\hspace*{1em}{n}="{one}"\hspace*{1em}{xml:id}="{TXT0101}">}\mbox{}\newline 
\textit{<!-- ... -->}\mbox{}\newline 
\hspace*{1em}{<\textbf{div}\hspace*{1em}{type}="{stanza}"\hspace*{1em}{n}="{xlii}">}\mbox{}\newline 
\textit{<!-- ... -->}\mbox{}\newline 
\hspace*{1em}{</\textbf{div}>}\mbox{}\newline 
{</\textbf{div}>}\end{shaded}\egroup\par \par
As noted above there is no requirement to record a value for either the {\itshape xml:id} or the {\itshape n} attribute. Any XML processor can identify the sequential position of one element within another in an XML document without any additional tagging. An encoding in which each line of a long poem is explicitly labelled with its numerical sequence such as the following \par\bgroup\index{l=<l>|exampleindex}\index{n=@n!<l>|exampleindex}\index{l=<l>|exampleindex}\index{n=@n!<l>|exampleindex}\index{l=<l>|exampleindex}\index{n=@n!<l>|exampleindex}\index{l=<l>|exampleindex}\index{n=@n!<l>|exampleindex}\exampleFont \begin{shaded}\noindent\mbox{}{<\textbf{l}\hspace*{1em}{n}="{1}">}\mbox{}\newline 
\textit{<!-- ... -->}\mbox{}\newline 
{</\textbf{l}>}\mbox{}\newline 
{<\textbf{l}\hspace*{1em}{n}="{2}">}\mbox{}\newline 
\textit{<!-- ... -->}\mbox{}\newline 
{</\textbf{l}>}\mbox{}\newline 
{<\textbf{l}\hspace*{1em}{n}="{3}">}\mbox{}\newline 
\textit{<!-- ... -->}\mbox{}\newline 
{</\textbf{l}>}\mbox{}\newline 
\textit{<!-- ... -->}\mbox{}\newline 
{<\textbf{l}\hspace*{1em}{n}="{100}">}\mbox{}\newline 
\textit{<!-- ... -->}\mbox{}\newline 
{</\textbf{l}>}\end{shaded}\egroup\par \noindent  is therefore probably redundant.
\subparagraph[{Language Indicators}]{Language Indicators}\label{STGAla}\par
The {\itshape xml:lang} attribute indicates the natural language and writing system applicable to the content of a given element. If it is not specified, the value is inherited from that of the immediately enclosing element. As a rule, therefore, it is simplest to specify the base language of the text on the \hyperref[TEI.TEI]{<TEI>} element, and allow most elements to take the default value for {\itshape xml:lang}; the language of an element then need be explicitly specified only for elements in languages other than the base language. For this reason, it is recommended practice to supply a default value for the {\itshape xml:lang} attribute, either on the \hyperref[TEI.TEI]{<TEI>} root element, or on both the \hyperref[TEI.teiHeader]{<teiHeader>} and the \hyperref[TEI.text]{<text>} element. The latter is appropriate in the not uncommon case where the text element in a TEI document uses a different default language from that of the TEI header attached to it. Other language shifts in the source should be explicitly identified by use of the {\itshape xml:lang} attribute on an element at an appropriate level wherever possible.\par
In the following example schematic, an English language TEI header is attached to an English language text: \par\bgroup\index{TEI=<TEI>|exampleindex}\index{teiHeader=<teiHeader>|exampleindex}\index{text=<text>|exampleindex}\exampleFont \begin{shaded}\noindent\mbox{}{<\textbf{TEI}\hspace*{1em}{xml:lang}="{en}" xmlns="http://www.tei-c.org/ns/1.0">}\mbox{}\newline 
\hspace*{1em}{<\textbf{teiHeader}>}\mbox{}\newline 
\textit{<!-- ... -->}\mbox{}\newline 
\hspace*{1em}{</\textbf{teiHeader}>}\mbox{}\newline 
\hspace*{1em}{<\textbf{text}>}\mbox{}\newline 
\textit{<!-- ... -->}\mbox{}\newline 
\hspace*{1em}{</\textbf{text}>}\mbox{}\newline 
{</\textbf{TEI}>}\end{shaded}\egroup\par \par
The same effect would be obtained by specifying the default language for both header and text: \par\bgroup\index{TEI=<TEI>|exampleindex}\index{teiHeader=<teiHeader>|exampleindex}\index{text=<text>|exampleindex}\exampleFont \begin{shaded}\noindent\mbox{}{<\textbf{TEI} xmlns="http://www.tei-c.org/ns/1.0">}\mbox{}\newline 
\hspace*{1em}{<\textbf{teiHeader}\hspace*{1em}{xml:lang}="{en}">}\mbox{}\newline 
\textit{<!-- ... -->}\mbox{}\newline 
\hspace*{1em}{</\textbf{teiHeader}>}\mbox{}\newline 
\hspace*{1em}{<\textbf{text}\hspace*{1em}{xml:lang}="{en}">}\mbox{}\newline 
\textit{<!-- ... -->}\mbox{}\newline 
\hspace*{1em}{</\textbf{text}>}\mbox{}\newline 
{</\textbf{TEI}>}\end{shaded}\egroup\par \par
The latter approach is necessary in the case where the two differ: for example, where an English language header is applied to a French text: \par\bgroup\index{TEI=<TEI>|exampleindex}\index{teiHeader=<teiHeader>|exampleindex}\index{text=<text>|exampleindex}\exampleFont \begin{shaded}\noindent\mbox{}{<\textbf{TEI} xmlns="http://www.tei-c.org/ns/1.0">}\mbox{}\newline 
\hspace*{1em}{<\textbf{teiHeader}\hspace*{1em}{xml:lang}="{en}">}\mbox{}\newline 
\textit{<!-- ... -->}\mbox{}\newline 
\hspace*{1em}{</\textbf{teiHeader}>}\mbox{}\newline 
\hspace*{1em}{<\textbf{text}\hspace*{1em}{xml:lang}="{fr}">}\mbox{}\newline 
\textit{<!-- ... -->}\mbox{}\newline 
\hspace*{1em}{</\textbf{text}>}\mbox{}\newline 
{</\textbf{TEI}>}\end{shaded}\egroup\par \par
The same principle applies at any hierarchic level. In the following example, the default language of the text is French, but one section of it is in German: \par\bgroup\index{TEI=<TEI>|exampleindex}\index{teiHeader=<teiHeader>|exampleindex}\index{text=<text>|exampleindex}\index{body=<body>|exampleindex}\index{div=<div>|exampleindex}\index{div=<div>|exampleindex}\index{div=<div>|exampleindex}\exampleFont \begin{shaded}\noindent\mbox{}{<\textbf{TEI} xmlns="http://www.tei-c.org/ns/1.0">}\mbox{}\newline 
\hspace*{1em}{<\textbf{teiHeader}\hspace*{1em}{xml:lang}="{en}">}\mbox{}\newline 
\textit{<!-- ... -->}\mbox{}\newline 
\hspace*{1em}{</\textbf{teiHeader}>}\mbox{}\newline 
\hspace*{1em}{<\textbf{text}\hspace*{1em}{xml:lang}="{fr}">}\mbox{}\newline 
\hspace*{1em}\hspace*{1em}{<\textbf{body}>}\mbox{}\newline 
\hspace*{1em}\hspace*{1em}\hspace*{1em}{<\textbf{div}>}\mbox{}\newline 
\textit{<!-- chapter one is in French -->}\mbox{}\newline 
\hspace*{1em}\hspace*{1em}\hspace*{1em}{</\textbf{div}>}\mbox{}\newline 
\hspace*{1em}\hspace*{1em}\hspace*{1em}{<\textbf{div}\hspace*{1em}{xml:lang}="{de}">}\mbox{}\newline 
\textit{<!-- chapter two is in German -->}\mbox{}\newline 
\hspace*{1em}\hspace*{1em}\hspace*{1em}{</\textbf{div}>}\mbox{}\newline 
\hspace*{1em}\hspace*{1em}\hspace*{1em}{<\textbf{div}>}\mbox{}\newline 
\textit{<!-- chapter three is French -->}\mbox{}\newline 
\hspace*{1em}\hspace*{1em}\hspace*{1em}{</\textbf{div}>}\mbox{}\newline 
\textit{<!-- ... -->}\mbox{}\newline 
\hspace*{1em}\hspace*{1em}{</\textbf{body}>}\mbox{}\newline 
\hspace*{1em}{</\textbf{text}>}\mbox{}\newline 
{</\textbf{TEI}>}\end{shaded}\egroup\par \par
Similarly, in the following example the {\itshape xml:lang} attribute on the \hyperref[TEI.term]{<term>} element allows us to record the fact that the technical terms used are Latin rather than English; no {\itshape xml:lang} attribute is needed on the \hyperref[TEI.q]{<q>} element, by contrast, because it is in the same language as its parent. \par\bgroup\index{p=<p>|exampleindex}\index{q=<q>|exampleindex}\index{term=<term>|exampleindex}\exampleFont \begin{shaded}\noindent\mbox{}{<\textbf{p}\hspace*{1em}{xml:lang}="{en}">}The\mbox{}\newline 
 constitution declares {<\textbf{q}>}that no bill of attainder or {<\textbf{term}\hspace*{1em}{xml:lang}="{la}">}ex post\mbox{}\newline 
\hspace*{1em}\hspace*{1em}\hspace*{1em}\hspace*{1em} facto{</\textbf{term}>} law shall be passed.{</\textbf{q}>} ...{</\textbf{p}>}\end{shaded}\egroup\par \par
Note that in cases where it is advisable or necessary to identify the language of the text that is pointed at, the (non-global) attribute {\itshape targetLang} should be used, for example in \par\bgroup\index{ptr=<ptr>|exampleindex}\index{target=@target!<ptr>|exampleindex}\index{targetLang=@targetLang!<ptr>|exampleindex}\exampleFont \begin{shaded}\noindent\mbox{}{<\textbf{ptr}\hspace*{1em}{target}="{x12}"\hspace*{1em}{targetLang}="{fr}"/>}\end{shaded}\egroup\par \noindent  the pointer references text written in French.\par
The values used for the {\itshape xml:lang} and {\itshape targetLang} attributes must be constructed in a particular way, using values from standard lists. See further \textit{\hyperref[CHSH]{vi.1\ Language Identification}}.\par
Additional information about a particular language may be supplied in the \hyperref[TEI.language]{<language>} element within the header (see section \textit{\hyperref[HD41]{2.4.2.\ Language Usage}}).
\subparagraph[{Rendition Indicators}]{Rendition Indicators}\label{STGAre}\par
The {\itshape rend}, {\itshape rendition}, and {\itshape style} attributes are all used to give information about the physical presentation of the text in the source. In the following example, {\itshape rend} is used to indicate that both the emphasized word and the proper name are printed in italics: \par\bgroup\index{p=<p>|exampleindex}\index{emph=<emph>|exampleindex}\index{rend=@rend!<emph>|exampleindex}\index{name=<name>|exampleindex}\index{rend=@rend!<name>|exampleindex}\exampleFont \begin{shaded}\noindent\mbox{}{<\textbf{p}>} ... Their motives {<\textbf{emph}\hspace*{1em}{rend}="{italics}">}might{</\textbf{emph}>} be pure\mbox{}\newline 
 and pious; but he was equally alarmed by his knowledge of the ambitious {<\textbf{name}\hspace*{1em}{rend}="{italics}">}Bohemond{</\textbf{name}>}, and his ignorance of the Transalpine chiefs:\mbox{}\newline 
 ...{</\textbf{p}>}\end{shaded}\egroup\par \noindent  The same effect might be achieved using the {\itshape style} attribute, as follows: \par\bgroup\index{p=<p>|exampleindex}\index{emph=<emph>|exampleindex}\index{style=@style!<emph>|exampleindex}\index{name=<name>|exampleindex}\index{style=@style!<name>|exampleindex}\exampleFont \begin{shaded}\noindent\mbox{}{<\textbf{p}>} ... Their motives {<\textbf{emph}\hspace*{1em}{style}="{font-style: italic}">}might{</\textbf{emph}>} be pure and pious; but he was equally alarmed by his knowledge of\mbox{}\newline 
 the ambitious {<\textbf{name}\hspace*{1em}{style}="{font-style: italic}">}Bohemond{</\textbf{name}>}, and his ignorance of\mbox{}\newline 
 the Transalpine chiefs: ...{</\textbf{p}>}\end{shaded}\egroup\par \noindent  If all or most \hyperref[TEI.emph]{<emph>} and \hyperref[TEI.name]{<name>} elements are rendered in the text by italics, it will be more convenient to register that fact in the TEI header once and for all (using the \hyperref[TEI.rendition]{<rendition>} element discussed below) and specify a {\itshape rend} or {\itshape style} value only for any elements which deviate from the stated rendition.\par
The main difference between {\itshape rend} attribute and {\itshape style} is that the value used for the former may contain one or more tokens from any vocabulary devised by the encoder, separated by space characters, whereas the value used for the latter must be a single string taken from a formally-defined style definition language such as CSS. The {\itshape rend} attribute values are sequence-indeterminate set of whitespace-separated tokens, whereas {\itshape style} values allow whitespace and sequence relationships as part of the formally-defined style definition language.\par
The \hyperref[TEI.rendition]{<rendition>} element defined in \textit{\hyperref[HD57]{2.3.4.\ The Tagging Declaration}} may be used to hold repeatedly-used format descriptions. A \hyperref[TEI.rendition]{<rendition>} element can then be associated with any element, either by default, or by means of the global {\itshape rendition} attribute. For example: \par\bgroup\index{tagsDecl=<tagsDecl>|exampleindex}\index{rendition=<rendition>|exampleindex}\index{scheme=@scheme!<rendition>|exampleindex}\index{selector=@selector!<rendition>|exampleindex}\index{rendition=<rendition>|exampleindex}\index{scheme=@scheme!<rendition>|exampleindex}\index{selector=@selector!<rendition>|exampleindex}\index{text=<text>|exampleindex}\index{body=<body>|exampleindex}\index{div=<div>|exampleindex}\index{p=<p>|exampleindex}\index{rendition=@rendition!<p>|exampleindex}\index{p=<p>|exampleindex}\exampleFont \begin{shaded}\noindent\mbox{}{<\textbf{tagsDecl}>}\mbox{}\newline 
\textit{<!-- define italic style using CSS, selecting it as default for emph and hi elements -->}\mbox{}\newline 
\hspace*{1em}{<\textbf{rendition}\hspace*{1em}{xml:id}="{IT}"\hspace*{1em}{scheme}="{css}"\mbox{}\newline 
\hspace*{1em}\hspace*{1em}{selector}="{emph, hi}">}font-style: italic;{</\textbf{rendition}>}\mbox{}\newline 
\textit{<!-- define a serif font family, selecting it as default for the text element -->}\mbox{}\newline 
\hspace*{1em}{<\textbf{rendition}\hspace*{1em}{xml:id}="{FontRoman}"\hspace*{1em}{scheme}="{css}"\mbox{}\newline 
\hspace*{1em}\hspace*{1em}{selector}="{text}">}font-family: serif;{</\textbf{rendition}>}\mbox{}\newline 
{</\textbf{tagsDecl}>}\mbox{}\newline 
\textit{<!-- ... -->}\mbox{}\newline 
{<\textbf{text}>}\mbox{}\newline 
\hspace*{1em}{<\textbf{body}>}\mbox{}\newline 
\hspace*{1em}\hspace*{1em}{<\textbf{div}>}\mbox{}\newline 
\hspace*{1em}\hspace*{1em}\hspace*{1em}{<\textbf{p}\hspace*{1em}{rendition}="{\#IT}">}\mbox{}\newline 
\textit{<!-- this paragraph uses the seriffed font, but is in italic-->}\mbox{}\newline 
\hspace*{1em}\hspace*{1em}\hspace*{1em}{</\textbf{p}>}\mbox{}\newline 
\hspace*{1em}\hspace*{1em}\hspace*{1em}{<\textbf{p}>}\mbox{}\newline 
\textit{<!-- this paragraph uses the seriffed font, but is not in italic -->}\mbox{}\newline 
\hspace*{1em}\hspace*{1em}\hspace*{1em}{</\textbf{p}>}\mbox{}\newline 
\hspace*{1em}\hspace*{1em}{</\textbf{div}>}\mbox{}\newline 
\hspace*{1em}{</\textbf{body}>}\mbox{}\newline 
{</\textbf{text}>}\end{shaded}\egroup\par \par
The {\itshape rendition} attribute always points to one or more \hyperref[TEI.rendition]{<rendition>} elements, each of which defines some aspect of the rendering or appearance of the text in its original form. These details may most conveniently be described using a formal style definition language, such as CSS (\cite{CSS1}) or XSL-FO (\cite{XSL11}); in some other formal language developed for a specific project; or even informally in running prose. Although languages such as CSS and XSL-FO are generally used to describe document output to screen or print, they nonetheless provide formal and precise mechanisms for describing the appearance of source documents, especially print documents, but also many aspects of manuscript documents. For example, both CSS and XSL-FO provide mechanisms for describing typefaces, weight, and styles; character and line spacing; and so on.\par
As noted above, the {\itshape style} attribute is provided for encoders wishing to describe the appearance of individual source elements using a language such as CSS directly rather than by reference to a \hyperref[TEI.rendition]{<rendition>} element. Its value may be any expression in the chosen formal style definition language.\par
Formal definition languages such as CSS typically identity a series of \textit{properties} (such as font-style or margin-left) for which \textit{values} are specified. A sequence of such property-value pairs makes up a stylesheet. The TEI uses such languages simply to describe the appearance of a source document, rather than to control how it should be formatted.\par
In the TEI scheme, it is possible to supply information about the appearance of elements within a source document in the following distinct ways: \begin{enumerate}
\item One or more properties may be specified as the default for a set of elements (based on an external scheme, by default CSS), using \hyperref[TEI.rendition]{<rendition>} elements and their {\itshape selector} attributes;
\item One or more properties may be specified for individual element occurrences, using the {\itshape rend} attribute with any convenient set of one or more sequence-indeterminate tokens;
\item One or more properties may be specified for individual element occurrences, using the {\itshape rendition} attribute to point to \hyperref[TEI.rendition]{<rendition>} elements;
\item One or more properties may be supplied explicitly for individual element occurrences, using the {\itshape style} attribute.
\end{enumerate}\par
If the same property is specified in more than one of the above ways, the one with the highest number in the list above is understood to be applicable. The resulting properties from each way are then combined to provide the full set of property-value pairs applicable to the given element, and (by default) to all of its children. \par
For simplicity of processing, the same formal style definition should be used throughout; however, the architecture does permit this to be varied, by using the {\itshape scheme} attribute to indicate a different language for one or more \hyperref[TEI.rendition]{<rendition>} elements. Care should be taken to ensure that such values can be meaningfully combined. Similar considerations apply to the use of the {\itshape rend} attribute, if this is used in combination with either {\itshape rendition} or {\itshape style}.\par
Note that these TEI attributes always describe the rendition or appearance of the source document, \textit{not} intended output renditions, although often the two may be closely related.
\subparagraph[{Sources, certainty, and responsibility}]{Sources, certainty, and responsibility}\label{STGAso}\par
The {\itshape source} attribute is used to indicate the source of an element and its content, for example by pointing to a bibliographic citation for a quotation to indicate the source from which it derives. The target of the pointer may be an entry in a bibliographic list of some kind, or a pointer to a digital version of the source itself.\par
As with other TEI pointers, the value of this attribute is expressed as any form of URI, for example an absolute URL, a relative URL, or a private scheme URI that is expanded to a relative or absolute URL as documented in a \hyperref[TEI.prefixDef]{<prefixDef>}. In the following typical example a relative ‘bare name’ URL value is used to point to a \hyperref[TEI.bibl]{<bibl>} elsewhere in the bibliography of the document which contains a bibliographic source for the quotation itself: \par\bgroup\index{p=<p>|exampleindex}\index{quote=<quote>|exampleindex}\index{source=@source!<quote>|exampleindex}\index{bibl=<bibl>|exampleindex}\index{title=<title>|exampleindex}\index{level=@level!<title>|exampleindex}\index{edition=<edition>|exampleindex}\index{pubPlace=<pubPlace>|exampleindex}\index{publisher=<publisher>|exampleindex}\index{date=<date>|exampleindex}\index{biblScope=<biblScope>|exampleindex}\index{unit=@unit!<biblScope>|exampleindex}\exampleFont \begin{shaded}\noindent\mbox{}{<\textbf{p}>}\mbox{}\newline 
\textit{<!-- ... -->}\mbox{}\newline 
\hspace*{1em}{<\textbf{quote}\hspace*{1em}{source}="{\#chicago-15\textunderscore ed}">}Grammatical theories\mbox{}\newline 
\hspace*{1em}\hspace*{1em} are in flux, and the more we learn, the less we\mbox{}\newline 
\hspace*{1em}\hspace*{1em} seem to know.{</\textbf{quote}>}\mbox{}\newline 
\textit{<!-- ... -->}\mbox{}\newline 
{</\textbf{p}>}\mbox{}\newline 
\textit{<!-- ... -->}\mbox{}\newline 
{<\textbf{bibl}\hspace*{1em}{xml:id}="{chicago-15\textunderscore ed}">}\mbox{}\newline 
\hspace*{1em}{<\textbf{title}\hspace*{1em}{level}="{m}">}The Chicago Manual of Style{</\textbf{title}>},\mbox{}\newline 
{<\textbf{edition}>}15th edition{</\textbf{edition}>}.\mbox{}\newline 
{<\textbf{pubPlace}>}Chicago{</\textbf{pubPlace}>}:\mbox{}\newline 
{<\textbf{publisher}>}University of Chicago Press{</\textbf{publisher}>} \mbox{}\newline 
 ({<\textbf{date}>}2003{</\textbf{date}>}),\mbox{}\newline 
{<\textbf{biblScope}\hspace*{1em}{unit}="{page}">}p.147{</\textbf{biblScope}>}.\mbox{}\newline 
\mbox{}\newline 
{</\textbf{bibl}>}\end{shaded}\egroup\par \par
Alternatively, the quotation might be directly linked to the online edition of this source using a full URI : \par\bgroup\index{p=<p>|exampleindex}\index{quote=<quote>|exampleindex}\index{source=@source!<quote>|exampleindex}\exampleFont \begin{shaded}\noindent\mbox{}{<\textbf{p}>}\mbox{}\newline 
\textit{<!-- ... -->}\mbox{}\newline 
\hspace*{1em}{<\textbf{quote}\hspace*{1em}{source}="{http://www.chicagomanualofstyle.org/15/ch05/ch05\textunderscore sec002.html}">}Grammatical theories\mbox{}\newline 
\hspace*{1em}\hspace*{1em} are in flux, and the more we learn, the less we\mbox{}\newline 
\hspace*{1em}\hspace*{1em} seem to know.{</\textbf{quote}>}\mbox{}\newline 
\textit{<!-- ... -->}\mbox{}\newline 
{</\textbf{p}>}\end{shaded}\egroup\par \noindent  \par
The {\itshape source} attribute is also used on schema documentation elements such as \hyperref[TEI.schemaSpec]{<schemaSpec>} or \hyperref[TEI.elementRef]{<elementRef>} to indicate the location from which declarations for the components being defined may be obtained by an ODD processor. For example, a customization wishing to include the \hyperref[TEI.p]{<p>} element specifically as it was in version 2.0.1 of TEI P5 would indicate the source for this on an \hyperref[TEI.elementRef]{<elementRef>} element like the following: \par\bgroup\index{elementRef=<elementRef>|exampleindex}\index{key=@key!<elementRef>|exampleindex}\index{source=@source!<elementRef>|exampleindex}\exampleFont \begin{shaded}\noindent\mbox{}{<\textbf{elementRef}\hspace*{1em}{key}="{p}"\hspace*{1em}{source}="{tei:2.0.1}"/>}\end{shaded}\egroup\par \noindent  Here the value of the {\itshape source} attribute is provided using private URI syntax, using a short cut predefined for the TEI Guidelines. More generally, an ODD customization can point to a URI from which a compiled version of any ODD can be downloaded. The above shortcut is equivalent to \par\bgroup\index{elementRef=<elementRef>|exampleindex}\index{key=@key!<elementRef>|exampleindex}\index{source=@source!<elementRef>|exampleindex}\exampleFont \begin{shaded}\noindent\mbox{}{<\textbf{elementRef}\hspace*{1em}{key}="{p}"\mbox{}\newline 
\hspace*{1em}{source}="{http://www.tei-c.org/Vault/P5/2.0.1/xml/tei/odd/p5subset.xml}"/>}\end{shaded}\egroup\par \noindent  Elements such as \hyperref[TEI.moduleRef]{<moduleRef>} or \hyperref[TEI.elementRef]{<elementRef>} can use the {\itshape source} attribute in this way to point to any previously compiled set of TEI ODD specifications which are to be included in a schema, as further discussed in section \textit{\hyperref[TDbuild]{22.8.1.\ TEI customizations}}\par
The {\itshape cert} attribute provides a method of indicating the encoder's certainty concerning an intervention or interpretation represented by the markup. It is typically used where the encoder wishes to supply one or more possible corrections to a text, indicating the certainty they wish to attach to each, as in the following example: \par\bgroup\index{choice=<choice>|exampleindex}\index{sic=<sic>|exampleindex}\index{corr=<corr>|exampleindex}\index{cert=@cert!<corr>|exampleindex}\index{corr=<corr>|exampleindex}\index{cert=@cert!<corr>|exampleindex}\exampleFont \begin{shaded}\noindent\mbox{} Blessed are the {<\textbf{choice}>}\mbox{}\newline 
\hspace*{1em}{<\textbf{sic}>}cheesemakers{</\textbf{sic}>}\mbox{}\newline 
\hspace*{1em}{<\textbf{corr}\hspace*{1em}{cert}="{high}">}peacemakers{</\textbf{corr}>}\mbox{}\newline 
\hspace*{1em}{<\textbf{corr}\hspace*{1em}{cert}="{low}">}placemakers{</\textbf{corr}>}\mbox{}\newline 
{</\textbf{choice}>}:\mbox{}\newline 
 for they shall be called the children of God.\end{shaded}\egroup\par \noindent  The {\itshape cert} attribute will usually, as here, characterize the degree of certainty simply as high, medium or low. In situations where a more detailed or nuanced indication is required, it can instead supply a probability value between 0 (minimal probability) and 1 (maximal probability). Other more sophisticated mechanisms are discussed in chapter \textit{\hyperref[CE]{21.\ Certainty, Precision, and Responsibility}}.\par
The {\itshape resp} attribute is used to indicate the person or organization considered responsible for some aspects of the information encoded by an element. For example, the preceding example might be revised as follows to indicate the editors responsible for the two corrections: \par\bgroup\index{corr=<corr>|exampleindex}\index{cert=@cert!<corr>|exampleindex}\index{resp=@resp!<corr>|exampleindex}\index{corr=<corr>|exampleindex}\index{cert=@cert!<corr>|exampleindex}\index{resp=@resp!<corr>|exampleindex}\exampleFont \begin{shaded}\noindent\mbox{} ... {<\textbf{corr}\hspace*{1em}{cert}="{high}"\hspace*{1em}{resp}="{\#ed1}">}peacemakers{</\textbf{corr}>}\mbox{}\newline 
{<\textbf{corr}\hspace*{1em}{cert}="{low}"\hspace*{1em}{resp}="{\#ed2}">}placemakers{</\textbf{corr}>}...\end{shaded}\egroup\par \noindent  When a more detailed or nuanced representation of responsibility is required, it is recommended that the element indicated by the {\itshape resp} attribute should not be a generic agent (for example a \hyperref[TEI.person]{<person>} or \hyperref[TEI.org]{<org>}) but a more precise element such as \hyperref[TEI.respStmt]{<respStmt>}, \hyperref[TEI.author]{<author>}, or \hyperref[TEI.editor]{<editor>} which can document the exact role played by the agent. In the following example, we indicate that the correction of n to u was made by a particular named transcriber: \par\bgroup\index{lg=<lg>|exampleindex}\index{l=<l>|exampleindex}\index{choice=<choice>|exampleindex}\index{sic=<sic>|exampleindex}\index{corr=<corr>|exampleindex}\index{resp=@resp!<corr>|exampleindex}\index{respStmt=<respStmt>|exampleindex}\index{resp=<resp>|exampleindex}\index{name=<name>|exampleindex}\exampleFont \begin{shaded}\noindent\mbox{}\mbox{}\newline 
\textit{<!-- in the <text> ... -->}{<\textbf{lg}>}\mbox{}\newline 
\textit{<!-- ... -->}\mbox{}\newline 
\hspace*{1em}{<\textbf{l}>}Punkes, Panders, baſe extortionizing\mbox{}\newline 
\hspace*{1em}\hspace*{1em} sla{<\textbf{choice}>}\mbox{}\newline 
\hspace*{1em}\hspace*{1em}\hspace*{1em}{<\textbf{sic}>}n{</\textbf{sic}>}\mbox{}\newline 
\hspace*{1em}\hspace*{1em}\hspace*{1em}{<\textbf{corr}\hspace*{1em}{resp}="{\#JENSJ}">}u{</\textbf{corr}>}\mbox{}\newline 
\hspace*{1em}\hspace*{1em}{</\textbf{choice}>}es,{</\textbf{l}>}\mbox{}\newline 
\textit{<!-- ... -->}\mbox{}\newline 
{</\textbf{lg}>}\mbox{}\newline 
\textit{<!-- in the <teiHeader> ... -->}\mbox{}\newline 
\textit{<!-- ... -->}\mbox{}\newline 
{<\textbf{respStmt}\hspace*{1em}{xml:id}="{JENSJ}">}\mbox{}\newline 
\hspace*{1em}{<\textbf{resp}>}Transcriber{</\textbf{resp}>}\mbox{}\newline 
\hspace*{1em}{<\textbf{name}>}Janelle Jenstad{</\textbf{name}>}\mbox{}\newline 
{</\textbf{respStmt}>}\end{shaded}\egroup\par \noindent  Pointing to multiple \hyperref[TEI.respStmt]{<respStmt>}s allows the encoder to specify clearly each of the roles played in part of a TEI file (creating, transcribing, encoding, editing, proofing etc.). If appropriate, the \hyperref[TEI.name]{<name>} element inside a \hyperref[TEI.respStmt]{<respStmt>} may also be associated with a more detailed \hyperref[TEI.person]{<person>} or \hyperref[TEI.org]{<org>} element using methods discussed in chapter \textit{\hyperref[ND]{13.\ Names, Dates, People, and Places}}.
\subparagraph[{Evaluation of Links}]{Evaluation of Links}\label{STGAba}\par
Several TEI elements carry attributes whose values are defined as \texttt{anyURI}, meaning that such attributes supply a link or pointer, typically expressed as a URL. Like other XML applications, the TEI allows use of a special attribute to set the context within which relative URLs are to be evaluated. The global attribute {\itshape xml:base} is defined as part of the XML specification and belongs to the XML namespace rather than the TEI namespace. We do not describe it in detail here: reference information about {\itshape xml:base} is provided by \cite{XMLBASE}\par
In essence {\itshape xml:base} is used to set a context for all relative URLs within the scope of the element on which it is specified. For example: \par\bgroup\index{body=<body>|exampleindex}\index{div=<div>|exampleindex}\index{p=<p>|exampleindex}\index{ptr=<ptr>|exampleindex}\index{target=@target!<ptr>|exampleindex}\index{div=<div>|exampleindex}\index{p=<p>|exampleindex}\index{ptr=<ptr>|exampleindex}\index{target=@target!<ptr>|exampleindex}\exampleFont \begin{shaded}\noindent\mbox{}{<\textbf{body}>}\mbox{}\newline 
\hspace*{1em}{<\textbf{div}\hspace*{1em}{xml:base}="{http://www.example.org/somewhere.xml}">}\mbox{}\newline 
\hspace*{1em}\hspace*{1em}{<\textbf{p}>}\mbox{}\newline 
\textit{<!--... -->}\mbox{}\newline 
\hspace*{1em}\hspace*{1em}\hspace*{1em}{<\textbf{ptr}\hspace*{1em}{target}="{elsewhere.xml}"/>}\mbox{}\newline 
\textit{<!--... -->}\mbox{}\newline 
\hspace*{1em}\hspace*{1em}{</\textbf{p}>}\mbox{}\newline 
\hspace*{1em}{</\textbf{div}>}\mbox{}\newline 
\hspace*{1em}{<\textbf{div}>}\mbox{}\newline 
\hspace*{1em}\hspace*{1em}{<\textbf{p}>}\mbox{}\newline 
\textit{<!--... -->}\mbox{}\newline 
\hspace*{1em}\hspace*{1em}\hspace*{1em}{<\textbf{ptr}\hspace*{1em}{target}="{elsewhere.xml}"/>}\mbox{}\newline 
\textit{<!--... -->}\mbox{}\newline 
\hspace*{1em}\hspace*{1em}{</\textbf{p}>}\mbox{}\newline 
\hspace*{1em}{</\textbf{div}>}\mbox{}\newline 
{</\textbf{body}>}\end{shaded}\egroup\par \noindent The first \hyperref[TEI.ptr]{<ptr>} element here is within the scope of a \hyperref[TEI.div]{<div>} which supplies a value for {\itshape xml:base}; its target is therefore to be found at \texttt{http://www.example.org/elsewhere.xml}. The second \hyperref[TEI.ptr]{<ptr>}, however, is within the scope of a \hyperref[TEI.div]{<div>} which does not change the default context, and its target is therefore a document in the same directory as the current document.\par
The {\itshape xml:base} attribute is intended to enable the stable resolution of relative URIs in a document after that document's context may have changed (for example as a result of being embedded in another document via XInclude). Setting the {\itshape xml:base} simply as a way to allow encoders to write shorter URIs is not recommended. In particular, {\itshape xml:base} may cause ambiguity as to the referent of same-document references in the form \texttt{\#id} (where \texttt{id} is an {\itshape xml:id}). \xref{https://tools.ietf.org/html/rfc3986\#section-4.4}{RFC 3986} states that URIs of this type should not result in the loading of a different document. The RFC therefore assumes that such references are internal to the document in which they are located. Using {\itshape xml:base} to denote arbitrary external bases while also using same-document references may mean that software agents deal with these links in unexpected and inconsistent ways. Further discussion of this attribute and its effect on TEI linking methods is provided in chapter \textit{\hyperref[SA]{16.\ Linking, Segmentation, and Alignment}}.
\subparagraph[{XML Whitespace}]{XML Whitespace}\label{STGAxs}\par
The global attribute {\itshape xml:space} provides a mechanism for indicating to systems processing an XML file how they should treat whitespace, that is, any sequences of consecutive tab (\#x09), space (\#x20), carriage return (\#x0D) or linefeed (\#x0A) characters. Like {\itshape xml:id} this attribute is defined as part of the XML specification and belongs to the XML namespace rather than the TEI namespace. Complete information about this attribute is provided by \xref{http://www.w3.org/TR/REC-xml/\#sec-white-space}{section 2.10 of the XML Specification}; here we provide a summary of how its use affects users of the TEI scheme.\par
The {\itshape xml:space} attribute has only two permitted values: preserve and default. The first indicates that whitespace in a text node—every carriage return, every tab, etc.—should be maintained as is when the document is processed. The second (which is implied when the attribute is not supplied), indicates that whitespace should be handled ‘as appropriate’. Exactly what is deemed appropriate is left unspecified by the XML Recommendation.\par
These Guidelines assume one of two different ways of processing whitespace will apply in a given case, depending on an element's content model. For an element that can contain only other elements with no intervening non-whitespace characters, whitespace is considered to have no semantic significance, and should therefore be discarded by a processor. For example, in a \hyperref[TEI.choice]{<choice>} element, such as \par\bgroup\index{choice=<choice>|exampleindex}\index{sic=<sic>|exampleindex}\index{corr=<corr>|exampleindex}\exampleFont \begin{shaded}\noindent\mbox{}{<\textbf{choice}>}\mbox{}\newline 
\hspace*{1em}{<\textbf{sic}>}1724{</\textbf{sic}>}\mbox{}\newline 
\hspace*{1em}{<\textbf{corr}>}1728{</\textbf{corr}>}\mbox{}\newline 
{</\textbf{choice}>}\end{shaded}\egroup\par \noindent  since non-whitespace text is not permitted between the \hyperref[TEI.choice]{<choice>} start-tag and the \hyperref[TEI.sic]{<sic>} tags or between the \hyperref[TEI.sic]{<sic>} and \hyperref[TEI.corr]{<corr>} tags, any whitespace found there has no significance and can be ignored completely by a processor.\par
Similarly, the \hyperref[TEI.address]{<address>} element has a content model containing only elements: any punctuation or whitespace required between the lines of an address must therefore be supplied by the processor, as any whitespace present in the input document will be ignored.\par
Elements with content models of this type are comparatively unusual in the TEI: a list of them is provided in the TEI release file \textsf{stripspace.xsl.model}, formatted there for use as an \texttt{<xsl:strip-space>} command for XSL stylesheets.\par
Most TEI elements permit what is known as mixed-content: that is, they can contain both text and other elements. Here the assumption of these Guidelines is that whitespace will be normalized. This means that all space, carriage return, linefeed, and tab characters are converted into spaces, all consecutive spaces are then deleted and replaced by one space, and then space immediately after a start-tag or immediately before an end-tag is deleted. The result is that this encoding, \par\bgroup\index{persName=<persName>|exampleindex}\index{forename=<forename>|exampleindex}\index{forename=<forename>|exampleindex}\index{surname=<surname>|exampleindex}\index{type=@type!<surname>|exampleindex}\index{roleName=<roleName>|exampleindex}\index{placeName=<placeName>|exampleindex}\exampleFont \begin{shaded}\noindent\mbox{}{<\textbf{persName}>}\mbox{}\newline 
\hspace*{1em}{<\textbf{forename}>}Edward{</\textbf{forename}>}\mbox{}\newline 
\hspace*{1em}{<\textbf{forename}>}George{</\textbf{forename}>}\mbox{}\newline 
\hspace*{1em}{<\textbf{surname}\hspace*{1em}{type}="{linked}">}Bulwer-Lytton{</\textbf{surname}>}, {<\textbf{roleName}>}Baron Lytton of\mbox{}\newline 
\hspace*{1em}{<\textbf{placeName}>}Knebworth{</\textbf{placeName}>}\mbox{}\newline 
\hspace*{1em}{</\textbf{roleName}>}\mbox{}\newline 
{</\textbf{persName}>}\end{shaded}\egroup\par \noindent  would be rendered as ‘Edward George Bulwer-Lytton, Baron Lytton of Knebworth’. The space before his name has been removed, a space is included between his forenames, the comma is preserved, and the newlines within his name have all been removed.\par
If the default treatment described above is not appropriate for a mixed content element, the processing required may be described in the \hyperref[TEI.encodingDesc]{<encodingDesc>} element of the TEI header, but generic XML processing tools may not take note of this.\par
Alternatively, the {\itshape xml:space} attribute may be supplied with a value of preserve in order to indicate that every space, tab, carriage return and linefeed character found within that element in the document being processed is significant. Typically, the result of that processing will be to retain the whitespace characters in the output. Thus if the above example began <persName xml:space="preserve">, the resulting text would most likely be rendered over five lines, indented, and with a blank line following.\par
The \texttt{xml:space="preserve"} attribute is rarely used in TEI documents because such layout features are generally captured with less risk and more precision by using native TEI elements such as \hyperref[TEI.lb]{<lb>} or \hyperref[TEI.space]{<space>}, or by using the renditional attributes described in section \textit{\hyperref[STGAre]{1.3.1.1.3.\ Rendition Indicators}}.
\subparagraph[{Other Globally Available Attributes}]{Other Globally Available Attributes}\label{STGAothers}\par
The following table lists for convenience other potentially available global attributes. The table specifies the name of the attribute class providing the attributes concerned, the module which must be included in a schema if the attributes are to be made available, and the section of these Guidelines where the class is discussed.  \par 
\begin{longtable}{P{0.17954545454545456\textwidth}P{0.0909090909090909\textwidth}P{0.5795454545454545\textwidth}}
\rowcolor{label}class name\tabcellsep module name\tabcellsep see further\\\hline 
att.global.linking\tabcellsep linking\tabcellsep \textit{\hyperref[SA]{16.\ Linking, Segmentation, and Alignment}}\\
att.global.analytic\tabcellsep analysis\tabcellsep \textit{\hyperref[AI]{17.\ Simple Analytic Mechanisms}}\\
att.global.facs\tabcellsep transcr\tabcellsep \textit{\hyperref[PHFAX]{11.1.\ Digital Facsimiles}}\\
att.global.change\tabcellsep transcr\tabcellsep \textit{\hyperref[PH-changes]{11.7.\ Identifying Changes and Revisions}}\end{longtable} \par
 
\subsubsection[{Model Classes}]{Model Classes}\label{STECCM}\par
As noted above, the members of a given TEI model class share the property that they can all appear in the same location within a document. Wherever possible, the content model of a TEI element is expressed not directly in terms of specific elements, but indirectly in terms of particular model classes. This makes content models simpler and more consistent; it also makes them much easier to understand and to modify.\par
Like attribute classes, model classes may have subclasses or superclasses. Just as elements inherit from a class the ability to appear in certain locations of a document (wherever the class can appear), so all members of a subclass inherit the ability to appear wherever any superclass can appear. To some extent, the class system thus provides a way of reducing the whole TEI galaxy of elements into a tidy hierarchy. This is however not entirely the case.\par
In fact, the nature of a given class of elements can be considered along two dimensions: as noted, it defines a set of places where the class members are permitted within the document hierarchy; it also implies a semantic grouping of some kind. For example, the very large class of elements which can appear within a paragraph comprises a number of other classes, all of which have the same structural property, but which differ in their field of application. Some are related to highlighting, while others relate to names or places, and so on. In some cases, the ‘set of places where class members are permitted’ is very constrained: it may just be within one specific element, or one class of element, for example. In other cases, elements may be permitted to appear in very many places, or in more than one such set of places.\par
These factors are reflected in the way that model classes are named. If a model class has a name containing part, such as \textsf{model.divPart} or \textsf{model.biblPart} then it is primarily defined in terms of its structural location. For example, those elements (or classes of element) which appear as content of a \hyperref[TEI.div]{<div>} constitute the \textsf{model.divPart} class; those which appear as content of a \hyperref[TEI.bibl]{<bibl>} constitute the \textsf{model.biblPart} class. If, however, a model class has a name containing like, such as \textsf{model.biblLike} or \textsf{model.nameLike}, the implication is that its members all have some additional semantic property in common, for example containing a bibliographic description, or containing some form of name, respectively. These semantically-motivated classes often provide a useful way of dividing up large structurally-motivated classes: for example, the very general structural class \textsf{model.pPart.data} (‘data elements that form part of a paragraph’) has four semantically-motivated member classes (\textsf{model.addressLike}, \textsf{model.dateLike}, \textsf{model.measureLike}, and \textsf{model.nameLike}), the last of these being itself a superclass with several members.\par
Although most classes are defined by the \textsf{tei} infrastructure module, a class cannot be populated unless some other specific module is included in a schema, since element declarations are contained by modules. Classes are not declared ‘top down’, but instead gain their members as a consequence of individual elements' declaration of their membership. The same class may therefore contain different members, depending on which modules are active. Consequently, the content model of a given element (being expressed in terms of model classes) may differ depending on which modules are active.\par
Some classes contain only a single member, even when all modules are loaded. One reason for declaring such a class is to make it easier for a customization to add new member elements in a specific place, particularly in areas where the TEI does not make fully elaborated proposals. For example, the TEI class \textsf{model.rdgLike}, initially empty, is expanded by the \textsf{textcrit} module to include just the TEI \hyperref[TEI.rdg]{<rdg>} element. A project wishing to add an alternative way of structuring text-critical information could do so by defining their own elements and adding it to this class.\par
Another reason for declaring single-member classes is where the class members are not needed in all documents, but appear in the same place as elements which are very frequently required. For example, the specialized element \hyperref[TEI.g]{<g>} used to represent a non-Unicode character or glyph is provided as the only member of the \textsf{model.gLike} class when the \textsf{gaiji} module is added to a schema. References to this class are included in almost every content model, since if it is used at all the \hyperref[TEI.g]{<g>} must be available wherever text is available; however these references have no effect unless the gaiji module is loaded.\par
At the other end of the scale, a few of the classes predefined by the tei module are subsequently populated with very many members. For example, the class \textsf{model.pPart.edit} groups all the classes of element for simple editorial correction and transcription which can appear within a \hyperref[TEI.p]{<p>} or paragraph element. The \textsf{core} module alone adds more than fifty elements to this class; the \textsf{namesdates} module adds another twenty, as does the \textsf{tagdocs} module. Since the \hyperref[TEI.p]{<p>} element is one of the basic building blocks of a TEI document it is not surprising that each module will need to add elements to it. The class system here provides a very convenient way of controlling the resulting complexity. Typically, elements are not added directly to these very general classes, but via some intermediate semantically-motivated class.\par
Just as there are a few classes which have a single member, so there are some classes which are used only once in the TEI architecture. These classes, which have no superclass and therefore do not fit into the class hierarchy defined here, are a convenient way of maintaining elements which are highly structured internally, but which appear from the outside to be uniform objects like others at the same level.\footnote{In former editions of these Guidelines, such elements were known metaphorically as ‘crystals’.} Members of such classes can only ever appear within one element, or one class of elements. For example, the class \textsf{model.addrPart} is used only to express the content model for the element \hyperref[TEI.address]{<address>}; it references some other classes of elements, which can appear elsewhere, and also some elements which can only appear inside an address.
\paragraph[{Informal Element Classifications }]{Informal Element Classifications }\label{STBTC}\par
Most TEI elements may also be informally classified as belonging to one of the following groupings: \begin{description}

\item[{\textit{divisions}}]high level, possibly self-nesting, major divisions of texts. These elements populate such classes as \textsf{model.divLike} or \textsf{model.div1Like}, and typically form the largest component units of a text.
\item[{\textit{chunks}}]elements such as paragraphs and other paragraph-level elements, which can appear directly within texts or within divisions of them, but not (usually) within other chunks. These elements populate the class \textsf{model.divPart}, either directly or by means of other classes such as \textsf{model.pLike} (paragraph-like elements), \textsf{model.entryLike}, etc.
\item[{\textit{phrase-level elements}}]elements such as highlighted phrases, book titles, or editorial corrections which can occur only within chunks, but not between them (and thus cannot appear directly within a division). These elements populate the class \textsf{model.phrase}.\footnote{Note that in this context, \textit{phrase} means any string of characters, and can apply to individual words, parts of words, and groups of words indifferently; it does not refer only to linguistically-motivated phrasal units. This may cause confusion for readers accustomed to applying the word in a more restrictive sense.}
\end{description} \par
The TEI also identifies two further groupings derived from these three: \begin{description}

\item[{\textit{inter-level elements}}]elements such as lists, notes, quotations, etc. which can appear either between chunks (as children of a \hyperref[TEI.div]{<div>}) or within them; these elements populate the class \textsf{model.inter}. Note that this class is not a superset of the \textsf{model.phrase} and \textsf{model.divPart} classes but rather a distinct grouping of elements which are both chunk-like and phrase-like. However, the classes \textsf{model.phrase}, \textsf{model.pLike}, and \textsf{model.inter} are all disjoint.
\item[{\textit{components}}]elements which can appear directly within texts or text divisions; this is a combination of the inter- and chunk- level elements defined above. These elements populate the class \textsf{model.common}, which is defined as a superset of the classes \textsf{model.divPart}, \textsf{model.inter}, and (when the dictionary module is included in a schema) \textsf{model.entryLike}.
\end{description}  Broadly speaking, the front, body, and back of a text each comprises a series of components, optionally grouped into divisions.\par
As noted above, some elements do not belong to any model class, and some model classes are not readily associated with any of the above informal groupings. However, over two-thirds of the 589 elements defined in the present edition of these Guidelines are classified in this way, and future editions of these recommendations will extend and develop this classification scheme.\par
A complete alphabetical list of all model classes is provided in \textit{\hyperref[REF-CLASSES-MODEL]{Appendix A\ Model Classes}}.
\subsection[{Macros}]{Macros}\label{STmacros}\par
The infrastructure module defined by this chapter also declares a number of \textit{macros}, or shortcut names for frequently occurring parts of other declarations. Macros are used in two ways in the TEI scheme: to stand for frequently-encountered content models, or parts of content models (\textit{\hyperref[STECST]{1.4.1.\ Standard Content Models}}); and to stand for attribute datatypes (\textit{\hyperref[DTYPES]{1.4.2.\ Datatype Specifications}}).
\subsubsection[{Standard Content Models}]{Standard Content Models}\label{STECST}\par
As far as possible, the TEI schemas use the following set of frequently-encountered content models to help achieve consistency among different elements. 
\begin{sansreflist}
  
\item [\textbf{macro.paraContent}] (paragraph content) defines the content of paragraphs and similar elements.
\item [\textbf{macro.limitedContent}] (paragraph content) defines the content of prose elements that are not used for transcription of extant materials.
\item [\textbf{macro.phraseSeq}] (phrase sequence) defines a sequence of character data and phrase-level elements.
\item [\textbf{macro.phraseSeq.limited}] (limited phrase sequence) defines a sequence of character data and those phrase-level elements that are not typically used for transcribing extant documents.
\item [\textbf{macro.specialPara}] ('special' paragraph content) defines the content model of elements such as notes or list items, which either contain a series of component-level elements or else have the same structure as a paragraph, containing a series of phrase-level and inter-level elements.
\item [\textbf{macro.xtext}] (extended text) defines a sequence of character data and gaiji elements.
\end{sansreflist}
\par
The present version of the TEI Guidelines includes some 589 different elements. \hyperref[tab-content-models]{Table 1} shows, in descending order of frequency, the seven most commonly used content models.\begin{table}\begin{center} \begin{small} \begin{tabular}{P{.25\textwidth}P{.15\textwidth}P{.5\textwidth}}
\rowcolor{label}Content model\tabcellsep Number of elements using this\tabcellsep Description\\\hline 
macro.phraseSeq\tabcellsep 86\tabcellsep defines a sequence of character data and phrase-level elements.\\
macro.paraContent\tabcellsep 53\tabcellsep defines the content of paragraphs and similar elements.\\
macro.specialPara\tabcellsep 33\tabcellsep defines the content model of elements such as notes or list items, which either contain a series of component-level elements or else have the same structure as a paragraph, containing a series of phrase-level and inter-level elements.\\
macro.phraseSeq.limited\tabcellsep 25\tabcellsep defines a sequence of character data and those phrase-level elements that are not typically used for transcribing extant documents.\\
macro.xtext\tabcellsep 10\tabcellsep defines a sequence of character data and gaiji elements.\\
macro.limitedContent\tabcellsep 7\tabcellsep defines the content of prose elements that are not used for transcription of extant materials.\end{tabular} 
      \caption{\label{tab-content-models}}
     \end{small} 
     \end{center}
     \end{table}
\subsubsection[{Datatype Specifications}]{Datatype Specifications}\label{DTYPES}\par
The values which attributes may take in a TEI schema are defined, for the most part, by reference to a TEI \textit{datatype specification}. Each such specification is defined in terms of other primitive datatypes, derived mostly from \hyperref[XSD2]{W3C Schema Datatypes}, literal values, or other datatypes. This indirection makes it possible for a TEI application to set constraints either globally or in individual cases, by redefining the datatype definition or the reference to it respectively. In some cases, the TEI datatype includes additional usage constraints which cannot be enforced by existing schema languages, although a TEI-compliant processor should attempt to validate them (see further discussion in chapter \textit{\hyperref[CF]{23.4.\ Conformance}}).\par
The following element is used to define a TEI datatype: 
\begin{sansreflist}
  
\item [\textbf{<dataSpec>}] (datatype specification) documents a datatype.
\end{sansreflist}
\par
TEI-defined datatypes may be grouped into those which define normalized values for numeric quantities, probabilities, or temporal expressions, those which define various kinds of shorthand codes or keys, and those which define pointers or links.\par
The following datatypes are used for attributes which are intended to hold normalized values of various kinds. First, expressions of quantity or probability: 
\begin{sansreflist}
  
\item {\bfseries \hyperref[TEI.teidata.certainty]{teidata.certainty}} defines the range of attribute values expressing a degree of certainty.
\item {\bfseries \hyperref[TEI.teidata.probability]{teidata.probability}} defines the range of attribute values expressing a probability.
\item {\bfseries \hyperref[TEI.teidata.numeric]{teidata.numeric}} defines the range of attribute values used for numeric values.
\item {\bfseries \hyperref[TEI.teidata.interval]{teidata.interval}} defines attribute values used to express an interval value.
\item {\bfseries \hyperref[TEI.teidata.count]{teidata.count}} defines the range of attribute values used for a non-negative integer value used as a count.
\end{sansreflist}
\par
Examples of attributes using the \textsf{teidata.probability} datatype include {\itshape degree} on \hyperref[TEI.damage]{<damage>} or \hyperref[TEI.certainty]{<certainty>}; examples of \textsf{teidata.numeric} include {\itshape quantity} on members of the \textsf{att.measurement} class or {\itshape value} on \hyperref[TEI.numeric]{<numeric>}; examples of \textsf{teidata.count} include {\itshape cols} on \hyperref[TEI.cell]{<cell>} and \hyperref[TEI.table]{<table>}.\par
Next, the datatypes used for attributes which are intended to hold normalized dates or times, durations, truth values, and language identifiers: 
\begin{sansreflist}
  
\item {\bfseries \hyperref[TEI.teidata.duration.w3c]{teidata.duration.w3c}} defines the range of attribute values available for representation of a duration in time using W3C datatypes.
\item {\bfseries \hyperref[TEI.teidata.temporal.w3c]{teidata.temporal.w3c}} defines the range of attribute values expressing a temporal expression such as a date, a time, or a combination of them, that conform to the W3C \textit{XML Schema Part 2: Datatypes Second Edition} specification.
\item {\bfseries \hyperref[TEI.teidata.temporal.working]{teidata.temporal.working}} defines the range of values, conforming to the W3C \textit{XML Schema Part 2: Datatypes Second Edition} specification, expressing a date or a date and a time within the working life of the document.
\item {\bfseries \hyperref[TEI.teidata.truthValue]{teidata.truthValue}} defines the range of attribute values used to express a truth value.
\item {\bfseries \hyperref[TEI.teidata.xTruthValue]{teidata.xTruthValue}} (extended truth value) defines the range of attribute values used to express a truth value which may be unknown.
\item {\bfseries \hyperref[TEI.teidata.language]{teidata.language}} defines the range of attribute values used to identify a particular combination of human language and writing system.
\end{sansreflist}
\par
Note that in each of these cases the values used are those recommended by existing international standards: ISO 8601 as profiled by \textit{XML Schema Part 2: Datatypes Second Edition} in the case of durations, times, and date; W3C Schema datatypes in the case of truth values; and BCP 47 in the case of language.\par
The following datatypes have more specialized uses: 
\begin{sansreflist}
  
\item {\bfseries \hyperref[TEI.teidata.namespace]{teidata.namespace}} defines the range of attribute values used to indicate XML namespaces as defined by the W3C \xref{http://www.w3.org/TR/1999/REC-xml-names-19990114/}{Namespaces in XML} Technical Recommendation.
\item {\bfseries \hyperref[TEI.teidata.namespaceOrName]{teidata.namespaceOrName}} defines attribute values which contain either an absolute namespace URI or a qualified XML name.
\item {\bfseries \hyperref[TEI.teidata.outputMeasurement]{teidata.outputMeasurement}} defines a range of values for use in specifying the size of an object that is intended for display.
\item {\bfseries \hyperref[TEI.teidata.pattern]{teidata.pattern}} defines attribute values which are expressed as a regular expression.
\item {\bfseries \hyperref[TEI.teidata.point]{teidata.point}} defines the data type used to express a point in cartesian space.
\item {\bfseries \hyperref[TEI.teidata.pointer]{teidata.pointer}} defines the range of attribute values used to provide a single URI, absolute or relative, pointing to some other resource, either within the current document or elsewhere.
\item {\bfseries \hyperref[TEI.teidata.authority]{teidata.authority}} defines attribute values which derive from an authority list, which may be an enumerated list defined in the document's schema, a list or taxonomy elsewhere in the document, or an online taxonomy, gazetteer, or other authority.
\item {\bfseries \hyperref[TEI.teidata.version]{teidata.version}} defines the range of attribute values which may be used to specify a TEI or Unicode version number.
\item {\bfseries \hyperref[TEI.teidata.versionNumber]{teidata.versionNumber}} defines the range of attribute values used for version numbers.
\item {\bfseries \hyperref[TEI.teidata.replacement]{teidata.replacement}} defines attribute values which contain a replacement template.
\item {\bfseries \hyperref[TEI.teidata.xpath]{teidata.xpath}} defines attribute values which contain an XPath expression.
\end{sansreflist}
\par
By far the largest number of TEI attributes take values which are coded values or names of some kind. These values may be constrained or defined in a number of different ways, each of which is given a different name, as follows: 
\begin{sansreflist}
  
\item {\bfseries \hyperref[TEI.teidata.word]{teidata.word}} defines the range of attribute values expressed as a single word or token.
\item {\bfseries \hyperref[TEI.teidata.text]{teidata.text}} defines the range of attribute values used to express some kind of identifying string as a single sequence of Unicode characters possibly including whitespace.
\item {\bfseries \hyperref[TEI.teidata.name]{teidata.name}} defines the range of attribute values expressed as an XML Name.
\item {\bfseries \hyperref[TEI.teidata.enumerated]{teidata.enumerated}} defines the range of attribute values expressed as a single XML name taken from a list of documented possibilities.
\item {\bfseries \hyperref[TEI.teidata.sex]{teidata.sex}} defines the range of attribute values used to identify human or animal sex.
\item {\bfseries \hyperref[TEI.teidata.xmlName]{teidata.xmlName}} defines attribute values which contain an XML name.
\item {\bfseries \hyperref[TEI.teidata.prefix]{teidata.prefix}} defines a range of values that may function as a URI scheme name.
\end{sansreflist}
\par
Attributes of type \textsf{teidata.word}, such as {\itshape age} on \hyperref[TEI.person]{<person>}, are used to supply an identifier expressed as any kind of single token or word. The TEI places a few constraints on the characters which may be used for this purpose: only Unicode characters classified as letters, digits, punctuation characters, or symbols can appear in an attribute value of this kind. Note in particular that such values cannot include whitespace characters. Legal values include cholmondeley, été, 1234, e\textunderscore content, or xml:id, but not grand wazoo. Attributes of this kind are sometimes used to associate (by co-reference) elements of different types.\par
Where identifiers are defined externally, for example as part of a database or file system, the inability to include whitespace or other special characters in a value may be problematic. In other cases, it may also be simply more convenient to supply a short sequence of natural language words including spaces as a single value. For these reasons, we also provide a datatype \textsf{teidata.text} which does permit whitespace and indeed any other Unicode character. Legal values include cholmondeley, été, 1234, e-content, xml:id, and grand wazoo. This datatype should be used with care since XML will not normalize whitespace characters within it: for example the values \texttt{n="a  b"} (two spaces) and \texttt{n="a   b"} (three spaces) would be considered distinct. This case should be distinguished from that of an attribute permitting multiple values, each of which may be separated by whitespace which \textit{will} be normalized (see further \textit{\hyperref[TD-datatypes]{22.5.3.1.\ Datatypes}}).\par
Attributes of type \textsf{teidata.name} are similar to those of type \textsf{teidata.word}, but with the additional constraint that they must be legal XML identifiers, as defined by the XML 1.0 specification, or successors. Hence, they may not begin with digits or punctuation characters. Legal identifiers include cholmondeley, été, e\textunderscore content, or xml:id, but not grand wazoo or 1234. Attributes of this kind are typically used to represent XML element or attribute names.\par
Attributes of type \textsf{teidata.xmlName} are similar to those of type \textsf{teidata.name}, but with the additional constraint that they must not contain a colon character (\textit{:}, U+003A). Thus attributes of this kind are used to represent XML element or attribute names that do not have a namespace prefix.\par
Attributes of type \textsf{teidata.prefix}, such as {\itshape ident} of \hyperref[TEI.prefixDef]{<prefixDef>}, are restricted to strings that form legal URI prefixes.\footnote{Technically \xref{https://tools.ietf.org/html/rfc3986\#section-3.1}{the specification} permits the 26 uppercase letters \textit{A-Z}; however, since ‘the canonical form is lowercase and documents that specify schemes must do so with lowercase letters’, the TEI \textsf{teidata.prefix} datatype does not permit uppercase letters.} Examples of valid values are http, https, tn3270, xmlrpc.beep, and view-source.\par
Attributes of type \textsf{teidata.enumerated}, such as {\itshape new} on \hyperref[TEI.shift]{<shift>} or {\itshape evidence} supplied by \textsf{att.editLike}, have the same definition as \textsf{teidata.word} above, with the added constraint that the word supplied is taken from a specific list of possibilities. In each case, the element or class specification which includes the definition for the attribute will also contain a list of possible values, together with a prose description of their intended significance. This list may be open (in which case the list is advisory), or closed (in which case it determines the range of legal values). In this latter case, the datatype will not be \textsf{teidata.enumerated}, but an explicit list of the possible values.\par
An attribute may, of course, take more than one value of a given type, for example a list of pointer values, or a list of words. In the TEI scheme, this information is regarded as a property of the \hyperref[TEI.datatype]{<datatype>} element used to document the attribute in question rather than as a distinct ‘datatype’, and is provided by the {\itshape minOccurs} or {\itshape maxOccurs} attribute. See further \textit{\hyperref[TD-datatypes]{22.5.3.1.\ Datatypes}}.\par
In a small number of cases, an attribute may take a value of either one datatype or another. These cases are considered as distinct datatypes: 
\begin{sansreflist}
  
\item {\bfseries \hyperref[TEI.teidata.probCert]{teidata.probCert}} defines a range of attribute values which can be expressed either as a numeric probability or as a coded certainty value.
\item {\bfseries \hyperref[TEI.teidata.unboundedInt]{teidata.unboundedInt}} defines an attribute value which can be either any non-negative integer or the string "unbounded".
\item {\bfseries \hyperref[TEI.teidata.nullOrName]{teidata.nullOrName}} defines attribute values which contain either the null string or an XML name.
\end{sansreflist}

\subsection[{The TEI Infrastructure Module}]{The TEI Infrastructure Module}\label{STOV}\par
The \textsf{tei} module defined by this chapter is a required component of any TEI schema. It provides declarations for all datatypes, and initial declarations for the attribute classes, model classes, and macros used by other modules in the TEI scheme. Its components are listed below in alphabetical order: \begin{description}

\item[{Module tei: Declarations for classes, datatypes, and macros available to all TEI modules}]\hspace{1em}\hfill\linebreak
\mbox{}\\[-10pt] \begin{itemize}
\item {\itshape Classes defined}: \hyperref[TEI.att.anchoring]{att.anchoring} \hyperref[TEI.att.ascribed]{att.ascribed} \hyperref[TEI.att.ascribed.directed]{att.ascribed.directed} \hyperref[TEI.att.breaking]{att.breaking} \hyperref[TEI.att.cReferencing]{att.cReferencing} \hyperref[TEI.att.canonical]{att.canonical} \hyperref[TEI.att.citeStructurePart]{att.citeStructurePart} \hyperref[TEI.att.citing]{att.citing} \hyperref[TEI.att.damaged]{att.damaged} \hyperref[TEI.att.datable]{att.datable} \hyperref[TEI.att.datable.w3c]{att.datable.w3c} \hyperref[TEI.att.datcat]{att.datcat} \hyperref[TEI.att.declarable]{att.declarable} \hyperref[TEI.att.declaring]{att.declaring} \hyperref[TEI.att.dimensions]{att.dimensions} \hyperref[TEI.att.divLike]{att.divLike} \hyperref[TEI.att.docStatus]{att.docStatus} \hyperref[TEI.att.duration.iso]{att.duration.iso} \hyperref[TEI.att.duration.w3c]{att.duration.w3c} \hyperref[TEI.att.editLike]{att.editLike} \hyperref[TEI.att.edition]{att.edition} \hyperref[TEI.att.formula]{att.formula} \hyperref[TEI.att.fragmentable]{att.fragmentable} \hyperref[TEI.att.global]{att.global} \hyperref[TEI.att.global.rendition]{att.global.rendition} \hyperref[TEI.att.global.responsibility]{att.global.responsibility} \hyperref[TEI.att.global.source]{att.global.source} \hyperref[TEI.att.handFeatures]{att.handFeatures} \hyperref[TEI.att.internetMedia]{att.internetMedia} att.interpLike \hyperref[TEI.att.measurement]{att.measurement} \hyperref[TEI.att.media]{att.media} \hyperref[TEI.att.naming]{att.naming} \hyperref[TEI.att.notated]{att.notated} \hyperref[TEI.att.partials]{att.partials} \hyperref[TEI.att.personal]{att.personal} \hyperref[TEI.att.placement]{att.placement} \hyperref[TEI.att.pointing]{att.pointing} \hyperref[TEI.att.pointing.group]{att.pointing.group} \hyperref[TEI.att.ranging]{att.ranging} \hyperref[TEI.att.resourced]{att.resourced} \hyperref[TEI.att.scoping]{att.scoping} \hyperref[TEI.att.segLike]{att.segLike} \hyperref[TEI.att.sortable]{att.sortable} \hyperref[TEI.att.spanning]{att.spanning} \hyperref[TEI.att.styleDef]{att.styleDef} \hyperref[TEI.att.timed]{att.timed} \hyperref[TEI.att.transcriptional]{att.transcriptional} \hyperref[TEI.att.typed]{att.typed} \hyperref[TEI.att.written]{att.written} \hyperref[TEI.model.addrPart]{model.addrPart} \hyperref[TEI.model.addressLike]{model.addressLike} \hyperref[TEI.model.annotationLike]{model.annotationLike} \hyperref[TEI.model.annotationPart.body]{model.annotationPart.body} \hyperref[TEI.model.applicationLike]{model.applicationLike} \hyperref[TEI.model.attributable]{model.attributable} \hyperref[TEI.model.availabilityPart]{model.availabilityPart} \hyperref[TEI.model.biblLike]{model.biblLike} \hyperref[TEI.model.biblPart]{model.biblPart} \hyperref[TEI.model.castItemPart]{model.castItemPart} \hyperref[TEI.model.catDescPart]{model.catDescPart} \hyperref[TEI.model.certLike]{model.certLike} \hyperref[TEI.model.choicePart]{model.choicePart} \hyperref[TEI.model.common]{model.common} \hyperref[TEI.model.correspActionPart]{model.correspActionPart} \hyperref[TEI.model.correspContextPart]{model.correspContextPart} \hyperref[TEI.model.correspDescPart]{model.correspDescPart} \hyperref[TEI.model.dateLike]{model.dateLike} \hyperref[TEI.model.descLike]{model.descLike} \hyperref[TEI.model.describedResource]{model.describedResource} \hyperref[TEI.model.dimLike]{model.dimLike} \hyperref[TEI.model.div1Like]{model.div1Like} \hyperref[TEI.model.div2Like]{model.div2Like} \hyperref[TEI.model.div3Like]{model.div3Like} \hyperref[TEI.model.div4Like]{model.div4Like} \hyperref[TEI.model.div5Like]{model.div5Like} \hyperref[TEI.model.div6Like]{model.div6Like} \hyperref[TEI.model.div7Like]{model.div7Like} \hyperref[TEI.model.divBottom]{model.divBottom} \hyperref[TEI.model.divBottomPart]{model.divBottomPart} \hyperref[TEI.model.divGenLike]{model.divGenLike} \hyperref[TEI.model.divLike]{model.divLike} \hyperref[TEI.model.divPart]{model.divPart} \hyperref[TEI.model.divTop]{model.divTop} \hyperref[TEI.model.divTopPart]{model.divTopPart} \hyperref[TEI.model.divWrapper]{model.divWrapper} \hyperref[TEI.model.editorialDeclPart]{model.editorialDeclPart} \hyperref[TEI.model.egLike]{model.egLike} \hyperref[TEI.model.emphLike]{model.emphLike} \hyperref[TEI.model.encodingDescPart]{model.encodingDescPart} \hyperref[TEI.model.entryPart]{model.entryPart} \hyperref[TEI.model.entryPart.top]{model.entryPart.top} \hyperref[TEI.model.eventLike]{model.eventLike} \hyperref[TEI.model.featureVal]{model.featureVal} \hyperref[TEI.model.featureVal.complex]{model.featureVal.complex} \hyperref[TEI.model.featureVal.single]{model.featureVal.single} \hyperref[TEI.model.frontPart]{model.frontPart} \hyperref[TEI.model.frontPart.drama]{model.frontPart.drama} \hyperref[TEI.model.gLike]{model.gLike} \hyperref[TEI.model.global]{model.global} \hyperref[TEI.model.global.edit]{model.global.edit} \hyperref[TEI.model.global.meta]{model.global.meta} \hyperref[TEI.model.glossLike]{model.glossLike} \hyperref[TEI.model.graphicLike]{model.graphicLike} \hyperref[TEI.model.headLike]{model.headLike} \hyperref[TEI.model.hiLike]{model.hiLike} \hyperref[TEI.model.highlighted]{model.highlighted} \hyperref[TEI.model.imprintPart]{model.imprintPart} \hyperref[TEI.model.inter]{model.inter} \hyperref[TEI.model.lLike]{model.lLike} \hyperref[TEI.model.lPart]{model.lPart} \hyperref[TEI.model.labelLike]{model.labelLike} \hyperref[TEI.model.limitedPhrase]{model.limitedPhrase} \hyperref[TEI.model.linePart]{model.linePart} \hyperref[TEI.model.listLike]{model.listLike} \hyperref[TEI.model.measureLike]{model.measureLike} \hyperref[TEI.model.milestoneLike]{model.milestoneLike} \hyperref[TEI.model.msItemPart]{model.msItemPart} \hyperref[TEI.model.msQuoteLike]{model.msQuoteLike} \hyperref[TEI.model.nameLike]{model.nameLike} \hyperref[TEI.model.nameLike.agent]{model.nameLike.agent} \hyperref[TEI.model.noteLike]{model.noteLike} \hyperref[TEI.model.objectLike]{model.objectLike} \hyperref[TEI.model.oddDecl]{model.oddDecl} \hyperref[TEI.model.oddRef]{model.oddRef} \hyperref[TEI.model.offsetLike]{model.offsetLike} \hyperref[TEI.model.orgPart]{model.orgPart} \hyperref[TEI.model.orgStateLike]{model.orgStateLike} \hyperref[TEI.model.pLike]{model.pLike} \hyperref[TEI.model.pLike.front]{model.pLike.front} \hyperref[TEI.model.pPart.data]{model.pPart.data} \hyperref[TEI.model.pPart.edit]{model.pPart.edit} \hyperref[TEI.model.pPart.editorial]{model.pPart.editorial} \hyperref[TEI.model.pPart.msdesc]{model.pPart.msdesc} \hyperref[TEI.model.pPart.transcriptional]{model.pPart.transcriptional} \hyperref[TEI.model.persStateLike]{model.persStateLike} \hyperref[TEI.model.personLike]{model.personLike} \hyperref[TEI.model.personPart]{model.personPart} \hyperref[TEI.model.phrase]{model.phrase} \hyperref[TEI.model.phrase.xml]{model.phrase.xml} \hyperref[TEI.model.placeLike]{model.placeLike} \hyperref[TEI.model.placeNamePart]{model.placeNamePart} \hyperref[TEI.model.placeStateLike]{model.placeStateLike} \hyperref[TEI.model.profileDescPart]{model.profileDescPart} \hyperref[TEI.model.ptrLike]{model.ptrLike} \hyperref[TEI.model.publicationStmtPart.agency]{model.publicationStmtPart.agency} \hyperref[TEI.model.publicationStmtPart.detail]{model.publicationStmtPart.detail} \hyperref[TEI.model.quoteLike]{model.quoteLike} \hyperref[TEI.model.resource]{model.resource} \hyperref[TEI.model.respLike]{model.respLike} \hyperref[TEI.model.segLike]{model.segLike} \hyperref[TEI.model.settingPart]{model.settingPart} \hyperref[TEI.model.sourceDescPart]{model.sourceDescPart} \hyperref[TEI.model.specDescLike]{model.specDescLike} \hyperref[TEI.model.stageLike]{model.stageLike} \hyperref[TEI.model.standOffPart]{model.standOffPart} \hyperref[TEI.model.teiHeaderPart]{model.teiHeaderPart} \hyperref[TEI.model.textDescPart]{model.textDescPart} \hyperref[TEI.model.titlepagePart]{model.titlepagePart}
\item Macros defined: \hyperref[TEI.macro.limitedContent]{macro.limitedContent} \hyperref[TEI.macro.paraContent]{macro.paraContent} \hyperref[TEI.macro.phraseSeq]{macro.phraseSeq} \hyperref[TEI.macro.phraseSeq.limited]{macro.phraseSeq.limited} \hyperref[TEI.macro.specialPara]{macro.specialPara} \hyperref[TEI.macro.xtext]{macro.xtext} \hyperref[TEI.teidata.authority]{teidata.authority} \hyperref[TEI.teidata.certainty]{teidata.certainty} \hyperref[TEI.teidata.count]{teidata.count} \hyperref[TEI.teidata.duration.iso]{teidata.duration.iso} \hyperref[TEI.teidata.duration.w3c]{teidata.duration.w3c} \hyperref[TEI.teidata.enumerated]{teidata.enumerated} \hyperref[TEI.teidata.interval]{teidata.interval} \hyperref[TEI.teidata.language]{teidata.language} \hyperref[TEI.teidata.name]{teidata.name} \hyperref[TEI.teidata.namespace]{teidata.namespace} \hyperref[TEI.teidata.namespaceOrName]{teidata.namespaceOrName} \hyperref[TEI.teidata.nullOrName]{teidata.nullOrName} \hyperref[TEI.teidata.numeric]{teidata.numeric} \hyperref[TEI.teidata.outputMeasurement]{teidata.outputMeasurement} \hyperref[TEI.teidata.pattern]{teidata.pattern} \hyperref[TEI.teidata.point]{teidata.point} \hyperref[TEI.teidata.pointer]{teidata.pointer} \hyperref[TEI.teidata.prefix]{teidata.prefix} \hyperref[TEI.teidata.probCert]{teidata.probCert} \hyperref[TEI.teidata.probability]{teidata.probability} \hyperref[TEI.teidata.replacement]{teidata.replacement} \hyperref[TEI.teidata.sex]{teidata.sex} \hyperref[TEI.teidata.temporal.iso]{teidata.temporal.iso} \hyperref[TEI.teidata.temporal.w3c]{teidata.temporal.w3c} \hyperref[TEI.teidata.temporal.working]{teidata.temporal.working} \hyperref[TEI.teidata.text]{teidata.text} \hyperref[TEI.teidata.truthValue]{teidata.truthValue} \hyperref[TEI.teidata.unboundedInt]{teidata.unboundedInt} \hyperref[TEI.teidata.version]{teidata.version} \hyperref[TEI.teidata.versionNumber]{teidata.versionNumber} \hyperref[TEI.teidata.word]{teidata.word} \hyperref[TEI.teidata.xTruthValue]{teidata.xTruthValue} \hyperref[TEI.teidata.xmlName]{teidata.xmlName} \hyperref[TEI.teidata.xpath]{teidata.xpath}
\end{itemize} 
\end{description} \par
The order in which declarations are made within the infrastructure module is critical, since several class declarations refer to others, which must therefore precede them. Other constraints on the order of declarations derive from the way in which the modularity of the TEI scheme is implemented in different schema languages. The XML DTD fragment implementing this TEI module makes extensive use of \textit{parameter entities} and \textit{marked sections} to effect a kind of conditional construction; the RELAX NG schema fragment similarly predeclares a number of patterns with null (‘notAllowed’) values. These issues are further discussed in chapter \textit{\hyperref[IM]{23.5.\ Implementation of an ODD System}}.