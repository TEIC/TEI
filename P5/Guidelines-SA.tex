
\section[{Linking, Segmentation, and Alignment}]{Linking, Segmentation, and Alignment}\label{SA}\par
This chapter discusses a number of ways in which encoders may represent analyses of the structure of a text which are not necessarily linear or hierarchic. The module defined by this chapter provides for the following common requirements: \begin{itemize}
\item to link disparate elements using the {\itshape xml:id} attribute (section \textit{\hyperref[SAPT]{16.1.\ Links}});
\item to link disparate elements without using the {\itshape xml:id} attribute (sections \textit{\hyperref[SAUR]{16.2.1.\ Pointing Elsewhere}} and \textit{\hyperref[SATS]{16.2.4.\ TEI XPointer Schemes}});
\item to segment text into elements convenient for the encoder and to mark arbitrary points within documents (section \textit{\hyperref[SASE]{16.3.\ Blocks, Segments, and Anchors}});
\item to represent correspondence or alignment among groups of text elements, both those with content and those which are empty (section \textit{\hyperref[SACS]{16.5.\ Correspondence and Alignment}});\footnote{We use the term \textit{alignment} as a special case for the more general notion of correspondence. Using A as a short form for ‘an element with its attribute {\itshape xml:id} set to the value A’, and suppose elements A1, A2, and A3 occur in that order and form one group, while elements B1, B2, and B3 occur in that order and form another group. Then a relation in which A1 corresponds to B1, A2 corresponds to B2, and A3 corresponds to B3 is an alignment. On the other hand, a relation in which A1 corresponds to B2, B1 to C2, and C1 to A2 is not an alignment.}
\item to synchronize elements of a text, that is to represent temporal correspondences and alignments among text elements (section \textit{\hyperref[SASY]{16.4.\ Synchronization}}) and also to align them with specific points in time (section \textit{\hyperref[SASYMP]{16.4.2.\ Placing Synchronous Events in Time}});
\item to specify that one text element is identical to or a copy of another (section \textit{\hyperref[SAIE]{16.6.\ Identical Elements and Virtual Copies}});
\item to aggregate possibly noncontiguous elements (section \textit{\hyperref[SAAG]{16.7.\ Aggregation}});
\item to specify that different elements are alternatives to one another and to express preferences among the alternatives (section \textit{\hyperref[SAAT]{16.8.\ Alternation}});
\item to store markup separately from the data it describes or is related to (section \textit{\hyperref[SASO]{16.9.\ Stand-off Markup}});
\item to associate segments of a text with interpretations or analyses of their significance (section \textit{\hyperref[SAAN]{16.12.\ Connecting Analytic and Textual Markup}});
\item to group together elements used to provide stand-off annotation, including contextual information (section \textit{\hyperref[SASOstdf]{16.10.\ The standOff Container}}).
\end{itemize} \par
These facilities all use the same set of techniques based on the W3C XPointer framework (\cite{XPTRFMWK}) This provides a variety of \textit{schemes}; the most convenient of which, and that recommended by these Guidelines, makes use of the global {\itshape xml:id} attribute, as defined in section \textit{\hyperref[STGA]{1.3.1.1.\ Global Attributes}}, and introduced in the section of \textit{\hyperref[SG]{v\ A Gentle Introduction to XML}} titled \textit{\hyperref[SG-id]{v.6.2\ Identifiers and Indicators}}. When the \textsf{linking} module is included in a schema, the attribute class \textsf{att.global} is extended to include eight additional attributes to support the various kinds of linking listed above. Each of these attributes is introduced in the appropriate section below. In addition, for many of the topics discussed, a choice of methods of encoding is offered, ranging from simple but less general ones, which use attribute values only, to more elaborate and more general ones, which use specialized elements.
\subsection[{Links}]{Links}\label{SAPT}\par
We say that one element \textit{points} to others if the first has an attribute whose value is a reference to the others: such an element is called a \textit{pointer element}, or simply a \textit{pointer}. Among the pointers that have been introduced up to this point in these Guidelines are \hyperref[TEI.note]{<note>}, \hyperref[TEI.ref]{<ref>}, and \hyperref[TEI.ptr]{<ptr>}. These elements all indicate an association between one place in the document (the location of the pointer itself) and one or more others (the elements whose identifiers are specified by the pointer's {\itshape target} attribute). The module described in this chapter introduces a variation on this basic kind of pointer, known as a \textit{link}, which specifies both ‘ends’ of an association. In addition, we define a syntax for representing locations in a document by a variety of means not dependent on the use of {\itshape xml:id} attributes.
\subsubsection[{Pointers and Links}]{Pointers and Links}\label{SAPTL}\par
In section \textit{\hyperref[COXR]{3.7.\ Simple Links and Cross-References}} we introduced the simplest pointer elements, \hyperref[TEI.ptr]{<ptr>} and \hyperref[TEI.ref]{<ref>}. Here we introduce additionally the \hyperref[TEI.link]{<link>} element, which represents an association between two (or more) locations by specifying each location explicitly. Its own location is irrelevant to the intended linkage. All three elements use the attribute {\itshape target}, provided by the \textsf{att.pointing} class as a means of indicating the location or locations referenced or pointed to. 
\begin{sansreflist}
  
\item [\textbf{att.pointing}] provides a set of attributes used by all elements which point to other elements by means of one or more URI references.\hfil\\[-10pt]\begin{sansreflist}
    \item[@{\itshape target}]
  specifies the destination of the reference by supplying one or more URI References
\end{sansreflist}  
\item [\textbf{<link>}] (link) defines an association or hypertextual link among elements or passages, of some type not more precisely specifiable by other elements.
\end{sansreflist}
 The \hyperref[TEI.ptr]{<ptr>} element may be called a ‘pure pointer’, because its primary function is simply to point. A pointer sets up a \textit{connection} between an element (which, in the case of a pure pointer, is simply a location in a document), and one or more others, known collectively as its \textit{target}. The \hyperref[TEI.ptr]{<ptr>} and \hyperref[TEI.ref]{<ref>} elements  point, conceptually, at a single target, even if that target may be discontinuous in the document. The \hyperref[TEI.link]{<link>} element  specifies at least two targets and represents an association between them, independent of its own location.\par
These three elements also share a common set of attributes, derived from the \textsf{att.pointing} and \textsf{att.typed} classes: 
\begin{sansreflist}
  
\item [\textbf{att.pointing}] provides a set of attributes used by all elements which point to other elements by means of one or more URI references.\hfil\\[-10pt]\begin{sansreflist}
    \item[@{\itshape evaluate}]
  (evaluate) specifies the intended meaning when the target of a pointer is itself a pointer.
\end{sansreflist}  
\item [\textbf{att.typed}] provides attributes which can be used to classify or subclassify elements in any way.\hfil\\[-10pt]\begin{sansreflist}
    \item[@{\itshape type}]
  characterizes the element in some sense, using any convenient classification scheme or typology.
    \item[@{\itshape subtype}]
  (subtype) provides a sub-categorization of the element, if needed
\end{sansreflist}  
\end{sansreflist}
\par
Double connection among elements could also be expressed by a combination of pointer elements, for example, two \hyperref[TEI.ptr]{<ptr>} elements, or one \hyperref[TEI.ptr]{<ptr>} element and one \hyperref[TEI.note]{<note>} element. All that is required is that the value of the {\itshape target} (or other pointing) attribute of the one be the value of the {\itshape xml:id} attribute of the other. What the \hyperref[TEI.link]{<link>} element accomplishes is the handling of double connection by means of a single element. Thus, in the following encoding: \par\bgroup\index{ptr=<ptr>|exampleindex}\index{target=@target!<ptr>|exampleindex}\index{ptr=<ptr>|exampleindex}\index{target=@target!<ptr>|exampleindex}\exampleFont \begin{shaded}\noindent\mbox{}{<\textbf{ptr}\hspace*{1em}{xml:id}="{sa-p1}"\hspace*{1em}{target}="{\#sa-p2}"/>}\mbox{}\newline 
{<\textbf{ptr}\hspace*{1em}{xml:id}="{sa-p2}"\hspace*{1em}{target}="{\#sa-p1}"/>}\end{shaded}\egroup\par \noindent  sa-p1 points to sa-p2, and sa-p2 points to sa-p1. This is logically equivalent to the more compact encoding: \par\bgroup\index{link=<link>|exampleindex}\index{target=@target!<link>|exampleindex}\exampleFont \begin{shaded}\noindent\mbox{}{<\textbf{link}\hspace*{1em}{target}="{\#sa-p1 \#sa-p2}"/>}\end{shaded}\egroup\par \par
As noted elsewhere, the {\itshape target}  attribute may take as value one or more URI reference. In the simplest case, each such reference will indicate an element in the current document (or in some other document), for example by supplying the value used for its global {\itshape xml:id} attribute. It may however carry as value any form of URI, such as a URL pointing to some other document or location on the Internet. Pointing or linking to external documents and pointing and linking where identifiers are not available is described below in section \textit{\hyperref[SAXP]{16.2.\ Pointing Mechanisms}}.
\subsubsection[{Using Pointers and Links}]{Using Pointers and Links}\label{SAPTEG}\par
As an example of the use of mechanisms which establish connections among elements, consider the practice (common in 18th century English verse and elsewhere) of providing footnotes citing parallel passages from classical authors. \begin{figure}[htbp]
\noindent\noindent\includegraphics[]{Images/dunpic.png}\end{figure}
 Such footnotes can of course simply be encoded using the \hyperref[TEI.note]{<note>} element (see section \textit{\hyperref[CONO]{3.9.\ Notes, Annotation, and Indexing}}) without a {\itshape target} attribute, placed adjacent to the passage to which the note refers:\footnote{The {\itshape type} attribute on the note is used to classify the notes using the typology established in the Advertisement to the work: ‘The \textit{Imitations} of the Ancients are added, to gratify those who either never read, or may have forgotten them; together with some of the Parodies, and Allusions to the most excellent of the Moderns.’ In the source text, the text of the poem shares the page with two sets of notes, one headed ‘Remarks’ and the other ‘Imitations’.} \par\bgroup\index{l=<l>|exampleindex}\index{l=<l>|exampleindex}\index{l=<l>|exampleindex}\index{note=<note>|exampleindex}\index{type=@type!<note>|exampleindex}\index{place=@place!<note>|exampleindex}\index{bibl=<bibl>|exampleindex}\index{quote=<quote>|exampleindex}\index{l=<l>|exampleindex}\index{l=<l>|exampleindex}\index{l=<l>|exampleindex}\exampleFont \begin{shaded}\noindent\mbox{}{<\textbf{l}>}(Diff'rent our parties, but with equal grace{</\textbf{l}>}\mbox{}\newline 
{<\textbf{l}>}The Goddess smiles on Whig and Tory race,{</\textbf{l}>}\mbox{}\newline 
{<\textbf{l}>}\mbox{}\newline 
\hspace*{1em}{<\textbf{note}\hspace*{1em}{type}="{imitation}"\hspace*{1em}{place}="{bottom}">}\mbox{}\newline 
\hspace*{1em}\hspace*{1em}{<\textbf{bibl}>}Virg. Æn. 10.{</\textbf{bibl}>}\mbox{}\newline 
\hspace*{1em}\hspace*{1em}{<\textbf{quote}>}\mbox{}\newline 
\hspace*{1em}\hspace*{1em}\hspace*{1em}{<\textbf{l}>}Tros Rutulusve fuat; nullo discrimine habebo.{</\textbf{l}>}\mbox{}\newline 
\hspace*{1em}\hspace*{1em}\hspace*{1em}{<\textbf{l}>}—— Rex Jupiter omnibus idem.{</\textbf{l}>}\mbox{}\newline 
\hspace*{1em}\hspace*{1em}{</\textbf{quote}>}\mbox{}\newline 
\hspace*{1em}{</\textbf{note}>}'Tis the same rope at sev'ral ends they twist,\mbox{}\newline 
{</\textbf{l}>}\mbox{}\newline 
{<\textbf{l}>}To Dulness, Ridpath is as dear as Mist){</\textbf{l}>}\end{shaded}\egroup\par \noindent  \par
This use of the \hyperref[TEI.note]{<note>} element can be called \textit{implicit pointing} (or \textit{implicit linking}). It relies on the juxtaposition of the note to the text being commented on for the connection to be understood. If it is felt that the mere juxtaposition of the note to the text does not make it sufficiently clear exactly what text segment is being commented on (for example, is it the immediately preceding line, or the immediately preceding two lines, or what?), or if it is decided to place the note at some distance from the text, then the pointing or the linking must be made explicit. We now consider various methods for doing that.\par
Firstly, a \hyperref[TEI.ptr]{<ptr>} element might be placed at an appropriate point within the text to link it with the annotation: \par\bgroup\index{l=<l>|exampleindex}\index{l=<l>|exampleindex}\index{ptr=<ptr>|exampleindex}\index{rend=@rend!<ptr>|exampleindex}\index{target=@target!<ptr>|exampleindex}\index{l=<l>|exampleindex}\index{l=<l>|exampleindex}\index{note=<note>|exampleindex}\index{type=@type!<note>|exampleindex}\index{place=@place!<note>|exampleindex}\index{bibl=<bibl>|exampleindex}\index{quote=<quote>|exampleindex}\index{l=<l>|exampleindex}\index{l=<l>|exampleindex}\exampleFont \begin{shaded}\noindent\mbox{}{<\textbf{l}>}(Diff'rent our parties, but with equal grace{</\textbf{l}>}\mbox{}\newline 
{<\textbf{l}>}The Goddess smiles on Whig and Tory race,\mbox{}\newline 
{<\textbf{ptr}\hspace*{1em}{rend}="{unmarked}"\hspace*{1em}{target}="{\#note3.284}"/>}\mbox{}\newline 
{</\textbf{l}>}\mbox{}\newline 
{<\textbf{l}>}'Tis the same rope at sev'ral ends they twist,{</\textbf{l}>}\mbox{}\newline 
{<\textbf{l}>}To Dulness, Ridpath is as dear as Mist){</\textbf{l}>}\mbox{}\newline 
{<\textbf{note}\hspace*{1em}{xml:id}="{note3.284}"\hspace*{1em}{type}="{imitation}"\mbox{}\newline 
\hspace*{1em}{place}="{bottom}">}\mbox{}\newline 
\hspace*{1em}{<\textbf{bibl}>}Virg. Æn. 10.{</\textbf{bibl}>}\mbox{}\newline 
\hspace*{1em}{<\textbf{quote}>}\mbox{}\newline 
\hspace*{1em}\hspace*{1em}{<\textbf{l}>}Tros Rutulusve fuat; nullo discrimine habebo.{</\textbf{l}>}\mbox{}\newline 
\hspace*{1em}\hspace*{1em}{<\textbf{l}>}—— Rex Jupiter omnibus idem.{</\textbf{l}>}\mbox{}\newline 
\hspace*{1em}{</\textbf{quote}>}\mbox{}\newline 
{</\textbf{note}>}\end{shaded}\egroup\par \noindent   The \hyperref[TEI.note]{<note>} element has been given an arbitrary identifier (note3.284) to enable it to be specified as the target of the pointer element. Because there is nothing in the text to signal the existence of the annotation, the {\itshape rend} attribute has been given the value unmarked.\par
Secondly, the {\itshape target} attribute of the \hyperref[TEI.note]{<note>} element can be used to point at its associated text, provided that an {\itshape xml:id} attribute has been supplied for the associated text: \par\bgroup\index{l=<l>|exampleindex}\index{l=<l>|exampleindex}\index{l=<l>|exampleindex}\index{l=<l>|exampleindex}\exampleFont \begin{shaded}\noindent\mbox{}{<\textbf{l}\hspace*{1em}{xml:id}="{L3.283}">}(Diff'rent our parties, but with equal grace{</\textbf{l}>}\mbox{}\newline 
{<\textbf{l}\hspace*{1em}{xml:id}="{L3.284}">}The Goddess smiles on Whig and Tory race,{</\textbf{l}>}\mbox{}\newline 
{<\textbf{l}\hspace*{1em}{xml:id}="{L3.285}">}'Tis the same rope at sev'ral ends they twist,{</\textbf{l}>}\mbox{}\newline 
{<\textbf{l}\hspace*{1em}{xml:id}="{L3.286}">}To Dulness, Ridpath is as dear as Mist){</\textbf{l}>}\mbox{}\newline 
\textit{<!-- ... -->}\end{shaded}\egroup\par \noindent  Given this encoding of the text itself, we can now link the various notes to it. In this case, the note itself contains a pointer to the place in the text which it is annotating; this could be encoded using a \hyperref[TEI.ref]{<ref>} element, which bears a {\itshape target} attribute of its own and contains a (slightly misquoted) extract from the text marked as a \hyperref[TEI.quote]{<quote>} element: \par\bgroup\index{note=<note>|exampleindex}\index{type=@type!<note>|exampleindex}\index{place=@place!<note>|exampleindex}\index{target=@target!<note>|exampleindex}\index{ref=<ref>|exampleindex}\index{rend=@rend!<ref>|exampleindex}\index{target=@target!<ref>|exampleindex}\index{quote=<quote>|exampleindex}\index{l=<l>|exampleindex}\index{l=<l>|exampleindex}\index{bibl=<bibl>|exampleindex}\index{quote=<quote>|exampleindex}\index{l=<l>|exampleindex}\index{l=<l>|exampleindex}\exampleFont \begin{shaded}\noindent\mbox{}{<\textbf{note}\hspace*{1em}{type}="{imitation}"\hspace*{1em}{place}="{bottom}"\mbox{}\newline 
\hspace*{1em}{target}="{\#L3.284}">}\mbox{}\newline 
\hspace*{1em}{<\textbf{ref}\hspace*{1em}{rend}="{sc}"\hspace*{1em}{target}="{\#L3.284}">}Verse 283–84.\mbox{}\newline 
\hspace*{1em}{<\textbf{quote}>}\mbox{}\newline 
\hspace*{1em}\hspace*{1em}\hspace*{1em}{<\textbf{l}>}——. With equal grace{</\textbf{l}>}\mbox{}\newline 
\hspace*{1em}\hspace*{1em}\hspace*{1em}{<\textbf{l}>}Our Goddess smiles on Whig and Tory race.{</\textbf{l}>}\mbox{}\newline 
\hspace*{1em}\hspace*{1em}{</\textbf{quote}>}\mbox{}\newline 
\hspace*{1em}{</\textbf{ref}>}\mbox{}\newline 
\hspace*{1em}{<\textbf{bibl}>}Virg. Æn. 10.{</\textbf{bibl}>}\mbox{}\newline 
\hspace*{1em}{<\textbf{quote}>}\mbox{}\newline 
\hspace*{1em}\hspace*{1em}{<\textbf{l}>}Tros Rutulusve fuat; nullo discrimine habebo.{</\textbf{l}>}\mbox{}\newline 
\hspace*{1em}\hspace*{1em}{<\textbf{l}>}—— Rex Jupiter omnibus idem. {</\textbf{l}>}\mbox{}\newline 
\hspace*{1em}{</\textbf{quote}>}\mbox{}\newline 
{</\textbf{note}>}\end{shaded}\egroup\par \noindent  \par
Combining these two approaches gives us the following associations: \begin{itemize}
\item a pointer within one line indicates the note
\item the note indicates the line
\item a pointer within the note indicates the line
\end{itemize}  Note that we do not have any way of pointing from the line itself to the note: the association is implied by containment of the pointer. We do not as yet have a true double link between text and note. To achieve that we will need to supply identifiers for the annotations as well as for the verse lines, and use a \hyperref[TEI.link]{<link>} element to associate the two. Note that the \hyperref[TEI.ptr]{<ptr>} element and the {\itshape target} attribute on the \hyperref[TEI.note]{<note>} may now be dispensed with: \par\bgroup\index{note=<note>|exampleindex}\index{type=@type!<note>|exampleindex}\index{place=@place!<note>|exampleindex}\index{ref=<ref>|exampleindex}\index{rend=@rend!<ref>|exampleindex}\index{target=@target!<ref>|exampleindex}\index{quote=<quote>|exampleindex}\index{l=<l>|exampleindex}\index{l=<l>|exampleindex}\index{bibl=<bibl>|exampleindex}\index{quote=<quote>|exampleindex}\index{l=<l>|exampleindex}\index{l=<l>|exampleindex}\index{link=<link>|exampleindex}\index{target=@target!<link>|exampleindex}\exampleFont \begin{shaded}\noindent\mbox{}{<\textbf{note}\hspace*{1em}{xml:id}="{n3.284}"\hspace*{1em}{type}="{imitation}"\mbox{}\newline 
\hspace*{1em}{place}="{bottom}">}\mbox{}\newline 
\hspace*{1em}{<\textbf{ref}\hspace*{1em}{rend}="{sc}"\hspace*{1em}{target}="{\#L3.284}">}Verse 283–84.\mbox{}\newline 
\hspace*{1em}{<\textbf{quote}>}\mbox{}\newline 
\hspace*{1em}\hspace*{1em}\hspace*{1em}{<\textbf{l}>}——. With equal grace{</\textbf{l}>}\mbox{}\newline 
\hspace*{1em}\hspace*{1em}\hspace*{1em}{<\textbf{l}>}Our Goddess smiles on Whig and Tory race.{</\textbf{l}>}\mbox{}\newline 
\hspace*{1em}\hspace*{1em}{</\textbf{quote}>}\mbox{}\newline 
\hspace*{1em}{</\textbf{ref}>}\mbox{}\newline 
\hspace*{1em}{<\textbf{bibl}>}Virg. Æn. 10.{</\textbf{bibl}>}\mbox{}\newline 
\hspace*{1em}{<\textbf{quote}>}\mbox{}\newline 
\hspace*{1em}\hspace*{1em}{<\textbf{l}>}Tros Rutulusve fuat; nullo discrimine habebo.{</\textbf{l}>}\mbox{}\newline 
\hspace*{1em}\hspace*{1em}{<\textbf{l}>}—— Rex Jupiter omnibus idem. {</\textbf{l}>}\mbox{}\newline 
\hspace*{1em}{</\textbf{quote}>}\mbox{}\newline 
{</\textbf{note}>}\mbox{}\newline 
{<\textbf{link}\hspace*{1em}{target}="{\#n3.284 \#L3.284}"/>}\end{shaded}\egroup\par \noindent  \par
The {\itshape target} attribute of the \hyperref[TEI.link]{<link>} element here bears the identifier of the note followed by that of the verse line. We could also allocate an identifier to the reference within the note and encode the association between it and the verse line in the same way: \par\bgroup\index{note=<note>|exampleindex}\index{type=@type!<note>|exampleindex}\index{place=@place!<note>|exampleindex}\index{ref=<ref>|exampleindex}\index{rend=@rend!<ref>|exampleindex}\index{target=@target!<ref>|exampleindex}\index{quote=<quote>|exampleindex}\index{l=<l>|exampleindex}\index{l=<l>|exampleindex}\index{link=<link>|exampleindex}\index{target=@target!<link>|exampleindex}\exampleFont \begin{shaded}\noindent\mbox{}{<\textbf{note}\hspace*{1em}{type}="{imitation}"\hspace*{1em}{place}="{bottom}">}\mbox{}\newline 
\hspace*{1em}{<\textbf{ref}\hspace*{1em}{rend}="{sc}"\hspace*{1em}{xml:id}="{r3.284}"\mbox{}\newline 
\hspace*{1em}\hspace*{1em}{target}="{\#L3.284}">}Verse 283–84.\mbox{}\newline 
\hspace*{1em}{<\textbf{quote}>}\mbox{}\newline 
\hspace*{1em}\hspace*{1em}\hspace*{1em}{<\textbf{l}>}——. With equal grace{</\textbf{l}>}\mbox{}\newline 
\hspace*{1em}\hspace*{1em}\hspace*{1em}{<\textbf{l}>}Our Goddess smiles on Whig and Tory race.{</\textbf{l}>}\mbox{}\newline 
\hspace*{1em}\hspace*{1em}{</\textbf{quote}>}\mbox{}\newline 
\hspace*{1em}{</\textbf{ref}>}\mbox{}\newline 
\textit{<!-- ... -->}\mbox{}\newline 
{</\textbf{note}>}\mbox{}\newline 
\textit{<!-- ... -->}\mbox{}\newline 
{<\textbf{link}\hspace*{1em}{target}="{\#r3.284 \#L3.284}"/>}\end{shaded}\egroup\par \noindent   Indeed, the two \hyperref[TEI.link]{<link>}s could be combined into one, as follows: \par\bgroup\index{link=<link>|exampleindex}\index{target=@target!<link>|exampleindex}\exampleFont \begin{shaded}\noindent\mbox{}{<\textbf{link}\hspace*{1em}{target}="{\#n3.284 \#r3.284 \#L3.284}"/>}\end{shaded}\egroup\par 
\subsubsection[{Groups of Links}]{Groups of Links}\label{SAPTLG}\par
Clearly, there are many reasons for which an encoder might wish to represent a link or association between different elements. For some of them, specific elements are provided in these Guidelines; some of these are discussed elsewhere in the present chapter. The \hyperref[TEI.link]{<link>} element is a general purpose element which may be used for any kind of association. The element \hyperref[TEI.linkGrp]{<linkGrp>} may be used to group links of a particular type together in a single part of the document; such a collection may be used to represent what is sometimes referred to in the literature of Hypertext as a \textit{web}, a term introduced by the Brown University FRESS project in 1969, and not to be confused with the World Wide Web. 
\begin{sansreflist}
  
\item [\textbf{<linkGrp>}] (link group) defines a collection of associations or hypertextual links.
\end{sansreflist}
 As a member of the class \textsf{att.pointing.group}, this element shares the following attributes with other members of that class: 
\begin{sansreflist}
  
\item [\textbf{att.pointing.group}] provides a set of attributes common to all elements which enclose groups of pointer elements.\hfil\\[-10pt]\begin{sansreflist}
    \item[@{\itshape domains}]
  optionally specifies the identifiers of the elements within which all elements indicated by the contents of this element lie.
    \item[@{\itshape targFunc}]
  (target function) describes the function of each of the values of the {\itshape target} attribute of the enclosed \hyperref[TEI.link]{<link>}, \hyperref[TEI.join]{<join>}, or \hyperref[TEI.alt]{<alt>} tags.
\end{sansreflist}  
\end{sansreflist}
 It is also a member of the \textsf{att.pointing} and \textsf{att.typed} classes, and therefore also carries the attributes specified in section \textit{\hyperref[SAPTL]{16.1.1.\ Pointers and Links}} above, in particular the {\itshape type} attribute.\par
The \hyperref[TEI.linkGrp]{<linkGrp>} element provides a convenient way of establishing a default for the {\itshape type} attribute on a group of links of the same type: by default, the {\itshape type} attribute on a \hyperref[TEI.link]{<link>} element has the same value as that given for {\itshape type} on the enclosing \hyperref[TEI.linkGrp]{<linkGrp>}.\par
Typical software might hide a web entirely from the user, but use it as a source of information about links, which are displayed independently at their referenced locations. Alternatively, software might provide a direct view of the link collection, along with added functions for manipulating the collection, as by filtering, sorting, and so on. To continue our previous example, this text contains many other notes of a kind similar to the one shown above. Here are a few more of the lines to which annotations have to be attached, followed by the annotations themselves: \par\bgroup\index{l=<l>|exampleindex}\index{l=<l>|exampleindex}\index{l=<l>|exampleindex}\index{note=<note>|exampleindex}\index{place=@place!<note>|exampleindex}\index{anchored=@anchored!<note>|exampleindex}\index{bibl=<bibl>|exampleindex}\index{quote=<quote>|exampleindex}\index{l=<l>|exampleindex}\index{l=<l>|exampleindex}\index{note=<note>|exampleindex}\index{place=@place!<note>|exampleindex}\index{anchored=@anchored!<note>|exampleindex}\index{bibl=<bibl>|exampleindex}\exampleFont \begin{shaded}\noindent\mbox{}{<\textbf{l}\hspace*{1em}{xml:id}="{L2.79}">}A place there is, betwixt earth, air and seas{</\textbf{l}>}\mbox{}\newline 
{<\textbf{l}\hspace*{1em}{xml:id}="{L2.80}">}Where from Ambrosia, Jove retires for ease.{</\textbf{l}>}\mbox{}\newline 
\textit{<!-- ... -->}\mbox{}\newline 
{<\textbf{l}\hspace*{1em}{xml:id}="{L2.88}">}Sign'd with that Ichor which from Gods distills.{</\textbf{l}>}\mbox{}\newline 
\textit{<!-- ... -->}\mbox{}\newline 
{<\textbf{note}\hspace*{1em}{xml:id}="{n2.79}"\hspace*{1em}{place}="{bottom}"\mbox{}\newline 
\hspace*{1em}{anchored}="{false}">}\mbox{}\newline 
\hspace*{1em}{<\textbf{bibl}>}Ovid Met. 12.{</\textbf{bibl}>}\mbox{}\newline 
\hspace*{1em}{<\textbf{quote}\hspace*{1em}{xml:lang}="{la}">}\mbox{}\newline 
\hspace*{1em}\hspace*{1em}{<\textbf{l}>}Orbe locus media est, inter terrasq; fretumq;{</\textbf{l}>}\mbox{}\newline 
\hspace*{1em}\hspace*{1em}{<\textbf{l}>}Cœlestesq; plagas —{</\textbf{l}>}\mbox{}\newline 
\hspace*{1em}{</\textbf{quote}>}\mbox{}\newline 
{</\textbf{note}>}\mbox{}\newline 
{<\textbf{note}\hspace*{1em}{xml:id}="{n2.88}"\hspace*{1em}{place}="{bottom}"\mbox{}\newline 
\hspace*{1em}{anchored}="{false}">} Alludes to {<\textbf{bibl}>}Homer, Iliad 5{</\textbf{bibl}>} ...\mbox{}\newline 
{</\textbf{note}>}\end{shaded}\egroup\par \noindent  To avoid having to repeat the specification of {\itshape type} as imitation on each \hyperref[TEI.note]{<note>}, we may specify it once for all on a \hyperref[TEI.linkGrp]{<linkGrp>} element containing all links of this type. \par\bgroup\index{linkGrp=<linkGrp>|exampleindex}\index{type=@type!<linkGrp>|exampleindex}\index{link=<link>|exampleindex}\index{target=@target!<link>|exampleindex}\index{link=<link>|exampleindex}\index{target=@target!<link>|exampleindex}\index{link=<link>|exampleindex}\index{target=@target!<link>|exampleindex}\exampleFont \begin{shaded}\noindent\mbox{}{<\textbf{linkGrp}\hspace*{1em}{type}="{imitation}">}\mbox{}\newline 
\hspace*{1em}{<\textbf{link}\hspace*{1em}{target}="{\#n2.79 \#L2.79}"/>}\mbox{}\newline 
\hspace*{1em}{<\textbf{link}\hspace*{1em}{target}="{\#n2.88 \#L2.88}"/>}\mbox{}\newline 
\hspace*{1em}{<\textbf{link}\hspace*{1em}{target}="{\#n3.284 \#L3.284}"/>}\mbox{}\newline 
{</\textbf{linkGrp}>}\end{shaded}\egroup\par \par
Additional information for applications that use \hyperref[TEI.linkGrp]{<linkGrp>} elements can be provided by means of special attributes. First, the {\itshape domains} attribute can be used to identify the text elements within which the individual targets of the links are to be found. Suppose that the text under discussion is organized into a \hyperref[TEI.body]{<body>} element, containing the text of the poem, and a \hyperref[TEI.back]{<back>} element containing the notes. Then the {\itshape domains} attribute can have as its value the identifiers of the \hyperref[TEI.body]{<body>} and the \hyperref[TEI.back]{<back>}, to enable an application to verify that the link targets are in fact contained by appropriate elements, or to limit its search space: \par\bgroup\index{linkGrp=<linkGrp>|exampleindex}\index{type=@type!<linkGrp>|exampleindex}\index{domains=@domains!<linkGrp>|exampleindex}\index{link=<link>|exampleindex}\index{target=@target!<link>|exampleindex}\index{link=<link>|exampleindex}\index{target=@target!<link>|exampleindex}\index{link=<link>|exampleindex}\index{target=@target!<link>|exampleindex}\exampleFont \begin{shaded}\noindent\mbox{}\mbox{}\newline 
\textit{<!-- ... -->}{<\textbf{linkGrp}\hspace*{1em}{type}="{imitation}"\mbox{}\newline 
\hspace*{1em}{domains}="{\#dunciad \#dunnotes}">}\mbox{}\newline 
\hspace*{1em}{<\textbf{link}\hspace*{1em}{target}="{\#n2.79 \#L2.79}"/>}\mbox{}\newline 
\hspace*{1em}{<\textbf{link}\hspace*{1em}{target}="{\#n2.88 \#L2.88}"/>}\mbox{}\newline 
\textit{<!-- ... -->}\mbox{}\newline 
\hspace*{1em}{<\textbf{link}\hspace*{1em}{target}="{\#n3.284 \#L3.284}"/>}\mbox{}\newline 
\textit{<!-- ... -->}\mbox{}\newline 
{</\textbf{linkGrp}>}\end{shaded}\egroup\par \par
Note that there must be a single parent element for each ‘domain’; if some notes are contained by a section with identifier dunnotes, and others by a section with identifier dunimits, an intermediate pointer must be provided (as described in section \textit{\hyperref[SAPTIP]{16.1.4.\ Intermediate Pointers}}) within the \hyperref[TEI.linkGrp]{<linkGrp>} and its identifier used instead.\par
Next, the {\itshape targFunc} attribute can be used to provide further information about the role or function of the various targets specified for each link in the group. The value of the {\itshape targFunc} attribute is a list of names (formally, name tokens), one for each of the targets in the link; these names can be chosen freely by the encoder, but their significance should be documented in the encoding description in the header.\footnote{Since no special element is provided for this purpose in the present version of these Guidelines, the information should be supplied as a series of paragraphs at the end of the \hyperref[TEI.encodingDesc]{<encodingDesc>} element described in section \textit{\hyperref[HD5]{2.3.\ The Encoding Description}}.} In the current example, we might think of the note as containing the \textit{source} of the imitation and the verse line as containing the \textit{goal} of the imitation. Accordingly, we can specify the \hyperref[TEI.linkGrp]{<linkGrp>} in the preceding example thus: \par\bgroup\index{linkGrp=<linkGrp>|exampleindex}\index{type=@type!<linkGrp>|exampleindex}\index{domains=@domains!<linkGrp>|exampleindex}\index{targFunc=@targFunc!<linkGrp>|exampleindex}\index{link=<link>|exampleindex}\index{target=@target!<link>|exampleindex}\index{link=<link>|exampleindex}\index{target=@target!<link>|exampleindex}\index{link=<link>|exampleindex}\index{target=@target!<link>|exampleindex}\exampleFont \begin{shaded}\noindent\mbox{}{<\textbf{linkGrp}\hspace*{1em}{type}="{imitation}"\mbox{}\newline 
\hspace*{1em}{domains}="{\#dunciad \#dunnotes}"\hspace*{1em}{targFunc}="{source goal}">}\mbox{}\newline 
\hspace*{1em}{<\textbf{link}\hspace*{1em}{target}="{\#n2.79 \#L2.79}"/>}\mbox{}\newline 
\hspace*{1em}{<\textbf{link}\hspace*{1em}{target}="{\#n2.88 \#L2.88}"/>}\mbox{}\newline 
\textit{<!-- ... -->}\mbox{}\newline 
\hspace*{1em}{<\textbf{link}\hspace*{1em}{target}="{\#n3.284 \#L3.284}"/>}\mbox{}\newline 
\textit{<!-- ... -->}\mbox{}\newline 
{</\textbf{linkGrp}>}\end{shaded}\egroup\par 
\subsubsection[{Intermediate Pointers}]{Intermediate Pointers}\label{SAPTIP}\par
In the preceding examples, we have shown various ways of linking an annotation and a single verse line. However, the example cited in fact requires us to encode an association between the note and a \textit{pair} of verse lines (lines 284 and 285); we call these two lines a \textit{span}.\par
There are a number of possible ways of correcting this error: one could use the {\itshape target} attribute to indicate one end of the span and the special purpose {\itshape targetEnd} attribute on the \hyperref[TEI.note]{<note>} element to point to the other. Another possibility might be to create an element which represents the whole span itself and assign that an {\itshape xml:id} attribute, which can then be linked to the \hyperref[TEI.note]{<note>} and \hyperref[TEI.ref]{<ref>} elements. This could be done using for example the \hyperref[TEI.lg]{<lg>} element defined in section \textit{\hyperref[COVE]{3.13.1.\ Core Tags for Verse}} or the ‘virtual’ \hyperref[TEI.join]{<join>} element discussed in section \textit{\hyperref[SAAG]{16.7.\ Aggregation}}.\par
A third possibility would be to use an ‘intermediate pointer’ as follows: \par\bgroup\index{ptr=<ptr>|exampleindex}\index{target=@target!<ptr>|exampleindex}\exampleFont \begin{shaded}\noindent\mbox{}{<\textbf{ptr}\hspace*{1em}{xml:id}="{L3.283-284}"\mbox{}\newline 
\hspace*{1em}{target}="{\#L3.283 \#L3.284}"/>}\end{shaded}\egroup\par \noindent  When the {\itshape target} attribute of a \hyperref[TEI.ptr]{<ptr>} or \hyperref[TEI.ref]{<ref>} element specifies more than one element, the indicated elements are intended to be combined or aggregated in some way to produce the object of the pointer. (Such aggregation is however the task of a processing application, and cannot be defined simply by the markup). The {\itshape xml:id} attribute of the \hyperref[TEI.ptr]{<ptr>} then provides an identifier which can be linked to the \hyperref[TEI.note]{<note>} and \hyperref[TEI.ref]{<ref>} elements: \par\bgroup\index{link=<link>|exampleindex}\index{evaluate=@evaluate!<link>|exampleindex}\index{target=@target!<link>|exampleindex}\exampleFont \begin{shaded}\noindent\mbox{}{<\textbf{link}\hspace*{1em}{evaluate}="{all}"\mbox{}\newline 
\hspace*{1em}{target}="{\#n3.284 \#r3.284 \#L3.283-284}"/>}\end{shaded}\egroup\par \noindent  \par
The all value of {\itshape evaluate} is used on the \hyperref[TEI.link]{<link>} element to specify that any pointer encountered as a target of that element is itself evaluated. If {\itshape evaluate} had the value none, the link target would be the pointer itself, rather than the objects it points to.\par
Where a \hyperref[TEI.linkGrp]{<linkGrp>} element is used to group a collection of \hyperref[TEI.link]{<link>} elements, any intermediate pointer elements used by those \hyperref[TEI.link]{<link>} elements should be included within the \hyperref[TEI.linkGrp]{<linkGrp>}.
\subsection[{Pointing Mechanisms}]{Pointing Mechanisms}\label{SAXP}\par
This section introduces more formally the pointing mechanisms available in the TEI. In addition to those discussed so far, the TEI provides methods of pointing: \begin{itemize}
\item into documents other than the current document;
\item to a particular element in a document other than the current document using its {\itshape xml:id};
\item to a particular element whether in the current document or not, using its position in the XML element tree;
\item at arbitrary content in any XML document using TEI-defined XPointer schemes.
\end{itemize} \par
All TEI attributes used to point at something else are declared as having the datatype \textsf{teidata.pointer}, which is defined as a URI reference\footnote{The URI (Universal Resource Indicator) is defined in \xref{http://www.ietf.org/rfc/rfc3986.txt}{RFC 3986}}; the cases so far discussed are all simple examples of a URI reference. Another familiar example is the mechanism used in XHTML to create represent hypertext links by means of the XHTML {\itshape href} attribute. A URI reference can reference the whole of an XML resource such as a document or an XML element, or a sub-portion of such a resource, identified by means of an appropriate \textit{fragment identifier}. Technically speaking, the ‘fragment identifier’ is that portion of a URI reference following the first unescaped ‘\#’ character; in practice, it provides a means of accessing some part of the resource described by the URI which is less than the whole. \par
The first three of the following subsections provide only a brief overview and some examples of the W3C mechanisms recommended. More detailed information on the use of these mechanisms is readily available elsewhere.
\subsubsection[{Pointing Elsewhere}]{Pointing Elsewhere}\label{SAUR}\par
Like the ubiquitous if misnamed XHTML pointing attribute {\itshape href}, the TEI pointing attributes can point to a document that is not the current document (the one that contains the pointing element) whether it is in the same local filesystem as the current document, or on a different system entirely. In either case, the pointing can be accomplished absolutely (using the entire address of the target document) or relatively (using an address relative to the current base URI in force). The ‘current base URI’ is defined according to \hyperref[XMLBASE]{Marsh and Tobin 2009}. If there is none, the base URI is that of the current document. In common practice the current base URI in force is likely to be the value of the {\itshape xml:base} attribute of the closest ancestor that has one. However this may not be the case, since {\itshape xml:base} attributes are accumulated through the hierarchy by concatenation of path segments, beginning at the top of the hierarchy and proceeding down to the context node.\par
The following example demonstrates an absolute URI reference that points to a remote document: \par\bgroup\index{ref=<ref>|exampleindex}\index{target=@target!<ref>|exampleindex}\exampleFont \begin{shaded}\noindent\mbox{}The current base URI in force is as defined in the\mbox{}\newline 
 W3C {<\textbf{ref}\hspace*{1em}{target}="{http://www.w3.org/TR/xmlbase/}">}XML\mbox{}\newline 
 Base{</\textbf{ref}>} recommendation.\end{shaded}\egroup\par \par
This example points explicitly to a location on the Web, accessible via HTTP. Suppose however that we wish to access a document stored locally in a file. Again we will supply an absolute URI reference, but this time using a different protocol: \par\bgroup\index{ref=<ref>|exampleindex}\index{target=@target!<ref>|exampleindex}\exampleFont \begin{shaded}\noindent\mbox{}This Debian package is distributed under the terms\mbox{}\newline 
 of the {<\textbf{ref}\hspace*{1em}{target}="{file:///usr/share/common-licenses/GPL-2}">}GNU General Public License{</\textbf{ref}>}.\end{shaded}\egroup\par \par
In the following example, we use a relative URI reference to point to a local document: \par\bgroup\index{figure=<figure>|exampleindex}\index{rend=@rend!<figure>|exampleindex}\index{graphic=<graphic>|exampleindex}\index{url=@url!<graphic>|exampleindex}\index{figDesc=<figDesc>|exampleindex}\index{title=<title>|exampleindex}\exampleFont \begin{shaded}\noindent\mbox{}{<\textbf{figure}\hspace*{1em}{rend}="{float fullpage}">}\mbox{}\newline 
\hspace*{1em}{<\textbf{graphic}\hspace*{1em}{url}="{Images/compic.png}"/>}\mbox{}\newline 
\hspace*{1em}{<\textbf{figDesc}>}The figure shows the page from the {<\textbf{title}>}Orbis\mbox{}\newline 
\hspace*{1em}\hspace*{1em}\hspace*{1em}\hspace*{1em} pictus{</\textbf{title}>} of Comenius which is discussed in the text.{</\textbf{figDesc}>}\mbox{}\newline 
{</\textbf{figure}>}\end{shaded}\egroup\par \noindent  Since no {\itshape xml:base} is specified here, the location of the resource \textsf{Images/compic.png} is determined relative to the resource indicated by the current base URI, which is the current document.\par
In the following example, however, we first change the current base URI by setting a new value for {\itshape xml:base}. The resource required is then identified by means of a relative URI: \par\bgroup\index{div=<div>|exampleindex}\index{type=@type!<div>|exampleindex}\index{head=<head>|exampleindex}\index{p=<p>|exampleindex}\index{ref=<ref>|exampleindex}\index{target=@target!<ref>|exampleindex}\index{title=<title>|exampleindex}\exampleFont \begin{shaded}\noindent\mbox{}{<\textbf{div}\hspace*{1em}{type}="{chap}"\mbox{}\newline 
\hspace*{1em}{xml:base}="{http://classics.mit.edu/}">}\mbox{}\newline 
\hspace*{1em}{<\textbf{head}>}On Ancient Persian Manners{</\textbf{head}>}\mbox{}\newline 
\hspace*{1em}{<\textbf{p}>}In the very first story of {<\textbf{ref}\hspace*{1em}{target}="{Sadi/gulistan.2.i.html}">}\mbox{}\newline 
\hspace*{1em}\hspace*{1em}\hspace*{1em}{<\textbf{title}>}The Gulistan of\mbox{}\newline 
\hspace*{1em}\hspace*{1em}\hspace*{1em}\hspace*{1em}\hspace*{1em}\hspace*{1em} Sa'di{</\textbf{title}>}\mbox{}\newline 
\hspace*{1em}\hspace*{1em}{</\textbf{ref}>},\mbox{}\newline 
\hspace*{1em}\hspace*{1em} Sa'di relates moral advice worthy of Miss Minners ...{</\textbf{p}>}\mbox{}\newline 
\textit{<!-- ... -->}\mbox{}\newline 
{</\textbf{div}>}\end{shaded}\egroup\par \par
As noted above, the current base URI is found on the nearest ancestor. It is technically possible to use {\itshape xml:base} as a means to shorten URIs, but this usage is not recommended. \hyperref[SAPU]{Abbreviated pointers} provide a more flexible and consistent method for creating shorthand links.
\subsubsection[{Pointing Locally}]{Pointing Locally}\label{SABN}\par
A \textit{shorthand pointer}, in which the URI consists only of \texttt{\#} followed by the value of an {\itshape xml:id} acts as a pointer to the element in the current document with that {\itshape xml:id}, as in the following example. \par\bgroup\index{div=<div>|exampleindex}\index{type=@type!<div>|exampleindex}\index{div=<div>|exampleindex}\index{type=@type!<div>|exampleindex}\index{n=@n!<div>|exampleindex}\index{head=<head>|exampleindex}\index{p=<p>|exampleindex}\index{ref=<ref>|exampleindex}\index{target=@target!<ref>|exampleindex}\index{list=<list>|exampleindex}\index{rend=@rend!<list>|exampleindex}\index{item=<item>|exampleindex}\index{n=@n!<item>|exampleindex}\index{item=<item>|exampleindex}\index{n=@n!<item>|exampleindex}\index{item=<item>|exampleindex}\index{n=@n!<item>|exampleindex}\index{item=<item>|exampleindex}\index{n=@n!<item>|exampleindex}\exampleFont \begin{shaded}\noindent\mbox{}{<\textbf{div}\hspace*{1em}{type}="{section}"\hspace*{1em}{xml:id}="{sect106}">}\mbox{}\newline 
\textit{<!-- ... -->}\mbox{}\newline 
{</\textbf{div}>}\mbox{}\newline 
{<\textbf{div}\hspace*{1em}{type}="{section}"\hspace*{1em}{n}="{107}"\hspace*{1em}{xml:id}="{sect107}">}\mbox{}\newline 
\hspace*{1em}{<\textbf{head}>}Limitations on exclusive rights: Fair use{</\textbf{head}>}\mbox{}\newline 
\hspace*{1em}{<\textbf{p}>}Notwithstanding the provisions of\mbox{}\newline 
\hspace*{1em}{<\textbf{ref}\hspace*{1em}{target}="{\#sect106}">}section 106{</\textbf{ref}>}, the fair use of a\mbox{}\newline 
\hspace*{1em}\hspace*{1em} copyrighted work, including such use by reproduction in copies\mbox{}\newline 
\hspace*{1em}\hspace*{1em} or phonorecords or by any other means specified by that section,\mbox{}\newline 
\hspace*{1em}\hspace*{1em} for purposes such as criticism, comment, news reporting,\mbox{}\newline 
\hspace*{1em}\hspace*{1em} teaching (including multiple copies for classroom use),\mbox{}\newline 
\hspace*{1em}\hspace*{1em} scholarship, or research, is not an infringement of copyright.\mbox{}\newline 
\hspace*{1em}\hspace*{1em} In determining whether the use made of a work in any particular\mbox{}\newline 
\hspace*{1em}\hspace*{1em} case is a fair use the factors to be considered shall\mbox{}\newline 
\hspace*{1em}\hspace*{1em} include — \mbox{}\newline 
\hspace*{1em}{<\textbf{list}\hspace*{1em}{rend}="{bulleted}">}\mbox{}\newline 
\hspace*{1em}\hspace*{1em}\hspace*{1em}{<\textbf{item}\hspace*{1em}{n}="{(1)}">}the purpose and character of the use, including\mbox{}\newline 
\hspace*{1em}\hspace*{1em}\hspace*{1em}\hspace*{1em}\hspace*{1em}\hspace*{1em} whether such use is of a commercial nature or is for nonprofit\mbox{}\newline 
\hspace*{1em}\hspace*{1em}\hspace*{1em}\hspace*{1em}\hspace*{1em}\hspace*{1em} educational purposes;{</\textbf{item}>}\mbox{}\newline 
\hspace*{1em}\hspace*{1em}\hspace*{1em}{<\textbf{item}\hspace*{1em}{n}="{(2)}">}the nature of the copyrighted work;{</\textbf{item}>}\mbox{}\newline 
\hspace*{1em}\hspace*{1em}\hspace*{1em}{<\textbf{item}\hspace*{1em}{n}="{(3)}">}the amount and substantiality of the portion\mbox{}\newline 
\hspace*{1em}\hspace*{1em}\hspace*{1em}\hspace*{1em}\hspace*{1em}\hspace*{1em} used in relation to the copyrighted work as a whole;\mbox{}\newline 
\hspace*{1em}\hspace*{1em}\hspace*{1em}\hspace*{1em}\hspace*{1em}\hspace*{1em} and{</\textbf{item}>}\mbox{}\newline 
\hspace*{1em}\hspace*{1em}\hspace*{1em}{<\textbf{item}\hspace*{1em}{n}="{(4)}">}the effect of the use upon the potential market\mbox{}\newline 
\hspace*{1em}\hspace*{1em}\hspace*{1em}\hspace*{1em}\hspace*{1em}\hspace*{1em} for or value of the copyrighted work.{</\textbf{item}>}\mbox{}\newline 
\hspace*{1em}\hspace*{1em}{</\textbf{list}>}\mbox{}\newline 
\hspace*{1em}\hspace*{1em} The fact that a work is unpublished shall not itself bar a\mbox{}\newline 
\hspace*{1em}\hspace*{1em} finding of fair use if such finding is made upon consideration\mbox{}\newline 
\hspace*{1em}\hspace*{1em} of all the above factors.{</\textbf{p}>}\mbox{}\newline 
{</\textbf{div}>}\end{shaded}\egroup\par \noindent  This method of pointing, by referring to the {\itshape xml:id} of the target element as a bare name only (e.g., \#sect106) is the simplest and often the best approach where it can be applied, i.e. where both the source element and target element are in the same XML document, and where the target element carries an identifier. It is the method used extensively in previous sections of this chapter and elsewhere in these Guidelines.
\subsubsection[{Using Abbreviated Pointers}]{Using Abbreviated Pointers}\label{SAPU}\par
Even in the case of relative links on the local file system, {\itshape ref} or {\itshape target} attributes may become quite lengthy and make XML code difficult to read. To deal with this problem, the TEI provides a useful method of using abbreviated pointers and documenting a way to dereference them automatically.\par
Imagine a project which has a large collection of XML documents organized like this:\begin{itemize}
\item anthology \mbox{}\\[-10pt] \begin{itemize}
\item poetry \mbox{}\\[-10pt] \begin{itemize}
\item \textsf{poem.xml}
\end{itemize} 
\item prose \mbox{}\\[-10pt] \begin{itemize}
\item \textsf{novel.xml}
\end{itemize} 
\end{itemize} 
\item references \mbox{}\\[-10pt] \begin{itemize}
\item people \mbox{}\\[-10pt] \begin{itemize}
\item \textsf{personography.xml}
\end{itemize} 
\end{itemize} 
\end{itemize} \par
If you want to link a \hyperref[TEI.name]{<name>} in the \textsf{novel.xml} file to a \hyperref[TEI.person]{<person>} in the \textsf{personography.xml} file, the link will look like this: \par\bgroup\index{name=<name>|exampleindex}\index{ref=@ref!<name>|exampleindex}\exampleFont \begin{shaded}\noindent\mbox{}{<\textbf{name}\hspace*{1em}{ref}="{../../references/people/personography.xml\#fred}">}Fred{</\textbf{name}>}\end{shaded}\egroup\par \noindent  If there are many names to tag in a single paragraph, the XML encoding will be congested, and such lengthy links are prone to typographical error. In addition, if the project organization is changed, every relative link will have to be found and altered.\par
One way to deal with this is to use what is often referred to as a "magic token". You could make such links using the {\itshape key} attribute: \par\bgroup\index{name=<name>|exampleindex}\index{key=@key!<name>|exampleindex}\exampleFont \begin{shaded}\noindent\mbox{}{<\textbf{name}\hspace*{1em}{key}="{fred}">}Fred{</\textbf{name}>}\end{shaded}\egroup\par \noindent  and document the meaning of the key using (for instance) a \hyperref[TEI.taxonomy]{<taxonomy>} element in the TEI header, as described in \textit{\hyperref[CONARS]{3.6.1.\ Referring Strings}}. However, such a link cannot be mechanically processed by an external system that does not know how to interpret it; a human will have to read the header explanation and write code explicitly to reconstruct the intended link.\par
A more robust alternative is to use a \textit{private URI scheme}. This is a method of constructing a simple, key-like token which functions as a \textsf{teidata.pointer}, and can therefore be used as the value of any attribute which has that datatype, such as {\itshape ref} and {\itshape target}. Such a scheme consists of a prefix with a colon, and then a value. You might, for example, use the prefix psn (for "person"), and structure your name tags like this: \par\bgroup\index{name=<name>|exampleindex}\index{ref=@ref!<name>|exampleindex}\exampleFont \begin{shaded}\noindent\mbox{}{<\textbf{name}\hspace*{1em}{ref}="{psn:fred}">}Fred{</\textbf{name}>}\end{shaded}\egroup\par \noindent  How is this different from a ‘magic token’? Essentially, it isn't, except that TEI provides a structured method of dereferencing it (turning it into a computable path, such as ../../references/people/personography.xml\#fred) by means of a declaration inside \hyperref[TEI.encodingDesc]{<encodingDesc>} in the TEI header, using the elements and attributes for prefix declaration: 
\begin{sansreflist}
  
\item [\textbf{<listPrefixDef>}] (list of prefix definitions) contains a list of definitions of prefixing schemes used in \textsf{teidata.pointer} values, showing how abbreviated URIs using each scheme may be expanded into full URIs.
\item [\textbf{<prefixDef>}] (prefix definition) defines a prefixing scheme used in \textsf{teidata.pointer} values, showing how abbreviated URIs using the scheme may be expanded into full URIs.\hfil\\[-10pt]\begin{sansreflist}
    \item[@{\itshape ident}]
  supplies a name which functions as the prefix for an abbreviated pointing scheme such as a private URI scheme. The prefix constitutes the text preceding the first colon.
\end{sansreflist}  
\item [\textbf{att.patternReplacement}] provides attributes for regular-expression matching and replacement.\hfil\\[-10pt]\begin{sansreflist}
    \item[@{\itshape matchPattern}]
  specifies a regular expression against which the values of other attributes can be matched.
    \item[@{\itshape replacementPattern}]
  specifies a ‘replacement pattern’, that is, the skeleton of a relative or absolute URI containing references to groups in the {\itshape matchPattern} which, once subpattern substitution has been performed, complete the URI.
\end{sansreflist}  
\end{sansreflist}
\par
This is how you might document a private URI scheme using the psn: prefix: \par\bgroup\index{listPrefixDef=<listPrefixDef>|exampleindex}\index{prefixDef=<prefixDef>|exampleindex}\index{ident=@ident!<prefixDef>|exampleindex}\index{matchPattern=@matchPattern!<prefixDef>|exampleindex}\index{replacementPattern=@replacementPattern!<prefixDef>|exampleindex}\index{p=<p>|exampleindex}\index{gi=<gi>|exampleindex}\exampleFont \begin{shaded}\noindent\mbox{}{<\textbf{listPrefixDef}>}\mbox{}\newline 
\hspace*{1em}{<\textbf{prefixDef}\hspace*{1em}{ident}="{psn}"\mbox{}\newline 
\hspace*{1em}\hspace*{1em}{matchPattern}="{([a-z]+)}"\mbox{}\newline 
\hspace*{1em}\hspace*{1em}{replacementPattern}="{../../references/people/personography.xml\#\$1}">}\mbox{}\newline 
\hspace*{1em}\hspace*{1em}{<\textbf{p}>} In the context of this project, private URIs with the prefix\mbox{}\newline 
\hspace*{1em}\hspace*{1em}\hspace*{1em}\hspace*{1em} "psn" point to {<\textbf{gi}>}person{</\textbf{gi}>} elements in the project's\mbox{}\newline 
\hspace*{1em}\hspace*{1em}\hspace*{1em}\hspace*{1em} personography.xml file.\mbox{}\newline 
\hspace*{1em}\hspace*{1em}{</\textbf{p}>}\mbox{}\newline 
\hspace*{1em}{</\textbf{prefixDef}>}\mbox{}\newline 
{</\textbf{listPrefixDef}>}\end{shaded}\egroup\par \noindent  This specifies that where a \textsf{teidata.pointer} value is constructed with a psn: prefix, a regular-expression replace operation can be performed on it to construct the full or relative URI to the target document or fragment. \hyperref[TEI.listPrefixDef]{<listPrefixDef>} is a child of \hyperref[TEI.encodingDesc]{<encodingDesc>}, and it contains any number of \hyperref[TEI.prefixDef]{<prefixDef>} elements. Each \hyperref[TEI.prefixDef]{<prefixDef>} element provides a method of dereferencing or expanding an abbreviated pointer, based on a regular expression. The {\itshape ident} attribute specifies the prefix to which the expansion applies (without the colon). The {\itshape matchPattern} attribute contains a regular expression which is matched against the component of the pointer following the first colon, and the {\itshape replacementPattern} provides the string which will be used as a replacement. In this example, using psn:fred, the value fred would be matched by the {\itshape matchPattern}, and also captured (through the parentheses in the regular expression); it would then be replaced by the value ../../references/people/personography.xml\#fred (with the the \$1 in the {\itshape replacementPattern} being replaced by the captured value). The \hyperref[TEI.p]{<p>} element inside the \hyperref[TEI.prefixDef]{<prefixDef>} can be used to provide a human-readable explanation of the usage of this prefix.\par
Through this mechanism, any processor which encounters a \textsf{teidata.pointer} with a protocol unknown to it can check the \hyperref[TEI.listPrefixDef]{<listPrefixDef>} in the header to see if there is an available expansion for it, and if there is, it can automatically provide the expansion and generate a full or relative URI.\par
For any given prefix, it may be useful to supply more than one expansion. For instance, in addition to pointing at the \hyperref[TEI.person]{<person>} element in the personography file, it might also be useful to point to an external source which is available on the network, representing the same information in a different way. So there might be a second \hyperref[TEI.prefixDef]{<prefixDef>} like this: \par\bgroup\index{prefixDef=<prefixDef>|exampleindex}\index{ident=@ident!<prefixDef>|exampleindex}\index{matchPattern=@matchPattern!<prefixDef>|exampleindex}\index{replacementPattern=@replacementPattern!<prefixDef>|exampleindex}\index{p=<p>|exampleindex}\exampleFont \begin{shaded}\noindent\mbox{}{<\textbf{prefixDef}\hspace*{1em}{ident}="{psn}"\mbox{}\newline 
\hspace*{1em}{matchPattern}="{([a-z]+)}"\mbox{}\newline 
\hspace*{1em}{replacementPattern}="{http://www.example.com/personography.html\#\$1}">}\mbox{}\newline 
\hspace*{1em}{<\textbf{p}>} Private URIs with the prefix "psn" can be converted to point\mbox{}\newline 
\hspace*{1em}\hspace*{1em} to a fragment on the Personography page of the project Website.\mbox{}\newline 
\hspace*{1em}{</\textbf{p}>}\mbox{}\newline 
{</\textbf{prefixDef}>}\end{shaded}\egroup\par \noindent  Any number of \hyperref[TEI.prefixDef]{<prefixDef>} elements may be provided for the same prefix. A processor may decide to process one or all of them; if it processes only one, it should choose the first one with the correct {\itshape ident} value, so the primary or most important \hyperref[TEI.prefixDef]{<prefixDef>} for any given prefix should appear first in its parent \hyperref[TEI.listPrefixDef]{<listPrefixDef>}.\par
When creating private URI schemes, it is recommended that you avoid using any existing registered prefix. A list of registered prefixes is maintained by IANA at \xref{http://www.iana.org/assignments/uri-schemes.html}{http://www.iana.org/assignments/uri-schemes.html}.\par
Note that this mechanism can also be used to dereference other abbreviated pointing systems which are based on prefixes, such as Tag URIs.\par
The {\itshape matchPattern} and {\itshape replacementPattern} attributes are also used in dereferencing canonical reference patterns, and further examples of the use of regular expressions are shown in \textit{\hyperref[SACR]{16.2.5.\ Canonical References}}.
\subsubsection[{TEI XPointer Schemes}]{TEI XPointer Schemes}\label{SATS}\par
The pointing schemes described in this chapter are part of a number of such schemes envisaged by the W3C, which together constitute a framework for addressing data within XML documents, known as the XPointer Framework (\hyperref[XPTRFMWK]{Grosso et al 2003}). This framework permits the definition of many other named addressing methods, each of which is known as an \textit{XPointer Scheme}. The W3C has predefined a set of such schemes, and maintains a register for their expansion.\par
One important scheme, also defined by the W3C, and recommended by these Guidelines is the  {\name xpath()} pointer scheme, which allows for any part of an XML structure to be selected using the syntax defined by the XPath specification. This is further discussed below, \textit{\hyperref[SATSXP]{16.2.4.2.\ xpath()}}. These Guidelines also define six other pointer schemes, which provide access to parts of an XML document such as points within data content or stretches of data content. These additional TEI pointer schemes are defined in sections \textit{\hyperref[SATSL]{16.2.4.3.\ left()}} to \textit{\hyperref[SATSMA]{16.2.4.8.\ match()}} below.
\paragraph[{Introduction to TEI Pointers}]{Introduction to TEI Pointers}\label{SATSin}\par
Before discussing the TEI pointer schemes, we introduce slightly more formally the terminology used to define them. So far, we have discussed only ways of pointing at components of the XML information set node such as elements and attributes. However, there is often a need in text analysis to address additional types of location such as the ‘point’ locations \textit{between} ‘nodes’, and ‘sequences’ that may arbitrarily cross the boundaries of nodes in a document. The content of an XML document is organized sequentially as well as hierarchically, and it makes sense to consider ranges of characters within a document independently of the nodes to which they belong. From the perspective of most of the pointer schemes discussed below, a TEI document is a tree structure superimposed upon a character stream. Nodes are entities available only in the tree, while points are available only in the stream. For this reason, the schemes below that rely upon character positions (\texttt{string-index()}, \texttt{string-range()}, and \texttt{match()}) cannot take nodes into account. Conversely, XPath (disregarding functions that return atomic values) is a method for locating nodes in the tree and treats those nodes as indivisible units, meaning it is unable to address parts of nodes in their document context.\par
The TEI pointer scheme thus distinguishes the following kinds of object: \begin{description}

\item[{Node}]A node is an instance of one of the node kinds defined in the \xref{http://www.w3.org/TR/xpath-datamodel/}{XQuery 1.0 and XPath 2.0 Data Model (Second Edition)}. It represents a single item in the XML information set for a document. For pointing purposes, the only nodes that are of interest are Text Nodes, Element Nodes, and Attribute Nodes.
\item[{Sequence}]A Sequence follows the definition in the XPath 2.0 Data Model, with one alteration. A Sequence is an ordered collection of zero or more items, where an item is either a node or a partial text node. 
\item[{Text Stream}]A Text Stream is the concatenation of the text nodes in a document and behaves as though all tags had been removed. A text stream begins at a reference node and encompasses all of the text inside that node (if any) and all the text following it in document order. In XPath terms, this would encompass all of the text nodes beginning at a particular node, and following it on the \xref{http://www.w3.org/TR/xpath20/\#axes}{following axis}.
\item[{Point}]A Point represents a dimensionless point between nodes or characters in a document. Every point is adjacent to either characters or elements, and never to another point. Points can only be referenced in relation to an element or text node in the document (i.e. something addressable by either an XPath or a fragment identifier). Points occur either immediately before or after an element, or at a numbered position inside a text stream. Position zero in the stream would be immediately before the first character. Note that points within attribute values cannot mark the beginning or end of a range extending beyond the attribute value, because points indicate a position within a document. Since attribute nodes are by definition un-ordered, they cannot be said to have a fixed position. 
\end{description} \par
The TEI recommends the following seven pointer schemes: \begin{description}

\item[{ {\name xpath()}}]Addresses a node or node sequence using the XPath syntax. (\textit{\hyperref[SATSXP]{16.2.4.2.\ xpath()}})
\item[{ {\name left()} and  {\name right()}}]addresses the point before (left) or after (right) a node or node sequence (\textit{\hyperref[SATSL]{16.2.4.3.\ left()}} and \textit{\hyperref[SATSR]{16.2.4.4.\ right()}})
\item[{ {\name string-index()}}]addresses a point inside a text node (\textit{\hyperref[SATSSI]{16.2.4.5.\ string-index()}}
\item[{ {\name range()}}]addresses the range between two points (\textit{\hyperref[SATSRN]{16.2.4.6.\ range()}})
\item[{ {\name string-range()}}]addresses a range of a specified length starting from a specified point (\textit{\hyperref[SATSSR]{16.2.4.7.\ string-range()}})
\item[{ {\name match()}}]addresses a range which matches a specified string within a node (\textit{\hyperref[SATSMA]{16.2.4.8.\ match()}})
\end{description} \par
The  {\name xpath()} scheme refers to the existing XPath specification which is adopted with one modification: the default namespace for any XPath used as a parameter to this scheme is assumed to be the TEI namespace \texttt{http://www.tei-c.org/ns/1.0}.\par
The other six schemes overlap in functionality with a W3C draft specification known as the  {\name XPointer scheme} draft, but are individually much simpler. At the time of this writing, there is no current or scheduled activity at the W3C towards revising this draft or issuing it as a recommendation.\par
{\bfseries A note on namespaces}: The W3C defines an  {\name xmlns()} scheme (see \xref{http://www.w3.org/TR/xptr-xmlns/}{XPointer xmlns() Scheme}) which when prepended to a resolvable pointer allows for the definition of namespace prefixes to be used in XPaths in subsequent pointers. TEI Pointer schemes assume that un-prefixed element names in TEI Pointer XPaths are in the TEI namespace, \texttt{http://www.tei-c.org/ns/1.0}. The use of  {\name xmlns()} is thus optional, provided no new prefixes need to be defined. If the schemes described here are used to address non-TEI elements, then any new prefixes to be used in pointer XPaths may be defined using the  {\name xmlns()} scheme.
\paragraph[{xpath()}]{xpath()}\label{SATSXP}\par
\texttt{Sequence xpath(XPATH)}\par
The  {\name xpath()} scheme locates zero or more nodes within an XML Information Set. The single argument XPATH is an XPath selection pattern, as defined in \xref{https://www.w3.org/TR/xslt-30/\#dt-selection-pattern}{XSLT 3.0}, that returns a node or sequence of nodes. XPaths returning atomic values (e.g.  {\name substring()}) are illegal in the  {\name xpath()} scheme because they represent extracted values rather than locations in the source document. Because the schemes below involve starting at a node and navigating from there, and because attribute nodes have no intrinsic order, XPath expressions that address attribute nodes should be avoided in schemes other than  {\name xpath()}.\par
The example below, and all subsequent examples in this section refer to the following TEI fragment\label{SATSXP-ex}:  \par\bgroup\index{div=<div>|exampleindex}\index{type=@type!<div>|exampleindex}\index{ab=<ab>|exampleindex}\index{lb=<lb>|exampleindex}\index{n=@n!<lb>|exampleindex}\index{supplied=<supplied>|exampleindex}\index{reason=@reason!<supplied>|exampleindex}\index{choice=<choice>|exampleindex}\index{reg=<reg>|exampleindex}\index{orig=<orig>|exampleindex}\index{lb=<lb>|exampleindex}\index{n=@n!<lb>|exampleindex}\index{gap=<gap>|exampleindex}\index{reason=@reason!<gap>|exampleindex}\index{quantity=@quantity!<gap>|exampleindex}\index{unit=@unit!<gap>|exampleindex}\index{gap=<gap>|exampleindex}\index{reason=@reason!<gap>|exampleindex}\index{quantity=@quantity!<gap>|exampleindex}\index{unit=@unit!<gap>|exampleindex}\index{unclear=<unclear>|exampleindex}\index{lb=<lb>|exampleindex}\index{n=@n!<lb>|exampleindex}\index{unclear=<unclear>|exampleindex}\index{unclear=<unclear>|exampleindex}\index{choice=<choice>|exampleindex}\index{reg=<reg>|exampleindex}\index{orig=<orig>|exampleindex}\index{choice=<choice>|exampleindex}\index{reg=<reg>|exampleindex}\index{orig=<orig>|exampleindex}\index{lb=<lb>|exampleindex}\index{n=@n!<lb>|exampleindex}\index{gap=<gap>|exampleindex}\index{reason=@reason!<gap>|exampleindex}\index{extent=@extent!<gap>|exampleindex}\index{unit=@unit!<gap>|exampleindex}\index{lb=<lb>|exampleindex}\index{n=@n!<lb>|exampleindex}\index{unclear=<unclear>|exampleindex}\exampleFont \begin{shaded}\noindent\mbox{}{<\textbf{div}\hspace*{1em}{xml:lang}="{la}"\hspace*{1em}{type}="{edition}"\mbox{}\newline 
\hspace*{1em}{xml:space}="{preserve}">}{<\textbf{ab}>}\newline
{<\textbf{lb}\hspace*{1em}{n}="{1}"\hspace*{1em}{xml:id}="{line1}"/>}{<\textbf{supplied}\hspace*{1em}{reason}="{lost}">}si{</\textbf{supplied}>} non {<\textbf{choice}>}{<\textbf{reg}>}habui{</\textbf{reg}>}{<\textbf{orig}>}abui{</\textbf{orig}>}{</\textbf{choice}>} quidquam vaco \newline
{<\textbf{lb}\hspace*{1em}{n}="{2}"/>}si{<\textbf{gap}\hspace*{1em}{reason}="{illegible}"\hspace*{1em}{quantity}="{3}"\mbox{}\newline 
\hspace*{1em}{unit}="{character}"/>}b{<\textbf{gap}\hspace*{1em}{reason}="{illegible}"\hspace*{1em}{quantity}="{3}"\mbox{}\newline 
\hspace*{1em}{unit}="{character}"/>} \newline
  cohort{<\textbf{unclear}>}e{</\textbf{unclear}>} mi rescribas \newline
{<\textbf{lb}\hspace*{1em}{n}="{3}"/>}{<\textbf{unclear}>}s{</\textbf{unclear}>}emp{<\textbf{unclear}>}er{</\textbf{unclear}>} in {<\textbf{choice}>}{<\textbf{reg}>}mente{</\textbf{reg}>}{<\textbf{orig}>}mentem{</\textbf{orig}>}{</\textbf{choice}>} \newline
  {<\textbf{choice}>}{<\textbf{reg}>}habe{</\textbf{reg}>}{<\textbf{orig}>}abe{</\textbf{orig}>}{</\textbf{choice}>} supra res \newline
{<\textbf{lb}\hspace*{1em}{n}="{4}"/>}scriptas{<\textbf{gap}\hspace*{1em}{reason}="{lost}"\hspace*{1em}{extent}="{unknown}"\mbox{}\newline 
\hspace*{1em}{unit}="{character}"/>} \newline
{<\textbf{lb}\hspace*{1em}{n}="{5}"/>}auge et opto u{<\textbf{unclear}>}t{</\textbf{unclear}>} bene valeas{</\textbf{ab}>}\mbox{}\newline 
{</\textbf{div}>}\end{shaded}\egroup\par \par
A TEI Pointer that referenced the \hyperref[TEI.reg]{<reg>} element in the \texttt{choice} in line 1 of the example might look like: \mbox{}\newline 
\texttt{\#xpath(//lb[@n='1']/following-sibling::choice[1]/reg)}. Note that XPath values must be assumed to start from the document root. They cannot be relative to the element bearing the attribute that uses the pointer because TEI Pointers are URIs. Care should be taken to ensure that XPaths used in TEI Pointers match only a single node, unless multiple matches are desired. The examples that follow are relatively simple because the document they refer to is short and does not contain many elements.\par
When an XPath is interpreted by a TEI processor, the information set of the referenced document is interpreted without any additional information supplied by any schema processing that may or may not be present. In particular this means that no whitespace normalization is applied to a document before the XPath is interpreted. \par
This pointer scheme allows easy, direct use of the most widely-implemented XML query method. It is probably the most robust pointing mechanism for the common situation of selecting an XML element or its contents where an {\itshape xml:id} is not present. The ability to use element names and attribute names and values makes  {\name xpath()} pointers more robust than the other mechanisms discussed in this section even if the designated document changes. For durability in the presence of editing, use of {\itshape xml:id} is always recommended when possible.
\paragraph[{left()}]{left()}\label{SATSL}\par
Point \texttt{left( IDREF | XPATH )}\par
The  {\name left()} scheme locates the point immediately preceding the node addressed by its argument, which is either an XPATH as defined above or an IDREF, the value of an {\itshape xml:id} occurring in the document addressed by the base URI in effect for the pointer.\par
Example: the pointer \texttt{\#left(//supplied[1])} indicates the point between the first \texttt{lb} and the first \texttt{supplied} in the \hyperref[SATSXP-ex]{example} above.\par
Example: \texttt{\#left(//gap[1])} indicates the point immediately before the first \texttt{gap} element in line two and the string \texttt{si}.\par
Example: \texttt{\#left(line1)} indicates the point immediately before the \texttt{<lb n="1"/>} element.
\paragraph[{right()}]{right()}\label{SATSR}\par
Point \texttt{right( IDREF | XPATH )}\par
The  {\name right()} scheme locates the point immediately following the node addressed by its argument.\par
Example: the pointer \texttt{\#right(//lb[@n='3'])} indicates the point between the third \texttt{lb} and the \texttt{<unclear>s</unclear>} element in the \hyperref[SATSXP-ex]{example}.
\paragraph[{string-index()}]{string-index()}\label{SATSSI}\par
Point \texttt{string-index( IDREF | XPATH, OFFSET )}\par
The  {\name string-index()} scheme locates a point based on character positions in a text stream relative to the node identified by the IDREF or XPATH parameter. The OFFSET parameter is a positive, negative, or zero integer which determines the position of the point. An offset of 0 represents the position immediately before the first character in either the first text node descendant of the node addressed in the first parameter or the first following text node, if the addressed element contains no text node descendants.\par
Example: \texttt{\#string-index(//lb[@n='2'],1)} indicates the point between the ‘s’ and the ‘i’ in the word ‘si’ in line 2.\par
{\bfseries Note}: The OFFSET parameter (and similarly the LENGTH parameter found below in the  {\name string-range()} scheme) are measured in characters. What is considered a single character will depend (assuming the document being evaluated is in Unicode) on the Normalization Form in use (see \xref{http://unicode.org/reports/tr15/}{UNICODE NORMALIZATION FORMS}). A letter followed by a combining diacritic counts as two characters, but the same diacritic precombined with a letter would count as a single character. Compare, for example, é (\texttt{\textbackslash u0060} followed by \texttt{\textbackslash u0301}) and é (\texttt{\textbackslash u00E9}). These are equivalent, and a conversion between Normalization Forms C and D will transform one into the other. This specification does not mandate a particular Normalization Form (see \textit{\hyperref[D4-46-2]{Precomposed and Combining Characters and Normalization}}), but users and implementers should be aware that it affects the character count and therefore the result of evaluating pointers that rely on character counting.
\paragraph[{range()}]{range()}\label{SATSRN}\par
Sequence \texttt{range( POINTER, POINTER[, POINTER, POINTER ...])}\par
The  {\name range()} scheme takes as parameters one or more pairs of POINTERs, which are each members of the set IDREF, XPATH,  {\name left()},  {\name right()}, or  {\name string-index()}. A  {\name range()} locates a (possibly non-contiguous) sequence beginning at the first POINTER parameter and ending at the last. If the POINTER locates a node (i.e. is an XPATH or IDREF), then that node is a member of the addressed sequence. If a sequence addressed by a range pointer overlaps, but does not wholly contain, an element (i.e. it contains only the start but not the end tag or vice-versa), then that element is not part of the sequence.\par
 {\name Range()}s may address sequences of non-contiguous nodes. For example, a range() might select text beginning before an \hyperref[TEI.app]{<app>}, encompassing the content of a single \hyperref[TEI.rdg]{<rdg>} and continuing after the \hyperref[TEI.app]{<app>}.\par
Example: \texttt{\#range(left(//lb[@n='3']),left(//lb[@n='4']))} indicates the whole of \hyperref[SATSXP-ex]{line 3} from the \texttt{<lb n="3"/>} to the point right before the following \texttt{<lb n="4"/>}.\par
Example: \texttt{\#range(right(//lb[@n='3']),string-index(//lb[@n='3'],15))} indicates the sequence \texttt{<unclear>s</unclear>emp<unclear>er</unclear> in mente}.\par
Example: \texttt{\#range(string-index(//lb[@n='3'],7),string-index(//lb[@n='3'],10),string-index(//lb[@n='3'],15),string-index(//lb[@n='3'],21))} indicates the non-contiguous sequence ‘in mentem’.
\paragraph[{string-range()}]{string-range()}\label{SATSSR}\par
Sequence \texttt{string-range(IDREF | XPATH, OFFSET, LENGTH[, OFFSET, LENGTH ...])}\par
The string-range() scheme locates a sequence based on character positions in a text stream relative to the node identified by the first parameter. The location of the beginning of the addressed sequence is determined precisely as for  {\name string-index()}. The OFFSET parameter is defined as above in  {\name string-index()}. The LENGTH parameter is a positive integer that denotes the length of the text stream captured by the sequence. As with  {\name range()}, the addressed sequence may contain text nodes and/or elements. The  {\name string-range()} scheme can accept multiple OFFSET, LENGTH pairs to address a non-contiguous sequence in much the same way that range() can accept multiple pairs of pointers.\par
Because string-range() addresses points in the text stream, tags are invisible to it. For example, if an empty tag like \hyperref[TEI.lb]{<lb>} is encountered while processing a string-range(), it will be included in the resulting sequence, but the LENGTH count will not increment when it is captured.\par
Example: \texttt{\#string-range(//lb[@n='5'],0,27)} indicates the whole of \hyperref[SATSXP-ex]{line 5} from the text immediately following the \texttt{lb} to the point right before the closing \texttt{ab} tag.\par
Example: \texttt{\#string-range(//lb[@n='3'],7,8)} indicates the sequence ‘in mente’.\par
Example: \texttt{\#string-range(//lb[@n='3'],7,3,15,6)} indicates the non-contiguous sequence ‘in mentem’.
\paragraph[{match()}]{match()}\label{SATSMA}\par
Sequence \texttt{match(IDREF | XPATH, 'REGEX' [, INDEX])}\par
The match scheme locates a sequence based on matching the REGEX parameter against a text stream relative to the reference node identified by the first parameter. REGEX is a regular expression as defined by \xref{http://www.w3.org/TR/xpath-functions/\#regex-syntax}{XQuery 1.0 and XPath 2.0 Functions and Operators (Second Edition)}, with some modifications: \begin{itemize}
\item Because the regular expression is delimited by apostrophe characters, any such characters (\texttt{'} or \texttt{\textbackslash u0027}) occurring inside the expression must be escaped using the URI percent-encoding scheme \texttt{\%27}.
\item Regular expressions in \texttt{match()} are assumed to operate in single-line mode. The end of the string to be matched against is either the end of the text contained by the element in the first parameter or the end of the document, if that parameter indicates an empty element. The meta-character \texttt{\textasciicircum } therefore matches the beginning of the text stream inside or following the reference node, and the meta-character \texttt{\$} matches the end of that stream.
\end{itemize}  The optional INDEX parameter is an integer greater than 0 which specifies which match should be chosen when there is more than one possibility. If omitted, the first match in the text stream will be used.\par
Like \texttt{string-range()}, \texttt{match()} may capture elements in the indicated sequence, even though they are ignored for purposes of evaluating the match.\par
Example: \texttt{\#match(//lb[@n='5'],'opto.*valeas')} indicates the sequence \texttt{opto u<unclear>t</unclear> bene valeas} in \hyperref[SATSXP-ex]{line 5}.\par
Example: \texttt{\#match(//lb[@n='3'],'semper')} would indicate the word ‘semper’, but would not capture the \texttt{unclear} elements in \texttt{<unclear>s</unclear>emp<unclear>er</unclear>}, just their text children.
\subsubsection[{Canonical References}]{Canonical References}\label{SACR}\par
By ‘canonical’ reference we mean any means of pointing into documents, specific to a community or corpus. For example, biblical scholars might understand ‘Matt 5:7’ to mean ‘the book called \textit{Matthew}, chapter 5, verse 7.’ They might then wish to translate the string ‘Matt 5:7’ into a pointer into a TEI-encoded document, selecting the element which corresponds to the seventh \hyperref[TEI.div]{<div>} element within the fifth \hyperref[TEI.div]{<div>} element within the \hyperref[TEI.div]{<div>} element with the {\itshape n} attribute valued ‘Matt.’\par
Several elements in the TEI scheme (\hyperref[TEI.gloss]{<gloss>}, \hyperref[TEI.ptr]{<ptr>}, \hyperref[TEI.ref]{<ref>}, and \hyperref[TEI.term]{<term>}) bear a special attribute, {\itshape cRef}, just for this purpose. Using the system described in this section, an encoder may specify references to canonical works in a discipline-familiar format, and expect software to derive a complete URI from it. The value of the {\itshape cRef} attribute is processed as described in this section, and the resulting URI reference is treated as if it were the value of the {\itshape target} attribute. The {\itshape cRef} and {\itshape target} attributes are mutually exclusive: only one or the other may be specified on any given occurrence of an element.\par
For the {\itshape cRef} attribute to function as required, a mechanism is needed to define the mapping between (for example) ‘the book called \textit{Matt}’ and the part of the XML structure which corresponds with it. This is provided by the \hyperref[TEI.refsDecl]{<refsDecl>} element  in the TEI header, which contains an algorithm for translating a canonical reference string (like Matt 5:7) into a URI such as \texttt{\#xpath(//div[@n='Matt']/div[5]/div[7])}. The \hyperref[TEI.refsDecl]{<refsDecl>} element is described in section \textit{\hyperref[HD54]{2.3.6.\ The Reference System Declaration}}; the following example is discussed in more detail below in section \textit{\hyperref[SACRWE]{16.2.5.1.\ Worked Example}}. An alternative and less verbose method is described in section \textit{\hyperref[SACRCS]{16.2.5.4.\ Citation Structures}}. \par\bgroup\index{refsDecl=<refsDecl>|exampleindex}\index{cRefPattern=<cRefPattern>|exampleindex}\index{matchPattern=@matchPattern!<cRefPattern>|exampleindex}\index{replacementPattern=@replacementPattern!<cRefPattern>|exampleindex}\index{p=<p>|exampleindex}\index{q=<q>|exampleindex}\index{q=<q>|exampleindex}\index{q=<q>|exampleindex}\index{cRefPattern=<cRefPattern>|exampleindex}\index{matchPattern=@matchPattern!<cRefPattern>|exampleindex}\index{replacementPattern=@replacementPattern!<cRefPattern>|exampleindex}\index{p=<p>|exampleindex}\index{q=<q>|exampleindex}\index{q=<q>|exampleindex}\index{cRefPattern=<cRefPattern>|exampleindex}\index{matchPattern=@matchPattern!<cRefPattern>|exampleindex}\index{replacementPattern=@replacementPattern!<cRefPattern>|exampleindex}\index{p=<p>|exampleindex}\index{q=<q>|exampleindex}\exampleFont \begin{shaded}\noindent\mbox{}{<\textbf{refsDecl}\hspace*{1em}{xml:id}="{biblical}">}\mbox{}\newline 
\hspace*{1em}{<\textbf{cRefPattern}\hspace*{1em}{matchPattern}="{(.+) (.+):(.+)}"\mbox{}\newline 
\hspace*{1em}\hspace*{1em}{replacementPattern}="{\#xpath(//div[@n='\$1']/div[@n='\$2']/div[@n='\$3]')}">}\mbox{}\newline 
\hspace*{1em}\hspace*{1em}{<\textbf{p}>}This pointer pattern extracts and references the {<\textbf{q}>}book,{</\textbf{q}>}\mbox{}\newline 
\hspace*{1em}\hspace*{1em}\hspace*{1em}{<\textbf{q}>}chapter,{</\textbf{q}>} and {<\textbf{q}>}verse{</\textbf{q}>} parts of a biblical reference.{</\textbf{p}>}\mbox{}\newline 
\hspace*{1em}{</\textbf{cRefPattern}>}\mbox{}\newline 
\hspace*{1em}{<\textbf{cRefPattern}\hspace*{1em}{matchPattern}="{(.+) (.+)}"\mbox{}\newline 
\hspace*{1em}\hspace*{1em}{replacementPattern}="{\#xpath(//div[@n='\$1']/div[\$2])}">}\mbox{}\newline 
\hspace*{1em}\hspace*{1em}{<\textbf{p}>}This pointer pattern extracts and references the {<\textbf{q}>}book{</\textbf{q}>} and\mbox{}\newline 
\hspace*{1em}\hspace*{1em}{<\textbf{q}>}chapter{</\textbf{q}>} parts of a biblical reference.{</\textbf{p}>}\mbox{}\newline 
\hspace*{1em}{</\textbf{cRefPattern}>}\mbox{}\newline 
\hspace*{1em}{<\textbf{cRefPattern}\hspace*{1em}{matchPattern}="{(.+)}"\mbox{}\newline 
\hspace*{1em}\hspace*{1em}{replacementPattern}="{\#xpath(//div[@n='\$1'])}">}\mbox{}\newline 
\hspace*{1em}\hspace*{1em}{<\textbf{p}>}This pointer pattern extracts and references just the {<\textbf{q}>}book{</\textbf{q}>}\mbox{}\newline 
\hspace*{1em}\hspace*{1em}\hspace*{1em}\hspace*{1em} part of a biblical reference.{</\textbf{p}>}\mbox{}\newline 
\hspace*{1em}{</\textbf{cRefPattern}>}\mbox{}\newline 
{</\textbf{refsDecl}>}\end{shaded}\egroup\par \par
When an application encounters a canonical reference as the value of {\itshape cRef} attribute, it might follow this sequence of specific steps to transform it into a URI reference: \begin{enumerate}
\item Ascertain the correct \hyperref[TEI.refsDecl]{<refsDecl>} following the rules summarized in section \textit{\hyperref[CCAS3]{15.3.3.\ Summary}}.
\item For each \hyperref[TEI.cRefPattern]{<cRefPattern>} element encountered in the appropriate \hyperref[TEI.refsDecl]{<refsDecl>}, in the order encountered: \mbox{}\\[-10pt] \begin{enumerate}
\item match the value of the {\itshape cRef} attribute to the regular expression found as the value of the {\itshape matchPattern} attribute
\item if the value of the {\itshape cRef} attribute matches: \mbox{}\\[-10pt] \begin{enumerate}
\item take the value of the {\itshape replacementPattern} attribute and substitute the back references (\$1, \$2, etc.) with the corresponding matched substrings
\item the result is taken as if it were a relative or absolute URI reference specified on the {\itshape target} attribute; i.e., it should be used as is or combined with the current {\itshape xml:base} attribute value as usual
\item no further processing of this value of the {\itshape cRef} attribute against the \hyperref[TEI.refsDecl]{<refsDecl>} should take place
\end{enumerate}
\item if, however, the value of the {\itshape cRef} attribute does not match the regular expression specified in the value of the {\itshape matchPattern} attribute, proceed to the next \hyperref[TEI.cRefPattern]{<cRefPattern>}
\end{enumerate}
\item If all the \hyperref[TEI.cRefPattern]{<cRefPattern>} elements are examined in turn and none matches, the pointer fails.
\end{enumerate}\par
The regular expression language used as the value of the {\itshape matchPattern} attribute is that used for the \textit{pattern} facet of the World Wide Web Consortium's XML Schema Language in an \xref{http://www.w3.org/TR/xmlschema-2/\#regexs}{Appendix to XML Schema Part 2}.\footnote{As always seems to be the case, no two regular expression languages are precisely the same. For those used to Perl regular expressions, be warned that while in Perl the pattern \texttt{tei} matches any string that contains \textit{tei}, in the W3C language it only matches the string ‘tei’.} The value of the {\itshape replacementPattern} attribute is simply a string, except that occurrences of ‘\$1’ through ‘\$9’ are replaced by the corresponding substring match. Note that since a maximum of nine substring matches are permitted, the string ‘\$18’ means ‘the value of the first matched substring followed by the character ‘8’’ as opposed to ‘the eighteenth matched substring’. If there is a need for an actual string including a dollar sign followed by a digit that is not supposed to be replaced, the dollar sign should be written as \texttt{\$\$}. Implementations must convert \texttt{\$\$} to \texttt{\$} during processing.
\paragraph[{Worked Example}]{Worked Example}\label{SACRWE}\par
Let us presume that with the example \hyperref[TEI.refsDecl]{<refsDecl>} above, an application comes across a {\itshape cRef} value of Matt 5:7. The application would first apply the regular expression \texttt{(.+) (.+):(.+)} to ‘Matt 5:7’. This regular expression would successfully match. The first matched substring would be ‘Matt’, the second ‘5’, and the third ‘7’. The application would then apply these substrings to the pattern \texttt{\#xpath(//div[@n='\$1']/div[\$2]/div[\$3])}, producing \texttt{\#xpath(//div[@n='Matt']/div[5]/div[7])}.\par
If, however, the input string had been ‘Matt 5’, the first regular expression would not have matched. The application would have then tried the second, \texttt{(.+) (.+)}, producing a successful match, and the matched substrings ‘Matt’ and ‘5’. It would then have substituted those matched substrings into the pattern \texttt{\#xpath(//div[@n='\$1']/div[\$2])} to produce a fragment identifier indicating the referenced element.\par
If the input string had been ‘Matt’, neither the first nor the second regular expressions would have successfully matched. The application would have then tried the third, \texttt{(.+)}, producing the matched substring ‘Matt’, and the URI Reference \texttt{\#xpath(//div[@n='Matt'])}.\par
a \hyperref[TEI.cRefPattern]{<cRefPattern>} should not reference more matched substrings. For example: \par\bgroup\index{cRefPattern=<cRefPattern>|exampleindex}\index{matchPattern=@matchPattern!<cRefPattern>|exampleindex}\index{replacementPattern=@replacementPattern!<cRefPattern>|exampleindex}\exampleFont \begin{shaded}\noindent\mbox{}{<\textbf{cRefPattern}\hspace*{1em}{matchPattern}="{(.+) (.+):(.+)}"\mbox{}\newline 
\hspace*{1em}{replacementPattern}="{//div[@n='\$1']/div[\$2]/div[\$3]/p[\$4]}"/>}\end{shaded}\egroup\par \noindent  is faulty, since only three matched substrings would have been produced, but a fourth (\texttt{\$4}) was referenced.
\paragraph[{Complete and Partial URI Examples}]{Complete and Partial URI Examples}\label{SACRex}\par
In the above example, the value of {\itshape cRef} was used to generate a Fragment Identifier. An absolute URI could be generated directly, as in the following example. \par\bgroup\index{refsDecl=<refsDecl>|exampleindex}\index{cRefPattern=<cRefPattern>|exampleindex}\index{matchPattern=@matchPattern!<cRefPattern>|exampleindex}\index{replacementPattern=@replacementPattern!<cRefPattern>|exampleindex}\index{p=<p>|exampleindex}\index{val=<val>|exampleindex}\index{val=<val>|exampleindex}\index{val=<val>|exampleindex}\index{cRefPattern=<cRefPattern>|exampleindex}\index{matchPattern=@matchPattern!<cRefPattern>|exampleindex}\index{replacementPattern=@replacementPattern!<cRefPattern>|exampleindex}\index{p=<p>|exampleindex}\index{val=<val>|exampleindex}\index{val=<val>|exampleindex}\index{val=<val>|exampleindex}\index{cRefPattern=<cRefPattern>|exampleindex}\index{matchPattern=@matchPattern!<cRefPattern>|exampleindex}\index{replacementPattern=@replacementPattern!<cRefPattern>|exampleindex}\index{p=<p>|exampleindex}\index{val=<val>|exampleindex}\index{val=<val>|exampleindex}\index{val=<val>|exampleindex}\index{p=<p>|exampleindex}\index{ref=<ref>|exampleindex}\index{cRef=@cRef!<ref>|exampleindex}\exampleFont \begin{shaded}\noindent\mbox{}{<\textbf{refsDecl}\hspace*{1em}{xml:id}="{USC}">}\mbox{}\newline 
\hspace*{1em}{<\textbf{cRefPattern}\hspace*{1em}{matchPattern}="{([0-9][0-9])⃥s*U⃥.?S⃥.?C⃥.?⃥s*[Cc](h(⃥.|ap(ter|⃥.)?)?)?⃥s*([1-9][0-9]*)}"\mbox{}\newline 
\hspace*{1em}\hspace*{1em}{replacementPattern}="{http://uscode.house.gov/download/pls/\$1C\$5.txt}">}\mbox{}\newline 
\hspace*{1em}\hspace*{1em}{<\textbf{p}>}Matches most standard references to particular\mbox{}\newline 
\hspace*{1em}\hspace*{1em}\hspace*{1em}\hspace*{1em} chapters of the United States Code, e.g.\mbox{}\newline 
\hspace*{1em}\hspace*{1em}{<\textbf{val}>}11USCC7{</\textbf{val}>}, {<\textbf{val}>}17 U.S.C. Chapter 3{</\textbf{val}>}, or\mbox{}\newline 
\hspace*{1em}\hspace*{1em}{<\textbf{val}>}14 USC Ch. 5{</\textbf{val}>}. Note that a leading zero is\mbox{}\newline 
\hspace*{1em}\hspace*{1em}\hspace*{1em}\hspace*{1em} required for the title (must be two digits), but is not\mbox{}\newline 
\hspace*{1em}\hspace*{1em}\hspace*{1em}\hspace*{1em} permitted for the chapter number.{</\textbf{p}>}\mbox{}\newline 
\hspace*{1em}{</\textbf{cRefPattern}>}\mbox{}\newline 
\hspace*{1em}{<\textbf{cRefPattern}\hspace*{1em}{matchPattern}="{([0-9][0-9])⃥s*U⃥.?S⃥.?C⃥.?⃥s*[Pp](re(lim(inary)?)?)?⃥s*[Mm](at(erial)?)?}"\mbox{}\newline 
\hspace*{1em}\hspace*{1em}{replacementPattern}="{http://uscode.house.gov/download/pls/\$1T.txt}">}\mbox{}\newline 
\hspace*{1em}\hspace*{1em}{<\textbf{p}>}Matches references to the preliminary material for a\mbox{}\newline 
\hspace*{1em}\hspace*{1em}\hspace*{1em}\hspace*{1em} given title, e.g. {<\textbf{val}>}11USCP{</\textbf{val}>}, {<\textbf{val}>}17 U.S.C.\mbox{}\newline 
\hspace*{1em}\hspace*{1em}\hspace*{1em}\hspace*{1em}\hspace*{1em}\hspace*{1em} Prelim Mat{</\textbf{val}>}, or {<\textbf{val}>}14 USC pm{</\textbf{val}>}.{</\textbf{p}>}\mbox{}\newline 
\hspace*{1em}{</\textbf{cRefPattern}>}\mbox{}\newline 
\hspace*{1em}{<\textbf{cRefPattern}\hspace*{1em}{matchPattern}="{([0-9][0-9])⃥s*U⃥.?S⃥.?C⃥.?⃥s*[Aa](ppend(ix)?)?}"\mbox{}\newline 
\hspace*{1em}\hspace*{1em}{replacementPattern}="{http://uscode.house.gov/download/pls/\$1A.txt}">}\mbox{}\newline 
\hspace*{1em}\hspace*{1em}{<\textbf{p}>}Matches references to the appendix of a given tile,\mbox{}\newline 
\hspace*{1em}\hspace*{1em}\hspace*{1em}\hspace*{1em} e.g. {<\textbf{val}>}05USCA{</\textbf{val}>}, {<\textbf{val}>}11 U.S.C. Appendix{</\textbf{val}>},\mbox{}\newline 
\hspace*{1em}\hspace*{1em}\hspace*{1em}\hspace*{1em} or {<\textbf{val}>}18 USC Append{</\textbf{val}>}.{</\textbf{p}>}\mbox{}\newline 
\hspace*{1em}{</\textbf{cRefPattern}>}\mbox{}\newline 
{</\textbf{refsDecl}>}\mbox{}\newline 
\textit{<!-- ... -->}\mbox{}\newline 
{<\textbf{p}>}The example in section 10 is taken\mbox{}\newline 
 from {<\textbf{ref}\hspace*{1em}{cRef}="{17 USC Ch 1}">}Subject Matter and Scope of\mbox{}\newline 
\hspace*{1em}\hspace*{1em} Copyright{</\textbf{ref}>}.{</\textbf{p}>}\end{shaded}\egroup\par \par
See \textit{\hyperref[SAPU]{16.2.3.\ Using Abbreviated Pointers}} for another related use of the {\itshape matchPattern} and {\itshape replacementPattern} attributes.
\paragraph[{Miscellaneous Usages}]{Miscellaneous Usages}\label{SACRmu}\par
Canonical reference pointers are intended for use by TEI encoders. However, this specification might be useful to the development of a process for recognizing canonical references in non-TEI documents (such as plain text documents), possibly as part of their conversion to TEI.
\paragraph[{Citation Structures}]{Citation Structures}\label{SACRCS}\par
Citation structures provide a more thorough and concise mechanism for describing canonical references and the ways those references map on to parts of a TEI document. A \hyperref[TEI.citeStructure]{<citeStructure>} element describes a single step in a reference, such as ‘Matt’, and may nest to handle multi-part references.\par
The equivalent structure to the set of \hyperref[TEI.cRefPattern]{<cRefPattern>}s in \textit{\hyperref[SACRWE]{16.2.5.1.\ Worked Example}} would be: \par\bgroup\index{refsDecl=<refsDecl>|exampleindex}\index{citeStructure=<citeStructure>|exampleindex}\index{unit=@unit!<citeStructure>|exampleindex}\index{match=@match!<citeStructure>|exampleindex}\index{use=@use!<citeStructure>|exampleindex}\index{citeStructure=<citeStructure>|exampleindex}\index{unit=@unit!<citeStructure>|exampleindex}\index{match=@match!<citeStructure>|exampleindex}\index{use=@use!<citeStructure>|exampleindex}\index{delim=@delim!<citeStructure>|exampleindex}\index{citeStructure=<citeStructure>|exampleindex}\index{unit=@unit!<citeStructure>|exampleindex}\index{match=@match!<citeStructure>|exampleindex}\index{use=@use!<citeStructure>|exampleindex}\index{delim=@delim!<citeStructure>|exampleindex}\exampleFont \begin{shaded}\noindent\mbox{}{<\textbf{refsDecl}\hspace*{1em}{xml:id}="{biblical2}">}\mbox{}\newline 
\hspace*{1em}{<\textbf{citeStructure}\hspace*{1em}{unit}="{book}"\hspace*{1em}{match}="{//div}"\mbox{}\newline 
\hspace*{1em}\hspace*{1em}{use}="{@n}">}\mbox{}\newline 
\hspace*{1em}\hspace*{1em}{<\textbf{citeStructure}\hspace*{1em}{unit}="{chapter}"\hspace*{1em}{match}="{div}"\mbox{}\newline 
\hspace*{1em}\hspace*{1em}\hspace*{1em}{use}="{@n}"\hspace*{1em}{delim}="{ }">}\mbox{}\newline 
\hspace*{1em}\hspace*{1em}\hspace*{1em}{<\textbf{citeStructure}\hspace*{1em}{unit}="{verse}"\hspace*{1em}{match}="{div}"\mbox{}\newline 
\hspace*{1em}\hspace*{1em}\hspace*{1em}\hspace*{1em}{use}="{@n}"\hspace*{1em}{delim}="{:}"/>}\mbox{}\newline 
\hspace*{1em}\hspace*{1em}{</\textbf{citeStructure}>}\mbox{}\newline 
\hspace*{1em}{</\textbf{citeStructure}>}\mbox{}\newline 
{</\textbf{refsDecl}>}\end{shaded}\egroup\par \par
An application wishing to resolve a canonical reference such as Matt 5:7 might follow this procedure: \begin{enumerate}
\item Ascertain the correct \hyperref[TEI.refsDecl]{<refsDecl>} following the rules summarized in section \textit{\hyperref[CCAS3]{15.3.3.\ Summary}}.
\item Begin with the outer <citeStructure>. If it has a {\itshape delim} attribute and the reference begins with the value of {\itshape delim} then take the portion of the reference after the value of {\itshape delim} as input for the next child <citeStructure>.
\item For each nested <citeStructure>, if the input reference string contains the value of the {\itshape delim}, then split the string on the value of the {\itshape delim} attribute. If the input string does not contain {\itshape delim}, then stop. Take the portion of the input string after the value of {\itshape delim} and use it as the input string for the child <citeStructure>. After processing the outer <citeStructure>, the output will be \texttt{('Matt 5:7')}, after the second, \texttt{('Matt', '5:7')}. The end result will be a sequence like \texttt{('Matt','5','7')}.
\item For each item in the resulting sequence, resolve the matching node by evaluating the XPath in {\itshape match} with the predicate found in {\itshape use}, using the context of the previously matched node, if any. Start with the outer \hyperref[TEI.citeStructure]{<citeStructure>} and move to the next child <citeStructure> for each step in the sequence. For example, for the first <citeStructure>, we could construct an XPath \texttt{//div[@n='Matt']}. The full XPath after the reference Matt 5:7 is resolved will be \texttt{//div[@n='Matt']/div[@n='5']/div[@n='7']}.
\end{enumerate} One advantage \hyperref[TEI.citeStructure]{<citeStructure>} has is that it can be used to \textit{generate} canonical references, using the declared citation structure to query the text structure. This means it is possible to automatically produce a list of resolvable citations for a TEI document. It also enables the automatic breaking of documents into smaller chunks for presentation and automated generation of tables of contents.\par
Citation structures may in addition specify how informational properties are to be extracted from the document sections they identify, using the \hyperref[TEI.citeData]{<citeData>} element. For example, if a TEI document is divided into chapters with a \hyperref[TEI.div]{<div>} per chapter and those chapters have titles, contained in \hyperref[TEI.head]{<head>} elements, then we might declare a citation structure for the document thus: \par\bgroup\index{citeStructure=<citeStructure>|exampleindex}\index{unit=@unit!<citeStructure>|exampleindex}\index{match=@match!<citeStructure>|exampleindex}\index{use=@use!<citeStructure>|exampleindex}\index{delim=@delim!<citeStructure>|exampleindex}\index{citeData=<citeData>|exampleindex}\index{property=@property!<citeData>|exampleindex}\index{use=@use!<citeData>|exampleindex}\exampleFont \begin{shaded}\noindent\mbox{}{<\textbf{citeStructure}\hspace*{1em}{unit}="{chapter}"\mbox{}\newline 
\hspace*{1em}{match}="{/TEI/text/body/div}"\hspace*{1em}{use}="{position()}"\hspace*{1em}{delim}="{ch. }">}\mbox{}\newline 
\hspace*{1em}{<\textbf{citeData}\hspace*{1em}{property}="{http://purl.org/dc/terms/title}"\mbox{}\newline 
\hspace*{1em}\hspace*{1em}{use}="{head}"/>}\mbox{}\newline 
{</\textbf{citeStructure}>}\end{shaded}\egroup\par \par
This specifies that chapter references are given in the form ‘ch. n’, where ‘n’ is the position of the \hyperref[TEI.div]{<div>} in the \hyperref[TEI.body]{<body>} of the document, and that we may obtain the title of the chapter (identified by the Dublin Core property \texttt{title}) from the chapter heading. This would, for example, enable the automated generation of a chapter listing for the document.
\subsection[{Blocks, Segments, and Anchors}]{Blocks, Segments, and Anchors}\label{SASE}\par
In this section, we discuss three general purposes elements which may be used to mark and categorize both a span of text and a point within one. These elements have several uses, most notably to provide elements which can be given identifiers for use when aligning or linking to parts of a document, as discussed elsewhere in this chapter. They also provide a convenient way of extending the semantics of the TEI markup scheme in a theory-neutral manner, by providing for two neutral or ‘anonymous’ elements to which the encoder can add any meaning not supplied by other TEI defined elements. 
\begin{sansreflist}
  
\item [\textbf{<anchor>}] (anchor point) attaches an identifier to a point within a text, whether or not it corresponds with a textual element.
\item [\textbf{<ab>}] (anonymous block) contains any arbitrary component-level unit of text, acting as an anonymous container for phrase or inter level elements analogous to, but without the semantic baggage of, a paragraph.
\item [\textbf{<seg>}] (arbitrary segment) represents any segmentation of text below the ‘chunk’ level.
\end{sansreflist}
 The elements \hyperref[TEI.anchor]{<anchor>}, \hyperref[TEI.ab]{<ab>}, and \hyperref[TEI.seg]{<seg>} are members of the class \textsf{att.typed}, from which they inherit the following attributes: 
\begin{sansreflist}
  
\item [\textbf{att.typed}] provides attributes which can be used to classify or subclassify elements in any way.\hfil\\[-10pt]\begin{sansreflist}
    \item[@{\itshape type}]
  characterizes the element in some sense, using any convenient classification scheme or typology.
    \item[@{\itshape subtype}]
  (subtype) provides a sub-categorization of the element, if needed
\end{sansreflist}  
\end{sansreflist}
 The elements \hyperref[TEI.ab]{<ab>}, and \hyperref[TEI.seg]{<seg>} are members of the class \textsf{att.fragmentable}, from which they inherit the following attribute: 
\begin{sansreflist}
  
\item [\textbf{att.fragmentable}] provides an attribute for representing fragmentation of a structural element, typically as a consequence of some overlapping hierarchy.\hfil\\[-10pt]\begin{sansreflist}
    \item[@{\itshape part}]
  specifies whether or not its parent element is fragmented in some way, typically by some other overlapping structure: for example a speech which is divided between two or more verse stanzas, a paragraph which is split across a page division, a verse line which is divided between two speakers.
\end{sansreflist}  
\end{sansreflist}
 The \hyperref[TEI.seg]{<seg>} element is also a member of the class \textsf{att.segLike} from which it inherits the following attribute: 
\begin{sansreflist}
  
\item [\textbf{att.segLike}] provides attributes for elements used for arbitrary segmentation.\hfil\\[-10pt]\begin{sansreflist}
    \item[@{\itshape function}]
  (function) characterizes the function of the segment.
\end{sansreflist}  
\end{sansreflist}
\par
The \hyperref[TEI.anchor]{<anchor>} element may be thought of as an empty \hyperref[TEI.seg]{<seg>}, or as an artifice enabling an identifier to be attached to any position in a text. Like the \hyperref[TEI.milestone]{<milestone>} element discussed in section \textit{\hyperref[CORS]{3.11.\ Reference Systems}}, it is useful where multiple views of a document are to be combined, for example, when a logical view based on paragraphs or verse lines is to be mapped on to a physical view based on manuscript lines. Like those elements, it is a member of the class \textsf{model.global} and can therefore appear anywhere within a document when the module defined by this chapter is included in a schema. Unlike the other elements in its class, the \hyperref[TEI.anchor]{<anchor>} element is primarily intended to mark an arbitrary point used for alignment, or as the target of a spanning element such as those discussed in section \textit{\hyperref[PHAD]{11.3.1.4.\ Additions and Deletions}}, rather than as a means of marking segment boundaries for some arbitrary segmentation of a text.\par
For example, suppose that we wish to mark the end of the fifth word following each occurrence of some term in a particular text, perhaps to assist with some collocational analysis. This can most easily be done with the help of the \hyperref[TEI.anchor]{<anchor>} element, as follows:  \par\bgroup\index{anchor=<anchor>|exampleindex}\index{anchor=<anchor>|exampleindex}\index{anchor=<anchor>|exampleindex}\index{anchor=<anchor>|exampleindex}\exampleFont \begin{shaded}\noindent\mbox{}English language. Except for not very{<\textbf{anchor}\hspace*{1em}{xml:id}="{eng1}"/>}\mbox{}\newline 
 English at all at the time{<\textbf{anchor}\hspace*{1em}{xml:id}="{eng2}"/>}\mbox{}\newline 
 English was still full of flaws{<\textbf{anchor}\hspace*{1em}{xml:id}="{eng3}"/>}\mbox{}\newline 
 English. This was revised by young\mbox{}\newline 
{<\textbf{anchor}\hspace*{1em}{xml:id}="{eng4}"/>}\end{shaded}\egroup\par \noindent  In section \textit{\hyperref[SACS1]{16.5.1.\ Correspondence}} we discuss ways in which these \hyperref[TEI.anchor]{<anchor>} points might be used to represent an alignment such as one might get in a keyword-in-context concordance.\par
The \hyperref[TEI.seg]{<seg>} element may be used at the encoder's discretion to mark almost any segment of the text of interest for processing. One use of the element is to mark text features for which no appropriate markup is otherwise defined, i.e. as a simple extension mechanism. Another use is to provide an identifier for some segment which is to be pointed at by some other element, i.e. to provide a target, or a part of a target, for a \hyperref[TEI.ptr]{<ptr>} or other similar element.\par
Several examples of uses for the \hyperref[TEI.seg]{<seg>} element are provided elsewhere in these Guidelines. For example: \begin{itemize}
\item as a means of marking segments significant in a metrical or rhyming analysis (see section \textit{\hyperref[VEME]{6.4.\ Rhyme and Metrical Analysis}})
\item as a means of marking typographic lines in drama (see section \textit{\hyperref[DRBOD]{7.2.\ The Body of a Performance Text}}) or title pages (see section \textit{\hyperref[DSTITL]{4.6.\ Title Pages}})
\item as a means of marking prosody- or pause-defined units in transcribed speech (see section \textit{\hyperref[TSSASE]{8.4.1.\ Segmentation}})
\item as a means of marking linguistic or other analyses in a theory-neutral manner (see chapter \textit{\hyperref[AI]{17.\ Simple Analytic Mechanisms}} passim)
\end{itemize} \par
In the following simple example, the \hyperref[TEI.seg]{<seg>} element simply delimits the extent of a stutter, a textual feature for which no element is provided in these Guidelines. \par\bgroup\index{q=<q>|exampleindex}\index{q=<q>|exampleindex}\index{seg=<seg>|exampleindex}\index{type=@type!<seg>|exampleindex}\index{q=<q>|exampleindex}\exampleFont \begin{shaded}\noindent\mbox{}{<\textbf{q}>}Don't say {<\textbf{q}>}\mbox{}\newline 
\hspace*{1em}\hspace*{1em}{<\textbf{seg}\hspace*{1em}{type}="{stutter}">}I-I-I{</\textbf{seg}>}'m afraid,{</\textbf{q}>} Melvin, just say {<\textbf{q}>}I'm\mbox{}\newline 
\hspace*{1em}\hspace*{1em} afraid.{</\textbf{q}>}\mbox{}\newline 
{</\textbf{q}>}\end{shaded}\egroup\par \noindent   The \hyperref[TEI.seg]{<seg>} element is particularly useful for the markup of linguistically significant constituents such as the phrases that may be the output of an automatic parsing system. This example also demonstrates the use of the {\itshape xml:id} attribute to carry an identifier which other parts of a document may use to point to, or align with: \par\bgroup\index{seg=<seg>|exampleindex}\index{type=@type!<seg>|exampleindex}\index{seg=<seg>|exampleindex}\index{type=@type!<seg>|exampleindex}\index{seg=<seg>|exampleindex}\index{type=@type!<seg>|exampleindex}\index{seg=<seg>|exampleindex}\index{type=@type!<seg>|exampleindex}\exampleFont \begin{shaded}\noindent\mbox{}{<\textbf{seg}\hspace*{1em}{xml:id}="{bl0034}"\hspace*{1em}{type}="{sentence}">}\mbox{}\newline 
\hspace*{1em}{<\textbf{seg}\hspace*{1em}{xml:id}="{bl0034.1}"\hspace*{1em}{type}="{phrase}">}Literate and illiterate speech{</\textbf{seg}>}\mbox{}\newline 
\hspace*{1em}{<\textbf{seg}\hspace*{1em}{xml:id}="{bl0034.2}"\hspace*{1em}{type}="{phrase}">}in a language like English{</\textbf{seg}>}\mbox{}\newline 
\hspace*{1em}{<\textbf{seg}\hspace*{1em}{xml:id}="{bl0034.3}"\hspace*{1em}{type}="{phrase}">}are plainly different.{</\textbf{seg}>}\mbox{}\newline 
{</\textbf{seg}>}\end{shaded}\egroup\par \noindent  \par
As the above example shows, \hyperref[TEI.seg]{<seg>} elements may be nested directly within one another, to any degree of analysis considered appropriate. This is taken a little further in the following example, where the {\itshape type} and {\itshape subtype} attributes have been used to further categorize each word of the sentence (the {\itshape xml:id} attributes have been removed to reduce the complexity of the example): \par\bgroup\index{seg=<seg>|exampleindex}\index{type=@type!<seg>|exampleindex}\index{subtype=@subtype!<seg>|exampleindex}\index{seg=<seg>|exampleindex}\index{type=@type!<seg>|exampleindex}\index{subtype=@subtype!<seg>|exampleindex}\index{seg=<seg>|exampleindex}\index{type=@type!<seg>|exampleindex}\index{subtype=@subtype!<seg>|exampleindex}\index{seg=<seg>|exampleindex}\index{type=@type!<seg>|exampleindex}\index{subtype=@subtype!<seg>|exampleindex}\index{seg=<seg>|exampleindex}\index{type=@type!<seg>|exampleindex}\index{subtype=@subtype!<seg>|exampleindex}\index{seg=<seg>|exampleindex}\index{type=@type!<seg>|exampleindex}\index{subtype=@subtype!<seg>|exampleindex}\index{seg=<seg>|exampleindex}\index{type=@type!<seg>|exampleindex}\index{subtype=@subtype!<seg>|exampleindex}\index{seg=<seg>|exampleindex}\index{type=@type!<seg>|exampleindex}\index{subtype=@subtype!<seg>|exampleindex}\index{seg=<seg>|exampleindex}\index{type=@type!<seg>|exampleindex}\index{subtype=@subtype!<seg>|exampleindex}\index{seg=<seg>|exampleindex}\index{type=@type!<seg>|exampleindex}\index{subtype=@subtype!<seg>|exampleindex}\index{seg=<seg>|exampleindex}\index{type=@type!<seg>|exampleindex}\index{subtype=@subtype!<seg>|exampleindex}\index{seg=<seg>|exampleindex}\index{type=@type!<seg>|exampleindex}\index{subtype=@subtype!<seg>|exampleindex}\index{seg=<seg>|exampleindex}\index{type=@type!<seg>|exampleindex}\index{subtype=@subtype!<seg>|exampleindex}\index{seg=<seg>|exampleindex}\index{type=@type!<seg>|exampleindex}\index{subtype=@subtype!<seg>|exampleindex}\index{seg=<seg>|exampleindex}\index{type=@type!<seg>|exampleindex}\index{subtype=@subtype!<seg>|exampleindex}\index{seg=<seg>|exampleindex}\index{type=@type!<seg>|exampleindex}\index{subtype=@subtype!<seg>|exampleindex}\index{seg=<seg>|exampleindex}\index{type=@type!<seg>|exampleindex}\exampleFont \begin{shaded}\noindent\mbox{}{<\textbf{seg}\hspace*{1em}{type}="{sentence}"\hspace*{1em}{subtype}="{declarative}">}\mbox{}\newline 
\hspace*{1em}{<\textbf{seg}\hspace*{1em}{type}="{phrase}"\hspace*{1em}{subtype}="{noun}">}\mbox{}\newline 
\hspace*{1em}\hspace*{1em}{<\textbf{seg}\hspace*{1em}{type}="{word}"\hspace*{1em}{subtype}="{adjective}">}Literate{</\textbf{seg}>}\mbox{}\newline 
\hspace*{1em}\hspace*{1em}{<\textbf{seg}\hspace*{1em}{type}="{word}"\hspace*{1em}{subtype}="{conjunction}">}and{</\textbf{seg}>}\mbox{}\newline 
\hspace*{1em}\hspace*{1em}{<\textbf{seg}\hspace*{1em}{type}="{word}"\hspace*{1em}{subtype}="{adjective}">}illiterate{</\textbf{seg}>}\mbox{}\newline 
\hspace*{1em}\hspace*{1em}{<\textbf{seg}\hspace*{1em}{type}="{word}"\hspace*{1em}{subtype}="{noun}">}speech{</\textbf{seg}>}\mbox{}\newline 
\hspace*{1em}{</\textbf{seg}>}\mbox{}\newline 
\hspace*{1em}{<\textbf{seg}\hspace*{1em}{type}="{phrase}"\hspace*{1em}{subtype}="{preposition}">}\mbox{}\newline 
\hspace*{1em}\hspace*{1em}{<\textbf{seg}\hspace*{1em}{type}="{word}"\hspace*{1em}{subtype}="{preposition}">}in{</\textbf{seg}>}\mbox{}\newline 
\hspace*{1em}\hspace*{1em}{<\textbf{seg}\hspace*{1em}{type}="{word}"\hspace*{1em}{subtype}="{article}">}a{</\textbf{seg}>}\mbox{}\newline 
\hspace*{1em}\hspace*{1em}{<\textbf{seg}\hspace*{1em}{type}="{word}"\hspace*{1em}{subtype}="{noun}">}language{</\textbf{seg}>}\mbox{}\newline 
\hspace*{1em}\hspace*{1em}{<\textbf{seg}\hspace*{1em}{type}="{word}"\hspace*{1em}{subtype}="{preposition}">}like{</\textbf{seg}>}\mbox{}\newline 
\hspace*{1em}\hspace*{1em}{<\textbf{seg}\hspace*{1em}{type}="{word}"\hspace*{1em}{subtype}="{noun}">}English{</\textbf{seg}>}\mbox{}\newline 
\hspace*{1em}{</\textbf{seg}>}\mbox{}\newline 
\hspace*{1em}{<\textbf{seg}\hspace*{1em}{type}="{phrase}"\hspace*{1em}{subtype}="{verb}">}\mbox{}\newline 
\hspace*{1em}\hspace*{1em}{<\textbf{seg}\hspace*{1em}{type}="{word}"\hspace*{1em}{subtype}="{verb}">}are{</\textbf{seg}>}\mbox{}\newline 
\hspace*{1em}\hspace*{1em}{<\textbf{seg}\hspace*{1em}{type}="{word}"\hspace*{1em}{subtype}="{adverb}">}plainly{</\textbf{seg}>}\mbox{}\newline 
\hspace*{1em}\hspace*{1em}{<\textbf{seg}\hspace*{1em}{type}="{word}"\hspace*{1em}{subtype}="{adjective}">}different{</\textbf{seg}>}\mbox{}\newline 
\hspace*{1em}{</\textbf{seg}>}\mbox{}\newline 
\hspace*{1em}{<\textbf{seg}\hspace*{1em}{type}="{punct}">}.{</\textbf{seg}>}\mbox{}\newline 
{</\textbf{seg}>}\end{shaded}\egroup\par \par
(The example values shown are chosen for simplicity of comprehension, rather than verisimilitude). It should also be noted that specialized segment elements are defined in section \textit{\hyperref[AILC]{17.1.\ Linguistic Segment Categories}} to facilitate this particular kind of analysis. These allow for the explicit markup of units called \textit{s-units}, \textit{clauses}, \textit{phrases}, \textit{words}, \textit{morphemes}, and \textit{characters}, which may be felt preferable to the more generic approach typified by use of the \hyperref[TEI.seg]{<seg>} element. Using these, the first phrase above might be encoded simply as \par\bgroup\index{phr=<phr>|exampleindex}\index{type=@type!<phr>|exampleindex}\index{w=<w>|exampleindex}\index{type=@type!<w>|exampleindex}\index{w=<w>|exampleindex}\index{type=@type!<w>|exampleindex}\index{w=<w>|exampleindex}\index{type=@type!<w>|exampleindex}\index{w=<w>|exampleindex}\index{type=@type!<w>|exampleindex}\exampleFont \begin{shaded}\noindent\mbox{}{<\textbf{phr}\hspace*{1em}{type}="{noun}">}\mbox{}\newline 
\hspace*{1em}{<\textbf{w}\hspace*{1em}{type}="{adjective}">}Literate{</\textbf{w}>}\mbox{}\newline 
\hspace*{1em}{<\textbf{w}\hspace*{1em}{type}="{conjunction}">}and{</\textbf{w}>}\mbox{}\newline 
\hspace*{1em}{<\textbf{w}\hspace*{1em}{type}="{adjective}">}illiterate{</\textbf{w}>}\mbox{}\newline 
\hspace*{1em}{<\textbf{w}\hspace*{1em}{type}="{noun}">}speech{</\textbf{w}>}\mbox{}\newline 
{</\textbf{phr}>}\end{shaded}\egroup\par \noindent  Note the way in which the {\itshape type} attribute of these specialized elements now carries the value carried by the {\itshape subtype} attribute of the more general \hyperref[TEI.seg]{<seg>} element. For an analysis not using these traditional linguistic categories however, the \hyperref[TEI.seg]{<seg>} element provides a simple but powerful mechanism.\par
In language corpora and similar material, the \hyperref[TEI.seg]{<seg>} element may be used to provide an end-to-end segmentation as an alternative to the more specific \hyperref[TEI.s]{<s>} element proposed in chapter \textit{\hyperref[AILC]{17.1.\ Linguistic Segment Categories}} for the markup of orthographic sentences, or \textit{s-units}. However, it may be more useful to use the \hyperref[TEI.s]{<s>} element for this purpose, since this means that the \hyperref[TEI.seg]{<seg>} element can then be used to mark both features within s-units and segments composed of s-units, as in the following example:\footnote{See section \textit{\hyperref[AISP]{17.3.\ Spans and Interpretations}}, where the text from which this fragment is taken is analyzed.} \par\bgroup\index{seg=<seg>|exampleindex}\index{type=@type!<seg>|exampleindex}\index{s=<s>|exampleindex}\index{seg=<seg>|exampleindex}\index{type=@type!<seg>|exampleindex}\index{s=<s>|exampleindex}\index{s=<s>|exampleindex}\exampleFont \begin{shaded}\noindent\mbox{}{<\textbf{seg}\hspace*{1em}{xml:id}="{s1s3}"\hspace*{1em}{type}="{narrative\textunderscore unit}">}\mbox{}\newline 
\hspace*{1em}{<\textbf{s}\hspace*{1em}{xml:id}="{s1}">}Sigmund, the {<\textbf{seg}\hspace*{1em}{type}="{patronymic}">}son of Volsung{</\textbf{seg}>},\mbox{}\newline 
\hspace*{1em}\hspace*{1em} was a king in Frankish country.{</\textbf{s}>}\mbox{}\newline 
\hspace*{1em}{<\textbf{s}\hspace*{1em}{xml:id}="{s2}">}Sinfiotli was the eldest of his sons.{</\textbf{s}>}\mbox{}\newline 
\hspace*{1em}{<\textbf{s}\hspace*{1em}{xml:id}="{s3}">} ... {</\textbf{s}>}\mbox{}\newline 
{</\textbf{seg}>}\end{shaded}\egroup\par \par
Like other elements, the \hyperref[TEI.seg]{<seg>} tag must be properly enclosed within other elements. Thus, a single \hyperref[TEI.seg]{<seg>} element can be used to group together words in different sentences only if the sentences are not themselves tagged. The first of the following two encodings is legal, but the second is not.  \par\bgroup\index{seg=<seg>|exampleindex}\index{type=@type!<seg>|exampleindex}\exampleFont \begin{shaded}\noindent\mbox{}Give me {<\textbf{seg}\hspace*{1em}{type}="{phrase}">}a dozen. Or two or three.{</\textbf{seg}>}\end{shaded}\egroup\par \noindent  \par\hfill\bgroup\exampleFont\vskip 10pt\begin{shaded}
\obeyspaces <!-- Illegal! -->\newline
<s>Give me <seg type="phrase">a dozen.</s>\newline
<s>Or two or three.</s></seg>\end{shaded}
\par\egroup 
\par
The {\itshape part} attribute may be used as one simple method of overcoming this restriction: \par\bgroup\index{s=<s>|exampleindex}\index{seg=<seg>|exampleindex}\index{type=@type!<seg>|exampleindex}\index{part=@part!<seg>|exampleindex}\index{s=<s>|exampleindex}\index{seg=<seg>|exampleindex}\index{part=@part!<seg>|exampleindex}\exampleFont \begin{shaded}\noindent\mbox{}{<\textbf{s}>}Give me {<\textbf{seg}\hspace*{1em}{type}="{phrase}"\hspace*{1em}{part}="{I}">}a dozen.{</\textbf{seg}>}\mbox{}\newline 
{</\textbf{s}>}\mbox{}\newline 
{<\textbf{s}>}\mbox{}\newline 
\hspace*{1em}{<\textbf{seg}\hspace*{1em}{part}="{F}">}Or two or three.{</\textbf{seg}>}\mbox{}\newline 
{</\textbf{s}>}\end{shaded}\egroup\par \noindent  Another solution is to use the \hyperref[TEI.join]{<join>} element discussed in section \textit{\hyperref[SAAG]{16.7.\ Aggregation}}; this requires that each of the \hyperref[TEI.seg]{<seg>} elements be given an identifier. For further discussion of this generic encoding problem, see also chapter \textit{\hyperref[NH]{20.\ Non-hierarchical Structures}}.\par
The \hyperref[TEI.seg]{<seg>} element has the same content as a paragraph in prose: it can therefore be used to group together consecutive sequences of \textsf{model.inter} class elements, such as lists, quotations, notes, stage directions, etc. as well as to contain sequences of phrase-level elements. It cannot however be used to group together sequences of paragraphs or similar text units such as verse lines; for this purpose, the encoder should use intermediate pointers, as described in section \textit{\hyperref[SAPTIP]{16.1.4.\ Intermediate Pointers}} or the methods described in section \textit{\hyperref[SAAG]{16.7.\ Aggregation}}. It is particularly important that the encoder provide a clear description of the principles by which a text has been segmented, and the way in which that segmentation is represented. This should include a description of the method used and the significance of any categorization codes. The description should be provided as a series of paragraphs within the \hyperref[TEI.segmentation]{<segmentation>} element of the encoding description in the TEI header, as described in section \textit{\hyperref[HD53]{2.3.3.\ The Editorial Practices Declaration}}.\par
The \hyperref[TEI.seg]{<seg>} element may also be used to encode simultaneous or mutually exclusive variants of a text when the more special purpose elements for simple editorial changes, abbreviation and expansion, addition and deletion, or for a critical apparatus are not appropriate. In these circumstances, one \hyperref[TEI.seg]{<seg>} is encoded for each possible variant, and the set of them is enclosed in a \hyperref[TEI.choice]{<choice>} element.\par
For example, if one were writing dual-platform instructions for installation of software, it might be useful to use \hyperref[TEI.seg]{<seg>} to record platform-specific pieces of mutually exclusive text. \par\bgroup\index{choice=<choice>|exampleindex}\index{seg=<seg>|exampleindex}\index{type=@type!<seg>|exampleindex}\index{subtype=@subtype!<seg>|exampleindex}\index{seg=<seg>|exampleindex}\index{type=@type!<seg>|exampleindex}\index{subtype=@subtype!<seg>|exampleindex}\exampleFont \begin{shaded}\noindent\mbox{}…pressing {<\textbf{choice}>}\mbox{}\newline 
\hspace*{1em}{<\textbf{seg}\hspace*{1em}{type}="{platform}"\hspace*{1em}{subtype}="{Mac}">}option{</\textbf{seg}>}\mbox{}\newline 
\hspace*{1em}{<\textbf{seg}\hspace*{1em}{type}="{platform}"\hspace*{1em}{subtype}="{PC}">}alt{</\textbf{seg}>}\mbox{}\newline 
{</\textbf{choice}>}-f will …\end{shaded}\egroup\par \par
Elsewhere in this chapter we provide a number of examples where the \hyperref[TEI.seg]{<seg>} element is used simply to provide an element to which an identifier may be attached, for example so that another segment may be linked or related to it in some way.\par
The \hyperref[TEI.ab]{<ab>} (anonymous block) element performs a similar function to that of the \hyperref[TEI.seg]{<seg>} element, but is used for portions of the text which occur not within paragraphs or other component-level elements, but at the component level themselves. It is therefore a member of the \textsf{model.pLike} class.\par
The \hyperref[TEI.ab]{<ab>} element may be used, for example, to tag the canonical verse divisions of Biblical texts: \par\bgroup\index{div1=<div1>|exampleindex}\index{n=@n!<div1>|exampleindex}\index{type=@type!<div1>|exampleindex}\index{head=<head>|exampleindex}\index{head=<head>|exampleindex}\index{type=@type!<head>|exampleindex}\index{div2=<div2>|exampleindex}\index{n=@n!<div2>|exampleindex}\index{type=@type!<div2>|exampleindex}\index{ab=<ab>|exampleindex}\index{n=@n!<ab>|exampleindex}\index{ab=<ab>|exampleindex}\index{n=@n!<ab>|exampleindex}\index{hi=<hi>|exampleindex}\index{ab=<ab>|exampleindex}\index{n=@n!<ab>|exampleindex}\exampleFont \begin{shaded}\noindent\mbox{}{<\textbf{div1}\hspace*{1em}{n}="{Gen}"\hspace*{1em}{type}="{book}">}\mbox{}\newline 
\hspace*{1em}{<\textbf{head}>}The First Book of Moses, Called{</\textbf{head}>}\mbox{}\newline 
\hspace*{1em}{<\textbf{head}\hspace*{1em}{type}="{main}">}Genesis{</\textbf{head}>}\mbox{}\newline 
\hspace*{1em}{<\textbf{div2}\hspace*{1em}{n}="{1}"\hspace*{1em}{type}="{chapter}">}\mbox{}\newline 
\hspace*{1em}\hspace*{1em}{<\textbf{ab}\hspace*{1em}{n}="{1}">}In the beginning God created the heaven and the\mbox{}\newline 
\hspace*{1em}\hspace*{1em}\hspace*{1em}\hspace*{1em} earth.{</\textbf{ab}>}\mbox{}\newline 
\hspace*{1em}\hspace*{1em}{<\textbf{ab}\hspace*{1em}{n}="{2}">}And the earth was without form, and void; and darkness\mbox{}\newline 
\hspace*{1em}\hspace*{1em}{<\textbf{hi}>}was{</\textbf{hi}>} upon the face of the deep. And the Spirit of God\mbox{}\newline 
\hspace*{1em}\hspace*{1em}\hspace*{1em}\hspace*{1em} moved upon the face of the waters.{</\textbf{ab}>}\mbox{}\newline 
\hspace*{1em}\hspace*{1em}{<\textbf{ab}\hspace*{1em}{n}="{3}">}And God said, Let there be light: and there was\mbox{}\newline 
\hspace*{1em}\hspace*{1em}\hspace*{1em}\hspace*{1em} light.{</\textbf{ab}>}\mbox{}\newline 
\hspace*{1em}{</\textbf{div2}>}\mbox{}\newline 
{</\textbf{div1}>}\end{shaded}\egroup\par \noindent  \par
In other cases, where the text clearly indicates paragraph divisions containing one or more verses, the \hyperref[TEI.p]{<p>} element may be used to tag the paragraphs, and the \hyperref[TEI.seg]{<seg>} element used to subdivide them. The \hyperref[TEI.ab]{<ab>} element is provided as an alternative to the \hyperref[TEI.p]{<p>} element; it may \textit{not} be used within paragraphs. The \hyperref[TEI.seg]{<seg>} element, by contrast, may appear only within and not between paragraphs (or anonymous block elements). \par\bgroup\index{div1=<div1>|exampleindex}\index{n=@n!<div1>|exampleindex}\index{type=@type!<div1>|exampleindex}\index{head=<head>|exampleindex}\index{div2=<div2>|exampleindex}\index{n=@n!<div2>|exampleindex}\index{type=@type!<div2>|exampleindex}\index{p=<p>|exampleindex}\index{seg=<seg>|exampleindex}\index{n=@n!<seg>|exampleindex}\index{seg=<seg>|exampleindex}\index{n=@n!<seg>|exampleindex}\index{p=<p>|exampleindex}\index{seg=<seg>|exampleindex}\index{n=@n!<seg>|exampleindex}\exampleFont \begin{shaded}\noindent\mbox{}{<\textbf{div1}\hspace*{1em}{n}="{Gen}"\hspace*{1em}{type}="{book}">}\mbox{}\newline 
\hspace*{1em}{<\textbf{head}>}Das Erste Buch Mose.{</\textbf{head}>}\mbox{}\newline 
\hspace*{1em}{<\textbf{div2}\hspace*{1em}{n}="{1}"\hspace*{1em}{type}="{chapter}">}\mbox{}\newline 
\hspace*{1em}\hspace*{1em}{<\textbf{p}>}\mbox{}\newline 
\hspace*{1em}\hspace*{1em}\hspace*{1em}{<\textbf{seg}\hspace*{1em}{n}="{1}">}Am Anfang schuff Gott Himel vnd Erden.{</\textbf{seg}>}\mbox{}\newline 
\hspace*{1em}\hspace*{1em}\hspace*{1em}{<\textbf{seg}\hspace*{1em}{n}="{2}">}Vnd die Erde war wüst vnd leer / vnd es war\mbox{}\newline 
\hspace*{1em}\hspace*{1em}\hspace*{1em}\hspace*{1em}\hspace*{1em}\hspace*{1em} finster auff der Tieffe / Vnd der Geist Gottes schwebet auff\mbox{}\newline 
\hspace*{1em}\hspace*{1em}\hspace*{1em}\hspace*{1em}\hspace*{1em}\hspace*{1em} dem Wasser.{</\textbf{seg}>}\mbox{}\newline 
\hspace*{1em}\hspace*{1em}{</\textbf{p}>}\mbox{}\newline 
\hspace*{1em}\hspace*{1em}{<\textbf{p}>}\mbox{}\newline 
\hspace*{1em}\hspace*{1em}\hspace*{1em}{<\textbf{seg}\hspace*{1em}{n}="{3}">}Vnd Gott sprach / Es werde Liecht / Vnd es ward\mbox{}\newline 
\hspace*{1em}\hspace*{1em}\hspace*{1em}\hspace*{1em}\hspace*{1em}\hspace*{1em} Liecht.{</\textbf{seg}>}\mbox{}\newline 
\hspace*{1em}\hspace*{1em}{</\textbf{p}>}\mbox{}\newline 
\hspace*{1em}{</\textbf{div2}>}\mbox{}\newline 
{</\textbf{div1}>}\end{shaded}\egroup\par \noindent  \par
The \hyperref[TEI.ab]{<ab>} element is also useful for marking dramatic speeches when it is not clear whether the speech is to be regarded as prose or verse. If, for example, an encoder does not wish to express an opinion as to whether the opening lines of Shakespeare's \textit{The Tempest} are to be regarded as prose or as verse, they might be tagged as follows: \par\bgroup\index{div1=<div1>|exampleindex}\index{n=@n!<div1>|exampleindex}\index{type=@type!<div1>|exampleindex}\index{div2=<div2>|exampleindex}\index{n=@n!<div2>|exampleindex}\index{type=@type!<div2>|exampleindex}\index{head=<head>|exampleindex}\index{rend=@rend!<head>|exampleindex}\index{stage=<stage>|exampleindex}\index{rend=@rend!<stage>|exampleindex}\index{type=@type!<stage>|exampleindex}\index{sp=<sp>|exampleindex}\index{speaker=<speaker>|exampleindex}\index{ab=<ab>|exampleindex}\index{sp=<sp>|exampleindex}\index{speaker=<speaker>|exampleindex}\index{ab=<ab>|exampleindex}\index{sp=<sp>|exampleindex}\index{speaker=<speaker>|exampleindex}\index{ab=<ab>|exampleindex}\index{stage=<stage>|exampleindex}\index{type=@type!<stage>|exampleindex}\index{stage=<stage>|exampleindex}\index{type=@type!<stage>|exampleindex}\index{sp=<sp>|exampleindex}\index{speaker=<speaker>|exampleindex}\index{ab=<ab>|exampleindex}\exampleFont \begin{shaded}\noindent\mbox{}{<\textbf{div1}\hspace*{1em}{n}="{I}"\hspace*{1em}{type}="{act}">}\mbox{}\newline 
\hspace*{1em}{<\textbf{div2}\hspace*{1em}{n}="{1}"\hspace*{1em}{type}="{scene}">}\mbox{}\newline 
\hspace*{1em}\hspace*{1em}{<\textbf{head}\hspace*{1em}{rend}="{italic}">}Actus primus, Scena prima.{</\textbf{head}>}\mbox{}\newline 
\hspace*{1em}\hspace*{1em}{<\textbf{stage}\hspace*{1em}{rend}="{italic}"\hspace*{1em}{type}="{setting}">} A tempestuous noise of\mbox{}\newline 
\hspace*{1em}\hspace*{1em}\hspace*{1em}\hspace*{1em} Thunder and Lightning heard:\mbox{}\newline 
\hspace*{1em}\hspace*{1em}\hspace*{1em}\hspace*{1em} Enter a Ship-master, and a Boteswaine.{</\textbf{stage}>}\mbox{}\newline 
\hspace*{1em}\hspace*{1em}{<\textbf{sp}>}\mbox{}\newline 
\hspace*{1em}\hspace*{1em}\hspace*{1em}{<\textbf{speaker}>}Master.{</\textbf{speaker}>}\mbox{}\newline 
\hspace*{1em}\hspace*{1em}\hspace*{1em}{<\textbf{ab}>}Bote-swaine.{</\textbf{ab}>}\mbox{}\newline 
\hspace*{1em}\hspace*{1em}{</\textbf{sp}>}\mbox{}\newline 
\hspace*{1em}\hspace*{1em}{<\textbf{sp}>}\mbox{}\newline 
\hspace*{1em}\hspace*{1em}\hspace*{1em}{<\textbf{speaker}>}Botes.{</\textbf{speaker}>}\mbox{}\newline 
\hspace*{1em}\hspace*{1em}\hspace*{1em}{<\textbf{ab}>}Heere Master: What cheere?{</\textbf{ab}>}\mbox{}\newline 
\hspace*{1em}\hspace*{1em}{</\textbf{sp}>}\mbox{}\newline 
\hspace*{1em}\hspace*{1em}{<\textbf{sp}>}\mbox{}\newline 
\hspace*{1em}\hspace*{1em}\hspace*{1em}{<\textbf{speaker}>}Mast.{</\textbf{speaker}>}\mbox{}\newline 
\hspace*{1em}\hspace*{1em}\hspace*{1em}{<\textbf{ab}>}Good: Speake to th' Mariners: fall too't, yarely,\mbox{}\newline 
\hspace*{1em}\hspace*{1em}\hspace*{1em}\hspace*{1em}\hspace*{1em}\hspace*{1em} or we run our selues a ground, bestirre, bestirre.\mbox{}\newline 
\hspace*{1em}\hspace*{1em}\hspace*{1em}{<\textbf{stage}\hspace*{1em}{type}="{move}">}Exit.{</\textbf{stage}>}\mbox{}\newline 
\hspace*{1em}\hspace*{1em}\hspace*{1em}{</\textbf{ab}>}\mbox{}\newline 
\hspace*{1em}\hspace*{1em}{</\textbf{sp}>}\mbox{}\newline 
\hspace*{1em}\hspace*{1em}{<\textbf{stage}\hspace*{1em}{type}="{move}">}Enter Mariners.{</\textbf{stage}>}\mbox{}\newline 
\hspace*{1em}\hspace*{1em}{<\textbf{sp}>}\mbox{}\newline 
\hspace*{1em}\hspace*{1em}\hspace*{1em}{<\textbf{speaker}>}Botes.{</\textbf{speaker}>}\mbox{}\newline 
\hspace*{1em}\hspace*{1em}\hspace*{1em}{<\textbf{ab}>}Heigh my hearts, cheerely, cheerely my harts: yare, yare:\mbox{}\newline 
\hspace*{1em}\hspace*{1em}\hspace*{1em}\hspace*{1em}\hspace*{1em}\hspace*{1em} Take in the toppe-sale: Tend to th' Masters whistle: Blow\mbox{}\newline 
\hspace*{1em}\hspace*{1em}\hspace*{1em}\hspace*{1em}\hspace*{1em}\hspace*{1em} till thou burst thy winde, if roome e-nough.{</\textbf{ab}>}\mbox{}\newline 
\hspace*{1em}\hspace*{1em}{</\textbf{sp}>}\mbox{}\newline 
\hspace*{1em}{</\textbf{div2}>}\mbox{}\newline 
{</\textbf{div1}>}\end{shaded}\egroup\par \noindent  See further \textit{\hyperref[CODR]{3.13.2.\ Core Tags for Drama}} and \textit{\hyperref[DRPAL]{7.2.5.\ Speech Contents}}.
\subsection[{Synchronization}]{Synchronization}\label{SASY}\par
In the previous section we discussed two particular kinds of alignment: alignment of parallel texts in different languages; and alignment of texts and portions of an image. In this section we address another specialized form of alignment: synchronization. The need to mark the relative positions of text components with respect to time arises most naturally and frequently in transcribed spoken texts, but it may arise in any text in which quoted speech occurs, or events are described within a time frame. The methods described here are also generalizable for other kinds of alignment (for example, alignment of text elements with respect to space).
\subsubsection[{Aligning Synchronous Events}]{Aligning Synchronous Events}\label{SASYNC}\par
Provided that explicit elements are available to represent the parts or places to be synchronized, then the global linking attribute {\itshape synch} may be used to encode such synchronization, once it has been identified. 
\begin{sansreflist}
  
\item [\textbf{att.global.linking}] provides a set of attributes for hypertextual linking.\hfil\\[-10pt]\begin{sansreflist}
    \item[@{\itshape synch}]
  (synchronous) points to elements that are synchronous with the current element.
\end{sansreflist}  
\end{sansreflist}
 This is another of the attributes made globally available by the mechanism described in the introduction to this chapter. Alternatively, the \hyperref[TEI.link]{<link>} and \hyperref[TEI.linkGrp]{<linkGrp>} elements may be used to make explicit the fact that the synchronous elements are aligned.\par
To illustrate the use of these mechanisms for marking synchrony, consider the following representation of a spoken text:  \par\hfill\bgroup\exampleFont\vskip 10pt\begin{shaded}
\obeyspaces B: The first time in twenty five years, we've cooked Christmas\newline
   (unclear) for a blooming great load of people.\newline
A: So you're [1] (unclear) [2]\newline
B: [1] It will be [2] nice in a way, but, [3] be strange. [4]\newline
A: [3] Yeah [4], yeah, cos it, it's [5] the [6]\newline
B: [5] not [6]\end{shaded}
\par\egroup 
\par
This representation uses numbers in brackets to mark the points at which speakers overlap each other. For example, the \textit{[1]} in A's first speech is to be understood as coinciding with the \textit{[1]} in B's second speech.\footnote{This sample is taken from a conversation collected and transcribed for the British National Corpus.}\par
To encode this we use the spoken texts module, described in chapter \textit{\hyperref[TS]{8.\ Transcriptions of Speech}}, together with the module described in the present chapter. First, we transcribe this text, marking the synchronous points with \hyperref[TEI.anchor]{<anchor>} elements, and providing a {\itshape synch} attribute on one of each of the pairs of synchronous anchors. As noted in the example given above (section \textit{\hyperref[SACSAL]{16.5.2.\ Alignment of Parallel Texts}}), correspondence, and hence synchrony, is a symmetric relation; therefore the attribute need only be specified on one of the pairs of synchronous anchors. \par\bgroup\index{div=<div>|exampleindex}\index{type=@type!<div>|exampleindex}\index{u=<u>|exampleindex}\index{who=@who!<u>|exampleindex}\index{unclear=<unclear>|exampleindex}\index{u=<u>|exampleindex}\index{who=@who!<u>|exampleindex}\index{anchor=<anchor>|exampleindex}\index{synch=@synch!<anchor>|exampleindex}\index{unclear=<unclear>|exampleindex}\index{anchor=<anchor>|exampleindex}\index{synch=@synch!<anchor>|exampleindex}\index{u=<u>|exampleindex}\index{who=@who!<u>|exampleindex}\index{anchor=<anchor>|exampleindex}\index{anchor=<anchor>|exampleindex}\index{anchor=<anchor>|exampleindex}\index{anchor=<anchor>|exampleindex}\index{u=<u>|exampleindex}\index{who=@who!<u>|exampleindex}\index{anchor=<anchor>|exampleindex}\index{synch=@synch!<anchor>|exampleindex}\index{anchor=<anchor>|exampleindex}\index{synch=@synch!<anchor>|exampleindex}\index{anchor=<anchor>|exampleindex}\index{synch=@synch!<anchor>|exampleindex}\index{anchor=<anchor>|exampleindex}\index{synch=@synch!<anchor>|exampleindex}\index{u=<u>|exampleindex}\index{who=@who!<u>|exampleindex}\index{anchor=<anchor>|exampleindex}\index{anchor=<anchor>|exampleindex}\exampleFont \begin{shaded}\noindent\mbox{}{<\textbf{div}\hspace*{1em}{xml:id}="{BNC-d1}"\hspace*{1em}{type}="{convers}">}\mbox{}\newline 
\hspace*{1em}{<\textbf{u}\hspace*{1em}{xml:id}="{u2b}"\hspace*{1em}{who}="{\#b}">} The first time in twenty five years,\mbox{}\newline 
\hspace*{1em}\hspace*{1em} we've cooked Christmas {<\textbf{unclear}>} for a blooming great\mbox{}\newline 
\hspace*{1em}\hspace*{1em}\hspace*{1em}\hspace*{1em} load of people.{</\textbf{unclear}>}\mbox{}\newline 
\hspace*{1em}{</\textbf{u}>}\mbox{}\newline 
\hspace*{1em}{<\textbf{u}\hspace*{1em}{xml:id}="{u3a}"\hspace*{1em}{who}="{\#a}">}So you're\mbox{}\newline 
\hspace*{1em}{<\textbf{anchor}\hspace*{1em}{synch}="{\#t1b}"\hspace*{1em}{xml:id}="{t1a}"/>}\mbox{}\newline 
\hspace*{1em}\hspace*{1em}{<\textbf{unclear}>}\mbox{}\newline 
\hspace*{1em}\hspace*{1em}\hspace*{1em}{<\textbf{anchor}\hspace*{1em}{synch}="{\#t2b}"\hspace*{1em}{xml:id}="{t2a}"/>}\mbox{}\newline 
\hspace*{1em}\hspace*{1em}{</\textbf{unclear}>}\mbox{}\newline 
\hspace*{1em}{</\textbf{u}>}\mbox{}\newline 
\hspace*{1em}{<\textbf{u}\hspace*{1em}{xml:id}="{u3b}"\hspace*{1em}{who}="{\#b}">}\mbox{}\newline 
\hspace*{1em}\hspace*{1em}{<\textbf{anchor}\hspace*{1em}{xml:id}="{t1b}"/>}It will be {<\textbf{anchor}\hspace*{1em}{xml:id}="{t2b}"/>}\mbox{}\newline 
\hspace*{1em}\hspace*{1em} nice in a way, but, {<\textbf{anchor}\hspace*{1em}{xml:id}="{t3b}"/>}\mbox{}\newline 
\hspace*{1em}\hspace*{1em} be strange.{<\textbf{anchor}\hspace*{1em}{xml:id}="{t4b}"/>}\mbox{}\newline 
\hspace*{1em}{</\textbf{u}>}\mbox{}\newline 
\hspace*{1em}{<\textbf{u}\hspace*{1em}{xml:id}="{u4a}"\hspace*{1em}{who}="{\#a}">}\mbox{}\newline 
\hspace*{1em}\hspace*{1em}{<\textbf{anchor}\hspace*{1em}{synch}="{\#t3b}"\hspace*{1em}{xml:id}="{t3a}"/>}Yeah\mbox{}\newline 
\hspace*{1em}{<\textbf{anchor}\hspace*{1em}{synch}="{\#t4b}"\hspace*{1em}{xml:id}="{t4a}"/>}, yeah, cos it, its\mbox{}\newline 
\hspace*{1em}{<\textbf{anchor}\hspace*{1em}{synch}="{\#t5b}"\hspace*{1em}{xml:id}="{t5a}"/>}the\mbox{}\newline 
\hspace*{1em}{<\textbf{anchor}\hspace*{1em}{synch}="{\#t6b}"\hspace*{1em}{xml:id}="{t6a}"/>}\mbox{}\newline 
\hspace*{1em}{</\textbf{u}>}\mbox{}\newline 
\hspace*{1em}{<\textbf{u}\hspace*{1em}{xml:id}="{u4b}"\hspace*{1em}{who}="{\#b}">}\mbox{}\newline 
\hspace*{1em}\hspace*{1em}{<\textbf{anchor}\hspace*{1em}{xml:id}="{t5b}"/>}not{<\textbf{anchor}\hspace*{1em}{xml:id}="{t6b}"/>}\mbox{}\newline 
\hspace*{1em}{</\textbf{u}>}\mbox{}\newline 
\textit{<!-- ... -->}\mbox{}\newline 
{</\textbf{div}>}\end{shaded}\egroup\par \par
We can encode this same example using \hyperref[TEI.link]{<link>} and \hyperref[TEI.linkGrp]{<linkGrp>} elements to make the temporal alignment explicit. A \hyperref[TEI.back]{<back>} element has been used to enclose the \hyperref[TEI.linkGrp]{<linkGrp>} element, but the links may be located anywhere the encoder finds convenient: \par\bgroup\index{back=<back>|exampleindex}\index{linkGrp=<linkGrp>|exampleindex}\index{domains=@domains!<linkGrp>|exampleindex}\index{targFunc=@targFunc!<linkGrp>|exampleindex}\index{type=@type!<linkGrp>|exampleindex}\index{link=<link>|exampleindex}\index{target=@target!<link>|exampleindex}\index{link=<link>|exampleindex}\index{target=@target!<link>|exampleindex}\index{link=<link>|exampleindex}\index{target=@target!<link>|exampleindex}\index{link=<link>|exampleindex}\index{target=@target!<link>|exampleindex}\index{link=<link>|exampleindex}\index{target=@target!<link>|exampleindex}\index{link=<link>|exampleindex}\index{target=@target!<link>|exampleindex}\exampleFont \begin{shaded}\noindent\mbox{}{<\textbf{back}>}\mbox{}\newline 
\hspace*{1em}{<\textbf{linkGrp}\hspace*{1em}{xml:id}="{lg1}"\mbox{}\newline 
\hspace*{1em}\hspace*{1em}{domains}="{\#BNC-d1 \#BNC-d1}"\hspace*{1em}{targFunc}="{speaker.a speaker.b}"\mbox{}\newline 
\hspace*{1em}\hspace*{1em}{type}="{synchronous\textunderscore alignment}">}\mbox{}\newline 
\hspace*{1em}\hspace*{1em}{<\textbf{link}\hspace*{1em}{xml:id}="{L1}"\hspace*{1em}{target}="{\#t1a \#t1b}"/>}\mbox{}\newline 
\hspace*{1em}\hspace*{1em}{<\textbf{link}\hspace*{1em}{xml:id}="{L2}"\hspace*{1em}{target}="{\#t2a \#t2b}"/>}\mbox{}\newline 
\hspace*{1em}\hspace*{1em}{<\textbf{link}\hspace*{1em}{xml:id}="{L3}"\hspace*{1em}{target}="{\#t3a \#t3b}"/>}\mbox{}\newline 
\hspace*{1em}\hspace*{1em}{<\textbf{link}\hspace*{1em}{xml:id}="{l4}"\hspace*{1em}{target}="{\#t4a \#t4b}"/>}\mbox{}\newline 
\hspace*{1em}\hspace*{1em}{<\textbf{link}\hspace*{1em}{xml:id}="{l5}"\hspace*{1em}{target}="{\#t5a \#t5b}"/>}\mbox{}\newline 
\hspace*{1em}\hspace*{1em}{<\textbf{link}\hspace*{1em}{xml:id}="{l6}"\hspace*{1em}{target}="{\#t6a \#t6b}"/>}\mbox{}\newline 
\hspace*{1em}{</\textbf{linkGrp}>}\mbox{}\newline 
{</\textbf{back}>}\end{shaded}\egroup\par \noindent  The {\itshape xml:id} attributes are provided for the \hyperref[TEI.link]{<link>} and \hyperref[TEI.linkGrp]{<linkGrp>} elements here for reasons discussed in the next section, \textit{\hyperref[SASYMP]{16.4.2.\ Placing Synchronous Events in Time}}.\par
As with other forms of alignment, synchronization may be expressed between stretches of speech as well as between points. When complete utterances are synchronous, for example, if one person says \textit{What?} and another \textit{No!} at the same time, that can be represented without \hyperref[TEI.anchor]{<anchor>} elements as follows. \par\bgroup\index{u=<u>|exampleindex}\index{synch=@synch!<u>|exampleindex}\index{who=@who!<u>|exampleindex}\index{u=<u>|exampleindex}\index{who=@who!<u>|exampleindex}\exampleFont \begin{shaded}\noindent\mbox{}{<\textbf{u}\hspace*{1em}{synch}="{\#u02}"\hspace*{1em}{xml:id}="{u01}"\hspace*{1em}{who}="{\#a}">}What?{</\textbf{u}>}\mbox{}\newline 
{<\textbf{u}\hspace*{1em}{xml:id}="{u02}"\hspace*{1em}{who}="{\#b}">}No!{</\textbf{u}>}\end{shaded}\egroup\par \par
A simple way of expressing \textit{overlap} (where one speaker starts speaking before another has finished) is thus to use the \hyperref[TEI.seg]{<seg>} element to encode the overlapping portions of speech. For example, \par\bgroup\index{u=<u>|exampleindex}\index{who=@who!<u>|exampleindex}\index{unclear=<unclear>|exampleindex}\index{synch=@synch!<unclear>|exampleindex}\index{u=<u>|exampleindex}\index{who=@who!<u>|exampleindex}\index{seg=<seg>|exampleindex}\index{seg=<seg>|exampleindex}\index{synch=@synch!<seg>|exampleindex}\index{u=<u>|exampleindex}\index{who=@who!<u>|exampleindex}\index{seg=<seg>|exampleindex}\index{seg=<seg>|exampleindex}\index{synch=@synch!<seg>|exampleindex}\index{u=<u>|exampleindex}\index{who=@who!<u>|exampleindex}\exampleFont \begin{shaded}\noindent\mbox{}{<\textbf{u}\hspace*{1em}{who}="{\#a}">} So you're {<\textbf{unclear}\hspace*{1em}{synch}="{\#u-b1}"/>}\mbox{}\newline 
{</\textbf{u}>}\mbox{}\newline 
{<\textbf{u}\hspace*{1em}{who}="{\#b}">}\mbox{}\newline 
\hspace*{1em}{<\textbf{seg}\hspace*{1em}{xml:id}="{u-b1}">} It will be {</\textbf{seg}>} nice in a way, but,\mbox{}\newline 
{<\textbf{seg}\hspace*{1em}{synch}="{\#u-a3}">} be strange. {</\textbf{seg}>}\mbox{}\newline 
{</\textbf{u}>}\mbox{}\newline 
{<\textbf{u}\hspace*{1em}{who}="{\#a}">}\mbox{}\newline 
\hspace*{1em}{<\textbf{seg}\hspace*{1em}{xml:id}="{u-a3}">} Yeah {</\textbf{seg}>}, yeah, cos it,\mbox{}\newline 
 its {<\textbf{seg}\hspace*{1em}{synch}="{\#u-b2}">} the {</\textbf{seg}>}\mbox{}\newline 
{</\textbf{u}>}\mbox{}\newline 
{<\textbf{u}\hspace*{1em}{xml:id}="{u-b2}"\hspace*{1em}{who}="{\#b}">} not {</\textbf{u}>}\end{shaded}\egroup\par \noindent  Note in this encoding how synchronization has been effected between an empty \hyperref[TEI.unclear]{<unclear>} element and the content of a \hyperref[TEI.seg]{<seg>} element, and between the content of a \hyperref[TEI.u]{<u>} element and that of another \hyperref[TEI.seg]{<seg>}, using the {\itshape synch} attribute. Alternatively, a \hyperref[TEI.linkGrp]{<linkGrp>} could be used in the same way as above.
\subsubsection[{Placing Synchronous Events in Time}]{Placing Synchronous Events in Time}\label{SASYMP}\par
A synchronous alignment specifies which points in a spoken text occur at the same time, and the order in which they occur, but does not say at what time those points actually occur. If that information is available to the encoder it can be represented by means of the \hyperref[TEI.when]{<when>} and \hyperref[TEI.timeline]{<timeline>} elements, whose description and attributes are the following: 
\begin{sansreflist}
  
\item [\textbf{<when>}] indicates a point in time either relative to other elements in the same timeline tag, or absolutely.\hfil\\[-10pt]\begin{sansreflist}
    \item[@{\itshape absolute}]
  supplies an absolute value for the time.
    \item[@{\itshape interval}]
  specifies a time interval either as a number or as one of the keywords defined by the datatype \textsf{teidata.interval}
    \item[@{\itshape unit}]
  specifies the unit of time in which the {\itshape interval} value is expressed, if this is not inherited from the parent \hyperref[TEI.timeline]{<timeline>}.
    \item[@{\itshape since}]
  identifies the reference point for determining the time of the current \hyperref[TEI.when]{<when>} element, which is obtained by adding the interval to the time of the reference point.
\end{sansreflist}  
\item [\textbf{<timeline>}] (timeline) provides a set of ordered points in time which can be linked to elements of a spoken text to create a temporal alignment of that text.\hfil\\[-10pt]\begin{sansreflist}
    \item[@{\itshape origin}]
  designates the origin of the timeline, i.e. the time at which it begins.
    \item[@{\itshape interval}]
  specifies a time interval either as a positive integral value or using one of a set of predefined codes.
    \item[@{\itshape unit}]
  specifies the unit of time corresponding to the {\itshape interval} value of the timeline or of its constituent points in time.
\end{sansreflist}  
\end{sansreflist}
\par
Each \hyperref[TEI.when]{<when>} element indicates a point in time, either directly by means of the {\itshape absolute} attribute, whose value is a string which specifies a particular time, or indirectly by means of the {\itshape since} attribute, which points to another \hyperref[TEI.when]{<when>}. If the {\itshape since} is used, then the {\itshape interval} and {\itshape unit} attributes should also be used to indicate the amount of time that has elapsed since the time specified by the element pointed to by the {\itshape since} attribute; the value -1 can be given to indicate that the interval is unknown.\par
If the \hyperref[TEI.when]{<when>} elements are uniformly spaced in time, then the {\itshape interval} and {\itshape unit} values need be given once in the \hyperref[TEI.timeline]{<timeline>}, and not repeated in any of the \hyperref[TEI.when]{<when>} elements. If the intervals vary, but the units are all the same, then the {\itshape unit} attribute alone can be given in the \hyperref[TEI.timeline]{<timeline>} element, and the {\itshape interval} attribute given in the \hyperref[TEI.when]{<when>} element.\par
The {\itshape origin} attribute in the \hyperref[TEI.timeline]{<timeline>} element points to a \hyperref[TEI.when]{<when>} element which specifies the reference or origin for the timings within the \hyperref[TEI.timeline]{<timeline>}; this must, of course, specify its position in time absolutely. If the origin of a timeline is unknown, then this attribute may be omitted.\par
The following \hyperref[TEI.timeline]{<timeline>} might be used to accompany the marked up conversation shown in the preceding section: \par\bgroup\index{timeline=<timeline>|exampleindex}\index{origin=@origin!<timeline>|exampleindex}\index{unit=@unit!<timeline>|exampleindex}\index{when=<when>|exampleindex}\index{absolute=@absolute!<when>|exampleindex}\index{when=<when>|exampleindex}\index{interval=@interval!<when>|exampleindex}\index{since=@since!<when>|exampleindex}\index{when=<when>|exampleindex}\index{interval=@interval!<when>|exampleindex}\index{since=@since!<when>|exampleindex}\index{when=<when>|exampleindex}\index{interval=@interval!<when>|exampleindex}\index{since=@since!<when>|exampleindex}\index{when=<when>|exampleindex}\index{interval=@interval!<when>|exampleindex}\index{since=@since!<when>|exampleindex}\index{when=<when>|exampleindex}\index{interval=@interval!<when>|exampleindex}\index{since=@since!<when>|exampleindex}\index{when=<when>|exampleindex}\index{interval=@interval!<when>|exampleindex}\index{since=@since!<when>|exampleindex}\exampleFont \begin{shaded}\noindent\mbox{}{<\textbf{timeline}\hspace*{1em}{xml:id}="{tL1}"\hspace*{1em}{origin}="{\#w0}"\mbox{}\newline 
\hspace*{1em}{unit}="{ms}">}\mbox{}\newline 
\hspace*{1em}{<\textbf{when}\hspace*{1em}{xml:id}="{w0}"\hspace*{1em}{absolute}="{11:30:00}"/>}\mbox{}\newline 
\hspace*{1em}{<\textbf{when}\hspace*{1em}{xml:id}="{w1}"\hspace*{1em}{interval}="{unknown}"\mbox{}\newline 
\hspace*{1em}\hspace*{1em}{since}="{\#w0}"/>}\mbox{}\newline 
\hspace*{1em}{<\textbf{when}\hspace*{1em}{xml:id}="{w2}"\hspace*{1em}{interval}="{100}"\mbox{}\newline 
\hspace*{1em}\hspace*{1em}{since}="{\#w1}"/>}\mbox{}\newline 
\hspace*{1em}{<\textbf{when}\hspace*{1em}{xml:id}="{w3}"\hspace*{1em}{interval}="{200}"\mbox{}\newline 
\hspace*{1em}\hspace*{1em}{since}="{\#w2}"/>}\mbox{}\newline 
\hspace*{1em}{<\textbf{when}\hspace*{1em}{xml:id}="{w4}"\hspace*{1em}{interval}="{150}"\mbox{}\newline 
\hspace*{1em}\hspace*{1em}{since}="{\#w3}"/>}\mbox{}\newline 
\hspace*{1em}{<\textbf{when}\hspace*{1em}{xml:id}="{w5}"\hspace*{1em}{interval}="{250}"\mbox{}\newline 
\hspace*{1em}\hspace*{1em}{since}="{\#w4}"/>}\mbox{}\newline 
\hspace*{1em}{<\textbf{when}\hspace*{1em}{xml:id}="{w6}"\hspace*{1em}{interval}="{100}"\mbox{}\newline 
\hspace*{1em}\hspace*{1em}{since}="{\#w5}"/>}\mbox{}\newline 
{</\textbf{timeline}>}\end{shaded}\egroup\par \noindent  The information in this \hyperref[TEI.timeline]{<timeline>} could now be linked to the information in the \hyperref[TEI.linkGrp]{<linkGrp>} which provides the temporal alignment (synchronization) for the text, as follows: \par\bgroup\index{linkGrp=<linkGrp>|exampleindex}\index{type=@type!<linkGrp>|exampleindex}\index{domains=@domains!<linkGrp>|exampleindex}\index{targFunc=@targFunc!<linkGrp>|exampleindex}\index{link=<link>|exampleindex}\index{target=@target!<link>|exampleindex}\index{link=<link>|exampleindex}\index{target=@target!<link>|exampleindex}\index{link=<link>|exampleindex}\index{target=@target!<link>|exampleindex}\index{link=<link>|exampleindex}\index{target=@target!<link>|exampleindex}\index{link=<link>|exampleindex}\index{target=@target!<link>|exampleindex}\index{link=<link>|exampleindex}\index{target=@target!<link>|exampleindex}\exampleFont \begin{shaded}\noindent\mbox{}{<\textbf{linkGrp}\hspace*{1em}{type}="{temporal\textunderscore specification}"\mbox{}\newline 
\hspace*{1em}{domains}="{\#lg1 \#tL1}"\hspace*{1em}{targFunc}="{synch.points when}">}\mbox{}\newline 
\hspace*{1em}{<\textbf{link}\hspace*{1em}{target}="{\#L1 \#w1}"/>}\mbox{}\newline 
\hspace*{1em}{<\textbf{link}\hspace*{1em}{target}="{\#L2 \#w2}"/>}\mbox{}\newline 
\hspace*{1em}{<\textbf{link}\hspace*{1em}{target}="{\#L3 \#w3}"/>}\mbox{}\newline 
\hspace*{1em}{<\textbf{link}\hspace*{1em}{target}="{\#l4 \#w4}"/>}\mbox{}\newline 
\hspace*{1em}{<\textbf{link}\hspace*{1em}{target}="{\#l5 \#w5}"/>}\mbox{}\newline 
\hspace*{1em}{<\textbf{link}\hspace*{1em}{target}="{\#l6 \#w6}"/>}\mbox{}\newline 
{</\textbf{linkGrp}>}\end{shaded}\egroup\par \par
To avoid the need for two distinct link groups (one marking the synchronization of anchors with each other, and the other marking their alignment with points on the time line) it would be better to link the \hyperref[TEI.when]{<when>} elements with the synchronous points directly: \par\bgroup\index{linkGrp=<linkGrp>|exampleindex}\index{type=@type!<linkGrp>|exampleindex}\index{domains=@domains!<linkGrp>|exampleindex}\index{targFunc=@targFunc!<linkGrp>|exampleindex}\index{link=<link>|exampleindex}\index{target=@target!<link>|exampleindex}\index{link=<link>|exampleindex}\index{target=@target!<link>|exampleindex}\index{link=<link>|exampleindex}\index{target=@target!<link>|exampleindex}\index{link=<link>|exampleindex}\index{target=@target!<link>|exampleindex}\index{link=<link>|exampleindex}\index{target=@target!<link>|exampleindex}\index{link=<link>|exampleindex}\index{target=@target!<link>|exampleindex}\exampleFont \begin{shaded}\noindent\mbox{}{<\textbf{linkGrp}\hspace*{1em}{type}="{temporal\textunderscore specification}"\mbox{}\newline 
\hspace*{1em}{domains}="{\#BNC-d1 \#BNC-d1 \#tL1}"\hspace*{1em}{targFunc}="{speaker.a speaker.b when}">}\mbox{}\newline 
\hspace*{1em}{<\textbf{link}\hspace*{1em}{target}="{\#t1a \#t1b \#w1}"/>}\mbox{}\newline 
\hspace*{1em}{<\textbf{link}\hspace*{1em}{target}="{\#t2a \#t2b \#w2}"/>}\mbox{}\newline 
\hspace*{1em}{<\textbf{link}\hspace*{1em}{target}="{\#t3a \#t3b \#w3}"/>}\mbox{}\newline 
\hspace*{1em}{<\textbf{link}\hspace*{1em}{target}="{\#t4a \#t4b \#w4}"/>}\mbox{}\newline 
\hspace*{1em}{<\textbf{link}\hspace*{1em}{target}="{\#t5a \#t5b \#w5}"/>}\mbox{}\newline 
\hspace*{1em}{<\textbf{link}\hspace*{1em}{target}="{\#t6a \#t6b \#w6}"/>}\mbox{}\newline 
{</\textbf{linkGrp}>}\end{shaded}\egroup\par \par
Finally, suppose that a digitized audio recording is also available, and an XML file that assigns identifiers to the various temporal spans of sound is available. For example, the following Synchronized Multimedia Integration Language (SMIL, pronounced "smile") fragment: \par\bgroup\exampleFont \begin{shaded}\noindent\mbox{}{<\textbf{audio} xmlns="http://www.w3.org/2001/SMIL20/Language"\hspace*{1em}{src}="{rtsp://soundstage.pi.cnr.it:554/home/az/bncSound/xmas4lots.mp3}"\mbox{}\newline 
\hspace*{1em}{xml:id}="{au1}"\hspace*{1em}{begin}="{05.2s}"/>}\mbox{}\newline 
{<\textbf{audio} xmlns="http://www.w3.org/2001/SMIL20/Language"\hspace*{1em}{src}="{rtsp://soundstage.pi.cnr.it:554/home/az/bncSound/xmas4lots.mp3}"\mbox{}\newline 
\hspace*{1em}{xml:id}="{au2}"\hspace*{1em}{begin}="{05.7s}"/>}\mbox{}\newline 
{<\textbf{audio} xmlns="http://www.w3.org/2001/SMIL20/Language"\hspace*{1em}{src}="{rtsp://soundstage.pi.cnr.it:554/home/az/bncSound/xmas4lots.mp3}"\mbox{}\newline 
\hspace*{1em}{xml:id}="{au3}"\hspace*{1em}{begin}="{05.9s}"/>}\mbox{}\newline 
{<\textbf{audio} xmlns="http://www.w3.org/2001/SMIL20/Language"\hspace*{1em}{src}="{rtsp://soundstage.pi.cnr.it:554/home/az/bncSound/xmas4lots.mp3}"\mbox{}\newline 
\hspace*{1em}{xml:id}="{au4}"\hspace*{1em}{begin}="{06.3s}"/>}\mbox{}\newline 
{<\textbf{audio} xmlns="http://www.w3.org/2001/SMIL20/Language"\hspace*{1em}{src}="{rtsp://soundstage.pi.cnr.it:554/home/az/bncSound/xmas4lots.mp3}"\mbox{}\newline 
\hspace*{1em}{xml:id}="{au5}"\hspace*{1em}{begin}="{06.9s}"/>}\mbox{}\newline 
{<\textbf{audio} xmlns="http://www.w3.org/2001/SMIL20/Language"\hspace*{1em}{src}="{rtsp://soundstage.pi.cnr.it:554/home/az/bncSound/xmas4lots.mp3}"\mbox{}\newline 
\hspace*{1em}{xml:id}="{au6}"\hspace*{1em}{begin}="{07.4s}"/>}\end{shaded}\egroup\par \noindent  URIs pointing to the \texttt{<audio>} elements could also be included as a fourth component in each of the above \hyperref[TEI.link]{<link>} elements, thus providing a synchronized audio track to complement the transcribed text.\par
For further discussion of this and related aspects of encoding transcribed speech, refer to chapter \textit{\hyperref[TS]{8.\ Transcriptions of Speech}}.
\subsection[{Correspondence and Alignment}]{Correspondence and Alignment}\label{SACS}\par
In this section we introduce the notions of \textit{correspondence}, expressed by the {\itshape corresp} attribute, and of \textit{alignment}, which is a special kind of correspondence involving an ordered set of correspondences. Both cases may be represented using the \hyperref[TEI.link]{<link>} and \hyperref[TEI.linkGrp]{<linkGrp>} elements introduced in section \textit{\hyperref[SAPT]{16.1.\ Links}}. We also discuss the special case of alignment in time or \textit{synchronization}, for which special purpose elements are proposed in section \textit{\hyperref[SASY]{16.4.\ Synchronization}}.
\subsubsection[{Correspondence}]{Correspondence}\label{SACS1}\par
A common requirement in text analysis is to represent correspondences between two or more parts of a single document, or between places in different documents. Provided that explicit elements are available to represent the parts or places to be linked, then the global linking attribute {\itshape corresp} may be used to encode such correspondence, once it has been identified. 
\begin{sansreflist}
  
\item [\textbf{att.global.linking}] provides a set of attributes for hypertextual linking.\hfil\\[-10pt]\begin{sansreflist}
    \item[@{\itshape corresp}]
  (corresponds) points to elements that correspond to the current element in some way.
\end{sansreflist}  
\end{sansreflist}
 This is one of the attributes made available by the mechanism described in the introduction to this chapter (\textit{\hyperref[SA]{16.\ Linking, Segmentation, and Alignment}}). Correspondence can also be expressed by means of the \hyperref[TEI.link]{<link>} element introduced in section \textit{\hyperref[SAPT]{16.1.\ Links}}.\par
Where the correspondence is between \textit{spans}, the \hyperref[TEI.seg]{<seg>} element should be used, if no other element is available. Where the correspondence is between \textit{points}, the \hyperref[TEI.anchor]{<anchor>} element should be used, if no other element is available.\par
The use of the {\itshape corresp} attribute with spans of content is illustrated by the following example: \par\bgroup\index{title=<title>|exampleindex}\index{name=<name>|exampleindex}\index{seg=<seg>|exampleindex}\index{corresp=@corresp!<seg>|exampleindex}\index{seg=<seg>|exampleindex}\index{corresp=@corresp!<seg>|exampleindex}\exampleFont \begin{shaded}\noindent\mbox{}{<\textbf{title}\hspace*{1em}{xml:id}="{SHIRLEY}">}Shirley{</\textbf{title}>}, which made\mbox{}\newline 
 its Friday night debut only a month ago, was\mbox{}\newline 
 not listed on {<\textbf{name}\hspace*{1em}{xml:id}="{NBC}">}NBC{</\textbf{name}>}'s new schedule,\mbox{}\newline 
 although {<\textbf{seg}\hspace*{1em}{corresp}="{\#NBC}"\hspace*{1em}{xml:id}="{NETWORK}">}the network{</\textbf{seg}>}\mbox{}\newline 
 says {<\textbf{seg}\hspace*{1em}{corresp}="{\#SHIRLEY}"\hspace*{1em}{xml:id}="{SHOW}">}the show{</\textbf{seg}>}\mbox{}\newline 
 still is being considered.\end{shaded}\egroup\par \noindent     Here the anaphoric phrases \textit{the network} and \textit{the show} have been associated directly with the elements to which they refer by means of {\itshape corresp} attributes. This mechanism is simple to apply, but has the drawback that it is not possible to specify more exactly what kind of correspondence is intended. Where this attribute is used, therefore, encoders are encouraged to specify their intent in the associated encoding description in the TEI header.\par
Essentially, what the {\itshape corresp} attribute does is to specify that elements bearing this attribute and those to which the attribute points are doubly linked. In the example above, the use of the {\itshape corresp} attribute indicates that the \hyperref[TEI.seg]{<seg>} element containing ‘the show’ and the \hyperref[TEI.title]{<title>} element containing ‘Shirley’ correspond to each other: the correspondence relationship is not ‘from’ one to the other, but ‘between’ the two objects. It is thus different from the {\itshape target} attribute, and provides functionality more similar to that of the \hyperref[TEI.link]{<link>} and \hyperref[TEI.linkGrp]{<linkGrp>} elements defined in section \textit{\hyperref[SAPT]{16.1.\ Links}}, although it lacks the ability to indicate more precisely what kind of correspondence is intended as in the following retagging of the preceding example. \par\bgroup\index{title=<title>|exampleindex}\index{name=<name>|exampleindex}\index{seg=<seg>|exampleindex}\index{seg=<seg>|exampleindex}\index{linkGrp=<linkGrp>|exampleindex}\index{type=@type!<linkGrp>|exampleindex}\index{targFunc=@targFunc!<linkGrp>|exampleindex}\index{link=<link>|exampleindex}\index{target=@target!<link>|exampleindex}\index{link=<link>|exampleindex}\index{target=@target!<link>|exampleindex}\exampleFont \begin{shaded}\noindent\mbox{}{<\textbf{title}\hspace*{1em}{xml:id}="{shirley}">}Shirley{</\textbf{title}>}, which made\mbox{}\newline 
 its Friday night debut only a month ago, was not\mbox{}\newline 
 listed on {<\textbf{name}\hspace*{1em}{xml:id}="{nbc}">}NBC{</\textbf{name}>}'s new schedule,\mbox{}\newline 
 although {<\textbf{seg}\hspace*{1em}{xml:id}="{network}">}the network{</\textbf{seg}>} says\mbox{}\newline 
{<\textbf{seg}\hspace*{1em}{xml:id}="{show}">}the show{</\textbf{seg}>} still is being considered.\mbox{}\newline 
\mbox{}\newline 
{<\textbf{linkGrp}\hspace*{1em}{type}="{anaphoric\textunderscore link}"\mbox{}\newline 
\hspace*{1em}{targFunc}="{antecedent anaphor}">}\mbox{}\newline 
\hspace*{1em}{<\textbf{link}\hspace*{1em}{target}="{\#shirley \#show}"/>}\mbox{}\newline 
\hspace*{1em}{<\textbf{link}\hspace*{1em}{target}="{\#nbc \#network}"/>}\mbox{}\newline 
{</\textbf{linkGrp}>}\end{shaded}\egroup\par \par
In the following example, we use the same mechanism to express a correspondence amongst the anchors introduced following the fifth word after \textit{English} in a text: \par\bgroup\index{anchor=<anchor>|exampleindex}\index{anchor=<anchor>|exampleindex}\index{anchor=<anchor>|exampleindex}\index{anchor=<anchor>|exampleindex}\index{linkGrp=<linkGrp>|exampleindex}\index{type=@type!<linkGrp>|exampleindex}\index{link=<link>|exampleindex}\index{type=@type!<link>|exampleindex}\index{target=@target!<link>|exampleindex}\exampleFont \begin{shaded}\noindent\mbox{}English language. Except for not very{<\textbf{anchor}\hspace*{1em}{xml:id}="{en1}"/>}\mbox{}\newline 
\textit{<!-- ... -->}\mbox{}\newline 
 English at all at the time\mbox{}\newline 
{<\textbf{anchor}\hspace*{1em}{xml:id}="{en2}"/>}\mbox{}\newline 
\textit{<!-- ... -->}\mbox{}\newline 
 English was still full of flaws\mbox{}\newline 
{<\textbf{anchor}\hspace*{1em}{xml:id}="{en3}"/>}\mbox{}\newline 
\textit{<!-- ... -->}\mbox{}\newline 
 English. This was revised by young\mbox{}\newline 
{<\textbf{anchor}\hspace*{1em}{xml:id}="{en4}"/>}\mbox{}\newline 
\textit{<!-- ... -->}\mbox{}\newline 
{<\textbf{linkGrp}\hspace*{1em}{type}="{five-word\textunderscore collocates}">}\mbox{}\newline 
\hspace*{1em}{<\textbf{link}\hspace*{1em}{type}="{collocates\textunderscore of\textunderscore ENGLISH}"\mbox{}\newline 
\hspace*{1em}\hspace*{1em}{target}="{\#en1 \#en2 \#en3 \#en4}"/>}\mbox{}\newline 
\textit{<!-- ... -->}\mbox{}\newline 
{</\textbf{linkGrp}>}\end{shaded}\egroup\par 
\subsubsection[{Alignment of Parallel Texts}]{Alignment of Parallel Texts}\label{SACSAL}\par
One very important application area for the alignment of parallel texts is multilingual corpora. Consider, for example, the need to align ‘translation pairs’ of sentences drawn from a corpus such as the Canadian Hansard, in which each sentence is given in both English and French. Concerning this problem, Gale and Church write: 
\begin{quote}Most English sentences match exactly one French sentence, but it is possible for an English sentence to match two or more French sentences. The first two English sentences [in the example below] illustrate a particularly hard case where two English sentences align to two French sentences. No smaller alignments are possible because the clause ‘...sales...were higher...’ in the first English sentence corresponds to (part of) the second French sentence. The next two alignments ... illustrate the more typical case where one English sentence aligns with exactly one French sentence. The final alignment matches two English sentences to a single French sentence. These alignments [which were produced by a computer program] agreed with the results produced by a human judge.\footnote{See \cite{SA-BIBL-1}, from which the example in the text is taken.}\end{quote}
\par
The alignment produced by Gale and Church's program can be expressed in four different ways. The encoder must first decide whether to represent the alignment in terms of points within each text (using the \hyperref[TEI.anchor]{<anchor>} element) or in terms of whole stretches of text, using the \hyperref[TEI.seg]{<seg>} element. To some extent the choice will depend on the process by which the software works out where alignment occurs, and the intention of the encoder. Secondly, the encoder may elect to represent the actual encoding using either {\itshape corresp} attributes attached to the individual \hyperref[TEI.anchor]{<anchor>} or \hyperref[TEI.seg]{<seg>} elements, or using a free-standing \hyperref[TEI.linkGrp]{<linkGrp>} element.\par
We present first a solution using \hyperref[TEI.anchor]{<anchor>} elements bearing only {\itshape corresp} attributes: \par\bgroup\index{div=<div>|exampleindex}\index{type=@type!<div>|exampleindex}\index{p=<p>|exampleindex}\index{anchor=<anchor>|exampleindex}\index{corresp=@corresp!<anchor>|exampleindex}\index{anchor=<anchor>|exampleindex}\index{corresp=@corresp!<anchor>|exampleindex}\index{anchor=<anchor>|exampleindex}\index{corresp=@corresp!<anchor>|exampleindex}\index{anchor=<anchor>|exampleindex}\index{corresp=@corresp!<anchor>|exampleindex}\index{div=<div>|exampleindex}\index{type=@type!<div>|exampleindex}\index{p=<p>|exampleindex}\index{anchor=<anchor>|exampleindex}\index{corresp=@corresp!<anchor>|exampleindex}\index{anchor=<anchor>|exampleindex}\index{corresp=@corresp!<anchor>|exampleindex}\index{anchor=<anchor>|exampleindex}\index{corresp=@corresp!<anchor>|exampleindex}\index{anchor=<anchor>|exampleindex}\index{corresp=@corresp!<anchor>|exampleindex}\exampleFont \begin{shaded}\noindent\mbox{}{<\textbf{div}\hspace*{1em}{xml:lang}="{en}"\hspace*{1em}{type}="{subsection}">}\mbox{}\newline 
\hspace*{1em}{<\textbf{p}>}\mbox{}\newline 
\hspace*{1em}\hspace*{1em}{<\textbf{anchor}\hspace*{1em}{corresp}="{\#fa1}"\hspace*{1em}{xml:id}="{ea1}"/>}According to our survey, 1988\mbox{}\newline 
\hspace*{1em}\hspace*{1em} sales of mineral water and soft drinks were much higher than in 1987,\mbox{}\newline 
\hspace*{1em}\hspace*{1em} reflecting the growing popularity of these products. Cola drink\mbox{}\newline 
\hspace*{1em}\hspace*{1em} manufacturers in particular achieved above-average growth rates.\mbox{}\newline 
\hspace*{1em}{<\textbf{anchor}\hspace*{1em}{corresp}="{\#fa2}"\hspace*{1em}{xml:id}="{ea2}"/>}The higher turnover was largely\mbox{}\newline 
\hspace*{1em}\hspace*{1em} due to an increase in the sales volume.\mbox{}\newline 
\hspace*{1em}{<\textbf{anchor}\hspace*{1em}{corresp}="{\#fa3}"\hspace*{1em}{xml:id}="{ea3}"/>}Employment and investment levels also climbed.\mbox{}\newline 
\hspace*{1em}{<\textbf{anchor}\hspace*{1em}{corresp}="{\#fa4}"\hspace*{1em}{xml:id}="{ea4}"/>}Following a two-year transitional period,\mbox{}\newline 
\hspace*{1em}\hspace*{1em} the new Foodstuffs Ordinance for Mineral Water came into effect on\mbox{}\newline 
\hspace*{1em}\hspace*{1em} April 1, 1988. Specifically, it contains more stringent requirements\mbox{}\newline 
\hspace*{1em}\hspace*{1em} regarding quality consistency and purity guarantees.{</\textbf{p}>}\mbox{}\newline 
{</\textbf{div}>}\mbox{}\newline 
{<\textbf{div}\hspace*{1em}{xml:lang}="{fr}"\hspace*{1em}{type}="{subsection}">}\mbox{}\newline 
\hspace*{1em}{<\textbf{p}>}\mbox{}\newline 
\hspace*{1em}\hspace*{1em}{<\textbf{anchor}\hspace*{1em}{corresp}="{\#ea1}"\hspace*{1em}{xml:id}="{fa1}"/>}Quant aux eaux minérales\mbox{}\newline 
\hspace*{1em}\hspace*{1em} et aux limonades, elles rencontrent toujours plus d'adeptes. En effet,\mbox{}\newline 
\hspace*{1em}\hspace*{1em} notre sondage fait ressortir des ventes nettement supérieures\mbox{}\newline 
\hspace*{1em}\hspace*{1em} à celles de 1987, pour les boissons à base de cola\mbox{}\newline 
\hspace*{1em}\hspace*{1em} notamment. {<\textbf{anchor}\hspace*{1em}{corresp}="{\#ea2}"\hspace*{1em}{xml:id}="{fa2}"/>}La progression des\mbox{}\newline 
\hspace*{1em}\hspace*{1em} chiffres d'affaires résulte en grande partie de l'accroissement\mbox{}\newline 
\hspace*{1em}\hspace*{1em} du volume des ventes. {<\textbf{anchor}\hspace*{1em}{corresp}="{\#ea3}"\hspace*{1em}{xml:id}="{fa3}"/>}L'emploi et\mbox{}\newline 
\hspace*{1em}\hspace*{1em} les investissements ont également augmenté.\mbox{}\newline 
\hspace*{1em}{<\textbf{anchor}\hspace*{1em}{corresp}="{\#ea4}"\hspace*{1em}{xml:id}="{fa4}"/>}La nouvelle ordonnance fédérale\mbox{}\newline 
\hspace*{1em}\hspace*{1em} sur les denrées alimentaires concernant entre autres les eaux\mbox{}\newline 
\hspace*{1em}\hspace*{1em} minérales, entrée en vigueur le 1er avril 1988 après\mbox{}\newline 
\hspace*{1em}\hspace*{1em} une période transitoire de deux ans, exige surtout une plus\mbox{}\newline 
\hspace*{1em}\hspace*{1em} grande constance dans la qualité et une garantie de la\mbox{}\newline 
\hspace*{1em}\hspace*{1em} pureté.{</\textbf{p}>}\mbox{}\newline 
{</\textbf{div}>}\end{shaded}\egroup\par \par
There is no requirement that the {\itshape corresp} attribute be specified in both English and French texts, since (as noted above) this attribute is defined as representing a mutual association. However, it may simplify processing to do so, and also avoids giving the impression that the English is translating the French, or vice versa. More seriously, this encoding does not make explicit that it is in fact the entire stretch of text between the anchors which is being aligned, not simply the points themselves. If for example one text contained material omitted from the other, this approach would not be appropriate.\par
We now present the same passage using the alternative \hyperref[TEI.linkGrp]{<linkGrp>} mechanism and marking explicitly the segments which have been aligned: \par\bgroup\index{div=<div>|exampleindex}\index{type=@type!<div>|exampleindex}\index{p=<p>|exampleindex}\index{seg=<seg>|exampleindex}\index{seg=<seg>|exampleindex}\index{seg=<seg>|exampleindex}\index{seg=<seg>|exampleindex}\index{div=<div>|exampleindex}\index{type=@type!<div>|exampleindex}\index{p=<p>|exampleindex}\index{seg=<seg>|exampleindex}\index{seg=<seg>|exampleindex}\index{seg=<seg>|exampleindex}\index{seg=<seg>|exampleindex}\index{linkGrp=<linkGrp>|exampleindex}\index{type=@type!<linkGrp>|exampleindex}\index{domains=@domains!<linkGrp>|exampleindex}\index{link=<link>|exampleindex}\index{target=@target!<link>|exampleindex}\index{link=<link>|exampleindex}\index{target=@target!<link>|exampleindex}\index{link=<link>|exampleindex}\index{target=@target!<link>|exampleindex}\index{link=<link>|exampleindex}\index{target=@target!<link>|exampleindex}\exampleFont \begin{shaded}\noindent\mbox{}{<\textbf{div}\hspace*{1em}{xml:id}="{div-e}"\hspace*{1em}{xml:lang}="{en}"\mbox{}\newline 
\hspace*{1em}{type}="{subsection}">}\mbox{}\newline 
\hspace*{1em}{<\textbf{p}>}\mbox{}\newline 
\hspace*{1em}\hspace*{1em}{<\textbf{seg}\hspace*{1em}{xml:id}="{e\textunderscore 1}">}According to our survey, 1988 sales of mineral\mbox{}\newline 
\hspace*{1em}\hspace*{1em}\hspace*{1em}\hspace*{1em} water and soft drinks were much higher than in 1987,\mbox{}\newline 
\hspace*{1em}\hspace*{1em}\hspace*{1em}\hspace*{1em} reflecting the growing popularity of these products. Cola\mbox{}\newline 
\hspace*{1em}\hspace*{1em}\hspace*{1em}\hspace*{1em} drink manufacturers in particular achieved above-average\mbox{}\newline 
\hspace*{1em}\hspace*{1em}\hspace*{1em}\hspace*{1em} growth rates.{</\textbf{seg}>}\mbox{}\newline 
\hspace*{1em}\hspace*{1em}{<\textbf{seg}\hspace*{1em}{xml:id}="{e\textunderscore 2}">}The higher turnover was largely due to an\mbox{}\newline 
\hspace*{1em}\hspace*{1em}\hspace*{1em}\hspace*{1em} increase in the sales volume.{</\textbf{seg}>}\mbox{}\newline 
\hspace*{1em}\hspace*{1em}{<\textbf{seg}\hspace*{1em}{xml:id}="{e\textunderscore 3}">}Employment and investment levels also climbed.{</\textbf{seg}>}\mbox{}\newline 
\hspace*{1em}\hspace*{1em}{<\textbf{seg}\hspace*{1em}{xml:id}="{e\textunderscore 4}">}Following a two-year transitional period, the new\mbox{}\newline 
\hspace*{1em}\hspace*{1em}\hspace*{1em}\hspace*{1em} Foodstuffs Ordinance for Mineral Water came into effect on\mbox{}\newline 
\hspace*{1em}\hspace*{1em}\hspace*{1em}\hspace*{1em} April 1, 1988. Specifically, it contains more stringent\mbox{}\newline 
\hspace*{1em}\hspace*{1em}\hspace*{1em}\hspace*{1em} requirements regarding quality consistency and purity\mbox{}\newline 
\hspace*{1em}\hspace*{1em}\hspace*{1em}\hspace*{1em} guarantees.{</\textbf{seg}>}\mbox{}\newline 
\hspace*{1em}{</\textbf{p}>}\mbox{}\newline 
{</\textbf{div}>}\mbox{}\newline 
{<\textbf{div}\hspace*{1em}{xml:id}="{div-f}"\hspace*{1em}{xml:lang}="{fr}"\mbox{}\newline 
\hspace*{1em}{type}="{subsection}">}\mbox{}\newline 
\hspace*{1em}{<\textbf{p}>}\mbox{}\newline 
\hspace*{1em}\hspace*{1em}{<\textbf{seg}\hspace*{1em}{xml:id}="{f\textunderscore 1}">}Quant aux eaux minérales et aux limonades,\mbox{}\newline 
\hspace*{1em}\hspace*{1em}\hspace*{1em}\hspace*{1em} elles rencontrent toujours plus d'adeptes. En effet, notre\mbox{}\newline 
\hspace*{1em}\hspace*{1em}\hspace*{1em}\hspace*{1em} sondage fait ressortir des ventes nettement\mbox{}\newline 
\hspace*{1em}\hspace*{1em}\hspace*{1em}\hspace*{1em} supérieures à celles de 1987, pour les\mbox{}\newline 
\hspace*{1em}\hspace*{1em}\hspace*{1em}\hspace*{1em} boissons à base de cola notamment.{</\textbf{seg}>}\mbox{}\newline 
\hspace*{1em}\hspace*{1em}{<\textbf{seg}\hspace*{1em}{xml:id}="{f\textunderscore 2}">}La progression des chiffres d'affaires\mbox{}\newline 
\hspace*{1em}\hspace*{1em}\hspace*{1em}\hspace*{1em} résulte en grande partie de l'accroissement du volume\mbox{}\newline 
\hspace*{1em}\hspace*{1em}\hspace*{1em}\hspace*{1em} des ventes.{</\textbf{seg}>}\mbox{}\newline 
\hspace*{1em}\hspace*{1em}{<\textbf{seg}\hspace*{1em}{xml:id}="{f\textunderscore 3}">}L'emploi et les investissements ont\mbox{}\newline 
\hspace*{1em}\hspace*{1em}\hspace*{1em}\hspace*{1em} également augmenté.{</\textbf{seg}>}\mbox{}\newline 
\hspace*{1em}\hspace*{1em}{<\textbf{seg}\hspace*{1em}{xml:id}="{f\textunderscore 4}">}La nouvelle ordonnance fédérale sur\mbox{}\newline 
\hspace*{1em}\hspace*{1em}\hspace*{1em}\hspace*{1em} les denrées alimentaires concernant entre autres les\mbox{}\newline 
\hspace*{1em}\hspace*{1em}\hspace*{1em}\hspace*{1em} eaux minérales, entrée en vigueur le 1er avril\mbox{}\newline 
\hspace*{1em}\hspace*{1em}\hspace*{1em}\hspace*{1em} 1988 après une période transitoire de deux\mbox{}\newline 
\hspace*{1em}\hspace*{1em}\hspace*{1em}\hspace*{1em} ans, exige surtout une plus grande constance dans la\mbox{}\newline 
\hspace*{1em}\hspace*{1em}\hspace*{1em}\hspace*{1em} qualité et une garantie de la pureté.{</\textbf{seg}>}\mbox{}\newline 
\hspace*{1em}{</\textbf{p}>}\mbox{}\newline 
{</\textbf{div}>}\mbox{}\newline 
{<\textbf{linkGrp}\hspace*{1em}{type}="{alignment}"\mbox{}\newline 
\hspace*{1em}{domains}="{\#div-e \#div-f}">}\mbox{}\newline 
\hspace*{1em}{<\textbf{link}\hspace*{1em}{target}="{\#e\textunderscore 1 \#f\textunderscore 1}"/>}\mbox{}\newline 
\hspace*{1em}{<\textbf{link}\hspace*{1em}{target}="{\#e\textunderscore 2 \#f\textunderscore 2}"/>}\mbox{}\newline 
\hspace*{1em}{<\textbf{link}\hspace*{1em}{target}="{\#e\textunderscore 3 \#f\textunderscore 3}"/>}\mbox{}\newline 
\hspace*{1em}{<\textbf{link}\hspace*{1em}{target}="{\#e\textunderscore 4 \#f\textunderscore 4}"/>}\mbox{}\newline 
{</\textbf{linkGrp}>}\end{shaded}\egroup\par \par
Note that use of the \hyperref[TEI.ab]{<ab>} element allows us to mark up the orthographic sentences in both languages independently of the alignment: the first translation pair in this example might be marked up as follows: \par\bgroup\index{div=<div>|exampleindex}\index{type=@type!<div>|exampleindex}\index{ab=<ab>|exampleindex}\index{s=<s>|exampleindex}\index{s=<s>|exampleindex}\index{div=<div>|exampleindex}\index{type=@type!<div>|exampleindex}\index{ab=<ab>|exampleindex}\index{s=<s>|exampleindex}\index{s=<s>|exampleindex}\exampleFont \begin{shaded}\noindent\mbox{}{<\textbf{div}\hspace*{1em}{xml:id}="{english}"\hspace*{1em}{xml:lang}="{en}"\mbox{}\newline 
\hspace*{1em}{type}="{subsection}">}\mbox{}\newline 
\hspace*{1em}{<\textbf{ab}\hspace*{1em}{xml:id}="{english1}">}\mbox{}\newline 
\hspace*{1em}\hspace*{1em}{<\textbf{s}>}According to our survey, 1988 sales of mineral water and soft\mbox{}\newline 
\hspace*{1em}\hspace*{1em}\hspace*{1em}\hspace*{1em} drinks were much higher than in 1987, reflecting the growing popularity\mbox{}\newline 
\hspace*{1em}\hspace*{1em}\hspace*{1em}\hspace*{1em} of these products.{</\textbf{s}>}\mbox{}\newline 
\hspace*{1em}\hspace*{1em}{<\textbf{s}>}Cola drink manufacturers in particular achieved above-average\mbox{}\newline 
\hspace*{1em}\hspace*{1em}\hspace*{1em}\hspace*{1em} growth rates.{</\textbf{s}>}\mbox{}\newline 
\hspace*{1em}{</\textbf{ab}>}\mbox{}\newline 
{</\textbf{div}>}\mbox{}\newline 
{<\textbf{div}\hspace*{1em}{xml:id}="{french}"\hspace*{1em}{xml:lang}="{fr}"\mbox{}\newline 
\hspace*{1em}{type}="{subsection}">}\mbox{}\newline 
\hspace*{1em}{<\textbf{ab}\hspace*{1em}{xml:id}="{french1}">}\mbox{}\newline 
\hspace*{1em}\hspace*{1em}{<\textbf{s}\hspace*{1em}{xml:id}="{fs1}">}Quant aux eaux minérales et aux limonades, elles\mbox{}\newline 
\hspace*{1em}\hspace*{1em}\hspace*{1em}\hspace*{1em} rencontrent toujours plus d'adeptes.{</\textbf{s}>}\mbox{}\newline 
\hspace*{1em}\hspace*{1em}{<\textbf{s}\hspace*{1em}{xml:id}="{fs2}">}En effet, notre sondage fait ressortir des ventes nettement\mbox{}\newline 
\hspace*{1em}\hspace*{1em}\hspace*{1em}\hspace*{1em} supérieures à celles de 1987, pour les boissons à\mbox{}\newline 
\hspace*{1em}\hspace*{1em}\hspace*{1em}\hspace*{1em} base de cola notamment.{</\textbf{s}>}\mbox{}\newline 
\hspace*{1em}{</\textbf{ab}>}\mbox{}\newline 
{</\textbf{div}>}\end{shaded}\egroup\par 
\subsubsection[{A Three-way Alignment}]{A Three-way Alignment}\label{SACSXA}\par
The preceding encoding of the alignment of parallel passages from two texts requires that those texts and the alignment all be part of the same document. If the texts are in separate documents, then complete URIs, whether absolute or relative (section \textit{\hyperref[SA]{16.\ Linking, Segmentation, and Alignment}}), will be required. These external pointers may appear anywhere within the document, but if they are created solely for use in encoding links, they may for convenience be grouped within the \hyperref[TEI.linkGrp]{<linkGrp>} (or other grouping element that uses them for linking).\par
To demonstrate this facility, we consider how we might encode the alignments in an extract from Comenius' \textit{Orbis Sensualium Pictus}, in the English translation of Charles Hoole (1659).  \begin{figure}[htbp]
\noindent\noindent\includegraphics[]{Images/compic.png}\end{figure}
 Each topic covered in this work has three parts: a picture, a prose text in Latin describing the topic, and a carefully-aligned translation of the Latin into English, German, or some other vernacular. Key terms in the two texts are typographically distinct, and are linked to the picture by numbers, which appear in the two texts and within the picture as well.\par
First, we consider the text portions. The English and Latin portions have been encoded as distinct \hyperref[TEI.div]{<div>} elements. Identifiers have been attached to each typographic line, but no other encoding added, to simplify the example.  \par\bgroup\index{div=<div>|exampleindex}\index{type=@type!<div>|exampleindex}\index{head=<head>|exampleindex}\index{p=<p>|exampleindex}\index{seg=<seg>|exampleindex}\index{seg=<seg>|exampleindex}\index{seg=<seg>|exampleindex}\index{seg=<seg>|exampleindex}\index{seg=<seg>|exampleindex}\index{seg=<seg>|exampleindex}\index{seg=<seg>|exampleindex}\index{seg=<seg>|exampleindex}\index{div=<div>|exampleindex}\index{type=@type!<div>|exampleindex}\index{head=<head>|exampleindex}\index{p=<p>|exampleindex}\index{seg=<seg>|exampleindex}\index{seg=<seg>|exampleindex}\index{seg=<seg>|exampleindex}\index{seg=<seg>|exampleindex}\index{seg=<seg>|exampleindex}\index{seg=<seg>|exampleindex}\index{seg=<seg>|exampleindex}\index{seg=<seg>|exampleindex}\exampleFont \begin{shaded}\noindent\mbox{}{<\textbf{div}\hspace*{1em}{xml:id}="{e98}"\hspace*{1em}{xml:lang}="{en}"\mbox{}\newline 
\hspace*{1em}{type}="{lesson}">}\mbox{}\newline 
\hspace*{1em}{<\textbf{head}>}The Study{</\textbf{head}>}\mbox{}\newline 
\hspace*{1em}{<\textbf{p}>}\mbox{}\newline 
\hspace*{1em}\hspace*{1em}{<\textbf{seg}\hspace*{1em}{xml:id}="{e9801}">}The Study{</\textbf{seg}>}\mbox{}\newline 
\hspace*{1em}\hspace*{1em}{<\textbf{seg}\hspace*{1em}{xml:id}="{e9802}">}is a place{</\textbf{seg}>}\mbox{}\newline 
\hspace*{1em}\hspace*{1em}{<\textbf{seg}\hspace*{1em}{xml:id}="{e9803}">}where a Student,{</\textbf{seg}>}\mbox{}\newline 
\hspace*{1em}\hspace*{1em}{<\textbf{seg}\hspace*{1em}{xml:id}="{e9804}">}a part from men,{</\textbf{seg}>}\mbox{}\newline 
\hspace*{1em}\hspace*{1em}{<\textbf{seg}\hspace*{1em}{xml:id}="{e9805}">}sitteth alone,{</\textbf{seg}>}\mbox{}\newline 
\hspace*{1em}\hspace*{1em}{<\textbf{seg}\hspace*{1em}{xml:id}="{e9806}">}addicted to his Studies,{</\textbf{seg}>}\mbox{}\newline 
\hspace*{1em}\hspace*{1em}{<\textbf{seg}\hspace*{1em}{xml:id}="{e9807}">}whilst he readeth{</\textbf{seg}>}\mbox{}\newline 
\hspace*{1em}\hspace*{1em}{<\textbf{seg}\hspace*{1em}{xml:id}="{e9808}">}Books,{</\textbf{seg}>}\mbox{}\newline 
\hspace*{1em}{</\textbf{p}>}\mbox{}\newline 
{</\textbf{div}>}\mbox{}\newline 
{<\textbf{div}\hspace*{1em}{xml:id}="{l98}"\hspace*{1em}{xml:lang}="{la}"\mbox{}\newline 
\hspace*{1em}{type}="{lesson}">}\mbox{}\newline 
\hspace*{1em}{<\textbf{head}>}Muséum{</\textbf{head}>}\mbox{}\newline 
\hspace*{1em}{<\textbf{p}>}\mbox{}\newline 
\hspace*{1em}\hspace*{1em}{<\textbf{seg}\hspace*{1em}{xml:id}="{l9801}">}Museum{</\textbf{seg}>}\mbox{}\newline 
\hspace*{1em}\hspace*{1em}{<\textbf{seg}\hspace*{1em}{xml:id}="{l9802}">}est locus{</\textbf{seg}>}\mbox{}\newline 
\hspace*{1em}\hspace*{1em}{<\textbf{seg}\hspace*{1em}{xml:id}="{l9803}">}ubi Studiosus,{</\textbf{seg}>}\mbox{}\newline 
\hspace*{1em}\hspace*{1em}{<\textbf{seg}\hspace*{1em}{xml:id}="{l9804}">}secretus ab hominibus,{</\textbf{seg}>}\mbox{}\newline 
\hspace*{1em}\hspace*{1em}{<\textbf{seg}\hspace*{1em}{xml:id}="{l9805}">}solus sedet,{</\textbf{seg}>}\mbox{}\newline 
\hspace*{1em}\hspace*{1em}{<\textbf{seg}\hspace*{1em}{xml:id}="{l9806}">}Studiis deditus,{</\textbf{seg}>}\mbox{}\newline 
\hspace*{1em}\hspace*{1em}{<\textbf{seg}\hspace*{1em}{xml:id}="{l9807}">}dum lectitat{</\textbf{seg}>}\mbox{}\newline 
\hspace*{1em}\hspace*{1em}{<\textbf{seg}\hspace*{1em}{xml:id}="{l9808}">}Libros,{</\textbf{seg}>}\mbox{}\newline 
\hspace*{1em}{</\textbf{p}>}\mbox{}\newline 
{</\textbf{div}>}\end{shaded}\egroup\par \par
Next we consider the non-textual parts of the page. Encoding this requires providing two distinct components: firstly a digitized rendering of the page itself, and secondly a representation of the areas within that image which are to be aligned. In section \textit{\hyperref[PHFAX]{11.1.\ Digital Facsimiles}} we present a simple way of doing this using the TEI-defined markup for alignment of facsimiles. In the present chapter we demonstrate a more powerful means of aligning arbitrary polygons and points, which uses the XML notation SVG (see \hyperref[SVG-11]{SVG}). This provides appropriate facilities for both these requirements: \par\bgroup\exampleFont \begin{shaded}\noindent\mbox{}{<\textbf{svg} xmlns="http://www.w3.org/2000/svg"\mbox{}\newline 
   xmlns:xlink="http://www.w3.org/1999/xlink">}\mbox{}\newline 
\hspace*{1em}{<\textbf{image}\hspace*{1em}{xlink:href}="{p1764.png}"\hspace*{1em}{width}="{597}"\mbox{}\newline 
\hspace*{1em}\hspace*{1em}{height}="{897}"\hspace*{1em}{id}="{p981}"/>}\mbox{}\newline 
\hspace*{1em}{<\textbf{rect}\hspace*{1em}{id}="{p982}"\hspace*{1em}{x}="{75}"\hspace*{1em}{y}="{75}"\hspace*{1em}{width}="{25}"\mbox{}\newline 
\hspace*{1em}\hspace*{1em}{height}="{10}"/>}\mbox{}\newline 
\hspace*{1em}{<\textbf{rect}\hspace*{1em}{id}="{p983}"\hspace*{1em}{x}="{55}"\hspace*{1em}{y}="{42}"\hspace*{1em}{width}="{25}"\mbox{}\newline 
\hspace*{1em}\hspace*{1em}{height}="{10}"/>}\mbox{}\newline 
{</\textbf{svg}>}\end{shaded}\egroup\par \noindent  This example of SVG defines two rectangles at the locations with the specified x and y coordinates. A view is defined on these, enabling them to be mapped by an SVG processor to the image found at the URL specified (\textsf{p1764.png}). It also defines unique identifiers for the whole image, and the two views of it, which we will use within our alignment, as shown next (for further discussion of the handling of images and graphics, see section \textit{\hyperref[FTGRA]{14.4.\ Specific Elements for Graphic Images}}; for further discussion of using non-TEI XML vocabularies such as SVG within a TEI document, see section \textit{\hyperref[ST-aliens]{22.8.2.\ Combining TEI and Non-TEI Modules}}).\par
As printed, the Comenius text exhibits three kinds of alignment. \begin{enumerate}
\item[1] The English and Latin portions are printed in two parallel columns, with corresponding phrases, (represented above by \hyperref[TEI.seg]{<seg>} elements), more or less next to each other.
\item[2] Particular words or phrases are marked as terms in the two languages by a change of rendition: the English text, which otherwise uses black letter type throughout, has the words \textit{The Study}, \textit{a Student}, \textit{Studies}, and \textit{Books} in a roman font; in the Latin text, which is printed in roman, the corresponding words (\textit{Museum}, \textit{Studiosus}, \textit{Studiis}, and \textit{Libros}) are all in italic.
\item[3] Numbered labels appear within the text portions, linking keywords to each other and to sections of the picture. These labels, which have been left out of the above encoding, are attached to the first, third, and last segments in each language quoted below, and also appear (rather indistinctly) within the picture itself. Thus, the images of the study, the student, and his books are each aligned with the correct term for them in the two languages.
\end{enumerate}\par
The first kind of alignment might be represented by using the {\itshape corresp} attribute on the \hyperref[TEI.seg]{<seg>} element. The second kind might be represented by using the \hyperref[TEI.gloss]{<gloss>} and \hyperref[TEI.term]{<term>} mechanism described in section \textit{\hyperref[COHTG]{3.4.1.\ Terms and Glosses}}. The third kind of alignment might be represented using pointers embedded within the texts, for example: \par\bgroup\index{seg=<seg>|exampleindex}\index{ref=<ref>|exampleindex}\index{n=@n!<ref>|exampleindex}\index{target=@target!<ref>|exampleindex}\index{seg=<seg>|exampleindex}\index{ref=<ref>|exampleindex}\index{n=@n!<ref>|exampleindex}\index{target=@target!<ref>|exampleindex}\exampleFont \begin{shaded}\noindent\mbox{}\mbox{}\newline 
\textit{<!--... -->}{<\textbf{seg}\hspace*{1em}{xml:id}="{xe9803}">}where a {<\textbf{ref}\hspace*{1em}{n}="{2}"\hspace*{1em}{target}="{\#xp982}">}Student{</\textbf{ref}>},{</\textbf{seg}>}\mbox{}\newline 
{<\textbf{seg}\hspace*{1em}{xml:id}="{xl9803}">}ubi {<\textbf{ref}\hspace*{1em}{n}="{2}"\hspace*{1em}{target}="{\#xp982}">}Studiosus{</\textbf{ref}>},{</\textbf{seg}>}\mbox{}\newline 
\textit{<!--... -->}\end{shaded}\egroup\par \noindent  We choose however to use the \hyperref[TEI.link]{<link>} element, since this provides a more efficient way of representing the three-way alignment between English, Latin, and picture without redundancy. \par\bgroup\index{linkGrp=<linkGrp>|exampleindex}\index{type=@type!<linkGrp>|exampleindex}\index{link=<link>|exampleindex}\index{target=@target!<link>|exampleindex}\index{link=<link>|exampleindex}\index{target=@target!<link>|exampleindex}\index{link=<link>|exampleindex}\index{target=@target!<link>|exampleindex}\index{link=<link>|exampleindex}\index{target=@target!<link>|exampleindex}\index{link=<link>|exampleindex}\index{target=@target!<link>|exampleindex}\index{link=<link>|exampleindex}\index{target=@target!<link>|exampleindex}\index{link=<link>|exampleindex}\index{target=@target!<link>|exampleindex}\index{link=<link>|exampleindex}\index{target=@target!<link>|exampleindex}\exampleFont \begin{shaded}\noindent\mbox{}{<\textbf{linkGrp}\hspace*{1em}{type}="{alignment}">}\mbox{}\newline 
\hspace*{1em}{<\textbf{link}\hspace*{1em}{target}="{\#xe9801 \#xl9801 \#xp981}"/>}\mbox{}\newline 
\hspace*{1em}{<\textbf{link}\hspace*{1em}{target}="{\#xe9802 \#xl9802}"/>}\mbox{}\newline 
\hspace*{1em}{<\textbf{link}\hspace*{1em}{target}="{\#xe9803 \#xl9803 \#xp982}"/>}\mbox{}\newline 
\hspace*{1em}{<\textbf{link}\hspace*{1em}{target}="{\#xe9804 \#xl9804}"/>}\mbox{}\newline 
\hspace*{1em}{<\textbf{link}\hspace*{1em}{target}="{\#xe9805 \#xl9805}"/>}\mbox{}\newline 
\hspace*{1em}{<\textbf{link}\hspace*{1em}{target}="{\#xe9806 \#xl9806}"/>}\mbox{}\newline 
\hspace*{1em}{<\textbf{link}\hspace*{1em}{target}="{\#xe9807 \#xl9807}"/>}\mbox{}\newline 
\hspace*{1em}{<\textbf{link}\hspace*{1em}{target}="{\#xe9808 \#xl9808 \#xp983}"/>}\mbox{}\newline 
{</\textbf{linkGrp}>}\end{shaded}\egroup\par \par
This map, of course, only aligns whole segments and image portions, since these are the only parts of our encoding which bear identifiers and can therefore be pointed to. To add to it the alignment between the typographically distinct words mentioned above, new elements must be defined, either within the text itself or externally by using stand off techniques. Encoding these word pairs as \hyperref[TEI.term]{<term>} and \hyperref[TEI.gloss]{<gloss>}, although intuitively obvious, requires a non-trivial decision as to whether the Latin text is glossing the English, or vice versa. Tagging all the marked words as \hyperref[TEI.term]{<term>} avoids the difficult decision, but might be thought by some encoders to convey the wrong information about the words in question. Simply tagging them as additional embedded \hyperref[TEI.seg]{<seg>} elements with identifiers that can be aligned like the others is also a possibility.\par
These solutions all require the addition of further markup to the text. This may pose no problems, or it may be infeasible, for example because the text is held on a read-only medium. If it is not feasible to add more markup to the original text, some form of stand-off markup will be needed. Any item within the text that can be pointed to using the various pointer schemes discussed in this chapter may be used, not simply those which rely on the existence of an {\itshape xml:id} attribute. Suppose our example had been more lightly tagged, as follows: \par\bgroup\index{div=<div>|exampleindex}\index{type=@type!<div>|exampleindex}\index{head=<head>|exampleindex}\index{ab=<ab>|exampleindex}\index{ab=<ab>|exampleindex}\index{ab=<ab>|exampleindex}\index{div=<div>|exampleindex}\index{type=@type!<div>|exampleindex}\index{head=<head>|exampleindex}\index{ab=<ab>|exampleindex}\index{ab=<ab>|exampleindex}\index{ab=<ab>|exampleindex}\exampleFont \begin{shaded}\noindent\mbox{}{<\textbf{div}\hspace*{1em}{xml:id}="{E98}"\hspace*{1em}{xml:lang}="{en}"\mbox{}\newline 
\hspace*{1em}{type}="{lesson}">}\mbox{}\newline 
\hspace*{1em}{<\textbf{head}>}The Study{</\textbf{head}>}\mbox{}\newline 
\hspace*{1em}{<\textbf{ab}>}The Study{</\textbf{ab}>}\mbox{}\newline 
\hspace*{1em}{<\textbf{ab}>}is a place{</\textbf{ab}>}\mbox{}\newline 
\hspace*{1em}{<\textbf{ab}>}where a Student,{</\textbf{ab}>}\mbox{}\newline 
{</\textbf{div}>}\mbox{}\newline 
{<\textbf{div}\hspace*{1em}{xml:id}="{L98}"\hspace*{1em}{xml:lang}="{la}"\mbox{}\newline 
\hspace*{1em}{type}="{lesson}">}\mbox{}\newline 
\hspace*{1em}{<\textbf{head}>}Muséum{</\textbf{head}>}\mbox{}\newline 
\hspace*{1em}{<\textbf{ab}>}Museum{</\textbf{ab}>}\mbox{}\newline 
\hspace*{1em}{<\textbf{ab}>}est locus{</\textbf{ab}>}\mbox{}\newline 
\hspace*{1em}{<\textbf{ab}>}ubi Studiosus,{</\textbf{ab}>}\mbox{}\newline 
{</\textbf{div}>}\end{shaded}\egroup\par \par
To express the same alignment mentioned above, we could use an XPath expression to identify the required \hyperref[TEI.ab]{<ab>} elements: \par\bgroup\index{linkGrp=<linkGrp>|exampleindex}\index{type=@type!<linkGrp>|exampleindex}\index{link=<link>|exampleindex}\index{target=@target!<link>|exampleindex}\index{link=<link>|exampleindex}\index{target=@target!<link>|exampleindex}\exampleFont \begin{shaded}\noindent\mbox{}{<\textbf{linkGrp}\hspace*{1em}{type}="{alignment}">}\mbox{}\newline 
\hspace*{1em}{<\textbf{link}\hspace*{1em}{target}="{\#xpath(//div[@xml:id='L98']/ab[1]) \#xpath(//div[@xml:id='E98']/ab[1])}"/>}\mbox{}\newline 
\hspace*{1em}{<\textbf{link}\hspace*{1em}{target}="{\#xpath(//div[@xml:id='L98']/ab[2]) \#xpath(//div[@xml:id='E98']/ab[2])}"/>}\mbox{}\newline 
{</\textbf{linkGrp}>}\end{shaded}\egroup\par \noindent  In the absence of any markup around individual substrings of the element content, the string-range pointer scheme discussed in \textit{\hyperref[SATSSR]{16.2.4.7.\ string-range()}} may also be helpful: for example, to indicate that the words \textit{Studies} and \textit{Studiis} correspond, we might express the link between them as follows: \par\bgroup\index{link=<link>|exampleindex}\index{target=@target!<link>|exampleindex}\exampleFont \begin{shaded}\noindent\mbox{}{<\textbf{link}\hspace*{1em}{target}="{\#string-range(e9806,16,7) \#string-range(l9806,0,7)}"/>}\end{shaded}\egroup\par 
\subsection[{Identical Elements and Virtual Copies}]{Identical Elements and Virtual Copies}\label{SAIE}\par
This section introduces the notion of a \textit{virtual element}, that is, an element which is not explicitly present in a text, but the presence of which an application can infer from the encoding supplied. In this section, we are concerned with virtual elements made by simply cloning existing elements. In the next section (\textit{\hyperref[SAAG]{16.7.\ Aggregation}}), we discuss virtual elements made by aggregating existing elements.\par
Provided that explicit elements are available to represent the parts or places to be linked, then the global linking attributes {\itshape sameAs} and {\itshape copyOf} may be used to encode this kind of equivalence: 
\begin{sansreflist}
  
\item [\textbf{att.global.linking}] provides a set of attributes for hypertextual linking.\hfil\\[-10pt]\begin{sansreflist}
    \item[@{\itshape sameAs}]
  points to an element that is the same as the current element.
    \item[@{\itshape copyOf}]
  points to an element of which the current element is a copy.
\end{sansreflist}  
\end{sansreflist}
\par
It is useful to be able to represent the fact that one element of text is identical to others, for analytical purposes, or (especially if the elements have lengthy content) to obviate the need to repeat the content. For example, consider the repetition of the \hyperref[TEI.date]{<date>} element in the following material: \par\bgroup\index{p=<p>|exampleindex}\index{q=<q>|exampleindex}\index{rend=@rend!<q>|exampleindex}\index{date=<date>|exampleindex}\index{p=<p>|exampleindex}\index{p=<p>|exampleindex}\index{q=<q>|exampleindex}\index{rend=@rend!<q>|exampleindex}\index{date=<date>|exampleindex}\exampleFont \begin{shaded}\noindent\mbox{}{<\textbf{p}>}In small clumsy letters he wrote:\mbox{}\newline 
{<\textbf{q}\hspace*{1em}{rend}="{centered italic}">}\mbox{}\newline 
\hspace*{1em}\hspace*{1em}{<\textbf{date}\hspace*{1em}{xml:id}="{d840404}">}April 4th,\mbox{}\newline 
\hspace*{1em}\hspace*{1em}\hspace*{1em}\hspace*{1em} 1984{</\textbf{date}>}.{</\textbf{q}>}\mbox{}\newline 
{</\textbf{p}>}\mbox{}\newline 
{<\textbf{p}>}He sat back. A sense of complete helplessness had\mbox{}\newline 
 descended upon him. ...{</\textbf{p}>}\mbox{}\newline 
{<\textbf{p}>}His small but childish handwriting straggled up\mbox{}\newline 
 and down the page, shedding first its capital letters\mbox{}\newline 
 and finally even its full stops:\mbox{}\newline 
{<\textbf{q}\hspace*{1em}{rend}="{italic}">}\mbox{}\newline 
\hspace*{1em}\hspace*{1em}{<\textbf{date}>}April 4th, 1984{</\textbf{date}>}.\mbox{}\newline 
\hspace*{1em}\hspace*{1em} Last night to the flicks. ... {</\textbf{q}>}\mbox{}\newline 
{</\textbf{p}>}\end{shaded}\egroup\par \noindent  Suppose now that we wish to encode the fact that the second \hyperref[TEI.date]{<date>} element above has identical content to the first. The {\itshape sameAs} attribute is provided for this purpose. Using it, we could recode the last line of the above example as follows: \par\bgroup\index{date=<date>|exampleindex}\index{sameAs=@sameAs!<date>|exampleindex}\exampleFont \begin{shaded}\noindent\mbox{}{<\textbf{date}\hspace*{1em}{sameAs}="{\#d840404}">}April 4th,\mbox{}\newline 
 1984{</\textbf{date}>}\mbox{}\newline 
 Last night to the flicks ... \end{shaded}\egroup\par \par
The {\itshape sameAs} attribute may be used to document the fact that two elements have identical content. It may be regarded as a special kind of link. It should only be attached to an element with identical content to that which it targets, or to one the content of which clearly designates it as a repetition, such as the word \textit{repeat} or \textit{bis} in the representation of the chorus of a song, the second time it is to be sung. The relation specified by the {\itshape sameAs} attribute is symmetric: if a chorus is repeated three times and each repetition bears a {\itshape sameAs} attribute indicating the first occurrence of the element concerned, it is implied that each chorus is identical, and there is no need for the first occurrence to specify any of its copies.\par
The {\itshape copyOf} attribute is used in a similar way to indicate that the content of the element bearing it is identical to that of another. The difference is that the content is not itself repeated. The effect of this attribute is thus to create a \textit{virtual copy} of the element indicated. Using this attribute, the repeated date in the first example above could be recoded as follows: \par\bgroup\index{date=<date>|exampleindex}\index{rend=@rend!<date>|exampleindex}\index{copyOf=@copyOf!<date>|exampleindex}\exampleFont \begin{shaded}\noindent\mbox{}{<\textbf{date}\hspace*{1em}{rend}="{italic}"\hspace*{1em}{copyOf}="{\#d840404}"/>}\end{shaded}\egroup\par \par
An application program should replace whatever is the actual content of an element bearing a {\itshape copyOf} attribute with the content of the element specified by it. If the content of the element specified includes other elements, these will become embedded within the element bearing the attribute. Care must be taken to ensure that the document is valid both before and after this embedding takes place. If, for example, the element bearing a {\itshape copyOf} attribute requires a mandatory sub-component, then this component must be present (though possibly empty), even though it will be replaced by the content of the targetted element.\par
The following example demonstrates how the {\itshape copyOf} attribute may be used in conjunction with the \hyperref[TEI.seg]{<seg>} element to highlight the differences between almost identical repetitions: \par\bgroup\index{sp=<sp>|exampleindex}\index{speaker=<speaker>|exampleindex}\index{l=<l>|exampleindex}\index{seg=<seg>|exampleindex}\index{l=<l>|exampleindex}\index{seg=<seg>|exampleindex}\index{l=<l>|exampleindex}\index{seg=<seg>|exampleindex}\index{l=<l>|exampleindex}\index{seg=<seg>|exampleindex}\index{copyOf=@copyOf!<seg>|exampleindex}\index{l=<l>|exampleindex}\index{l=<l>|exampleindex}\index{l=<l>|exampleindex}\index{seg=<seg>|exampleindex}\index{l=<l>|exampleindex}\index{seg=<seg>|exampleindex}\index{copyOf=@copyOf!<seg>|exampleindex}\index{sp=<sp>|exampleindex}\index{speaker=<speaker>|exampleindex}\index{l=<l>|exampleindex}\index{seg=<seg>|exampleindex}\index{copyOf=@copyOf!<seg>|exampleindex}\index{l=<l>|exampleindex}\index{seg=<seg>|exampleindex}\index{copyOf=@copyOf!<seg>|exampleindex}\index{l=<l>|exampleindex}\index{copyOf=@copyOf!<l>|exampleindex}\index{l=<l>|exampleindex}\index{copyOf=@copyOf!<l>|exampleindex}\index{l=<l>|exampleindex}\index{copyOf=@copyOf!<l>|exampleindex}\index{l=<l>|exampleindex}\index{copyOf=@copyOf!<l>|exampleindex}\index{l=<l>|exampleindex}\index{copyOf=@copyOf!<l>|exampleindex}\index{l=<l>|exampleindex}\index{copyOf=@copyOf!<l>|exampleindex}\exampleFont \begin{shaded}\noindent\mbox{}{<\textbf{sp}>}\mbox{}\newline 
\hspace*{1em}{<\textbf{speaker}>}Mikado{</\textbf{speaker}>}\mbox{}\newline 
\hspace*{1em}{<\textbf{l}>}My {<\textbf{seg}\hspace*{1em}{xml:id}="{Mik-L1s}">}object all sublime{</\textbf{seg}>}\mbox{}\newline 
\hspace*{1em}{</\textbf{l}>}\mbox{}\newline 
\hspace*{1em}{<\textbf{l}>}I shall {<\textbf{seg}\hspace*{1em}{xml:id}="{Mik-L2s}">}achieve in time{</\textbf{seg}>}—{</\textbf{l}>}\mbox{}\newline 
\hspace*{1em}{<\textbf{l}\hspace*{1em}{xml:id}="{Mik-L3}">}To let {<\textbf{seg}\hspace*{1em}{xml:id}="{L3s}">}the punishment fit the crime{</\textbf{seg}>},{</\textbf{l}>}\mbox{}\newline 
\hspace*{1em}{<\textbf{l}\hspace*{1em}{xml:id}="{Mik-l4}">}\mbox{}\newline 
\hspace*{1em}\hspace*{1em}{<\textbf{seg}\hspace*{1em}{copyOf}="{\#Mik-L3s}"/>};{</\textbf{l}>}\mbox{}\newline 
\hspace*{1em}{<\textbf{l}\hspace*{1em}{xml:id}="{Mik-l5}">}And make each pris'ner pent{</\textbf{l}>}\mbox{}\newline 
\hspace*{1em}{<\textbf{l}\hspace*{1em}{xml:id}="{Mik-l6}">}Unwillingly represent{</\textbf{l}>}\mbox{}\newline 
\hspace*{1em}{<\textbf{l}\hspace*{1em}{xml:id}="{Mik-l7}">}A source {<\textbf{seg}\hspace*{1em}{xml:id}="{Mik-l7s}">}of innocent merriment{</\textbf{seg}>},{</\textbf{l}>}\mbox{}\newline 
\hspace*{1em}{<\textbf{l}\hspace*{1em}{xml:id}="{Mik-l8}">}\mbox{}\newline 
\hspace*{1em}\hspace*{1em}{<\textbf{seg}\hspace*{1em}{copyOf}="{\#Mik-l7s}"/>}!{</\textbf{l}>}\mbox{}\newline 
{</\textbf{sp}>}\mbox{}\newline 
{<\textbf{sp}>}\mbox{}\newline 
\hspace*{1em}{<\textbf{speaker}>}Chorus{</\textbf{speaker}>}\mbox{}\newline 
\hspace*{1em}{<\textbf{l}>}His {<\textbf{seg}\hspace*{1em}{copyOf}="{\#Mik-L1s}"/>}\mbox{}\newline 
\hspace*{1em}{</\textbf{l}>}\mbox{}\newline 
\hspace*{1em}{<\textbf{l}>}He will {<\textbf{seg}\hspace*{1em}{copyOf}="{\#Mik-L2s}"/>}\mbox{}\newline 
\hspace*{1em}{</\textbf{l}>}\mbox{}\newline 
\hspace*{1em}{<\textbf{l}\hspace*{1em}{copyOf}="{\#Mik-L3}"/>}\mbox{}\newline 
\hspace*{1em}{<\textbf{l}\hspace*{1em}{copyOf}="{\#Mik-l4}"/>}\mbox{}\newline 
\hspace*{1em}{<\textbf{l}\hspace*{1em}{copyOf}="{\#Mik-l5}"/>}\mbox{}\newline 
\hspace*{1em}{<\textbf{l}\hspace*{1em}{copyOf}="{\#Mik-l6}"/>}\mbox{}\newline 
\hspace*{1em}{<\textbf{l}\hspace*{1em}{copyOf}="{\#Mik-l7}"/>}\mbox{}\newline 
\hspace*{1em}{<\textbf{l}\hspace*{1em}{copyOf}="{\#Mik-l8}"/>}\mbox{}\newline 
{</\textbf{sp}>}\end{shaded}\egroup\par \par
For further examples of the use of this attribute, see \textit{\hyperref[SAAT]{16.8.\ Alternation}} and \textit{\hyperref[GDAT]{19.3.\ Another Tree Notation}}.
\subsection[{Aggregation}]{Aggregation}\label{SAAG}\par
Because of the strict hierarchical organization of elements, or for other reasons, it may not always be possible or desirable to include all the parts of a possibly fragmented text segment within a single element. In section \textit{\hyperref[SAPTIP]{16.1.4.\ Intermediate Pointers}} we introduced the notion of an intermediate pointer as a way of pointing to discontinuous segments of this kind. In this section we first describe another way of linking the parts of a discontinuous whole, using a set of linking attributes, which are made available for any tag by following the procedure described at the beginning of this chapter. We then describe how the \hyperref[TEI.link]{<link>} element may be used to aggregate such segments, and finally introduce the \hyperref[TEI.join]{<join>} element, which is a special-purpose linking element specifically for representing the aggregation of parts, and the \hyperref[TEI.joinGrp]{<joinGrp>} for grouping \hyperref[TEI.join]{<join>} elements.\par
The linking attributes for aggregation are {\itshape next} and {\itshape prev}; each of these attributes has a single identifier as its value: 
\begin{sansreflist}
  
\item [\textbf{att.global.linking}] provides a set of attributes for hypertextual linking.\hfil\\[-10pt]\begin{sansreflist}
    \item[@{\itshape next}]
  points to the next element of a virtual aggregate of which the current element is part.
    \item[@{\itshape prev}]
  (previous) points to the previous element of a virtual aggregate of which the current element is part.
\end{sansreflist}  
\end{sansreflist}
\par
It is recommended that the elements indicated by these attributes be of the same type as the element bearing them.\par
The \hyperref[TEI.join]{<join>} element is also a member of the class of \textsf{att.pointing} elements, and so may carry any of the attributes of that class; for the list, see section \textit{\hyperref[SAPT]{16.1.\ Links}}.\par
Here is the material on which we base our first illustration of the use of these mechanisms. Our problem is to represent the s-units identified below as qs3 and qs4 as a single (but discontinuous) whole: \par\bgroup\index{q=<q>|exampleindex}\index{s=<s>|exampleindex}\index{emph=<emph>|exampleindex}\index{s=<s>|exampleindex}\index{emph=<emph>|exampleindex}\index{s=<s>|exampleindex}\index{q=<q>|exampleindex}\index{s=<s>|exampleindex}\index{s=<s>|exampleindex}\index{s=<s>|exampleindex}\exampleFont \begin{shaded}\noindent\mbox{}{<\textbf{q}>}\mbox{}\newline 
\hspace*{1em}{<\textbf{s}\hspace*{1em}{xml:id}="{qs2}">}Monsieur Paul, after he has taken equal\mbox{}\newline 
\hspace*{1em}\hspace*{1em} parts of goose breast and the finest pork, and\mbox{}\newline 
\hspace*{1em}\hspace*{1em} broken a certain number of egg yolks into them,\mbox{}\newline 
\hspace*{1em}\hspace*{1em} and ground them {<\textbf{emph}>}very{</\textbf{emph}>}, very fine,\mbox{}\newline 
\hspace*{1em}\hspace*{1em} cooks all with seasoning for some three hours.{</\textbf{s}>}\mbox{}\newline 
\hspace*{1em}{<\textbf{s}\hspace*{1em}{xml:id}="{qs3}">}\mbox{}\newline 
\hspace*{1em}\hspace*{1em}{<\textbf{emph}>}But{</\textbf{emph}>},{</\textbf{s}>}\mbox{}\newline 
{</\textbf{q}>}\mbox{}\newline 
{<\textbf{s}\hspace*{1em}{xml:id}="{ps2}">}she pushed her face nearer, and looked with\mbox{}\newline 
 ferocious gloating at the pâté\mbox{}\newline 
 inside me, her eyes like X rays,{</\textbf{s}>}\mbox{}\newline 
{<\textbf{q}>}\mbox{}\newline 
\hspace*{1em}{<\textbf{s}\hspace*{1em}{xml:id}="{qs4}">}he never stops stirring it!{</\textbf{s}>}\mbox{}\newline 
\hspace*{1em}{<\textbf{s}\hspace*{1em}{xml:id}="{qs5}">}Figure to yourself the work of it —{</\textbf{s}>}\mbox{}\newline 
\hspace*{1em}{<\textbf{s}\hspace*{1em}{xml:id}="{qs6}">}stir, stir, never stopping!{</\textbf{s}>}\mbox{}\newline 
{</\textbf{q}>}\end{shaded}\egroup\par \noindent  \par
Using the {\itshape prev} and {\itshape next} attributes, we can link the s-units with identifiers qs3 and qs4, either singly or doubly as follows: \par\hfill\bgroup\exampleFont\vskip 10pt\begin{shaded}
\obeyspaces   <s xml:id="qs3" next="\#qs4"><emph>But</emph>,</s>\newline
  <s xml:id="qs4">he never stops stirring it!</s>\end{shaded}
\par\egroup 
 \par\hfill\bgroup\exampleFont\vskip 10pt\begin{shaded}
\obeyspaces   <s xml:id="qs3"><emph>But</emph>,</s>\newline
  <s xml:id="qs4" prev="\#qs3">he never stops stirring it!</s>\end{shaded}
\par\egroup 
 \par\hfill\bgroup\exampleFont\vskip 10pt\begin{shaded}
\obeyspaces   <s xml:id="qs3" next="\#qs4"><emph>But</emph>,</s>\newline
  <s xml:id="qs4" prev="\#qs3">he never stops stirring it!</s>\end{shaded}
\par\egroup 
 Double linking of the two s-units, as illustrated by the last of these encodings, is equivalent to specifying a \hyperref[TEI.link]{<link>} element: \par\bgroup\index{link=<link>|exampleindex}\index{type=@type!<link>|exampleindex}\index{target=@target!<link>|exampleindex}\exampleFont \begin{shaded}\noindent\mbox{}{<\textbf{link}\hspace*{1em}{type}="{join}"\hspace*{1em}{target}="{\#qs3 \#qs4}"/>}\end{shaded}\egroup\par \par
Such a \hyperref[TEI.link]{<link>} element must carry a {\itshape type} attribute with a value of join to specify that the link is to be understood as joining its targets into a single aggregate.\par
The \hyperref[TEI.join]{<join>} element is equivalent to a \hyperref[TEI.link]{<link>} element of type join.  Unlike the \hyperref[TEI.link]{<link>} element, the \hyperref[TEI.join]{<join>} element can additionally specify information about the virtual element which it represents, by means of its {\itshape result} attribute. And finally, unlike the \hyperref[TEI.link]{<link>} element, the position of a \hyperref[TEI.join]{<join>} element within a text is significant: it must be supplied at a position where the element indicated by its {\itshape result} attribute would be contextually legal. 
\begin{sansreflist}
  
\item [\textbf{<join>}] (join) identifies a possibly fragmented segment of text, by pointing at the possibly discontiguous elements which compose it.\hfil\\[-10pt]\begin{sansreflist}
    \item[@{\itshape result}]
  specifies the name of an element which this aggregation may be understood to represent.
\end{sansreflist}  
\item [\textbf{<joinGrp>}] (join group) groups a collection of \hyperref[TEI.join]{<join>} elements and possibly pointers.\hfil\\[-10pt]\begin{sansreflist}
    \item[@{\itshape result}]
  supplies the default value for the {\itshape result} on each \hyperref[TEI.join]{<join>} included within the group.
\end{sansreflist}  
\end{sansreflist}
 To conclude the above example, we now use a \hyperref[TEI.join]{<join>} element to represent the virtual sentence formed by the aggregation of s1 and s2: \par\bgroup\index{join=<join>|exampleindex}\index{target=@target!<join>|exampleindex}\index{result=@result!<join>|exampleindex}\exampleFont \begin{shaded}\noindent\mbox{}{<\textbf{join}\hspace*{1em}{target}="{\#qs3 \#qs4}"\hspace*{1em}{result}="{s}"/>}\end{shaded}\egroup\par \noindent  As a further example, consider the following list of authors' names. The object of the \hyperref[TEI.join]{<join>} element here is to provide another list, composed of those authors from the larger list who happen to come from Heidelberg: \par\bgroup\index{list=<list>|exampleindex}\index{head=<head>|exampleindex}\index{item=<item>|exampleindex}\index{item=<item>|exampleindex}\index{item=<item>|exampleindex}\index{item=<item>|exampleindex}\index{item=<item>|exampleindex}\index{item=<item>|exampleindex}\index{join=<join>|exampleindex}\index{target=@target!<join>|exampleindex}\index{result=@result!<join>|exampleindex}\index{desc=<desc>|exampleindex}\exampleFont \begin{shaded}\noindent\mbox{}{<\textbf{list}>}\mbox{}\newline 
\hspace*{1em}{<\textbf{head}>}Authors{</\textbf{head}>}\mbox{}\newline 
\hspace*{1em}{<\textbf{item}\hspace*{1em}{xml:id}="{a\textunderscore uf}">}Figge, Udo {</\textbf{item}>}\mbox{}\newline 
\hspace*{1em}{<\textbf{item}\hspace*{1em}{xml:id}="{a\textunderscore ch}">}Heibach, Christiane {</\textbf{item}>}\mbox{}\newline 
\hspace*{1em}{<\textbf{item}\hspace*{1em}{xml:id}="{a\textunderscore gh}">}Heyer, Gerhard {</\textbf{item}>}\mbox{}\newline 
\hspace*{1em}{<\textbf{item}\hspace*{1em}{xml:id}="{a\textunderscore bp}">}Philipp, Bettina {</\textbf{item}>}\mbox{}\newline 
\hspace*{1em}{<\textbf{item}\hspace*{1em}{xml:id}="{a\textunderscore ms}">}Samiec, Monika {</\textbf{item}>}\mbox{}\newline 
\hspace*{1em}{<\textbf{item}\hspace*{1em}{xml:id}="{a\textunderscore ss}">}Schierholz, Stefan {</\textbf{item}>}\mbox{}\newline 
{</\textbf{list}>}\mbox{}\newline 
{<\textbf{join}\hspace*{1em}{target}="{\#a\textunderscore ch \#a\textunderscore bp \#a\textunderscore ss}"\mbox{}\newline 
\hspace*{1em}{result}="{list}">}\mbox{}\newline 
\hspace*{1em}{<\textbf{desc}>}Authors from Heidelberg{</\textbf{desc}>}\mbox{}\newline 
{</\textbf{join}>}\end{shaded}\egroup\par \par
The following example shows how \hyperref[TEI.join]{<join>} can be used to reconstruct a text cited in fragments presented out of order. The poem being remembered (an unusual translation of a well-known poem by Basho) runs ‘When the old pond / gets a new frog, / it's a new pond.’\par\bgroup\index{sp=<sp>|exampleindex}\index{speaker=<speaker>|exampleindex}\index{p=<p>|exampleindex}\index{q=<q>|exampleindex}\index{l=<l>|exampleindex}\index{l=<l>|exampleindex}\index{l=<l>|exampleindex}\index{sp=<sp>|exampleindex}\index{speaker=<speaker>|exampleindex}\index{p=<p>|exampleindex}\index{q=<q>|exampleindex}\index{l=<l>|exampleindex}\index{l=<l>|exampleindex}\index{sp=<sp>|exampleindex}\index{speaker=<speaker>|exampleindex}\index{p=<p>|exampleindex}\index{q=<q>|exampleindex}\index{l=<l>|exampleindex}\index{join=<join>|exampleindex}\index{target=@target!<join>|exampleindex}\index{result=@result!<join>|exampleindex}\index{scope=@scope!<join>|exampleindex}\exampleFont \begin{shaded}\noindent\mbox{}{<\textbf{sp}>}\mbox{}\newline 
\hspace*{1em}{<\textbf{speaker}>}Hughie{</\textbf{speaker}>}\mbox{}\newline 
\hspace*{1em}{<\textbf{p}>}How does it go?\mbox{}\newline 
\hspace*{1em}{<\textbf{q}>}\mbox{}\newline 
\hspace*{1em}\hspace*{1em}\hspace*{1em}{<\textbf{l}\hspace*{1em}{xml:id}="{frog-x1}">}da-da-da{</\textbf{l}>}\mbox{}\newline 
\hspace*{1em}\hspace*{1em}\hspace*{1em}{<\textbf{l}\hspace*{1em}{xml:id}="{frog-L2}">}gets a new frog{</\textbf{l}>}\mbox{}\newline 
\hspace*{1em}\hspace*{1em}\hspace*{1em}{<\textbf{l}>}...{</\textbf{l}>}\mbox{}\newline 
\hspace*{1em}\hspace*{1em}{</\textbf{q}>}\mbox{}\newline 
\hspace*{1em}{</\textbf{p}>}\mbox{}\newline 
{</\textbf{sp}>}\mbox{}\newline 
{<\textbf{sp}>}\mbox{}\newline 
\hspace*{1em}{<\textbf{speaker}>}Louie{</\textbf{speaker}>}\mbox{}\newline 
\hspace*{1em}{<\textbf{p}>}\mbox{}\newline 
\hspace*{1em}\hspace*{1em}{<\textbf{q}>}\mbox{}\newline 
\hspace*{1em}\hspace*{1em}\hspace*{1em}{<\textbf{l}\hspace*{1em}{xml:id}="{frog-L1}">}When the old pond{</\textbf{l}>}\mbox{}\newline 
\hspace*{1em}\hspace*{1em}\hspace*{1em}{<\textbf{l}>}...{</\textbf{l}>}\mbox{}\newline 
\hspace*{1em}\hspace*{1em}{</\textbf{q}>}\mbox{}\newline 
\hspace*{1em}{</\textbf{p}>}\mbox{}\newline 
{</\textbf{sp}>}\mbox{}\newline 
{<\textbf{sp}>}\mbox{}\newline 
\hspace*{1em}{<\textbf{speaker}>}Dewey{</\textbf{speaker}>}\mbox{}\newline 
\hspace*{1em}{<\textbf{p}>}\mbox{}\newline 
\hspace*{1em}\hspace*{1em}{<\textbf{q}>}...\mbox{}\newline 
\hspace*{1em}\hspace*{1em}{<\textbf{l}\hspace*{1em}{xml:id}="{frog-L3}">}It's a new pond.{</\textbf{l}>}\mbox{}\newline 
\hspace*{1em}\hspace*{1em}{</\textbf{q}>}\mbox{}\newline 
\hspace*{1em}{</\textbf{p}>}\mbox{}\newline 
\hspace*{1em}{<\textbf{join}\hspace*{1em}{target}="{\#frog-L1 \#frog-L2 \#frog-L3}"\mbox{}\newline 
\hspace*{1em}\hspace*{1em}{result}="{lg}"\hspace*{1em}{scope}="{root}"/>}\mbox{}\newline 
{</\textbf{sp}>}\end{shaded}\egroup\par \par
As with other forms of link, a grouping element \hyperref[TEI.joinGrp]{<joinGrp>} is available for use when a number of \hyperref[TEI.join]{<join>} elements of the same kind co-occur. This avoids the need to specify the {\itshape result} attribute for each \hyperref[TEI.join]{<join>} if they are all of the same type, and also allows us to restrict the domain within which their target elements are to be found, in the same way as for \hyperref[TEI.linkGrp]{<linkGrp>} elements (see \textit{\hyperref[SAPTLG]{16.1.3.\ Groups of Links}}). Like a \hyperref[TEI.join]{<join>}, a \hyperref[TEI.joinGrp]{<joinGrp>} may appear only where the elements represented by its contents are legal. Thus if we had created many \hyperref[TEI.join]{<join>} tags of the sort just described, we could group them together, and require that their components are all contained by an element with the identifier MFKFhungry as follows: \par\bgroup\index{joinGrp=<joinGrp>|exampleindex}\index{domains=@domains!<joinGrp>|exampleindex}\index{result=@result!<joinGrp>|exampleindex}\index{join=<join>|exampleindex}\index{target=@target!<join>|exampleindex}\index{join=<join>|exampleindex}\index{target=@target!<join>|exampleindex}\exampleFont \begin{shaded}\noindent\mbox{}{<\textbf{joinGrp}\hspace*{1em}{domains}="{\#mfkfhungry \#mfkfhungry}"\mbox{}\newline 
\hspace*{1em}{result}="{s}">}\mbox{}\newline 
\hspace*{1em}{<\textbf{join}\hspace*{1em}{target}="{\#qs3 \#qs4}"/>}\mbox{}\newline 
\hspace*{1em}{<\textbf{join}\hspace*{1em}{target}="{\#qs5 \#qs6}"/>}\mbox{}\newline 
{</\textbf{joinGrp}>}\end{shaded}\egroup\par \par
The \hyperref[TEI.join]{<join>} element is useful as a means of representing non-hierarchic structures (as further discussed in chapter \textit{\hyperref[NH]{20.\ Non-hierarchical Structures}}). It may also be used as a convenient way of representing a variety of analytic units, like the \hyperref[TEI.span]{<span>} and \hyperref[TEI.interp]{<interp>} elements discussed in chapter \textit{\hyperref[AI]{17.\ Simple Analytic Mechanisms}}. As an example, consider the following famous Zen koan: 
\begin{quote}\par
Zui-Gan called out to himself every day, ‘Master.’ \par
Then he answered himself, ‘Yes, sir.’ \par
And then he added, ‘Become sober.’ \par
Again he answered, ‘Yes, sir.’ \par
‘And after that,’ he continued, ‘do not be deceived by others.’ \par
‘Yes, sir; yes, sir,’ he replied.\end{quote}
\par
Suppose now that we wish to represent an interpretation of the above passage in which we distinguish between the various ‘voices’ adopted by Zui-Gan. In the following encoding, the {\itshape who} attribute has been used for this purpose; its value on each occasion supplies a pointer to the ‘voice’ to which each speech is attributed. (For convenience in this example, we use simply the first occurrence of the names used for each voice as the target for these pointers.) Note also that we add {\itshape xml:id} attributes to each distinct speech fragment, which we can then use to link the material spoken by each voice: \par\bgroup\index{text=<text>|exampleindex}\index{body=<body>|exampleindex}\index{p=<p>|exampleindex}\index{name=<name>|exampleindex}\index{q=<q>|exampleindex}\index{next=@next!<q>|exampleindex}\index{who=@who!<q>|exampleindex}\index{name=<name>|exampleindex}\index{p=<p>|exampleindex}\index{q=<q>|exampleindex}\index{next=@next!<q>|exampleindex}\index{who=@who!<q>|exampleindex}\index{p=<p>|exampleindex}\index{q=<q>|exampleindex}\index{next=@next!<q>|exampleindex}\index{who=@who!<q>|exampleindex}\index{p=<p>|exampleindex}\index{q=<q>|exampleindex}\index{next=@next!<q>|exampleindex}\index{who=@who!<q>|exampleindex}\index{p=<p>|exampleindex}\index{q=<q>|exampleindex}\index{next=@next!<q>|exampleindex}\index{who=@who!<q>|exampleindex}\index{q=<q>|exampleindex}\index{who=@who!<q>|exampleindex}\index{p=<p>|exampleindex}\index{q=<q>|exampleindex}\index{who=@who!<q>|exampleindex}\exampleFont \begin{shaded}\noindent\mbox{}{<\textbf{text}\hspace*{1em}{xml:id}="{zuitxt}">}\mbox{}\newline 
\hspace*{1em}{<\textbf{body}>}\mbox{}\newline 
\hspace*{1em}\hspace*{1em}{<\textbf{p}>}\mbox{}\newline 
\hspace*{1em}\hspace*{1em}\hspace*{1em}{<\textbf{name}\hspace*{1em}{xml:id}="{zuigan}">}Zui-Gan{</\textbf{name}>} called out to himself every day,\mbox{}\newline 
\hspace*{1em}\hspace*{1em}{<\textbf{q}\hspace*{1em}{next}="{\#zuiq2}"\hspace*{1em}{xml:id}="{zuiq1}"\mbox{}\newline 
\hspace*{1em}\hspace*{1em}\hspace*{1em}\hspace*{1em}{who}="{\#zuigan}">}\mbox{}\newline 
\hspace*{1em}\hspace*{1em}\hspace*{1em}\hspace*{1em}{<\textbf{name}\hspace*{1em}{xml:id}="{master}">}Master{</\textbf{name}>}.{</\textbf{q}>}\mbox{}\newline 
\hspace*{1em}\hspace*{1em}{</\textbf{p}>}\mbox{}\newline 
\hspace*{1em}\hspace*{1em}{<\textbf{p}>}Then he answered himself,\mbox{}\newline 
\hspace*{1em}\hspace*{1em}{<\textbf{q}\hspace*{1em}{next}="{\#zuiq4}"\hspace*{1em}{xml:id}="{zuiq2}"\mbox{}\newline 
\hspace*{1em}\hspace*{1em}\hspace*{1em}\hspace*{1em}{who}="{\#zuigan}">}Yes, sir.{</\textbf{q}>}\mbox{}\newline 
\hspace*{1em}\hspace*{1em}{</\textbf{p}>}\mbox{}\newline 
\hspace*{1em}\hspace*{1em}{<\textbf{p}>}And then he added,\mbox{}\newline 
\hspace*{1em}\hspace*{1em}{<\textbf{q}\hspace*{1em}{next}="{\#zuiq5}"\hspace*{1em}{xml:id}="{zuiq3}"\mbox{}\newline 
\hspace*{1em}\hspace*{1em}\hspace*{1em}\hspace*{1em}{who}="{\#master}">}Become sober.{</\textbf{q}>}\mbox{}\newline 
\hspace*{1em}\hspace*{1em}{</\textbf{p}>}\mbox{}\newline 
\hspace*{1em}\hspace*{1em}{<\textbf{p}>}Again he answered,\mbox{}\newline 
\hspace*{1em}\hspace*{1em}{<\textbf{q}\hspace*{1em}{next}="{\#zuiq7}"\hspace*{1em}{xml:id}="{zuiq4}"\mbox{}\newline 
\hspace*{1em}\hspace*{1em}\hspace*{1em}\hspace*{1em}{who}="{\#zuigan}">}Yes, sir.{</\textbf{q}>}\mbox{}\newline 
\hspace*{1em}\hspace*{1em}{</\textbf{p}>}\mbox{}\newline 
\hspace*{1em}\hspace*{1em}{<\textbf{p}>}\mbox{}\newline 
\hspace*{1em}\hspace*{1em}\hspace*{1em}{<\textbf{q}\hspace*{1em}{next}="{\#zuiq6}"\hspace*{1em}{xml:id}="{zuiq5}"\mbox{}\newline 
\hspace*{1em}\hspace*{1em}\hspace*{1em}\hspace*{1em}{who}="{\#master}">}And after that,{</\textbf{q}>}\mbox{}\newline 
\hspace*{1em}\hspace*{1em}\hspace*{1em}\hspace*{1em} he continued,\mbox{}\newline 
\hspace*{1em}\hspace*{1em}{<\textbf{q}\hspace*{1em}{xml:id}="{zuiq6}"\hspace*{1em}{who}="{\#master}">}do not be deceived by others.{</\textbf{q}>}\mbox{}\newline 
\hspace*{1em}\hspace*{1em}{</\textbf{p}>}\mbox{}\newline 
\hspace*{1em}\hspace*{1em}{<\textbf{p}>}\mbox{}\newline 
\hspace*{1em}\hspace*{1em}\hspace*{1em}{<\textbf{q}\hspace*{1em}{xml:id}="{zuiq7}"\hspace*{1em}{who}="{\#zuigan}">}Yes, sir; yes, sir,{</\textbf{q}>}\mbox{}\newline 
\hspace*{1em}\hspace*{1em}\hspace*{1em}\hspace*{1em} he replied.{</\textbf{p}>}\mbox{}\newline 
\hspace*{1em}{</\textbf{body}>}\mbox{}\newline 
{</\textbf{text}>}\end{shaded}\egroup\par \par
However, by using the \hyperref[TEI.join]{<join>} element, we can directly represent the complete speech attributed to each voice: \par\bgroup\index{joinGrp=<joinGrp>|exampleindex}\index{result=@result!<joinGrp>|exampleindex}\index{join=<join>|exampleindex}\index{target=@target!<join>|exampleindex}\index{desc=<desc>|exampleindex}\index{join=<join>|exampleindex}\index{target=@target!<join>|exampleindex}\index{desc=<desc>|exampleindex}\exampleFont \begin{shaded}\noindent\mbox{}{<\textbf{joinGrp}\hspace*{1em}{result}="{q}">}\mbox{}\newline 
\hspace*{1em}{<\textbf{join}\hspace*{1em}{target}="{\#zuiq1 \#zuiq2 \#zuiq4 \#zuiq7}">}\mbox{}\newline 
\hspace*{1em}\hspace*{1em}{<\textbf{desc}>}what Zui-Gan said{</\textbf{desc}>}\mbox{}\newline 
\hspace*{1em}{</\textbf{join}>}\mbox{}\newline 
\hspace*{1em}{<\textbf{join}\hspace*{1em}{target}="{\#zuiq3 \#zuiq5 \#zuiq6}">}\mbox{}\newline 
\hspace*{1em}\hspace*{1em}{<\textbf{desc}>}what Master said{</\textbf{desc}>}\mbox{}\newline 
\hspace*{1em}{</\textbf{join}>}\mbox{}\newline 
{</\textbf{joinGrp}>}\end{shaded}\egroup\par \par
Note the use of the \hyperref[TEI.desc]{<desc>} child element within the two \hyperref[TEI.join]{<join>}s making up the \hyperref[TEI.q]{<q>} element here. These enable us to document the speakers of the two virtual \hyperref[TEI.q]{<q>} elements represented by the \hyperref[TEI.join]{<join>} elements; this is necessary because the there is no way of specifying the attributes to be associated with a virtual element, in particular there is no way to specify a {\itshape who} value for them.\par
Suppose now that {\itshape xml:id} attributes, for whatever reasons, are not available. Then \hyperref[TEI.ptr]{<ptr>} elements may be created using any of the methods described in section \textit{\hyperref[SATS]{16.2.4.\ TEI XPointer Schemes}}. The {\itshape xml:id} attributes of \textit{these} elements may now be specified by the {\itshape target} attribute on the \hyperref[TEI.join]{<join>} elements. \par\bgroup\index{text=<text>|exampleindex}\index{body=<body>|exampleindex}\index{div1=<div1>|exampleindex}\index{p=<p>|exampleindex}\index{q=<q>|exampleindex}\index{p=<p>|exampleindex}\index{q=<q>|exampleindex}\index{p=<p>|exampleindex}\index{q=<q>|exampleindex}\index{p=<p>|exampleindex}\index{q=<q>|exampleindex}\index{p=<p>|exampleindex}\index{q=<q>|exampleindex}\index{q=<q>|exampleindex}\index{p=<p>|exampleindex}\index{q=<q>|exampleindex}\index{ab=<ab>|exampleindex}\index{type=@type!<ab>|exampleindex}\index{ptr=<ptr>|exampleindex}\index{target=@target!<ptr>|exampleindex}\index{ptr=<ptr>|exampleindex}\index{target=@target!<ptr>|exampleindex}\index{ptr=<ptr>|exampleindex}\index{target=@target!<ptr>|exampleindex}\index{ptr=<ptr>|exampleindex}\index{target=@target!<ptr>|exampleindex}\index{ptr=<ptr>|exampleindex}\index{target=@target!<ptr>|exampleindex}\index{ptr=<ptr>|exampleindex}\index{target=@target!<ptr>|exampleindex}\index{ptr=<ptr>|exampleindex}\index{target=@target!<ptr>|exampleindex}\index{joinGrp=<joinGrp>|exampleindex}\index{evaluate=@evaluate!<joinGrp>|exampleindex}\index{result=@result!<joinGrp>|exampleindex}\index{join=<join>|exampleindex}\index{target=@target!<join>|exampleindex}\index{desc=<desc>|exampleindex}\index{join=<join>|exampleindex}\index{target=@target!<join>|exampleindex}\index{desc=<desc>|exampleindex}\exampleFont \begin{shaded}\noindent\mbox{}{<\textbf{text}>}\mbox{}\newline 
\hspace*{1em}{<\textbf{body}>}\mbox{}\newline 
\textit{<!-- five div1 elements -->}\mbox{}\newline 
\hspace*{1em}\hspace*{1em}{<\textbf{div1}>}\mbox{}\newline 
\hspace*{1em}\hspace*{1em}\hspace*{1em}{<\textbf{p}>}Zui-Gan called out to himself every day, {<\textbf{q}>}Master.{</\textbf{q}>}\mbox{}\newline 
\hspace*{1em}\hspace*{1em}\hspace*{1em}{</\textbf{p}>}\mbox{}\newline 
\hspace*{1em}\hspace*{1em}\hspace*{1em}{<\textbf{p}>}Then he answered himself, {<\textbf{q}>}Yes, sir.{</\textbf{q}>}\mbox{}\newline 
\hspace*{1em}\hspace*{1em}\hspace*{1em}{</\textbf{p}>}\mbox{}\newline 
\hspace*{1em}\hspace*{1em}\hspace*{1em}{<\textbf{p}>}And then he added, {<\textbf{q}>}Become sober.{</\textbf{q}>}\mbox{}\newline 
\hspace*{1em}\hspace*{1em}\hspace*{1em}{</\textbf{p}>}\mbox{}\newline 
\hspace*{1em}\hspace*{1em}\hspace*{1em}{<\textbf{p}>}Again he answered, {<\textbf{q}>}Yes, sir.{</\textbf{q}>}\mbox{}\newline 
\hspace*{1em}\hspace*{1em}\hspace*{1em}{</\textbf{p}>}\mbox{}\newline 
\hspace*{1em}\hspace*{1em}\hspace*{1em}{<\textbf{p}>}\mbox{}\newline 
\hspace*{1em}\hspace*{1em}\hspace*{1em}\hspace*{1em}{<\textbf{q}>}And after that,{</\textbf{q}>} he continued, {<\textbf{q}>}do not be deceived by others.{</\textbf{q}>}\mbox{}\newline 
\hspace*{1em}\hspace*{1em}\hspace*{1em}{</\textbf{p}>}\mbox{}\newline 
\hspace*{1em}\hspace*{1em}\hspace*{1em}{<\textbf{p}>}\mbox{}\newline 
\hspace*{1em}\hspace*{1em}\hspace*{1em}\hspace*{1em}{<\textbf{q}>}Yes, sir; yes, sir,{</\textbf{q}>} he replied.{</\textbf{p}>}\mbox{}\newline 
\hspace*{1em}\hspace*{1em}\hspace*{1em}{<\textbf{ab}\hspace*{1em}{type}="{aggregation}">}\mbox{}\newline 
\hspace*{1em}\hspace*{1em}\hspace*{1em}\hspace*{1em}{<\textbf{ptr}\hspace*{1em}{xml:id}="{rzuiq1}"\mbox{}\newline 
\hspace*{1em}\hspace*{1em}\hspace*{1em}\hspace*{1em}\hspace*{1em}{target}="{\#xpath(//div1[6]/p[1]/q[1])}"/>}\mbox{}\newline 
\hspace*{1em}\hspace*{1em}\hspace*{1em}\hspace*{1em}{<\textbf{ptr}\hspace*{1em}{xml:id}="{rzuiq2}"\mbox{}\newline 
\hspace*{1em}\hspace*{1em}\hspace*{1em}\hspace*{1em}\hspace*{1em}{target}="{\#xpath(//div1[6]/p[2]/q[1])}"/>}\mbox{}\newline 
\hspace*{1em}\hspace*{1em}\hspace*{1em}\hspace*{1em}{<\textbf{ptr}\hspace*{1em}{xml:id}="{rzuiq3}"\mbox{}\newline 
\hspace*{1em}\hspace*{1em}\hspace*{1em}\hspace*{1em}\hspace*{1em}{target}="{\#xpath(//div1[6]/p[3]/q[1])}"/>}\mbox{}\newline 
\hspace*{1em}\hspace*{1em}\hspace*{1em}\hspace*{1em}{<\textbf{ptr}\hspace*{1em}{xml:id}="{rzuiq4}"\mbox{}\newline 
\hspace*{1em}\hspace*{1em}\hspace*{1em}\hspace*{1em}\hspace*{1em}{target}="{\#xpath(//div1[6]/p[4]/q[1])}"/>}\mbox{}\newline 
\hspace*{1em}\hspace*{1em}\hspace*{1em}\hspace*{1em}{<\textbf{ptr}\hspace*{1em}{xml:id}="{rzuiq5}"\mbox{}\newline 
\hspace*{1em}\hspace*{1em}\hspace*{1em}\hspace*{1em}\hspace*{1em}{target}="{\#xpath(//div1[6]/p[5]/q[1])}"/>}\mbox{}\newline 
\hspace*{1em}\hspace*{1em}\hspace*{1em}\hspace*{1em}{<\textbf{ptr}\hspace*{1em}{xml:id}="{rzuiq6}"\mbox{}\newline 
\hspace*{1em}\hspace*{1em}\hspace*{1em}\hspace*{1em}\hspace*{1em}{target}="{\#xpath(//div1[6]/p[5]/q[2])}"/>}\mbox{}\newline 
\hspace*{1em}\hspace*{1em}\hspace*{1em}\hspace*{1em}{<\textbf{ptr}\hspace*{1em}{xml:id}="{rzuiq7}"\mbox{}\newline 
\hspace*{1em}\hspace*{1em}\hspace*{1em}\hspace*{1em}\hspace*{1em}{target}="{\#xpath(//div1[6]/p[6]/q[1])}"/>}\mbox{}\newline 
\hspace*{1em}\hspace*{1em}\hspace*{1em}\hspace*{1em}{<\textbf{joinGrp}\hspace*{1em}{evaluate}="{one}"\hspace*{1em}{result}="{q}">}\mbox{}\newline 
\hspace*{1em}\hspace*{1em}\hspace*{1em}\hspace*{1em}\hspace*{1em}{<\textbf{join}\hspace*{1em}{target}="{\#rzuiq1 \#rzuiq2 \#rzuiq4 \#rzuiq7}">}\mbox{}\newline 
\hspace*{1em}\hspace*{1em}\hspace*{1em}\hspace*{1em}\hspace*{1em}\hspace*{1em}{<\textbf{desc}>}what Zui-Gan said{</\textbf{desc}>}\mbox{}\newline 
\hspace*{1em}\hspace*{1em}\hspace*{1em}\hspace*{1em}\hspace*{1em}{</\textbf{join}>}\mbox{}\newline 
\hspace*{1em}\hspace*{1em}\hspace*{1em}\hspace*{1em}\hspace*{1em}{<\textbf{join}\hspace*{1em}{target}="{\#rzuiq3 \#rzuiq5 \#rzuiq6}">}\mbox{}\newline 
\hspace*{1em}\hspace*{1em}\hspace*{1em}\hspace*{1em}\hspace*{1em}\hspace*{1em}{<\textbf{desc}>}what Master said{</\textbf{desc}>}\mbox{}\newline 
\hspace*{1em}\hspace*{1em}\hspace*{1em}\hspace*{1em}\hspace*{1em}{</\textbf{join}>}\mbox{}\newline 
\hspace*{1em}\hspace*{1em}\hspace*{1em}\hspace*{1em}{</\textbf{joinGrp}>}\mbox{}\newline 
\hspace*{1em}\hspace*{1em}\hspace*{1em}{</\textbf{ab}>}\mbox{}\newline 
\hspace*{1em}\hspace*{1em}{</\textbf{div1}>}\mbox{}\newline 
\hspace*{1em}{</\textbf{body}>}\mbox{}\newline 
{</\textbf{text}>}\end{shaded}\egroup\par \par
The extended pointer with identifier rzuiq2, for example, may be read as ‘the first \hyperref[TEI.q]{<q>} in the first \hyperref[TEI.p]{<p>}, within the sixth \hyperref[TEI.div1]{<div1>} element of the current document.’
\subsection[{Alternation}]{Alternation}\label{SAAT}\par
This section proposes elements for the representation of alternation. We say that two or more elements are in \textit{exclusive alternation} if any of those elements could be present in a text, but one and only one of them is; in addition, we say that those elements are \textit{mutually exclusive}. We say that the elements are in \textit{inclusive alternation} if at least one (and possibly more) of them is present. The elements that are in alternation may also be called \textit{alternants}.\par
The need to mark exclusive alternation arises frequently in text encoding. A common situation is one in which it can be determined that exactly one of several different words appears in a given location, but it cannot be determined which one. One way to mark such an exclusive alternation is to use the linking attribute {\itshape exclude}. Having marked an exclusive alternation, it can sometimes later be determined which of the alternants actually appears in the given location. To preserve the fact that an alternation was posited, one can add the linking attribute {\itshape select} to a tag which hierarchically encompasses the alternants, which points to the one which actually appears. To assign responsibility and degree of certainty to the choice, one can use the \hyperref[TEI.certainty]{<certainty>} tag described in chapter \textit{\hyperref[CE]{21.\ Certainty, Precision, and Responsibility}}. Also see that chapter for further discussion of certainty in general.\par
The {\itshape exclude} and {\itshape select} attributes may be used with any element assuming that they have been declared following the procedure discussed in the introduction to this chapter. 
\begin{sansreflist}
  
\item [\textbf{att.global.linking}] provides a set of attributes for hypertextual linking.\hfil\\[-10pt]\begin{sansreflist}
    \item[@{\itshape exclude}]
  points to elements that are in exclusive alternation with the current element.
    \item[@{\itshape select}]
  selects one or more alternants; if one alternant is selected, the ambiguity or uncertainty is marked as resolved. If more than one alternant is selected, the degree of ambiguity or uncertainty is marked as reduced by the number of alternants not selected.
\end{sansreflist}  
\end{sansreflist}
\par
A more general way to mark alternation, encompassing both exclusive and inclusive alternation, is to use the linking element \hyperref[TEI.alt]{<alt>}. The description and attributes of this tag and of the associated grouping tag \hyperref[TEI.altGrp]{<altGrp>} are as follows. These elements are also members of the \textsf{att.pointing} class and therefore have all the attributes associated with that class. 
\begin{sansreflist}
  
\item [\textbf{<alt>}] (alternation) identifies an alternation or a set of choices among elements or passages.\hfil\\[-10pt]\begin{sansreflist}
    \item[@{\itshape weights}]
  If {\itshape mode} is excl, each weight states the probability that the corresponding alternative occurs. If {\itshape mode} is incl each weight states the probability that the corresponding alternative occurs given that at least one of the other alternatives occurs.
\end{sansreflist}  
\item [\textbf{<altGrp>}] (alternation group) groups a collection of \hyperref[TEI.alt]{<alt>} elements and possibly pointers.
\end{sansreflist}
\par
To take a simple hypothetical example, suppose in transcribing a spoken text, we encounter an utterance that we can understand either as \textit{We had fun at the beach today.} or as \textit{We had sun at the beach today.} We can represent the exclusive alternation of these two possibilities by means of the {\itshape exclude} attribute as follows. \par\bgroup\index{div=<div>|exampleindex}\index{type=@type!<div>|exampleindex}\index{u=<u>|exampleindex}\index{exclude=@exclude!<u>|exampleindex}\index{u=<u>|exampleindex}\index{exclude=@exclude!<u>|exampleindex}\exampleFont \begin{shaded}\noindent\mbox{}{<\textbf{div}\hspace*{1em}{type}="{interview}">}\mbox{}\newline 
\hspace*{1em}{<\textbf{u}\hspace*{1em}{exclude}="{\#we.sun1}"\hspace*{1em}{xml:id}="{we.fun1}">}We had fun at the beach today.{</\textbf{u}>}\mbox{}\newline 
\hspace*{1em}{<\textbf{u}\hspace*{1em}{exclude}="{\#we.fun1}"\hspace*{1em}{xml:id}="{we.sun1}">}We had sun at the beach today.{</\textbf{u}>}\mbox{}\newline 
{</\textbf{div}>}\end{shaded}\egroup\par \par
If it is then determined that the speaker said \textit{fun}, not \textit{sun}, the encoder could amend the text by deleting the alternant containing \textit{sun} and the {\itshape exclude} attribute on the remaining alternant. Alternatively, the encoder could preserve the fact that there was uncertainty in the original transcription by retaining the alternants, and assigning the we.fun value to the {\itshape select} attribute value on the \hyperref[TEI.div]{<div>} element that encompasses the alternants, as in: \par\bgroup\index{div=<div>|exampleindex}\index{select=@select!<div>|exampleindex}\index{type=@type!<div>|exampleindex}\index{u=<u>|exampleindex}\index{exclude=@exclude!<u>|exampleindex}\index{u=<u>|exampleindex}\index{exclude=@exclude!<u>|exampleindex}\exampleFont \begin{shaded}\noindent\mbox{}{<\textbf{div}\hspace*{1em}{select}="{\#we.fun2}"\hspace*{1em}{type}="{interview}">}\mbox{}\newline 
\hspace*{1em}{<\textbf{u}\hspace*{1em}{exclude}="{\#we.sun2}"\hspace*{1em}{xml:id}="{we.fun2}">}We had fun at the beach\mbox{}\newline 
\hspace*{1em}\hspace*{1em} today.{</\textbf{u}>}\mbox{}\newline 
\hspace*{1em}{<\textbf{u}\hspace*{1em}{exclude}="{\#we.fun2}"\hspace*{1em}{xml:id}="{we.sun2}">}We had sun at the beach today.{</\textbf{u}>}\mbox{}\newline 
{</\textbf{div}>}\end{shaded}\egroup\par \par
The above alternation (including the {\itshape select} attribute) could be recoded by assigning the {\itshape exclude} attributes to tags that enclose just the words or even the characters that are mutually exclusive, as in:\footnote{See section \textit{\hyperref[AILC]{17.1.\ Linguistic Segment Categories}} for discussion of the \hyperref[TEI.w]{<w>} and \hyperref[TEI.c]{<c>} tags that can be used in the following examples instead of the <seg type="word"> and <seg type="character"> tags.} \par\bgroup\index{div=<div>|exampleindex}\index{type=@type!<div>|exampleindex}\index{u=<u>|exampleindex}\index{select=@select!<u>|exampleindex}\index{seg=<seg>|exampleindex}\index{exclude=@exclude!<seg>|exampleindex}\index{type=@type!<seg>|exampleindex}\index{seg=<seg>|exampleindex}\index{exclude=@exclude!<seg>|exampleindex}\index{type=@type!<seg>|exampleindex}\exampleFont \begin{shaded}\noindent\mbox{}{<\textbf{div}\hspace*{1em}{type}="{interview}">}\mbox{}\newline 
\hspace*{1em}{<\textbf{u}\hspace*{1em}{select}="{\#fun3}">}We had\mbox{}\newline 
\hspace*{1em}{<\textbf{seg}\hspace*{1em}{exclude}="{\#sun3}"\hspace*{1em}{xml:id}="{fun3}"\mbox{}\newline 
\hspace*{1em}\hspace*{1em}\hspace*{1em}{type}="{word}">}fun{</\textbf{seg}>}\mbox{}\newline 
\hspace*{1em}\hspace*{1em}{<\textbf{seg}\hspace*{1em}{exclude}="{\#fun3}"\hspace*{1em}{xml:id}="{sun3}"\mbox{}\newline 
\hspace*{1em}\hspace*{1em}\hspace*{1em}{type}="{word}">}sun{</\textbf{seg}>}\mbox{}\newline 
\hspace*{1em}\hspace*{1em} at the beach today.{</\textbf{u}>}\mbox{}\newline 
{</\textbf{div}>}\end{shaded}\egroup\par \noindent  \par\bgroup\index{div=<div>|exampleindex}\index{type=@type!<div>|exampleindex}\index{u=<u>|exampleindex}\index{seg=<seg>|exampleindex}\index{select=@select!<seg>|exampleindex}\index{type=@type!<seg>|exampleindex}\index{seg=<seg>|exampleindex}\index{exclude=@exclude!<seg>|exampleindex}\index{type=@type!<seg>|exampleindex}\index{seg=<seg>|exampleindex}\index{exclude=@exclude!<seg>|exampleindex}\index{type=@type!<seg>|exampleindex}\exampleFont \begin{shaded}\noindent\mbox{}{<\textbf{div}\hspace*{1em}{type}="{interview}">}\mbox{}\newline 
\hspace*{1em}{<\textbf{u}>}We had\mbox{}\newline 
\hspace*{1em}{<\textbf{seg}\hspace*{1em}{select}="{\#id-f}"\hspace*{1em}{type}="{word}">}\mbox{}\newline 
\hspace*{1em}\hspace*{1em}\hspace*{1em}{<\textbf{seg}\hspace*{1em}{exclude}="{\#id-s}"\hspace*{1em}{xml:id}="{id-f}"\mbox{}\newline 
\hspace*{1em}\hspace*{1em}\hspace*{1em}\hspace*{1em}{type}="{character}">}f{</\textbf{seg}>}\mbox{}\newline 
\hspace*{1em}\hspace*{1em}\hspace*{1em}{<\textbf{seg}\hspace*{1em}{exclude}="{\#id-f}"\hspace*{1em}{xml:id}="{id-s}"\mbox{}\newline 
\hspace*{1em}\hspace*{1em}\hspace*{1em}\hspace*{1em}{type}="{character}">}s{</\textbf{seg}>}\mbox{}\newline 
\hspace*{1em}\hspace*{1em}\hspace*{1em}\hspace*{1em} un{</\textbf{seg}>}\mbox{}\newline 
\hspace*{1em}\hspace*{1em} at the beach today.{</\textbf{u}>}\mbox{}\newline 
{</\textbf{div}>}\end{shaded}\egroup\par \par
Now suppose that the transcriber is uncertain whether the first word in the utterance is \textit{We} or \textit{Lee}, but is certain that if it is \textit{Lee}, then the other uncertain word is definitely \textit{fun} and not \textit{sun}. The three utterances that are in mutual exclusion can be encoded as follows. \par\bgroup\index{div=<div>|exampleindex}\index{type=@type!<div>|exampleindex}\index{u=<u>|exampleindex}\index{exclude=@exclude!<u>|exampleindex}\index{u=<u>|exampleindex}\index{exclude=@exclude!<u>|exampleindex}\index{u=<u>|exampleindex}\index{exclude=@exclude!<u>|exampleindex}\exampleFont \begin{shaded}\noindent\mbox{}{<\textbf{div}\hspace*{1em}{type}="{interview}">}\mbox{}\newline 
\textit{<!-- ... -->}\mbox{}\newline 
\hspace*{1em}{<\textbf{u}\hspace*{1em}{exclude}="{\#we.sun4 \#lee.fun4}"\mbox{}\newline 
\hspace*{1em}\hspace*{1em}{xml:id}="{we.fun4}">}We had fun at the beach today.{</\textbf{u}>}\mbox{}\newline 
\hspace*{1em}{<\textbf{u}\hspace*{1em}{exclude}="{\#we.fun4 \#lee.fun4}"\mbox{}\newline 
\hspace*{1em}\hspace*{1em}{xml:id}="{we.sun4}">}We had sun at the beach today.{</\textbf{u}>}\mbox{}\newline 
\hspace*{1em}{<\textbf{u}\hspace*{1em}{exclude}="{\#we.fun4 \#we.sun4}"\mbox{}\newline 
\hspace*{1em}\hspace*{1em}{xml:id}="{lee.fun4}">}Lee had fun at the beach today.{</\textbf{u}>}\mbox{}\newline 
\textit{<!-- ... -->}\mbox{}\newline 
{</\textbf{div}>}\end{shaded}\egroup\par \par
The preceding example can also be encoded with {\itshape exclude} attributes on the word segments \textit{We}, \textit{Lee}, \textit{fun}, and \textit{sun}: \par\bgroup\index{u=<u>|exampleindex}\index{seg=<seg>|exampleindex}\index{exclude=@exclude!<seg>|exampleindex}\index{type=@type!<seg>|exampleindex}\index{seg=<seg>|exampleindex}\index{exclude=@exclude!<seg>|exampleindex}\index{type=@type!<seg>|exampleindex}\index{seg=<seg>|exampleindex}\index{exclude=@exclude!<seg>|exampleindex}\index{type=@type!<seg>|exampleindex}\index{seg=<seg>|exampleindex}\index{exclude=@exclude!<seg>|exampleindex}\index{type=@type!<seg>|exampleindex}\exampleFont \begin{shaded}\noindent\mbox{}{<\textbf{u}>}\mbox{}\newline 
\hspace*{1em}{<\textbf{seg}\hspace*{1em}{exclude}="{\#lee}"\hspace*{1em}{xml:id}="{we}"\hspace*{1em}{type}="{word}">}We{</\textbf{seg}>}\mbox{}\newline 
\hspace*{1em}{<\textbf{seg}\hspace*{1em}{exclude}="{\#we \#sun}"\hspace*{1em}{xml:id}="{lee}"\mbox{}\newline 
\hspace*{1em}\hspace*{1em}{type}="{word}">}Lee{</\textbf{seg}>}\mbox{}\newline 
 had\mbox{}\newline 
{<\textbf{seg}\hspace*{1em}{exclude}="{\#sun}"\hspace*{1em}{xml:id}="{fun}"\mbox{}\newline 
\hspace*{1em}\hspace*{1em}{type}="{word}">}fun{</\textbf{seg}>}\mbox{}\newline 
\hspace*{1em}{<\textbf{seg}\hspace*{1em}{exclude}="{\#fun \#lee}"\hspace*{1em}{xml:id}="{sun}"\mbox{}\newline 
\hspace*{1em}\hspace*{1em}{type}="{word}">}sun{</\textbf{seg}>}\mbox{}\newline 
 at the beach today.\mbox{}\newline 
{</\textbf{u}>}\end{shaded}\egroup\par \par
The value of the {\itshape select} attribute is defined as a list of identifiers; hence it can also be used to narrow down the range of alternants, as in: \par\bgroup\index{div=<div>|exampleindex}\index{select=@select!<div>|exampleindex}\index{type=@type!<div>|exampleindex}\index{u=<u>|exampleindex}\index{exclude=@exclude!<u>|exampleindex}\index{u=<u>|exampleindex}\index{exclude=@exclude!<u>|exampleindex}\index{u=<u>|exampleindex}\index{exclude=@exclude!<u>|exampleindex}\exampleFont \begin{shaded}\noindent\mbox{}{<\textbf{div}\hspace*{1em}{select}="{\#we.fun5 \#lee.fun5}"\mbox{}\newline 
\hspace*{1em}{type}="{interview}">}\mbox{}\newline 
\hspace*{1em}{<\textbf{u}\hspace*{1em}{exclude}="{\#we.sun5 \#lee.fun5}"\mbox{}\newline 
\hspace*{1em}\hspace*{1em}{xml:id}="{we.fun5}">}We had fun at the beach today.{</\textbf{u}>}\mbox{}\newline 
\hspace*{1em}{<\textbf{u}\hspace*{1em}{exclude}="{\#we.fun5 \#lee.fun5}"\mbox{}\newline 
\hspace*{1em}\hspace*{1em}{xml:id}="{we.sun5}">}We had sun at the beach today.{</\textbf{u}>}\mbox{}\newline 
\hspace*{1em}{<\textbf{u}\hspace*{1em}{exclude}="{\#we.fun5 \#we.sun5}"\mbox{}\newline 
\hspace*{1em}\hspace*{1em}{xml:id}="{lee.fun5}">}Lee had fun at the beach today.{</\textbf{u}>}\mbox{}\newline 
{</\textbf{div}>}\end{shaded}\egroup\par \noindent  This is interpreted to mean that either the first or the third \hyperref[TEI.u]{<u>} element tag appears, and is thus equivalent to just the alternation of those two tags: \par\bgroup\index{div=<div>|exampleindex}\index{type=@type!<div>|exampleindex}\index{u=<u>|exampleindex}\index{exclude=@exclude!<u>|exampleindex}\index{u=<u>|exampleindex}\index{exclude=@exclude!<u>|exampleindex}\exampleFont \begin{shaded}\noindent\mbox{}{<\textbf{div}\hspace*{1em}{type}="{interview}">}\mbox{}\newline 
\hspace*{1em}{<\textbf{u}\hspace*{1em}{exclude}="{\#lee.fun6}"\hspace*{1em}{xml:id}="{we.fun6}">}We had fun at the beach\mbox{}\newline 
\hspace*{1em}\hspace*{1em} today.{</\textbf{u}>}\mbox{}\newline 
\hspace*{1em}{<\textbf{u}\hspace*{1em}{exclude}="{\#we.fun6}"\hspace*{1em}{xml:id}="{lee.fun6}">}Lee had fun at the beach today.{</\textbf{u}>}\mbox{}\newline 
{</\textbf{div}>}\end{shaded}\egroup\par \par
The {\itshape exclude} attribute can also be used in case there is uncertainty about the tag that appears in a certain position. For example, the occurrence of the word \textit{May} in the s-unit \textit{Let's go to May} can be interpreted, in the absence of other information, either as a person's name or as a date. The uncertainty can be rendered as follows, using the {\itshape exclude} attribute. \par\bgroup\index{s=<s>|exampleindex}\index{name=<name>|exampleindex}\index{exclude=@exclude!<name>|exampleindex}\index{date=<date>|exampleindex}\index{copyOf=@copyOf!<date>|exampleindex}\index{exclude=@exclude!<date>|exampleindex}\exampleFont \begin{shaded}\noindent\mbox{}{<\textbf{s}>}Let's go to\mbox{}\newline 
{<\textbf{name}\hspace*{1em}{exclude}="{\#mayn}"\hspace*{1em}{xml:id}="{mayd}">}May{</\textbf{name}>}\mbox{}\newline 
\hspace*{1em}{<\textbf{date}\hspace*{1em}{copyOf}="{\#mayd}"\hspace*{1em}{exclude}="{\#mayd}"\mbox{}\newline 
\hspace*{1em}\hspace*{1em}{xml:id}="{mayn}"/>}.{</\textbf{s}>}\end{shaded}\egroup\par \par
Note the use of the {\itshape copyOf} attribute discussed in section \textit{\hyperref[SAIE]{16.6.\ Identical Elements and Virtual Copies}}; this avoids having to repeat the content of the element whose correct tagging is in doubt.\par
The {\itshape copyOf} and the {\itshape exclude} attributes also provide for a simple way of indicating uncertainty about exactly where a particular element occurs in a document.\footnote{An alternative way of representing this problem is discussed in chapter \textit{\hyperref[CE]{21.\ Certainty, Precision, and Responsibility}}.} For example suppose that a particular \hyperref[TEI.div2]{<div2>} element appears either as the third and last of the \hyperref[TEI.div2]{<div2>} elements within the first \hyperref[TEI.div1]{<div1>} element in the body of a document, or as the first \hyperref[TEI.div2]{<div2>} of the second \hyperref[TEI.div1]{<div1>}. One solution would be to record the \hyperref[TEI.div2]{<div2>} in its entirety in the first of these positions, and a virtual copy of it in the second, and mark them as excluding each other as follows: \par\bgroup\index{body=<body>|exampleindex}\index{div1=<div1>|exampleindex}\index{div2=<div2>|exampleindex}\index{exclude=@exclude!<div2>|exampleindex}\index{div1=<div1>|exampleindex}\index{div2=<div2>|exampleindex}\index{copyOf=@copyOf!<div2>|exampleindex}\index{exclude=@exclude!<div2>|exampleindex}\exampleFont \begin{shaded}\noindent\mbox{}{<\textbf{body}>}\mbox{}\newline 
\hspace*{1em}{<\textbf{div1}\hspace*{1em}{xml:id}="{C1}">}\mbox{}\newline 
\hspace*{1em}\hspace*{1em}{<\textbf{div2}\hspace*{1em}{xml:id}="{C1S3}"\hspace*{1em}{exclude}="{\#C2S1}"/>}\mbox{}\newline 
\hspace*{1em}{</\textbf{div1}>}\mbox{}\newline 
\hspace*{1em}{<\textbf{div1}\hspace*{1em}{xml:id}="{C2}">}\mbox{}\newline 
\hspace*{1em}\hspace*{1em}{<\textbf{div2}\hspace*{1em}{xml:id}="{C2S1}"\hspace*{1em}{copyOf}="{\#C1S3}"\mbox{}\newline 
\hspace*{1em}\hspace*{1em}\hspace*{1em}{exclude}="{\#C1S3}"/>}\mbox{}\newline 
\hspace*{1em}{</\textbf{div1}>}\mbox{}\newline 
{</\textbf{body}>}\end{shaded}\egroup\par \noindent  In this case, the {\itshape select} attribute, if used, would appear on the \hyperref[TEI.body]{<body>} element.\par
Mutual exclusion can also be expressed using a \hyperref[TEI.link]{<link>}; the first example in this section can be recoded by removing the {\itshape exclude} attributes from the \hyperref[TEI.u]{<u>} elements, and adding a \hyperref[TEI.link]{<link>} element as follows:\footnote{In this example, we have placed the \hyperref[TEI.link]{<link>} next to the elements that represent the alternants. It could also have been placed elsewhere in the document, perhaps within a \hyperref[TEI.linkGrp]{<linkGrp>}.} \par\bgroup\index{div=<div>|exampleindex}\index{type=@type!<div>|exampleindex}\index{u=<u>|exampleindex}\index{u=<u>|exampleindex}\index{link=<link>|exampleindex}\index{type=@type!<link>|exampleindex}\index{target=@target!<link>|exampleindex}\exampleFont \begin{shaded}\noindent\mbox{}{<\textbf{div}\hspace*{1em}{type}="{interview}">}\mbox{}\newline 
\hspace*{1em}{<\textbf{u}\hspace*{1em}{xml:id}="{we.had.fun}">}We had fun at the beach today.{</\textbf{u}>}\mbox{}\newline 
\hspace*{1em}{<\textbf{u}\hspace*{1em}{xml:id}="{we.had.sun}">}We had sun at the beach today.{</\textbf{u}>}\mbox{}\newline 
\hspace*{1em}{<\textbf{link}\hspace*{1em}{type}="{exclusiveAlternation}"\mbox{}\newline 
\hspace*{1em}\hspace*{1em}{target}="{\#we.had.fun \#we.had.sun}"/>}\mbox{}\newline 
{</\textbf{div}>}\end{shaded}\egroup\par \par
Now we define the specialized linking element \hyperref[TEI.alt]{<alt>}, making it a member of the class \textsf{att.pointing}, and assigning it a {\itshape mode} attribute, which can have either of the values excl (for exclusive) or incl (for inclusive). Then the following equivalence holds: \par\bgroup\index{alt=<alt>|exampleindex}\index{target=@target!<alt>|exampleindex}\index{mode=@mode!<alt>|exampleindex}\exampleFont \begin{shaded}\noindent\mbox{}{<\textbf{alt}\hspace*{1em}{target}="{\#a \#b}"\hspace*{1em}{mode}="{excl}"/>}\end{shaded}\egroup\par \noindent  = \par\bgroup\index{link=<link>|exampleindex}\index{target=@target!<link>|exampleindex}\index{type=@type!<link>|exampleindex}\exampleFont \begin{shaded}\noindent\mbox{}{<\textbf{link}\hspace*{1em}{target}="{\#a \#b}"\mbox{}\newline 
\hspace*{1em}{type}="{exclusive\textunderscore alternation}"/>}\end{shaded}\egroup\par \par
The preceding \hyperref[TEI.link]{<link>} element may therefore be recoded as the following \hyperref[TEI.alt]{<alt>} element. \par\bgroup\index{alt=<alt>|exampleindex}\index{target=@target!<alt>|exampleindex}\index{mode=@mode!<alt>|exampleindex}\exampleFont \begin{shaded}\noindent\mbox{}{<\textbf{alt}\hspace*{1em}{target}="{\#we.had.fun \#we.had.sun}"\mbox{}\newline 
\hspace*{1em}{mode}="{excl}"/>}\end{shaded}\egroup\par \par
Another attribute that is defined specifically for the \hyperref[TEI.alt]{<alt>} element is {\itshape weights}, which is to be used if one wishes to assign \textit{probabilistic weights} to the targets (alternants). Its value is a list of numbers, corresponding to the targets, expressing the probability that each target appears.  If the alternants are mutually exclusive, then the weights must sum to 1.0.\par
Suppose in the preceding example that it is equiprobable whether \textit{fun} or \textit{sun} appears. Then the \hyperref[TEI.alt]{<alt>} element that represents the alternation may be stated as follows: \par\bgroup\index{alt=<alt>|exampleindex}\index{target=@target!<alt>|exampleindex}\index{mode=@mode!<alt>|exampleindex}\index{weights=@weights!<alt>|exampleindex}\exampleFont \begin{shaded}\noindent\mbox{}{<\textbf{alt}\hspace*{1em}{target}="{\#we.fun \#we.had.sun}"\mbox{}\newline 
\hspace*{1em}{mode}="{excl}"\hspace*{1em}{weights}="{0.5 0.5}"/>}\end{shaded}\egroup\par \par
The assignment of a weight of 1.0 to one target (and weights of 0 to all the others) is equivalent to selecting that target. Thus the following encoding is equivalent to the second example at the beginning of this section. \par\bgroup\index{u=<u>|exampleindex}\index{u=<u>|exampleindex}\index{alt=<alt>|exampleindex}\index{target=@target!<alt>|exampleindex}\index{mode=@mode!<alt>|exampleindex}\index{weights=@weights!<alt>|exampleindex}\exampleFont \begin{shaded}\noindent\mbox{}{<\textbf{u}\hspace*{1em}{xml:id}="{we.fun}">}We had fun at the beach today.{</\textbf{u}>}\mbox{}\newline 
{<\textbf{u}\hspace*{1em}{xml:id}="{we.sun}">}We had sun at the beach today.{</\textbf{u}>}\mbox{}\newline 
{<\textbf{alt}\hspace*{1em}{target}="{\#we.fun \#we.sun}"\hspace*{1em}{mode}="{excl}"\mbox{}\newline 
\hspace*{1em}{weights}="{1 0}"/>}\end{shaded}\egroup\par \noindent                                          The sum of the weights for <alt mode="incl"> tags ranges from 0\% to (100 × \texttt{k})\%, where \texttt{k} is the number of targets. If the sum is 0\%, then the alternation is equivalent to exclusive alternation; if the sum is (100 x k)\%, then all of the alternants must appear, and the situation is better encoded without an \hyperref[TEI.alt]{<alt>} tag.\par
If it is desired, \hyperref[TEI.alt]{<alt>} elements may be grouped together in an \hyperref[TEI.altGrp]{<altGrp>} element, and attribute values shared by the individual \hyperref[TEI.alt]{<alt>} elements may be identified on the \hyperref[TEI.altGrp]{<altGrp>} element. The {\itshape targFunc} attribute defaults to the value first.alternant next.alternant. \par
To illustrate, consider again the example of a transcribed utterance, in which it is uncertain whether the first word is \textit{We} or \textit{Lee}, whether the third word is \textit{fun} or \textit{sun}, but that if the first word is \textit{Lee}, then the third word is \textit{fun}. Now suppose we have the following additional information: if \textit{we} occurs, then the probability that \textit{fun} occurs is 50\% and that \textit{sun} occurs is 50\%; if \textit{fun} occurs, then the probability that \textit{we} occurs is 40\% and that \textit{Lee} occurs is 60\%. This situation can be encoded as follows. \par\bgroup\index{u=<u>|exampleindex}\index{seg=<seg>|exampleindex}\index{exclude=@exclude!<seg>|exampleindex}\index{type=@type!<seg>|exampleindex}\index{seg=<seg>|exampleindex}\index{exclude=@exclude!<seg>|exampleindex}\index{type=@type!<seg>|exampleindex}\index{seg=<seg>|exampleindex}\index{exclude=@exclude!<seg>|exampleindex}\index{type=@type!<seg>|exampleindex}\index{seg=<seg>|exampleindex}\index{exclude=@exclude!<seg>|exampleindex}\index{type=@type!<seg>|exampleindex}\index{altGrp=<altGrp>|exampleindex}\index{alt=<alt>|exampleindex}\index{target=@target!<alt>|exampleindex}\index{alt=<alt>|exampleindex}\index{target=@target!<alt>|exampleindex}\index{alt=<alt>|exampleindex}\index{target=@target!<alt>|exampleindex}\index{mode=@mode!<alt>|exampleindex}\index{weights=@weights!<alt>|exampleindex}\index{alt=<alt>|exampleindex}\index{target=@target!<alt>|exampleindex}\index{mode=@mode!<alt>|exampleindex}\index{weights=@weights!<alt>|exampleindex}\exampleFont \begin{shaded}\noindent\mbox{}{<\textbf{u}>}\mbox{}\newline 
\hspace*{1em}{<\textbf{seg}\hspace*{1em}{exclude}="{\#lee2}"\hspace*{1em}{xml:id}="{we2}"\mbox{}\newline 
\hspace*{1em}\hspace*{1em}{type}="{word}">}We{</\textbf{seg}>}\mbox{}\newline 
\hspace*{1em}{<\textbf{seg}\hspace*{1em}{exclude}="{\#we2}"\hspace*{1em}{xml:id}="{lee2}"\mbox{}\newline 
\hspace*{1em}\hspace*{1em}{type}="{word}">}Lee{</\textbf{seg}>}\mbox{}\newline 
 had\mbox{}\newline 
{<\textbf{seg}\hspace*{1em}{exclude}="{\#sun2}"\hspace*{1em}{xml:id}="{fun2}"\mbox{}\newline 
\hspace*{1em}\hspace*{1em}{type}="{word}">}fun{</\textbf{seg}>}\mbox{}\newline 
\hspace*{1em}{<\textbf{seg}\hspace*{1em}{exclude}="{\#fun2}"\hspace*{1em}{xml:id}="{sun2}"\mbox{}\newline 
\hspace*{1em}\hspace*{1em}{type}="{word}">}sun{</\textbf{seg}>}\mbox{}\newline 
 at the beach today.\mbox{}\newline 
{</\textbf{u}>}\mbox{}\newline 
{<\textbf{altGrp}>}\mbox{}\newline 
\hspace*{1em}{<\textbf{alt}\hspace*{1em}{target}="{\#we2 \#lee2}"/>}\mbox{}\newline 
\hspace*{1em}{<\textbf{alt}\hspace*{1em}{target}="{\#fun2 \#sun2}"/>}\mbox{}\newline 
\hspace*{1em}{<\textbf{alt}\hspace*{1em}{target}="{\#we2 \#fun2}"\hspace*{1em}{mode}="{incl}"\mbox{}\newline 
\hspace*{1em}\hspace*{1em}{weights}="{0.5 0.5}"/>}\mbox{}\newline 
\hspace*{1em}{<\textbf{alt}\hspace*{1em}{target}="{\#lee2 \#fun2}"\hspace*{1em}{mode}="{incl}"\mbox{}\newline 
\hspace*{1em}\hspace*{1em}{weights}="{1.0 0.6}"/>}\mbox{}\newline 
{</\textbf{altGrp}>}\end{shaded}\egroup\par \noindent  As noted above, when the {\itshape mode} attribute on an \hyperref[TEI.alt]{<alt>} has the value incl, then each weight states the probability that the corresponding alternative occurs, given that at least one of the other alternatives occurs.\par
From the information in this encoding, we can determine that the probability is about 28.5\% that the utterance is ‘We had fun at the beach today’, 28.5\% that it is \textit{We had sun at the beach today}, and 43\% that it is \textit{Lee had fun at the beach today}.\par
Another very similar example is the following regarding the text of a Broadway song. In three different versions of the song, the same line reads ‘Her skin is tender as a leather glove’, ‘Her skin is tender as a baseball glove’, and ‘Her skin is tender as Dimaggio's glove.’\footnote{The variant readings are found in the commercial sheet music, the performance score, and the Broadway cast recording.}\par
If we wish to express this textual variation using the \hyperref[TEI.alt]{<alt>} element, we can record our relative confidence in the readings \textit{Dimaggio's} (with probability 50\%), \textit{a leather} (25\%), and \textit{a baseball} (25\%).\par
Let us extend the example with a further (imaginary) variation, supposing for the sake of the argument that the next line is variously given as \textit{and she bats from right to left} (with probability 50\%) or \textit{now ain't that too damn bad} (with probability 50\%). Using the \hyperref[TEI.alt]{<alt>} element, we can express the conviction that if the first choice for the second line is correct, then the probability that the first line contains \textit{Dimaggio's} is 90\%, and each of the others 5\%; whereas if the second choice for the second line is correct, then the probability that the first line contains \textit{Dimaggio's} is 10\%, and each of the others is 45\%. This can be encoded, with an \hyperref[TEI.altGrp]{<altGrp>} element containing a combination of exclusive and inclusive \hyperref[TEI.alt]{<alt>} elements, as follows.  \par\bgroup\index{div=<div>|exampleindex}\index{type=@type!<div>|exampleindex}\index{l=<l>|exampleindex}\index{seg=<seg>|exampleindex}\index{seg=<seg>|exampleindex}\index{seg=<seg>|exampleindex}\index{l=<l>|exampleindex}\index{l=<l>|exampleindex}\index{altGrp=<altGrp>|exampleindex}\index{alt=<alt>|exampleindex}\index{target=@target!<alt>|exampleindex}\index{mode=@mode!<alt>|exampleindex}\index{weights=@weights!<alt>|exampleindex}\index{alt=<alt>|exampleindex}\index{target=@target!<alt>|exampleindex}\index{mode=@mode!<alt>|exampleindex}\index{weights=@weights!<alt>|exampleindex}\index{altGrp=<altGrp>|exampleindex}\index{mode=@mode!<altGrp>|exampleindex}\index{alt=<alt>|exampleindex}\index{target=@target!<alt>|exampleindex}\index{weights=@weights!<alt>|exampleindex}\index{alt=<alt>|exampleindex}\index{target=@target!<alt>|exampleindex}\index{weights=@weights!<alt>|exampleindex}\index{alt=<alt>|exampleindex}\index{target=@target!<alt>|exampleindex}\index{weights=@weights!<alt>|exampleindex}\index{alt=<alt>|exampleindex}\index{target=@target!<alt>|exampleindex}\index{weights=@weights!<alt>|exampleindex}\index{alt=<alt>|exampleindex}\index{target=@target!<alt>|exampleindex}\index{weights=@weights!<alt>|exampleindex}\index{alt=<alt>|exampleindex}\index{target=@target!<alt>|exampleindex}\index{weights=@weights!<alt>|exampleindex}\exampleFont \begin{shaded}\noindent\mbox{}{<\textbf{div}\hspace*{1em}{xml:id}="{bm}"\hspace*{1em}{type}="{song}">}\mbox{}\newline 
\hspace*{1em}{<\textbf{l}>}Her skin is tender as\mbox{}\newline 
\hspace*{1em}{<\textbf{seg}\hspace*{1em}{xml:id}="{dm}">}Dimaggio's{</\textbf{seg}>}\mbox{}\newline 
\hspace*{1em}\hspace*{1em}{<\textbf{seg}\hspace*{1em}{xml:id}="{lt}">}a leather{</\textbf{seg}>}\mbox{}\newline 
\hspace*{1em}\hspace*{1em}{<\textbf{seg}\hspace*{1em}{xml:id}="{bb}">}a baseball{</\textbf{seg}>}\mbox{}\newline 
\hspace*{1em}\hspace*{1em} glove,{</\textbf{l}>}\mbox{}\newline 
\hspace*{1em}{<\textbf{l}\hspace*{1em}{xml:id}="{rl}">}and she bats from right to left.{</\textbf{l}>}\mbox{}\newline 
\hspace*{1em}{<\textbf{l}\hspace*{1em}{xml:id}="{db}">}now ain't that too damn bad.{</\textbf{l}>}\mbox{}\newline 
{</\textbf{div}>}\mbox{}\newline 
{<\textbf{altGrp}>}\mbox{}\newline 
\hspace*{1em}{<\textbf{alt}\hspace*{1em}{target}="{\#dm \#lt \#bb}"\hspace*{1em}{mode}="{excl}"\mbox{}\newline 
\hspace*{1em}\hspace*{1em}{weights}="{0.5 0.25 0.25}"/>}\mbox{}\newline 
\hspace*{1em}{<\textbf{alt}\hspace*{1em}{target}="{\#rl \#db}"\hspace*{1em}{mode}="{excl}"\mbox{}\newline 
\hspace*{1em}\hspace*{1em}{weights}="{0.50 0.50}"/>}\mbox{}\newline 
{</\textbf{altGrp}>}\mbox{}\newline 
{<\textbf{altGrp}\hspace*{1em}{mode}="{incl}">}\mbox{}\newline 
\hspace*{1em}{<\textbf{alt}\hspace*{1em}{target}="{\#dm \#rl}"\hspace*{1em}{weights}="{0.90 0.90}"/>}\mbox{}\newline 
\hspace*{1em}{<\textbf{alt}\hspace*{1em}{target}="{\#lt \#rl}"\hspace*{1em}{weights}="{0.5 0.5}"/>}\mbox{}\newline 
\hspace*{1em}{<\textbf{alt}\hspace*{1em}{target}="{\#bb \#rl}"\hspace*{1em}{weights}="{0.5 0.5}"/>}\mbox{}\newline 
\hspace*{1em}{<\textbf{alt}\hspace*{1em}{target}="{\#dm \#db}"\hspace*{1em}{weights}="{0.10 0.10}"/>}\mbox{}\newline 
\hspace*{1em}{<\textbf{alt}\hspace*{1em}{target}="{\#lt \#db}"\hspace*{1em}{weights}="{0.45 0.90}"/>}\mbox{}\newline 
\hspace*{1em}{<\textbf{alt}\hspace*{1em}{target}="{\#bb \#db}"\hspace*{1em}{weights}="{0.45 0.90}"/>}\mbox{}\newline 
{</\textbf{altGrp}>}\end{shaded}\egroup\par 
\subsection[{Stand-off Markup}]{Stand-off Markup}\label{SASO}
\subsubsection[{Introduction}]{Introduction}\label{SASOin}\par
Most of the mechanisms defined in this chapter rely to a greater or lesser extent on the fact that tags in a marked-up document can both assert a property for a span of text which they enclose, and assert the existence of an association between themselves and some other span of text elsewhere. In stand-off markup, there is a clear separation of these two behaviours: the markup does not directly contain any part of the text, but instead includes it by reference. One specific mechanism recommended by these Guidelines for this purpose is the standard XInclude mechanism defined by the W3C; another is to use pointers as demonstrated elsewhere in this chapter. \par
There are many reasons for using stand-off markup: the source text might be read-only so that additional markup cannot be added, or a single text may need to be marked up according to several hierarchically incompatible schemes, or a single scheme may need to accommodate multiple hierarchical ambiguities, so that a single markup tree is not the most faithful representation of the source material.\par
This section describes a generic mechanism for expressing \textit{all} kinds of markup externally as stand-off tags, for use whenever it is appropriate; and a place in the TEI structure (\hyperref[TEI.standOff]{<standOff>}) to contain certain common kinds of stand-off markup.\par
Throughout this section the following terms will be systematically used in specific senses. \begin{description}

\item[{\textit{source document}}]a document to which the stand-off markup refers (a source document can be either XML or plain text); there may be more than one source document.
\item[{\textit{internal markup}}]markup that is already present in an XML source document
\item[{\textit{stand-off markup}}]markup that is either outside of the source document and points in to it to the data it describes, or is pointed at by the data that refers to it; or alternatively is in another part of the source document and points elsewhere within the document to the data it describes, or is pointed at by data elsewhere that refers to it.
\item[{\textit{external document}}]a document that contains stand-off markup that points to a different, source document
\item[{\textit{internalize}}]the action of creating a new XML document with external markup and data integrated with the source document data, and possibly some source document markup as well
\item[{\textit{externalize}}]a process applied to markup from a pre-existing XML document, which splits it into two documents, an XML (external) document containing some of the markup of the original document, and another (source) XML document containing whatever text content and markup has not been extracted into the stand-off document; if all markup has been externalized from a document, the new source may be a plain text document
\end{description} \par
The three major requirements satisfied by this scheme for stand-off markup are: \begin{enumerate}
\item[a] any valid TEI markup can be either internal or external,
\item[b] external markup can be internalized by applying it to the document content by either substituting the existing markup or adding to it, to form a valid TEI document, and
\item[c] the external markup itself specifies whether an internalized document is to be created by substituting the existing internal markup or by adding to it.
\end{enumerate}
\subsubsection[{Overview of XInclude }]{Overview of XInclude }\label{SASOov}\par
Stand-off markup which relies on the inclusion of virtual content is adequately supported by the W3C XInclude recommendation, which is also recommended for use by these Guidelines.\footnote{The version on which this text is based is the \xref{http://www.w3.org/TR/2004/REC-xinclude-20041220/}{W3C Recommendation dated 20 December 2004.}.} XInclude defines a namespace (\textit{http://www.w3.org/2001/XInclude}), which in these Guidelines will be associated with the prefix \textit{xi:}, and exactly two elements, \texttt{<xi:include>} and \texttt{<xi:fallback>}. XInclude relies on the \xref{http://www.w3.org/TR/xptr-framework/}{XPointer framework} discussed elsewhere in this chapter to point to the actual fragments of text to be internalized. Although XInclude only requires support for the \xref{http://www.w3.org/TR/xptr-element/}{\texttt{element()}} scheme of XPointer, these Guidelines permit the use of any of the pointing schemes discussed in section \textit{\hyperref[SAXP]{16.2.\ Pointing Mechanisms}}.\par
XInclude is a W3C recommendation which specifies a syntax for the inclusion within an XML document of data fragments placed in different resources. Included resources can be either plain text or XML. XInclude instructions within an XML document are meant to be replaced by a resource targetted by a URI, possibly augmented by an XPointer that identifies the exact subresource to be included. \par
The \texttt{<xi:include>} element uses the {\itshape href} attribute to specify the location of the resource to be included; its value is an URI containing, if necessary, an XPointer. Additionally, it uses the {\itshape parse} attribute (whose only valid values are text and xml) to specify whether the included content is plain text or an XML fragment, and the {\itshape encoding} attribute to provide a hint, when the included fragment is text, of the character encoding of the fragment. An optional \texttt{<xi:fallback>} element is also permitted within an \texttt{<xi:include>}; it specifies alternative content to be used when the external resource cannot be fetched for some reason. Its use is not however recommended for stand-off markup.
\subsubsection[{Stand-off Markup in TEI}]{Stand-off Markup in TEI}\label{SASOso}\par
The operations of internalizing and externalizing markup are very useful and practically important. XInclude processing as defined by the W3C \textit{is} internalization of one or more source documents' content into a stand-off document. TEI use of XInclude for stand-off markup enables use of XInclude-conformant software to perform this useful operation. However, internalization is not clearly defined for all stand-off files, because the structure of the internal and external markup trees may overlap. In particular, when an external markup document selects a range that overlaps partial elements in the source document, it is not clear how the semantics of internalization (inclusion) should work, since partial elements are not XML objects.\footnote{This corresponds to the observation that overlapping XML tags reflecting a textual version of such an inclusion would not even be well-formed XML. This kind of overlap in textual phenomena of interest is in fact the major reason that stand-off markup is needed.} XInclude defines a semantics for this case that involves only complete elements.\par
When a range selection partially overlaps a number of elements in a source document, XInclude specifies that the partially overlapping elements should be included as well as all completely overlapping elements and characters (partially overlapping characters are not possible). The effect of this is that elements that straddle the start or end of a selected range will be included as wrappers for those of their children that are completely or partially selected by the range. For example, given the following source document: \par\bgroup\index{body=<body>|exampleindex}\index{p=<p>|exampleindex}\index{emph=<emph>|exampleindex}\index{p=<p>|exampleindex}\index{emph=<emph>|exampleindex}\exampleFont \begin{shaded}\noindent\mbox{}{<\textbf{body}>}\mbox{}\newline 
\hspace*{1em}{<\textbf{p}\hspace*{1em}{xml:id}="{par1}">}home, {<\textbf{emph}>}home{</\textbf{emph}>} on Brokeback Mountain.{</\textbf{p}>}\mbox{}\newline 
\hspace*{1em}{<\textbf{p}\hspace*{1em}{xml:id}="{par2}">}That was the {<\textbf{emph}>}song{</\textbf{emph}>} that I sang{</\textbf{p}>}\mbox{}\newline 
{</\textbf{body}>}\end{shaded}\egroup\par \noindent  and the following external document: \par\hfill\bgroup\exampleFont\vskip 10pt\begin{shaded}
\obeyspaces   <body>\newline
     <div><include href="example1.xml" xmlns="http://www.w3.org/2001/XInclude"\newline
xpointer="range(xpath(id('par1')//emph),xpath(id('par2')//emph))"/>\newline
     </div>\newline
 </body>   \end{shaded}
\par\egroup 
 the resulting document after XInclude processing of this external document would be: \par\hfill\bgroup\exampleFont\vskip 10pt\begin{shaded}
\obeyspaces    <body>\newline
   <div>\newline
     <p xml:id="par1">home, <emph>home</emph> on Brokeback Mountain.</p>\newline
     <p xml:id="par2">That was the <emph>song</emph> that I sang</p>\newline
   </div>\newline
   </body>\end{shaded}
\par\egroup 
 The result of the inclusion is two paragraph elements, while the original range designated in the source document overlapped two paragraph fragments.  The semantics of XInclude require the creation of well-formed XML results even though the pointing mechanisms it uses do not necessarily respect the hierarchical structure of XML documents, as in this case. While this is a good way to ensure that internalization is always possible, it has implications for the use of XInclude as a notation for the \textit{description} of overlapping markup structures.\par
When overlapping hierarchies need to be represented for a single document, each hierarchy must be represented by a separate set of XInclude tags pointing to a common source document. This sort of structure corresponds to common practice in work with linguistic text corpora. In such corpora, each potentially overlapping hierarchy of elements for the text is represented as a separate stream of stand-off markup. Generally the source text contains markup for the smallest significant units of analysis in the corpus, such as words or morphemes, this information and its markup representing a layer of common information that is shared by all the various hierarchies. As a way of organizing the representation of complex data, this technique generally allows a large number of {\itshape xml:id} attributes to be attached to the shared elements, providing robust anchors for links and facilitating adjustments to the source document without breaking external documents that reference it.\par
Any tag can be externalized by  removing its content and replacing it with an \texttt{<xi:include>} element that contains an XPointer pointing to the desired content.\par
For instance the following portion of a TEI document: \par\bgroup\index{text=<text>|exampleindex}\index{body=<body>|exampleindex}\index{head=<head>|exampleindex}\index{l=<l>|exampleindex}\index{l=<l>|exampleindex}\index{l=<l>|exampleindex}\index{l=<l>|exampleindex}\index{l=<l>|exampleindex}\exampleFont \begin{shaded}\noindent\mbox{}{<\textbf{text}>}\mbox{}\newline 
\hspace*{1em}{<\textbf{body}>}\mbox{}\newline 
\hspace*{1em}\hspace*{1em}{<\textbf{head}>}1755{</\textbf{head}>}\mbox{}\newline 
\hspace*{1em}\hspace*{1em}{<\textbf{l}>}To make a prairie it takes a clover and one bee,{</\textbf{l}>}\mbox{}\newline 
\hspace*{1em}\hspace*{1em}{<\textbf{l}>}One clover, and a bee,{</\textbf{l}>}\mbox{}\newline 
\hspace*{1em}\hspace*{1em}{<\textbf{l}>}And revery.{</\textbf{l}>}\mbox{}\newline 
\hspace*{1em}\hspace*{1em}{<\textbf{l}>}The revery alone will do,{</\textbf{l}>}\mbox{}\newline 
\hspace*{1em}\hspace*{1em}{<\textbf{l}>}If bees are few.{</\textbf{l}>}\mbox{}\newline 
\hspace*{1em}{</\textbf{body}>}\mbox{}\newline 
{</\textbf{text}>}\end{shaded}\egroup\par \noindent  can be externalized by placing the actual text in a separate document, and providing exactly the same markup with the \texttt{<xi:include>} elements: \mbox{}\newline 
  \textbf{Source.xml} \par\hfill\bgroup\exampleFont\vskip 10pt\begin{shaded}
\obeyspaces <content>To make a prairie it takes a clover and one bee,⃥n\newline
One clover, and a bee,⃥n\newline
And revery.⃥n\newline
The revery alone will do,⃥n\newline
If bees are few.⃥n\newline
</content>\end{shaded}
\par\egroup 
 \mbox{}\newline 
  \textbf{External.xml} \par\hfill\bgroup\exampleFont\vskip 10pt\begin{shaded}
\obeyspaces <text xmlns:xi="http://www.w3.org/2001/XInclude">\newline
 <body>\newline
  <head>1755</head>\newline
   <l>\newline
    <xi:include href="Source.xml" parse="xml"\newline
 xpointer="string-range(element(/1),  0, 48)"/>\newline
   </l>\newline
   <l>\newline
    <xi:include href="Source.xml" parse="xml"\newline
 xpointer="string-range(element(/1), 49, 71)"/>\newline
   </l>\newline
   <l>\newline
    <xi:include href="Source.xml" parse="xml"\newline
 xpointer="string-range(element(/1), 72, 83)"/>\newline
   </l>\newline
   <l>\newline
    <xi:include href="Source.xml" parse="xml"\newline
 xpointer="string-range(element(/1), 84,109)"/>\newline
   </l>\newline
   <l>\newline
    <xi:include href="Source.xml" parse="xml"\newline
 xpointer="string-range(element(/1),110,126)"/>\newline
   </l>\newline
 </body>\newline
</text>\end{shaded}
\par\egroup 
\par
Please note that this specification requires that the XInclude namespace declaration is present in all cases. The \texttt{<xi:fallback>} element contains text or XML fragments to be placed in the document if the inclusion fails for any reason (for instance due to inaccessibility of an external resource). The \texttt{<xi:fallback>} element is optional; if it is not present an XInclude processor must signal a fatal error when a resource is not found. This is the preferred behaviour for use with stand-off markup. These Guidelines recommend against the use of \texttt{<xi:fallback>} for stand-off markup.
\subsubsection[{Well-formedness and Validity of Stand-off Markup}]{Well-formedness and Validity of Stand-off Markup}\label{SASOva}\par
The whole source fragment identified by an XInclude element, as well as any markup therein contained is inserted in the position specified, and an XInclude processor is required to ensure that the resulting internalized document is well-formed. This has obvious implications when the external document contains XML markup. A plain text source document will always create a well-formed internalized document. \par
While a TEI customization may permit \texttt{<xi:include>} elements in various places in a TEI document instance, in general these Guidelines suggest that validity be verified after the resolution of all the \texttt{<xi:include>} elements.
\subsubsection[{Including Text or XML Fragments}]{Including Text or XML Fragments}\label{SASOfr}\par
When the source text is plain text the overall form of the XPointer pointing to it is of minimal importance. The form of the XPointer matters considerably, on the other hand, when the source document is XML.\par
In this case, it is rather important to distinguish whether we intend to substitute the source XML with the new one, or just to add new markup to it. The XPointers used in the references can express both cases.\par
A simple way is to make sure to select only textual data in the XPointer. For instance, given the following document: \mbox{}\newline 
  \textbf{Source.xhtml} \par\bgroup\exampleFont \begin{shaded}\noindent\mbox{}{<\textbf{xhtml:html}>}\mbox{}\newline 
\hspace*{1em}{<\textbf{xhtml:body}>}\mbox{}\newline 
\hspace*{1em}\hspace*{1em}{<\textbf{xhtml:div}>}To make a prairie it takes a {<\textbf{xhtml:a}\hspace*{1em}{href}="{clover.gif}">}clover{</\textbf{xhtml:a}>}\mbox{}\newline 
\hspace*{1em}\hspace*{1em}\hspace*{1em}\hspace*{1em} and one {<\textbf{xhtml:a}\hspace*{1em}{href}="{bee.gif}">}bee{</\textbf{xhtml:a}>},{</\textbf{xhtml:div}>}\mbox{}\newline 
\hspace*{1em}\hspace*{1em}{<\textbf{xhtml:div}>}One {<\textbf{xhtml:a}\hspace*{1em}{href}="{clover.gif}">}clover{</\textbf{xhtml:a}>}, and\mbox{}\newline 
\hspace*{1em}\hspace*{1em}\hspace*{1em}\hspace*{1em} a {<\textbf{xhtml:a}\hspace*{1em}{href}="{bee.gif}">}bee{</\textbf{xhtml:a}>},{</\textbf{xhtml:div}>}\mbox{}\newline 
\hspace*{1em}\hspace*{1em}{<\textbf{xhtml:div}>}And revery.{</\textbf{xhtml:div}>}\mbox{}\newline 
\hspace*{1em}\hspace*{1em}{<\textbf{xhtml:div}>}The revery alone will do,{</\textbf{xhtml:div}>}\mbox{}\newline 
\hspace*{1em}\hspace*{1em}{<\textbf{xhtml:div}>}If bees are few.{</\textbf{xhtml:div}>}\mbox{}\newline 
\hspace*{1em}{</\textbf{xhtml:body}>}\mbox{}\newline 
{</\textbf{xhtml:html}>}\end{shaded}\egroup\par \noindent  the expression \texttt{range(element(/1/2/1.0),element(/1/2/11.1))} will select the whole poem, text content \textit{and} \hyperref[TEI.div]{<div>} elements \textit{and} hypertext links (NB: in XPointer whitespace-only text nodes count).\par
On the contrary, the expressions \texttt{xpointer(//text()/range-to(.))} and \texttt{xpointer(string-range(//text(),"To")/range-to(//text(),"few.")} will only select the text of the poem, with no markup inside.\par
Thus, the following could be a valid stand-off document for the \textit{Source.xhtml} document: \mbox{}\newline 
  \textbf{External2.xml} \par\hfill\bgroup\exampleFont\vskip 10pt\begin{shaded}
\obeyspaces <text xmlns:xi="http://www.w3.org/2001/XInclude">\newline
 <body>\newline
  <head>1755</head>\newline
  <l>\newline
   <xi:include href="Source.xhtml"\newline
 xpointer='xpointer(string-range(//div[1]/text(),"To")/range-to(//div[1]/text(),"bee,")'/>\newline
  </l>\newline
  <l>\newline
   <xi:include href="Source.xhtml"\newline
 xpointer='xpointer(string-range(//div[2]/text(),"One")/range-to(//div[2]/text(),"bee,")'/>\newline
  </l>\newline
  <l>\newline
   <xi:include href="Source.xhtml"\newline
 xpointer='xpointer(string-range(//div[3]/text(),"And")/range-to(//div[3]/text(),".")'/>\newline
  </l>\newline
  <l>\newline
   <xi:include href="Source.xhtml"\newline
 xpointer='xpointer(string-range(//div[4]/text(),"The")/range-to(//div[4]/text(),",")'/>\newline
  </l>\newline
  <l>\newline
   <xi:include href="Source.xhtml"\newline
 xpointer='xpointer(string-range(//div[5]/text(),"If")/range-to(//div[5]/text(),".")'/>\newline
  </l>\newline
 </body>\newline
</text>\end{shaded}
\par\egroup 

\subsection[{The standOff Container}]{The \texttt{<standOff>} Container}\label{SASOstdf}\par
The \hyperref[TEI.standOff]{<standOff>} element is intended to hold content that does not fit well in the \hyperref[TEI.text]{<text>} (e.g. because it is not transcribed from the source), nor in the \hyperref[TEI.teiHeader]{<teiHeader>} (e.g. because it is not metadata about the source or transcription). Examples include contextual information about named entities (typically encoded using \hyperref[TEI.listBibl]{<listBibl>}, \hyperref[TEI.listOrg]{<listOrg>}, \hyperref[TEI.listNym]{<listNym>}, \hyperref[TEI.listPerson]{<listPerson>}, or \hyperref[TEI.listPlace]{<listPlace>}), annotations indicating the morphosyntactic features of a text, and annotations commenting on or associating parts of a text with additional information. 
\begin{sansreflist}
  
\item [\textbf{<standOff>}] Functions as a container element for linked data, contextual information, and stand-off annotations embedded in a TEI document.
\end{sansreflist}
\par
As a member of \textsf{model.resource}, \hyperref[TEI.standOff]{<standOff>} may occur as a child of \hyperref[TEI.TEI]{<TEI>} (or \hyperref[TEI.teiCorpus]{<teiCorpus>}). If the metadata that describes the \hyperref[TEI.standOff]{<standOff>} is largely the same as the metadata that describes the associated resource (e.g., the transcribed text in \hyperref[TEI.text]{<text>}), then the \hyperref[TEI.standOff]{<standOff>} and the encoded associated resource may appear as children of the same \hyperref[TEI.TEI]{<TEI>} element. The example below has a transcription with \texttt{<placename>} elements in the text linked to a list of \hyperref[TEI.place]{<place>} elements in the \hyperref[TEI.standOff]{<standOff>} section. \par\bgroup\index{TEI=<TEI>|exampleindex}\index{teiHeader=<teiHeader>|exampleindex}\index{standOff=<standOff>|exampleindex}\index{listPlace=<listPlace>|exampleindex}\index{place=<place>|exampleindex}\index{placeName=<placeName>|exampleindex}\index{idno=<idno>|exampleindex}\index{type=@type!<idno>|exampleindex}\index{place=<place>|exampleindex}\index{placeName=<placeName>|exampleindex}\index{placeName=<placeName>|exampleindex}\index{idno=<idno>|exampleindex}\index{type=@type!<idno>|exampleindex}\index{place=<place>|exampleindex}\index{placeName=<placeName>|exampleindex}\index{placeName=<placeName>|exampleindex}\index{idno=<idno>|exampleindex}\index{type=@type!<idno>|exampleindex}\index{place=<place>|exampleindex}\index{placeName=<placeName>|exampleindex}\index{idno=<idno>|exampleindex}\index{type=@type!<idno>|exampleindex}\index{text=<text>|exampleindex}\index{body=<body>|exampleindex}\index{div=<div>|exampleindex}\index{type=@type!<div>|exampleindex}\index{div=<div>|exampleindex}\index{type=@type!<div>|exampleindex}\index{n=@n!<div>|exampleindex}\index{head=<head>|exampleindex}\index{p=<p>|exampleindex}\index{n=@n!<p>|exampleindex}\index{seg=<seg>|exampleindex}\index{n=@n!<seg>|exampleindex}\index{persName=<persName>|exampleindex}\index{ref=@ref!<persName>|exampleindex}\index{app=<app>|exampleindex}\index{lem=<lem>|exampleindex}\index{placeName=<placeName>|exampleindex}\index{ref=@ref!<placeName>|exampleindex}\index{rdg=<rdg>|exampleindex}\index{wit=@wit!<rdg>|exampleindex}\index{ana=@ana!<rdg>|exampleindex}\index{placeName=<placeName>|exampleindex}\index{ref=@ref!<placeName>|exampleindex}\index{placeName=<placeName>|exampleindex}\index{ref=@ref!<placeName>|exampleindex}\index{app=<app>|exampleindex}\index{lem=<lem>|exampleindex}\index{placeName=<placeName>|exampleindex}\index{ref=@ref!<placeName>|exampleindex}\index{rdg=<rdg>|exampleindex}\index{wit=@wit!<rdg>|exampleindex}\index{ana=@ana!<rdg>|exampleindex}\index{orgName=<orgName>|exampleindex}\index{ref=@ref!<orgName>|exampleindex}\index{persName=<persName>|exampleindex}\index{ref=@ref!<persName>|exampleindex}\exampleFont \begin{shaded}\noindent\mbox{}{<\textbf{TEI} xmlns="http://www.tei-c.org/ns/1.0">}\mbox{}\newline 
\hspace*{1em}{<\textbf{teiHeader}>}\mbox{}\newline 
\textit{<!-- ... -->}\mbox{}\newline 
\hspace*{1em}{</\textbf{teiHeader}>}\mbox{}\newline 
\hspace*{1em}{<\textbf{standOff}>}\mbox{}\newline 
\hspace*{1em}\hspace*{1em}{<\textbf{listPlace}>}\mbox{}\newline 
\textit{<!-- ... -->}\mbox{}\newline 
\hspace*{1em}\hspace*{1em}\hspace*{1em}{<\textbf{place}\hspace*{1em}{xml:id}="{Cilicia}">}\mbox{}\newline 
\hspace*{1em}\hspace*{1em}\hspace*{1em}\hspace*{1em}{<\textbf{placeName}>}Cilicia{</\textbf{placeName}>}\mbox{}\newline 
\hspace*{1em}\hspace*{1em}\hspace*{1em}\hspace*{1em}{<\textbf{idno}\hspace*{1em}{type}="{URI}">}https://pleiades.stoa.org/places/658440{</\textbf{idno}>}\mbox{}\newline 
\hspace*{1em}\hspace*{1em}\hspace*{1em}{</\textbf{place}>}\mbox{}\newline 
\hspace*{1em}\hspace*{1em}\hspace*{1em}{<\textbf{place}\hspace*{1em}{xml:id}="{Creta}">}\mbox{}\newline 
\hspace*{1em}\hspace*{1em}\hspace*{1em}\hspace*{1em}{<\textbf{placeName}\hspace*{1em}{xml:lang}="{la}">}Creta{</\textbf{placeName}>}\mbox{}\newline 
\hspace*{1em}\hspace*{1em}\hspace*{1em}\hspace*{1em}{<\textbf{placeName}\hspace*{1em}{xml:lang}="{en}">}Crete{</\textbf{placeName}>}\mbox{}\newline 
\hspace*{1em}\hspace*{1em}\hspace*{1em}\hspace*{1em}{<\textbf{idno}\hspace*{1em}{type}="{URI}">}https://pleiades.stoa.org/places/589748{</\textbf{idno}>}\mbox{}\newline 
\hspace*{1em}\hspace*{1em}\hspace*{1em}{</\textbf{place}>}\mbox{}\newline 
\textit{<!-- ... -->}\mbox{}\newline 
\hspace*{1em}\hspace*{1em}\hspace*{1em}{<\textbf{place}\hspace*{1em}{xml:id}="{Rhodus}">}\mbox{}\newline 
\hspace*{1em}\hspace*{1em}\hspace*{1em}\hspace*{1em}{<\textbf{placeName}\hspace*{1em}{xml:lang}="{la}">}Rhodus{</\textbf{placeName}>}\mbox{}\newline 
\hspace*{1em}\hspace*{1em}\hspace*{1em}\hspace*{1em}{<\textbf{placeName}\hspace*{1em}{xml:lang}="{en}">}Rhodes{</\textbf{placeName}>}\mbox{}\newline 
\hspace*{1em}\hspace*{1em}\hspace*{1em}\hspace*{1em}{<\textbf{idno}\hspace*{1em}{type}="{URI}">}https://pleiades.stoa.org/places/590031{</\textbf{idno}>}\mbox{}\newline 
\hspace*{1em}\hspace*{1em}\hspace*{1em}{</\textbf{place}>}\mbox{}\newline 
\hspace*{1em}\hspace*{1em}\hspace*{1em}{<\textbf{place}\hspace*{1em}{xml:id}="{Syria}">}\mbox{}\newline 
\hspace*{1em}\hspace*{1em}\hspace*{1em}\hspace*{1em}{<\textbf{placeName}>}Syria{</\textbf{placeName}>}\mbox{}\newline 
\hspace*{1em}\hspace*{1em}\hspace*{1em}\hspace*{1em}{<\textbf{idno}\hspace*{1em}{type}="{URI}">}https://pleiades.stoa.org/places/1306{</\textbf{idno}>}\mbox{}\newline 
\hspace*{1em}\hspace*{1em}\hspace*{1em}{</\textbf{place}>}\mbox{}\newline 
\textit{<!-- ... -->}\mbox{}\newline 
\hspace*{1em}\hspace*{1em}{</\textbf{listPlace}>}\mbox{}\newline 
\hspace*{1em}{</\textbf{standOff}>}\mbox{}\newline 
\hspace*{1em}{<\textbf{text}>}\mbox{}\newline 
\hspace*{1em}\hspace*{1em}{<\textbf{body}>}\mbox{}\newline 
\hspace*{1em}\hspace*{1em}\hspace*{1em}{<\textbf{div}\hspace*{1em}{type}="{edition}"\mbox{}\newline 
\hspace*{1em}\hspace*{1em}\hspace*{1em}\hspace*{1em}{xml:id}="{edition-text}">}\mbox{}\newline 
\hspace*{1em}\hspace*{1em}\hspace*{1em}\hspace*{1em}{<\textbf{div}\hspace*{1em}{type}="{textpart}"\hspace*{1em}{n}="{1}"\mbox{}\newline 
\hspace*{1em}\hspace*{1em}\hspace*{1em}\hspace*{1em}\hspace*{1em}{xml:id}="{part1}">}\mbox{}\newline 
\hspace*{1em}\hspace*{1em}\hspace*{1em}\hspace*{1em}\hspace*{1em}{<\textbf{head}>}Bellum Alexandrinum{</\textbf{head}>}\mbox{}\newline 
\hspace*{1em}\hspace*{1em}\hspace*{1em}\hspace*{1em}\hspace*{1em}{<\textbf{p}\hspace*{1em}{n}="{1}"\hspace*{1em}{xml:id}="{p1}">}\mbox{}\newline 
\hspace*{1em}\hspace*{1em}\hspace*{1em}\hspace*{1em}\hspace*{1em}\hspace*{1em}{<\textbf{seg}\hspace*{1em}{n}="{1}"\hspace*{1em}{xml:id}="{seg-1.1}">}Bello Alexandrino conflato {<\textbf{persName}\hspace*{1em}{ref}="{\#Caesar}">}Caesar{</\textbf{persName}>}\mbox{}\newline 
\hspace*{1em}\hspace*{1em}\hspace*{1em}\hspace*{1em}\hspace*{1em}\hspace*{1em}\hspace*{1em}{<\textbf{app}>}\mbox{}\newline 
\hspace*{1em}\hspace*{1em}\hspace*{1em}\hspace*{1em}\hspace*{1em}\hspace*{1em}\hspace*{1em}\hspace*{1em}{<\textbf{lem}>}\mbox{}\newline 
\hspace*{1em}\hspace*{1em}\hspace*{1em}\hspace*{1em}\hspace*{1em}\hspace*{1em}\hspace*{1em}\hspace*{1em}\hspace*{1em}{<\textbf{placeName}\hspace*{1em}{ref}="{\#Rhodus}">}Rhodo{</\textbf{placeName}>}\mbox{}\newline 
\hspace*{1em}\hspace*{1em}\hspace*{1em}\hspace*{1em}\hspace*{1em}\hspace*{1em}\hspace*{1em}\hspace*{1em}{</\textbf{lem}>}\mbox{}\newline 
\hspace*{1em}\hspace*{1em}\hspace*{1em}\hspace*{1em}\hspace*{1em}\hspace*{1em}\hspace*{1em}\hspace*{1em}{<\textbf{rdg}\hspace*{1em}{wit}="{\#S}"\hspace*{1em}{ana}="{\#orthographical}">}Ordo{</\textbf{rdg}>}\mbox{}\newline 
\hspace*{1em}\hspace*{1em}\hspace*{1em}\hspace*{1em}\hspace*{1em}\hspace*{1em}\hspace*{1em}{</\textbf{app}>} atque ex {<\textbf{placeName}\hspace*{1em}{ref}="{\#Syria}">}Syria{</\textbf{placeName}>}\mbox{}\newline 
\hspace*{1em}\hspace*{1em}\hspace*{1em}\hspace*{1em}\hspace*{1em}\hspace*{1em}\hspace*{1em}{<\textbf{placeName}\hspace*{1em}{ref}="{\#Cilicia}">}Cilicia{</\textbf{placeName}>}que omnem classem arcessit; {<\textbf{app}>}\mbox{}\newline 
\hspace*{1em}\hspace*{1em}\hspace*{1em}\hspace*{1em}\hspace*{1em}\hspace*{1em}\hspace*{1em}\hspace*{1em}{<\textbf{lem}>}\mbox{}\newline 
\hspace*{1em}\hspace*{1em}\hspace*{1em}\hspace*{1em}\hspace*{1em}\hspace*{1em}\hspace*{1em}\hspace*{1em}\hspace*{1em}{<\textbf{placeName}\hspace*{1em}{ref}="{\#Creta}">}Creta{</\textbf{placeName}>}\mbox{}\newline 
\hspace*{1em}\hspace*{1em}\hspace*{1em}\hspace*{1em}\hspace*{1em}\hspace*{1em}\hspace*{1em}\hspace*{1em}{</\textbf{lem}>}\mbox{}\newline 
\hspace*{1em}\hspace*{1em}\hspace*{1em}\hspace*{1em}\hspace*{1em}\hspace*{1em}\hspace*{1em}\hspace*{1em}{<\textbf{rdg}\hspace*{1em}{wit}="{\#S}"\hspace*{1em}{ana}="{\#orthographical}">}certa{</\textbf{rdg}>}\mbox{}\newline 
\hspace*{1em}\hspace*{1em}\hspace*{1em}\hspace*{1em}\hspace*{1em}\hspace*{1em}\hspace*{1em}{</\textbf{app}>} sagittarios, equites ab rege {<\textbf{orgName}\hspace*{1em}{ref}="{\#Nabataei}">}Nabataeorum{</\textbf{orgName}>}\mbox{}\newline 
\hspace*{1em}\hspace*{1em}\hspace*{1em}\hspace*{1em}\hspace*{1em}\hspace*{1em}\hspace*{1em}{<\textbf{persName}\hspace*{1em}{ref}="{\#Malchus}">}Malcho{</\textbf{persName}>} euocat; tormenta undique conquiri et\mbox{}\newline 
\hspace*{1em}\hspace*{1em}\hspace*{1em}\hspace*{1em}\hspace*{1em}\hspace*{1em}\hspace*{1em}\hspace*{1em}\hspace*{1em}\hspace*{1em}\hspace*{1em}\hspace*{1em} frumentum mitti, auxilia adduci iubet.{</\textbf{seg}>}\mbox{}\newline 
\textit{<!-- ... -->}\mbox{}\newline 
\hspace*{1em}\hspace*{1em}\hspace*{1em}\hspace*{1em}\hspace*{1em}{</\textbf{p}>}\mbox{}\newline 
\textit{<!-- ... -->}\mbox{}\newline 
\hspace*{1em}\hspace*{1em}\hspace*{1em}\hspace*{1em}{</\textbf{div}>}\mbox{}\newline 
\textit{<!-- ... -->}\mbox{}\newline 
\hspace*{1em}\hspace*{1em}\hspace*{1em}{</\textbf{div}>}\mbox{}\newline 
\hspace*{1em}\hspace*{1em}{</\textbf{body}>}\mbox{}\newline 
\hspace*{1em}{</\textbf{text}>}\mbox{}\newline 
{</\textbf{TEI}>}\end{shaded}\egroup\par 
\subsection[{Annotations}]{Annotations}\label{SASOann}\par
The \hyperref[TEI.annotation]{<annotation>} element's structure is based on the \hyperref[WADM]{Web Annotation Data Model} (WADM). A Web Annotation may have one or more targets, which are URIs, and zero or more bodies, which may be either URIs or embedded text. A Web Annotation may also contain metadata about the annotation, such as the creator, creation and modification dates, and license information. The \hyperref[TEI.annotation]{<annotation>} element implements a subset of WADM, using TEI elements and attributes to encode the same information, with a focus on annotating TEI documents. Targets are represented using the {\itshape target} attribute on \hyperref[TEI.annotation]{<annotation>}. URI bodies are represented using \hyperref[TEI.ref]{<ref>} or \hyperref[TEI.ptr]{<ptr>} and embedded text bodies using \hyperref[TEI.note]{<note>}. Lifecycle and license information may be given using \hyperref[TEI.respStmt]{<respStmt>}, \hyperref[TEI.revisionDesc]{<revisionDesc>}, and \hyperref[TEI.licence]{<licence>}. 
\begin{sansreflist}
  
\item [\textbf{<listAnnotation>}] contains a list of annotations, typically encoded as \hyperref[TEI.annotation]{<annotation>}, \hyperref[TEI.annotationBlock]{<annotationBlock>}, or \hyperref[TEI.note]{<note>}, possibly organized with nested \hyperref[TEI.listAnnotation]{<listAnnotation>} elements.
\item [\textbf{<annotation>}] represents an annotation following the \hyperref[WADM]{Web Annotation Data Model}.
\end{sansreflist}
\par
TEI annotations are, in general, intended to capture the output of processes that annotate TEI texts without altering the text and markup in the \hyperref[TEI.text]{<text>}. They allow this kind of output to be represented directly in TEI, and thus to be processed using the same toolchains as the texts they annotate. A named entity recognition workflow might use \hyperref[TEI.annotation]{<annotation>}, for example to associate names in the text with \hyperref[TEI.person]{<person>} elements instead of attempting to rewrite the TEI text using inline \hyperref[TEI.persName]{<persName>}. Projects may wish to use this mechanism to layer information onto a TEI text in case where using inline elements might result in complicated markup.\par
The example below illustrates how stand-off annotations can be used to connect words in a text with \hyperref[TEI.place]{<place>} elements in a list. The words ‘Gallia’ and ‘Galliae’ in the edition are connected by an annotation in the \hyperref[TEI.standOff]{<standOff>} section which points to them using \texttt{string-range()} pointers and references the definition of the place (also in the \hyperref[TEI.standOff]{<standOff>} section). If the set of annotations were created in a process separate from the creation of the transcription and then imported into the transcription document, then wrapping them in a \hyperref[TEI.TEI]{<TEI>} element with its own \hyperref[TEI.teiHeader]{<teiHeader>} providing metadata for the annotations might be advisable. \par\bgroup\index{TEI=<TEI>|exampleindex}\index{teiHeader=<teiHeader>|exampleindex}\index{fileDesc=<fileDesc>|exampleindex}\index{titleStmt=<titleStmt>|exampleindex}\index{title=<title>|exampleindex}\index{text=<text>|exampleindex}\index{body=<body>|exampleindex}\index{div=<div>|exampleindex}\index{type=@type!<div>|exampleindex}\index{div=<div>|exampleindex}\index{type=@type!<div>|exampleindex}\index{subtype=@subtype!<div>|exampleindex}\index{n=@n!<div>|exampleindex}\index{p=<p>|exampleindex}\index{n=@n!<p>|exampleindex}\index{seg=<seg>|exampleindex}\index{n=@n!<seg>|exampleindex}\index{seg=<seg>|exampleindex}\index{n=@n!<seg>|exampleindex}\index{seg=<seg>|exampleindex}\index{n=@n!<seg>|exampleindex}\index{standOff=<standOff>|exampleindex}\index{listPlace=<listPlace>|exampleindex}\index{place=<place>|exampleindex}\index{placeName=<placeName>|exampleindex}\index{placeName=<placeName>|exampleindex}\index{idno=<idno>|exampleindex}\index{type=@type!<idno>|exampleindex}\index{TEI=<TEI>|exampleindex}\index{teiHeader=<teiHeader>|exampleindex}\index{standOff=<standOff>|exampleindex}\index{listAnnotation=<listAnnotation>|exampleindex}\index{annotation=<annotation>|exampleindex}\index{motivation=@motivation!<annotation>|exampleindex}\index{target=@target!<annotation>|exampleindex}\index{respStmt=<respStmt>|exampleindex}\index{resp=<resp>|exampleindex}\index{persName=<persName>|exampleindex}\index{revisionDesc=<revisionDesc>|exampleindex}\index{change=<change>|exampleindex}\index{status=@status!<change>|exampleindex}\index{when=@when!<change>|exampleindex}\index{who=@who!<change>|exampleindex}\index{change=<change>|exampleindex}\index{status=@status!<change>|exampleindex}\index{when=@when!<change>|exampleindex}\index{who=@who!<change>|exampleindex}\index{licence=<licence>|exampleindex}\index{target=@target!<licence>|exampleindex}\index{ptr=<ptr>|exampleindex}\index{target=@target!<ptr>|exampleindex}\index{ptr=<ptr>|exampleindex}\index{target=@target!<ptr>|exampleindex}\exampleFont \begin{shaded}\noindent\mbox{}{<\textbf{TEI} xmlns="http://www.tei-c.org/ns/1.0">}\mbox{}\newline 
\hspace*{1em}{<\textbf{teiHeader}>}\mbox{}\newline 
\hspace*{1em}\hspace*{1em}{<\textbf{fileDesc}>}\mbox{}\newline 
\hspace*{1em}\hspace*{1em}\hspace*{1em}{<\textbf{titleStmt}>}\mbox{}\newline 
\hspace*{1em}\hspace*{1em}\hspace*{1em}\hspace*{1em}{<\textbf{title}>}De Bello Gallico{</\textbf{title}>}\mbox{}\newline 
\hspace*{1em}\hspace*{1em}\hspace*{1em}{</\textbf{titleStmt}>}\mbox{}\newline 
\textit{<!-- ... -->}\mbox{}\newline 
\hspace*{1em}\hspace*{1em}{</\textbf{fileDesc}>}\mbox{}\newline 
\hspace*{1em}{</\textbf{teiHeader}>}\mbox{}\newline 
\hspace*{1em}{<\textbf{text}>}\mbox{}\newline 
\hspace*{1em}\hspace*{1em}{<\textbf{body}>}\mbox{}\newline 
\hspace*{1em}\hspace*{1em}\hspace*{1em}{<\textbf{div}\hspace*{1em}{type}="{edition}">}\mbox{}\newline 
\hspace*{1em}\hspace*{1em}\hspace*{1em}\hspace*{1em}{<\textbf{div}\hspace*{1em}{type}="{textpart}"\hspace*{1em}{subtype}="{chapter}"\mbox{}\newline 
\hspace*{1em}\hspace*{1em}\hspace*{1em}\hspace*{1em}\hspace*{1em}{n}="{1}"\hspace*{1em}{xml:id}="{ch1}">}\mbox{}\newline 
\hspace*{1em}\hspace*{1em}\hspace*{1em}\hspace*{1em}\hspace*{1em}{<\textbf{p}\hspace*{1em}{n}="{1}"\hspace*{1em}{xml:id}="{c1p1}">}\mbox{}\newline 
\hspace*{1em}\hspace*{1em}\hspace*{1em}\hspace*{1em}\hspace*{1em}\hspace*{1em}{<\textbf{seg}\hspace*{1em}{n}="{1}"\hspace*{1em}{xml:id}="{c1p1s1}">}Gallia est omnis divisa in partes tres, quarum unam\mbox{}\newline 
\hspace*{1em}\hspace*{1em}\hspace*{1em}\hspace*{1em}\hspace*{1em}\hspace*{1em}\hspace*{1em}\hspace*{1em}\hspace*{1em}\hspace*{1em}\hspace*{1em}\hspace*{1em} incolunt Belgae, aliam Aquitani, tertiam qui ipsorum lingua Celtae, nostra\mbox{}\newline 
\hspace*{1em}\hspace*{1em}\hspace*{1em}\hspace*{1em}\hspace*{1em}\hspace*{1em}\hspace*{1em}\hspace*{1em}\hspace*{1em}\hspace*{1em}\hspace*{1em}\hspace*{1em} Galli appellantur.{</\textbf{seg}>}\mbox{}\newline 
\textit{<!-- ... -->}\mbox{}\newline 
\hspace*{1em}\hspace*{1em}\hspace*{1em}\hspace*{1em}\hspace*{1em}\hspace*{1em}{<\textbf{seg}\hspace*{1em}{n}="{6}"\hspace*{1em}{xml:id}="{c1p1s6}">}Belgae ab extremis Galliae finibus oriuntur,\mbox{}\newline 
\hspace*{1em}\hspace*{1em}\hspace*{1em}\hspace*{1em}\hspace*{1em}\hspace*{1em}\hspace*{1em}\hspace*{1em}\hspace*{1em}\hspace*{1em}\hspace*{1em}\hspace*{1em} pertinent ad inferiorem partem fluminis Rheni, spectant in septentrionem et\mbox{}\newline 
\hspace*{1em}\hspace*{1em}\hspace*{1em}\hspace*{1em}\hspace*{1em}\hspace*{1em}\hspace*{1em}\hspace*{1em}\hspace*{1em}\hspace*{1em}\hspace*{1em}\hspace*{1em} orientem solem.{</\textbf{seg}>}\mbox{}\newline 
\hspace*{1em}\hspace*{1em}\hspace*{1em}\hspace*{1em}\hspace*{1em}\hspace*{1em}{<\textbf{seg}\hspace*{1em}{n}="{7}"\hspace*{1em}{xml:id}="{c1p1s7}">}Aquitania a Garumna flumine ad Pyrenaeos montes et\mbox{}\newline 
\hspace*{1em}\hspace*{1em}\hspace*{1em}\hspace*{1em}\hspace*{1em}\hspace*{1em}\hspace*{1em}\hspace*{1em}\hspace*{1em}\hspace*{1em}\hspace*{1em}\hspace*{1em} eam partem Oceani quae est ad Hispaniam pertinet; spectat inter occasum solis\mbox{}\newline 
\hspace*{1em}\hspace*{1em}\hspace*{1em}\hspace*{1em}\hspace*{1em}\hspace*{1em}\hspace*{1em}\hspace*{1em}\hspace*{1em}\hspace*{1em}\hspace*{1em}\hspace*{1em} et septentriones.]{</\textbf{seg}>}\mbox{}\newline 
\textit{<!-- ... -->}\mbox{}\newline 
\hspace*{1em}\hspace*{1em}\hspace*{1em}\hspace*{1em}\hspace*{1em}{</\textbf{p}>}\mbox{}\newline 
\textit{<!-- ... -->}\mbox{}\newline 
\hspace*{1em}\hspace*{1em}\hspace*{1em}\hspace*{1em}{</\textbf{div}>}\mbox{}\newline 
\textit{<!-- Ch. 2 etc. ... -->}\mbox{}\newline 
\hspace*{1em}\hspace*{1em}\hspace*{1em}{</\textbf{div}>}\mbox{}\newline 
\hspace*{1em}\hspace*{1em}{</\textbf{body}>}\mbox{}\newline 
\hspace*{1em}{</\textbf{text}>}\mbox{}\newline 
\hspace*{1em}{<\textbf{standOff}>}\mbox{}\newline 
\hspace*{1em}\hspace*{1em}{<\textbf{listPlace}>}\mbox{}\newline 
\hspace*{1em}\hspace*{1em}\hspace*{1em}{<\textbf{place}\hspace*{1em}{xml:id}="{Gallia}">}\mbox{}\newline 
\hspace*{1em}\hspace*{1em}\hspace*{1em}\hspace*{1em}{<\textbf{placeName}\hspace*{1em}{xml:lang}="{la}">}Gallia{</\textbf{placeName}>}\mbox{}\newline 
\hspace*{1em}\hspace*{1em}\hspace*{1em}\hspace*{1em}{<\textbf{placeName}\hspace*{1em}{xml:lang}="{en}">}Gaul{</\textbf{placeName}>}\mbox{}\newline 
\hspace*{1em}\hspace*{1em}\hspace*{1em}\hspace*{1em}{<\textbf{idno}\hspace*{1em}{type}="{URI}">}https://pleiades.stoa.org/places/993{</\textbf{idno}>}\mbox{}\newline 
\hspace*{1em}\hspace*{1em}\hspace*{1em}{</\textbf{place}>}\mbox{}\newline 
\textit{<!-- ... -->}\mbox{}\newline 
\hspace*{1em}\hspace*{1em}{</\textbf{listPlace}>}\mbox{}\newline 
\hspace*{1em}{</\textbf{standOff}>}\mbox{}\newline 
\hspace*{1em}{<\textbf{TEI} xmlns="http://www.tei-c.org/ns/1.0">}\mbox{}\newline 
\hspace*{1em}\hspace*{1em}{<\textbf{teiHeader}>}\mbox{}\newline 
\textit{<!-- Metadata applying just to the set of annotations -->}\mbox{}\newline 
\hspace*{1em}\hspace*{1em}{</\textbf{teiHeader}>}\mbox{}\newline 
\hspace*{1em}\hspace*{1em}{<\textbf{standOff}>}\mbox{}\newline 
\hspace*{1em}\hspace*{1em}\hspace*{1em}{<\textbf{listAnnotation}>}\mbox{}\newline 
\hspace*{1em}\hspace*{1em}\hspace*{1em}\hspace*{1em}{<\textbf{annotation}\hspace*{1em}{xml:id}="{ann01}"\mbox{}\newline 
\hspace*{1em}\hspace*{1em}\hspace*{1em}\hspace*{1em}\hspace*{1em}{motivation}="{linking}"\hspace*{1em}{target}="{\#Gallia}">}\mbox{}\newline 
\hspace*{1em}\hspace*{1em}\hspace*{1em}\hspace*{1em}\hspace*{1em}{<\textbf{respStmt}\hspace*{1em}{xml:id}="{ed}">}\mbox{}\newline 
\hspace*{1em}\hspace*{1em}\hspace*{1em}\hspace*{1em}\hspace*{1em}\hspace*{1em}{<\textbf{resp}>}creator{</\textbf{resp}>}\mbox{}\newline 
\hspace*{1em}\hspace*{1em}\hspace*{1em}\hspace*{1em}\hspace*{1em}\hspace*{1em}{<\textbf{persName}>}Fred Editor{</\textbf{persName}>}\mbox{}\newline 
\hspace*{1em}\hspace*{1em}\hspace*{1em}\hspace*{1em}\hspace*{1em}{</\textbf{respStmt}>}\mbox{}\newline 
\hspace*{1em}\hspace*{1em}\hspace*{1em}\hspace*{1em}\hspace*{1em}{<\textbf{revisionDesc}>}\mbox{}\newline 
\hspace*{1em}\hspace*{1em}\hspace*{1em}\hspace*{1em}\hspace*{1em}\hspace*{1em}{<\textbf{change}\hspace*{1em}{status}="{created}"\mbox{}\newline 
\hspace*{1em}\hspace*{1em}\hspace*{1em}\hspace*{1em}\hspace*{1em}\hspace*{1em}\hspace*{1em}{when}="{2020-05-21T13:59:00Z}"\hspace*{1em}{who}="{\#ed}"/>}\mbox{}\newline 
\hspace*{1em}\hspace*{1em}\hspace*{1em}\hspace*{1em}\hspace*{1em}\hspace*{1em}{<\textbf{change}\hspace*{1em}{status}="{modified}"\mbox{}\newline 
\hspace*{1em}\hspace*{1em}\hspace*{1em}\hspace*{1em}\hspace*{1em}\hspace*{1em}\hspace*{1em}{when}="{2020-05-21T19:48:00Z}"\hspace*{1em}{who}="{\#ed}"/>}\mbox{}\newline 
\hspace*{1em}\hspace*{1em}\hspace*{1em}\hspace*{1em}\hspace*{1em}{</\textbf{revisionDesc}>}\mbox{}\newline 
\hspace*{1em}\hspace*{1em}\hspace*{1em}\hspace*{1em}\hspace*{1em}{<\textbf{licence}\hspace*{1em}{target}="{http://creativecommons.org/licenses/by/3.0/}"/>}\mbox{}\newline 
\textit{<!-- Gallia in seg 1 -->}\mbox{}\newline 
\hspace*{1em}\hspace*{1em}\hspace*{1em}\hspace*{1em}\hspace*{1em}{<\textbf{ptr}\hspace*{1em}{target}="{\#string-range(c1p1s1,0,6)}"/>}\mbox{}\newline 
\textit{<!-- Galliae in seg 6 -->}\mbox{}\newline 
\hspace*{1em}\hspace*{1em}\hspace*{1em}\hspace*{1em}\hspace*{1em}{<\textbf{ptr}\hspace*{1em}{target}="{\#string-range(c1p1s6,19,7)}"/>}\mbox{}\newline 
\hspace*{1em}\hspace*{1em}\hspace*{1em}\hspace*{1em}{</\textbf{annotation}>}\mbox{}\newline 
\textit{<!-- ... -->}\mbox{}\newline 
\hspace*{1em}\hspace*{1em}\hspace*{1em}{</\textbf{listAnnotation}>}\mbox{}\newline 
\hspace*{1em}\hspace*{1em}{</\textbf{standOff}>}\mbox{}\newline 
\hspace*{1em}{</\textbf{TEI}>}\mbox{}\newline 
{</\textbf{TEI}>}\end{shaded}\egroup\par 
\subsection[{Connecting Analytic and Textual Markup}]{Connecting Analytic and Textual Markup}\label{SAAN}\par
In chapters \textit{\hyperref[AI]{17.\ Simple Analytic Mechanisms}} and \textit{\hyperref[FS]{18.\ Feature Structures}} and elsewhere, provision is made for analytic and interpretive markup to be represented outside of textual markup, either in the same document or in a different document. The elements in these separate domains can be connected, either with the pointing attributes {\itshape ana} (for \textit{analysis}) and {\itshape inst} (for \textit{instance}), or by means of \hyperref[TEI.link]{<link>} and \hyperref[TEI.linkGrp]{<linkGrp>} elements. Numerous examples are given in these chapters.\par
Another more specific form of annotation is available through the TEI \hyperref[TEI.ruby]{<ruby>} element and its children, described in detail in \textit{\hyperref[COHTGRB]{3.4.2.\ Ruby Annotations}}.
\subsection[{Module for Linking, Segmentation, and Alignment}]{Module for Linking, Segmentation, and Alignment}\label{SAref}\par
The module described in this chapter makes available the following components: \begin{description}

\item[{Module linking: Linking, segmentation, and alignment}]\hspace{1em}\hfill\linebreak
\mbox{}\\[-10pt] \begin{itemize}
\item {\itshape Elements defined}: \hyperref[TEI.ab]{ab} \hyperref[TEI.alt]{alt} \hyperref[TEI.altGrp]{altGrp} \hyperref[TEI.anchor]{anchor} \hyperref[TEI.annotation]{annotation} \hyperref[TEI.join]{join} \hyperref[TEI.joinGrp]{joinGrp} \hyperref[TEI.link]{link} \hyperref[TEI.linkGrp]{linkGrp} \hyperref[TEI.listAnnotation]{listAnnotation} \hyperref[TEI.seg]{seg} \hyperref[TEI.standOff]{standOff} \hyperref[TEI.timeline]{timeline} \hyperref[TEI.when]{when}
\item {\itshape Classes defined}: \hyperref[TEI.att.global.linking]{att.global.linking}
\end{itemize} 
\end{description}  The selection and combination of modules to form a TEI schema is described in \textit{\hyperref[STIN]{1.2.\ Defining a TEI Schema}}.