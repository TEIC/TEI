
\section[{Default Text Structure}]{Default Text Structure}\label{DS}\par
This chapter describes the default high-level structure for TEI documents. A full TEI document combines metadata describing it, represented by a \hyperref[TEI.teiHeader]{<teiHeader>} element, with the document itself, represented by one or more \hyperref[TEI.text]{<text>} elements or other elements taken from the \textsf{model.resource} class. That is, the \hyperref[TEI.TEI]{<TEI>} element is used to group together metadata about an encoded resource (in \hyperref[TEI.teiHeader]{<teiHeader>}, specified by the \textsf{header} module, which is fully described in chapter \textit{\hyperref[HD]{2.\ The TEI Header}}) with an encoded resource. Possible encoded resources are \begin{itemize}
\item a logical transcription of a source document in a \hyperref[TEI.text]{<text>} element; the \hyperref[TEI.text]{<text>} element is specified along with its high-level constituents in the \textsf{textstructure} module and described in the remainder of the current chapter
\item a diplomatic transcription of a source document in a \hyperref[TEI.sourceDoc]{<sourceDoc>} element, which is specified in the \textsf{transcr} module and described in chapter \textit{\hyperref[PH]{11.\ Representation of Primary Sources}}
\item an encoded representation of a text-bearing object as images in a \hyperref[TEI.facsimile]{<facsimile>} element, which is also specified in the \textsf{transcr} module and described in chapter \textit{\hyperref[PH]{11.\ Representation of Primary Sources}}
\item a collection of contextual information or annotations that provides more detail about another encoded resource (whether in the same or a different TEI document) in a \hyperref[TEI.standOff]{<standOff>} element, which is specified in the \textsf{linking} module and described in section \textit{\hyperref[SASOstdf]{16.10.\ The standOff Container}}
\item a feature system declaration which can be used to declare the use of \hyperref[TEI.fs]{<fs>} elements in the rest of the document, which is specified in the \textsf{iso-fs} module and described in section \textit{\hyperref[FD]{18.11.\ Feature System Declaration}}
\end{itemize} \par
In a case in which more than one resource related to the same source document share the same metadata, they may be grouped together in a \hyperref[TEI.TEI]{<TEI>} element following a single \hyperref[TEI.teiHeader]{<teiHeader>}.\par
Because the \hyperref[TEI.TEI]{<TEI>} can be a child of itself, a set or collection of documents may be represented by an outermost \hyperref[TEI.TEI]{<TEI>} element that contains a \hyperref[TEI.teiHeader]{<teiHeader>} with metadata that is applicable to the entire set or collection of transcriptions, and then a complete \hyperref[TEI.TEI]{<TEI>} element for each document in the collection or set; each of these \hyperref[TEI.TEI]{<TEI>} elements contains a \hyperref[TEI.teiHeader]{<teiHeader>} with metadata that is applicable to the individual document, and one or more \hyperref[TEI.text]{<text>} or other elements taken from the \textsf{model.resource} class.\par
A variant on this basic form, the \hyperref[TEI.teiCorpus]{<teiCorpus>}, is also defined for the representation of language corpora, or other collections of encoded texts. A \hyperref[TEI.teiCorpus]{<teiCorpus>} consists of its own metadata in a \hyperref[TEI.teiHeader]{<teiHeader>}, followed by one or more complete \hyperref[TEI.TEI]{<TEI>} elements, each combining a \hyperref[TEI.teiHeader]{<teiHeader>} with one or more elements from the \textsf{model.resource} class. This permits the encoder to distinguish metadata applicable to the whole collection of encoded texts, which is represented by the outermost \hyperref[TEI.teiHeader]{<teiHeader>}, from that applicable to each of the individual \hyperref[TEI.TEI]{<TEI>} elements within the corpus. Further information about the organization and encoding of language corpora is given in chapter \textit{\hyperref[CC]{15.\ Language Corpora}}.\par
Alternatively, the corpus may be represented with a \hyperref[TEI.TEI]{<TEI>} element (perhaps with a {\itshape type} of corpus) in the same manner as a \hyperref[TEI.teiCorpus]{<teiCorpus>}.\par
In summary, when the default structure module is included in a schema, the following elements are available for the representation of the outermost structure of a TEI document: 
\begin{sansreflist}
  
\item [\textbf{<TEI>}] (TEI document) contains a single TEI-conformant document, combining a single TEI header with one or more members of the \textsf{model.resource} class. Multiple \hyperref[TEI.TEI]{<TEI>} elements may be combined within a \hyperref[TEI.TEI]{<TEI>} (or \hyperref[TEI.teiCorpus]{<teiCorpus>}) element.\hfil\\[-10pt]\begin{sansreflist}
    \item[@{\itshape version}]
  specifies the version number of the TEI Guidelines against which this document is valid.
\end{sansreflist}  
\item [\textbf{<teiCorpus>}] (TEI corpus) contains the whole of a TEI encoded corpus, comprising a single corpus header and one or more \hyperref[TEI.TEI]{<TEI>} elements, each containing a single text header and a text.
\item [\textbf{<teiHeader>}] (TEI header) supplies descriptive and declarative metadata associated with a digital resource or set of resources.
\item [\textbf{<text>}] (text) contains a single text of any kind, whether unitary or composite, for example a poem or drama, a collection of essays, a novel, a dictionary, or a corpus sample.
\end{sansreflist}
 As noted above, the \hyperref[TEI.teiHeader]{<teiHeader>} element is formally declared in the \textsf{header} module (see chapter \textit{\hyperref[HD]{2.\ The TEI Header}}). A TEI document may also contain elements from the \textsf{model.resource} class (such as a collection of facsimile images, or a feature system declaration) if the appropriate module is included in a schema (see further \textit{\hyperref[PHFAX]{11.1.\ Digital Facsimiles}} and \textit{\hyperref[FD]{18.11.\ Feature System Declaration}} respectively). By default, however, this class is not populated and hence only the elements \hyperref[TEI.TEI]{<TEI>}, \hyperref[TEI.text]{<text>}, and \hyperref[TEI.teiCorpus]{<teiCorpus>} are available as major parts of a TEI document. These three elements are provided by the \textsf{textstructure} module described by the present chapter.\par
TEI texts may be regarded either as \textit{unitary}, that is, forming an organic whole, or as \textit{composite,} that is, consisting of several components which are in some important sense independent of each other. The distinction is not always entirely obvious: for example a collection of essays might be regarded as a single item in some circumstances, or as a number of distinct items in others. In such borderline cases, the encoder must choose whether to treat the text as unitary or composite; each may have advantages and disadvantages in a given situation.\par
Whether unitary or composite, the text is marked with the \hyperref[TEI.text]{<text>} tag and may contain front matter, a text body, and back matter. In unitary texts, the text body is tagged \hyperref[TEI.body]{<body>}; in composite texts, where the text body consists of a series of subordinate texts or groups, it is tagged \hyperref[TEI.group]{<group>}. The overall structure of any text, unitary or composite, is thus defined by the following elements: 
\begin{sansreflist}
  
\item [\textbf{<front>}] (front matter) contains any prefatory matter (headers, abstracts, title page, prefaces, dedications, etc.) found at the start of a document, before the main body.
\item [\textbf{<body>}] (text body) contains the whole body of a single unitary text, excluding any front or back matter.
\item [\textbf{<group>}] (group) contains the body of a composite text, grouping together a sequence of distinct texts (or groups of such texts) which are regarded as a unit for some purpose, for example the collected works of an author, a sequence of prose essays, etc.
\item [\textbf{<back>}] (back matter) contains any appendixes, etc. following the main part of a text.
\end{sansreflist}
\par
The overall structure of a unitary text is: \par\bgroup\index{TEI=<TEI>|exampleindex}\index{teiHeader=<teiHeader>|exampleindex}\index{text=<text>|exampleindex}\index{front=<front>|exampleindex}\index{body=<body>|exampleindex}\index{back=<back>|exampleindex}\exampleFont \begin{shaded}\noindent\mbox{}{<\textbf{TEI} xmlns="http://www.tei-c.org/ns/1.0">}\mbox{}\newline 
\hspace*{1em}{<\textbf{teiHeader}>}\mbox{}\newline 
\textit{<!-- ... -->}\mbox{}\newline 
\hspace*{1em}{</\textbf{teiHeader}>}\mbox{}\newline 
\hspace*{1em}{<\textbf{text}>}\mbox{}\newline 
\hspace*{1em}\hspace*{1em}{<\textbf{front}>}\mbox{}\newline 
\textit{<!-- front matter of copy text, if any, goes here -->}\mbox{}\newline 
\hspace*{1em}\hspace*{1em}{</\textbf{front}>}\mbox{}\newline 
\hspace*{1em}\hspace*{1em}{<\textbf{body}>}\mbox{}\newline 
\textit{<!-- body of copy text goes here -->}\mbox{}\newline 
\hspace*{1em}\hspace*{1em}{</\textbf{body}>}\mbox{}\newline 
\hspace*{1em}\hspace*{1em}{<\textbf{back}>}\mbox{}\newline 
\textit{<!-- back matter of copy text, if any, goes here -->}\mbox{}\newline 
\hspace*{1em}\hspace*{1em}{</\textbf{back}>}\mbox{}\newline 
\hspace*{1em}{</\textbf{text}>}\mbox{}\newline 
{</\textbf{TEI}>}\end{shaded}\egroup\par \par
The overall structure of a composite text made up of two unitary texts is: \par\bgroup\index{TEI=<TEI>|exampleindex}\index{teiHeader=<teiHeader>|exampleindex}\index{text=<text>|exampleindex}\index{front=<front>|exampleindex}\index{group=<group>|exampleindex}\index{text=<text>|exampleindex}\index{front=<front>|exampleindex}\index{body=<body>|exampleindex}\index{back=<back>|exampleindex}\index{text=<text>|exampleindex}\index{body=<body>|exampleindex}\index{back=<back>|exampleindex}\exampleFont \begin{shaded}\noindent\mbox{}{<\textbf{TEI} xmlns="http://www.tei-c.org/ns/1.0">}\mbox{}\newline 
\hspace*{1em}{<\textbf{teiHeader}>}\mbox{}\newline 
\textit{<!-- ... -->}\mbox{}\newline 
\hspace*{1em}{</\textbf{teiHeader}>}\mbox{}\newline 
\hspace*{1em}{<\textbf{text}>}\mbox{}\newline 
\hspace*{1em}\hspace*{1em}{<\textbf{front}>}\mbox{}\newline 
\textit{<!-- front matter for composite text -->}\mbox{}\newline 
\hspace*{1em}\hspace*{1em}{</\textbf{front}>}\mbox{}\newline 
\hspace*{1em}\hspace*{1em}{<\textbf{group}>}\mbox{}\newline 
\hspace*{1em}\hspace*{1em}\hspace*{1em}{<\textbf{text}>}\mbox{}\newline 
\hspace*{1em}\hspace*{1em}\hspace*{1em}\hspace*{1em}{<\textbf{front}>}\mbox{}\newline 
\textit{<!-- front matter of first unitary text, if any -->}\mbox{}\newline 
\hspace*{1em}\hspace*{1em}\hspace*{1em}\hspace*{1em}{</\textbf{front}>}\mbox{}\newline 
\hspace*{1em}\hspace*{1em}\hspace*{1em}\hspace*{1em}{<\textbf{body}>}\mbox{}\newline 
\textit{<!-- body of first unitary text -->}\mbox{}\newline 
\hspace*{1em}\hspace*{1em}\hspace*{1em}\hspace*{1em}{</\textbf{body}>}\mbox{}\newline 
\hspace*{1em}\hspace*{1em}\hspace*{1em}\hspace*{1em}{<\textbf{back}>}\mbox{}\newline 
\textit{<!-- back matter of first unitary text, if any -->}\mbox{}\newline 
\hspace*{1em}\hspace*{1em}\hspace*{1em}\hspace*{1em}{</\textbf{back}>}\mbox{}\newline 
\hspace*{1em}\hspace*{1em}\hspace*{1em}{</\textbf{text}>}\mbox{}\newline 
\hspace*{1em}\hspace*{1em}\hspace*{1em}{<\textbf{text}>}\mbox{}\newline 
\hspace*{1em}\hspace*{1em}\hspace*{1em}\hspace*{1em}{<\textbf{body}>}\mbox{}\newline 
\textit{<!-- body of second unitary text -->}\mbox{}\newline 
\hspace*{1em}\hspace*{1em}\hspace*{1em}\hspace*{1em}{</\textbf{body}>}\mbox{}\newline 
\hspace*{1em}\hspace*{1em}\hspace*{1em}{</\textbf{text}>}\mbox{}\newline 
\hspace*{1em}\hspace*{1em}{</\textbf{group}>}\mbox{}\newline 
\hspace*{1em}\hspace*{1em}{<\textbf{back}>}\mbox{}\newline 
\textit{<!-- back matter for composite text, if any -->}\mbox{}\newline 
\hspace*{1em}\hspace*{1em}{</\textbf{back}>}\mbox{}\newline 
\hspace*{1em}{</\textbf{text}>}\mbox{}\newline 
{</\textbf{TEI}>}\end{shaded}\egroup\par \par
Finally, a \hyperref[TEI.floatingText]{<floatingText>} element is provided for the case where one text is embedded within another, but does not contribute to its hierarchical organization, for example because it interrupts it, or simply quoted within it. This is useful in such common literary contexts as the ‘play within a play’ or the narrative interrupted by other (often deeply nested) multiple narratives.\par
Each of these elements is further described in the remainder of this chapter. Elements \hyperref[TEI.front]{<front>} and \hyperref[TEI.back]{<back>} are further discussed in sections \textit{\hyperref[DSFRONT]{4.5.\ Front Matter}} and \textit{\hyperref[DSBACK]{4.7.\ Back Matter}}. The \hyperref[TEI.group]{<group>} and \hyperref[TEI.floatingText]{<floatingText>} elements, used for more complex or composite text structures, are further discussed in section \textit{\hyperref[DSGRPF]{4.3.\ Grouped and Floating Texts}}. Other textual elements, such as paragraphs, lists or phrases, which nest within these major structural elements, are discussed in chapter \textit{\hyperref[CO]{3.\ Elements Available in All TEI Documents}}, in the case of elements which can appear in any kind of document, or elsewhere in the case of elements specific to particular kinds of document.
\subsection[{Divisions of the Body}]{Divisions of the Body}\label{DSDIV}\par
In some texts, the body consists simply of a sequence of low-level structural items, referred to here as \textit{components} or \textit{component-level elements} (see section \textit{\hyperref[STEC]{1.3.\ The TEI Class System}}). Examples in prose texts include paragraphs or lists; in dramatic texts, speeches and stage directions; in dictionaries, dictionary entries. In other cases sequences of such elements will be grouped together hierarchically into textual divisions and subdivisions, such as chapters or sections. The names used for these structural subdivisions of texts vary with the genre and period of the text, or even at the whim of the author, editor, or publisher. For example, a major subdivision of an epic or of the Bible is generally called a ‘book’, that of a report is usually called a ‘part’ or ‘section’, that of a novel a ‘chapter’—unless it is an epistolary novel, in which case it may be called a ‘letter’. Even texts which are not organized as linear prose narratives, or not as narratives at all, will frequently be subdivided in a similar way: a drama into ‘acts’ and ‘scenes’; a reference book into ‘sections’; a diary or day book into ‘entries’; a newspaper into ‘issues’ and ‘sections’, and so forth.\par
Because of this variety, these Guidelines propose that all such textual divisions be regarded as occurrences of the same neutrally named elements, with an attribute {\itshape type} used to categorize elements independently of their hierarchic level. Two alternative styles are provided for the marking of these neutral divisions: \textit{numbered} and \textit{un-numbered}. Numbered divisions are named \hyperref[TEI.div1]{<div1>}, \hyperref[TEI.div2]{<div2>}, etc., where the number indicates the depth of this particular division within the hierarchy, the largest such division being ‘div1’, any subdivision within it being ‘div2’, any further sub-sub-division being ‘div3’ and so on. Un-numbered divisions are simply named \hyperref[TEI.div]{<div>}, and allowed to nest recursively to indicate their hierarchic depth. The two styles must \textit{not} be combined within a single \hyperref[TEI.front]{<front>}, \hyperref[TEI.body]{<body>}, or \hyperref[TEI.back]{<back>} element.
\subsubsection[{Un-numbered Divisions}]{Un-numbered Divisions}\label{DSDIV1}\par
The following element is used to identify textual subdivisions in the un-numbered style: 
\begin{sansreflist}
  
\item [\textbf{<div>}] (text division) contains a subdivision of the front, body, or back of a text.
\end{sansreflist}
 As a member of the class \textsf{att.typed}, this element has the following additional attributes: 
\begin{sansreflist}
  
\item [\textbf{att.typed}] provides attributes which can be used to classify or subclassify elements in any way.\hfil\\[-10pt]\begin{sansreflist}
    \item[@{\itshape type}]
  characterizes the element in some sense, using any convenient classification scheme or typology.
    \item[@{\itshape subtype}]
  (subtype) provides a sub-categorization of the element, if needed
\end{sansreflist}  
\end{sansreflist}
\par
Using this style, the body of a text containing two parts, each composed of two chapters, might be represented as follows: \par\bgroup\index{body=<body>|exampleindex}\index{div=<div>|exampleindex}\index{type=@type!<div>|exampleindex}\index{n=@n!<div>|exampleindex}\index{div=<div>|exampleindex}\index{type=@type!<div>|exampleindex}\index{n=@n!<div>|exampleindex}\index{div=<div>|exampleindex}\index{type=@type!<div>|exampleindex}\index{n=@n!<div>|exampleindex}\index{div=<div>|exampleindex}\index{type=@type!<div>|exampleindex}\index{n=@n!<div>|exampleindex}\index{div=<div>|exampleindex}\index{n=@n!<div>|exampleindex}\index{type=@type!<div>|exampleindex}\index{div=<div>|exampleindex}\index{n=@n!<div>|exampleindex}\index{type=@type!<div>|exampleindex}\exampleFont \begin{shaded}\noindent\mbox{}{<\textbf{body}>}\mbox{}\newline 
\hspace*{1em}{<\textbf{div}\hspace*{1em}{type}="{part}"\hspace*{1em}{n}="{1}">}\mbox{}\newline 
\hspace*{1em}\hspace*{1em}{<\textbf{div}\hspace*{1em}{type}="{chapter}"\hspace*{1em}{n}="{1}">}\mbox{}\newline 
\textit{<!-- text of part 1, chapter 1 -->}\mbox{}\newline 
\hspace*{1em}\hspace*{1em}{</\textbf{div}>}\mbox{}\newline 
\hspace*{1em}\hspace*{1em}{<\textbf{div}\hspace*{1em}{type}="{chapter}"\hspace*{1em}{n}="{2}">}\mbox{}\newline 
\textit{<!-- text of part 1, chapter 2 -->}\mbox{}\newline 
\hspace*{1em}\hspace*{1em}{</\textbf{div}>}\mbox{}\newline 
\hspace*{1em}{</\textbf{div}>}\mbox{}\newline 
\hspace*{1em}{<\textbf{div}\hspace*{1em}{type}="{part}"\hspace*{1em}{n}="{2}">}\mbox{}\newline 
\hspace*{1em}\hspace*{1em}{<\textbf{div}\hspace*{1em}{n}="{1}"\hspace*{1em}{type}="{chapter}">}\mbox{}\newline 
\textit{<!-- text of part 2, chapter 1 -->}\mbox{}\newline 
\hspace*{1em}\hspace*{1em}{</\textbf{div}>}\mbox{}\newline 
\hspace*{1em}\hspace*{1em}{<\textbf{div}\hspace*{1em}{n}="{2}"\hspace*{1em}{type}="{chapter}">}\mbox{}\newline 
\textit{<!-- text of part 2, chapter 2 -->}\mbox{}\newline 
\hspace*{1em}\hspace*{1em}{</\textbf{div}>}\mbox{}\newline 
\hspace*{1em}{</\textbf{div}>}\mbox{}\newline 
{</\textbf{body}>}\end{shaded}\egroup\par 
\subsubsection[{Numbered Divisions}]{Numbered Divisions}\label{DSDIV2}\par
The following elements are used to identify textual subdivisions in the numbered style: 
\begin{sansreflist}
  
\item [\textbf{<div1>}] (level-1 text division) contains a first-level subdivision of the front, body, or back of a text.
\item [\textbf{<div2>}] (level-2 text division) contains a second-level subdivision of the front, body, or back of a text.
\item [\textbf{<div3>}] (level-3 text division) contains a third-level subdivision of the front, body, or back of a text.
\item [\textbf{<div4>}] (level-4 text division) contains a fourth-level subdivision of the front, body, or back of a text.
\item [\textbf{<div5>}] (level-5 text division) contains a fifth-level subdivision of the front, body, or back of a text.
\item [\textbf{<div6>}] (level-6 text division) contains a sixth-level subdivision of the front, body, or back of a text.
\item [\textbf{<div7>}] (level-7 text division) contains the smallest possible subdivision of the front, body or back of a text, larger than a paragraph.
\end{sansreflist}
 As members of the class \textsf{att.typed} these elements all bear the following additional attributes: 
\begin{sansreflist}
  
\item [\textbf{att.typed}] provides attributes which can be used to classify or subclassify elements in any way.\hfil\\[-10pt]\begin{sansreflist}
    \item[@{\itshape type}]
  characterizes the element in some sense, using any convenient classification scheme or typology.
    \item[@{\itshape subtype}]
  (subtype) provides a sub-categorization of the element, if needed
\end{sansreflist}  
\end{sansreflist}
\par
The largest possible subdivision of the body is \hyperref[TEI.div1]{<div1>} element and the smallest possible \hyperref[TEI.div7]{<div7>}. If numbered divisions are in use, a division at any one level (say, \hyperref[TEI.div3]{<div3>}), may contain only numbered divisions at the next lowest level (in this case, \hyperref[TEI.div4]{<div4>}).\par
Using this style, the body of a text containing two parts, each composed of two chapters, might be represented as follows: \par\bgroup\index{body=<body>|exampleindex}\index{div1=<div1>|exampleindex}\index{type=@type!<div1>|exampleindex}\index{n=@n!<div1>|exampleindex}\index{div2=<div2>|exampleindex}\index{type=@type!<div2>|exampleindex}\index{n=@n!<div2>|exampleindex}\index{div2=<div2>|exampleindex}\index{type=@type!<div2>|exampleindex}\index{n=@n!<div2>|exampleindex}\index{div1=<div1>|exampleindex}\index{type=@type!<div1>|exampleindex}\index{n=@n!<div1>|exampleindex}\index{div2=<div2>|exampleindex}\index{n=@n!<div2>|exampleindex}\index{type=@type!<div2>|exampleindex}\index{div2=<div2>|exampleindex}\index{n=@n!<div2>|exampleindex}\index{type=@type!<div2>|exampleindex}\exampleFont \begin{shaded}\noindent\mbox{}{<\textbf{body}>}\mbox{}\newline 
\hspace*{1em}{<\textbf{div1}\hspace*{1em}{type}="{part}"\hspace*{1em}{n}="{1}">}\mbox{}\newline 
\hspace*{1em}\hspace*{1em}{<\textbf{div2}\hspace*{1em}{type}="{chapter}"\hspace*{1em}{n}="{1}">}\mbox{}\newline 
\textit{<!-- text of part 1, chapter 1 -->}\mbox{}\newline 
\hspace*{1em}\hspace*{1em}{</\textbf{div2}>}\mbox{}\newline 
\hspace*{1em}\hspace*{1em}{<\textbf{div2}\hspace*{1em}{type}="{chapter}"\hspace*{1em}{n}="{2}">}\mbox{}\newline 
\textit{<!-- text of part 1, chapter 2 -->}\mbox{}\newline 
\hspace*{1em}\hspace*{1em}{</\textbf{div2}>}\mbox{}\newline 
\hspace*{1em}{</\textbf{div1}>}\mbox{}\newline 
\hspace*{1em}{<\textbf{div1}\hspace*{1em}{type}="{part}"\hspace*{1em}{n}="{2}">}\mbox{}\newline 
\hspace*{1em}\hspace*{1em}{<\textbf{div2}\hspace*{1em}{n}="{1}"\hspace*{1em}{type}="{chapter}">}\mbox{}\newline 
\textit{<!-- text of part 2, chapter 1 -->}\mbox{}\newline 
\hspace*{1em}\hspace*{1em}{</\textbf{div2}>}\mbox{}\newline 
\hspace*{1em}\hspace*{1em}{<\textbf{div2}\hspace*{1em}{n}="{2}"\hspace*{1em}{type}="{chapter}">}\mbox{}\newline 
\textit{<!-- text of part 2, chapter 2 -->}\mbox{}\newline 
\hspace*{1em}\hspace*{1em}{</\textbf{div2}>}\mbox{}\newline 
\hspace*{1em}{</\textbf{div1}>}\mbox{}\newline 
{</\textbf{body}>}\end{shaded}\egroup\par \noindent          
\subsubsection[{Numbered or Un-numbered?}]{Numbered or Un-numbered?}\label{DSDIV3}\par
Within the same \hyperref[TEI.front]{<front>}, \hyperref[TEI.body]{<body>}, or \hyperref[TEI.back]{<back>} element, all hierarchic subdivisions must be marked using either nested \hyperref[TEI.div]{<div>} elements, or \hyperref[TEI.div1]{<div1>}, \hyperref[TEI.div2]{<div2>} etc. elements nested as appropriate; the two styles must \textit{not} be mixed.\par
The choice between numbered and un-numbered divisions will depend to some extent on the complexity of the material: un-numbered divisions allow for an arbitrary depth of nesting, while numbered divisions limit the depth of the tree which can be constructed. Where divisions at different levels should be processed differently (for example to ensure that chapters, but not sections, begin on a new page), numbered divisions slightly simplify the task of defining the desired processing for each level, though this distinction could also be made by supplying this information on the {\itshape type} attribute of an un-numbered \hyperref[TEI.div]{<div>}. Some software may find numbered divisions easier to process, as there is no need to maintain knowledge of the whole document structure in order to know the level at which a division occurs; such software may, however, find it difficult to cope with some other aspects of the TEI scheme. On the other hand, in a collection of many works it may prove difficult or impossible to ensure that the same numbered division always corresponds with the same type of textual feature: a ‘chapter’ may be at level 1 in one work and level 3 in another.\par
Whichever style is used, the global {\itshape n} and {\itshape xml:id} attributes (section \textit{\hyperref[STGA]{1.3.1.1.\ Global Attributes}}) may be used to provide reference strings or labels for each division of a text, where appropriate. Such labels should be provided for each section which is regarded as significant for referencing purposes (on reference systems, see further section \textit{\hyperref[CORS]{3.11.\ Reference Systems}}).\par
As indicated above, the {\itshape type} and {\itshape subtype} attributes provided by the \textsf{att.typed} class may be used to provide a name or description for the division. Typical values might be ‘book’, ‘chapter’, ‘section’, ‘part’, or (for verse texts) ‘book’, ‘canto’, ‘stanza’, or (for dramatic texts) ‘act’, ‘scene’. The following extended example uses numbered divisions to indicate the structure of a novel, and illustrates the use of the attributes discussed above. It also uses some elements discussed in section \textit{\hyperref[DSDTB]{4.2.\ Elements Common to All Divisions}} and the \hyperref[TEI.p]{<p>} element discussed in section \textit{\hyperref[COPA]{3.1.\ Paragraphs}}.  \par\bgroup\index{div1=<div1>|exampleindex}\index{type=@type!<div1>|exampleindex}\index{n=@n!<div1>|exampleindex}\index{head=<head>|exampleindex}\index{div2=<div2>|exampleindex}\index{type=@type!<div2>|exampleindex}\index{n=@n!<div2>|exampleindex}\index{head=<head>|exampleindex}\index{p=<p>|exampleindex}\index{div2=<div2>|exampleindex}\index{type=@type!<div2>|exampleindex}\index{n=@n!<div2>|exampleindex}\index{head=<head>|exampleindex}\index{p=<p>|exampleindex}\index{trailer=<trailer>|exampleindex}\index{div1=<div1>|exampleindex}\index{type=@type!<div1>|exampleindex}\index{n=@n!<div1>|exampleindex}\index{head=<head>|exampleindex}\index{div2=<div2>|exampleindex}\index{type=@type!<div2>|exampleindex}\index{n=@n!<div2>|exampleindex}\index{head=<head>|exampleindex}\index{p=<p>|exampleindex}\index{term=<term>|exampleindex}\index{term=<term>|exampleindex}\index{p=<p>|exampleindex}\index{div2=<div2>|exampleindex}\index{type=@type!<div2>|exampleindex}\index{n=@n!<div2>|exampleindex}\index{head=<head>|exampleindex}\index{p=<p>|exampleindex}\exampleFont \begin{shaded}\noindent\mbox{}{<\textbf{div1}\hspace*{1em}{type}="{book}"\hspace*{1em}{n}="{I}"\hspace*{1em}{xml:id}="{JA0100}">}\mbox{}\newline 
\hspace*{1em}{<\textbf{head}>}Book I.{</\textbf{head}>}\mbox{}\newline 
\hspace*{1em}{<\textbf{div2}\hspace*{1em}{type}="{chapter}"\hspace*{1em}{n}="{1}"\hspace*{1em}{xml:id}="{JA0101}">}\mbox{}\newline 
\hspace*{1em}\hspace*{1em}{<\textbf{head}>}Of writing lives in general, and particularly of Pamela, with a word\mbox{}\newline 
\hspace*{1em}\hspace*{1em}\hspace*{1em}\hspace*{1em} by the bye of Colley Cibber and others.{</\textbf{head}>}\mbox{}\newline 
\hspace*{1em}\hspace*{1em}{<\textbf{p}>}It is a trite but true observation, that examples work more forcibly on\mbox{}\newline 
\hspace*{1em}\hspace*{1em}\hspace*{1em}\hspace*{1em} the mind than precepts: ... {</\textbf{p}>}\mbox{}\newline 
\textit{<!-- remainder of chapter 1 here -->}\mbox{}\newline 
\hspace*{1em}{</\textbf{div2}>}\mbox{}\newline 
\hspace*{1em}{<\textbf{div2}\hspace*{1em}{type}="{chapter}"\hspace*{1em}{n}="{2}"\hspace*{1em}{xml:id}="{JA0102}">}\mbox{}\newline 
\hspace*{1em}\hspace*{1em}{<\textbf{head}>}Of Mr. Joseph Andrews, his birth, parentage, education, and great\mbox{}\newline 
\hspace*{1em}\hspace*{1em}\hspace*{1em}\hspace*{1em} endowments; with a word or two concerning ancestors.{</\textbf{head}>}\mbox{}\newline 
\hspace*{1em}\hspace*{1em}{<\textbf{p}>}Mr. Joseph Andrews, the hero of our ensuing history, was esteemed to\mbox{}\newline 
\hspace*{1em}\hspace*{1em}\hspace*{1em}\hspace*{1em} be the only son of Gaffar and Gammar Andrews, and brother to the\mbox{}\newline 
\hspace*{1em}\hspace*{1em}\hspace*{1em}\hspace*{1em} illustrious Pamela, whose virtue is at present so famous ... {</\textbf{p}>}\mbox{}\newline 
\textit{<!-- remainder of chapter 2 here -->}\mbox{}\newline 
\hspace*{1em}{</\textbf{div2}>}\mbox{}\newline 
\textit{<!-- remaining chapters of Book 1 here -->}\mbox{}\newline 
\hspace*{1em}{<\textbf{trailer}>}The end of the first Book{</\textbf{trailer}>}\mbox{}\newline 
{</\textbf{div1}>}\mbox{}\newline 
{<\textbf{div1}\hspace*{1em}{type}="{book}"\hspace*{1em}{n}="{II}"\hspace*{1em}{xml:id}="{JA0200}">}\mbox{}\newline 
\hspace*{1em}{<\textbf{head}>}Book II{</\textbf{head}>}\mbox{}\newline 
\hspace*{1em}{<\textbf{div2}\hspace*{1em}{type}="{chapter}"\hspace*{1em}{n}="{1}"\hspace*{1em}{xml:id}="{JA0201}">}\mbox{}\newline 
\hspace*{1em}\hspace*{1em}{<\textbf{head}>}Of divisions in authors{</\textbf{head}>}\mbox{}\newline 
\hspace*{1em}\hspace*{1em}{<\textbf{p}>}There are certain mysteries or secrets in all trades, from the highest\mbox{}\newline 
\hspace*{1em}\hspace*{1em}\hspace*{1em}\hspace*{1em} to the lowest, from that of {<\textbf{term}>}prime-ministering{</\textbf{term}>}, to this of\mbox{}\newline 
\hspace*{1em}\hspace*{1em}{<\textbf{term}>}authoring{</\textbf{term}>}, which are seldom discovered unless to members of\mbox{}\newline 
\hspace*{1em}\hspace*{1em}\hspace*{1em}\hspace*{1em} the same calling ... {</\textbf{p}>}\mbox{}\newline 
\hspace*{1em}\hspace*{1em}{<\textbf{p}>}I will dismiss this chapter with the following observation: that it\mbox{}\newline 
\hspace*{1em}\hspace*{1em}\hspace*{1em}\hspace*{1em} becomes an author generally to divide a book, as it does a butcher to\mbox{}\newline 
\hspace*{1em}\hspace*{1em}\hspace*{1em}\hspace*{1em} joint his meat, for such assistance is of great help to both the reader\mbox{}\newline 
\hspace*{1em}\hspace*{1em}\hspace*{1em}\hspace*{1em} and the carver. And now having indulged myself a little I will endeavour\mbox{}\newline 
\hspace*{1em}\hspace*{1em}\hspace*{1em}\hspace*{1em} to indulge the curiosity of my reader, who is no doubt impatient to know\mbox{}\newline 
\hspace*{1em}\hspace*{1em}\hspace*{1em}\hspace*{1em} what he will find in the subsequent chapters of this book.{</\textbf{p}>}\mbox{}\newline 
\hspace*{1em}{</\textbf{div2}>}\mbox{}\newline 
\hspace*{1em}{<\textbf{div2}\hspace*{1em}{type}="{chapter}"\hspace*{1em}{n}="{2}"\hspace*{1em}{xml:id}="{JA0202}">}\mbox{}\newline 
\hspace*{1em}\hspace*{1em}{<\textbf{head}>}A surprising instance of Mr. Adams's short memory, with the\mbox{}\newline 
\hspace*{1em}\hspace*{1em}\hspace*{1em}\hspace*{1em} unfortunate consequences which it brought on Joseph.\mbox{}\newline 
\hspace*{1em}\hspace*{1em}{</\textbf{head}>}\mbox{}\newline 
\hspace*{1em}\hspace*{1em}{<\textbf{p}>}Mr. Adams and Joseph were now ready to depart different ways ... {</\textbf{p}>}\mbox{}\newline 
\hspace*{1em}{</\textbf{div2}>}\mbox{}\newline 
{</\textbf{div1}>}\end{shaded}\egroup\par \noindent  \par
As an alternative (or complement) to this use of the {\itshape type} attribute to characterize neutrally named division elements, the modification mechanisms discussed in section \textit{\hyperref[MD]{23.3.\ Customization}} may be used to define new elements such as \texttt{<chapter>}, \texttt{<part>}, etc. To make this simpler, a single member model class is defined for each of the neutrally named division elements: \textsf{model.divLike} (containing \hyperref[TEI.div]{<div>}), \textsf{model.div1Like} (containing \hyperref[TEI.div1]{<div1>}), \textsf{model.div2Like} (containing \hyperref[TEI.div2]{<div2>}), etc. For example, suppose that the body of a text consists of a series of diary entries, each of which is potentially divided into entries for the morning and the afternoon. This might be represented in any of the following ways. First, using the un-numbered style: \par\bgroup\index{body=<body>|exampleindex}\index{div=<div>|exampleindex}\index{type=@type!<div>|exampleindex}\index{n=@n!<div>|exampleindex}\index{div=<div>|exampleindex}\index{type=@type!<div>|exampleindex}\index{n=@n!<div>|exampleindex}\index{p=<p>|exampleindex}\index{div=<div>|exampleindex}\index{type=@type!<div>|exampleindex}\index{n=@n!<div>|exampleindex}\index{p=<p>|exampleindex}\index{div=<div>|exampleindex}\index{type=@type!<div>|exampleindex}\index{n=@n!<div>|exampleindex}\index{div=<div>|exampleindex}\index{type=@type!<div>|exampleindex}\index{n=@n!<div>|exampleindex}\index{p=<p>|exampleindex}\index{div=<div>|exampleindex}\index{type=@type!<div>|exampleindex}\index{n=@n!<div>|exampleindex}\index{p=<p>|exampleindex}\exampleFont \begin{shaded}\noindent\mbox{}{<\textbf{body}>}\mbox{}\newline 
\hspace*{1em}{<\textbf{div}\hspace*{1em}{type}="{entry}"\hspace*{1em}{n}="{1}">}\mbox{}\newline 
\hspace*{1em}\hspace*{1em}{<\textbf{div}\hspace*{1em}{type}="{morning}"\hspace*{1em}{n}="{1.1}">}\mbox{}\newline 
\hspace*{1em}\hspace*{1em}\hspace*{1em}{<\textbf{p}>}[...]{</\textbf{p}>}\mbox{}\newline 
\hspace*{1em}\hspace*{1em}{</\textbf{div}>}\mbox{}\newline 
\hspace*{1em}\hspace*{1em}{<\textbf{div}\hspace*{1em}{type}="{afternoon}"\hspace*{1em}{n}="{1.2}">}\mbox{}\newline 
\hspace*{1em}\hspace*{1em}\hspace*{1em}{<\textbf{p}>}[...]{</\textbf{p}>}\mbox{}\newline 
\hspace*{1em}\hspace*{1em}{</\textbf{div}>}\mbox{}\newline 
\hspace*{1em}{</\textbf{div}>}\mbox{}\newline 
\hspace*{1em}{<\textbf{div}\hspace*{1em}{type}="{entry}"\hspace*{1em}{n}="{2}">}\mbox{}\newline 
\hspace*{1em}\hspace*{1em}{<\textbf{div}\hspace*{1em}{type}="{morning}"\hspace*{1em}{n}="{2.1}">}\mbox{}\newline 
\hspace*{1em}\hspace*{1em}\hspace*{1em}{<\textbf{p}>}[...]{</\textbf{p}>}\mbox{}\newline 
\hspace*{1em}\hspace*{1em}{</\textbf{div}>}\mbox{}\newline 
\hspace*{1em}\hspace*{1em}{<\textbf{div}\hspace*{1em}{type}="{afternoon}"\hspace*{1em}{n}="{2.2}">}\mbox{}\newline 
\hspace*{1em}\hspace*{1em}\hspace*{1em}{<\textbf{p}>}[...]{</\textbf{p}>}\mbox{}\newline 
\hspace*{1em}\hspace*{1em}{</\textbf{div}>}\mbox{}\newline 
\hspace*{1em}{</\textbf{div}>}\mbox{}\newline 
\textit{<!-- ...-->}\mbox{}\newline 
{</\textbf{body}>}\end{shaded}\egroup\par \noindent  Equivalently, using the numbered style: \par\bgroup\index{body=<body>|exampleindex}\index{div1=<div1>|exampleindex}\index{type=@type!<div1>|exampleindex}\index{n=@n!<div1>|exampleindex}\index{div2=<div2>|exampleindex}\index{type=@type!<div2>|exampleindex}\index{n=@n!<div2>|exampleindex}\index{p=<p>|exampleindex}\index{div2=<div2>|exampleindex}\index{type=@type!<div2>|exampleindex}\index{n=@n!<div2>|exampleindex}\index{p=<p>|exampleindex}\index{div1=<div1>|exampleindex}\index{type=@type!<div1>|exampleindex}\index{n=@n!<div1>|exampleindex}\index{div2=<div2>|exampleindex}\index{type=@type!<div2>|exampleindex}\index{n=@n!<div2>|exampleindex}\index{p=<p>|exampleindex}\index{div2=<div2>|exampleindex}\index{type=@type!<div2>|exampleindex}\index{n=@n!<div2>|exampleindex}\index{p=<p>|exampleindex}\exampleFont \begin{shaded}\noindent\mbox{}{<\textbf{body}>}\mbox{}\newline 
\hspace*{1em}{<\textbf{div1}\hspace*{1em}{type}="{entry}"\hspace*{1em}{n}="{1}">}\mbox{}\newline 
\hspace*{1em}\hspace*{1em}{<\textbf{div2}\hspace*{1em}{type}="{morning}"\hspace*{1em}{n}="{1.1}">}\mbox{}\newline 
\hspace*{1em}\hspace*{1em}\hspace*{1em}{<\textbf{p}>}[...]{</\textbf{p}>}\mbox{}\newline 
\hspace*{1em}\hspace*{1em}{</\textbf{div2}>}\mbox{}\newline 
\hspace*{1em}\hspace*{1em}{<\textbf{div2}\hspace*{1em}{type}="{afternoon}"\hspace*{1em}{n}="{1.2}">}\mbox{}\newline 
\hspace*{1em}\hspace*{1em}\hspace*{1em}{<\textbf{p}>}[...]{</\textbf{p}>}\mbox{}\newline 
\hspace*{1em}\hspace*{1em}{</\textbf{div2}>}\mbox{}\newline 
\hspace*{1em}{</\textbf{div1}>}\mbox{}\newline 
\hspace*{1em}{<\textbf{div1}\hspace*{1em}{type}="{entry}"\hspace*{1em}{n}="{2}">}\mbox{}\newline 
\hspace*{1em}\hspace*{1em}{<\textbf{div2}\hspace*{1em}{type}="{morning}"\hspace*{1em}{n}="{2.1}">}\mbox{}\newline 
\hspace*{1em}\hspace*{1em}\hspace*{1em}{<\textbf{p}>}[...]{</\textbf{p}>}\mbox{}\newline 
\hspace*{1em}\hspace*{1em}{</\textbf{div2}>}\mbox{}\newline 
\hspace*{1em}\hspace*{1em}{<\textbf{div2}\hspace*{1em}{type}="{afternoon}"\hspace*{1em}{n}="{2.2}">}\mbox{}\newline 
\hspace*{1em}\hspace*{1em}\hspace*{1em}{<\textbf{p}>}[...]{</\textbf{p}>}\mbox{}\newline 
\hspace*{1em}\hspace*{1em}{</\textbf{div2}>}\mbox{}\newline 
\hspace*{1em}{</\textbf{div1}>}\mbox{}\newline 
\textit{<!-- ...-->}\mbox{}\newline 
{</\textbf{body}>}\end{shaded}\egroup\par \noindent  Now, assuming a customization in which a new element \texttt{<diaryEntry>} has been added to the \textsf{model.divLike} class: \par\bgroup\index{body=<body>|exampleindex}\index{p=<p>|exampleindex}\index{p=<p>|exampleindex}\index{p=<p>|exampleindex}\index{p=<p>|exampleindex}\exampleFont \begin{shaded}\noindent\mbox{}{<\textbf{body}\mbox{}\newline 
   xmlns:my="http://www.example.org/ns/nonTEI">}\mbox{}\newline 
\hspace*{1em}{<\textbf{my:diaryEntry}\hspace*{1em}{type}="{entry}"\hspace*{1em}{n}="{1}">}\mbox{}\newline 
\hspace*{1em}\hspace*{1em}{<\textbf{my:diaryEntry}\hspace*{1em}{type}="{morning}"\hspace*{1em}{n}="{1.1}">}\mbox{}\newline 
\hspace*{1em}\hspace*{1em}\hspace*{1em}{<\textbf{p}>}[...]{</\textbf{p}>}\mbox{}\newline 
\hspace*{1em}\hspace*{1em}{</\textbf{my:diaryEntry}>}\mbox{}\newline 
\hspace*{1em}\hspace*{1em}{<\textbf{my:diaryEntry}\hspace*{1em}{type}="{afternoon}"\hspace*{1em}{n}="{1.2}">}\mbox{}\newline 
\hspace*{1em}\hspace*{1em}\hspace*{1em}{<\textbf{p}>}[...]{</\textbf{p}>}\mbox{}\newline 
\hspace*{1em}\hspace*{1em}{</\textbf{my:diaryEntry}>}\mbox{}\newline 
\hspace*{1em}{</\textbf{my:diaryEntry}>}\mbox{}\newline 
\hspace*{1em}{<\textbf{my:diaryEntry}\hspace*{1em}{type}="{entry}"\hspace*{1em}{n}="{1}">}\mbox{}\newline 
\hspace*{1em}\hspace*{1em}{<\textbf{my:diaryEntry}\hspace*{1em}{type}="{morning}"\hspace*{1em}{n}="{1.1}">}\mbox{}\newline 
\hspace*{1em}\hspace*{1em}\hspace*{1em}{<\textbf{p}>}[...]{</\textbf{p}>}\mbox{}\newline 
\hspace*{1em}\hspace*{1em}{</\textbf{my:diaryEntry}>}\mbox{}\newline 
\hspace*{1em}\hspace*{1em}{<\textbf{my:diaryEntry}\hspace*{1em}{type}="{afternoon}"\hspace*{1em}{n}="{1.2}">}\mbox{}\newline 
\hspace*{1em}\hspace*{1em}\hspace*{1em}{<\textbf{p}>}[...]{</\textbf{p}>}\mbox{}\newline 
\hspace*{1em}\hspace*{1em}{</\textbf{my:diaryEntry}>}\mbox{}\newline 
\hspace*{1em}{</\textbf{my:diaryEntry}>}\mbox{}\newline 
\textit{<!-- ...-->}\mbox{}\newline 
{</\textbf{body}>}\end{shaded}\egroup\par \noindent  And finally, assuming a customization in which three new elements have been added: \texttt{<diaryEntry>} to the \textsf{model.div1Like} class, and \texttt{<amEntry>} and \texttt{<pmEntry>} both to the \textsf{model.div2Like} class: \par\bgroup\index{body=<body>|exampleindex}\index{p=<p>|exampleindex}\index{p=<p>|exampleindex}\index{p=<p>|exampleindex}\index{p=<p>|exampleindex}\index{p=<p>|exampleindex}\exampleFont \begin{shaded}\noindent\mbox{}{<\textbf{body}\mbox{}\newline 
   xmlns:my="http://www.example.org/ns/nonTEI">}\mbox{}\newline 
\hspace*{1em}{<\textbf{p}>}\mbox{}\newline 
\textit{<!-- ... -->}\mbox{}\newline 
\hspace*{1em}{</\textbf{p}>}\mbox{}\newline 
\hspace*{1em}{<\textbf{my:diaryEntry}\hspace*{1em}{type}="{entry}"\hspace*{1em}{n}="{1}">}\mbox{}\newline 
\hspace*{1em}\hspace*{1em}{<\textbf{my:amEntry}\hspace*{1em}{type}="{morning}"\hspace*{1em}{n}="{1.1}">}\mbox{}\newline 
\hspace*{1em}\hspace*{1em}\hspace*{1em}{<\textbf{p}>}[...]{</\textbf{p}>}\mbox{}\newline 
\hspace*{1em}\hspace*{1em}{</\textbf{my:amEntry}>}\mbox{}\newline 
\hspace*{1em}\hspace*{1em}{<\textbf{my:pmEntry}\hspace*{1em}{type}="{afternoon}"\hspace*{1em}{n}="{1.2}">}\mbox{}\newline 
\hspace*{1em}\hspace*{1em}\hspace*{1em}{<\textbf{p}>}[...]{</\textbf{p}>}\mbox{}\newline 
\hspace*{1em}\hspace*{1em}{</\textbf{my:pmEntry}>}\mbox{}\newline 
\hspace*{1em}{</\textbf{my:diaryEntry}>}\mbox{}\newline 
\hspace*{1em}{<\textbf{my:diaryEntry}\hspace*{1em}{type}="{entry}"\hspace*{1em}{n}="{1}">}\mbox{}\newline 
\hspace*{1em}\hspace*{1em}{<\textbf{my:amEntry}\hspace*{1em}{type}="{morning}"\hspace*{1em}{n}="{1.1}">}\mbox{}\newline 
\hspace*{1em}\hspace*{1em}\hspace*{1em}{<\textbf{p}>}[...]{</\textbf{p}>}\mbox{}\newline 
\hspace*{1em}\hspace*{1em}{</\textbf{my:amEntry}>}\mbox{}\newline 
\hspace*{1em}\hspace*{1em}{<\textbf{my:pmEntry}\hspace*{1em}{type}="{afternoon}"\hspace*{1em}{n}="{1.1}">}\mbox{}\newline 
\hspace*{1em}\hspace*{1em}\hspace*{1em}{<\textbf{p}>}[...]{</\textbf{p}>}\mbox{}\newline 
\hspace*{1em}\hspace*{1em}{</\textbf{my:pmEntry}>}\mbox{}\newline 
\hspace*{1em}{</\textbf{my:diaryEntry}>}\mbox{}\newline 
\textit{<!-- ... -->}\mbox{}\newline 
{</\textbf{body}>}\end{shaded}\egroup\par \par
More information about the customization techniques exemplified here is provided in \textit{\hyperref[MD]{23.3.\ Customization}}.
\subsubsection[{Partial and Composite Divisions}]{Partial and Composite Divisions}\label{DSDIV3X}\par
In most situations, the textual subdivisions marked by \hyperref[TEI.div]{<div>} or \hyperref[TEI.div1]{<div1>} (etc.) elements will be both complete and identically organized with reference to the original source. For some purposes however, in particular where dealing with unusually large or unusually small texts, encoders may find it convenient to present as textual divisions sequences of text which are incomplete with reference to the original text, or which are in fact an ad hoc agglomeration of tiny texts. Moreover, in some kinds of texts it is difficult or impossible to determine the order in which individual subdivisions should be combined to form the next higher level of subdivision, as noted below.\par
To overcome these problems, the following additional attributes are defined for all elements in the \textsf{att.divLike} class: 
\begin{sansreflist}
  
\item [\textbf{att.divLike}] provides attributes common to all elements which behave in the same way as divisions.\hfil\\[-10pt]\begin{sansreflist}
    \item[@{\itshape org}]
  (organization) specifies how the content of the division is organized.
    \item[@{\itshape sample}]
  indicates whether this division is a sample of the original source and if so, from which part.
\end{sansreflist}  
\item [\textbf{att.fragmentable}] provides an attribute for representing fragmentation of a structural element, typically as a consequence of some overlapping hierarchy.\hfil\\[-10pt]\begin{sansreflist}
    \item[@{\itshape part}]
  specifies whether or not its parent element is fragmented in some way, typically by some other overlapping structure: for example a speech which is divided between two or more verse stanzas, a paragraph which is split across a page division, a verse line which is divided between two speakers.
\end{sansreflist}  
\end{sansreflist}
\par
For example, an encoder might choose to transcribe only the first two thousand words of each chapter from a novel. In such a case, each chapter might conveniently be regarded as a partial division, and tagged with a \hyperref[TEI.div]{<div>} element in the following form: \par\bgroup\index{div=<div>|exampleindex}\index{n=@n!<div>|exampleindex}\index{sample=@sample!<div>|exampleindex}\index{part=@part!<div>|exampleindex}\index{type=@type!<div>|exampleindex}\index{p=<p>|exampleindex}\exampleFont \begin{shaded}\noindent\mbox{}{<\textbf{div}\hspace*{1em}{n}="{xx}"\hspace*{1em}{sample}="{initial}"\hspace*{1em}{part}="{Y}"\mbox{}\newline 
\hspace*{1em}{type}="{chapter}">}\mbox{}\newline 
\hspace*{1em}{<\textbf{p}>} ... {</\textbf{p}>}\mbox{}\newline 
{</\textbf{div}>}\end{shaded}\egroup\par \noindent  where xx represents a number for the chapter, and the {\itshape part} attribute takes the value Y to indicate that this division is incomplete in some respect. Other possible values for this attribute indicate whether material has been omitted initially (I), finally (F), or in the middle (M) of the division, while the \hyperref[TEI.gap]{<gap>} element (\textit{\hyperref[COEDADD]{3.5.3.\ Additions, Deletions, and Omissions}}) may be used to indicate exactly where material has been omitted: \par\bgroup\index{div=<div>|exampleindex}\index{n=@n!<div>|exampleindex}\index{part=@part!<div>|exampleindex}\index{type=@type!<div>|exampleindex}\index{p=<p>|exampleindex}\index{gap=<gap>|exampleindex}\index{extent=@extent!<gap>|exampleindex}\index{reason=@reason!<gap>|exampleindex}\index{p=<p>|exampleindex}\exampleFont \begin{shaded}\noindent\mbox{}{<\textbf{div}\hspace*{1em}{n}="{xx}"\hspace*{1em}{part}="{M}"\hspace*{1em}{type}="{chapter}">}\mbox{}\newline 
\hspace*{1em}{<\textbf{p}>} ... {</\textbf{p}>}\mbox{}\newline 
\hspace*{1em}{<\textbf{gap}\hspace*{1em}{extent}="{2}"\hspace*{1em}{reason}="{sampling}"/>}\mbox{}\newline 
\hspace*{1em}{<\textbf{p}>} ... {</\textbf{p}>}\mbox{}\newline 
{</\textbf{div}>}\end{shaded}\egroup\par \noindent  The \hyperref[TEI.samplingDecl]{<samplingDecl>} element in the TEI header should also be used to record the principles underlying the selection of incomplete samples, as further described in section \textit{\hyperref[HD52]{2.3.2.\ The Sampling Declaration}}.\par
The following example demonstrates how a newspaper column composed of very short unrelated snippets may be encoded using these attributes: \par\bgroup\index{div1=<div1>|exampleindex}\index{type=@type!<div1>|exampleindex}\index{org=@org!<div1>|exampleindex}\index{head=<head>|exampleindex}\index{div2=<div2>|exampleindex}\index{type=@type!<div2>|exampleindex}\index{head=<head>|exampleindex}\index{soCalled=<soCalled>|exampleindex}\index{p=<p>|exampleindex}\index{div2=<div2>|exampleindex}\index{type=@type!<div2>|exampleindex}\index{head=<head>|exampleindex}\index{p=<p>|exampleindex}\index{div2=<div2>|exampleindex}\index{type=@type!<div2>|exampleindex}\index{head=<head>|exampleindex}\index{p=<p>|exampleindex}\exampleFont \begin{shaded}\noindent\mbox{}{<\textbf{div1}\hspace*{1em}{type}="{storylist}"\hspace*{1em}{org}="{composite}">}\mbox{}\newline 
\hspace*{1em}{<\textbf{head}>}News in brief{</\textbf{head}>}\mbox{}\newline 
\hspace*{1em}{<\textbf{div2}\hspace*{1em}{type}="{story}">}\mbox{}\newline 
\hspace*{1em}\hspace*{1em}{<\textbf{head}>}Police deny {<\textbf{soCalled}>}losing{</\textbf{soCalled}>} bomb{</\textbf{head}>}\mbox{}\newline 
\hspace*{1em}\hspace*{1em}{<\textbf{p}>}Scotland Yard yesterday denied claims in the Sunday\mbox{}\newline 
\hspace*{1em}\hspace*{1em}\hspace*{1em}\hspace*{1em} Express that anti-terrorist officers trailing an IRA van\mbox{}\newline 
\hspace*{1em}\hspace*{1em}\hspace*{1em}\hspace*{1em} loaded with explosives in north London had lost track of\mbox{}\newline 
\hspace*{1em}\hspace*{1em}\hspace*{1em}\hspace*{1em} it 10 days ago.{</\textbf{p}>}\mbox{}\newline 
\hspace*{1em}{</\textbf{div2}>}\mbox{}\newline 
\hspace*{1em}{<\textbf{div2}\hspace*{1em}{type}="{story}">}\mbox{}\newline 
\hspace*{1em}\hspace*{1em}{<\textbf{head}>}Hotel blaze{</\textbf{head}>}\mbox{}\newline 
\hspace*{1em}\hspace*{1em}{<\textbf{p}>}Nearly 200 guests were evacuated before dawn\mbox{}\newline 
\hspace*{1em}\hspace*{1em}\hspace*{1em}\hspace*{1em} yesterday after fire broke out at the Scandic\mbox{}\newline 
\hspace*{1em}\hspace*{1em}\hspace*{1em}\hspace*{1em} Crown hotel in the Royal Mile, Edinburgh.{</\textbf{p}>}\mbox{}\newline 
\hspace*{1em}{</\textbf{div2}>}\mbox{}\newline 
\hspace*{1em}{<\textbf{div2}\hspace*{1em}{type}="{story}">}\mbox{}\newline 
\hspace*{1em}\hspace*{1em}{<\textbf{head}>}Test match split{</\textbf{head}>}\mbox{}\newline 
\hspace*{1em}\hspace*{1em}{<\textbf{p}>}Test Match Special next summer will be split\mbox{}\newline 
\hspace*{1em}\hspace*{1em}\hspace*{1em}\hspace*{1em} between Radio 5 and Radio 3, after protests this\mbox{}\newline 
\hspace*{1em}\hspace*{1em}\hspace*{1em}\hspace*{1em} year that it disrupted Radio 3's music schedule.{</\textbf{p}>}\mbox{}\newline 
\hspace*{1em}{</\textbf{div2}>}\mbox{}\newline 
{</\textbf{div1}>}\end{shaded}\egroup\par \noindent  \par
The {\itshape org} attribute on the \hyperref[TEI.div1]{<div1>} element is used here to indicate that individual stories in this group, marked here as \hyperref[TEI.div2]{<div2>}, are really quite independent of each other, although they are all marked as subdivisions of the whole group. They can be read in any order without affecting the sense of the piece; indeed, in some cases, divisions of this nature are printed in such a way as to make it impossible to determine the order in which they are intended to be read. Individual stories can be added or removed without affecting the existing components.\par
This method of encoding composite texts as composite divisions has some limitations compared with the more general and powerful mechanisms discussed in section \textit{\hyperref[DSGRP]{4.3.1.\ Grouped Texts}}. However, it may be preferable in some circumstances, notably where the individual texts are very small.
\subsection[{Elements Common to All Divisions}]{Elements Common to All Divisions}\label{DSDTB}\par
The divisions of any kind of text may sometimes begin with a brief heading or descriptive title, with or without a byline, an epigraph or brief quotation, or a salutation such as one finds at the start of a letter. They may also conclude with a brief trailer, byline, postscript, or signature. Many of these (e.g. a byline) may appear either at the start or at the end of a text division proper.\par
To support this heterogeneity, the TEI architecture defines five classes, all of which are populated by this module: 
\begin{sansreflist}
  
\item [\textbf{model.divTop}] groups elements appearing at the beginning of a text division. \par 
\begin{longtable}{P{0.3332\textwidth}P{0.5167999999999999\textwidth}}
\hyperref[TEI.model.divTopPart]{model.divTopPart}\tabcellsep groups elements which can occur only at the beginning of a text division.\\
\hyperref[TEI.model.divWrapper]{model.divWrapper}\tabcellsep groups elements which can appear at either top or bottom of a textual division.\end{longtable} \par
 
\item [\textbf{model.divBottom}] groups elements appearing at the end of a text division. \par 
\begin{longtable}{P{0.35359999999999997\textwidth}P{0.49639999999999995\textwidth}}
\hyperref[TEI.model.divBottomPart]{model.divBottomPart}\tabcellsep groups elements which can occur only at the end of a text division.\\
\hyperref[TEI.model.divWrapper]{model.divWrapper}\tabcellsep groups elements which can appear at either top or bottom of a textual division.\end{longtable} \par
 
\item [\textbf{model.divTopPart}] groups elements which can occur only at the beginning of a text division. \par 
\begin{longtable}{P{0.298828125\textwidth}P{0.551171875\textwidth}}
\hyperref[TEI.model.headLike]{model.headLike}\tabcellsep groups elements used to provide a title or heading at the start of a text division.\end{longtable} \par
  \par 
\begin{longtable}{P{0.145\textwidth}P{0.7050000000000001\textwidth}}
\hyperref[TEI.opener]{opener}\tabcellsep (opener) groups together dateline, byline, salutation, and similar phrases appearing as a preliminary group at the start of a division, especially of a letter.\\
\hyperref[TEI.signed]{signed}\tabcellsep (signature) contains the closing salutation, etc., appended to a foreword, dedicatory epistle, or other division of a text.\end{longtable} \par
 
\item [\textbf{model.divBottomPart}] groups elements which can occur only at the end of a text division. \par 
\begin{longtable}{P{0.2083657587548638\textwidth}P{0.6416342412451361\textwidth}}
\hyperref[TEI.closer]{closer}\tabcellsep (closer) groups together salutations, datelines, and similar phrases appearing as a final group at the end of a division, especially of a letter.\\
\hyperref[TEI.postscript]{postscript}\tabcellsep contains a postscript, e.g. to a letter.\\
\hyperref[TEI.signed]{signed}\tabcellsep (signature) contains the closing salutation, etc., appended to a foreword, dedicatory epistle, or other division of a text.\\
\hyperref[TEI.trailer]{trailer}\tabcellsep contains a closing title or footer appearing at the end of a division of a text.\end{longtable} \par
 
\item [\textbf{model.divWrapper}] groups elements which can appear at either top or bottom of a textual division. \par 
\begin{longtable}{P{0.15225669957686883\textwidth}P{0.6977433004231312\textwidth}}
\hyperref[TEI.argument]{argument}\tabcellsep (argument) contains a formal list or prose description of the topics addressed by a subdivision of a text.\\
\hyperref[TEI.byline]{byline}\tabcellsep (byline) contains the primary statement of responsibility given for a work on its title page or at the head or end of the work.\\
\hyperref[TEI.dateline]{dateline}\tabcellsep (dateline) contains a brief description of the place, date, time, etc. of production of a letter, newspaper story, or other work, prefixed or suffixed to it as a kind of heading or trailer.\\
\hyperref[TEI.docAuthor]{docAuthor}\tabcellsep (document author) contains the name of the author of the document, as given on the title page (often but not always contained in a byline).\\
\hyperref[TEI.docDate]{docDate}\tabcellsep (document date) contains the date of a document, as given on a title page or in a dateline.\\
\hyperref[TEI.epigraph]{epigraph}\tabcellsep (epigraph) contains a quotation, anonymous or attributed, appearing at the start or end of a section or on a title page.\\
\hyperref[TEI.meeting]{meeting}\tabcellsep contains the formalized descriptive title for a meeting or conference, for use in a bibliographic description for an item derived from such a meeting, or as a heading or preamble to publications emanating from it.\\
\hyperref[TEI.salute]{salute}\tabcellsep (salutation) contains a salutation or greeting prefixed to a foreword, dedicatory epistle, or other division of a text, or the salutation in the closing of a letter, preface, etc.\end{longtable} \par
 
\end{sansreflist}

\subsubsection[{Headings and Trailers}]{Headings and Trailers}\label{DSHD}\par
The \hyperref[TEI.head]{<head>} element is used to identify a heading prefixed to the start of any textual division, at any level. A given division may contain more than one such element, as in the following example: \par\bgroup\index{div1=<div1>|exampleindex}\index{n=@n!<div1>|exampleindex}\index{head=<head>|exampleindex}\index{head=<head>|exampleindex}\index{p=<p>|exampleindex}\exampleFont \begin{shaded}\noindent\mbox{}{<\textbf{div1}\hspace*{1em}{n}="{Etym}">}\mbox{}\newline 
\hspace*{1em}{<\textbf{head}>}Etymology{</\textbf{head}>}\mbox{}\newline 
\hspace*{1em}{<\textbf{head}>}(Supplied by a late consumptive usher to a\mbox{}\newline 
\hspace*{1em}\hspace*{1em} grammar school){</\textbf{head}>}\mbox{}\newline 
\hspace*{1em}{<\textbf{p}>}The pale Usher — threadbare in coat, heart,\mbox{}\newline 
\hspace*{1em}\hspace*{1em} body and brain; I see him now. He was ever\mbox{}\newline 
\hspace*{1em}\hspace*{1em} dusting his old lexicons and grammars, ...{</\textbf{p}>}\mbox{}\newline 
{</\textbf{div1}>}\end{shaded}\egroup\par \noindent  \par
Unlike some other markup schemes, the TEI scheme does \textit{not} require that headings attached to textual subdivisions at different hierarchic levels have different identifiers. All kinds of heading are marked identically using the \hyperref[TEI.head]{<head>} tag; the type or level of heading intended is implied by the immediate parent of the \hyperref[TEI.head]{<head>} element, which may for example be a \hyperref[TEI.div1]{<div1>}, \hyperref[TEI.div2]{<div2>}, etc., an un-numbered \hyperref[TEI.div]{<div>}, or any member of the \textsf{model.listLike} class. However, as with \hyperref[TEI.div]{<div>} elements, the encoder may choose to extend the \textsf{model.headLike} class of which \hyperref[TEI.head]{<head>} is the sole member to include other such elements if required.\par
In certain kinds of text (notably newspapers), there may be a need to categorize individual headings within the sequence at the start of a division, for example as ‘main’ headings, or ‘detail’ headings: this may readily be done using the {\itshape type} or {\itshape subtype} attribute. Specific elements are provided for certain kinds of heading-like features, (notably \hyperref[TEI.byline]{<byline>}, \hyperref[TEI.dateline]{<dateline>}, and \hyperref[TEI.salute]{<salute>}; see further section \textit{\hyperref[DSOC]{4.2.2.\ Openers and Closers}}), but the {\itshape type} or {\itshape subtype} attributes must be used to discriminate among other forms of heading. These attributes are provided, as elsewhere, by the \textsf{att.typed} attribute class of which the \hyperref[TEI.head]{<head>} element is a member.\par
In the following example, taken from a British newspaper, the lead story and its associated headlines have been encoded as a \hyperref[TEI.div]{<div>} element, with appropriate \textsf{model.divTop} elements attached: \par\bgroup\index{div=<div>|exampleindex}\index{type=@type!<div>|exampleindex}\index{head=<head>|exampleindex}\index{rend=@rend!<head>|exampleindex}\index{type=@type!<head>|exampleindex}\index{head=<head>|exampleindex}\index{rend=@rend!<head>|exampleindex}\index{type=@type!<head>|exampleindex}\index{byline=<byline>|exampleindex}\index{p=<p>|exampleindex}\exampleFont \begin{shaded}\noindent\mbox{}{<\textbf{div}\hspace*{1em}{type}="{story}">}\mbox{}\newline 
\hspace*{1em}{<\textbf{head}\hspace*{1em}{rend}="{underlined}"\hspace*{1em}{type}="{sub}">}President pledges safeguards for 2,400 British\mbox{}\newline 
\hspace*{1em}\hspace*{1em} troops in Bosnia{</\textbf{head}>}\mbox{}\newline 
\hspace*{1em}{<\textbf{head}\hspace*{1em}{rend}="{scream}"\hspace*{1em}{type}="{main}">}Major agrees to enforced no-fly zone{</\textbf{head}>}\mbox{}\newline 
\hspace*{1em}{<\textbf{byline}>}By George Jones, Political Editor, in Washington{</\textbf{byline}>}\mbox{}\newline 
\hspace*{1em}{<\textbf{p}>}Greater Western intervention in the conflict in\mbox{}\newline 
\hspace*{1em}\hspace*{1em} former Yugoslavia was pledged by President Bush ...{</\textbf{p}>}\mbox{}\newline 
{</\textbf{div}>}\end{shaded}\egroup\par \noindent  \par
In older writings, the headings or \textit{incipits} may be longer than in modern works. When heading-like material appears in the middle of a text, the encoder must decide whether or not to treat it as the start of a new division. If the phrase in question appears to be more closely connected with what follows than with what precedes it, then it may be regarded as a heading and tagged as the \hyperref[TEI.head]{<head>} of a new \hyperref[TEI.div]{<div>} element. If it appears to be simply inserted or superimposed—as for example the kind of ‘pull quotes’ often found in newspapers or magazines, then the \hyperref[TEI.quote]{<quote>}, \hyperref[TEI.q]{<q>}, or \hyperref[TEI.cit]{<cit>} element may be more appropriate.\par
The \hyperref[TEI.trailer]{<trailer>} element, which can appear at the end of a division only, is used to mark any heading-like feature appearing in this position, as in this example: \par\bgroup\index{div=<div>|exampleindex}\index{type=@type!<div>|exampleindex}\index{n=@n!<div>|exampleindex}\index{head=<head>|exampleindex}\index{div=<div>|exampleindex}\index{head=<head>|exampleindex}\index{list=<list>|exampleindex}\index{item=<item>|exampleindex}\index{div=<div>|exampleindex}\index{head=<head>|exampleindex}\index{p=<p>|exampleindex}\index{p=<p>|exampleindex}\index{trailer=<trailer>|exampleindex}\exampleFont \begin{shaded}\noindent\mbox{}{<\textbf{div}\hspace*{1em}{type}="{book}"\hspace*{1em}{n}="{I}">}\mbox{}\newline 
\hspace*{1em}{<\textbf{head}>}In the name of Christ here begins the\mbox{}\newline 
\hspace*{1em}\hspace*{1em} first book of the ecclesiastical history of Georgius Florentinus,\mbox{}\newline 
\hspace*{1em}\hspace*{1em} known as Gregory, Bishop of Tours.{</\textbf{head}>}\mbox{}\newline 
\hspace*{1em}{<\textbf{div}>}\mbox{}\newline 
\hspace*{1em}\hspace*{1em}{<\textbf{head}>}Chapter Headings{</\textbf{head}>}\mbox{}\newline 
\hspace*{1em}\hspace*{1em}{<\textbf{list}>}\mbox{}\newline 
\hspace*{1em}\hspace*{1em}\hspace*{1em}{<\textbf{item}>}\mbox{}\newline 
\textit{<!-- chapter head -->}\mbox{}\newline 
\hspace*{1em}\hspace*{1em}\hspace*{1em}{</\textbf{item}>}\mbox{}\newline 
\textit{<!-- further chapter heads omitted -->}\mbox{}\newline 
\hspace*{1em}\hspace*{1em}{</\textbf{list}>}\mbox{}\newline 
\hspace*{1em}{</\textbf{div}>}\mbox{}\newline 
\hspace*{1em}{<\textbf{div}>}\mbox{}\newline 
\hspace*{1em}\hspace*{1em}{<\textbf{head}>}In the name of Christ here begins Book I of the history.{</\textbf{head}>}\mbox{}\newline 
\hspace*{1em}\hspace*{1em}{<\textbf{p}>}Proposing as I do ...{</\textbf{p}>}\mbox{}\newline 
\hspace*{1em}\hspace*{1em}{<\textbf{p}>}From the Passion of our Lord until the death of Saint Martin four\mbox{}\newline 
\hspace*{1em}\hspace*{1em}\hspace*{1em}\hspace*{1em} hundred and twelve years passed.{</\textbf{p}>}\mbox{}\newline 
\hspace*{1em}\hspace*{1em}{<\textbf{trailer}>}Here ends the first Book, which covers five thousand, five\mbox{}\newline 
\hspace*{1em}\hspace*{1em}\hspace*{1em}\hspace*{1em} hundred and ninety-six years from the beginning of the world down\mbox{}\newline 
\hspace*{1em}\hspace*{1em}\hspace*{1em}\hspace*{1em} to the death of Saint Martin.{</\textbf{trailer}>}\mbox{}\newline 
\hspace*{1em}{</\textbf{div}>}\mbox{}\newline 
{</\textbf{div}>}\end{shaded}\egroup\par 
\subsubsection[{Openers and Closers}]{Openers and Closers}\label{DSOC}\par
In addition to headings of various kinds, divisions sometimes include more or less formulaic opening or closing passages, typically conveying such information as the name and address of the person to whom the division is addressed, the place or time of its production, a salutation or exhortation to the reader, and so on. Divisions in epistolary form are particularly liable to include such features. Additional elements for the detailed encoding of personal names, dates, and places are provided in chapter \textit{\hyperref[ND]{13.\ Names, Dates, People, and Places}}. For simple cases, the following elements should be adequate: 
\begin{sansreflist}
  
\item [\textbf{<byline>}] (byline) contains the primary statement of responsibility given for a work on its title page or at the head or end of the work.
\item [\textbf{<dateline>}] (dateline) contains a brief description of the place, date, time, etc. of production of a letter, newspaper story, or other work, prefixed or suffixed to it as a kind of heading or trailer.
\item [\textbf{<salute>}] (salutation) contains a salutation or greeting prefixed to a foreword, dedicatory epistle, or other division of a text, or the salutation in the closing of a letter, preface, etc.
\item [\textbf{<signed>}] (signature) contains the closing salutation, etc., appended to a foreword, dedicatory epistle, or other division of a text.
\end{sansreflist}
\par
The \hyperref[TEI.byline]{<byline>} and \hyperref[TEI.dateline]{<dateline>} elements are used to encode headings which identify the authorship and provenance of a division. Although the terminology derives from newspaper usage, there is no implication that \hyperref[TEI.dateline]{<dateline>} or \hyperref[TEI.byline]{<byline>} elements apply only to newspaper texts. The following example illustrates use of the \hyperref[TEI.dateline]{<dateline>} and \hyperref[TEI.signed]{<signed>} elements at the end of the preface to a novel: \par\bgroup\index{div=<div>|exampleindex}\index{type=@type!<div>|exampleindex}\index{head=<head>|exampleindex}\index{p=<p>|exampleindex}\index{closer=<closer>|exampleindex}\index{signed=<signed>|exampleindex}\index{dateline=<dateline>|exampleindex}\exampleFont \begin{shaded}\noindent\mbox{}{<\textbf{div}\hspace*{1em}{type}="{preface}">}\mbox{}\newline 
\hspace*{1em}{<\textbf{head}>}To Henry Hope.{</\textbf{head}>}\mbox{}\newline 
\hspace*{1em}{<\textbf{p}>}It is not because this volume was conceived and partly\mbox{}\newline 
\hspace*{1em}\hspace*{1em} executed amid the glades and galleries of the Deepdene,\mbox{}\newline 
\hspace*{1em}\hspace*{1em} that I have inscribed it with your name. ... I shall find a\mbox{}\newline 
\hspace*{1em}\hspace*{1em} reflex to their efforts in your own generous spirit and\mbox{}\newline 
\hspace*{1em}\hspace*{1em} enlightened mind.\mbox{}\newline 
\hspace*{1em}{</\textbf{p}>}\mbox{}\newline 
\hspace*{1em}{<\textbf{closer}>}\mbox{}\newline 
\hspace*{1em}\hspace*{1em}{<\textbf{signed}\hspace*{1em}{xml:lang}="{el}">}D.{</\textbf{signed}>}\mbox{}\newline 
\hspace*{1em}\hspace*{1em}{<\textbf{dateline}>}Grosvenor Gate, May-Day, 1844{</\textbf{dateline}>}\mbox{}\newline 
\hspace*{1em}{</\textbf{closer}>}\mbox{}\newline 
{</\textbf{div}>}\end{shaded}\egroup\par \noindent  \par
Where a sequence of such elements appear together, either at the beginning or end of an element, it may be convenient to group them together using one of the following elements: 
\begin{sansreflist}
  
\item [\textbf{<opener>}] (opener) groups together dateline, byline, salutation, and similar phrases appearing as a preliminary group at the start of a division, especially of a letter.
\item [\textbf{<closer>}] (closer) groups together salutations, datelines, and similar phrases appearing as a final group at the end of a division, especially of a letter.
\end{sansreflist}
 The following examples demonstrate the use of the \hyperref[TEI.opener]{<opener>} and \hyperref[TEI.closer]{<closer>} grouping elements: \par\bgroup\index{div=<div>|exampleindex}\index{type=@type!<div>|exampleindex}\index{n=@n!<div>|exampleindex}\index{head=<head>|exampleindex}\index{head=<head>|exampleindex}\index{div=<div>|exampleindex}\index{type=@type!<div>|exampleindex}\index{n=@n!<div>|exampleindex}\index{opener=<opener>|exampleindex}\index{dateline=<dateline>|exampleindex}\index{name=<name>|exampleindex}\index{type=@type!<name>|exampleindex}\index{date=<date>|exampleindex}\index{salute=<salute>|exampleindex}\index{name=<name>|exampleindex}\index{p=<p>|exampleindex}\index{closer=<closer>|exampleindex}\index{salute=<salute>|exampleindex}\index{signed=<signed>|exampleindex}\index{name=<name>|exampleindex}\exampleFont \begin{shaded}\noindent\mbox{}{<\textbf{div}\hspace*{1em}{type}="{narrative}"\hspace*{1em}{n}="{6}">}\mbox{}\newline 
\hspace*{1em}{<\textbf{head}>}Sixth Narrative{</\textbf{head}>}\mbox{}\newline 
\hspace*{1em}{<\textbf{head}>}contributed by Sergeant Cuff{</\textbf{head}>}\mbox{}\newline 
\hspace*{1em}{<\textbf{div}\hspace*{1em}{type}="{fragment}"\hspace*{1em}{n}="{6.1}">}\mbox{}\newline 
\hspace*{1em}\hspace*{1em}{<\textbf{opener}>}\mbox{}\newline 
\hspace*{1em}\hspace*{1em}\hspace*{1em}{<\textbf{dateline}>}\mbox{}\newline 
\hspace*{1em}\hspace*{1em}\hspace*{1em}\hspace*{1em}{<\textbf{name}\hspace*{1em}{type}="{place}">}Dorking, Surrey,{</\textbf{name}>}\mbox{}\newline 
\hspace*{1em}\hspace*{1em}\hspace*{1em}\hspace*{1em}{<\textbf{date}>}July 30th, 1849{</\textbf{date}>}\mbox{}\newline 
\hspace*{1em}\hspace*{1em}\hspace*{1em}{</\textbf{dateline}>}\mbox{}\newline 
\hspace*{1em}\hspace*{1em}\hspace*{1em}{<\textbf{salute}>}To {<\textbf{name}>}Franklin Blake, Esq.{</\textbf{name}>} Sir, —{</\textbf{salute}>}\mbox{}\newline 
\hspace*{1em}\hspace*{1em}{</\textbf{opener}>}\mbox{}\newline 
\hspace*{1em}\hspace*{1em}{<\textbf{p}>}I beg to apologize for the delay that has occurred in the\mbox{}\newline 
\hspace*{1em}\hspace*{1em}\hspace*{1em}\hspace*{1em} production of the Report, with which I engaged to furnish you.\mbox{}\newline 
\hspace*{1em}\hspace*{1em}\hspace*{1em}\hspace*{1em} I have waited to make it a complete Report ...{</\textbf{p}>}\mbox{}\newline 
\hspace*{1em}\hspace*{1em}{<\textbf{closer}>}\mbox{}\newline 
\hspace*{1em}\hspace*{1em}\hspace*{1em}{<\textbf{salute}>}I have the honour to remain, dear sir, your\mbox{}\newline 
\hspace*{1em}\hspace*{1em}\hspace*{1em}\hspace*{1em}\hspace*{1em}\hspace*{1em} obedient servant {</\textbf{salute}>}\mbox{}\newline 
\hspace*{1em}\hspace*{1em}\hspace*{1em}{<\textbf{signed}>}\mbox{}\newline 
\hspace*{1em}\hspace*{1em}\hspace*{1em}\hspace*{1em}{<\textbf{name}>}RICHARD CUFF{</\textbf{name}>} (late sergeant in the\mbox{}\newline 
\hspace*{1em}\hspace*{1em}\hspace*{1em}\hspace*{1em}\hspace*{1em}\hspace*{1em} Detective Force, Scotland Yard, London). {</\textbf{signed}>}\mbox{}\newline 
\hspace*{1em}\hspace*{1em}{</\textbf{closer}>}\mbox{}\newline 
\hspace*{1em}{</\textbf{div}>}\mbox{}\newline 
{</\textbf{div}>}\end{shaded}\egroup\par \noindent   \par\bgroup\index{div=<div>|exampleindex}\index{type=@type!<div>|exampleindex}\index{n=@n!<div>|exampleindex}\index{head=<head>|exampleindex}\index{opener=<opener>|exampleindex}\index{dateline=<dateline>|exampleindex}\index{p=<p>|exampleindex}\index{p=<p>|exampleindex}\index{closer=<closer>|exampleindex}\index{salute=<salute>|exampleindex}\index{signed=<signed>|exampleindex}\exampleFont \begin{shaded}\noindent\mbox{}{<\textbf{div}\hspace*{1em}{type}="{letter}"\hspace*{1em}{n}="{14}">}\mbox{}\newline 
\hspace*{1em}{<\textbf{head}>}Letter XIV: Miss Clarissa Harlowe to Miss Howe{</\textbf{head}>}\mbox{}\newline 
\hspace*{1em}{<\textbf{opener}>}\mbox{}\newline 
\hspace*{1em}\hspace*{1em}{<\textbf{dateline}>}Thursday evening, March 2.{</\textbf{dateline}>}\mbox{}\newline 
\hspace*{1em}{</\textbf{opener}>}\mbox{}\newline 
\hspace*{1em}{<\textbf{p}>}On Hannah's depositing my long letter ...{</\textbf{p}>}\mbox{}\newline 
\hspace*{1em}{<\textbf{p}>}An interruption obliges me to conclude myself\mbox{}\newline 
\hspace*{1em}\hspace*{1em} in some hurry, as well as fright, what I must ever be,{</\textbf{p}>}\mbox{}\newline 
\hspace*{1em}{<\textbf{closer}>}\mbox{}\newline 
\hspace*{1em}\hspace*{1em}{<\textbf{salute}>}Yours more than my own,{</\textbf{salute}>}\mbox{}\newline 
\hspace*{1em}\hspace*{1em}{<\textbf{signed}>}Clarissa Harlowe{</\textbf{signed}>}\mbox{}\newline 
\hspace*{1em}{</\textbf{closer}>}\mbox{}\newline 
{</\textbf{div}>}\end{shaded}\egroup\par \noindent  \par
For further discussion of the encoding of dates and of names of persons and places, see section \textit{\hyperref[CONADA]{3.6.4.\ Dates and Times}} and chapter \textit{\hyperref[ND]{13.\ Names, Dates, People, and Places}}.
\subsubsection[{Arguments, Epigraphs, and Postscripts}]{Arguments, Epigraphs, and Postscripts}\label{DSAE}\par
The \hyperref[TEI.argument]{<argument>} element may be used to encode the prefatory list of topics sometimes found at the start of a chapter or other division. It is most conveniently encoded as a list, since this allows each item to be distinguished, but may also simply be presented as a paragraph. The following are thus both equally valid ways of encoding the same argument: \par\bgroup\index{div=<div>|exampleindex}\index{type=@type!<div>|exampleindex}\index{n=@n!<div>|exampleindex}\index{argument=<argument>|exampleindex}\index{p=<p>|exampleindex}\index{p=<p>|exampleindex}\exampleFont \begin{shaded}\noindent\mbox{}{<\textbf{div}\hspace*{1em}{type}="{chap}"\hspace*{1em}{n}="{6}">}\mbox{}\newline 
\hspace*{1em}{<\textbf{argument}>}\mbox{}\newline 
\hspace*{1em}\hspace*{1em}{<\textbf{p}>}Kingston — Instructive remarks on early English history\mbox{}\newline 
\hspace*{1em}\hspace*{1em}\hspace*{1em}\hspace*{1em} — Instructive observations on carved oak and life in general\mbox{}\newline 
\hspace*{1em}\hspace*{1em}\hspace*{1em}\hspace*{1em} — Sad case of Stivvings, junior — Musings on antiquity\mbox{}\newline 
\hspace*{1em}\hspace*{1em}\hspace*{1em}\hspace*{1em} — I forget that I am steering — Interesting result\mbox{}\newline 
\hspace*{1em}\hspace*{1em}\hspace*{1em}\hspace*{1em} — Hampton Court Maze — Harris as a guide.{</\textbf{p}>}\mbox{}\newline 
\hspace*{1em}{</\textbf{argument}>}\mbox{}\newline 
\hspace*{1em}{<\textbf{p}>}It was a glorious morning, late spring or early summer, as you\mbox{}\newline 
\hspace*{1em}\hspace*{1em} care to take it ...{</\textbf{p}>}\mbox{}\newline 
{</\textbf{div}>}\end{shaded}\egroup\par \noindent  \par\bgroup\index{div=<div>|exampleindex}\index{type=@type!<div>|exampleindex}\index{n=@n!<div>|exampleindex}\index{argument=<argument>|exampleindex}\index{list=<list>|exampleindex}\index{type=@type!<list>|exampleindex}\index{item=<item>|exampleindex}\index{item=<item>|exampleindex}\index{item=<item>|exampleindex}\index{item=<item>|exampleindex}\index{item=<item>|exampleindex}\index{item=<item>|exampleindex}\index{item=<item>|exampleindex}\index{item=<item>|exampleindex}\index{item=<item>|exampleindex}\index{p=<p>|exampleindex}\exampleFont \begin{shaded}\noindent\mbox{}{<\textbf{div}\hspace*{1em}{type}="{chap}"\hspace*{1em}{n}="{6}">}\mbox{}\newline 
\hspace*{1em}{<\textbf{argument}>}\mbox{}\newline 
\hspace*{1em}\hspace*{1em}{<\textbf{list}\hspace*{1em}{type}="{inline}">}\mbox{}\newline 
\hspace*{1em}\hspace*{1em}\hspace*{1em}{<\textbf{item}>}Kingston{</\textbf{item}>}\mbox{}\newline 
\hspace*{1em}\hspace*{1em}\hspace*{1em}{<\textbf{item}>}Instructive remarks on early English history{</\textbf{item}>}\mbox{}\newline 
\hspace*{1em}\hspace*{1em}\hspace*{1em}{<\textbf{item}>}Instructive observations on carved oak and life in\mbox{}\newline 
\hspace*{1em}\hspace*{1em}\hspace*{1em}\hspace*{1em}\hspace*{1em}\hspace*{1em} general{</\textbf{item}>}\mbox{}\newline 
\hspace*{1em}\hspace*{1em}\hspace*{1em}{<\textbf{item}>}Sad case of Stivvings, junior{</\textbf{item}>}\mbox{}\newline 
\hspace*{1em}\hspace*{1em}\hspace*{1em}{<\textbf{item}>}Musings on antiquity{</\textbf{item}>}\mbox{}\newline 
\hspace*{1em}\hspace*{1em}\hspace*{1em}{<\textbf{item}>}I forget that I am steering{</\textbf{item}>}\mbox{}\newline 
\hspace*{1em}\hspace*{1em}\hspace*{1em}{<\textbf{item}>}Interesting result{</\textbf{item}>}\mbox{}\newline 
\hspace*{1em}\hspace*{1em}\hspace*{1em}{<\textbf{item}>}Hampton Court Maze{</\textbf{item}>}\mbox{}\newline 
\hspace*{1em}\hspace*{1em}\hspace*{1em}{<\textbf{item}>}Harris as a guide.{</\textbf{item}>}\mbox{}\newline 
\hspace*{1em}\hspace*{1em}{</\textbf{list}>}\mbox{}\newline 
\hspace*{1em}{</\textbf{argument}>}\mbox{}\newline 
\hspace*{1em}{<\textbf{p}>}It was a glorious morning, late spring or early summer, as you\mbox{}\newline 
\hspace*{1em}\hspace*{1em} care to take it ...{</\textbf{p}>}\mbox{}\newline 
{</\textbf{div}>}\end{shaded}\egroup\par \par
An \textit{epigraph} is a quotation from some other work, a saying, or a motto, appearing on a title page, or at the start of a division. It may be encoded using the special-purpose \hyperref[TEI.epigraph]{<epigraph>} element, as in the following example: \par\bgroup\index{titlePage=<titlePage>|exampleindex}\index{docAuthor=<docAuthor>|exampleindex}\index{docTitle=<docTitle>|exampleindex}\index{titlePart=<titlePart>|exampleindex}\index{epigraph=<epigraph>|exampleindex}\index{q=<q>|exampleindex}\exampleFont \begin{shaded}\noindent\mbox{}{<\textbf{titlePage}>}\mbox{}\newline 
\hspace*{1em}{<\textbf{docAuthor}>}E. M. Forster{</\textbf{docAuthor}>}\mbox{}\newline 
\hspace*{1em}{<\textbf{docTitle}>}\mbox{}\newline 
\hspace*{1em}\hspace*{1em}{<\textbf{titlePart}>}Howards End{</\textbf{titlePart}>}\mbox{}\newline 
\hspace*{1em}{</\textbf{docTitle}>}\mbox{}\newline 
\hspace*{1em}{<\textbf{epigraph}>}\mbox{}\newline 
\hspace*{1em}\hspace*{1em}{<\textbf{q}>}Only connect...{</\textbf{q}>}\mbox{}\newline 
\hspace*{1em}{</\textbf{epigraph}>}\mbox{}\newline 
{</\textbf{titlePage}>}\end{shaded}\egroup\par \noindent  When an epigraph contains a quotation, this may often be associated with a bibliographic reference. In such cases, it is recommended additionally to group the quotation and its source together using the \hyperref[TEI.cit]{<cit>} element, as in the following example: \par\bgroup\index{div=<div>|exampleindex}\index{n=@n!<div>|exampleindex}\index{type=@type!<div>|exampleindex}\index{head=<head>|exampleindex}\index{epigraph=<epigraph>|exampleindex}\index{cit=<cit>|exampleindex}\index{quote=<quote>|exampleindex}\index{q=<q>|exampleindex}\index{bibl=<bibl>|exampleindex}\index{p=<p>|exampleindex}\exampleFont \begin{shaded}\noindent\mbox{}{<\textbf{div}\hspace*{1em}{n}="{19}"\hspace*{1em}{type}="{chap}">}\mbox{}\newline 
\hspace*{1em}{<\textbf{head}>}Chapter 19{</\textbf{head}>}\mbox{}\newline 
\hspace*{1em}{<\textbf{epigraph}>}\mbox{}\newline 
\hspace*{1em}\hspace*{1em}{<\textbf{cit}>}\mbox{}\newline 
\hspace*{1em}\hspace*{1em}\hspace*{1em}{<\textbf{quote}>}I pity the man who can travel\mbox{}\newline 
\hspace*{1em}\hspace*{1em}\hspace*{1em}\hspace*{1em}\hspace*{1em}\hspace*{1em} from Dan to Beersheba, and say {<\textbf{q}>}'Tis all\mbox{}\newline 
\hspace*{1em}\hspace*{1em}\hspace*{1em}\hspace*{1em}\hspace*{1em}\hspace*{1em}\hspace*{1em}\hspace*{1em} barren;{</\textbf{q}>} and so is all the world to him\mbox{}\newline 
\hspace*{1em}\hspace*{1em}\hspace*{1em}\hspace*{1em}\hspace*{1em}\hspace*{1em} who will not cultivate the fruits it offers.\mbox{}\newline 
\hspace*{1em}\hspace*{1em}\hspace*{1em}{</\textbf{quote}>}\mbox{}\newline 
\hspace*{1em}\hspace*{1em}\hspace*{1em}{<\textbf{bibl}>}Sterne: Sentimental Journey.{</\textbf{bibl}>}\mbox{}\newline 
\hspace*{1em}\hspace*{1em}{</\textbf{cit}>}\mbox{}\newline 
\hspace*{1em}{</\textbf{epigraph}>}\mbox{}\newline 
\hspace*{1em}{<\textbf{p}>}To say that Deronda was romantic would be to\mbox{}\newline 
\hspace*{1em}\hspace*{1em} misrepresent him: but under his calm and somewhat\mbox{}\newline 
\hspace*{1em}\hspace*{1em} self-repressed exterior ...{</\textbf{p}>}\mbox{}\newline 
{</\textbf{div}>}\end{shaded}\egroup\par \noindent  \par
For discussion of quotations appearing other than as epigraphs refer to section \textit{\hyperref[COHQQ]{3.3.3.\ Quotation}}.\par
A \textit{postscript} is a passage added after the signature of a letter or, less frequently, the main portion of the body of a book, article, or essay. In English a postscript is often abbreviated as \textit{P.S.} or \textit{PS}, and postscripts are often introduced by labels with one of these abbreviations, as in the following example. \par\bgroup\index{div=<div>|exampleindex}\index{type=@type!<div>|exampleindex}\index{opener=<opener>|exampleindex}\index{dateline=<dateline>|exampleindex}\index{placeName=<placeName>|exampleindex}\index{date=<date>|exampleindex}\index{when=@when!<date>|exampleindex}\index{salute=<salute>|exampleindex}\index{p=<p>|exampleindex}\index{lb=<lb>|exampleindex}\index{lb=<lb>|exampleindex}\index{lb=<lb>|exampleindex}\index{lb=<lb>|exampleindex}\index{lb=<lb>|exampleindex}\index{lb=<lb>|exampleindex}\index{lb=<lb>|exampleindex}\index{lb=<lb>|exampleindex}\index{unclear=<unclear>|exampleindex}\index{lb=<lb>|exampleindex}\index{lb=<lb>|exampleindex}\index{lb=<lb>|exampleindex}\index{closer=<closer>|exampleindex}\index{salute=<salute>|exampleindex}\index{signed=<signed>|exampleindex}\index{postscript=<postscript>|exampleindex}\index{label=<label>|exampleindex}\index{p=<p>|exampleindex}\index{lb=<lb>|exampleindex}\index{lb=<lb>|exampleindex}\index{lb=<lb>|exampleindex}\exampleFont \begin{shaded}\noindent\mbox{}{<\textbf{div}\hspace*{1em}{type}="{letter}">}\mbox{}\newline 
\hspace*{1em}{<\textbf{opener}>}\mbox{}\newline 
\hspace*{1em}\hspace*{1em}{<\textbf{dateline}>}\mbox{}\newline 
\hspace*{1em}\hspace*{1em}\hspace*{1em}{<\textbf{placeName}>}Newport{</\textbf{placeName}>}\mbox{}\newline 
\hspace*{1em}\hspace*{1em}\hspace*{1em}{<\textbf{date}\hspace*{1em}{when}="{1761-05-27}">}May ye 27th 1761{</\textbf{date}>}\mbox{}\newline 
\hspace*{1em}\hspace*{1em}{</\textbf{dateline}>}\mbox{}\newline 
\hspace*{1em}\hspace*{1em}{<\textbf{salute}>}Gentlemen{</\textbf{salute}>}\mbox{}\newline 
\hspace*{1em}{</\textbf{opener}>}\mbox{}\newline 
\hspace*{1em}{<\textbf{p}>}Capt Stoddard's Business\mbox{}\newline 
\hspace*{1em}{<\textbf{lb}/>}calling him to Providence, have\mbox{}\newline 
\hspace*{1em}{<\textbf{lb}/>}got him to look at Hopkins brigantine\mbox{}\newline 
\hspace*{1em}{<\textbf{lb}/>}\& if can agree to Purchase her, shall\mbox{}\newline 
\hspace*{1em}{<\textbf{lb}/>}be much oblig'd for your further\mbox{}\newline 
\hspace*{1em}{<\textbf{lb}/>}assistance herein, \& will acquiesce with\mbox{}\newline 
\hspace*{1em}{<\textbf{lb}/>}whatever you \& he shall Contract\mbox{}\newline 
\hspace*{1em}{<\textbf{lb}/>}for — I Thank you for your\mbox{}\newline 
\hspace*{1em}{<\textbf{lb}/>}\mbox{}\newline 
\hspace*{1em}\hspace*{1em}{<\textbf{unclear}>}Line{</\textbf{unclear}>} respecting the brigantine \& Beg\mbox{}\newline 
\hspace*{1em}{<\textbf{lb}/>}leave to Recommend the Bearer\mbox{}\newline 
\hspace*{1em}{<\textbf{lb}/>}to you for your advice \& Friendship\mbox{}\newline 
\hspace*{1em}{<\textbf{lb}/>}in this matter{</\textbf{p}>}\mbox{}\newline 
\hspace*{1em}{<\textbf{closer}>}\mbox{}\newline 
\hspace*{1em}\hspace*{1em}{<\textbf{salute}>}I am your most humble servant{</\textbf{salute}>}\mbox{}\newline 
\hspace*{1em}\hspace*{1em}{<\textbf{signed}>}Joseph Wanton Jr{</\textbf{signed}>}\mbox{}\newline 
\hspace*{1em}{</\textbf{closer}>}\mbox{}\newline 
\hspace*{1em}{<\textbf{postscript}>}\mbox{}\newline 
\hspace*{1em}\hspace*{1em}{<\textbf{label}>}P.S.{</\textbf{label}>}\mbox{}\newline 
\hspace*{1em}\hspace*{1em}{<\textbf{p}>}I have Mollases, Sugar,\mbox{}\newline 
\hspace*{1em}\hspace*{1em}{<\textbf{lb}/>}Coffee \& Rum, which\mbox{}\newline 
\hspace*{1em}\hspace*{1em}{<\textbf{lb}/>}will Exchange with you\mbox{}\newline 
\hspace*{1em}\hspace*{1em}{<\textbf{lb}/>}for Candles or Oyl{</\textbf{p}>}\mbox{}\newline 
\hspace*{1em}{</\textbf{postscript}>}\mbox{}\newline 
{</\textbf{div}>}\end{shaded}\egroup\par \noindent  
\subsubsection[{Content of Textual Divisions}]{Content of Textual Divisions}\label{DSCO}\par
Other than elements from the \textsf{model.divWrapper}, \textsf{model.divTop}, or \textsf{model.divBottom} classes, every textual division (numbered or un-numbered) consists of a sequence of ungrouped \textsf{macro.component} elements (see \textit{\hyperref[STEC]{1.3.\ The TEI Class System}}). The actual elements available will depend on the modules in use; in all cases, at least the component-level structural elements defined in the core will be available (paragraphs, lists, dramatic speeches, verse lines and line groups etc.). If the drama module has been selected, then other component- or phrase- level items specialized for performance texts (for example, cast lists or camera angles) will be available, as defined in chapter \textit{\hyperref[DR]{7.\ Performance Texts}}) will be available. If the dictionary module is in use, then dictionary entries, related entries, etc. (as defined in chapter \textit{\hyperref[DI]{9.\ Dictionaries}}) will also be available; if the module for transcribed speech is in use, then utterances, pauses, vocals, kinesics, etc., as defined in chapter \textit{\hyperref[TSBA]{8.3.\ Elements Unique to Spoken Texts}} will be available; and so on.\par
Where a text contains low-level elements from more than one module these may appear at any point; there is no requirement that elements from the same module be kept together.
\subsection[{Grouped and Floating Texts}]{Grouped and Floating Texts}\label{DSGRPF}\par
The \hyperref[TEI.group]{<group>} element discussed in \textit{\hyperref[DSGRP]{4.3.1.\ Grouped Texts}} should be used to represent a collection of independent texts which is to be regarded as a single unit for processing or other purposes. The \hyperref[TEI.floatingText]{<floatingText>} element discussed in \textit{\hyperref[DSFLT]{4.3.2.\ Floating Texts}} should be used to represent an independent text which interrupts the text containing it at any point but after which the surrounding text resumes. 
\begin{sansreflist}
  
\item [\textbf{<group>}] (group) contains the body of a composite text, grouping together a sequence of distinct texts (or groups of such texts) which are regarded as a unit for some purpose, for example the collected works of an author, a sequence of prose essays, etc.
\item [\textbf{<floatingText>}] (floating text) contains a single text of any kind, whether unitary or composite, which interrupts the text containing it at any point and after which the surrounding text resumes.
\end{sansreflist}

\subsubsection[{Grouped Texts}]{Grouped Texts}\label{DSGRP}\par
Examples of composite texts which should be represented using the \hyperref[TEI.group]{<group>} element include anthologies and other collections. The presence of common front matter referring to the whole collection, possibly in addition to front matter relating to each individual text, is a good indication that a given text might usefully be encoded in this way; this structure may be found useful in other circumstances too.\par
For example, the overall structure of a collection of short stories might be encoded as follows: \par\bgroup\index{text=<text>|exampleindex}\index{front=<front>|exampleindex}\index{docTitle=<docTitle>|exampleindex}\index{titlePart=<titlePart>|exampleindex}\index{docImprint=<docImprint>|exampleindex}\index{title=<title>|exampleindex}\index{group=<group>|exampleindex}\index{text=<text>|exampleindex}\index{front=<front>|exampleindex}\index{head=<head>|exampleindex}\index{rend=@rend!<head>|exampleindex}\index{docTitle=<docTitle>|exampleindex}\index{titlePart=<titlePart>|exampleindex}\index{titlePart=<titlePart>|exampleindex}\index{byline=<byline>|exampleindex}\index{body=<body>|exampleindex}\index{p=<p>|exampleindex}\index{emph=<emph>|exampleindex}\index{text=<text>|exampleindex}\index{front=<front>|exampleindex}\index{head=<head>|exampleindex}\index{rend=@rend!<head>|exampleindex}\index{docTitle=<docTitle>|exampleindex}\index{titlePart=<titlePart>|exampleindex}\index{titlePart=<titlePart>|exampleindex}\index{byline=<byline>|exampleindex}\index{body=<body>|exampleindex}\index{p=<p>|exampleindex}\index{text=<text>|exampleindex}\index{front=<front>|exampleindex}\index{head=<head>|exampleindex}\index{rend=@rend!<head>|exampleindex}\index{docTitle=<docTitle>|exampleindex}\index{titlePart=<titlePart>|exampleindex}\index{titlePart=<titlePart>|exampleindex}\index{byline=<byline>|exampleindex}\index{body=<body>|exampleindex}\index{p=<p>|exampleindex}\index{q=<q>|exampleindex}\exampleFont \begin{shaded}\noindent\mbox{}{<\textbf{text}>}\mbox{}\newline 
\hspace*{1em}{<\textbf{front}>}\mbox{}\newline 
\hspace*{1em}\hspace*{1em}{<\textbf{docTitle}>}\mbox{}\newline 
\hspace*{1em}\hspace*{1em}\hspace*{1em}{<\textbf{titlePart}>} The Adventures of Sherlock Holmes\mbox{}\newline 
\hspace*{1em}\hspace*{1em}\hspace*{1em}{</\textbf{titlePart}>}\mbox{}\newline 
\hspace*{1em}\hspace*{1em}{</\textbf{docTitle}>}\mbox{}\newline 
\hspace*{1em}\hspace*{1em}{<\textbf{docImprint}>}First published in {<\textbf{title}>}The Strand{</\textbf{title}>}\mbox{}\newline 
\hspace*{1em}\hspace*{1em}\hspace*{1em}\hspace*{1em} between July 1891 and December 1892{</\textbf{docImprint}>}\mbox{}\newline 
\textit{<!-- any other front matter specific to this collection -->}\mbox{}\newline 
\hspace*{1em}{</\textbf{front}>}\mbox{}\newline 
\hspace*{1em}{<\textbf{group}>}\mbox{}\newline 
\hspace*{1em}\hspace*{1em}{<\textbf{text}>}\mbox{}\newline 
\hspace*{1em}\hspace*{1em}\hspace*{1em}{<\textbf{front}>}\mbox{}\newline 
\hspace*{1em}\hspace*{1em}\hspace*{1em}\hspace*{1em}{<\textbf{head}\hspace*{1em}{rend}="{italic}">}Adventures of Sherlock\mbox{}\newline 
\hspace*{1em}\hspace*{1em}\hspace*{1em}\hspace*{1em}\hspace*{1em}\hspace*{1em}\hspace*{1em}\hspace*{1em} Holmes{</\textbf{head}>}\mbox{}\newline 
\hspace*{1em}\hspace*{1em}\hspace*{1em}\hspace*{1em}{<\textbf{docTitle}>}\mbox{}\newline 
\hspace*{1em}\hspace*{1em}\hspace*{1em}\hspace*{1em}\hspace*{1em}{<\textbf{titlePart}>}Adventure I. —{</\textbf{titlePart}>}\mbox{}\newline 
\hspace*{1em}\hspace*{1em}\hspace*{1em}\hspace*{1em}\hspace*{1em}{<\textbf{titlePart}>}A Scandal in Bohemia{</\textbf{titlePart}>}\mbox{}\newline 
\hspace*{1em}\hspace*{1em}\hspace*{1em}\hspace*{1em}{</\textbf{docTitle}>}\mbox{}\newline 
\hspace*{1em}\hspace*{1em}\hspace*{1em}\hspace*{1em}{<\textbf{byline}>}By A. Conan Doyle.{</\textbf{byline}>}\mbox{}\newline 
\hspace*{1em}\hspace*{1em}\hspace*{1em}{</\textbf{front}>}\mbox{}\newline 
\hspace*{1em}\hspace*{1em}\hspace*{1em}{<\textbf{body}>}\mbox{}\newline 
\hspace*{1em}\hspace*{1em}\hspace*{1em}\hspace*{1em}{<\textbf{p}>}To Sherlock Holmes she is always\mbox{}\newline 
\hspace*{1em}\hspace*{1em}\hspace*{1em}\hspace*{1em}{<\textbf{emph}>}the{</\textbf{emph}>} woman. ... {</\textbf{p}>}\mbox{}\newline 
\textit{<!-- remainder of A Scandal in Bohemia here -->}\mbox{}\newline 
\hspace*{1em}\hspace*{1em}\hspace*{1em}{</\textbf{body}>}\mbox{}\newline 
\hspace*{1em}\hspace*{1em}{</\textbf{text}>}\mbox{}\newline 
\hspace*{1em}\hspace*{1em}{<\textbf{text}>}\mbox{}\newline 
\hspace*{1em}\hspace*{1em}\hspace*{1em}{<\textbf{front}>}\mbox{}\newline 
\hspace*{1em}\hspace*{1em}\hspace*{1em}\hspace*{1em}{<\textbf{head}\hspace*{1em}{rend}="{italic}">}Adventures of Sherlock Holmes{</\textbf{head}>}\mbox{}\newline 
\hspace*{1em}\hspace*{1em}\hspace*{1em}\hspace*{1em}{<\textbf{docTitle}>}\mbox{}\newline 
\hspace*{1em}\hspace*{1em}\hspace*{1em}\hspace*{1em}\hspace*{1em}{<\textbf{titlePart}>}Adventure II. —{</\textbf{titlePart}>}\mbox{}\newline 
\hspace*{1em}\hspace*{1em}\hspace*{1em}\hspace*{1em}\hspace*{1em}{<\textbf{titlePart}>}The Red-Headed League{</\textbf{titlePart}>}\mbox{}\newline 
\hspace*{1em}\hspace*{1em}\hspace*{1em}\hspace*{1em}{</\textbf{docTitle}>}\mbox{}\newline 
\hspace*{1em}\hspace*{1em}\hspace*{1em}\hspace*{1em}{<\textbf{byline}>}By A. Conan Doyle.{</\textbf{byline}>}\mbox{}\newline 
\hspace*{1em}\hspace*{1em}\hspace*{1em}{</\textbf{front}>}\mbox{}\newline 
\hspace*{1em}\hspace*{1em}\hspace*{1em}{<\textbf{body}>}\mbox{}\newline 
\hspace*{1em}\hspace*{1em}\hspace*{1em}\hspace*{1em}{<\textbf{p}>}I had called upon my friend, Mr. Sherlock Holmes, one day\mbox{}\newline 
\hspace*{1em}\hspace*{1em}\hspace*{1em}\hspace*{1em}\hspace*{1em}\hspace*{1em}\hspace*{1em}\hspace*{1em} in the autumn of last year and found him in deep conversation\mbox{}\newline 
\hspace*{1em}\hspace*{1em}\hspace*{1em}\hspace*{1em}\hspace*{1em}\hspace*{1em}\hspace*{1em}\hspace*{1em} with a very stout, florid-faced, elderly gentleman with fiery red hair …\mbox{}\newline 
\hspace*{1em}\hspace*{1em}\hspace*{1em}\hspace*{1em}{</\textbf{p}>}\mbox{}\newline 
\textit{<!-- remainder of The Red Headed League here -->}\mbox{}\newline 
\hspace*{1em}\hspace*{1em}\hspace*{1em}{</\textbf{body}>}\mbox{}\newline 
\hspace*{1em}\hspace*{1em}{</\textbf{text}>}\mbox{}\newline 
\hspace*{1em}\hspace*{1em}{<\textbf{text}>}\mbox{}\newline 
\hspace*{1em}\hspace*{1em}\hspace*{1em}{<\textbf{front}>}\mbox{}\newline 
\hspace*{1em}\hspace*{1em}\hspace*{1em}\hspace*{1em}{<\textbf{head}\hspace*{1em}{rend}="{italic}">}Adventures of Sherlock Holmes{</\textbf{head}>}\mbox{}\newline 
\hspace*{1em}\hspace*{1em}\hspace*{1em}\hspace*{1em}{<\textbf{docTitle}>}\mbox{}\newline 
\hspace*{1em}\hspace*{1em}\hspace*{1em}\hspace*{1em}\hspace*{1em}{<\textbf{titlePart}>}Adventure XII. —{</\textbf{titlePart}>}\mbox{}\newline 
\hspace*{1em}\hspace*{1em}\hspace*{1em}\hspace*{1em}\hspace*{1em}{<\textbf{titlePart}>}The Adventure of the Copper Beeches{</\textbf{titlePart}>}\mbox{}\newline 
\hspace*{1em}\hspace*{1em}\hspace*{1em}\hspace*{1em}{</\textbf{docTitle}>}\mbox{}\newline 
\hspace*{1em}\hspace*{1em}\hspace*{1em}\hspace*{1em}{<\textbf{byline}>}By A. Conan Doyle.{</\textbf{byline}>}\mbox{}\newline 
\hspace*{1em}\hspace*{1em}\hspace*{1em}{</\textbf{front}>}\mbox{}\newline 
\hspace*{1em}\hspace*{1em}\hspace*{1em}{<\textbf{body}>}\mbox{}\newline 
\hspace*{1em}\hspace*{1em}\hspace*{1em}\hspace*{1em}{<\textbf{p}>}\mbox{}\newline 
\hspace*{1em}\hspace*{1em}\hspace*{1em}\hspace*{1em}\hspace*{1em}{<\textbf{q}>}To the man who loves art for its\mbox{}\newline 
\hspace*{1em}\hspace*{1em}\hspace*{1em}\hspace*{1em}\hspace*{1em}\hspace*{1em}\hspace*{1em}\hspace*{1em}\hspace*{1em}\hspace*{1em} own sake,{</\textbf{q}>} remarked Sherlock Holmes ...\mbox{}\newline 
\hspace*{1em}\hspace*{1em}\hspace*{1em}\hspace*{1em}\hspace*{1em}\hspace*{1em}\hspace*{1em}\hspace*{1em} \mbox{}\newline 
\hspace*{1em}\hspace*{1em}\hspace*{1em}\hspace*{1em}\mbox{}\newline 
\textit{<!-- remainder of The Copper Beeches here -->}\mbox{}\newline 
\hspace*{1em}\hspace*{1em}\hspace*{1em}\hspace*{1em}\hspace*{1em}\hspace*{1em}\hspace*{1em}\hspace*{1em} \mbox{}\newline 
\hspace*{1em}\hspace*{1em}\hspace*{1em}\hspace*{1em}\hspace*{1em}\hspace*{1em}\hspace*{1em}\hspace*{1em} ... she is now the head of a private school\mbox{}\newline 
\hspace*{1em}\hspace*{1em}\hspace*{1em}\hspace*{1em}\hspace*{1em}\hspace*{1em}\hspace*{1em}\hspace*{1em} at Walsall, where I believe that she has\mbox{}\newline 
\hspace*{1em}\hspace*{1em}\hspace*{1em}\hspace*{1em}\hspace*{1em}\hspace*{1em}\hspace*{1em}\hspace*{1em} met with considerable success.{</\textbf{p}>}\mbox{}\newline 
\hspace*{1em}\hspace*{1em}\hspace*{1em}{</\textbf{body}>}\mbox{}\newline 
\hspace*{1em}\hspace*{1em}{</\textbf{text}>}\mbox{}\newline 
\textit{<!-- end of The Copper Beeches -->}\mbox{}\newline 
\hspace*{1em}{</\textbf{group}>}\mbox{}\newline 
{</\textbf{text}>}\mbox{}\newline 
\textit{<!-- end of the Adventures of Sherlock Holmes -->}\end{shaded}\egroup\par \noindent  \par
A text which is a member of a group may itself contain groups. This is quite common in collections of verse, but may happen in any kind of text. As an example, consider the overall structure of a typical collection, such as the \textit{Muses Library} edition of Crashaw's poetry. Following a critical introduction and table of contents, this work contains the following major sections: \begin{itemize}
\item \textit{Steps to the Temple} (a collection of verse first published in 1648)
\item \textit{Carmen deo Nostro} (a second collection, published in 1652)
\item \textit{The Delights of the Muses} (a third collection, published in 1648)
\item \textit{Posthumous Poems,} I (a collection of fragments all taken from a single manuscript)
\item \textit{Posthumous Poems,} II (a further collection of fragments, taken from a different manuscript)
\end{itemize} \par
Each of the three collections published in Crashaw's lifetime has a reasonable claim to be considered as a text in its own right, and may therefore be encoded as such. It is rather more arbitrary as to whether the two posthumous collections should be treated as two groups, following the practice of the \textit{Muses Library} edition. An encoder might elect to combine the two into a single group or simply to treat each fragment as an ungrouped unitary text.\par
The \textit{Muses Library} edition reprints the whole of each of the three original collections, including their original front matter (title pages, dedications etc.). These should be encoded using the \hyperref[TEI.front]{<front>} element and its constituents (on which see further section \textit{\hyperref[DSFRONT]{4.5.\ Front Matter}}), while the body of each collection should be encoded as a single \hyperref[TEI.group]{<group>} element. Each individual poem within the collections should be encoded as a distinct \hyperref[TEI.text]{<text>} element. The beginning of the whole collection would thus appear as follows (for further discussion of the use of the elements \hyperref[TEI.div]{<div>} and \hyperref[TEI.lg]{<lg>} for textual subdivision of verse, see section \textit{\hyperref[COVE]{3.13.1.\ Core Tags for Verse}} and chapter \textit{\hyperref[VE]{6.\ Verse}}): \par\bgroup\index{text=<text>|exampleindex}\index{front=<front>|exampleindex}\index{titlePage=<titlePage>|exampleindex}\index{docTitle=<docTitle>|exampleindex}\index{titlePart=<titlePart>|exampleindex}\index{byline=<byline>|exampleindex}\index{div=<div>|exampleindex}\index{type=@type!<div>|exampleindex}\index{head=<head>|exampleindex}\index{p=<p>|exampleindex}\index{group=<group>|exampleindex}\index{text=<text>|exampleindex}\index{front=<front>|exampleindex}\index{titlePage=<titlePage>|exampleindex}\index{docTitle=<docTitle>|exampleindex}\index{titlePart=<titlePart>|exampleindex}\index{div=<div>|exampleindex}\index{type=@type!<div>|exampleindex}\index{head=<head>|exampleindex}\index{p=<p>|exampleindex}\index{group=<group>|exampleindex}\index{text=<text>|exampleindex}\index{front=<front>|exampleindex}\index{docTitle=<docTitle>|exampleindex}\index{titlePart=<titlePart>|exampleindex}\index{body=<body>|exampleindex}\index{div1=<div1>|exampleindex}\index{type=@type!<div1>|exampleindex}\index{n=@n!<div1>|exampleindex}\index{head=<head>|exampleindex}\index{epigraph=<epigraph>|exampleindex}\index{l=<l>|exampleindex}\index{lg=<lg>|exampleindex}\index{n=@n!<lg>|exampleindex}\index{type=@type!<lg>|exampleindex}\index{l=<l>|exampleindex}\index{l=<l>|exampleindex}\index{l=<l>|exampleindex}\index{l=<l>|exampleindex}\index{l=<l>|exampleindex}\index{text=<text>|exampleindex}\index{front=<front>|exampleindex}\index{docTitle=<docTitle>|exampleindex}\index{titlePart=<titlePart>|exampleindex}\index{body=<body>|exampleindex}\index{lg=<lg>|exampleindex}\index{n=@n!<lg>|exampleindex}\index{l=<l>|exampleindex}\index{l=<l>|exampleindex}\index{back=<back>|exampleindex}\index{text=<text>|exampleindex}\index{front=<front>|exampleindex}\index{group=<group>|exampleindex}\index{text=<text>|exampleindex}\index{text=<text>|exampleindex}\index{text=<text>|exampleindex}\index{group=<group>|exampleindex}\index{text=<text>|exampleindex}\index{text=<text>|exampleindex}\index{back=<back>|exampleindex}\exampleFont \begin{shaded}\noindent\mbox{}{<\textbf{text}>}\mbox{}\newline 
\hspace*{1em}{<\textbf{front}>}\mbox{}\newline 
\hspace*{1em}\hspace*{1em}{<\textbf{titlePage}>}\mbox{}\newline 
\hspace*{1em}\hspace*{1em}\hspace*{1em}{<\textbf{docTitle}>}\mbox{}\newline 
\hspace*{1em}\hspace*{1em}\hspace*{1em}\hspace*{1em}{<\textbf{titlePart}>}The poems of Richard Crashaw{</\textbf{titlePart}>}\mbox{}\newline 
\hspace*{1em}\hspace*{1em}\hspace*{1em}{</\textbf{docTitle}>}\mbox{}\newline 
\hspace*{1em}\hspace*{1em}\hspace*{1em}{<\textbf{byline}>}Edited by J.R. Tutin{</\textbf{byline}>}\mbox{}\newline 
\hspace*{1em}\hspace*{1em}{</\textbf{titlePage}>}\mbox{}\newline 
\hspace*{1em}\hspace*{1em}{<\textbf{div}\hspace*{1em}{type}="{preface}">}\mbox{}\newline 
\hspace*{1em}\hspace*{1em}\hspace*{1em}{<\textbf{head}>}Editor's Note{</\textbf{head}>}\mbox{}\newline 
\hspace*{1em}\hspace*{1em}\hspace*{1em}{<\textbf{p}>}A few words are necessary ... {</\textbf{p}>}\mbox{}\newline 
\hspace*{1em}\hspace*{1em}{</\textbf{div}>}\mbox{}\newline 
\hspace*{1em}{</\textbf{front}>}\mbox{}\newline 
\hspace*{1em}{<\textbf{group}>}\mbox{}\newline 
\hspace*{1em}\hspace*{1em}{<\textbf{text}>}\mbox{}\newline 
\hspace*{1em}\hspace*{1em}\hspace*{1em}{<\textbf{front}>}\mbox{}\newline 
\hspace*{1em}\hspace*{1em}\hspace*{1em}\hspace*{1em}{<\textbf{titlePage}>}\mbox{}\newline 
\hspace*{1em}\hspace*{1em}\hspace*{1em}\hspace*{1em}\hspace*{1em}{<\textbf{docTitle}>}\mbox{}\newline 
\hspace*{1em}\hspace*{1em}\hspace*{1em}\hspace*{1em}\hspace*{1em}\hspace*{1em}{<\textbf{titlePart}>}Steps to the Temple, Sacred Poems{</\textbf{titlePart}>}\mbox{}\newline 
\hspace*{1em}\hspace*{1em}\hspace*{1em}\hspace*{1em}\hspace*{1em}{</\textbf{docTitle}>}\mbox{}\newline 
\hspace*{1em}\hspace*{1em}\hspace*{1em}\hspace*{1em}{</\textbf{titlePage}>}\mbox{}\newline 
\hspace*{1em}\hspace*{1em}\hspace*{1em}\hspace*{1em}{<\textbf{div}\hspace*{1em}{type}="{address}">}\mbox{}\newline 
\hspace*{1em}\hspace*{1em}\hspace*{1em}\hspace*{1em}\hspace*{1em}{<\textbf{head}>}The Preface to the Reader{</\textbf{head}>}\mbox{}\newline 
\hspace*{1em}\hspace*{1em}\hspace*{1em}\hspace*{1em}\hspace*{1em}{<\textbf{p}>}Learned Reader, The Author's friend will not usurp much\mbox{}\newline 
\hspace*{1em}\hspace*{1em}\hspace*{1em}\hspace*{1em}\hspace*{1em}\hspace*{1em}\hspace*{1em}\hspace*{1em}\hspace*{1em}\hspace*{1em} upon thy eye ... {</\textbf{p}>}\mbox{}\newline 
\hspace*{1em}\hspace*{1em}\hspace*{1em}\hspace*{1em}{</\textbf{div}>}\mbox{}\newline 
\hspace*{1em}\hspace*{1em}\hspace*{1em}{</\textbf{front}>}\mbox{}\newline 
\hspace*{1em}\hspace*{1em}\hspace*{1em}{<\textbf{group}>}\mbox{}\newline 
\hspace*{1em}\hspace*{1em}\hspace*{1em}\hspace*{1em}{<\textbf{text}>}\mbox{}\newline 
\hspace*{1em}\hspace*{1em}\hspace*{1em}\hspace*{1em}\hspace*{1em}{<\textbf{front}>}\mbox{}\newline 
\hspace*{1em}\hspace*{1em}\hspace*{1em}\hspace*{1em}\hspace*{1em}\hspace*{1em}{<\textbf{docTitle}>}\mbox{}\newline 
\hspace*{1em}\hspace*{1em}\hspace*{1em}\hspace*{1em}\hspace*{1em}\hspace*{1em}\hspace*{1em}{<\textbf{titlePart}>}Sospetto D'Herode{</\textbf{titlePart}>}\mbox{}\newline 
\hspace*{1em}\hspace*{1em}\hspace*{1em}\hspace*{1em}\hspace*{1em}\hspace*{1em}{</\textbf{docTitle}>}\mbox{}\newline 
\hspace*{1em}\hspace*{1em}\hspace*{1em}\hspace*{1em}\hspace*{1em}{</\textbf{front}>}\mbox{}\newline 
\hspace*{1em}\hspace*{1em}\hspace*{1em}\hspace*{1em}\hspace*{1em}{<\textbf{body}>}\mbox{}\newline 
\hspace*{1em}\hspace*{1em}\hspace*{1em}\hspace*{1em}\hspace*{1em}\hspace*{1em}{<\textbf{div1}\hspace*{1em}{type}="{book}"\hspace*{1em}{n}="{Herod I}">}\mbox{}\newline 
\hspace*{1em}\hspace*{1em}\hspace*{1em}\hspace*{1em}\hspace*{1em}\hspace*{1em}\hspace*{1em}{<\textbf{head}>}Libro Primo{</\textbf{head}>}\mbox{}\newline 
\hspace*{1em}\hspace*{1em}\hspace*{1em}\hspace*{1em}\hspace*{1em}\hspace*{1em}\hspace*{1em}{<\textbf{epigraph}>}\mbox{}\newline 
\hspace*{1em}\hspace*{1em}\hspace*{1em}\hspace*{1em}\hspace*{1em}\hspace*{1em}\hspace*{1em}\hspace*{1em}{<\textbf{l}>}Casting the times with their strong signs{</\textbf{l}>}\mbox{}\newline 
\hspace*{1em}\hspace*{1em}\hspace*{1em}\hspace*{1em}\hspace*{1em}\hspace*{1em}\hspace*{1em}{</\textbf{epigraph}>}\mbox{}\newline 
\hspace*{1em}\hspace*{1em}\hspace*{1em}\hspace*{1em}\hspace*{1em}\hspace*{1em}\hspace*{1em}{<\textbf{lg}\hspace*{1em}{n}="{I.1}"\hspace*{1em}{type}="{stanza}">}\mbox{}\newline 
\hspace*{1em}\hspace*{1em}\hspace*{1em}\hspace*{1em}\hspace*{1em}\hspace*{1em}\hspace*{1em}\hspace*{1em}{<\textbf{l}>}Muse! now the servant of soft loves no more{</\textbf{l}>}\mbox{}\newline 
\hspace*{1em}\hspace*{1em}\hspace*{1em}\hspace*{1em}\hspace*{1em}\hspace*{1em}\hspace*{1em}\hspace*{1em}{<\textbf{l}>}Hate is thy theme and Herod whose unblest{</\textbf{l}>}\mbox{}\newline 
\hspace*{1em}\hspace*{1em}\hspace*{1em}\hspace*{1em}\hspace*{1em}\hspace*{1em}\hspace*{1em}\hspace*{1em}{<\textbf{l}>}Hand (O, what dares not jealous greatness?) tore{</\textbf{l}>}\mbox{}\newline 
\hspace*{1em}\hspace*{1em}\hspace*{1em}\hspace*{1em}\hspace*{1em}\hspace*{1em}\hspace*{1em}\hspace*{1em}{<\textbf{l}>}A thousand sweet babes from their mothers' breast,{</\textbf{l}>}\mbox{}\newline 
\hspace*{1em}\hspace*{1em}\hspace*{1em}\hspace*{1em}\hspace*{1em}\hspace*{1em}\hspace*{1em}\hspace*{1em}{<\textbf{l}>}The blooms of martyrdom ...{</\textbf{l}>}\mbox{}\newline 
\hspace*{1em}\hspace*{1em}\hspace*{1em}\hspace*{1em}\hspace*{1em}\hspace*{1em}\hspace*{1em}{</\textbf{lg}>}\mbox{}\newline 
\hspace*{1em}\hspace*{1em}\hspace*{1em}\hspace*{1em}\hspace*{1em}\hspace*{1em}{</\textbf{div1}>}\mbox{}\newline 
\hspace*{1em}\hspace*{1em}\hspace*{1em}\hspace*{1em}\hspace*{1em}{</\textbf{body}>}\mbox{}\newline 
\hspace*{1em}\hspace*{1em}\hspace*{1em}\hspace*{1em}{</\textbf{text}>}\mbox{}\newline 
\hspace*{1em}\hspace*{1em}\hspace*{1em}\hspace*{1em}{<\textbf{text}>}\mbox{}\newline 
\hspace*{1em}\hspace*{1em}\hspace*{1em}\hspace*{1em}\hspace*{1em}{<\textbf{front}>}\mbox{}\newline 
\hspace*{1em}\hspace*{1em}\hspace*{1em}\hspace*{1em}\hspace*{1em}\hspace*{1em}{<\textbf{docTitle}>}\mbox{}\newline 
\hspace*{1em}\hspace*{1em}\hspace*{1em}\hspace*{1em}\hspace*{1em}\hspace*{1em}\hspace*{1em}{<\textbf{titlePart}>}The Tear{</\textbf{titlePart}>}\mbox{}\newline 
\hspace*{1em}\hspace*{1em}\hspace*{1em}\hspace*{1em}\hspace*{1em}\hspace*{1em}{</\textbf{docTitle}>}\mbox{}\newline 
\hspace*{1em}\hspace*{1em}\hspace*{1em}\hspace*{1em}\hspace*{1em}{</\textbf{front}>}\mbox{}\newline 
\hspace*{1em}\hspace*{1em}\hspace*{1em}\hspace*{1em}\hspace*{1em}{<\textbf{body}>}\mbox{}\newline 
\hspace*{1em}\hspace*{1em}\hspace*{1em}\hspace*{1em}\hspace*{1em}\hspace*{1em}{<\textbf{lg}\hspace*{1em}{n}="{I}">}\mbox{}\newline 
\hspace*{1em}\hspace*{1em}\hspace*{1em}\hspace*{1em}\hspace*{1em}\hspace*{1em}\hspace*{1em}{<\textbf{l}>}What bright soft thing is this{</\textbf{l}>}\mbox{}\newline 
\hspace*{1em}\hspace*{1em}\hspace*{1em}\hspace*{1em}\hspace*{1em}\hspace*{1em}\hspace*{1em}{<\textbf{l}>}Sweet Mary, thy fair eyes' expense?{</\textbf{l}>}\mbox{}\newline 
\hspace*{1em}\hspace*{1em}\hspace*{1em}\hspace*{1em}\hspace*{1em}\hspace*{1em}{</\textbf{lg}>}\mbox{}\newline 
\hspace*{1em}\hspace*{1em}\hspace*{1em}\hspace*{1em}\hspace*{1em}{</\textbf{body}>}\mbox{}\newline 
\hspace*{1em}\hspace*{1em}\hspace*{1em}\hspace*{1em}{</\textbf{text}>}\mbox{}\newline 
\textit{<!-- remaining poems of the Steps to the Temple appear\newline
	    here, each tagged as a distinct text element -->}\mbox{}\newline 
\hspace*{1em}\hspace*{1em}\hspace*{1em}{</\textbf{group}>}\mbox{}\newline 
\hspace*{1em}\hspace*{1em}\hspace*{1em}{<\textbf{back}>}\mbox{}\newline 
\textit{<!-- back matter for the Steps to the Temple -->}\mbox{}\newline 
\hspace*{1em}\hspace*{1em}\hspace*{1em}{</\textbf{back}>}\mbox{}\newline 
\hspace*{1em}\hspace*{1em}{</\textbf{text}>}\mbox{}\newline 
\hspace*{1em}\hspace*{1em}{<\textbf{text}>}\mbox{}\newline 
\textit{<!-- start of Carmen deo Nostro -->}\mbox{}\newline 
\hspace*{1em}\hspace*{1em}\hspace*{1em}{<\textbf{front}/>}\mbox{}\newline 
\hspace*{1em}\hspace*{1em}\hspace*{1em}{<\textbf{group}>}\mbox{}\newline 
\hspace*{1em}\hspace*{1em}\hspace*{1em}\hspace*{1em}{<\textbf{text}/>}\mbox{}\newline 
\hspace*{1em}\hspace*{1em}\hspace*{1em}\hspace*{1em}{<\textbf{text}/>}\mbox{}\newline 
\textit{<!-- more texts here -->}\mbox{}\newline 
\hspace*{1em}\hspace*{1em}\hspace*{1em}{</\textbf{group}>}\mbox{}\newline 
\hspace*{1em}\hspace*{1em}{</\textbf{text}>}\mbox{}\newline 
\hspace*{1em}\hspace*{1em}{<\textbf{text}>}\mbox{}\newline 
\textit{<!-- start of The Delights of the Muses -->}\mbox{}\newline 
\hspace*{1em}\hspace*{1em}\hspace*{1em}{<\textbf{group}>}\mbox{}\newline 
\hspace*{1em}\hspace*{1em}\hspace*{1em}\hspace*{1em}{<\textbf{text}/>}\mbox{}\newline 
\hspace*{1em}\hspace*{1em}\hspace*{1em}\hspace*{1em}{<\textbf{text}/>}\mbox{}\newline 
\textit{<!-- more texts here -->}\mbox{}\newline 
\hspace*{1em}\hspace*{1em}\hspace*{1em}{</\textbf{group}>}\mbox{}\newline 
\hspace*{1em}\hspace*{1em}{</\textbf{text}>}\mbox{}\newline 
\hspace*{1em}{</\textbf{group}>}\mbox{}\newline 
\hspace*{1em}{<\textbf{back}>}\mbox{}\newline 
\textit{<!-- back matter for the whole collection -->}\mbox{}\newline 
\hspace*{1em}{</\textbf{back}>}\mbox{}\newline 
{</\textbf{text}>}\end{shaded}\egroup\par \par
The \hyperref[TEI.group]{<group>} element may be used in this way to encode any kind of collection of which the constituents are regarded by the encoder as texts in their own right. Examples include anthologies or collections of verse or prose by multiple authors, florilegia, or commonplace books, journals, day books, etc. As a fairly typical example, we consider \textit{The Norton Book of Travel}, an anthology edited by Paul Fussell and published in 1987 by W. W. Norton. This work comprises the following major sections: \begin{enumerate}
\item Front matter (title page, acknowledgments, introductory essay)
\item The Beginnings
\item The Eighteenth Century and the Grand Tour
\item The Heyday
\item Touristic Tendencies
\item Post Tourism
\item Back matter (permissions list, index)
\end{enumerate} Each titled section listed above comprises a group of extracts or complete texts from writers of a given historical period, preceded by an introductory essay. For example, the second group listed above contains, inter alia, the following: \begin{enumerate}
\item Prefatory essay
\item Five letters by Lady Mary Wortley Montagu
\item An extract from Swift's \textit{Gulliver's Travels}
\item Two poems by Alexander Pope
\item Two extracts from Boswell's Journal
\item A poem by William Blake
\end{enumerate} Each group of writings by a single author is preceded by a brief biographical notice. Some of the extracts are quite lengthy, containing several chapters or other divisions; others are quite short. As the above list indicates, the texts included range across all kinds of material: verse, prose, journals and letters.\par
The easiest way of encoding such an anthology is to treat each individual extract as a text in its own right. A sequence of texts by a single author, together with the biographical note preceding it, can then be treated as a single \hyperref[TEI.group]{<group>} element within the larger \hyperref[TEI.group]{<group>} formed by the section. The sequence of single or composite texts making up a single section of the work is likewise treated, together with its prefatory essay, as a single \hyperref[TEI.group]{<group>} within the work. Schematically: \par\bgroup\index{text=<text>|exampleindex}\index{front=<front>|exampleindex}\index{group=<group>|exampleindex}\index{group=<group>|exampleindex}\index{head=<head>|exampleindex}\index{group=<group>|exampleindex}\index{text=<text>|exampleindex}\index{group=<group>|exampleindex}\index{text=<text>|exampleindex}\index{text=<text>|exampleindex}\index{text=<text>|exampleindex}\index{text=<text>|exampleindex}\index{front=<front>|exampleindex}\index{body=<body>|exampleindex}\index{group=<group>|exampleindex}\index{text=<text>|exampleindex}\index{text=<text>|exampleindex}\index{text=<text>|exampleindex}\index{group=<group>|exampleindex}\index{head=<head>|exampleindex}\index{back=<back>|exampleindex}\exampleFont \begin{shaded}\noindent\mbox{}{<\textbf{text}>}\mbox{}\newline 
\textit{<!-- the whole anthology -->}\mbox{}\newline 
\hspace*{1em}{<\textbf{front}>}\mbox{}\newline 
\textit{<!-- title page, acknowledgments, introductory essay -->}\mbox{}\newline 
\hspace*{1em}{</\textbf{front}>}\mbox{}\newline 
\hspace*{1em}{<\textbf{group}>}\mbox{}\newline 
\textit{<!-- body of anthology starts here -->}\mbox{}\newline 
\hspace*{1em}\hspace*{1em}{<\textbf{group}>}\mbox{}\newline 
\hspace*{1em}\hspace*{1em}\hspace*{1em}{<\textbf{head}>}The Beginnings{</\textbf{head}>}\mbox{}\newline 
\textit{<!-- sequence of texts or groups -->}\mbox{}\newline 
\hspace*{1em}\hspace*{1em}{</\textbf{group}>}\mbox{}\newline 
\hspace*{1em}\hspace*{1em}{<\textbf{group}>}\mbox{}\newline 
\textit{<!-- The Eighteenth Century and the Grand Tour -->}\mbox{}\newline 
\hspace*{1em}\hspace*{1em}\hspace*{1em}{<\textbf{text}>}\mbox{}\newline 
\textit{<!-- prefatory essay by editor -->}\mbox{}\newline 
\hspace*{1em}\hspace*{1em}\hspace*{1em}{</\textbf{text}>}\mbox{}\newline 
\hspace*{1em}\hspace*{1em}\hspace*{1em}{<\textbf{group}>}\mbox{}\newline 
\textit{<!-- Section on Lady Mary Wortley Montagu starts -->}\mbox{}\newline 
\hspace*{1em}\hspace*{1em}\hspace*{1em}\hspace*{1em}{<\textbf{text}>}\mbox{}\newline 
\textit{<!-- biographical notice by editor -->}\mbox{}\newline 
\hspace*{1em}\hspace*{1em}\hspace*{1em}\hspace*{1em}{</\textbf{text}>}\mbox{}\newline 
\hspace*{1em}\hspace*{1em}\hspace*{1em}\hspace*{1em}{<\textbf{text}>}\mbox{}\newline 
\textit{<!-- first letter -->}\mbox{}\newline 
\hspace*{1em}\hspace*{1em}\hspace*{1em}\hspace*{1em}{</\textbf{text}>}\mbox{}\newline 
\hspace*{1em}\hspace*{1em}\hspace*{1em}\hspace*{1em}{<\textbf{text}>}\mbox{}\newline 
\textit{<!-- second letter -->}\mbox{}\newline 
\hspace*{1em}\hspace*{1em}\hspace*{1em}\hspace*{1em}{</\textbf{text}>}\mbox{}\newline 
\textit{<!-- ... -->}\mbox{}\newline 
\hspace*{1em}\hspace*{1em}\hspace*{1em}{</\textbf{group}>}\mbox{}\newline 
\textit{<!-- end of Montagu section -->}\mbox{}\newline 
\hspace*{1em}\hspace*{1em}\hspace*{1em}{<\textbf{text}>}\mbox{}\newline 
\textit{<!-- single text by Jonathan Swift starts -->}\mbox{}\newline 
\hspace*{1em}\hspace*{1em}\hspace*{1em}\hspace*{1em}{<\textbf{front}>}\mbox{}\newline 
\textit{<!-- biographical notice by editor -->}\mbox{}\newline 
\hspace*{1em}\hspace*{1em}\hspace*{1em}\hspace*{1em}{</\textbf{front}>}\mbox{}\newline 
\hspace*{1em}\hspace*{1em}\hspace*{1em}\hspace*{1em}{<\textbf{body}/>}\mbox{}\newline 
\hspace*{1em}\hspace*{1em}\hspace*{1em}{</\textbf{text}>}\mbox{}\newline 
\textit{<!-- end of Swift section -->}\mbox{}\newline 
\hspace*{1em}\hspace*{1em}\hspace*{1em}{<\textbf{group}>}\mbox{}\newline 
\textit{<!-- Section on Alexander Pope starts -->}\mbox{}\newline 
\hspace*{1em}\hspace*{1em}\hspace*{1em}\hspace*{1em}{<\textbf{text}>}\mbox{}\newline 
\textit{<!-- biographical notice by editor -->}\mbox{}\newline 
\hspace*{1em}\hspace*{1em}\hspace*{1em}\hspace*{1em}{</\textbf{text}>}\mbox{}\newline 
\hspace*{1em}\hspace*{1em}\hspace*{1em}\hspace*{1em}{<\textbf{text}>}\mbox{}\newline 
\textit{<!-- first poem -->}\mbox{}\newline 
\hspace*{1em}\hspace*{1em}\hspace*{1em}\hspace*{1em}{</\textbf{text}>}\mbox{}\newline 
\hspace*{1em}\hspace*{1em}\hspace*{1em}\hspace*{1em}{<\textbf{text}>}\mbox{}\newline 
\textit{<!-- second poem -->}\mbox{}\newline 
\hspace*{1em}\hspace*{1em}\hspace*{1em}\hspace*{1em}{</\textbf{text}>}\mbox{}\newline 
\hspace*{1em}\hspace*{1em}\hspace*{1em}{</\textbf{group}>}\mbox{}\newline 
\textit{<!-- end of Pope section -->}\mbox{}\newline 
\textit{<!-- ... -->}\mbox{}\newline 
\hspace*{1em}\hspace*{1em}{</\textbf{group}>}\mbox{}\newline 
\textit{<!-- end of 18th century section -->}\mbox{}\newline 
\hspace*{1em}\hspace*{1em}{<\textbf{group}>}\mbox{}\newline 
\hspace*{1em}\hspace*{1em}\hspace*{1em}{<\textbf{head}>}The Heyday{</\textbf{head}>}\mbox{}\newline 
\textit{<!-- texts and subgroups -->}\mbox{}\newline 
\hspace*{1em}\hspace*{1em}{</\textbf{group}>}\mbox{}\newline 
\textit{<!-- ... -->}\mbox{}\newline 
\hspace*{1em}{</\textbf{group}>}\mbox{}\newline 
\textit{<!-- end of the anthology proper -->}\mbox{}\newline 
\hspace*{1em}{<\textbf{back}>}\mbox{}\newline 
\textit{<!-- back matter for anthology -->}\mbox{}\newline 
\hspace*{1em}{</\textbf{back}>}\mbox{}\newline 
{</\textbf{text}>}\end{shaded}\egroup\par \par
Note that the editor's introductory essays on each author may be treated as texts in their own right (as the essays on Lady Mary Wortley Montagu and Alexander Pope have been treated above), or as front matter to the embedded text, as the essay on Swift has been. The treatment in the example is intentionally inconsistent, to allow comparison of the two approaches. Consistency can be imposed either by treating the Swift section as a \hyperref[TEI.group]{<group>} containing one text by Swift and one by the editor, or by treating the Montagu and Pope sections as \hyperref[TEI.text]{<text>} elements containing the editor's essays as front matter. Marked in the second way, the Pope section of the book would look like this: \par\bgroup\index{text=<text>|exampleindex}\index{front=<front>|exampleindex}\index{group=<group>|exampleindex}\index{text=<text>|exampleindex}\index{text=<text>|exampleindex}\exampleFont \begin{shaded}\noindent\mbox{}{<\textbf{text}>}\mbox{}\newline 
\textit{<!-- Section on Alexander Pope starts -->}\mbox{}\newline 
\hspace*{1em}{<\textbf{front}>}\mbox{}\newline 
\textit{<!-- biographical notice by editor -->}\mbox{}\newline 
\hspace*{1em}{</\textbf{front}>}\mbox{}\newline 
\hspace*{1em}{<\textbf{group}>}\mbox{}\newline 
\hspace*{1em}\hspace*{1em}{<\textbf{text}>}\mbox{}\newline 
\textit{<!-- first poem -->}\mbox{}\newline 
\hspace*{1em}\hspace*{1em}{</\textbf{text}>}\mbox{}\newline 
\hspace*{1em}\hspace*{1em}{<\textbf{text}>}\mbox{}\newline 
\textit{<!-- second poem -->}\mbox{}\newline 
\hspace*{1em}\hspace*{1em}{</\textbf{text}>}\mbox{}\newline 
\hspace*{1em}{</\textbf{group}>}\mbox{}\newline 
{</\textbf{text}>}\mbox{}\newline 
\textit{<!-- end of Pope section-->}\end{shaded}\egroup\par \par
The essays on ‘The Eighteenth Century and the Grand Tour’ and other larger sections could also be tagged as ‘front’ matter in the same way, by treating the larger sections as \hyperref[TEI.text]{<text>} elements rather than \hyperref[TEI.group]{<group>} elements.\par
Where, as in this case, an anthology contains different kinds of text (for example, mixtures of prose and drama, or transcribed speech and dictionary entries, or letters and verse), the elements to be encoded will of course be drawn from more than one module. The elements provided by the core module described in chapter \textit{\hyperref[CO]{3.\ Elements Available in All TEI Documents}} should however prove adequate for most simple purposes, where prose, drama, and verse are combined in a single collection.\par
For anthologies of short extracts such as commonplace books, it may often be preferable to regard each extract not as a text in its own right but simply as a quotation or \hyperref[TEI.cit]{<cit>} element. The following component-level elements may be used to encode quotations of this kind: 
\begin{sansreflist}
  
\item [\textbf{<cit>}] (cited quotation) contains a quotation from some other document, together with a bibliographic reference to its source. In a dictionary it may contain an example text with at least one occurrence of the word form, used in the sense being described, or a translation of the headword, or an example.
\item [\textbf{<quote>}] (quotation) contains a phrase or passage attributed by the narrator or author to some agency external to the text.
\end{sansreflist}
 For example, the chapter of ‘extracts’ which appears in the front matter of Melville's \textit{Moby Dick} might be encoded as follows: \par\bgroup\index{div=<div>|exampleindex}\index{n=@n!<div>|exampleindex}\index{type=@type!<div>|exampleindex}\index{head=<head>|exampleindex}\index{head=<head>|exampleindex}\index{p=<p>|exampleindex}\index{p=<p>|exampleindex}\index{cit=<cit>|exampleindex}\index{quote=<quote>|exampleindex}\index{bibl=<bibl>|exampleindex}\index{cit=<cit>|exampleindex}\index{quote=<quote>|exampleindex}\index{l=<l>|exampleindex}\index{l=<l>|exampleindex}\index{bibl=<bibl>|exampleindex}\index{cit=<cit>|exampleindex}\index{quote=<quote>|exampleindex}\index{mentioned=<mentioned>|exampleindex}\index{bibl=<bibl>|exampleindex}\exampleFont \begin{shaded}\noindent\mbox{}{<\textbf{div}\hspace*{1em}{n}="{2}"\hspace*{1em}{type}="{chap}">}\mbox{}\newline 
\hspace*{1em}{<\textbf{head}>}Extracts{</\textbf{head}>}\mbox{}\newline 
\hspace*{1em}{<\textbf{head}>}(Supplied by a sub-sub-Librarian){</\textbf{head}>}\mbox{}\newline 
\hspace*{1em}{<\textbf{p}>}It will be seen that this mere painstaking burrower and\mbox{}\newline 
\hspace*{1em}\hspace*{1em} grubworm of a poor devil of a Sub-Sub appears to have gone\mbox{}\newline 
\hspace*{1em}\hspace*{1em} through the long Vaticans and street-stalls of the earth,\mbox{}\newline 
\hspace*{1em}\hspace*{1em} picking up whatever random allusions to whales he could\mbox{}\newline 
\hspace*{1em}\hspace*{1em} anyways find ...\mbox{}\newline 
\hspace*{1em}\hspace*{1em} Here ye strike but splintered hearts together — there,\mbox{}\newline 
\hspace*{1em}\hspace*{1em} ye shall strike unsplinterable glasses!{</\textbf{p}>}\mbox{}\newline 
\hspace*{1em}{<\textbf{p}>}\mbox{}\newline 
\hspace*{1em}\hspace*{1em}{<\textbf{cit}>}\mbox{}\newline 
\hspace*{1em}\hspace*{1em}\hspace*{1em}{<\textbf{quote}>}And God created great whales.{</\textbf{quote}>}\mbox{}\newline 
\hspace*{1em}\hspace*{1em}\hspace*{1em}{<\textbf{bibl}>}Genesis{</\textbf{bibl}>}\mbox{}\newline 
\hspace*{1em}\hspace*{1em}{</\textbf{cit}>}\mbox{}\newline 
\hspace*{1em}\hspace*{1em}{<\textbf{cit}>}\mbox{}\newline 
\hspace*{1em}\hspace*{1em}\hspace*{1em}{<\textbf{quote}>}\mbox{}\newline 
\hspace*{1em}\hspace*{1em}\hspace*{1em}\hspace*{1em}{<\textbf{l}>}Leviathan maketh a path to shine after him;{</\textbf{l}>}\mbox{}\newline 
\hspace*{1em}\hspace*{1em}\hspace*{1em}\hspace*{1em}{<\textbf{l}>}One would think the deep to be hoary.{</\textbf{l}>}\mbox{}\newline 
\hspace*{1em}\hspace*{1em}\hspace*{1em}{</\textbf{quote}>}\mbox{}\newline 
\hspace*{1em}\hspace*{1em}\hspace*{1em}{<\textbf{bibl}>}Job{</\textbf{bibl}>}\mbox{}\newline 
\hspace*{1em}\hspace*{1em}{</\textbf{cit}>}\mbox{}\newline 
\hspace*{1em}\hspace*{1em}{<\textbf{cit}>}\mbox{}\newline 
\hspace*{1em}\hspace*{1em}\hspace*{1em}{<\textbf{quote}>}By art is created that great Leviathan,\mbox{}\newline 
\hspace*{1em}\hspace*{1em}\hspace*{1em}\hspace*{1em}\hspace*{1em}\hspace*{1em} called a Commonwealth or State — (in Latin,\mbox{}\newline 
\hspace*{1em}\hspace*{1em}\hspace*{1em}{<\textbf{mentioned}\hspace*{1em}{xml:lang}="{la}">}civitas{</\textbf{mentioned}>}), which\mbox{}\newline 
\hspace*{1em}\hspace*{1em}\hspace*{1em}\hspace*{1em}\hspace*{1em}\hspace*{1em} is but an artificial man.{</\textbf{quote}>}\mbox{}\newline 
\hspace*{1em}\hspace*{1em}\hspace*{1em}{<\textbf{bibl}>}Opening sentence of Hobbes's Leviathan{</\textbf{bibl}>}\mbox{}\newline 
\hspace*{1em}\hspace*{1em}{</\textbf{cit}>}\mbox{}\newline 
\hspace*{1em}{</\textbf{p}>}\mbox{}\newline 
{</\textbf{div}>}\end{shaded}\egroup\par \noindent  For more information on the use of the \hyperref[TEI.quote]{<quote>} and \hyperref[TEI.bibl]{<bibl>} elements, see sections \textit{\hyperref[COHQQ]{3.3.3.\ Quotation}} and \textit{\hyperref[COBI]{3.12.\ Bibliographic Citations and References}} respectively.
\subsubsection[{Floating Texts}]{Floating Texts}\label{DSFLT}\par
An important characteristic of the unitary or composite text structures discussed so far is that they can be regarded as forming what is mathematically known as a \textit{tesselation} covering the whole of the available text (or text division) at each hierarchic level. Just as an XML document has a single root element containing a single tree, each node of which forms a properly nested sub-tree, so it seems natural to think of the internal structure of a text as decomposable hierarchically into subparts, each of which is a properly nested subtree. While this is undoubtedly true of a large number of documents, it is not true of all. In particular, it is not true of texts which are only partly tesselated at a given level. For example, if a text A is contained by text B in such a way that part of B precedes A and part follows it, we cannot tesselate the whole of B. In such a case, we say that text A is a ‘floating’ text.\par
The \hyperref[TEI.floatingText]{<floatingText>} element is a member of the \textsf{model.divPart} class, and can thus appear within any division level element in the same way as a paragraph. For example, texts such as the \textit{Decameron} or the \textit{Arabian Nights} might be regarded as containing many floating texts embedded within another single text, the framing narrative, rather than as groups of discrete texts in which the fragments of framing narrative are regarded as front or back matter.\par
As an example, we consider an 18th century text \textit{The Lining to the {\itshape Patch-Work Screen}}, by Jane Barker (1726). This lengthy narrative contains nearly a hundred distinct ‘tales’ embedded (as the title suggests) in a single patchwork. The work begins by introducing the central character, Galecia, but within a few pages launches into a distinct narrative, the story of Captain Manly: \par\bgroup\index{p=<p>|exampleindex}\index{p=<p>|exampleindex}\index{pb=<pb>|exampleindex}\index{n=@n!<pb>|exampleindex}\index{floatingText=<floatingText>|exampleindex}\index{body=<body>|exampleindex}\index{head=<head>|exampleindex}\index{hi=<hi>|exampleindex}\index{p=<p>|exampleindex}\index{pb=<pb>|exampleindex}\index{n=@n!<pb>|exampleindex}\exampleFont \begin{shaded}\noindent\mbox{}{<\textbf{p}>}Galecia one Evening setting alone in her Chamber by a clear Fire,\mbox{}\newline 
 and a clean Hearth [...] reflected on the Providence of our\mbox{}\newline 
 All-wise and Gracious Creator [...] {</\textbf{p}>}\mbox{}\newline 
{<\textbf{p}>}She was thus ruminating, when a Gentleman enter'd the Room, the\mbox{}\newline 
 Door being a jar [...] calling for a Candle, she beg'd a thousand\mbox{}\newline 
 Pardons, engaged him to sit down, and let her know, what had so long\mbox{}\newline 
 conceal'd him from her Correspondence.\mbox{}\newline 
{</\textbf{p}>}\mbox{}\newline 
{<\textbf{pb}\hspace*{1em}{n}="{5}"/>}\mbox{}\newline 
{<\textbf{floatingText}>}\mbox{}\newline 
\hspace*{1em}{<\textbf{body}>}\mbox{}\newline 
\hspace*{1em}\hspace*{1em}{<\textbf{head}>}The Story of {<\textbf{hi}>}Captain Manly{</\textbf{hi}>}\mbox{}\newline 
\hspace*{1em}\hspace*{1em}{</\textbf{head}>}\mbox{}\newline 
\hspace*{1em}\hspace*{1em}{<\textbf{p}>}Dear Galecia, said he, though you partly know the loose, or rather\mbox{}\newline 
\hspace*{1em}\hspace*{1em}\hspace*{1em}\hspace*{1em} lewd Life that I led in my Youth; yet I can't forbear relating part of\mbox{}\newline 
\hspace*{1em}\hspace*{1em}\hspace*{1em}\hspace*{1em} it to you by way of Abhorrence...\mbox{}\newline 
\hspace*{1em}\hspace*{1em}\mbox{}\newline 
\textit{<!-- Captain Manly's story here -->}\mbox{}\newline 
\hspace*{1em}\hspace*{1em}\hspace*{1em}\hspace*{1em} I had lost and spent all I had in the World; in which I verified the\mbox{}\newline 
\hspace*{1em}\hspace*{1em}\hspace*{1em}\hspace*{1em} Old Proverb, That a Rolling Stone never gathers Moss,\mbox{}\newline 
\hspace*{1em}\hspace*{1em}{</\textbf{p}>}\mbox{}\newline 
\hspace*{1em}{</\textbf{body}>}\mbox{}\newline 
{</\textbf{floatingText}>}\mbox{}\newline 
{<\textbf{pb}\hspace*{1em}{n}="{37}"/>}\end{shaded}\egroup\par \par
Following the conclusion of Captain Manly's tale, we are returned to Galecia, and almost immediately after that into two further stories.  However, the Galecia narrative returns between each of the texts, which is why we choose to represent them as \hyperref[TEI.floatingText]{<floatingText>}s: \par\bgroup\index{p=<p>|exampleindex}\index{floatingText=<floatingText>|exampleindex}\index{body=<body>|exampleindex}\index{p=<p>|exampleindex}\index{p=<p>|exampleindex}\index{hi=<hi>|exampleindex}\index{hi=<hi>|exampleindex}\index{pb=<pb>|exampleindex}\index{n=@n!<pb>|exampleindex}\index{p=<p>|exampleindex}\index{floatingText=<floatingText>|exampleindex}\index{body=<body>|exampleindex}\index{head=<head>|exampleindex}\index{hi=<hi>|exampleindex}\index{p=<p>|exampleindex}\index{p=<p>|exampleindex}\index{quote=<quote>|exampleindex}\index{l=<l>|exampleindex}\index{l=<l>|exampleindex}\index{p=<p>|exampleindex}\exampleFont \begin{shaded}\noindent\mbox{}{<\textbf{p}>}The Gentleman having finish'd his Story, Galecia waited on him to\mbox{}\newline 
 the Stairs-head; and at her return, casting her Eyes on the Table, she\mbox{}\newline 
 saw lying there an old dirty rumpled Book, and found in it the\mbox{}\newline 
 following story: {</\textbf{p}>}\mbox{}\newline 
{<\textbf{floatingText}>}\mbox{}\newline 
\hspace*{1em}{<\textbf{body}>}\mbox{}\newline 
\hspace*{1em}\hspace*{1em}{<\textbf{p}>} IN the time of the Holy War when\mbox{}\newline 
\hspace*{1em}\hspace*{1em}\hspace*{1em}\hspace*{1em} Christians from all parts went into the Holy Land to oppose the Turks;\mbox{}\newline 
\hspace*{1em}\hspace*{1em}\hspace*{1em}\hspace*{1em} Amongst these there was a certain English Knight...{</\textbf{p}>}\mbox{}\newline 
\textit{<!-- rest of story here -->}\mbox{}\newline 
\hspace*{1em}\hspace*{1em}{<\textbf{p}>}The King graciously pardoned the Knight; Richard was kindly receiv'd\mbox{}\newline 
\hspace*{1em}\hspace*{1em}\hspace*{1em}\hspace*{1em} into his Convent, and all things went on in good order: But from hence\mbox{}\newline 
\hspace*{1em}\hspace*{1em}\hspace*{1em}\hspace*{1em} came the Proverb, We must not strike {<\textbf{hi}>}Robert{</\textbf{hi}>} for\mbox{}\newline 
\hspace*{1em}\hspace*{1em}{<\textbf{hi}>}Richard.{</\textbf{hi}>}\mbox{}\newline 
\hspace*{1em}\hspace*{1em}{</\textbf{p}>}\mbox{}\newline 
\hspace*{1em}{</\textbf{body}>}\mbox{}\newline 
{</\textbf{floatingText}>}\mbox{}\newline 
{<\textbf{pb}\hspace*{1em}{n}="{43}"/>}\mbox{}\newline 
{<\textbf{p}>}By this time Galecia's Maid brought up her Supper; after which she\mbox{}\newline 
 cast her Eyes again on the foresaid little Book, where she found the\mbox{}\newline 
 following Story, which she read through before she went to bed.\mbox{}\newline 
{</\textbf{p}>}\mbox{}\newline 
{<\textbf{floatingText}>}\mbox{}\newline 
\hspace*{1em}{<\textbf{body}>}\mbox{}\newline 
\hspace*{1em}\hspace*{1em}{<\textbf{head}>}The Cause of the Moors Overrunning\mbox{}\newline 
\hspace*{1em}\hspace*{1em}{<\textbf{hi}>}Spain{</\textbf{hi}>}\mbox{}\newline 
\hspace*{1em}\hspace*{1em}{</\textbf{head}>}\mbox{}\newline 
\hspace*{1em}\hspace*{1em}{<\textbf{p}>}King ———— of Spain at his Death, committed the Government of his\mbox{}\newline 
\hspace*{1em}\hspace*{1em}\hspace*{1em}\hspace*{1em} Kingdom to his Brother Don ——— till his little Son should come of\mbox{}\newline 
\hspace*{1em}\hspace*{1em}\hspace*{1em}\hspace*{1em} Age ...{</\textbf{p}>}\mbox{}\newline 
\hspace*{1em}\hspace*{1em}{<\textbf{p}>}Thus the little Story ended, without telling what Misery\mbox{}\newline 
\hspace*{1em}\hspace*{1em}\hspace*{1em}\hspace*{1em} befel the King and Kingdom, by the Moors, who over ran the Country for\mbox{}\newline 
\hspace*{1em}\hspace*{1em}\hspace*{1em}\hspace*{1em} many Years after. To which, we may well apply the Proverb,\mbox{}\newline 
\hspace*{1em}\hspace*{1em}{<\textbf{quote}>}\mbox{}\newline 
\hspace*{1em}\hspace*{1em}\hspace*{1em}\hspace*{1em}{<\textbf{l}>}Who drives the Devil's Stages,{</\textbf{l}>}\mbox{}\newline 
\hspace*{1em}\hspace*{1em}\hspace*{1em}\hspace*{1em}{<\textbf{l}>}Deserves the Devil's Wages{</\textbf{l}>}\mbox{}\newline 
\hspace*{1em}\hspace*{1em}\hspace*{1em}{</\textbf{quote}>}\mbox{}\newline 
\hspace*{1em}\hspace*{1em}{</\textbf{p}>}\mbox{}\newline 
\hspace*{1em}{</\textbf{body}>}\mbox{}\newline 
{</\textbf{floatingText}>}\mbox{}\newline 
{<\textbf{p}>}The reading this Trifle of a Story detained Galecia from her Rest\mbox{}\newline 
 beyond her usual Hour; for she slept so sound the next Morning, that\mbox{}\newline 
 she did not rise, till a Lady's Footman came to tell her, that his\mbox{}\newline 
 Lady and another or two were coming to breakfast with her...\mbox{}\newline 
{</\textbf{p}>}\end{shaded}\egroup\par \par
In other multi-narrative texts, the individual nested tales may have greater significance than the framing narratives, and it may therefore be preferable to treat the fragments of framing narrative as front or back matter associated with each nested tale. This is commonly done, for example, in texts such as Chaucer's \textit{Canterbury Tales}, where each tale is typically presented with front matter in which the teller of the tale is introduced, and back matter in which the pilgrims comment on it.\par
It is important to distinguish between the uses of \hyperref[TEI.floatingText]{<floatingText>} and \hyperref[TEI.quote]{<quote>}. Whereas the semantics of \hyperref[TEI.quote]{<quote>} suggest that its content derives from a source external to the current text, \hyperref[TEI.floatingText]{<floatingText>} carries no such implication and is simply used whenever the richer content model that it provides is required to support the markup of a part of a text that is presented as a discrete ‘inclusion.’ In some cases, such inclusions could be considered external (e.g., enclosures, attachments, etc.); often however, as in the examples above, the included text bears no signs of emanating from outside.\par
\hyperref[TEI.floatingText]{<floatingText>} and \hyperref[TEI.quote]{<quote>} may be used in combination. For a text with rich internal structure that is quoted at length, \hyperref[TEI.floatingText]{<floatingText>} might be used within \hyperref[TEI.quote]{<quote>}. Also, like a unitary text, \hyperref[TEI.floatingText]{<floatingText>} may include one or more quoted sections, each marked with a \hyperref[TEI.quote]{<quote>} element.
\subsection[{Virtual Divisions}]{Virtual Divisions}\label{DSVIRT}\par
Where the whole of a division can be automatically generated, for example because it is derived from another part of this or another document, an encoder may prefer not to represent it explicitly but instead simply mark its location by means of a processing instruction, or by using the special purpose \hyperref[TEI.divGen]{<divGen>} element: 
\begin{sansreflist}
  
\item [\textbf{<divGen>}] (automatically generated text division) indicates the location at which a textual division generated automatically by a text-processing application is to appear.
\end{sansreflist}
\par
This element is made available by the \textsf{model.divGenLike} class of which it is the sole element. The \hyperref[TEI.divGen]{<divGen>} element is a member of the \textsf{att.typed} class, from which it inherits the {\itshape type} and {\itshape subtype} attributes. It may appear wherever a \hyperref[TEI.div]{<div>} or \hyperref[TEI.div1]{<div1>} (\hyperref[TEI.div2]{<div2>}, etc.) element may appear.\par
For example, if the table of contents (toc) for a given work is simply derived by copying the first \hyperref[TEI.head]{<head>} element from each \hyperref[TEI.div]{<div>} element in a text, it might be more easily encoded as follows: \par\bgroup\index{divGen=<divGen>|exampleindex}\index{type=@type!<divGen>|exampleindex}\exampleFont \begin{shaded}\noindent\mbox{}{<\textbf{divGen}\hspace*{1em}{type}="{toc}"/>}\end{shaded}\egroup\par \noindent  Similarly, in a digital edition combining a transcribed version of some text with a translated version of it, it may be desired to represent the transcript, the translation, and an aligned version of the two as three distinct divisions. This could be achieved by an encoding like the following: \par\bgroup\index{div=<div>|exampleindex}\index{div=<div>|exampleindex}\index{divGen=<divGen>|exampleindex}\index{type=@type!<divGen>|exampleindex}\exampleFont \begin{shaded}\noindent\mbox{}{<\textbf{div}>}\mbox{}\newline 
\textit{<!-- transcript here-->}\mbox{}\newline 
{</\textbf{div}>}\mbox{}\newline 
{<\textbf{div}>}\mbox{}\newline 
\textit{<!-- translation here -->}\mbox{}\newline 
{</\textbf{div}>}\mbox{}\newline 
{<\textbf{divGen}\hspace*{1em}{type}="{alignment}"/>}\end{shaded}\egroup\par \noindent  The processing to be carried out when a \hyperref[TEI.divGen]{<divGen>} element is rendered will be determined by the application program or stylesheet in use: the function of the TEI markup is simply to identify the location at which the virtual division is to be generated, and also to provide some information about the kind of division to be generated. As such it may be regarded as a special kind of processing instruction, and could equally well be represented by one.
\subsection[{Front Matter}]{Front Matter}\label{DSFRONT}\par
By \textit{front matter} we mean distinct sections of a text (usually, but not necessarily, a printed one), prefixed to it by way of introduction or identification as a part of its production. Features such as title pages or prefaces are clear examples; a less definite case might be the prologue attached to a play. The front matter of an encoded text should not be confused with the TEI header described in chapter \textit{\hyperref[HD]{2.\ The TEI Header}}, which serves as a kind of front matter for the computer file itself, not the text it encodes.\par
An encoder may choose simply to ignore the front matter in a text, if the original presentation of the work is of no interest, or for other reasons; alternatively some or all components of the front matter may be thought worth including with the text as components of the \hyperref[TEI.front]{<front>} element.\footnote{This decision should be recorded in the \hyperref[TEI.samplingDecl]{<samplingDecl>} element of the header.} With the exception of the title page, (on which see section \textit{\hyperref[DSTITL]{4.6.\ Title Pages}}), front matter should be encoded using the same elements as the rest of a text. As with the divisions of the text body, no other specific tags are proposed here for the various kinds of subdivision which may appear within front matter: instead either numbered or un-numbered \hyperref[TEI.div]{<div>} elements may be used. The following suggested values\footnote{As with all lists of ‘suggested values’ for attributes, it is recommended that software written to handle TEI-conformant texts be prepared to recognize and handle these values when they occur, without limiting the user to the values in this list.} for the {\itshape type} attribute may be used to distinguish various kinds of division characteristic of front matter: \begin{description}

\item[{preface}]A foreword or preface addressed to the reader in which the author or publisher explains the content, purpose, or origin of the text.
\item[{ack}]A formal declaration of acknowledgment by the author in which persons and institutions are thanked for their part in the creation of a text.
\item[{dedication}]A formal offering or dedication of a text to one or more persons or institutions by the author.
\item[{abstract}]A summary of the content of a text as continuous prose.
\item[{contents}]A table of contents, specifying the structure of a work and listing its constituents. The \hyperref[TEI.list]{<list>} element should be used to mark its structure.
\item[{frontispiece}]A pictorial frontispiece, possibly including some text.
\end{description} \par
The following extended example demonstrates how various parts of the front matter of a text may be encoded. The front part begins with a title page, which is presented in section \textit{\hyperref[DSTITL]{4.6.\ Title Pages}} below. This is followed by a dedication and a preface, each of which is encoded as a distinct \hyperref[TEI.div]{<div>}: \par\bgroup\index{div=<div>|exampleindex}\index{type=@type!<div>|exampleindex}\index{p=<p>|exampleindex}\index{div=<div>|exampleindex}\index{type=@type!<div>|exampleindex}\index{head=<head>|exampleindex}\index{p=<p>|exampleindex}\index{q=<q>|exampleindex}\index{p=<p>|exampleindex}\index{list=<list>|exampleindex}\index{item=<item>|exampleindex}\index{title=<title>|exampleindex}\index{item=<item>|exampleindex}\index{title=<title>|exampleindex}\index{signed=<signed>|exampleindex}\exampleFont \begin{shaded}\noindent\mbox{}{<\textbf{div}\hspace*{1em}{type}="{dedication}">}\mbox{}\newline 
\hspace*{1em}{<\textbf{p}>}To my parents, Ida and Max Fish{</\textbf{p}>}\mbox{}\newline 
{</\textbf{div}>}\mbox{}\newline 
{<\textbf{div}\hspace*{1em}{type}="{preface}">}\mbox{}\newline 
\hspace*{1em}{<\textbf{head}>}Preface{</\textbf{head}>}\mbox{}\newline 
\hspace*{1em}{<\textbf{p}>}The answer this book gives to its title question is {<\textbf{q}>}there is\mbox{}\newline 
\hspace*{1em}\hspace*{1em}\hspace*{1em}\hspace*{1em} and there isn't{</\textbf{q}>}.{</\textbf{p}>}\mbox{}\newline 
\hspace*{1em}{<\textbf{p}>}Chapters 1–12 have been previously published in the\mbox{}\newline 
\hspace*{1em}\hspace*{1em} following journals and collections:\mbox{}\newline 
\hspace*{1em}{<\textbf{list}>}\mbox{}\newline 
\hspace*{1em}\hspace*{1em}\hspace*{1em}{<\textbf{item}>}chapters 1 and 3 in {<\textbf{title}>}New literary History{</\textbf{title}>}\mbox{}\newline 
\hspace*{1em}\hspace*{1em}\hspace*{1em}{</\textbf{item}>}\mbox{}\newline 
\hspace*{1em}\hspace*{1em}\hspace*{1em}{<\textbf{item}>}chapter 10 in {<\textbf{title}>}Boundary II{</\textbf{title}>} (1980){</\textbf{item}>}\mbox{}\newline 
\hspace*{1em}\hspace*{1em}{</\textbf{list}>}.\mbox{}\newline 
\hspace*{1em}\hspace*{1em} I am grateful for permission to reprint.{</\textbf{p}>}\mbox{}\newline 
\hspace*{1em}{<\textbf{signed}>}S.F.{</\textbf{signed}>}\mbox{}\newline 
{</\textbf{div}>}\end{shaded}\egroup\par \par
The front matter concludes with another \hyperref[TEI.div]{<div>} element, shown in the next example, this time containing a table of contents, which contains a \hyperref[TEI.list]{<list>} element (as described in section \textit{\hyperref[COLI]{3.8.\ Lists}}). Note the use of the \hyperref[TEI.ptr]{<ptr>} element to provide page-references: the implication here is that the target identifiers supplied (fish1, fish2, etc.) will correspond with identifiers used for the \hyperref[TEI.div]{<div>} elements containing chapters of the text itself. (For the \hyperref[TEI.ptr]{<ptr>} element, see \textit{\hyperref[COXR]{3.7.\ Simple Links and Cross-References}}.) \par\bgroup\index{div=<div>|exampleindex}\index{type=@type!<div>|exampleindex}\index{head=<head>|exampleindex}\index{list=<list>|exampleindex}\index{item=<item>|exampleindex}\index{ptr=<ptr>|exampleindex}\index{target=@target!<ptr>|exampleindex}\index{item=<item>|exampleindex}\index{list=<list>|exampleindex}\index{head=<head>|exampleindex}\index{item=<item>|exampleindex}\index{n=@n!<item>|exampleindex}\index{ptr=<ptr>|exampleindex}\index{target=@target!<ptr>|exampleindex}\index{item=<item>|exampleindex}\index{n=@n!<item>|exampleindex}\index{ptr=<ptr>|exampleindex}\index{target=@target!<ptr>|exampleindex}\index{div=<div>|exampleindex}\index{head=<head>|exampleindex}\index{div=<div>|exampleindex}\index{head=<head>|exampleindex}\index{div=<div>|exampleindex}\index{head=<head>|exampleindex}\exampleFont \begin{shaded}\noindent\mbox{}{<\textbf{div}\hspace*{1em}{type}="{contents}">}\mbox{}\newline 
\hspace*{1em}{<\textbf{head}>}Contents{</\textbf{head}>}\mbox{}\newline 
\hspace*{1em}{<\textbf{list}>}\mbox{}\newline 
\hspace*{1em}\hspace*{1em}{<\textbf{item}>}Introduction, or How I stopped Worrying and Learned to Love\mbox{}\newline 
\hspace*{1em}\hspace*{1em}\hspace*{1em}\hspace*{1em} Interpretation {<\textbf{ptr}\hspace*{1em}{target}="{\#fish1}"/>}\mbox{}\newline 
\hspace*{1em}\hspace*{1em}{</\textbf{item}>}\mbox{}\newline 
\hspace*{1em}\hspace*{1em}{<\textbf{item}>}\mbox{}\newline 
\hspace*{1em}\hspace*{1em}\hspace*{1em}{<\textbf{list}>}\mbox{}\newline 
\hspace*{1em}\hspace*{1em}\hspace*{1em}\hspace*{1em}{<\textbf{head}>}Part One: Literature in the Reader{</\textbf{head}>}\mbox{}\newline 
\hspace*{1em}\hspace*{1em}\hspace*{1em}\hspace*{1em}{<\textbf{item}\hspace*{1em}{n}="{1}">}Literature in the Reader: Affective Stylistics\mbox{}\newline 
\hspace*{1em}\hspace*{1em}\hspace*{1em}\hspace*{1em}{<\textbf{ptr}\hspace*{1em}{target}="{\#fish2}"/>}\mbox{}\newline 
\hspace*{1em}\hspace*{1em}\hspace*{1em}\hspace*{1em}{</\textbf{item}>}\mbox{}\newline 
\hspace*{1em}\hspace*{1em}\hspace*{1em}\hspace*{1em}{<\textbf{item}\hspace*{1em}{n}="{2}">}What is Stylistics and Why Are They Saying Such\mbox{}\newline 
\hspace*{1em}\hspace*{1em}\hspace*{1em}\hspace*{1em}\hspace*{1em}\hspace*{1em}\hspace*{1em}\hspace*{1em} Terrible Things About It? {<\textbf{ptr}\hspace*{1em}{target}="{\#fish3}"/>}\mbox{}\newline 
\hspace*{1em}\hspace*{1em}\hspace*{1em}\hspace*{1em}{</\textbf{item}>}\mbox{}\newline 
\hspace*{1em}\hspace*{1em}\hspace*{1em}{</\textbf{list}>}\mbox{}\newline 
\hspace*{1em}\hspace*{1em}{</\textbf{item}>}\mbox{}\newline 
\hspace*{1em}{</\textbf{list}>}\mbox{}\newline 
{</\textbf{div}>}\mbox{}\newline 
{<\textbf{div}\hspace*{1em}{xml:id}="{fish1}">}\mbox{}\newline 
\hspace*{1em}{<\textbf{head}>}Introduction{</\textbf{head}>}\mbox{}\newline 
\textit{<!-- ... -->}\mbox{}\newline 
{</\textbf{div}>}\mbox{}\newline 
{<\textbf{div}\hspace*{1em}{xml:id}="{fish2}">}\mbox{}\newline 
\hspace*{1em}{<\textbf{head}>}Literature in the Reader{</\textbf{head}>}\mbox{}\newline 
\textit{<!-- ... -->}\mbox{}\newline 
{</\textbf{div}>}\mbox{}\newline 
{<\textbf{div}\hspace*{1em}{xml:id}="{fish3}">}\mbox{}\newline 
\hspace*{1em}{<\textbf{head}>}What is stylistics?{</\textbf{head}>}\mbox{}\newline 
\textit{<!-- ... -->}\mbox{}\newline 
{</\textbf{div}>}\end{shaded}\egroup\par \noindent  Alternatively, the pointers in the index might link to the page breaks at which a chapter begins, assuming that these have been included in the markup: \par\bgroup\index{item=<item>|exampleindex}\index{n=@n!<item>|exampleindex}\index{ref=<ref>|exampleindex}\index{target=@target!<ref>|exampleindex}\index{div=<div>|exampleindex}\index{type=@type!<div>|exampleindex}\index{head=<head>|exampleindex}\index{pb=<pb>|exampleindex}\exampleFont \begin{shaded}\noindent\mbox{}\mbox{}\newline 
\textit{<!-- ... -->}{<\textbf{item}\hspace*{1em}{n}="{1}">}Literature in the Reader: Affective Stylistics\mbox{}\newline 
{<\textbf{ref}\hspace*{1em}{target}="{\#fish-p24}">}24{</\textbf{ref}>}\mbox{}\newline 
{</\textbf{item}>}\mbox{}\newline 
\textit{<!-- ... -->}\mbox{}\newline 
{<\textbf{div}\hspace*{1em}{type}="{chapter}">}\mbox{}\newline 
\hspace*{1em}{<\textbf{head}>}Literature in the Reader{</\textbf{head}>}\mbox{}\newline 
\hspace*{1em}{<\textbf{pb}\hspace*{1em}{xml:id}="{fish-p24}"/>}\mbox{}\newline 
\textit{<!-- ... -->}\mbox{}\newline 
{</\textbf{div}>}\mbox{}\newline 
\textit{<!-- ... -->}\end{shaded}\egroup\par \par
The following example uses numbered divisions to mark up the front matter of a medieval text. Note that in this case no title page in the modern sense occurs; the title is simply given as a heading at the start of the front matter. Note also the use of the {\itshape type} attribute on the \hyperref[TEI.div]{<div>} elements to indicate document elements comparatively unusual in modern books such as the initial prayer: \par\bgroup\index{front=<front>|exampleindex}\index{div1=<div1>|exampleindex}\index{type=@type!<div1>|exampleindex}\index{p=<p>|exampleindex}\index{title=<title>|exampleindex}\index{div1=<div1>|exampleindex}\index{type=@type!<div1>|exampleindex}\index{head=<head>|exampleindex}\index{p=<p>|exampleindex}\index{div1=<div1>|exampleindex}\index{type=@type!<div1>|exampleindex}\index{head=<head>|exampleindex}\index{p=<p>|exampleindex}\index{p=<p>|exampleindex}\index{div1=<div1>|exampleindex}\index{type=@type!<div1>|exampleindex}\index{head=<head>|exampleindex}\index{list=<list>|exampleindex}\index{label=<label>|exampleindex}\index{item=<item>|exampleindex}\index{label=<label>|exampleindex}\index{item=<item>|exampleindex}\index{label=<label>|exampleindex}\index{item=<item>|exampleindex}\index{trailer=<trailer>|exampleindex}\exampleFont \begin{shaded}\noindent\mbox{}{<\textbf{front}>}\mbox{}\newline 
\hspace*{1em}{<\textbf{div1}\hspace*{1em}{type}="{incipit}">}\mbox{}\newline 
\hspace*{1em}\hspace*{1em}{<\textbf{p}>}Here bygynniþ a book of contemplacyon, þe whiche\mbox{}\newline 
\hspace*{1em}\hspace*{1em}\hspace*{1em}\hspace*{1em} is clepyd {<\textbf{title}>}þE CLOWDE OF VNKNOWYNG{</\textbf{title}>},\mbox{}\newline 
\hspace*{1em}\hspace*{1em}\hspace*{1em}\hspace*{1em} in þe whiche a soule is onyd wiþ GOD.{</\textbf{p}>}\mbox{}\newline 
\hspace*{1em}{</\textbf{div1}>}\mbox{}\newline 
\hspace*{1em}{<\textbf{div1}\hspace*{1em}{type}="{prayer}">}\mbox{}\newline 
\hspace*{1em}\hspace*{1em}{<\textbf{head}>}Here biginneþ þe preyer on þe prologe.{</\textbf{head}>}\mbox{}\newline 
\hspace*{1em}\hspace*{1em}{<\textbf{p}>}God, unto whom alle hertes ben open, \& unto whome alle wille\mbox{}\newline 
\hspace*{1em}\hspace*{1em}\hspace*{1em}\hspace*{1em} spekiþ, \& unto whom no priue þing is hid: I beseche\mbox{}\newline 
\hspace*{1em}\hspace*{1em}\hspace*{1em}\hspace*{1em} þee so for to clense þe entent of myn hert wiþ þe\mbox{}\newline 
\hspace*{1em}\hspace*{1em}\hspace*{1em}\hspace*{1em} unspekable 3ift of þi grace, þat I may parfiteliche\mbox{}\newline 
\hspace*{1em}\hspace*{1em}\hspace*{1em}\hspace*{1em} loue þee \& worþilich preise þee. Amen.{</\textbf{p}>}\mbox{}\newline 
\hspace*{1em}{</\textbf{div1}>}\mbox{}\newline 
\hspace*{1em}{<\textbf{div1}\hspace*{1em}{type}="{preface}">}\mbox{}\newline 
\hspace*{1em}\hspace*{1em}{<\textbf{head}>}Here biginneþ þe prolog.{</\textbf{head}>}\mbox{}\newline 
\hspace*{1em}\hspace*{1em}{<\textbf{p}>}In þe name of þe Fader \& of þe Sone \&\mbox{}\newline 
\hspace*{1em}\hspace*{1em}\hspace*{1em}\hspace*{1em} of þe Holy Goost.{</\textbf{p}>}\mbox{}\newline 
\hspace*{1em}\hspace*{1em}{<\textbf{p}>}I charge þee \& I beseeche þee, wiþ as moche\mbox{}\newline 
\hspace*{1em}\hspace*{1em}\hspace*{1em}\hspace*{1em} power \& vertewe as þe bonde of charite is sufficient\mbox{}\newline 
\hspace*{1em}\hspace*{1em}\hspace*{1em}\hspace*{1em} to suffre, what-so-euer þou be þat þis book schalt\mbox{}\newline 
\hspace*{1em}\hspace*{1em}\hspace*{1em}\hspace*{1em} haue in possession ...{</\textbf{p}>}\mbox{}\newline 
\hspace*{1em}{</\textbf{div1}>}\mbox{}\newline 
\hspace*{1em}{<\textbf{div1}\hspace*{1em}{type}="{contents}">}\mbox{}\newline 
\hspace*{1em}\hspace*{1em}{<\textbf{head}>}Here biginneþ a table of þe chapitres.{</\textbf{head}>}\mbox{}\newline 
\hspace*{1em}\hspace*{1em}{<\textbf{list}>}\mbox{}\newline 
\hspace*{1em}\hspace*{1em}\hspace*{1em}{<\textbf{label}>}þe first chapitre {</\textbf{label}>}\mbox{}\newline 
\hspace*{1em}\hspace*{1em}\hspace*{1em}{<\textbf{item}>}Of foure degrees of Cristen mens leuing; \& of þe\mbox{}\newline 
\hspace*{1em}\hspace*{1em}\hspace*{1em}\hspace*{1em}\hspace*{1em}\hspace*{1em} cours of his cleping þat þis book was maad vnto.{</\textbf{item}>}\mbox{}\newline 
\hspace*{1em}\hspace*{1em}\hspace*{1em}{<\textbf{label}>}þe secound chapitre{</\textbf{label}>}\mbox{}\newline 
\hspace*{1em}\hspace*{1em}\hspace*{1em}{<\textbf{item}>}A schort stering to meeknes \& to þe werk of þis\mbox{}\newline 
\hspace*{1em}\hspace*{1em}\hspace*{1em}\hspace*{1em}\hspace*{1em}\hspace*{1em} book{</\textbf{item}>}\mbox{}\newline 
\hspace*{1em}\hspace*{1em}\hspace*{1em}{<\textbf{label}>}þe fiue and seuenti chapitre{</\textbf{label}>}\mbox{}\newline 
\hspace*{1em}\hspace*{1em}\hspace*{1em}{<\textbf{item}>}Of somme certein tokenes bi þe whiche a man may proue\mbox{}\newline 
\hspace*{1em}\hspace*{1em}\hspace*{1em}\hspace*{1em}\hspace*{1em}\hspace*{1em} wheþer he be clepid of God to worche in þis werk.{</\textbf{item}>}\mbox{}\newline 
\hspace*{1em}\hspace*{1em}{</\textbf{list}>}\mbox{}\newline 
\hspace*{1em}\hspace*{1em}{<\textbf{trailer}>}\& here eendeþ þe table of þe chapitres.{</\textbf{trailer}>}\mbox{}\newline 
\hspace*{1em}{</\textbf{div1}>}\mbox{}\newline 
{</\textbf{front}>}\end{shaded}\egroup\par \noindent  \par
If, however, the table of contents can be automatically generated from the remainder of the text, it may be preferable simply to mark its presence, either by means of an empty \hyperref[TEI.divGen]{<divGen>} element or by using an appropriate processing instruction.
\subsection[{Title Pages}]{Title Pages}\label{DSTITL}\par
Detailed analysis of the title page and other \textit{preliminaries} of older printed books and manuscripts is of major importance in descriptive bibliography and the cataloguing of printed books; such analysis may require a rather more detailed module than that proposed here. The following elements are suggested as a means of encoding the major features of most title pages: 
\begin{sansreflist}
  
\item [\textbf{<titlePage>}] (title page) contains the title page of a text, appearing within the front or back matter.
\item [\textbf{<docTitle>}] (document title) contains the title of a document, including all its constituents, as given on a title page.
\item [\textbf{<titlePart>}] (title part) contains a subsection or division of the title of a work, as indicated on a title page.\hfil\\[-10pt]\begin{sansreflist}
    \item[@{\itshape type}]
  (type) specifies the role of this subdivision of the title.
\end{sansreflist}  
\item [\textbf{<argument>}] (argument) contains a formal list or prose description of the topics addressed by a subdivision of a text.
\item [\textbf{<byline>}] (byline) contains the primary statement of responsibility given for a work on its title page or at the head or end of the work.
\item [\textbf{<docAuthor>}] (document author) contains the name of the author of the document, as given on the title page (often but not always contained in a byline).
\item [\textbf{<epigraph>}] (epigraph) contains a quotation, anonymous or attributed, appearing at the start or end of a section or on a title page.
\item [\textbf{<imprimatur>}] (imprimatur) contains a formal statement authorizing the publication of a work, sometimes required to appear on a title page or its verso.
\item [\textbf{<docEdition>}] (document edition) contains an edition statement as presented on a title page of a document.
\item [\textbf{<docImprint>}] (document imprint) contains the imprint statement (place and date of publication, publisher name), as given (usually) at the foot of a title page.
\item [\textbf{<docDate>}] (document date) contains the date of a document, as given on a title page or in a dateline.
\item [\textbf{<graphic>}] (graphic) indicates the location of a graphic or illustration, either forming part of a text, or providing an image of it.
\end{sansreflist}
\par
Together with the \hyperref[TEI.figure]{<figure>} element described in chapter \textit{\hyperref[FT]{14.\ Tables, Formulæ, Graphics and Notated Music}}, these elements constitute the \textsf{model.titlepagePart} class. Any number of elements from this class can appear grouped together within a \hyperref[TEI.titlePage]{<titlePage>} element. The \hyperref[TEI.figure]{<figure>} element is included so as to enable encoders to record the presence of complex non-textual material on a title page. For simple cases such as printers' ornaments or illustrations the \hyperref[TEI.graphic]{<graphic>} element discussed in section \textit{\hyperref[COGR]{3.10.\ Graphics and Other Non-textual Components}} should be adequate.\par
The elements listed above, together with the \hyperref[TEI.head]{<head>} element, also constitute the class \textsf{model.pLike.front}. The elements in this class can appear within a minimal \hyperref[TEI.front]{<front>} element without any need to group them together and encode a complete title page.\par
Encoders wishing to add new elements to either class may do so using the methods described in section \textit{\hyperref[MD]{23.3.\ Customization}}. Two examples of the use of these elements follow. First, the title page of the work discussed earlier in this section: \par\bgroup\index{front=<front>|exampleindex}\index{titlePage=<titlePage>|exampleindex}\index{docTitle=<docTitle>|exampleindex}\index{titlePart=<titlePart>|exampleindex}\index{type=@type!<titlePart>|exampleindex}\index{titlePart=<titlePart>|exampleindex}\index{type=@type!<titlePart>|exampleindex}\index{docAuthor=<docAuthor>|exampleindex}\index{docImprint=<docImprint>|exampleindex}\index{publisher=<publisher>|exampleindex}\index{pubPlace=<pubPlace>|exampleindex}\index{pubPlace=<pubPlace>|exampleindex}\exampleFont \begin{shaded}\noindent\mbox{}{<\textbf{front}>}\mbox{}\newline 
\hspace*{1em}{<\textbf{titlePage}>}\mbox{}\newline 
\hspace*{1em}\hspace*{1em}{<\textbf{docTitle}>}\mbox{}\newline 
\hspace*{1em}\hspace*{1em}\hspace*{1em}{<\textbf{titlePart}\hspace*{1em}{type}="{main}">}Is There a Text in This Class?{</\textbf{titlePart}>}\mbox{}\newline 
\hspace*{1em}\hspace*{1em}\hspace*{1em}{<\textbf{titlePart}\hspace*{1em}{type}="{sub}">}The Authority of Interpretive Communities{</\textbf{titlePart}>}\mbox{}\newline 
\hspace*{1em}\hspace*{1em}{</\textbf{docTitle}>}\mbox{}\newline 
\hspace*{1em}\hspace*{1em}{<\textbf{docAuthor}>}Stanley Fish{</\textbf{docAuthor}>}\mbox{}\newline 
\hspace*{1em}\hspace*{1em}{<\textbf{docImprint}>}\mbox{}\newline 
\hspace*{1em}\hspace*{1em}\hspace*{1em}{<\textbf{publisher}>}Harvard University Press{</\textbf{publisher}>}\mbox{}\newline 
\hspace*{1em}\hspace*{1em}\hspace*{1em}{<\textbf{pubPlace}>}Cambridge, Massachusetts{</\textbf{pubPlace}>}\mbox{}\newline 
\hspace*{1em}\hspace*{1em}\hspace*{1em}{<\textbf{pubPlace}>}London, England{</\textbf{pubPlace}>}\mbox{}\newline 
\hspace*{1em}\hspace*{1em}{</\textbf{docImprint}>}\mbox{}\newline 
\hspace*{1em}{</\textbf{titlePage}>}\mbox{}\newline 
{</\textbf{front}>}\end{shaded}\egroup\par \par
Second, a characteristically verbose 17th century example. Note the use of the \hyperref[TEI.lb]{<lb>} tag to mark the line breaks of the original where necessary: \par\bgroup\index{titlePage=<titlePage>|exampleindex}\index{docTitle=<docTitle>|exampleindex}\index{titlePart=<titlePart>|exampleindex}\index{type=@type!<titlePart>|exampleindex}\index{lb=<lb>|exampleindex}\index{lb=<lb>|exampleindex}\index{lb=<lb>|exampleindex}\index{lb=<lb>|exampleindex}\index{lb=<lb>|exampleindex}\index{titlePart=<titlePart>|exampleindex}\index{type=@type!<titlePart>|exampleindex}\index{lb=<lb>|exampleindex}\index{titlePart=<titlePart>|exampleindex}\index{type=@type!<titlePart>|exampleindex}\index{lb=<lb>|exampleindex}\index{lb=<lb>|exampleindex}\index{lb=<lb>|exampleindex}\index{epigraph=<epigraph>|exampleindex}\index{cit=<cit>|exampleindex}\index{quote=<quote>|exampleindex}\index{bibl=<bibl>|exampleindex}\index{byline=<byline>|exampleindex}\index{docAuthor=<docAuthor>|exampleindex}\index{imprimatur=<imprimatur>|exampleindex}\index{docImprint=<docImprint>|exampleindex}\index{pubPlace=<pubPlace>|exampleindex}\index{name=<name>|exampleindex}\index{lb=<lb>|exampleindex}\index{name=<name>|exampleindex}\index{name=<name>|exampleindex}\index{lb=<lb>|exampleindex}\index{name=<name>|exampleindex}\index{docDate=<docDate>|exampleindex}\exampleFont \begin{shaded}\noindent\mbox{}{<\textbf{titlePage}>}\mbox{}\newline 
\hspace*{1em}{<\textbf{docTitle}>}\mbox{}\newline 
\hspace*{1em}\hspace*{1em}{<\textbf{titlePart}\hspace*{1em}{type}="{main}">}THE\mbox{}\newline 
\hspace*{1em}\hspace*{1em}{<\textbf{lb}/>}Pilgrim's Progress\mbox{}\newline 
\hspace*{1em}\hspace*{1em}{<\textbf{lb}/>}FROM\mbox{}\newline 
\hspace*{1em}\hspace*{1em}{<\textbf{lb}/>}THIS WORLD,\mbox{}\newline 
\hspace*{1em}\hspace*{1em}{<\textbf{lb}/>}TO\mbox{}\newline 
\hspace*{1em}\hspace*{1em}{<\textbf{lb}/>}That which is to come:{</\textbf{titlePart}>}\mbox{}\newline 
\hspace*{1em}\hspace*{1em}{<\textbf{titlePart}\hspace*{1em}{type}="{sub}">}Delivered under the Similitude of a\mbox{}\newline 
\hspace*{1em}\hspace*{1em}{<\textbf{lb}/>}DREAM{</\textbf{titlePart}>}\mbox{}\newline 
\hspace*{1em}\hspace*{1em}{<\textbf{titlePart}\hspace*{1em}{type}="{desc}">}Wherein is Discovered,\mbox{}\newline 
\hspace*{1em}\hspace*{1em}{<\textbf{lb}/>}The manner of his setting out,\mbox{}\newline 
\hspace*{1em}\hspace*{1em}{<\textbf{lb}/>}His Dangerous Journey; And safe\mbox{}\newline 
\hspace*{1em}\hspace*{1em}{<\textbf{lb}/>}Arrival at the Desired Countrey.{</\textbf{titlePart}>}\mbox{}\newline 
\hspace*{1em}{</\textbf{docTitle}>}\mbox{}\newline 
\hspace*{1em}{<\textbf{epigraph}>}\mbox{}\newline 
\hspace*{1em}\hspace*{1em}{<\textbf{cit}>}\mbox{}\newline 
\hspace*{1em}\hspace*{1em}\hspace*{1em}{<\textbf{quote}>}I have used Similitudes,{</\textbf{quote}>}\mbox{}\newline 
\hspace*{1em}\hspace*{1em}\hspace*{1em}{<\textbf{bibl}>}Hos. 12.10{</\textbf{bibl}>}\mbox{}\newline 
\hspace*{1em}\hspace*{1em}{</\textbf{cit}>}\mbox{}\newline 
\hspace*{1em}{</\textbf{epigraph}>}\mbox{}\newline 
\hspace*{1em}{<\textbf{byline}>}By {<\textbf{docAuthor}>}John Bunyan{</\textbf{docAuthor}>}.{</\textbf{byline}>}\mbox{}\newline 
\hspace*{1em}{<\textbf{imprimatur}>}Licensed and Entred according to Order.{</\textbf{imprimatur}>}\mbox{}\newline 
\hspace*{1em}{<\textbf{docImprint}>}\mbox{}\newline 
\hspace*{1em}\hspace*{1em}{<\textbf{pubPlace}>}LONDON,{</\textbf{pubPlace}>}\mbox{}\newline 
\hspace*{1em}\hspace*{1em} Printed for {<\textbf{name}>}Nath. Ponder{</\textbf{name}>}\mbox{}\newline 
\hspace*{1em}\hspace*{1em}{<\textbf{lb}/>}at the {<\textbf{name}>}Peacock{</\textbf{name}>} in the {<\textbf{name}>}Poultrey{</\textbf{name}>}\mbox{}\newline 
\hspace*{1em}\hspace*{1em}{<\textbf{lb}/>}near {<\textbf{name}>}Cornhil{</\textbf{name}>}, {<\textbf{docDate}>}1678{</\textbf{docDate}>}.\mbox{}\newline 
\hspace*{1em}{</\textbf{docImprint}>}\mbox{}\newline 
{</\textbf{titlePage}>}\end{shaded}\egroup\par \par
Where, as here, it is considered important to encode salient features of the way a title page was originally rendered, the techniques exemplified in \textit{\hyperref[HD57]{2.3.4.\ The Tagging Declaration}} may also be useful.\par
Where title pages are encoded, their physical rendition is often of considerable importance. One approach to this requirement would be to use the \hyperref[TEI.seg]{<seg>} tag, described in chapter \textit{\hyperref[SA]{16.\ Linking, Segmentation, and Alignment}}, to segment the typographic content of each part of the title page, and then use the global {\itshape rend} attribute to specify its rendition. Another would be to use a module specialized for the description of typographic entities such as pages, lines, rules, etc., bearing special-purpose attributes to describe line-height, leading, degree of kerning, font, etc. Further discussion of these problems is provided in chapter \textit{\hyperref[PH]{11.\ Representation of Primary Sources}}.
\subsection[{Back Matter}]{Back Matter}\label{DSBACK}\par
Conventions vary as to which elements are grouped as back matter and which as front. For example, some books place the table of contents at the front, and others at the back. Even title pages may appear at the back of a book as well as at the front. The content model for \hyperref[TEI.back]{<back>} and \hyperref[TEI.front]{<front>} elements are therefore identical.\par
The following suggested values may be used for the {\itshape type} attribute on all division elements, in order to distinguish various kinds of division characteristic of back matter: \begin{description}

\item[{appendix}]An ancillary self-contained section of a work, often providing additional but in some sense extra-canonical text.
\item[{glossary}]A list of terms associated with definition texts (‘glosses’): this should be encoded as a <list type="gloss"> (see section \textit{\hyperref[COLI]{3.8.\ Lists}}).
\item[{notes}]A section in which textual or other kinds of notes are gathered together.
\item[{bibliogr}]A list of bibliographic citations: this should be encoded as a \hyperref[TEI.listBibl]{<listBibl>} (see section \textit{\hyperref[COBI]{3.12.\ Bibliographic Citations and References}}).
\item[{index}]Any form of index to the work.
\item[{colophon}]A statement appearing at the end of a book describing the conditions of its physical production.
\end{description} \par
No additional elements are proposed for the encoding of back matter at present. Some characteristic examples follow; first, an index (for the case in which a printed index is of sufficient interest to merit transcription): \par\bgroup\index{back=<back>|exampleindex}\index{div=<div>|exampleindex}\index{type=@type!<div>|exampleindex}\index{head=<head>|exampleindex}\index{list=<list>|exampleindex}\index{type=@type!<list>|exampleindex}\index{item=<item>|exampleindex}\index{ref=<ref>|exampleindex}\index{item=<item>|exampleindex}\index{ref=<ref>|exampleindex}\index{item=<item>|exampleindex}\index{list=<list>|exampleindex}\index{type=@type!<list>|exampleindex}\index{item=<item>|exampleindex}\index{ref=<ref>|exampleindex}\index{item=<item>|exampleindex}\index{ref=<ref>|exampleindex}\index{item=<item>|exampleindex}\index{ref=<ref>|exampleindex}\index{item=<item>|exampleindex}\index{ref=<ref>|exampleindex}\index{item=<item>|exampleindex}\index{ref=<ref>|exampleindex}\exampleFont \begin{shaded}\noindent\mbox{}{<\textbf{back}>}\mbox{}\newline 
\hspace*{1em}{<\textbf{div}\hspace*{1em}{type}="{index}">}\mbox{}\newline 
\hspace*{1em}\hspace*{1em}{<\textbf{head}>}Index{</\textbf{head}>}\mbox{}\newline 
\hspace*{1em}\hspace*{1em}{<\textbf{list}\hspace*{1em}{type}="{index}">}\mbox{}\newline 
\hspace*{1em}\hspace*{1em}\hspace*{1em}{<\textbf{item}>}Actors, public, paid for the contempt attending\mbox{}\newline 
\hspace*{1em}\hspace*{1em}\hspace*{1em}\hspace*{1em}\hspace*{1em}\hspace*{1em} their profession, {<\textbf{ref}>}263{</\textbf{ref}>}\mbox{}\newline 
\hspace*{1em}\hspace*{1em}\hspace*{1em}{</\textbf{item}>}\mbox{}\newline 
\hspace*{1em}\hspace*{1em}\hspace*{1em}{<\textbf{item}>}Africa, cause assigned for the barbarous state of\mbox{}\newline 
\hspace*{1em}\hspace*{1em}\hspace*{1em}\hspace*{1em}\hspace*{1em}\hspace*{1em} the interior parts of that continent, {<\textbf{ref}>}125{</\textbf{ref}>}\mbox{}\newline 
\hspace*{1em}\hspace*{1em}\hspace*{1em}{</\textbf{item}>}\mbox{}\newline 
\hspace*{1em}\hspace*{1em}\hspace*{1em}{<\textbf{item}>}Agriculture\mbox{}\newline 
\hspace*{1em}\hspace*{1em}\hspace*{1em}{<\textbf{list}\hspace*{1em}{type}="{indexentry}">}\mbox{}\newline 
\hspace*{1em}\hspace*{1em}\hspace*{1em}\hspace*{1em}\hspace*{1em}{<\textbf{item}>}ancient policy of Europe unfavourable to, {<\textbf{ref}>}371{</\textbf{ref}>}\mbox{}\newline 
\hspace*{1em}\hspace*{1em}\hspace*{1em}\hspace*{1em}\hspace*{1em}{</\textbf{item}>}\mbox{}\newline 
\hspace*{1em}\hspace*{1em}\hspace*{1em}\hspace*{1em}\hspace*{1em}{<\textbf{item}>}artificers necessary to carry it on, {<\textbf{ref}>}481{</\textbf{ref}>}\mbox{}\newline 
\hspace*{1em}\hspace*{1em}\hspace*{1em}\hspace*{1em}\hspace*{1em}{</\textbf{item}>}\mbox{}\newline 
\hspace*{1em}\hspace*{1em}\hspace*{1em}\hspace*{1em}\hspace*{1em}{<\textbf{item}>}cattle and tillage mutually improve each other, {<\textbf{ref}>}325{</\textbf{ref}>}\mbox{}\newline 
\hspace*{1em}\hspace*{1em}\hspace*{1em}\hspace*{1em}\hspace*{1em}{</\textbf{item}>}\mbox{}\newline 
\hspace*{1em}\hspace*{1em}\hspace*{1em}\hspace*{1em}\hspace*{1em}{<\textbf{item}>}wealth arising from more solid than that which proceeds\mbox{}\newline 
\hspace*{1em}\hspace*{1em}\hspace*{1em}\hspace*{1em}\hspace*{1em}\hspace*{1em}\hspace*{1em}\hspace*{1em}\hspace*{1em}\hspace*{1em} from commerce {<\textbf{ref}>}520{</\textbf{ref}>}\mbox{}\newline 
\hspace*{1em}\hspace*{1em}\hspace*{1em}\hspace*{1em}\hspace*{1em}{</\textbf{item}>}\mbox{}\newline 
\hspace*{1em}\hspace*{1em}\hspace*{1em}\hspace*{1em}{</\textbf{list}>}\mbox{}\newline 
\hspace*{1em}\hspace*{1em}\hspace*{1em}{</\textbf{item}>}\mbox{}\newline 
\hspace*{1em}\hspace*{1em}\hspace*{1em}{<\textbf{item}>}Alehouses, the number of, not the efficient cause of drunkenness, {<\textbf{ref}>}461{</\textbf{ref}>}\mbox{}\newline 
\hspace*{1em}\hspace*{1em}\hspace*{1em}{</\textbf{item}>}\mbox{}\newline 
\hspace*{1em}\hspace*{1em}{</\textbf{list}>}\mbox{}\newline 
\hspace*{1em}{</\textbf{div}>}\mbox{}\newline 
{</\textbf{back}>}\end{shaded}\egroup\par \noindent   Note that if the page breaks in the original source have also been explicitly encoded, and given identifiers, the references to them in the above index can more usefully be recorded as links. For example, assuming that the encoding of page 461 of the original source starts like this: \par\bgroup\index{pb=<pb>|exampleindex}\exampleFont \begin{shaded}\noindent\mbox{}{<\textbf{pb}\hspace*{1em}{xml:id}="{P461}"/>}\end{shaded}\egroup\par \noindent  then the last item above might be encoded more usefully in either of the following forms: \par\bgroup\index{item=<item>|exampleindex}\index{ref=<ref>|exampleindex}\index{target=@target!<ref>|exampleindex}\index{item=<item>|exampleindex}\index{ptr=<ptr>|exampleindex}\index{target=@target!<ptr>|exampleindex}\exampleFont \begin{shaded}\noindent\mbox{}{<\textbf{item}>}Alehouses, the number of, not\mbox{}\newline 
 the efficient cause of drunkenness, {<\textbf{ref}\hspace*{1em}{target}="{\#P461}">}461{</\textbf{ref}>}\mbox{}\newline 
{</\textbf{item}>}\mbox{}\newline 
{<\textbf{item}>}Alehouses, the number of, not the efficient cause of drunkenness, {<\textbf{ptr}\hspace*{1em}{target}="{\#P461}"/>}\mbox{}\newline 
{</\textbf{item}>}\end{shaded}\egroup\par \par
Next, a back-matter division in epistolary form: \par\bgroup\index{back=<back>|exampleindex}\index{div=<div>|exampleindex}\index{type=@type!<div>|exampleindex}\index{head=<head>|exampleindex}\index{p=<p>|exampleindex}\index{signed=<signed>|exampleindex}\index{name=<name>|exampleindex}\index{epigraph=<epigraph>|exampleindex}\index{p=<p>|exampleindex}\index{trailer=<trailer>|exampleindex}\exampleFont \begin{shaded}\noindent\mbox{}{<\textbf{back}>}\mbox{}\newline 
\hspace*{1em}{<\textbf{div}\hspace*{1em}{type}="{letter}">}\mbox{}\newline 
\hspace*{1em}\hspace*{1em}{<\textbf{head}>}A letter written to his wife, founde with this booke\mbox{}\newline 
\hspace*{1em}\hspace*{1em}\hspace*{1em}\hspace*{1em} after his death.{</\textbf{head}>}\mbox{}\newline 
\hspace*{1em}\hspace*{1em}{<\textbf{p}>}The remembrance of the many wrongs offred thee, and thy\mbox{}\newline 
\hspace*{1em}\hspace*{1em}\hspace*{1em}\hspace*{1em} unreproued vertues, adde greater sorrow to my miserable state,\mbox{}\newline 
\hspace*{1em}\hspace*{1em}\hspace*{1em}\hspace*{1em} than I can utter or thou conceiue. ...\mbox{}\newline 
\hspace*{1em}\hspace*{1em}\hspace*{1em}\hspace*{1em} ... yet trust I in the world to come to find mercie, by the\mbox{}\newline 
\hspace*{1em}\hspace*{1em}\hspace*{1em}\hspace*{1em} merites of my Saiuour to whom I commend thee, and commit\mbox{}\newline 
\hspace*{1em}\hspace*{1em}\hspace*{1em}\hspace*{1em} my soule.{</\textbf{p}>}\mbox{}\newline 
\hspace*{1em}\hspace*{1em}{<\textbf{signed}>}Thy repentant husband for his disloyaltie,\mbox{}\newline 
\hspace*{1em}\hspace*{1em}{<\textbf{name}>}Robert Greene.{</\textbf{name}>}\mbox{}\newline 
\hspace*{1em}\hspace*{1em}{</\textbf{signed}>}\mbox{}\newline 
\hspace*{1em}\hspace*{1em}{<\textbf{epigraph}\hspace*{1em}{xml:lang}="{la}">}\mbox{}\newline 
\hspace*{1em}\hspace*{1em}\hspace*{1em}{<\textbf{p}>}Faelicem fuisse infaustum{</\textbf{p}>}\mbox{}\newline 
\hspace*{1em}\hspace*{1em}{</\textbf{epigraph}>}\mbox{}\newline 
\hspace*{1em}\hspace*{1em}{<\textbf{trailer}>}FINIS{</\textbf{trailer}>}\mbox{}\newline 
\hspace*{1em}{</\textbf{div}>}\mbox{}\newline 
{</\textbf{back}>}\end{shaded}\egroup\par \noindent  \par
And finally, a list of corrigenda and addenda with pseudo-epistolary features: \par\bgroup\index{back=<back>|exampleindex}\index{div=<div>|exampleindex}\index{type=@type!<div>|exampleindex}\index{head=<head>|exampleindex}\index{salute=<salute>|exampleindex}\index{p=<p>|exampleindex}\index{name=<name>|exampleindex}\index{q=<q>|exampleindex}\index{q=<q>|exampleindex}\index{q=<q>|exampleindex}\index{q=<q>|exampleindex}\index{q=<q>|exampleindex}\index{q=<q>|exampleindex}\exampleFont \begin{shaded}\noindent\mbox{}{<\textbf{back}>}\mbox{}\newline 
\hspace*{1em}{<\textbf{div}\hspace*{1em}{type}="{corrigenda}">}\mbox{}\newline 
\hspace*{1em}\hspace*{1em}{<\textbf{head}>}Addenda{</\textbf{head}>}\mbox{}\newline 
\hspace*{1em}\hspace*{1em}{<\textbf{salute}\hspace*{1em}{xml:lang}="{la}">}M. Scriblerus Lectori{</\textbf{salute}>}\mbox{}\newline 
\hspace*{1em}\hspace*{1em}{<\textbf{p}>}Once more, gentle reader I appeal unto thee, from the shameful\mbox{}\newline 
\hspace*{1em}\hspace*{1em}\hspace*{1em}\hspace*{1em} ignorance of the Editor, by whom Our own Specimen of\mbox{}\newline 
\hspace*{1em}\hspace*{1em}{<\textbf{name}>}Virgil{</\textbf{name}>} hath been mangled in such miserable manner, that\mbox{}\newline 
\hspace*{1em}\hspace*{1em}\hspace*{1em}\hspace*{1em} scarce without tears can we behold it. At the very entrance, Instead\mbox{}\newline 
\hspace*{1em}\hspace*{1em}\hspace*{1em}\hspace*{1em} of {<\textbf{q}\hspace*{1em}{xml:lang}="{grc}">}προλεγομενα{</\textbf{q}>}, lo!\mbox{}\newline 
\hspace*{1em}\hspace*{1em}{<\textbf{q}\hspace*{1em}{xml:lang}="{grc}">}προλεγωμενα{</\textbf{q}>} with an Omega!\mbox{}\newline 
\hspace*{1em}\hspace*{1em}\hspace*{1em}\hspace*{1em} and in the same line {<\textbf{q}\hspace*{1em}{xml:lang}="{la}">}consulâs{</\textbf{q}>} with a circumflex!\mbox{}\newline 
\hspace*{1em}\hspace*{1em}\hspace*{1em}\hspace*{1em} In the next page thou findest {<\textbf{q}\hspace*{1em}{xml:lang}="{la}">}leviter perlabere{</\textbf{q}>},\mbox{}\newline 
\hspace*{1em}\hspace*{1em}\hspace*{1em}\hspace*{1em} which his ignorance took to be the infinitive mood of\mbox{}\newline 
\hspace*{1em}\hspace*{1em}{<\textbf{q}\hspace*{1em}{xml:lang}="{la}">}perlabor{</\textbf{q}>} but ought to be\mbox{}\newline 
\hspace*{1em}\hspace*{1em}{<\textbf{q}\hspace*{1em}{xml:lang}="{la}">}perlabi{</\textbf{q}>} ... Wipe away all these\mbox{}\newline 
\hspace*{1em}\hspace*{1em}\hspace*{1em}\hspace*{1em} monsters, Reader, with thy quill.{</\textbf{p}>}\mbox{}\newline 
\hspace*{1em}{</\textbf{div}>}\mbox{}\newline 
{</\textbf{back}>}\end{shaded}\egroup\par \noindent   
\subsection[{Module for Default Text Structure}]{Module for Default Text Structure}\label{DSSTRUC}\par
The module described by the present chapter has the following components: \begin{description}

\item[{Module textstructure: Default text structure}]\hspace{1em}\hfill\linebreak
\mbox{}\\[-10pt] \begin{itemize}
\item {\itshape Elements defined}: \hyperref[TEI.TEI]{TEI} \hyperref[TEI.argument]{argument} \hyperref[TEI.back]{back} \hyperref[TEI.body]{body} \hyperref[TEI.byline]{byline} \hyperref[TEI.closer]{closer} \hyperref[TEI.dateline]{dateline} \hyperref[TEI.div]{div} \hyperref[TEI.div1]{div1} \hyperref[TEI.div2]{div2} \hyperref[TEI.div3]{div3} \hyperref[TEI.div4]{div4} \hyperref[TEI.div5]{div5} \hyperref[TEI.div6]{div6} \hyperref[TEI.div7]{div7} \hyperref[TEI.docAuthor]{docAuthor} \hyperref[TEI.docDate]{docDate} \hyperref[TEI.docEdition]{docEdition} \hyperref[TEI.docImprint]{docImprint} \hyperref[TEI.docTitle]{docTitle} \hyperref[TEI.epigraph]{epigraph} \hyperref[TEI.floatingText]{floatingText} \hyperref[TEI.front]{front} \hyperref[TEI.group]{group} \hyperref[TEI.imprimatur]{imprimatur} \hyperref[TEI.opener]{opener} \hyperref[TEI.postscript]{postscript} \hyperref[TEI.salute]{salute} \hyperref[TEI.signed]{signed} \hyperref[TEI.text]{text} \hyperref[TEI.titlePage]{titlePage} \hyperref[TEI.titlePart]{titlePart} \hyperref[TEI.trailer]{trailer}
\end{itemize} 
\end{description}  The selection and combination of modules to form a TEI schema is described in \textit{\hyperref[STIN]{1.2.\ Defining a TEI Schema}}.